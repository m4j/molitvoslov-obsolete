

\mypart{Божественная литургия}
%http://www.molitvoslov.com/content/Liturgiya



В разделе \bfseries  Божественная Литургия\normalfont{} приведены песнопения Божественных литургий святителей Иоанна Златоуста и Василия Великого. Избранные песнопения литургии Преждеосвященных Даров помещены в главе "Песнопения из служб Триоди постной".


Литургия святителя Иоанна Златоуста совершается в Православной Церкви в течении всего года, кроме Великого Поста, когда она совершается по субботам, в Благовещение Пресвятой Богородицы и в Неделю ваий.


Литургия святителя Василия Великого совершается 10 раз в году: в воскресенье дни Великого поста, в Великие четверг и субботу, в день памяти святителя Василия Великого (1 января), в Рождественский и Крещенский сочельники (если же эти дни выпадают на субботу или воскресенье, то совершается в самые праздники Рождества Христова и Крещения Господня).

Литургия состоит из трех частей: проскомидии, литургии оглашенных и литургии верных.
 

\mychapter{Проскомидия}
%http://www.molitvoslov.com/content/Proskomidiya 
 


До чтения третьего и шестого часов, или во время чтения их, в алтаре совершаются священнодействия проскомидии, через которые из принесенных хлеба и вина приготовляется вещество для Св. Евхаристии и при этом совершается предварительное поминовение членов Церкви Христовой "--- небесной и земной.


От древнего обычая приносить в храм хлеб и вино для таинства Св. Евхаристии первая часть литургии и называется проскомидией "--- приношением.


\mychapterending

\mychapter{Литургия оглашенных}
%http://www.molitvoslov.com/text906.htm 
 


Вторая часть литургии называется литургией оглашенных. Такое название эта часть службы получила от содержания в ее составе молитвословий, песнопений, священнодействий и поучений, имеющих учительный, огласительный характер. В древней Церкви во время ее совершения могли присутствовать, вместе с верными, и оглашенные, т.е. лица, готовящиеся ко Св. Крещению, а также кающиеся, отлученные от Св. Причастия.


\itshape Диакон:\normalfont{} Благослови, владыко.
\itshape Иерей:\normalfont{} Благословено Царство Отца и Сына и Святаго Духа, ныне и присно и во веки веков. 


\itshape Хор:\normalfont{} Аминь.


\medskip
\bfseries Великая ектения\normalfont{}


\itshape Диакон:\normalfont{} Миром Господу помолимся. 


\itshape Хор:\normalfont{} Господи, помилуй.\itshape  (На каждое прошение)\normalfont{}


О свышнем мире и спасении душ наших, Господу помолимся. 


О мире всего мира, благостоянии Святых Божиих Церквей и соединении всех, Господу помолимся.


О святем храме сем и с верою, благоговением и страхом Божиим входящих в онь, Господу помолимся. 


О Великом Господине и Отце нашем Святейшем Патриархе Алексии, и о Господине нашем Преосвященнейшем митрополите\itshape  (или:\normalfont{} архиепископе\itshape , или:\normalfont{} епископе\itshape ) (имярек)\normalfont{}, честнем пресвитерстве, во Христе диаконстве, о всем причте и людех, Господу помолимся. 


О Богохранимей стране нашей, властех и воинстве ея, Господу помолимся.


О граде сем\itshape  (или:\normalfont{} o веси сей\itshape , если в монастыре, то: о\normalfont{} святей обители сей), всяком граде, стране и верою живущих в них, Господу помолимся. 


О благорастворении воздухов, о изобилии плодов земных и временех мирных, Господу помолимся.


О плавающих, путешествующих, недугующих, страждущих, плененных и о спасении их, Господу помолимся. 


О избавитися нам от всякия скорби, гнева и нужды, Господу помолимся. 


Заступи, спаси, помилуй и сохрани нас, Боже, Твоею благодатию. 


Пресвятую, Пречистую, Преблагословенную, Славную Владычицу нашу Богородицу и Приснодеву Марию, со всеми святыми помянувше, сами себе, и друг друга, и весь живот наш Христу Богу предадим. 


\itshape Хор:\normalfont{} Тебе, Господи. 


\itshape Иерей:\normalfont{} Яко подобает Тебе всякая слава, честь и поклонение, Отцу и Сыну и Святому Духу, ныне и присно и во веки веков. 


\itshape Хор:\normalfont{} Аминь.


\medskip
\bfseries Антифоны\normalfont{}


Антифоны на литургии бывают трех родов: праздничные, изобразительные и вседневные (будничные). Какие из них поются, определяется на каждый день церковным Уставом. Праздничные антифоны поются в праздники Господни, за исключением Сретения (праздничные антифоны в Неделю ваий, на Пасху, на Вознесение и в День Святой Троицы приведены в главе «Песнопения из служб Триоди цветной».


Вседневные антифоны положено петь в будни. Чаще всего в воскресные и праздничные дни, поются антифоны изобразительные (псалмы 102, 145 и Блаженны "--- Мф. 5, 3-12)


\medskip
\bfseries Первый антифон \normalfont{}


1. Благослови, душе моя, Господа. Благословен еси, Господи. Благослови, душе моя, Господа, и вся внутренняя моя Имя святое Его. 


2. Благослови, душе моя, Господа, и не забывай всех воздаяний Его. 


1. Очищающаго вся беззакония твоя,  исцеляющаго вся недуги твоя. 


2. Избавляющаго от истления живот твой,  венчающаго тя милостию и щедротами. 


1. Исполняющаго во благих желание твое:  обновится, яко орля, юность твоя. 


2. Творяй милостыни Господь, и судьбу всем обидимым. 


1. Сказа пути Своя Моисеови, сыновом Израилевым хотения Своя. 


2. Щедр и милостив Господь, долготерпелив и многомилостив.


1. Не до конца прогневается, ниже в век враждует. 


2. Не по беззаконием нашим сотворил есть нам, ниже по грехом нашим воздал есть нам. 


1. Яко по высоте небесней от земли,  утвердил есть Господь милость Свою на боящихся Его. 


2. Елико отстоят востоцы от запад,  удалил есть от нас беззакония наша. 


1. Якоже щедрит отец сыны, ущедри Господь боящихся Его. 


2. Яко Той позна создание наше, помяну, яко персть есмы. 


1. Человек, яко трава дние его, яко цвет сельный, тако оцветет. 


2. Яко дух пройде в нем, и не будет, и не познает ктому места своего. 


1. Милость же Господня от века и до века на боящихся Его. 


2. И правда Его на сынех сынов, хранящих завет Его, и помнящих заповеди Его творити я. 


1. Господь на небеси уготова Престол Свой, и Царство Его всеми обладает. 


2. Благословите Господа, ангели Его,  сильнии крепостию, творящии слово Его, услышати глас словес Его. 


1 . Благословите Господа, вся силы Его, слуги Его, творящии волю Его. 


2. Благословите Господа, вся дела Его, на всяком месте владычества Его. 


1. Слава Отцу и Сыну и Святому Духу. 


2. И ныне и присно и во веки веков. Аминь. 


1. Благослови, душе моя, Господа, и вся внутренняя моя, имя святое Его. Благословен еси, Господи.


\medskip
\bfseries Ектения малая \normalfont{}


\medskip
\itshape Диакон:\normalfont{} Паки и паки миром Господу помолимся. 


\itshape Хор:\normalfont{} Господи, помилуй. 


\itshape Диакон:\normalfont{} Заступи, спаси, помилуй и сохрани нас, Боже, Твоею благодатию. 


\itshape Хор:\normalfont{} Господи, помилуй. 


\itshape Диакон:\normalfont{} Пресвятую, Пречистую, Преблагословенную, Славную Владычицу нашу Богородицу и Приснодеву Марию, со всеми святыми помянувше, сами себе, и друг друга, и весь живот наш Христу Богу предадим. 


\itshape Хор:\normalfont{} Тебе, Господи. 


\itshape Иерей:\normalfont{} Яко Твоя держава, и Твое есть Царство, и сила, и слава, Отца и Сына и Святаго Духа, ныне и присно и во веки веков. 


\itshape Хор:\normalfont{} Аминь. 


\itshape Во время второго антифона зажигается пономарская свеча. Алтарник берёт свечу во время «Единородный Сыне...» и становится с ней на горнем месте.\normalfont{}


\medskip
\bfseries Второй антифон \normalfont{}


1. Хвали, душе моя, Господа. Восхвалю Господа в животе моем, пою Богу моему, дондеже есмь. 


2. Не надейтеся на князи, на сыны человеческия, в них же несть спасения. 


1. Изыдет дух его, и возвратится в землю свою: в той день погибнут вся помышления его. 


2. Блажен, емуже Бог Иаковль помощник его, упование его на Господа Бога своего. 


1. Сотворшаго небо и землю, море и вся, яже в них. 


2. Хранящаго истину в век, творящаго суд обидимым, дающаго пищу алчущим. 


1. Господь решит окованныя, Господь умудряет слепцы. 


2. Господь возводит низверженныя, Господь любит праведники. 


1. Господь хранит пришельцы, сира и вдову приимет, и путь грешных погубит. 


2. Воцарится Господь во век, Бог твой, Сионе, в род и род. И ныне и присно и во веки веков. Аминь. 


\medskip
\bfseries Песнь Господу Иисусу Христу \normalfont{}


Единородный Сыне и Слове Божий, Безсмертен Сый, и изволивый спасения нашего ради воплотитися от Святыя Богородицы и Приснодевы Марии, непреложно вочеловечивыйся; распныйся же, Христе Боже, смертию смерть поправый, един Сый Святыя Троицы, спрославляемый Отцу и Святому Духу, спаси нас. 


\medskip
\bfseries Ектения малая \normalfont{}


\itshape Диакон:\normalfont{} Паки и паки миром Господу помолимся. 


\itshape Хор:\normalfont{} Господи, помилуй. 


\itshape Диакон:\normalfont{} Заступи, спаси, помилуй и сохрани нас, Боже, Твоею благодатию. 


\itshape Хор:\normalfont{} Господи, помилуй. 


\itshape Диакон:\normalfont{} Пресвятую, Пречистую, Преблагословенную, Славную Владычицу нашу Богородицу и Приснодеву Марию, со всеми святыми помянувше, сами себе, и друг друга, и весь живот наш Христу Богу предадим.


\itshape Хор:\normalfont{} Тебе, Господи. 


\itshape Иерей:\normalfont{} Яко Твоя держава, и Твое есть Царство, и сила, и слава, Отца и Сына и Святаго Духа, ныне и присно и во веки веков. 


\itshape Хор:\normalfont{} Аминь.


\medskip
\bfseries Третий антифон; Блаженны \normalfont{}


«Блаженны» положено петь с тропарями, назначенными в этот день церковным Уставом: особыми тропарями на «Блаженных», или тропарями из песней утреннего канон празднику или святому.


1 . Во Царствии Твоем помяни нас, Господи, егда приидеши во Царствии Твоем.\itshape  На \normalfont{}12 


1 . Блажени нищий духом, яко тех есть Царство Небесное. 


2. Блажени плачущии, яко тии утешатся.\itshape  На \normalfont{}10 


1. Блажени кротции, яко тии наследят землю. 


2. Блажени алчущии и жаждущии правды, яко тии насытятся.\itshape  На\normalfont{} 8 


1. Блажени милостивии, яко тии помиловани будут. 


2. Блажени чистии сердцем, яко тии Бога узрят.\itshape  На\normalfont{} 6 


1. Блажени миротворцы, яко тии сынове Божии нарекутся. 


2. Блажени изгнани правды ради, яко тех есть Царство Небесное.\itshape  На\normalfont{} 4 


1. Блажени есте, егда поносят вам, и изженут, и рекут всяк зол глагол на вы, лжуще Мене ради. 


2. Радуйтеся и веселитеся, яко мзда ваша многа на небесех. 


Слава Отцу и Сыну и Святому Духу. 


И ныне и присно и во веки веков. Аминь. 


\medskip


\bfseries Антифоны вседневные (будничные) \normalfont{}


Антифон 1-й 


1. Благо есть исповедатися Господеви. Молитвами Богородицы, Спасе, спаси нас. 


2. Благо есть исповедатися Господеви, и пети имени Твоему, Вышний. Молитвами Богородицы, Спасе, спаси нас 


1. Возвещати заутра милость Твою, и истину Твою на всяку нощь. Молитвами Богородицы, Спасе, спаси нас 


2. Яко прав Господь Бог наш, и несть неправды в Нем. Молитвами Богородицы, Спасе, спаси нас


1. Слава Отцу и Сыну и Святому Духу: Молитвами Богородицы, Спасе, спаси нас 


2. И ныне и присно и во веки веков. Амин. Молитвами Богородицы, Спасе, спаси нас

Антифон 2-й 


1. Господь воцарися, в лепоту облечеся. Молитвами святых Твоих, Спасе, спаси нас. 


2. Господь воцарися, в лепоту облечеся,/ облечеся Господь в силу, и препоясася. Молитвами святых Твоих, Спасе, спаси нас 


1. Ибо утверди вселенную, яже не подвижится. Молитвами святых Твоих, Спасе, спаси нас 


2. Свидения Твоя уверишася зело: дому Твоему подобает святыня, Господи, в долготу дний. Молитвами святых Твоих, Спасе, спаси нас Слава, и ныне:


\medskip
\bfseries  Песнь Господу Иисусу Христу\normalfont{} 


Единородный Сыне и Слове Божий, Безсмертен Сый, и изволивый спасения нашего ради воплотитися от Святыя Богородицы и Приснодевы Марии, непреложно вочеловечивыйся; распныйся же, Христе Боже, смертию смерть поправый, един Сый Святыя Троицы, спрославляемый Отцу и Святому Духу, спаси нас. 

\medskip
Антифон 3-й 


1. Приидите возрадуемся Господеви, воскликнем Богу Спасителю нашему. Спаси ны, Сыне Божий, во святых дивен сый, поющия Ти: аллилуиа. 


2. Предварим лице Его во исповедании, и во псалмех воскликнем Ему: Спаси ны, Сыне Божий, во святых дивен сый, поющия Ти: аллилуиа. 


1. Яко Бог Велий Господь, и Царь Велий по всей земли. Спаси ны, Сыне Божий, во святых дивен сый, поющия Ти: аллилуиа. 


2. Яко в руце Его вси концы земли, и высоты гор Того суть. Спаси ны, Сыне Божий, во святых дивен сый, поющия Ти: аллилуиа. 


1 . Яко Того есть море, и Той сотвори е, и сушу руце Его создаете. Спаси ны, Сыне Божий, во святых дивен сый, поющия Ти: аллилуиа. 


\itshape Вход с Евангелием. Дьякон заходит в алтарь, открывает Царские врата, вместе со священником крестится и целует престол и берёт евангелие, алтарник в этот момент крестится с ними синхронно, кланяется, горнему месту, священнику и в момент перехода священника от престола к горнему месту идёт к северным вратам. Когда священник с диаконом тоже направятся к вратам, открывает дверь и по амвону проходит до царских врат, затем сворачивает к аналою и становится перед ним спиной к народу, когда священник зайдёт в алтарь, алтарник заходит через южные врата. В алтаре понамарь проходит до горнего места, крестится, кланяется горнему месту, священнику и проходит, чтобы поставить свечу на место.\normalfont{}


\medskip


\bfseries Вход с Евангелием \normalfont{}


\itshape Диакон:\normalfont{} Премудрость, прости.


\itshape  Хор:\normalfont{} Приидите, поклонимся и припадем ко Христу. Спаси Сыне Божий, воскресый из мертвых, поющия Ти: аллилуиа.


[B таком виде это песнопение поется во все обычные воскресенья, на Пасху и все дни пасхальной седмицы. Вместо слов «воскресый из мертвых» в будни поется «во святых Дивен сын», а на праздники "--- по смыслу праздника, как указано в Церковном уставе: на Рождество "--- «рождейся от Девы»; на Крещение "--- «во Иордане крестивыйся»; на Вознесение "--- «вознесыйся ко славе» в Пятидесятницу и в День Святого Духа "--- «Спаси ны. Утешителю Благий»; на Преображение "--- «преобразивыйся на горе»; на Воздвижение "--- «плотию распныйся»; в Неделю ваий "--- «возседый на жребя». В праздники Богородицы "--- «молитвами Богородицы». Праздничные входные стихи поются и в дни попразднства, до отдания.]


\medskip
\bfseries Тропари и кондаки «по входе» \normalfont{}


\itshape Хор поет тропари и кондаки «по входе», назначенные в этот день церковным Уставом (воскресные тропари и кондаки приведены в главе «Песнопения из служб воскресных», дневные "--- в главе «Песнопения из служб будничных», общие ликам святых "--- в главе «Песнопения из служб общих ликам святых», праздничные "--- в главе «Песнопения из служб праздничных»).\normalfont{}


\itshape Иерей:\normalfont{} Яко Свят еси, Боже наш, и Тебе славу возсылаем, Отцу и Сыну и Святому Духу, ныне и присно.


\itshape Диакон:\normalfont{} И во веки веков.


\itshape Хор:\normalfont{} Аминь.




\medskip


\bfseries Трисвятое \normalfont{}


(В праздники Рождества Христова, Богоявления, в Лазареву и Великую субботы, во все дни пасхальной седмицы и в период Пятидесятницы вместо Трисвятого поется: «Елицы во Христа крестистеся, во Христа облекостеся. Аллилуиа». В праздник Воздвижения Креста Господня и в Неделю крестопоклонную поется: «Кресту Твоему покланяемся, Владыко, и Святое Воскресение Твое славим)


Святый Боже, Святый Крепкий, Святый Безсмертный, помилуй нас. \itshape (Трижды)\normalfont{}


Слава Отцу и Сыну и Святому Духу, и ныне и присно и во веки веков. Аминь. Святый Безсмертный, помилуй нас.


Святый Боже, Святый Крепкий, Святый Бессмертный, помилуй нас.


[В праздники Рождества Христова, Богоявления, в Лазареву и Великую субботы, во все дни пасхальной седмицы и в период Пятидесятницы вместо Трисвятого поется: «Елицы во Христа крестистеся, во Христа облекостеся. Аллилуя». В праздник Воздвижения Креста Господня и в Неделю крестопоклонную поется: «Кресту Твоему поклоняемся Владыко, и Святое Воскресение Твое славим»]




\medskip


\bfseries Прокимен \normalfont{}


\itshape Дьякону подаётся кадило\normalfont{}


\itshape Диакон:\normalfont{} Вонмем.


\itshape Иерей:\normalfont{} Мир всем.


\itshape Чтец Апостола:\normalfont{} И духови твоему. Прокимен. Псалом Давидов, глас..




[В Богородичные праздники: «Прокимен, песнь Богородицы: Величит душа Моя Господа/ и возрадовася дух Мой о Бозе Спасе Моем». ]


Произносится один или два прокимна, назначенные в этот день на литургии церковным Уставом (воскресные прокимны со своими стихами приведены в главе «Песнопения из служб воскресных восьми гласов», дневные (будничные) "--- в главе «Песнопения из служб будничных», из служб Триодей постной и цветной "--- в главах «Песнопения из служб Триоди постной» и «Песнопения из служб Триоди цветной».


Чтец произносит Прокимен, называя глас его, хор поет прокимен, чтец произносит стих, хор повторяет прокимен, чтец произносит первую половину прокимна, хор поет вторую половину его. Когда Устав назначает два прокимна, первый поется дважды, т.е. чтец: прокимен, хор: прокимен, чтец: стих, хор: прокимен, затем чтец произносит второй прокимен, и хор поет его один раз.


\medskip
\bfseries Прокимны и аллилуиарии воскресные на литургии\normalfont{}


\itshape Глас 1-й:\normalfont{} Буди, Господи, милость Твоя на нас, якоже уповахом на Тя.


\itshape Стих:\normalfont{} Радуйтеся, праведнии, о Господе, правым подобает похвала.


\itshape Аллилуиа:\normalfont{} Бог даяй отмщение мне и покоривый люди под мя.


\itshape Стих:\normalfont{} Величай спасения царева и творяй милость Христу Своему Давиду и семени его до века.


\itshape Глас 2-й:\normalfont{} Крепость моя и пение мое Господь. и бысть мне во спасение.


\itshape Стих:\normalfont{} Наказуя наказа мя Господь, смерти же не предаде мя.


\itshape Аллилуиа:\normalfont{} Услышит тя Господь в день печали, защитит тя имя Бога Иаковля.


\itshape Стих:\normalfont{} Господи, спаси царя и услыши ны, в оньже аще день призовем Тя.


\itshape Глас 3-й:\normalfont{} Пойте Богу нашему, пойте пойте Цареви нашему, пойте.


\itshape Стих:\normalfont{} Вси языцы, восплещите руками, воскликните Богу гласом радования.


\itshape Аллилуиа:\normalfont{} На Тя, Господи, уповах, да не постыжуся во век.


\itshape Стих:\normalfont{} Буди ми в Бога Защитителя и в дом прибежища, еже спасти мя.


\itshape Глас 4-й:\normalfont{} Яко возвеличишася дела Твоя, Господи, вся премудростию сотворил еси.


\itshape Стих:\normalfont{} Благослови, душе моя, Господа, Господи Боже мой, возвеличился еси зело.


\itshape Аллилуиа:\normalfont{} Наляцы и успевай и царствуй, истины ради и кротости, и правды.


\itshape Стих:\normalfont{} Возлюбил еси правду и возненавидел еси беззконие.


\itshape Глас 5-й:\normalfont{} Ты, Господи, сохраниши ны и соблюдеши ны от рода сего и во век.


\itshape Стих:\normalfont{} Спаси мя, Господи, яко оскуде преподобный.


\itshape Аллилуиа:\normalfont{} Милости Твоя, Господи, во век воспою, в род и род возвещу истину Твою усты моими.


\itshape Стих:\normalfont{} Зане рекл еси: в век милость созиждется, на небесех уготовится истина Твоя.


\itshape Глас 6-п:\normalfont{} Спаси, Господи, люди Твоя и благослови достояние Твое.


\itshape Стих:\normalfont{} К Тебе, Господи, воззову, Боже мой, да не премолчиши от мене.


\itshape Аллилуиа:\normalfont{} Живый в помощи Вышняго, в крове Бога Небеснаго водворится.


\itshape Стих:\normalfont{} Речет Господеви: Заступник мой еси и Прибежище мое, Бог мой, и уповаю на Него.


\itshape Глас 7-й:\normalfont{} Господь крепость людем Своим даст Господь благословит люди Своя миром.


\itshape Стих:\normalfont{} Принесите Господеви сынове Божии, принесите Господеви сыны овни,


\itshape Аллилуиа:\normalfont{} Благо есть исповедатися Господеви и пети Имени Твоему, Вышний.


\itshape Стих:\normalfont{} Возвещати заутра милость Твою, и истину Твою на всяку нощь.


\itshape Глас 8-й:\normalfont{} Помолитеся и воздадите Господеви Богу нашему.


\itshape Стих:\normalfont{} Ведом во Иудеи Бог, во Израили велие Имя Его.


\itshape Аллилуиа:\normalfont{} Приидите, возрадуемся Господеви, воскликнем Богу Спасителю нашему.


\itshape Стих:\normalfont{} Предварим лице Его во исповедании, и во псалмех воскликнем Ему.




\medskip
\bfseries Прокимны и аллилуиарии дневные(будничные)\normalfont{}


\itshape В понедельник, гл. 4-й:\normalfont{} Творяй ангелы Своя духи, и слуги Своя пламень огненный.


\itshape Стих:\normalfont{} Благослови, душе моя. Господа, Господи Боже мой, возвеличился еси зело.


\itshape Аллилуиа, гл. 5-й:\normalfont{} Хвалите Господа, вси ангели Его, хвалите Его, вся силы Его.


\itshape Стих:\normalfont{} Яко Той рече, и быша; Той повеле, и создашася.


\itshape Во вторник, гл. 7-й:\normalfont{} Возвеселится праведник о Господе и уповает на Него.


\itshape Стих:\normalfont{} Услыши, Боже, глас мой, внегда молитися ми к Тебе.


\itshape Аллилуиа, гл. 4-й:\normalfont{} Праведник яко финикс процветет, яко кедр, иже в Ливане, умножится


\itshape Стих:\normalfont{} Насаждени в дому Господни, во дворех Бога нашего процветут.


\itshape В сред\normalfont{}у\itshape , гл. 3-й:\normalfont{} Величит душа Моя Господа, и возрадовася дух Мой о Бозе Спасе Моем


\itshape Стих:\normalfont{} Яко призре на смирение рабы Своея, се бо отныне ублажат Мя вси роди.


\itshape Аллилуиа, гл. 8-й:\normalfont{} Слыши, Дщи, и виждь, и приклони ухо Твое.


\itshape Стих:\normalfont{} Лицу Твоему помолятся богатии людстии.


\itshape В четверг, гл. 8-й:\normalfont{} Во всю землю изыде вещание их, и в концы вселенныя глаголы их.


\itshape Стих:\normalfont{} Небеса поведают славу Божию, творение же руку Его возвещает твердь.


\itshape Аллияуиа, гл. 1-й;\normalfont{} Исповедят небеса чудеса, Господи, ибо истину Твою в Церкви святых


\itshape Стих:\normalfont{} Бог прославляем в совете святых.


\itshape В пятницу, гл. 7-й:\normalfont{} Возносите Господа Бога нашего, и покланяйтеся подножию ногу Его, яко свято есть.


\itshape Стих:\normalfont{} Господь воцарися, да гневаются людие.


\itshape Аллилуиа, гл. 1-й:\normalfont{} Помяни сонм Твой, егоже стяжал еси исперва.


\itshape Стих:\normalfont{} Бог же Царь наш прежде века, содела спасение посреди земли. 


\itshape В субботу, гл. 8-й:\normalfont{} Веселитеся о Господе, и радуйтеся, праведнии.


\itshape  Стих:\normalfont{} Блажени, ихже оставишася беззакония и ихже прикрышася греси.


\itshape  Заупокойный, гл. 6-й:\normalfont{} Души их/ во благих водворятся.


\itshape  Аллилуиа, гл. 4-й:\normalfont{} Воззваша праведнии, и Господь услыша их, и от всех скорбей их избави их.


\itshape  Стих:\normalfont{} Многи скорби праведным, и от всех их избавит я Господь.


\itshape  Стих:\normalfont{} Блажени, яже избрал и приял еси, Господи, и память их в род и род.


\itshape Диакон:\normalfont{} Премудрость.


\itshape  Чтец:\normalfont{} Деяний святых апостол чтение.


\itshape (Или:\normalfont{} Соборнаго послания Петрова \itshape [или:\normalfont{} Иоаннова\itshape , причем не принято говорить, какое это послание "--- первое, или второе, или третье\normalfont{}] чтение\itshape . Или:\normalfont{} К римляном [К коринфяном; К галатом; К Тимофе\itshape ю и т.п.\normalfont{}] послания святаго апостола Павла чтение.)\itshape  Диакон:\normalfont{} Вонмем.


\medskip
\bfseries Чтение Апостола \normalfont{}

\itshape Во время чтения апостола, ставится аналой на амвоне для Евангелия. Когда чтение закончится, иерей говорит чтецу:\normalfont{} Мир ти.


\itshape  Чтец:\normalfont{} И духови твоему.


\medskip
\bfseries Аллилуиа \normalfont{}


\itshape Диакон:\normalfont{} Премудрость.
\itshape Чтец:\normalfont{} Аллилуиа, глас...\itshape  Если прислуживает один алтарник, то выносится пономарская свеча и ставится перед аналоем (с Евангелием), если два алтарника то во время пения аллилуя вдвоём подходят на горнее место со свечами, синхронно крестятся, кланяются горнему месту, священнику, друг другу, и выходят на амвон северными и южными вратами, до чтения евангелия стоят лицом к иконостасу, не кланяясь и не крестясь, в начале чтения поворачиваются лицом к евангелию, по окончанию кланяются иконам и заходят теми же вратами в алтарь, также крестятся и кланяются на горнем месте и проходят, чтобы поставить свечи на место. Не забудьте убрать аналой.\normalfont{}


\medskip
Хор поет «Аллилуиа»- трижды на указанный глас, чтец произносит первый стих аллилуиария, хор: «Аллилуиа», чтец произносит второй стих аллилуиария, хор поет в третий раз «Аллилуиа». В богослужебных книгах перед первым стихом аллилуария пишется «Аллилуиа, глас...», а перед вторым "--- «Стих» (воскресные аллилуиарии приведены в главе «Песнопения из служб воскресных восьми гласов», дневные (будничные) "--- в главе «Песнопения из служб будничных», аллилуиарии из служб Триодей постной и цветной "--- в главах «Песнопения из служб Триоди постной» и «Песнопения из служб Триоди цветной».)


\itshape Диакон:\normalfont{} Благослови, владыко, благовестителя святаго апостола и евангелиста \itshape (имя евангелиста)\normalfont{}.


\itshape Иерей, благословляя его, произносит:\normalfont{} Бог, молитвами святаго, славнаго, всехвальнаго апостола и евангелиста\itshape  (имярек)\normalfont{}, да даст тебе глагол, благовествующему силою многою, во исполнение Евангелиа Возлюбленнаго Сына Своего, Господа нашего Иисуса Христа.


\itshape Диакон:\normalfont{} Аминь.


\itshape Иерей:\normalfont{} Премудрость, прости, услышим святаго Евангелиа. Мир всем.
\itshape Хор:\normalfont{} И духови твоему.


\itshape Диакон:\normalfont{} От \itshape (имя\normalfont{}) святаго Евангелиа чтение.


\itshape Хор$<$\normalfont{}$>$: Слава Тебе, Господи, слава Тебе.


\itshape Иерей:\normalfont{} Вонмем.


\medskip
\bfseries Чтение Евангелия\normalfont{}


\itshape Читается Евангелие. Церковный Устав назначает определенные евангельские чтения на каждый день (евангельские чтения Пресвятой Богородице о общие ликам святых приведены в главе «Песнопения из служб общих ликам святых»).\normalfont{}


\itshape По окончании чтения хор:\normalfont{} Слава Тебе, Господи, слава Тебе.


\itshape Выносятся записки о здравии и о упокоении.\normalfont{}


\medskip
\bfseries Ектения сугубая\normalfont{}


\itshape Диакон:\normalfont{} Рцем вси от всея души, и от всего помышления нашего рцем.


\itshape  Хор:\normalfont{} Господи, помилуй. Господи Вседержителю, Боже отец наших, молим Ти ся, услыши и помилуй.


\itshape  Хор:\normalfont{} Господи, помилуй. Помилуй нас, Боже, по велицей милости Твоей, молим Ти ся, услыши и помилуй.


\itshape  Хор:\normalfont{} Господи, помилуй. \itshape (Трижды, на каждое прошение)\normalfont{} 


Еще молимся о Великом Господине и Отце нашем Святейшем Патриархе\itshape  (имярек)\normalfont{}, и о Господине нашем Преосвященнейшем митрополите\itshape  (или:\normalfont{} архиепископе\itshape , или:\normalfont{} епископе\itshape ) (имярек)\normalfont{}, и всей во Христе братии нашей. 


Еще молимся о Богохранимей стране нашей, властех и воинстве ея, да тихое и безмолвное житие поживем во всяком благочестии и чистоте. 


Еще молимся о братиях наших, священицех, священномонасех и всем во Христе братстве нашем.


Еще молимся о блаженных и приснопамятных создателех святаго храма сего \itshape (если в монастыре:\normalfont{} святыя обители сея), и о всех преждепочивших отцех и братиях. зде лежащих и повсюду, православных. 


Еще молимся о милости, жизни, мире, здравии, спасении, посещении, прощении и оставлении грехов рабов Божиих. братии святаго храма сего \itshape (если в монастыре:\normalfont{} святыя обители сея). 


Еще молимся о плодоносящих и добродеющих во святем и всечестнем храме сем, труждающихся, поющих и предстоящих людех, ожидающих от Тебе великия и богатыя милости.


\itshape Иерей:\normalfont{} Яко Милостив и Человеколюбец Бог еси, и Тебе славу возсылаем, Отцу и Сыну и Святому Духу, ныне и присно и во веки веков.


\itshape Хор\normalfont{}: Аминь.


\itshape [В некоторые дли церковного года (кроме двунадесятых и храмовых праздников) за сугубой ектенией читается следующая ектения об усопших, при открытых царских аратах и с кадильницей:\normalfont{}


\medskip
\bfseries Ектения заупокойная \normalfont{}


\itshape Диакон:\normalfont{} Помилуй нас, Боже, по велицей милости Твоей, молим Ти ся, услыши и помилуй.


\itshape  Хор:\normalfont{} Господи помилуй.\itshape  (на каждое прошение)\normalfont{}. 


Еще молимся о упокоении душ усопших рабов Божиих\itshape  (имена)\normalfont{} и о еже проститися им всякому прегрешению, вольному же и невольному.


Яко да Господь Бог учинит души их, идеже праведнии упокояются. Милости Божия, Царства Небеснаго и оставления грехов их у Христа, Безсмертнаго Царя и Бога нашего, просим.


\itshape Хор:\normalfont{} Подай, Господи.


\itshape Диакон:\normalfont{} Господу помолимся.


\itshape Хор:\normalfont{} Господи, помилуй.


\itshape Иерей:\normalfont{} Яко Ты еси воскресение, и живот, и покой усопших раб Твоих \itshape (имени)\normalfont{}, Христе Боже наш, и Тебе славу возсылаем, со Безначальным Твоим Отцем и Пресвятым и Благим и Животворящим Твоим Духом, ныне и присно и во веки веков.


\itshape Хор\normalfont{}: Аминь.\itshape  Царские врата закрываются.\normalfont{}


\medskip
\bfseries Ектения об оглашенных \normalfont{}


\itshape Диакон:\normalfont{} Помолитеся, оглашеннии, Господеви.


\itshape Хор:\normalfont{} Господи, помилуй, \itshape (На каждое прошение,)\normalfont{}. Вернии, о оглашенных помолимся, да Господь помилует их. Огласит их словом истины. Открыет им Евангелие правды. Соединит их Святей Своей, Соборней и Апостольстей Церкви. Спаси, помилуй, заступи и сохрани их, Боже, Твоею благодатию. Оглашеннии, главы ваша Господеви приклоните.


\itshape Хор:\normalfont{} Тебе, Господи. Да и тии с нами славят пречестное и великолепое Имя Твое, Отца и Сына и Святаго Духа, ныне и присно и во веки веков.


\itshape Хор:\normalfont{} Аминь.\itshape  Диакон:\normalfont{} Елицы оглашеннии, изыдите, оглашеннии, изыдите; елицы оглашеннии, изыдите. Да никто от оглашенных, елицы вернии, паки и паки миром Господу помолимся.


\itshape Хор:\normalfont{} Господи, помилуй.


\itshape Возглашением диакона: «Оглашеннии, изыдите...» заканчивается вторая часть литургии.\normalfont{}

\mychapterending

\mychapter{Литургия верных}
%http://www.molitvoslov.com/node/456 
 


\bfseries Литургия верных\normalfont{} "--- третья, самая важная часть литургии, на которой Св. Дары, приготовленные на проскомидии, силою и действием Святого Духа пресуществляются в Тело и Кровь Христовы и возносятся в спасительную для людей жертву Богу Отцу, а затем преподаются верующим для причащения. Эта часть литургии получила название оттого, что присутствовать при ее совершении и приступать к причащению Св. Тайн могут только верные, то есть лица, принявшие православную веру через Св. Крещение и оставшиеся верными обетам, данным при Св. Крещении.



  На литургии верных воспоминаются страдания Господа Иисуса Христа, Его смерть, погребение, Воскресение, Вознесение на небо, седение одесную Бога Отца и второе славное пришествие на землю.


\medskip


В состав этой части литургии входят важнейшие священнодействия:


  1. Перенесение Честных Даров с жертвенника на престол, приготовление верующих молитвенному участию при совершении Бескровной Жертвы.


  2. Самое совершение Таинства, с молитвенным воспоминанием членов Церкви Небесной и земной. 


  3. Приготовление к причащению и причащение священнослужителей и мирян.


  4. Благодарение за причащение и благословение на исход из храма (отпуст).


\medskip


 \bfseries Ектении\normalfont{}


\itshape Диакон от лица верных произносит две ектении:\normalfont{}


 \itshape  Диакон:\normalfont{} Заступи, спаси, помилуй и сохрани нас, Боже, Твоею благодатию.


\itshape Хор:\normalfont{} Господи, помилуй.


\itshape Диакон:\normalfont{} Премудрость.


\itshape  Иерей:\normalfont{} Яко подобает Тебе всякая слава, честь и поклонение, Отцу и Сыну и Святому Духу, ныне и присно и во веки веков.


\itshape  Хор:\normalfont{} Аминь.


\itshape  Готовится кадило, зажигаются понамарская свеча.\normalfont{}


 \itshape  Диакон:\normalfont{} Паки и паки миром Господу помолимся. \itshape  


  Хор:\normalfont{} Господи, помилуй.\itshape  (На каждое прошение)\normalfont{}. 


  О свышнем мире и спасении душ наших. Господу помолимся. 


  О мире всего мира, благостоянии святых Божиих Церквей и соединении всех, Господу помолимся. 


  О святем храме сем и с верою, благоговением и страхом Божиим входящих в онь, Господу помолимся. 


  О избавитися нам от всякия скорби, гнева и нужды, Господу помолимся. 


  Заступи, спаси, помилуй и сохрани нас, Боже, Твоею благодатию. \itshape 


  Диакон:\normalfont{} Премудрость. \itshape 


  Иерей:\normalfont{} Яко да под державою Твоею всегда храними, Тебе славу возсылаем, Отцу и Сыну и Святому Духу, ныне и присно и во веки веков. 


 \itshape (отверзаются царские врата.)\normalfont{}


 \itshape Подаётся кадило. Алтарник становится на горнем месте так, чтобы не мешать каждению. Когда каждение закончится и дьякон зайдёт в алтарь, синхронно крестится и кланяется со священнослужителями, на третий раз кланяется как обычно (горнее место, священник) и проходит к северным вратам. По сигналу священника открывает дверь и выходит как обычно к аналою. Стоит перед аналоем до тех пор пока не закроются Царские врата. По обычаю заходит в Алтарь\normalfont{}. 


\itshape  Хор:\normalfont{} Аминь, \itshape и поет Херувимскую песнь \normalfont{}


\medskip


 \bfseries Херувимская песнь \normalfont{}


  Иже Херувимы тайно образующе и Животворящей Троице Трисвятую песнь припевающе, всякое ныне житейское отложим попечение... 


\medskip


\bfseries  Великий вход \normalfont{}


\itshape  Диакон и иерей, взяв Святые Дары, выходят из алтаря на солею.\normalfont{}


\itshape  Диакон:\normalfont{} Великаго Господина и Отца нашего\itshape  (имярек),\normalfont{} Святейшаго Патриарха Московскаго и всея Руси, и Господина нашего Преосвященнейшаго\itshape  (имя епархиального архиерея),\normalfont{} да помянет Господь Бог во Царствии Своем, всегда, ныне и присно и во веки веков.


\itshape  Иерей:\normalfont{} Вас и всех православных христиан да помянет Господь Бог во Царствии Своем, всегда, ныне и прирно и во веки веков.


\itshape Хор:\normalfont{} Аминь. Яко да Царя всех подымем ангельскими невидимо дориносима чинми. Аллилуиа, аллилуиа, аллилуиа.


 \itshape  [ Вместо Херувимской на литургии в Великий четверг поется \normalfont{}"Вечери Твоея Тайныя..\itshape .", а в Великую субботу — \normalfont{}"Да молчит всякая плоть..\itshape ." (эти песнопения приведены в главе "Песнопения из служб Триоди постной").]\normalfont{}



\itshape  Режутся просфоры для причастников\normalfont{}. 


\medskip


 \bfseries Просительная ектения \normalfont{}


\itshape  Диакон:\normalfont{} Исполним молитву нашу Господеви. \itshape 


  Хор:\normalfont{} Господи, помилуй.\itshape  (На каждое прошение\normalfont{}). 


  О предложенных честных Дарех Господу помолимся. 


  О святем храме сем и с верою, благоговением и страхом Божиим входящих в онь, Господу помолимся. 


  О избавитися нам от всякия скорби, гнева и нужды, Господу помолимся. 


  Заступи, спаси, помилуй и сохрани нас, Боже, Твоею благодатию. 


  Дне всего совершенна, свята, мирна и безгрешна, у Господа просим. \itshape 


  Хор:\normalfont{} Подай, Господи.\itshape  (На каждое прошение)\normalfont{}. 


  Ангела мирна, верна наставника, хранителя душ и телес наших, у Господа просим. 


  Прощения и оставления грехов и прегрсшений наших, у Господа просим. 


  Добрых и полезных душам нашим и мира мирови, у Господа просим. 


  Прочее время живота нашего в мире и покаянии скончати, у Господа просим. 


  Христианския кончины живота нашего, безболезнены, непостыдны, мирны, и добраго ответа на Страшном Судищи Христове, просим. 


  Пресвятую, Пречистую, Преблагословенную, Славную Владычицу нашу Богородицу и Приснодеву Марию, со всеми святыми помянувше, сами себе, и друг друга, и весь живот наш Христу Богу предадим.


\itshape Хор:\normalfont{} Тебе, Господи.


\itshape Иерей:\normalfont{} Щедротами Единороднаго Сына Твоего, с Нимже благословен еси, со Пресвятым и Благим и Животворящим Твоим Духом, ныне и присно и во реки веков.


\itshape Хор:\normalfont{} Аминь. \itshape  Иерей:\normalfont{} Мир всем. \itshape  Хор:\normalfont{} И духови твоему. \itshape  Диакон:\normalfont{} Возлюбим друг друга, да единомыслием исповемы.


\itshape Хор:\normalfont{} Отца и Сына и Святаго Духа, Троицу Единосущую и Нераздельную. \itshape  Диакон:\normalfont{} Двери, двери, премудростию вонмем. \itshape  (Открывается завеса царских врат\normalfont{}.)

 


\itshape  Ставится кипятиться чайник\normalfont{}. 


\medskip


 \bfseries  Символ веры \normalfont{}


  Хор (или все молящиеся):


  1. Верую во Единаго Бога Отца Вседержителя, Творца небу и земли, видимым же всем и невидимым. 


  2. И во Единаго Господа Иисуса Христа, Сына Божия, Единороднаго, Иже от Отца рожденнаго прежде всех век. Света от Света, Бога истинна от Бога истинна, рожденна, несотворенна, единосущна Отцу, Имже вся быша.


  3. Нас ради, человек, и нашего ради спасения сшедшаго с Небес, и воплотившагося от Духа Свята и Марии Девы, и вочеловечшася.


  4. Распятаго же за ны при Понтийстем Пилате, и страдавша, и погребенна.


  5. И воскресшаго в третий день по Писанием.


  6. И восшедшаго на Небеса, и седяща одесную Отца.


  7. И паки грядущаго со славою судити живым и мертвым, Его же Царствию не будет конца.


  8. И в Духа Святаго, Господа Животворящаго, Иже от Отца исходящаго, Иже со Отцем и Сыном спокланяема и сславима, глаголавшаго пророки.


  9. Во едину Святую Соборную и Апостольскую Церковь.


  10. Исповедую едино Крещение во оставление грехов.


  11. Чаю воскресения мертвых,


  12. и жизни будущаго века. Аминь. 


\medskip


\bfseries  Евхаристический канон. \normalfont{}


\itshape  Диакон:\normalfont{} Станем добре, станем со страхом, вонмем, Святое Возношение в мире приносити\itshape . 


  Хор:\normalfont{} Милость мира, Жертву хваления. \itshape 


  Иерей:\normalfont{} Благодать Господа нашего Иисуса Христа, и любы Бога и Отца, и причастие Святаго Духа, буди со всеми вами.


\itshape Хор:\normalfont{} И со духом твоим. \itshape 


  Иерей:\normalfont{} Горе имеим сердца. \itshape 


  Хор:\normalfont{} Имамы ко Господу. \itshape 


  Иерей:\normalfont{} Благодарим Господа. \itshape 


  Хор:\normalfont{} Достойно и праведно есть покланятися Отцу и Сыну и Святому Духу, Троице Единосущной и Нераздельней. \itshape  Иерей:\normalfont{} Победную песнь ноюще, вопиюще, взывающе и глаголюще: \itshape 


  Хор:\normalfont{} Свят, Свят, Свят Господь Саваоф, исполнь Небо и земля славы Твоея; осанна в вышних, благословен Грядый во Имя Господне, осанна в вышних. \itshape 


  Иерей:\normalfont{} Приимите, ядите, Сие есть Тело Мое, еже за вы ломимое во оставление грехов. \itshape 


  Хор:\normalfont{} Аминь. \itshape  Иерей:\normalfont{} Пийте от нея вси, сия есть Кровь Моя Новаго Завета, яже за вы и за многи изливаемая во оставление грехов. \itshape  Хор:\normalfont{} Аминь. 


  (На литургии св. Василия Великого последние возгласа иерея начинаются словами: "Даде святым Своим учеником и апостолом, рек:".)


\itshape  Иерей:\normalfont{} Твоя от Твоих Тебе припосяще о всех и за вся. \itshape  Готовится кадило\normalfont{}. \itshape 


Хор:\normalfont{} Тебе поем. Тебе благословим, Тебе благодарим, Господи, и молим Ти ся. Боже наш. \itshape  Подаётся кадило во время "Тебе поем...", после слов священника в алтаре "Приложив Духом Твоим Святым. Аминь. Аминь.Аминь.\normalfont{}"  \itshape 


  Иерей:\normalfont{} Изрядно о Пресвятей, Пречистей, Преблагословенней, Славней Владычице нашей Богородице и Приснодеве Марии. \itshape 


Хор:\normalfont{} Достойно есть, яко воистинну блажити Тя, Богородицу, Присноблаженную и Пренепорочную и Матерь Бога нашего. Честнейшую Херувим и Славнейшую без сравнения Серафим, без истления Бога Слова рождшую, сущую Богородицу Тя величаем. 


 [В двунадесятые праздники и их попразднства вместо "Достойно..." поется припев и ирмос 9-й песни праздничного канона, так называемый "задостойник". В Великий четверг поется ирмос 9-й песни "Странствия Владычня...", в Великую субботу — "Не рыдай Мене, Мати...", в Неделю ваий — "Бог Господь..." (эти песнопения приведены в главах "Песнопения из служб Триоди постной" и "Песнопения из служб Триоди цветной").


 Если же литургия св. Василия Великого, вместо "Достойно... поем: 


      О Тебе радуется. Благодатная, всякая тварь, ангельский собор и человеческий род, освященный храме и раю словесный, девственная похвало, из Неяже Бог воплотися и Младенец бысть, прежде век сый Бог наш; ложесна бо Твоя престол сотвори и чрево Твое пространнее небес содела. О Тебе радуется, Благодатная, всякая тварь, слава Тебе.]


\itshape  Иерей:\normalfont{} В первых помяни Господи, Великаго Господина и Отца нашего\itshape  (имярек),\normalfont{} Святейшаго Патриарха Московскаго и всея Руси, и Господина нашего Преосвященнейшаго\itshape  (имя епархиального епископ\normalfont{}а), ихже даруй святым Твоим Церквам в мире, целых, честных, здравых, долгоденствующих, право правящих слово Твоей истины. \itshape  Хор:\normalfont{} И всех и вся. \itshape  Иерей:\normalfont{} И даждь нам единеми усты и единем сердцем славити и воспевати Пречестно\itshape е\normalfont{} и Всликолепое Имя Твое, Отца и Сына и Святаго Духа, ныне и присно и во веки веков. \itshape 

 


  Хор:\normalfont{} Аминь. \itshape  Иерей:\normalfont{} И да будут милости Великаго Бога и Спаса нашего Иисуса Христа со всеми вами. \itshape 


  Хор:\normalfont{} И со духом твоим. \itshape  Готовится чаша для теплоты и плат для причастия\normalfont{}. 


\medskip


 \bfseries  Ектения просительная \normalfont{} 


\itshape  Диакон:\normalfont{} Вся святыя помянувше, паки и паки миром Господу помолимся. 


\itshape Хор:\normalfont{} Господи, помилуй.\itshape  (На каждое прошение)\normalfont{}. 


  О принесенных и освященных Честных Дарех, Господу помолимся. 


  Яко да Человеколюбец Бог наш, приемь я во святый, и пренебесный, и мысленный Свой Жертвенник, в воню благоухания духовнаго, возниспослет нам Божественную благодать и дар Святаго Духа, помолимся. 


  О избавится нам от всякия скорби, гнева и нужды, Господу помолимся. 


  Заступи, спаси, помилуй и сохрани нас\itshape ,\normalfont{} Твоею благодатию. 


  Дне всего совершенна, свята, мирна и безгрешна,.у Господа просим. \itshape 


  Хор:\normalfont{} Подай, Господи.\itshape  (На каждое прошение)\normalfont{}. \itshape 


  Диакон:\normalfont{} Ангела мирна, верна наставника, хранителя душ и телес наших, у Господа просим. 


  Прощения и оставления грехов и прегрешений наших, у Господа просим. 


  Добрых и полезных душам нашим и мира мирови, у Господа просим. 


  Прочее время живота нашего в мире и покаянии скончати у Господа просим. 


  Христианския кончины живота нашего, безболезнены, непостыдны, мирны, и добраго ответа на Страшнем Судищи Христове, просим. 


  Соединение веры и причастие Святаго Духа испросивше, сами себе, и друг друга, и весь живот наш Христу Богу предадим. \itshape 


  Хор:\normalfont{} Тебе, Господи. 


  И сподоби нас, Владыко, со дерзновением, неосужденно смети призывати Тебе, Небеснаго Бога Отца, и глаголати: 


\medskip


 \bfseries  Отче наш \normalfont{}


\itshape  Хор (или все молящиеся):\normalfont{} Отче наш, Иже еси на Небесех! Да святится Имя Твое, да приидет Царствие Твое, да будет воля Твоя, яко на небеси и на земли. Хлеб наш насущный даждь нам днесь, и остави нам долги наша, якоже и мы оставляем должником нашим; и не введи нас во искушение, но избави нас от лукаваго. 


\itshape  Иерей:\normalfont{} Яко Твое есть Царство, и сила, и слава. Отца и Сына и Святаго Духа, ныне и присно и во веки веков. \itshape 


  Хор:\normalfont{} Аминь. \itshape 


  Иерей:\normalfont{} Мир всем. \itshape 


  Хор:\normalfont{} И духови твоему. \itshape  Диакон:\normalfont{} Главы ваша Господеви приклоните


\itshape Хор:\normalfont{} Тебе, Господи. \itshape  Иерей:\normalfont{} Благодатию, и щедротами, и человеколюбием Единороднаго Сына Твоего, с Нимже благословен еси, со Пресвятым и Благим и Животворящим Твоим Духом, ныне и присно и во веки веков. \itshape 


  Хор:\normalfont{} Аминь. \itshape  (Закрываются царские врата и завеса)


  Диакон:\normalfont{} Вонмем. \itshape 


  Подносится теплот\normalfont{}а \itshape 


  Иерей:\normalfont{} Святая святым. 


\itshape  Хор:\normalfont{} Един Свят, един Господь Иисус Христос, во славу Бога Отца. Аминь. 


\medskip


\bfseries  Причащение священнослужителей \normalfont{}


  В алтаре причащаются священнослужители. 


Хор поет назначенный в этот день церковным Уставом причастен "--- стих, оканчивающихся троекратным "Аллилуиа". Причастнов может быть назначено два, однако "Аллилуия" поется только после второго. 


\medskip


\bfseries  Причастны \normalfont{}


      Во время причастных выносится понамарская свеча и ставится перед Царскими вратами. Затем выносятся запивка и просфоры для причастников.


\itshape  В воскресенье:\normalfont{} Хвалите Господа с небес, хвалите Его в вышних. Аллилуиа, аллилуиа, аллилуиа. 


\itshape В понедельник:\normalfont{} Творяй ангелы Своя духи, и слуги Своя пламень огненный. 


\itshape Во вторник:\normalfont{} В память вечную будет праведник, от слуха зла не убоится. 


\itshape В среду:\normalfont{} Чашу спасения прииму и Имя Господне призову. 


\itshape В четверг:\normalfont{} Во всю землю изыде вещание их, и в концы вселенныя глаголы их. \itshape 


  В пятницу:\normalfont{} Спасение соделал еси посреде земли, Боже. \itshape 


  В субботу:\normalfont{} Радуйтеся, праведнии, о Господе, правым подобает похвала. \itshape 


  Заупокойный:\normalfont{} Блажени, яже избрал и приял еси, Господи, и память их в род и род. \itshape 


  В праздники Богородицы:\normalfont{} Чашу спасения прииму и Имя Господне призову. \itshape 


  В праздники апостолов:\normalfont{} Во всю землю изыде вещание их, и в концы вселенныя глаголы их. 


\itshape В дни памяти святых:\normalfont{} В память вечную будет праведник, от слуха зла не убоится. 


\itshape   Забирается свеча\normalfont{}. \itshape  Открываются царские врата. Диакон, вынося Святую Чашу, возглашает: Со страхом Божиим и верою приступите! 


  (Передает Чашу иерею.) \normalfont{}


\itshape Хор:\normalfont{} Благословен Грядый во Имя Господне, Бог Господь и явися нам.


\itshape [В пасхальную седмицу вместо этого поется "Христос воскресе... ".]\normalfont{}


\itshape  Иерей (и с ним все, желающие причаститься):\normalfont{} Верую, Господи, и исповедую, яко Ты еси воистину Христос, Сын Бога живаго, пришедый в мир грешныя спасти, от них же первый есмь аз. Еще верую, яко Cие есть самое Пречистое Тело Твое, и сия есть самая Честная Кровь Твоя. Молюся убо Тебе: помилуй мя, и прости ми прегрешения моя, вольная и невольная, яже словом, яже делом, яже ведением и неведением, и сподоби мя неосужденно причаститися Пречистых Твоих таинств, во оставление грехов, и в Жизнь Вечную. Аминь.


  Вечери Твоея Тайныя днесь, Сыне Божий, причастника мя приими; не бо врагом Твом тайну повем, ми лобзания Ти дам, яко Ииуда, на яко разбойник исповедаю Тя: помяни мя, Господи, во Царствии Твом.


  Да не в суд или во осуждение будет мне причащение Святых Таин, Господи, но во исцеление души и тела. Аминь.


\medskip


\bfseries Причащение мирян\normalfont{} 


\itshape  Причащая мирян, иерей говори\normalfont{}т: Причащается раб Божий (имя) Честнаго и Святаго Тела и Крове Господа и Бога и Спаса нашего Иисуса Христа, во оставление грехов своих и в Жизнь Вечную. 


Хор \itshape (во время причащения)\normalfont{}: Тело Христово приимите, Источника безсмертнаго вкусите.


 [В Великий четверг вместо этого поется "Вечери Твоея тайныя..." (это песнопение приведено в главе "Песнопения из служб Триоди постной"); в пасхальную седмицу — "Христос воскресе...".]


 \itshape Иерей:\normalfont{} Спаси, Боже, люди Твоя и благослови достояние Твое, 


\itshape Подносится кадило в алтаре.\normalfont{} 


      Хор: Видехом Свет истинный,/ прияхом Духа Небеснаго,/ обретохом веру истинную,/ Нераздельней Троице покланяемся: Та бо нас спасла есть.


      [Вместо "Видехом свет истинный..." от Пасхи до отдания поется "Христос воскресе..."; от Вознесения до отдания — тропарь Вознесения (эти песнопения приведены в главе "Песнопения из служб Триоди цветной"); в Троицкую родительскую субботу — "Глубиною мудрости..." (этот тропарь приведен в главе "Песнопения из служб Триоди цветной", в службе мясопустной родительской субботы).]


\itshape  Иерей:\normalfont{} Всегда, ныне и присно и во веки веков.


\itshape  Хор:\normalfont{} Аминь. Да исполнятся уста наша хваления Твоего, Господи, яко да поем славу Твою, яко сподобил еси нас причаститися Святым Твоим, Божественным, Безсмертным и Животворящим Тайнам; соблюди нас во Твоей святыни, весь день поучатися правде Твоей. Аллилуиа, аллилуиа, аллилуиа. 


\itshape [В Великий четверг вмест\normalfont{}о "Да исполнятся..." поется "Вечери Твоея тайныя..." (это песнопение приведено в главе "Песнопения из служб Триоди постной"); в пасхальную седмицу "--- "Христос воскресе...".]


\medskip


 \bfseries Ектения \normalfont{}


\itshape  Диакон:\normalfont{} Прости приимше Божественных, Святых, Пречистых, Безсмертных, Небесных и Животворящих, Страшных Христовых Тайн, достойно благодарим Господа. \itshape 


  Хор:\normalfont{} Господи, помилуй. 


  Заступи, спаси, помилуй и сохрани нас, Боже, Твоею благодатию. 


  День весь совершен, свят, мирен и безгрешен испросивше, сами себе, и друг друга, и весь живот наш Христу Богу предадим. \itshape 


  Хор:\normalfont{} Тебе, Господи. 


\itshape Иерей:\normalfont{} Яко Ты еси Освящение наше, и Тебе славу возсылаем, Отцу и Сыну и Святому Духу, ныне и присно и во веки веков, \itshape  Хор:\normalfont{} Аминь. \itshape  Иерей: С\normalfont{} миром изыдем, \itshape 


  Хор:\normalfont{} О имени Господни. \itshape 


  Диакон:\normalfont{} Господу помолимся. \itshape  Хор:\normalfont{} Господи, помилуй. 


\medskip


 \bfseries Молитва заамвонная \normalfont{}


\itshape   Иерей (стоя пред амвоном):\normalfont{} Благословляяй благословящия Тя, Господи, и освящаяй на Тя уповающия, спаси люди Твоя и благослови достояние Твое, исполнение Церкве Твоея сохрани, освяти любящия благолепие дому Твоего; Ты тех возпрослави Божественною Твоею силою, и не остави нас, уповающих на Тя. Мир мирови Твоему даруй, Церквам Твоим, священником, воинству и всем людем Твоим. Яко всякое даяние благо, и всяк дар совершен свыше есть, сходяй от Тебе, Отца Светов; и Тебе славу, и благодарение, и поклонение возсылаем, Отцу и Сыну и Святому Духу, ныне и присно и во веки веков. \itshape 


  Хор:\normalfont{} Аминь. Буди Имя Господне благословено отныне и до века \itshape (Tрижд\normalfont{}ы)


  [На пасхальной седмице вместо этого поется "Христос воскресе...".]


\medskip


\bfseries  Псалом 33 \normalfont{}


\bfseries  \normalfont{}\itshape  Хор:\normalfont{} Благословлю Господа на всякое время, выну хвала Его во устех моих. О Господе похвалится душа моя. Да услышат кротции, и возвеселятся. Возвеличите Господа со мною, и вознесем Имя Его вкупе. Взысках Господа, и услыша мя, и от всех скорбей моих избави мя. Приступите к Нему и просветитеся, и лица ваша не постыдятся. Сей нищий воззва, и Господь услыша и, и от всех скорбей его спасе и. Ополчится ангел Господень окрест боящихся Его, и избавит их. Вкусите, и видите, яко благ Господь; блажен муж, иже уповает Нань. Бойтеся Господа вси святии Его, яко несть лишения боящимся Его. Богатии обнищаша и взалкаша: взыскающии же Господа не лишатся всякаго блага. Приидйте, чада, послушайте мене, страху Господню научу вас. Кто есть человек хотяй живот, любяй дни видети благи? Удержи язык твой от зла, и устне твои, еже не глаголати льсти. Уклонися от зла, и сотвори благо, взыщи мира, и пожени и. Очи Господни на праведныя и уши Его в молитву их. Лице же Господне на творящия злая, еже потребити от земли память их. Воззваша праведнии, и Господь услыша их, и от всех скорбей их избави их. Близ Господь сокрушенных сердцем, и смиренныя духом спасет. Многи.скорби праведным, и от всех их избавит я Господь. Хранит Господь вся кости их, ни едина от них сокрушится. Смерть грешников люта, и ненавидящии праведнаго прегрешат. Избавит Господь души раб Своих, и не прегрешат вси уповающий на Него.


  [На пасхальной седмице вместо этого поется "Христос воскресе...".]


 \itshape  Иерей:\normalfont{} Благословение Господне на вас. Того благодатию и человеколюбием, всегда, ныне и присно и во веки веков.


\itshape  Хор:\normalfont{} Аминь. \itshape  Иерей:\normalfont{} Слава Тебе, Христе Боже, Упование наше, слава Тебе.


    [На Пасху, в пасхальную седмицу и в отдание Пасхи вместо "Слава Тебе, Христе Боже..." священнослужители поют "Христос воскресе из мертвых, смертию смерть поправ", а хор заканчивает: "и сущим во гробех живот даровав". 


    От Недели о Фоме до отдания Пасхи священник произносит: "Слава Тебе, Христе Боже, Упование наше, Слава Тебе", я хор поет "Христос воскресе..." (Трижды).] \itshape  Хор: Слава, и ныне.\normalfont{} Господи, помилуй\itshape  (Трижды).\normalfont{} Благослови. 


\medskip


 \bfseries \itshape  Отпуст \normalfont{}\normalfont{}


\itshape  Иерей произносит отпуст. В воскресенье:\normalfont{} Воскресый из мертвых, Христос, истинный Бог наш, молитвами Пречистыя Своея Матере, святых славных и всехвальных Апостол, иже во святых отца нашего Иоанна, архиепископа Константина града, Златоустаго\itshape  (или:\normalfont{} св. Василия Великаго, архиепископа Кесарии Каппадокийския), и святаго\itshape  (храма и святого, которого память в этот день),\normalfont{} святых и праведных Богоотец Иоакима и Анны и всех святых, помилует и спасет нас, яко Благ и Человеколюбец.


\medskip


 \bfseries  Многолетие \normalfont{}


\itshape  Хор:\normalfont{} Великаго Господина и Отца нашего\itshape  (имярек\normalfont{}), Святейшаго Патриарха Московскаго и всея Руси, и Господина нашего Прсосвященнейшаго\itshape  (имя)\normalfont{} митрополита\itshape  (или:\normalfont{} архиепископа,\itshape  или:\normalfont{} епископа)\itshape  (епархиальный титул его),\normalfont{} богохранимую державу нашу Российскую, настоятеля, братию и прихожан святаго храма сего и вся православныя христианы, Господи, сохрани их на многая лета.


\medskip


  По обычаю, перед отпустом священник берет крест с престола и после отпуста, осенив крестом народ и сам поцеловав крест, дает его для целования молящимся, а чтец читает благодарственные молитвы; затем священник опять осеняет крестом народ и возвращается в алтарь, причем царские врата и завеса закрываются.


  Алтарники убираются в алтаре, чистят кадило и готовятся к вечернему Богослужению


\mychapterending