

\mypart{МОЛИТВЫ ПРИ НАПАДЕНИИ НЕЧИСТОЙ СИЛЫ ТВОРИМЫЕ}
%http://www.molitvoslov.com/content/prinapadenii-nechistoy-sili

 

\mychapter{О прогнании лукавых духов от людей и животных, против вреда со стороны экстрасенсов, магов и т.д. и злых людей}
%http://www.molitvoslov.com/content/o-prognanii-lukavih-duhov-ot-lyudej-i-zhivotnih-protiv-vreda-so-storoni-ekstrasensov-magov-i-zlyh-lyudey

 

\section{Преподобной Марии Египетской}
%http://www.molitvoslov.com/text813.htm 
 


\itshape (читаются только по благословению духовника)\normalfont{}


\medskip


\bfseries Тропарь, глас 8-й:\normalfont{}


В тебе, мати, известно спасеся еже по образу: приимши бо Kрест, последовала еси Христу, и деющи учила еси презирати убо плоть, преходит бо, прилежати же о души, вещи безсмертней. Темже и со aнгелы срадуется, преподобная Марие, дух твой.


\medskip


\bfseries Кондак, глас 4-й:\normalfont{}


Греха мглы избежавши, покаяния светом озаривши твое сердце, славная, пришла еси ко Христу, Сего Всенепорочную и святую Матерь, Молитвенницу милостивную принесла еси. Отонудуже и прегрешений обрела еси оставление, и со ангелы присно срадуешися.


\medskip


\bfseries Молитва:\normalfont{}


О великая Христова угодница, преподобная мати Марие! Услыши недостойную молитву нас, грешных (\itshape имена\normalfont{}), избави нас, преподобная мати, от страстей, воюющих на души наша, от всякия печали и находяшия напасти, от внезапныя смерти и от всякаго зла, в час же разлучения души от тела отжени, святая угодница, всякую лукавую мысль и лукавые бесы, яко да приимет души наша с миром в место светло Христос Господь Бог наш, яко от Него очищение грехов, и Той есть спасение душ наших, Емуже подобает всякая слава, честь и поклонение, со Отцем и Святым Духом, ныне и присно и во веки веков.


\itshape Об исцелении бесноватых и страждущих от нечистых духов молятся перед иконами Божией Матери Свенско-Печерской, Владимиро-Оранской, Коневской и "Избавительницей".\normalfont{}


\itshape Также читают акафисты: святому мученику Трифону, святителю Тихону Воронежскому, святому Архангелу Михаилу.\normalfont{}


\section{Священномученику Киприану}
%http://www.molitvoslov.com/text803.htm 
 


\itshape (читаются только по благословению духовника)\normalfont{}


\medskip


\bfseries Тропарь, глас 4-й:\normalfont{}


И нравом причастник, и престолом наместник апостолом быв, деяние обрел еси, Богодухновенне, в видения восход: сего ради слово истины исправляя, и веры ради пострадал еси даже до крове, священномучениче Киприане, моли Христа Бога спастися душам нашим.


\medskip


\bfseries Кондак, глас 1-й:\normalfont{}


От художества волшебнаго обратився, богомудре, к познанию Божественному, показался еси миру врач мудрейший, исцеления даруя чествующим тя, Киприане со Иустиною: с неюже молися Человеколюбцу Владыце спасти души наша.


\medskip


\bfseries Икос:\normalfont{}


Целений твоих, святе, дарования мне низпослав, и недугующее мое сердце гноем греховным молитвами твоими исцели, яко да слово пения от скверных устен моих ныне принесу ти и воспою болезни твоя, яже показал еси, священномучениче, покаянием добрым и блаженным и Богу приближающимся. Того бо удержан рукою, направился еси, яко по лествице, к Небесным, непрестанно моляся спасти души наша.


\medskip


\bfseries Молитва:\normalfont{}


О, святый угодниче Божий, священномучениче Киприане, скорый помощниче и молитвенниче о всех к тебе прибегающих. Приими от нас недостойных хваление наше, и испроси нам у Господа Бога в немощех укрепление, в болезнех исцеление, в печалех утешение и всем вся полезная в жизни нашей. Вознеси ко Господу благомощную твою молитву, да оградит нас от падений греховных наших, да научит нас истинному покаянию, да избавит нас от пленения диавольскаго и всякаго действия духов нечистых и избавит от обидящих нас. Буди нам крепкий поборник на все враги видимыя и невидимыя. Во искушениях подаждь нам терпение и в час кончины нашея яви нам заступление от истязателей на воздушных мытарствах наших. Да водимыя тобою достигнем Горняго Иерусалима и сподобимся в Небеснем Царствии со всеми святыми славити и воспевати Пресвятое имя Отца и Сына и Святаго Духа во веки веков. Аминь.


\section{Cвятым угодникам: Иоанну Крестителю, Иоанну Богослову, Николаю Чудотворцу, священномученику Харлампию, вмч. Георгию Победоносцу и другим святым}
%http://www.molitvoslov.com/text812.htm 
 


\itshape (читаются только по благословению духовника)\normalfont{}




О, великие Христовы угодницы и чудотворцы: святый Предтече и Крестителю Христов Иоанне, святый всехвальный апостоле и наперстниче Христов Иоанне, святителю отче Николае, священномучениче Харлампие, великомучениче Георгие Победоносче, отче Феодоре, пророче Божий Илие, святителю Никите, мучениче Иоанне Воине, великомученице Варваро, великомученице Екатерино, преподобный отче Антоние! Услышите нас, вам молящихся, раб Божиих (\itshape имена\normalfont{}). Вы весте наши скорби и недуги, слышите воздыхания множества к вам притекающих. Сего ради к вам яко скорым помощникам и теплым молитвенникам нашим зовем: не оставляйте нас (\itshape имена\normalfont{}) вашим у Бога ходатайством. Мы присно заблуждаем от пути спасения, руководите нас, милостивые наставницы. Мы немощны есмы в вере, утвердите нас, правоверия учители. Мы зело убози сотворихомся добрых дел, обогатите нас, благосердия сокровища. Мы присно наветуеми от враг видимых и невидимых и озлобляеми, помозите нам, безпомощных заступницы. Гнев праведный, движимый на ны за беззакония наша, отвратите от нас вашим ходатайством у престола Судии Бога, Емуже вы предстоите на небеси, святые праведницы. Услышите, молим вас, велиции Христовы угодницы, вас с верою призывающия и испросите молитвами вашими у Отца Небеснаго всем нам прощение грехов наших и от бед избавление. Вы бо помощницы, заступницы и молитвенницы, и о вас славу возсылаем Отцу и Сыну и Святому Духу, ныне и присно и во веки веков. Аминь.


\section{Псалом 102}
%http://www.molitvoslov.com/text807.htm 
 


Благослови, душе моя, Господа, и вся внутренняя моя имя святое Его. Благослови, душе моя, Господа, и не забывай всех воздаяний Его, очищающаго вся беззакония твоя, исцеляющаго вся недуги твоя, избавляющаго от истления живот твой, венчающаго тя милостию и щедротами, исполняющаго во благих желание твое: обновится яко орля юность твоя. Творяй милостыни Господь, и судьбу всем обидимым. Сказа пути Своя Моисеови, сыновом Израилевым хотения Своя: Щедр и Милостив Господь, Долготерпелив и Многомилостив. Не до конца прогневается, ниже во век враждует, не по беззаконием нашим сотворил есть нам, ниже по грехом нашым воздал есть нам. Яко по высоте небесней от земли, утвердил есть Господь милость Свою на боявшихся Его. Елико отстоят востоцы от запад, удалил есть от нас беззакония наша. Якоже щедрит отец сыны, ущедри Господь боящихся Его. Яко той позна создание наше, помяну, яко персть есмы. Человек, яко трава дние его, яко цвет сельный, тако оцветет, яко дух пройде в нем, и не будет, и не познает ктому места своего. Милость же Господня от века и до века на боящихся Его, и правда Его на сынех сынов, хранящих завет Его, и помнящих заповеди Его творити я. Господь на Небеси уготова Престол Свой, и Царство Его всеми обладает. Благословите Господа вси Ангели Его, сильнии крепостию, творящии слово Его, услышати глас словес Его. Благословите Господа вся силы Его, слуги Его, творящии волю Его. Благословите Господа вся дела Его, на всяком месте владычества Его, благослови, душе моя. Господа.


\section{Святому праведному Иоанну, Русскому Исповеднику}
%http://www.molitvoslov.com/text811.htm 
 


\itshape (читаются только по благословению духовника)\normalfont{}


\medskip


\bfseries Тропарь, глас 4-й:\normalfont{}


От земли пленения твоего воззвавый тя к Небесным селением, Господь соблюдает невредимо и цельбоносно тело твое, праведне Иоанне, ты бо, в России ятый и во Асию проданный, посреде, агарянскаго злочестия благочестно пожил еси во мнозе терпении и, сеяв зде слезами, жнеши тамо неизглаголанною радостию. Темже моли Христа Бога спастися душам нашим.


\medskip


\bfseries Тропарь исповедникам, глас 8-й:\normalfont{}


Православия наставниче, благочестия учителю и чистоты, вселенныя светильниче, архиереев  богодуховенное удобрение, Иоанне  премудре, ученьми твоими вся просветил еси, цевнице духовная. Mоли Христа Бога, спастися душам нашым.


\medskip


\bfseries Кондак, глас 2-й:\normalfont{}


Насладився, богомудре, воздержания, и желания плотская ты успил еси, явився верою возвещаемь, и яко животадрево райское процвел еси, Иоанне  отче священный.


\medskip


\bfseries Молитва:\normalfont{}


О, святый новоявленный угодниче Божий, Иоанне Русский! Подвигом добрым подвизаяся на земли, восприял еси на небесех венец правды, еже уготовал есть Господь всем любящим Его. Темже взирающе на святый твой образ, радуемся о преславнем скончании жительства твоего и чтем святую память твою. Ты же, предстояй Престолу Божию, приими моления наша, раб Божиих (\itshape имена\normalfont{}), и ко Всемилостивому Богу принеси, о еже простити нам всякое прегрешение и помощи нам стати противу козней диавольских, да избавльшеся от скорбей, болезней, бед и напастей и всякаго зла, благочестно и праведно поживем в нынешнем веце и сподобимся предстательством твоим, аще и недостойни есмы, видети благая на земли живых, славяще Единаго во святых Своих славимаго Бога, Отца и Сына и Святаго Духа, ныне и во веки веков.


\section{Псалом 90}
%http://www.molitvoslov.com/text806.htm 
 


Живый в помощи Вышняго, в крове Бога Небеснаго водворится. Речет Господеви: Заступник мой еси и Прибежище мое, Бог мой, и уповаю на Него. Яко Той избавит тя от сети ловчи, и от словесе мятежна, плещма Своима осенит тя, и под криле Его надеешися: оружием обыдет тя истина Его. Не убоишися от страха нощнаго, от стрелы летящия во дни, от вещи во тме преходяшия, от сряща, и беса полуденнаго. Падет от страны твоея тысяща, и тма одесную тебе, к тебе же не приближится, обаче очима твоима смотриши, и воздаяние грешников узриши. Яко Ты, Господи, упование мое, Вышняго положил еси прибежище твое. Не приидет к тебе зло, и рана не приближится телеси твоему, яко Ангелом Своим заповесть о тебе, сохранити тя во всех путех твоих. На руках возмут тя, да не когда преткнеши о камень ногу твою, на аспида и василиска наступиши, и попереши льва и змия. Яко на Мя упова, и избавлю и: покрыю и, яко позна имя Мое. Воззовет ко Мне, и услышу его: с ним есмь в скорби, изму его, и прославлю его, долготою дней исполню его, и явлю ему спасение Мое.


\section{Преподобномученику Корнилию Псково-Печерскому}
%http://www.molitvoslov.com/text810.htm 
 


\itshape (читаются только по благословению духовника)\normalfont{}


\medskip


\bfseries Тропарь, глас 6-й:\normalfont{}


Псково-Печерская обитель, издревле славная чудесы иконы Богоматерни, многия иноки Богови воспита, тамо и преподобный Корнилий подвигом добрым подвизася, чудную Богоматерь славя, иноверныя просвещая, иноки и многия люди спасая, обитель же свою дивно украшая и ограждая. Тамо и мученичества венец по многих летех пастырства своего доблестно прият. Темже воспоим, людие, Христа Бога и Пречистую Его Матерь возблагодарим, яко дарова нам преподобномученика славна и молитвенника о душах наших достоблаженна.


\medskip


\bfseries Тропарь общий преподобномученику, глас 8-й:\normalfont{}


В тебе, отче, известно спасеся, еже по образу: приим бо крест последовал еси Христу, и дея учил еси презирати убо плоть, преходит бо, прилежати же о души, вещи безсмертней.Темже и со ангелы срадуется, преподобне Корнилиe, дух твой.


\medskip


\bfseries Кондак, глас 2:\normalfont{}


Яко постника благочестна и искусна, и страдальца произволением честна, и пустыни жителя сообразна, в песнех достойно восхвалим Корнилия, приснохвальнаго: той бо змия попрал есть.


\medskip


\bfseries Молитва:\normalfont{}


О святый преподобномучениче Корнилие!  Милостиво призри на скорби наша душевныя и телесныя, и избаву нам подаждь; помози нам, рабам Божиим (\itshape имена\normalfont{}), святче Божий, избавитися от навета злых человек, от нихже и сам ты невинно на земли пострадал еси; защити нас от насилия диавола, люте на нас воюющаго. Умоли Господа Бога и Пречистую Его Матерь даровати нам тихое и безгрешное житие, братскую нелицемерную любовь и мирную христианскую кончину, да чистою совестию предстанем нелицеприятному страшному судилищу Христову, и в Царствии Его прославим Животворящую Троицу, Отца и Сына и Святаго Духа, во веки веков. Аминь.


\section{Псалом 67}
%http://www.molitvoslov.com/text805.htm 
 


Да воскреснет Бог, и расточатся врази Еro, и да бежат от лица Его ненавидящии Его. Яко исчезает дым, да исчезнут, яко тает воск от лица огня, тако да погибнут грешницы от лица Божия, а праведницы да возвеселятся, да возрадуются пред Богом, да насладятся в веселии. Воспойте Богу, пойте имени Его, путесотворите возшедшему на запады. Господь имя Ему, и радуйтеся пред Ним. Да смятутся от лица Его, Отца сирых и Судии вдовиц: Бог в месте святем Своем. Бог вселяет единомысленныя в дом, изводя окованныя мужеством, такожде преогорчевающыя живущыя во гробех. Боже, внегда исходити Тебе пред людьми Твоими, внегда мимоходити Тебе в пустыни, земля потрясеся, ибо небеса кануша от лица Бога Синаина, от лица Бога Израилева. Дождь волен отлучиши, Боже, достоянию Твоему и изнеможе, Ты же совершил еси е. Животная Твоя живут на ней, уготовал еси благостию Твоею нищему, Боже. Господь даст глагол благовествующым силою многою. Царь сил возлюбленнаго, красотою дому разделити корысти. Аще поспите посреде предел, криле голубине посребрене, и междорамия ея в блещании злата. Внегда разнствит Небесный цари на ней, оснежатся в Селмоне. Гора Божия, гора тучная, гора усыренная, гора тучная. Вскую непщуете горы усыренныя? Гора, юже благоволи Бог жити в ней, ибо Господь вселится до конца. Колесница Божия тмами тем, тысяща гобзующих, Господь в них в Синаи во святем. Возшел еси на высоту, пленил еси плен, приял еси даяния в человецех, ибо не покаряющыяся, еже вселитися. Господь Бог благословен, благословен Господь день дне, поспешит нам Бог спасений наших. Бог наш, Бог еже спасати, и Господня, Господня исходища смертная. Обаче Бог сокрушит главы врагов Своих, верх влас преходящих в прегрешениих своих. Рече Господь: от Васана обращу, обращу во глубинах морских. Яко да омочится нога твоя в крови, язык пес твоих, от враг от него. Видена быша шествия Твоя, Боже, шествия Бога моего Царя, иже во святем, предвариша князи близ поющих, посреде дев тимпанниц. В церквах благословите Бога, Господа от источник Израилевых. Тамо Вениамин юнейший во ужасе, книзи Иудовы владыки их, князи Завулони, князи Неффалимли. Заповеждь, Боже, силою Твоею, укрепи, Боже, сие, еже соделал еси в нас. От храма Твоего во Иерусалим Тебе принесут царие дары. Запрети зверем\itshape  \normalfont{}тростным, сонм юнец в юницах людских, еже затворити искушенныя сребром, расточи языки хотящыя бранем. Приидут молитвенницы от Египта, Ефиопиа предварит руку свою к Богу. Царства земная, пойте Богу, воспойте Господеви, возшедшему на Небо небесе на востоки, се даст гласу Своему глас силы. Дадите славу Богови, на Израили велелепота Его, и сила Его на облацех. Дивен Бог во святых Своих, Бог Израилев: Той даст силу и державу людем Своим, благословен Бог.


\section{Священномученику Киприану и мученице Иустине}
%http://www.molitvoslov.com/text804.htm 
 


\itshape (читаются только по благословению духовника)\normalfont{}


\medskip


\bfseries Молитва:\normalfont{}


 О святие священномучениче Киприане и мученице Иустина! Внемлите смиренному молению нашему. Аще бо временное житие ваше мученически за Христа скончали есте, но духом от нас не отступаете есте, присно по заповедем Господним шествовати нас научающе и крест свой терпеливо нести нам пособствующе. Се, дерзновением ко Христу Богу и Пречистей Его Матери стяжали есте. Темже и ныне будите молитвенницы и ходатаи о нас недостойных (\itshape имена\normalfont{}). Будите нам заступницы крепции, да заступлением вашим сохраняеми невредимы, от бесов, волхвов и от человек злых пребудем, славяще Святую Троицу, Отца и Сына и Святаго Духа, ныне и присно и во веки веков. Аминь.


\section{Псалом 126}
%http://www.molitvoslov.com/text808.htm 
 


Аще не Господь созиждет дом, всуе трудишася зиждущии. Аще не Господь сохранит град, всуе бде стрегий. Всуе вам есть утреневати, востанете по седении, ядущии хлеб болезни, егда даст возлюбленным Своим сон. Се достояние Господне сынове, мзда плода чревняго. Яко стрелы в руце сильнаго, тако сынове оттрясенных. Блажен, иже исполнит желание свое от них. Не постыдятся, егда глаголют врагом своим во вратех.

\itshape Также рекомендуется читать вечернюю молитву  "Да воскреснет Бог ..." и целиком 4-ю кафизму, поминая на каждой "Славе..." имя страждущего.

\normalfont{} 


\mychapterending

\mychapter{От мысленных бесовских искушений}
%http://www.molitvoslov.com/content/ot-myslennyh-besovskih-iskusheniy

 

\section{Кратчайший образ избавления от хульных мыслей}
%http://www.molitvoslov.com/text931.htm 
 




\itshape Аще приидет хульный помысл на Бога, читай:\normalfont{} Верую во единаго Бога\itshape … до конца: и будет ли мощно, сотвори метаний, или поклонов что-либо по силе.\normalfont{}


\itshape Аще приидет ли хульный помысл на Пречистыя Тайны Христовы, читай:\normalfont{}  Верую, Господи, и исповедую, яко Ты воистину Христос…\itshape  до конца и сотвори поклонение.\normalfont{}


\itshape Аще приидет хульный помысл на Пречистую Богородицу, читай каковую-либо молитву к Пречистой Богородице; или:\normalfont{}  Под Твое благоутробие…\itshape  или:\normalfont{}  Богородице Дево, радуйся…\itshape  или какий-либо тропарь Богородичен, с поклонами, глаголя:\normalfont{} Пресвятая Богородице, спаси мя грешнаго!

\itshape Аще кий приидет хульный помысл на коего святаго, читай сие\normalfont{}: Моли Бога о мне грешном, святый (\itshape имя\normalfont{}), яко аз по Бозе к тебе прибегаю, скорому помощнику и молитвеннику о душах наших.\itshape  И сотвори поклонов по силе, глаголя:\normalfont{}  Святый (\itshape имя\normalfont{}), моли Бога о мне грешном.

\itshape Аще приидет хульный помысл на какую икону, сотвори пред тою иконою поклонов 15, или колико можеши, молящися тому, иже на той иконе изображен, и тако хульныя мысли ни во чтоже вмениши Божиею помощию. Аминь.\normalfont{}


\itshape Примечание: Слова, выделенные курсивом не являются частью молитв.\normalfont{}





\section{Преподобному  Серафиму Саровскому}
%http://www.molitvoslov.com/text815.htm 
 


\itshape (читаются только по благословению духовника)

\normalfont{}

\bfseries Тропарь, глас 4-й\normalfont{}\bfseries :  \normalfont{}


От юности Христа возлюбил еси, блаженне, и Тому Единому работати пламенне вожделев, непрестанною молитвою и трудом в пустыни подвизался еси, умиленным же сердцем любовь Христову стяжав, избранник возлюблен Божия Матери явился еси. Сего ради вопием ти: спасай нас молитвами твоими, Серафиме, преподобне отче наш. 

\bfseries 

Кондак, глас 2-й:\normalfont{}


Мира красоту и яже в нем тленная оставив, преподобне, в Саровскую обитель вселился еси; и тамо ангельски пожив, многим путь был еси ко спасению. Сего ради и Христос тебе, отче Серафиме, прослави, и даром исцелений и чудес обогати. Темже вопием ти: радуйся, Серафиме, преподобне отче наш. 

\bfseries 

Молитва:\normalfont{}


О, великий угодниче Божий, преподобне и Богоносне отче наш Серафиме! Призри от Горния славы на нас, смиренных и немощных, обремененных грехми многими, твоея помощи и утешения просящих. Приникни к нам благосердием твоим и помози нам заповеди Господни непорочно сохраняти, веру Православную крепко содержати, покаяние во гресех наших усердно Богу приносити, во благочестии христианстем благодатно преуспевати и достойны быти твоего о нас молитвеннаго к Богу предстательства. Ей, святче Божий, услыши нас, молящихся тебе с верою и любовию и не презри нас, требующих твоего заступления: ныне и в час кончины нашея помози нам и заступи нас молитвами твоими от злобных наветов диавольских, да не обладает нами тех сила, но да сподобимся помощию твоею наследовати блаженство обители райския. На тя бо упование наше ныне возлагаем, отче благосердный: буди нам воистинну ко спасению путевождь и приведи нас к невечернему свету жизни вечныя Богоприятным предстательством твоим у Престола Пресвятыя Троицы, да славим и поем со всеми святыми достопокланяемое имя Отца и Сына и Святаго Духа во веки веков. Аминь. 


\mychapterending

\mychapter{Во время брани плотской}
%http://www.molitvoslov.com/content/vo-vremya-brani-plotskoy

 

\section{Иная молитва}
%http://www.molitvoslov.com/text818.htm 
 


\itshape (читаются только по благословению духовника)

\normalfont{}


Предложение твое, враже, на главу твою, Мати Божия, помози ми. \itshape (И читать много раз "Богородице Дево, радуйся".)

\normalfont{}


\section{Молитва (старца Макария Оптинского)}
%http://www.molitvoslov.com/text817.htm 
 


\itshape (читаются только по благословению духовника)

\normalfont{}


О, Мати Господа моего Творца, Ты корень девства и неувядаемый цвет чистоты. О, Богородительнице! Ты ми помози, немощному плотскою страстью о болезненну сущу, едино бо Твое, и с Тобою Твоего Сына и Бога имею заступление. Аминь.





\section{Правило от осквернения}
%http://www.molitvoslov.com/text819.htm 
 


\itshape (читаются только по благословению духовника)\normalfont{}


\itshape Когда случится кому искуситися во сне по действу диаволю, востав от одра, творит поклоны, глаголя: 

\normalfont{}

Боже, милостив буди мне грешному.


\itshape Затем начало обычное: 

\normalfont{}Царю Небесный, Трисвятое, \itshape по\normalfont{} Отче наш: Господи помилуй. (12 раз). Слава, и ныне. Приидите, поклонимся... (\itshape Трижды\normalfont{}) Помилуй мя, Боже...


\itshape И тропари, глас 7-й: \normalfont{}

Пастырю добрый, душу Твою положивый о нас, сведый сокровенная, содеянная мною, Едине Блаже, упаси мя разумом заблуждшаго, и исхити мя от волка, Агнче Божий, и помилуй мя. 


Отягчен сном уныния, помрачаюся прелестию греховною: но даруй ми утро покаяния, просвещая очи мысленныя, Христе Боже, просвещение души моея, и спаси мя. 


Мглою греховною и сластьми житейскими сплетаем ум окаянныя души моея, страсти различныя раждает, и в помысл умиления не приходит. Но ущедри, Спасе, смирение мое, и даждь ми помысл умиления, да и аз спасаем прежде конца воззову благоутробию Твоему: Господи Христе Спасе мой, отчаяннаго спаси мя и недостойнаго. 


Яко впадый в разбойники, и уязвен, тако и аз впадох во многии грехи, и уязвена ми есть душа. К кому прибегну повинный аз: токмо к Тебе, милосердому душ Врачу: излей на мя, Боже, великую Твою милость. 


Слава: Яко блудный сын приидох и аз, Щедре: приими мя, Отче, возвращающася, яко единаго от наемник Твоих, Боже, и помилуй мя.


И ныне: \itshape Богородичен\normalfont{}. Избави, Богородице, от обдержащих нас грехов: яко иного упования вернии не имамы, разве Тебе, и от Тебе рождшагося Господа.


Господи, помилуй \itshape (40 раз).\normalfont{}


Затем 8 поклонов с молитвой: 

Боже, милостив ми буди, и прости мя блуднаго за имя Твое Святое.


\medskip


\bfseries Молитва 1, святого Василия Великого

\normalfont{}Паки запят бых окаянный умом и лукавым обычаем, работая греху. Паки князь тьмы и страстных сладостей родитель, пленена мя сотвори, и якоже раба смиреннаго, того хотением, и желанием плотским работати принуждает мя. И что сотворю, Господи мой, и Избавителю, и Заступниче уповающим на Тя, аще не к Тебе паки возвращуся и постеню, и милости испрошу о содеянных мною; но боюся и трепещу, да не како всегда исповедаяся, и отступити злых обещаваяся, и на кийждо час согрешая: и не воздав молитвы моея Тебе Богу моему, долготерпение Твое воздвигну к негодованию. И кто стерпит гнева Твоего, Господи; ведый убо множество щедрот Твоих, и пучину человеколюбия Твоего, паки возверзаю себе на милость Твою, и взываю Ти: еже согреших, прости. Помилуй мя падшаго, даждь ми руку помощи, в тине сластей погруженному. Не остави, Господи, создание Твое растлитися беззаконьми моими и грехи моими: но обычным Твоим милосердием и благостию принуждаем, избави от кала и скверны телесныя и страстных помышлений, оскверняющих всегда душу мою окаянную: се бо, Господи, якоже зриши, несть в ней места чиста, но вся проказися, и все тело объят язва. Сам убо, Человеколюбче, врачу душ и телес и милости источниче, очисти ту слез моих течением, сих изливая на мя обильно: излей на мя человеколюбие Твое, и даждь ми исцеление и очищение, и исцели сокрушение мое, и не отврати лица Твоего от мене, да не якоже вещь, пояст мя отчаяния огнь: но якоже рекл еси, неложный Боже, яко велия радость бывает на небеси о едином грешнице кающемся, сие сотвори и на мне грешнем, и не затвори ушию благоутробия Твоего, в молитве покаяния моего; но отверзи их, и яко кадило исправи ту пред Тобою: веси бо немощь естества Создателю, и удобь поползновение юности, и тяжесть тела, и презираеши грехи, и покаяние приемлеши призывающих Тя истиною. Яко благословися и прославися пречестное и великолепое имя Твое, Отца и Сына и Святаго Духа, ныне и присно и во веки веков. Аминь.


\medskip


\bfseries Молитва 2, его же\normalfont{}


Многомилостиве, нетленне, нескверне, безгрешне Господи, очисти мя, непотребнаго раба Твоего, от всякия скверны плотския и душевныя, и от невнимания и уныния моего прибывшую ми нечистоту, со инеми всеми беззаконии моими, и яви мя нескверна, Владыко, за благость Христа Твоего, и освяти мя нашествием Пресвятаго Твоего Духа: яко да возбнув от мглы нечистых привидений диавольских, и всякия скверны, сподоблюся чистою совестию отверсти скверная моя и нечистая уста, и воспевати всесвятое имя Твое, Отца и Сына и Святаго Духа, ныне и присно и во веки веков. Аминь.


\medskip


\bfseries Молитва 3\normalfont{}


Господи Боже наш, еже согреших во дни сем словом, делом и помышлением, яко Благ и Человеколюбец прости ми. Мирен сон и безмятежен даруй ми. Ангела Твоего хранителя посли, покрывающа и соблюдающа мя от всякаго зла, яко Ты еси хранитель душам и телесем нашим, и Тебе славу возсылаем, Отцу и Сыну и Святому Духу, ныне и присно и во веки веков. Аминь.


\medskip


\bfseries Молитва 4, святого Иоанна Златоуста\normalfont{}


\itshape (24 молитвы, по числу часов дня и ночи)\normalfont{}

Господи, не лиши мене небесных Твоих благ.

Господи, избави мя вечных мук.

Господи, умом ли или помышлением, словом или делом согреших, прости мя.

Господи, избави мя всякаго неведения и забвения, и малодушия, и окамененнаго нечувствия.

Господи, избави мя от всякаго искушения.

Господи, просвети мое сердце, еже помрачи лукавое похотение.

Господи, аз яко человек согреших, Ты же яко Бог щедр, помилуй мя, видя немощь души моея.

Господи, посли благодать Твою в помощь мне, да прославлю имя Твое святое.

Господи Иисусе Христе, напиши мя раба Твоего в книзе животней и даруй ми конец благий.

Господи, Боже мой, аще и ничтоже благо сотворих пред Тобою, но даждь ми по благодати Твоей положити начало благое.

Господи, окропи в сердце моем росу благодати Твоея.

Господи небесе и земли, помяни мя грешнаго раба Твоего, студнаго и нечистаго, во Царствии Твоем. Аминь.

Господи, в покаянии приими мя.

Господи, не остави мене.

Господи, не введи мене в напасть.

Господи, даждь ми мысль благу.

Господи, даждь ми слезы и память смертную, и умиление.

Господи, даждь ми помысл исповедания грехов моих.

Господи, даждь ми смирение, целомудрие и послушание.

Господи, даждь ми терпение, великодушие и кротость.

Господи, всели в мя корень благих, страх Твой в сердце мое.

Господи, сподоби мя любити Тя от всея души моея и помышления и творити во всем волю Твою.

Господи, покрый мя от человек некоторых, и бесов, и страстей, и от всякия иныя неподобныя вещи.

Господи, веси, яко творши, якоже Ты волиши, да будет воля Твоя и во мне грешнем, яко благословен еси во веки. Аминь.


\medskip


\bfseries Молитва 5, ко Пресвятой Богородице

\normalfont{}К Тебе Пречистей Божией Матери аз окаянный припадая молюся: веси, Царице, яко безпрестани согрешаю и прогневляю Сына Твоего и Бога моего, и многажды аще каюся, лож пред Богом обретаюся, и каюся трепеща: не ужели Господь поразит мя, и по часе паки таяжде творю; ведущи сия, Владычице моя Госпоже Богородице, молю, да помилуеши, да укрепиши, и благая творити да подаси ми. Веси бо, Владычице моя Богородице, яко отнюд имам в ненависти злая моя дела, и всею мыслию люблю закон Бога моего; но не вем, Госпоже Пречистая, откуду яже ненавижду, та и люблю, а благая преступаю. Не попущай, Пречистая, воли моей совершатися, не угодна бо есть, но да будет воля Сына Твоего и Бога моего: да мя спасет, и вразумит, и подаст благодать Святаго Духа, да бых аз отселе престал сквернодейства, и прочее пожил бых в повелении Сына Твоего, Емуже подобает всякая слава, честь и держава, со Безначальным Его Отцем, и Пресвятым и Благим и Животворящим Его Духом, ныне и присно и во веки веков. Аминь.


\itshape Затем \normalfont{}Честнейшую Херувим... 

Слава, и ныне. Господи, помилуй \itshape (трижды)\normalfont{}. 

Господи, благослови. 


\medskip


\bfseries И отпуст\normalfont{}


Господи Иисусе Христе, Боже наш, молитв ради Пречистыя Твоея Матере и всех святых, спаси мя грешнаго.


\bfseries Молитва от осквернения.\normalfont{}

Господи Боже наш, Едине Благий и Человеколюбче, Едине Святый и на святых почиваяй, иже верховному Твоему апостолу Петру явивый видением, ничтоже скверно, или нечисто мнети, от Тебе сотворенных на пищу и в наслаждение человеком, и сосудом Твоим избранным, апостолом Павлом вся чиста чистым заповедавый: Ты Сам Владыко Пресвятый призыванием страшнаго и пречистаго Твоего Имене, и знамением страшного и Животворящаго Креста, благослови и очисти мя раба твоего (\itshape имя\normalfont{}) осквернившагося от всякого непрязненного духа, от всякаго мечтания и гада ядовитаго, от всякаго беззакония и от всякия лести, от всякого потвора, и всякия язи, и от всякаго противнаго злодейства диавольского. И ныне недостойнаго мене раба Твоего (\itshape имя\normalfont{}): сподоби по Твоему милосердию служити пречистым Твоим Таинам. И прежде очисти ми душу и тело от всякия скверны, и остави всякое прегрешение, вольное и невольное, еже согреших во вся дни живота моего, делом, словом и помышлением, во дни и в нощи, и до нынешняго часа. Идаждь ми, Господи служение сие страшное небесных Чинов, и причастие Пречистых Твоих Таин, не в суд, ни во осуждение, но в прощение грехов, и в Духа Святаго пришествие, и живот присносущныя радости, егоже уготовал еси истинным Твоим служебником. Сохрани мя, Владыко Всесильне, от всякаго греха и злобы, соблюди нескверна и непорочна от всякия проказы противнаго диавола: и даждь ми, Господи, служити Тебе в преподобии и правде до последняго дне и часа и скончания моего: Ты бо еси благословляяй и освящаяй всяческая, Христе Боже наш, и Тебе славу возсылаем, со Безначальным Твоим Отцем, и с Пресвятым и Благим и Животворящим Твоим Духом, ныне и присно и во веки веков, аминь.


\mychapterending

\mychapter{Молитва для защиты от нечистой силы}
%http://www.molitvoslov.com/text797.htm 
 


\itshape (читаются только по благословению духовника)

\normalfont{}

Господи Иисусе Христе, Сыне Божий, огради мя святыми Твоими aнгелы и молитвами Всепречистыя Владычицы нашея Богородицы и Приснодевы Марии, Силою Честнаго и Животворящаго Креста, святаго архистратига Божия Михаила и прочих Небесных сил безплотных; святаго Пророка и Предтечи Крестителя Господня Иоанна; святаго Апостола и Евангелиста Иоанна Богослова; священномученика Киприана и мученицы Иустины; святителя Николая архиепископа Мир Ликийских, чудотворца; святителя Льва епископа Катанскаго; святителя Иоасафа Белгородскаго; святителя Митрофана Воронежскаго; преподобнаго Сергия игумена Радонежскаго; преподобнаго Серафима Саровскаго,  чудотворца; святых мучениц Веры, Надежды, Любови и матери их Софии; святых и праведных Богоотец Иоакима и Анны и всех святых Твоих, помоги мне, недостойному рабу твоему (\itshape имя молящегося\normalfont{}), избави мя от всех навет вражиих, от всякаго колдовства, волшебства, чародейства и от лукавых человек, да не возмогут они причинить мне некоего зла. Господи, светом Твоего сияния сохрани мя на утро, на день, на вечер, на сон грядущий, и силою Благодати Твоея отврати и удали всякия злыя нечестия, действуемые по наущению диавола. Яко Твое есть Царство и Сила, и Слава, Отца, и Сына, и Святаго Духа. Аминь. 


\mychapterending

\mychapter{Небесным Силам}
%http://www.molitvoslov.com/text801.htm 
 


\itshape (читаются только по благословению духовника)

\normalfont{}

\bfseries Тропарь, глас 4-й:\normalfont{}


Небесных воинств Архистратизи, молим вас присно мы недостойнии, да вашими молитвами оградите нас кровом крил невещественныя вашея славы; сохраняюще ны припадающия прилежно и вопиющия: от бед избавите ны, яко чиноначальницы вышних сил. 


\bfseries Кондак, глас-2-й\normalfont{}:


Архистратизи Божии, служителие Божественныя славы, ангелов начальницы, и человеков наставницы, полезное нам просите и велию милость, яко Безплотных Архистратизи. 


\mychapterending

\mychapter{Молитвы задержания}
%http://www.molitvoslov.com/text800.htm 
 


\itshape (читаются только по благословению духовника)

(Из сборника молитв старца Пансофия Афонского, 1848 г.)

Сила сих молитв в утаении от слуха и взора людскаго, в тайнодействовании ея.

\normalfont{}

Яко неплодную смоковницу, на посецы мене, Спасе, грешнаго, но на многая лета пождание ми даруй, напаяя душу мою слезами покаяния, да плод принесу Ти, Многомилостиве.*


\bfseries Молитва задержания

\normalfont{}


Милосердный Господи, Ты некогда устами служителя Моисея, Иисуса Навина, задерживал целый день движение Солнца и Луны, доколе народ Израильский мстил врагам своим. 

Молитвой Елисея пророка некогда поразил сириян, задерживая их, и вновь исцелил их.

Ты некогда вещал пророку Исаии: вот, Я возвращу назад на десять ступеней солнечную тень, которая прошла по ступеням Ахазовым, и возвратилось солнце на десять ступеней по ступеням, по которым оно сходило. (1)

Ты некогда устами пророка Иезекииля затворил бездны, останавливал реки, задерживал воды. (2)

И Ты некогда постом и молитвою пророка Твоего Даниила заграждал уста львов во рву. (3)

И ныне задержи и замедли до благовремения все замыслы вокруг стоящих мя о моем перемещении, увольнении, смещении, изгнании. 

Так и ныне, разруши злые хотения и требования всех осуждающих мя\bfseries  \normalfont{}, загради уста и сердца всех клевещущих, злобствующих и рыкающих на мя, и всех хулящих и унижающих мя.

Так и ныне, наведи духовную слепоту на глаза всех восстающих на мя и на врагов моих.

Не Ты ли вещал апостолу Павлу: говори и не умолкай, ибо Я с тобою, и никто не сделает тебе зла. (4)

Смягчи сердца всех противоборствующих благу и достоинству Церкви Христовой. Поэтому да не умолкнут уста Мои для обличения нечестивых и прославления праведных и всех дивных дел Твоих. И да исполнятся вся благая начинания наши и желания. 


К вам, праведницы и молитвенницы Божии, наши дерзновенные предстателие, некогда силою своих молитв сдерживающие нашествие иноплеменников, подход ненавидящих, разрушившие злые замыслы людей, заграждавшие уста львов, ныне обращаюсь я с молитвою моей, с моим прошением.


И Ты, преподобный великий Елий Египетский, некогда оградивший в круге крестным знамением место поселения ученика своего, повелел ему вооружиться именем Господним и не бояться отныне демонских искушений. (5) Огради дом мой, в коем я живу, в круге молитв твоих и сохрани его от огненного запаления, воровского нападения и всякого зла и страхования.


И ты, преподобный отче Поплие Сирийский, некогда своею непрестанною молитвою десять дней демона державший неподвижным и не могущим идти ни днем, ни ночью (6); ныне окрест келии моей и дома (\itshape моего\normalfont{}) сего удержи за оградою его вся сопротивныя силы и всех хулящих имя Божие и презирающих мя.


И Ты, преподобная девственница Пиама, некогда силою молитвы остановившая движение шедших погубить жителей тоя деревни, где жила, ныне приостанови все замыслы врагов моих, хотящих изгнати мя из града сего и погубити мя: не допускай им приближатися к дому сему, останови их силою молитвы своей: "Господи, Судия Вселенной, Ты, которому неугодна всякая неправда, когда приидет к Тебе молитва сия, пусть Святая Сила остановит их на том месте, где постигнет их". (7)


И Ты, блаженный Лаврентий Калужский, моли Бога о мне, как имеющий дерзновение пред Господом предстательствовать о страждущих от козней диавольских. Моли Бога о мне, да оградит Он мя от козней сатанинских.


И Ты, преподобный Василий Печерский, соверши свои молитвы запрещения над нападающими на меня и отжени все козни диавольские от мене. (8)


И вы, вси святии земли Российстей, развейте силою молитв своих обо мне все бесовские чары, все диавольские замыслы и козни "--- досадити мне и погубити мя и достояние мое.


И Ты, великий и грозный страже, архистратиже Михаиле, огненным мечом посекаяй все хотения врага рода человеческого и всех приспешников его, хотящих погубити мя. Стой нерушимо на страже дома сего, всех живущих в нем и всего достояния его.


И Ты, Владычице, не напрасно именуемая "Нерушимой стеной", буди для всех враждующих против меня и замышляющих пакостная творити мне, воистинну некоей преградой и нерушимой стеной, ограждающей мя от всякого зла и тяжких обстояний.


* В понедельник утра на стиховне стихира покаянная, глас 7-й

(1) Ис. 38, 8

(2) Иез. 31, 15

(3) Евр. 11, 33

(4) Деян. 18, 10

(5) Лавсаик

(6) Древний Патерик, стр. 238-240

(7) Лавсаик

(8) Киево-Печерский Патерик


\mychapterending

\mychapter{Правило об избавлении от злых помыслов}
%http://www.molitvoslov.com/text799.htm 
 


\itshape (читаются только по благословению духовника)

\normalfont{}

Начало обычное: Царю Небесный,  Трисвятое,  \itshape по\normalfont{} Отче наш: Господи помилуй. (12 раз) Приидите, поклонимся... (\itshape Трижды\normalfont{})


\medskip


\bfseries Псалом 3\normalfont{}


Господи, что ся умножиша стужающии ми? Мнози востают на мя, мнози глаголют души моей: несть спасения ему в Бозе его. Ты же, Господи, Заступник мой еси, слава моя и возносяй главу мою. Гласом моим ко Господу воззвах, и услыша мя от горы святыя Своея. Аз уснух, и спах, востах, яко Господь заступит мя. Не убоюся от тем людей, окрест нападающих на мя. Воскресни, Господи, спаси мя, Боже мой, яко Ты поразил еси вся враждующыя ми всуе: зубы грешников сокрушил еси. Господне есть спасение, и на людех Твоих благословение Твое.


\medskip


\bfseries Молитва 1\normalfont{}


Владыко Господи Боже мой, Егоже в руках жребий мой, заступи мя по милости Твоей, и не остави мя погибнути со беззаконии моими, ниже воли последовати похотствующия плоти на дух. Создание Твое есмь, не презри дело руку Твоею, не отвратися, ущедри, но не уничижи;  не презри мя, Господи, яко немощен есмь, яко к Тебе прибегох, Покровителю моему Богу, исцели душу мою, яко согреших Тебе. Спаси мя ради милости Твоея, яко к Тебе привержен есмь от юности моея: да посрамятся ищущий отринути от Тебе, деяньми нечистыми, помышленьми  нелепыми,  воспоминаньми неполезными. Отгони от мене всякую скверну, злобы излишество: яко Ты еси Един Свят, Един Крепкий, Един Безсмертный, во всех имеяй могущество безприкладное, и Тобою дается всем, яже на диавола и его воинства, крепость.

Яко подобает Тебе всякая слава, честь и поклонение, Отцу и Сыну, и Святому Духу, ныне и присно, и во веки веков. Аминь.


\medskip


\bfseries Молитва 2\normalfont{}


Да обратится болезнь твоя на главу твою, и на верх твой хула твоя да снидет, лукавый бесе и нечистный: аз бо Господу Богу моему кланяюся, и Того никогдаже похулю. Како бо ми возможно Сему досадити, или Его похулити, Егоже по вся дни и нощи и часы славословлю, и покланяюся от всея ду­ши моея, и крепости и мысли моея; но убо славословие мое есть, хула же твоя еси: ты узриши, о нихже нань клевещеши, и яко от­ступник на Бога глаголеши.


\medskip


\bfseries Молитва 3 ко Пресвятой Богородице\normalfont{}


Пресвятая Владычице моя Богородице, святыми Твоими и всесильными мольбами отжени от мене, окаяннаго раба Твоего, уны­ние, забвение, неразумие, нерадение, и вся скверная, лукавая и хульная помышления, от окаяннаго ми сердца, от помраченнаго ума моего, и погаси пламень страстей моих, яко нищ и окаянен есмь, и избави мя от мно­гих лютых воспоминаний и предприятий, и от всех действ злых свободи мя, яко благо­словенна еси от всех родов, и славится Пречестное имя Твое во веки. Аминь.

Честнейшую Херувим: Господи, помилуй. (\itshape Tрижды\normalfont{}) Господи, благослови, \itshape и отпуст.\normalfont{}


\mychapterending

\mychapter{Об умягчении злых сердец и при защите от напастей злых человек}
%http://www.molitvoslov.com/text798.htm 
 


\itshape (читаются только по благословению духовника)\normalfont{}


\medskip


\bfseries Тропарь:\normalfont{}


Умягчи наша злая сердца. Богородице, и напасти ненавидящих нас угаси, и всякую тесноту души нашея разреши. На Твой бо святый образ взирающе, Твоим страданием и милосердием о нас умиляемся и раны Твоя лобызаем, стрел же наших, Тя терзающих, ужасаемся. Не даждь нам, Мати Благосердая, в жестокосердии нашем и от жестокосердия ближних погибнути, Ты бо еси воистину злых сердец умягчение.


\mychapterending