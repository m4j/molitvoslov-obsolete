\mypart{ПЕСНОПЕНИЯ ИЗ СЛУЖБ ТРИОДИ ЦВЕТНОЙ}\label{_content_pesnopeniya-izsluzhb-triodi-cvetnoy}
%http://www.molitvoslov.com/content/pesnopeniya-izsluzhb-triodi-cvetnoy

\mychapter{Пасхальный канон, глас 1-й}\begin{mymulticols}
%http://www.molitvoslov.com/text901.htm 
 
\mysubtitle{Песнь 1 }

\irmos{Воскресения день, просветимся людие: Пасха, Господня Пасха! От смерти бо к жизни, и от земли к Небеси, Христос Бог нас преведе, победную поющия. }

\pripev{Христос воскресе из мертвых.}

Очистим чувствия, и узрим неприступным светом воскресения Христа блистающася, и радуйтеся рекуща ясно да услышим, победную поюще. 

\pripev{Христос воскресе из мертвых.}

Небеса убо достойно да веселятся, земля же да радуется, да празднует же мир, видимый же весь и невидимый: Христос бо воста, веселие вечное. 

\myemph{\centering Богородичны\symbolfootnote[1]{Припев к ним: «Пресвятая Богородице, спаси нас», или «Слава…», «И ныне…»}:\\\small(Поемыя со второго дня Пасхи то отдания)}

\pripev{Пресвятая Богородице, спаси нас.}

Умерщвления предел сломила еси, вечную жизнь рождшая Христа, из гроба возсиявшаго днесь, Дево всенепорочная, и мир просветившаго.    

\pripev{Пресвятая Богородице, спаси нас.}

Воскресшаго видевши Сына Твоего и Бога, радуйся со апостолы, Богоблагодатная чистая: и еже радуйся первее, яко всех радости вина, восприяла еси, Богомати всенепорочная. 

\mysubtitle{Песнь 3 }

\irmos{Приидите, пиво пием новое, не от камене неплодна чудодеемое, но нетления источник, из гроба одождивша Христа, в Немже утверждаемся. }

\pripev[]{Христос воскресе из мертвых.}

Ныне вся исполнишася света, Небо же и земля и преисподняя: да празднует убо вся тварь востание Христово, в Немже утверждается. 

\pripev[]{Христос воскресе из мертвых.}

Вчера спогребохся Тебе, Христе, совостаю днесь воскресшу Тебе, сраспинахся Тебе вчера, Сам мя спрослави, Спасе, во Царствии Твоем. 

\myemph{\centering Богородичны:\\}

\pripev[]{Пресвятая Богородице, спаси нас.}

На нетленную жизнь прихожду днесь, благостию Рождшагося из Тебе, Чистая, и всем концем свет облиставшаго. 

\pripev[]{Пресвятая Богородице, спаси нас.}

Бога, Егоже родила еси плотию, из мертвых, якоже рече, воставша видевши, Чистая, ликуй, и Сего яко Бога, Пречистая, возвеличай. 

\mysubtitle{Ипакои, глас 4-й:}

Предварившия утро яже о Марии, и обретшия камень отвален от гроба, слышаху от Ангела: во свете присносущнем Сущаго, с мертвыми что ищете, яко человека? Видите гробныя пелены, тецыте, и миру проповедите, яко воста Господь, умертвивый смерть, яко есть Сын Бога, спасающаго род человеческий.

\mysubtitle{Песнь 4 }

\irmos{На божественней стражи, богоглаголивый Аввакум да станет с нами и покажет светоносна ангела, ясно глаголюща: днесь спасение миру, яко воскресе Христос, яко всесилен. }

\pripev[]{Христос воскресе из мертвых.}

Мужеский убо пол, яко разверзый девственную утробу, явися Христос: яко человек же, Агнец наречеся: непорочен же, яко невкусен скверны, наша Пасха, и яко Бог истинен совершен речеся. 

\pripev[]{Христос воскресе из мертвых.}

Яко единолетный агнец, благословенный нам венец Христос, волею за всех заклан бысть, Пасха чистительная, и паки из гроба красное правды нам возсия Солнце. 

\pripev[]{Христос воскресе из мертвых.}

Богоотец убо Давид, пред сенным ковчегом скакаше играя, людие же Божии святии, образов сбытие зряще, веселимся божественне, яко воскресе Христос, яко всесилен. 

\myemph{\centering Богородичны:\\}

\pripev[]{Пресвятая Богородице, спаси нас.}

Создавый Адама, Твоего праотца, Чистая, зиждется от Тебе, и смертное жилище разори Своею смертию днесь, и озари вся божественными блистаньми воскресения. 

\pripev[]{Пресвятая Богородице, спаси нас.}

Егоже родила еси Христа, прекрасно из мертвых возсиявша, Чистая, зрящи, добрая и непорочная в женах и красная, днесь во спасение всех, со апостолы радующися, Того прославляй.

\mysubtitle{Песнь 5 }

\irmos{Утренюем утреннюю глубоку, и вместо мира песнь принесем Владыце, и Христа узрим, Правды Солнце, всем жизнь возсияюща. }

\pripev[]{Христос воскресе из мертвых.}

Безмерное Твое благоутробие адовыми узами содержимии зряще, к свету идяху Христе, веселыми ногами, Пасху хваляще вечную. 

\pripev[]{Христос воскресе из мертвых.}

Приступим, свещеноснии, исходящу Христу из гроба яко жениху, и спразднуим любопразднственными чинми Пасху Божию спасительную. 

\myemph{\centering Богородичны:\\}

\pripev[]{Пресвятая Богородице, спаси нас.}

Просвещается божественными лучами и живоносными воскресения Сына Твоего, Богомати Пречистая, и радости исполняется благочестивых собрание. 

\pripev[]{Пресвятая Богородице, спаси нас.}

Не разверзл еси врата девства в воплощении, гроба не разрушил еси печатей, Царю создания: отонудуже воскресшаго Тя зрящи, Мати радовашеся. 

\mysubtitle{Песнь 6 }

\irmos{Снизшел еси в преисподняя земли и сокрушил еси вереи вечныя, содержащия связанныя Христе, и тридневен, яко от кита Иона, воскресл еси от гроба. }

\pripev[]{Христос воскресе из мертвых.}

Сохранив цела знамения, Христе, воскресл еси от гроба, ключи Девы невредивый в рождестве Твоем, и отверзл еси нам райския двери. 

\pripev[]{Христос воскресе из мертвых.}

Спасе мой, живое же и нежертвенное заколение, яко Бог Сам Себе волею привед Отцу, совоскресил еси всероднаго Адама, воскрес от гроба. 

\myemph{\centering Богородичны:\\}

\pripev[]{Пресвятая Богородице, спаси нас.}

Возведеся древле держимое смертию и тлением, Воплотившимся от Твоего пречистаго чрева, к нетленней и присносущней жизни, Богородице Дево. 

\pripev[]{Пресвятая Богородице, спаси нас.}

Сниде в преисподняя земли, в ложесна Твоя, Чистая, cшедый, и вселивыйся и воплотивыйся паче ума, и воздвиже с Собою Адама, воскрес от гроба. 

\mysubtitle{Кондак, глас 8-й}

Аще и во гроб снизшел еси, Безсмертне, но адову разрушил еси силу, и воскресл еси яко победитель, Христе Боже, женам мироносицам вещавый: радуйтеся, и Твоим апостолом мир даруяй, падшим подаяй воскресение. 

\mysubtitle{Икос}

Еже прежде солнца, Солнце зашедшее иногда во гроб, предвариша ко утру, ищущия яко дне мироносицы девы, и друга ко друзей вопияху: О другини! приидите, вонями помажем тело живоносное и погребенное, плоть Воскресившаго падшаго Адама, лежащую во гробе. Идем, потщимся якоже волсви, и поклонимся, и принесем мира яко дары, не в пеленах, но в плащанице Обвитому, и плачим, и возопиим: о Владыко, востани, падшим подаяй воскресение. 

Воскресение Христово видевше, поклонимся Святому Господу Иисусу, Единому безгрешному, Кресту Твоему покланяемся, Христе, и святое воскресение Твое поем и славим: Ты бо еси Бог наш, разве Тебе иного не знаем, имя Твое именуем. Приидите вси вернии, поклонимся святому Христову воскресению: се, бо прииде Крестом радость всему миру. Всегда благословяще Господа, поем воскресение Его: распятие бо претерпев, смертию смерть разруши. \myemph{(Трижды)} 

Воскрес Иисус от гроба, якоже прорече, даде нам живот вечный и велию милость. \myemph{(Трижды)} 

\mysubtitle{Песнь 7 }

\irmos{Отроки от пещи избавивый, быв человек, страждет яко смертен, и страстию смертное в нетления облачит благолепие, Един благословен отцев Бог, и препрославлен. }

\pripev[]{Христос воскресе из мертвых.}

Жены с миры богомудрыя в след Тебе течаху: Егоже яко мертва со слезами искаху, поклонишася радующияся Живому Богу, и Пасху тайную Твоим, Христе, учеником благовестиша. 

\pripev[]{Христос воскресе из мертвых.}

Смерти празднуем умерщвление, адово разрушение, иного жития вечнаго начало, и играюще поем Виновнаго, единаго благословеннаго отцев Бога и препрославленнаго. 

\pripev[]{Христос воскресе из мертвых.}

Яко воистинну священная и всепразднственная, сия спасительная нощь, и светозарная, светоноснаго дне, востания сущи провозвестница: в нейже безлетный Свет из гроба плотски всем возсия. 

\myemph{\centering Богородичны:\\}

\pripev[]{Пресвятая Богородице, спаси нас.}

Умертвив Сын Твой смерть, Всенепорочная, днесь, всем смертным пребывающий живот во веки веков дарова, Един благословенный отцев Бог и препрославленный. 

\pripev[]{Пресвятая Богородице, спаси нас.}

Всем царствуяй созданием, быв человек, вселися в Твою, Богоблагодатная, утробу, и распятие претерпев и смерть, воскресе боголепно, совозставив нас яко всесилен. 

\mysubtitle{Песнь 8 }

\irmos{Сей нареченный и святый день, един суббот Царь и Господь, праздников праздник, и торжество есть торжеств: в оньже благословим Христа во веки. }

\pripev[]{Христос воскресе из мертвых.}

Приидите, новаго винограда рождения, божественнаго веселия, в нарочитом дни воскресения, Царствия Христова приобщимся, поюще Его яко Бога во веки. 

\pripev[]{Христос воскресе из мертвых.}

Возведи окрест очи твои, Сионе, и виждь: се бо приидоша к тебе, яко богосветлая светила, от запада, и севера, и моря, и востока чада твоя, в тебе благословящая Христа во веки. 

\pripev[Троичен:]{Пресвятая Троице Боже наш, слава Тебе.} 

Отче Вседержителю, и Слове, и Душе, треми соединяемое во ипостасех Естество, Пресущественне и Пребожественне, в Тя крестихомся, и Тя благословим во вся веки. 

\myemph{\centering Богородичны:\\}

\pripev[]{Пресвятая Богородице, спаси нас.}

Прииде Тобою в мир Господь, Дево Богородице, и чрево адово расторг, смертным нам воскресение дарова: темже благословим Его во веки. 

\pripev[]{Пресвятая Богородице, спаси нас.}

Всю низложив смерти державу Сын Твой, Дево, Своим воскресением, яко Бог крепкий совознесе нас и обожи: темже воспеваем Его во веки. 

\mysubtitle{Песнь 9 }

\pripev{Величит душа моя воскресшаго тридневно от гроба Христа Жизнодавца.}

\irmos{Светися, светися, новый Иерусалиме: слава бо Господня на тебе возсия, ликуй ныне, и веселися, Сионе! Ты же, Чистая, красуйся, Богородице, о востании Рождества Твоего. }

\pripev{Христос новая Пасха, жертва живая, Агнец Божий, вземляй грехи мира.}

О, божественнаго! О, любезнаго! О, сладчайшаго Твоего гласа! С нами бо неложно обещался еси быти, до скончания века, Христе, Егоже вернии,  утверждение надежды имуще, радуемся. 

\pripev{Ангел вопияше Благодатней: чистая Дево, радуйся, и паки реку, радуйся! Твой Сын воскресе тридневен от гроба, и мертвыя воздвигнувый, людие, веселитеся. }

О, Пасха велия и священнейшая, Христе! О мудросте, и Слове Божий, и Cило! Подавай нам истее Тебе причащатися, в невечернем дни Царствия Твоего. 

\myemph{\centering Богородичны:\\}

\pripev[]{Пресвятая Богородице, спаси нас.}

Согласно, Дево, Тебе блажим вернии: радуйся, двере Господня, радуйся граде одушевленный; радуйся, Еяже ради нам ныне возсия свет из Тебе Рожденнаго из мертвых воскресения. 

\pripev[]{Пресвятая Богородице, спаси нас.}

Веселися и радуйся, божественная двере Света: зашедый бо Иисус во гроб, возсия, просияв солнца светлее, и верныя вся озарив, богорадованная Владычице. 

\mysubtitle{Ексапостиларий самогласен}

Плотию уснув, яко мертв, Царю и Господи, тридневен воскресл еси, Адама воздвиг от тли, и упразднив смерть: Пасха нетления, мира спасение. \myemph{(Трижды)} 

\mysubtitle{Стихиры Пасхи, глас 5-й:}

\myemph{Стих:} Да воскреснет Бог, и расточатся врази Его. 

Пасха священная нам днесь показася: Пасха нова святая, Пасха таинственная, Пасха всечестная, Пасха Христос Избавитель: Пасха непорочная, Пасха великая, Пасха верных, Пасха, двери райския нам отверзающая, Пасха всех освящающая верных. 

\myemph{Стих:} Яко исчезает дым, да исчезнут. 

Приидите от видения жены благовестницы, и Сиону рцыте: приими от нас радости благовещения Воскресения Христова; красуйся, ликуй и радуйся, Иерусалиме, Царя Христа узрев из гроба, яко жениха происходяща. 

\myemph{Стих:} Тако да погибнут грешницы от лица Божия, а праведницы да возвеселятся. 

Мироносицы жены, утру глубоку, представша гробу Живодавца, обретоша Ангела, на камени седяща, и той провещав им, сице глаголаше: что ищете живаго с мертвыми? Что плачете Нетленнаго во тли? Шедше проповедите учеником Его. 

\myemph{Стих:} Сей день, егоже сотвори Господь, возрадуемся и возвеселимся в онь. 

Пасха красная, Пасха, Господня Пасха! Пасха всечестная нам возсия. Пасха! Радостию друг друга обымем. О Пасха! Избавление скорби, ибо из гроба днесь яко от чертога возсияв Христос, жены радости исполни, глаголя: проповедите апостолом.

\slavainyne

Воскресения день, и просветимся торжеством, и друг друга обымем. Рцем, братие, и ненавидящим нас, простим вся воскресением, и тако возопиим: Христос воскресе из мертвых, смертию смерть поправ, и сущим во гробех живот даровав. 

\end{mymulticols}

\mychapterending

\mychapter{Во Святую и Великую неделю Пасхи}\begin{mymulticols}
%http://www.molitvoslov.com/text900.htm 

\mysubtitle{Стихира в начале утрени, глас 6-й}

Воскресение Твое, Христе Спасе, Ангели поют на небесех, и нас на земли сподоби чистым сердцем Тебе славити.

\mysubtitle{Тропарь, глас 5-й:}

Христос воскресе из мертвых, смертию смерть поправ, и сущим во гробех живот даровав.

\end{mymulticols}

\mychapterending

\mychapter{Часы святой Пасхи}

{\centering\myemph{Это последование бывает во всю Светлую седмицу вместо повечерия и полунощницы, а также вместо утренних и вечерних молитв.}\par}

\begin{mymulticols}

%http://www.molitvoslov.com/text902.htm 

\myemph{Аще иерей:} Благословен Бог наш:

\myemph{Мирский же глаголет:} \MolitvamiSviatyhOtecNashih

Христос воскресе из мертвых, смертию смерть поправ, и сущим во гробех живот даровав. \myemph{(Трижды)} 

Воскресение Христово видевше, поклонимся святому Господу Иисусу, Единому безгрешному. Кресту Твоему покланяемся, Христе, и святое воскресение Твое поем и славим: Ты бо еси Бог наш, разве Тебе иного не знаем, имя Твое именуем. Приидите вси вернии, поклонимся святому Христову воскресению: се бо прииде Крестом радость всему миру. Всегда благословяще Господа, поем воскресение Его: распятие бо претерпев, смертию смерть разруши. \myemph{(Трижды)} 

\mysubtitle{Ипакои, глас 8-й}

Предварившия утро яже о Марии, и обретшия камень отвален от гроба, слышаху от ангела: во свете присносущнем Сущаго, с мертвыми что ищете яко человека? Видите гробныя пелены, тецыте и миру проповедите, яко воста Господь, умертвивый смерть, яко есть Сын Бога, спасающаго род человеческий. 

\mysubtitle{Кондак, глас 8-й}

Аще и во гроб снизшел еси, Безсмертне, но адову разрушил еси силу, и воскресл еси яко победитель, Христе Боже, женам мироносицам вещавый: радуйтеся, и Твоим апостолом мир даруяй, падшим подаяй воскресение. 

\mysubtitle{Тропари, глас 8-й}

Во гробе плотски, во аде же с душею яко Бог, в раи же с разбойником, и на престоле был еси, Христе, со Отцем и Духом, вся исполняяй, Неописанный. 

\slava

Яко живоносец, яко рая краснейший, воистинну и чертога всякаго царскаго показася светлейший, Христе, гроб Твой, источник нашего воскресения. 

\inyne

Вышняго освященное Божественное селение, радуйся: Тобою бо дадеся радость, Богородице, зовущим: благословенна Ты в женах еси, всенепорочная Владычице. 

Господи, помилуй.\myemph{ (40 раз)} 

Слава Отцу и Сыну и Святому Духу, и ныне и присно, и во веки веков, аминь. 

Честнейшую херувим и славнейшую без сравнения серафим, без истления Бога Слова рождшую, сущую Богородицу Тя величаем. 

Именем Господним благослови, отче. 

\myemph{Иерей:} \MolitvamiSviatyhOtecNashih 

\myemph{И глаголем тропарь:}
Христос воскресе из мертвых: \myemph{(Трижды)} 
\myemph{Слава, и ныне:} Господи, помилуй. \myemph{(Трижды)}
Благослови.
\myemph{ И отпуст от иерея.}

\myemph{ Мирский же глаголет:} Господи Иисусе Христе, Сыне Божий, молитв ради Пречистыя Твоея Матере, преподобных и богоносных отец наших и всех святых, помилуй нас. Аминь. 

\end{mymulticols}

\mychapterending
