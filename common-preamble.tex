% The next three lines enable
% 1) search in acrobat reader using unicode search strings
% 2) copy and paste of unicode text from acrobat reader
\input glyphtounicode.tex
\input glyphtounicode-cmr.tex
\pdfgentounicode=1

\documentclass[11pt,oneside]{book}
\usepackage[a4paper]{geometry}
\usepackage[utf8]{inputenc}
\usepackage[russian]{babel}
\usepackage{xcolor}
\usepackage{pdfcolparallel}
\usepackage{graphicx}
\graphicspath{{img/},{img/tall/},{img/wide/},{uzory/},{target/header/}}

\usepackage{wrapfig}
\usepackage{fancyhdr}
\usepackage{indentfirst}
\usepackage{lettrine}
\usepackage{truncate}
\usepackage{shorttoc}
\usepackage{multicol}
\usepackage{ragged2e}

% http://ru.wikipedia.org/wiki/%D0%9A%D0%B8%D0%BD%D0%BE%D0%B2%D0%B0%D1%80%D1%8C_%28%D1%86%D0%B2%D0%B5%D1%82%29
%\definecolor{kinovar}{cmyk}{0, 0.71, 0.77, 0.11}
\definecolor{kinovar}{RGB}{227, 66, 52}

\newcommand\myheadingcolor{\color{kinovar}}
\newcommand\mychaptercolor{\color{kinovar}}
\newcommand\myframecolor{\color{kinovar}}

\usepackage{tocloft}
\tocloftpagestyle{empty}
% fill with dots after chapter titles
\renewcommand{\cftchapleader}{\cftdotfill{\cftdotsep}}
% centered TOC title
\renewcommand{\cfttoctitlefont}{\hfill\Huge\bfseries\mychaptercolor}
\renewcommand{\cftaftertoctitle}{\hfill}
\renewcommand{\cftpartpresnum}{\S\ }

\makeatletter
\renewcommand{\@pnumwidth}{3em} \renewcommand{\@tocrmarg}{4em}
\makeatother

%\usepackage{pscyr}
%\usepackage{paratype}
\usepackage{PTSerif}
\usepackage{calc,layouts}
\usepackage{eso-pic}
\usepackage{pifont}
\usepackage{ifpdf}
%\usepackage{etoolbox}

\pagestyle{fancy}
\lhead{} \chead{\myheadingcolor\textbf{\rightmark}} \rhead{}
%\renewcommand{\headrulewidth}{0.4pt}
%\renewcommand{\headrule}{{\myheadingcolor%
%\hrule width\headwidth height\headrulewidth \vskip-\headrulewidth}}

\lfoot{} \cfoot{---~\thepage~---} \rfoot{}

\newcommand{\wb}[2]{\fontsize{#1}{#2}\usefont{U}{webo}{xl}{n}}

% What follows has been elaborated starting from an original
% post of Piet van Oostrum <piet@cs.uu.nl> on comp.text.tex .
%
% A4 pages are 210 mm wide and 297 mm high, i.e. about 595
% bp wide and 842 bp high.  Allowing a 1 in (72 bp) white
% margin on all sides (predefined inside TeX), that accounts
% for a decoration border 451 bp wide and 698 bp high.  If
% we want to use WeboMints characters in square cells with
% 12 bp side, we need 37 cells (2 angles and 35 border) in
% the X direction, for a total of 37*12 = 444 bp; and 58
% cells in the Y direction, for a total of 58*12 = 696 bp.
% The two white sides have a real dimension of (595-444)/2 =
% 75 and (842-696)/2 = 73 bp's.

%\fancyhf{}
%\pagestyle{fancy}
\newcommand\myborderding{\myframecolor\ding{68}}
\newcommand\mycornerding{\myframecolor\ding{107}}
\newcommand{\fancypageframe}{%
\renewcommand{\headrulewidth}{0pt}
\setlength{\unitlength}{1bp}

\fancyhead[HL]{\wb{15bp}{15bp}%         Offset (48,-16), found
  \begin{picture}(0,0)(25,-32)%           by trial and error.
    \put(0,0){\mycornerding}%                       Upper left corner
    \multiput(12,0)(12,0){37}{\myborderding}%       Upper border
    \put(456,0){\mycornerding}%                     Upper right corner
    \multiput(0,-12)(0,-12){56}{\myborderding}%     Left border
    \put(0,-684){\mycornerding}%                    Lower left corner
    \multiput(12,-684)(12,0){37}{\myborderding}%    Lower border
    \multiput(456,-12)(0,-12){56}{\myborderding}%   Right border
    \put(456,-684){\mycornerding}%                  Lower right corner
\end{picture}}}

\fancypageframe

\fancypagestyle{plain}{%
\fancyhf{} % clear all header and footer fields
\fancyfoot[C]{ ---~\thepage~--- } % except the center
\renewcommand{\headrulewidth}{0pt}
\renewcommand{\footrulewidth}{0pt}
\fancypageframe}

\renewcommand\sectionmark[1]{}
\newcommand{\partornament}{cross}
\newcommand{\ornament}{uzor_begin_10}
\newcommand{\ornamentending}{uzor_end_3}

% calculate header offset from text body
\normalsize
\setlength\headheight{2\baselineskip}
\newlength\headoff
\addtolength\headoff{\headheight}
\addtolength\headoff{\headsep}
\addtolength\headoff{18pt}

% position chapter and toc headings higher
\makeatletter
\def\@part[#1]#2{%
    \ifnum \c@secnumdepth >-2\relax
      \refstepcounter{part}%
      \addcontentsline{toc}{part}{\thepart\hspace{1em}#1}%
    \else
      \addcontentsline{toc}{part}{#1}%
    \fi
    %\addtocontents{toc}{\protect\mbox{}\protect\hrulefill\par}
    \markboth{}{}%
    \thispagestyle{empty}
    {\centering
     \interlinepenalty \@M
     \normalfont
     %~ \ifnum \c@secnumdepth >-2\relax
       %~ \huge\bfseries \partname\nobreakspace\thepart
       %~ \par
       %~ \vskip 20\p@
     %~ \fi
     \includegraphics[width=\textwidth]{\partornament}
     \par
     \vskip 20\p@
     \Huge \bfseries #2\par}%
    \@endpart}


\def\@makechapterhead#1{%
  {\parindent \z@ \raggedright \normalfont
    \ifnum \c@secnumdepth >\m@ne
        \LARGE\bfseries \@chapapp\space \thechapter
        \par\nobreak
        \vskip 20\p@
    \fi
    \interlinepenalty\@M
    \LARGE \bfseries \centering \mychaptercolor #1\par\nobreak
    \vskip 20\p@
  }}
\def\@makeschapterhead#1{%
  {\parindent \z@ \raggedright
    \vspace*{-\headoff}
    \includegraphics[width=\textwidth]{\ornament}
    \normalfont
    \interlinepenalty\@M
    \vskip 15\p@
    \huge \bfseries \centering \mychaptercolor  #1\par\nobreak
    \vskip 10\p@
  }}

\renewcommand\section{\@startsection {section}{1}{\z@}%
                                   {-3.5ex \@plus -1ex \@minus -.2ex}%
                                   {2.3ex \@plus.2ex}%
                                   {\normalfont\Large\bfseries\myheadingcolor\centering}}


\makeatother

\title{Полный православный молитвослов на всякую потребу}
\author{www.molitvoslov.com}
%\raggedbottom
% widow/orphane control
\widowpenalty=2000
\clubpenalty=2000
\setcounter{secnumdepth}{-2}
\righthyphenmin=2

\newcommand{\mychapter}[1]{
    \cleardoublepage
    \phantomsection
    \addcontentsline{toc}{chapter}{#1}
    \chapter*{#1}
    \markright{#1}
}

\newcommand{\mychapterz}[2]{
    \cleardoublepage
    \phantomsection
    \addcontentsline{toc}{chapter}{#1}
    \chapter*{#2}
    \markright{#1}
}

\setlength\intextsep{0pt}

\newcommand{\myfig}[2][0.41]{
    \begin{wrapfigure}{l}{0pt}
    \includegraphics[width=#1\textwidth]{#2}
    \end{wrapfigure}
}

\newcommand{\myfigh}[3][0.41]{
    \begin{wrapfigure}[#3]{l}{0pt}
    \includegraphics[width=#1\textwidth]{#2}
    \end{wrapfigure}
}

\newcommand{\myfigr}[2][0.41]{
    \begin{wrapfigure}{r}{0pt}
    \includegraphics[width=#1\textwidth]{#2}
    \end{wrapfigure}
}

\newcommand{\myfigrh}[3][0.41]{
    \begin{wrapfigure}[#3]{r}{0pt}
    \includegraphics[width=#1\textwidth]{#2}
    \end{wrapfigure}
}

% was 0.7
\newcommand{\myfigure}[2][0.95]{
    \begin{center}
    \includegraphics[width=#1\columnwidth]{#2}
    \end{center}
}

\newcommand{\mypart}[1]{
\part{#1}
\markright{#1}
}

\newenvironment{mymulticols}[1][]{\begin{multicols}{2}[#1]}{\end{multicols}}

% the idea of identity environment is to provide a mechanism
% to assign some id to the element it surrounds. Its main use is
% with htlatex configs
\newcommand\myid{}
\newenvironment{identity}[1]{}{}

% Here's the command with two optional
% arguments -- hight of the raisebox and width
% of the ornament. The main command has one optional
% argument and it relays to another command also with
% an optional argument.
%
%\newcommand{\mychapterending}[1][2.5]{%
%    \def\ArgI{{#1}}% first argument
%    \mychapterendingRelay
%}
%\newcommand{\mychapterendingRelay}[1][0.4]{% second argument
%    {\nopagebreak\centering%
%        \raisebox{0pt}[\ArgI\baselineskip][0pt]{%
%            \includegraphics[width=#1\textwidth]{\ornamentending}}\par}%
%}

\newcommand{\mychapterending}[1][0.35]{% second argument
    \vfill
    {\nopagebreak\centering%
        \includegraphics[width=#1\textwidth]{\ornamentending}\par}%
    \vfill
}

\usepackage{microtype}
\UseMicrotypeSet[protrusion]{basictext}
%~ \SetProtrusion
%~ {
%~ encoding = T2A,
%~ family = *
%~ }
%~ {
%~ « = {600,      },
%~ » = {    ,  600},
%~ „ = {1000,     },
%~ “ = {    , 1000},
%~ ( = {600,      },
%~ ) = {    ,  600},
%~ ! = {    , 1000},
%~ ? = {    ,  600},
%~ : = {    , 1000},
%~ ; = {    , 1000},
%~ . = {    , 1000},
%~ - = {    , 800},
%~ %— = {    ,  500},
%~ %– = {    ,  500},
%~ {,}= {    , 1000}
%~ }
%~ \DeclareMicrotypeSet{t2atext}{encoding=T2A}
%~ \UseMicrotypeSet{t2atext}

\usepackage[unicode=true]{hyperref}
\hypersetup{pdftex, colorlinks=true, linkcolor=black, citecolor=black,
filecolor=black, urlcolor=blue, pdftitle={Полный православный молитвослов на всякую потребу}, pdfauthor={www.molitvoslov.com},
pdfsubject=, pdfkeywords={Православие,церковь,молитва,молитвослов,канон,акафист,правило,святой,вера,исцеление}}

\sloppy

\newcommand\longpage[1][1]{\enlargethispage{#1\baselineskip}}
\newcommand\shortpage[1][1]{\enlargethispage{-#1\baselineskip}}

% save the value of current parindent to be used
% later in minipage environments
\newlength\myparindent
\setlength\myparindent{\parindent}
\newcommand\noparindent{\setlength\parindent{0pt}}
\newcommand\restoreparindent{\setlength\parindent{\myparindent}}

\long\def\symbolfootnote[#1]#2{\begingroup%
\def\thefootnote{\fnsymbol{footnote}}\footnote[#1]{#2}\endgroup}

\setlength\footskip{6\baselineskip}

% Make more compact lists
\setlength\itemsep{2pt}

\makeatletter
\newcommand\mysubtitle{\@startsection{subsubsection}{3}{\z@}%
                                     {-1.0ex\@plus -1ex \@minus -.2ex}%
                                     {1.0ex \@plus .2ex}%
                                     {\normalfont\normalsize\bfseries\centering\myheadingcolor}}
\makeatother

\newcommand\mysubsection[1]{%
  \subsubsection*{\centering\myheadingcolor #1}}

\newcommand\myparagraph[1]{%
{\centering\myheadingcolor\bfseries #1\par}}

% sectioning for dictionary
\newcommand\bukva[1]{
    \section*{#1}
}
\newcommand{\bukvaending}{}
\newcommand\myemph[1]{{\small\myheadingcolor #1}}
\newcommand{\firstletter}[1]{#1}%{\myheadingcolor #1}}

\newcommand{\minicolumns}[2]{
  \medskip
  \ParallelLText{#1}
  \ParallelRText{#2}
  \ParallelPar
}

\newcommand{\myparsep}[1][0.3]{%
  \nointerlineskip \vspace{0.7\baselineskip}%
  \hspace{\fill}\rule{#1\linewidth}{0.7pt}\hspace{\fill}
  \par\nointerlineskip \vspace{0.7\baselineskip}
}

\newcommand{\mybibleref}[1]{%
  \myemph{\footnotesize #1}%
}

\newcommand\pripev[2][Припев:]{{\small\myemph{#1} #2}}
\newcommand\irmos[1]{\pripev[Ирм\'{о}с:]{#1}}
\newcommand\Bogorodichen[1]{\myemph{Богородичен:} #1}
\newcommand\slavan{Слава Отцу и Сыну и Святому Духу.}
\newcommand\slava{{\small\slavan}}
\newcommand\inynen{И ныне и присно и во веки веков. Аминь.}
\newcommand\inyne{{\small\inynen}}
\newcommand\slavainynen{Слава Отцу и Сыну и Святому Духу. И ныне и присно и во веки веков. Аминь.}
\newcommand\slavainyne{{\small\slavainynen}}

\newcommand{\pripevc}[1]{{\small \centerline{#1} \nopagebreak}}
\newcommand{\pripevmskipc}[1]{\medskip\pripevc{#1}}
\newcommand{\pripevpomiluj}{\pripevmskipc{\pripev{\firstletter{П}омилуй мя, Боже, помилуй мя.}}}
\newcommand{\slavac}{\pripevmskipc{\slavan}}
\newcommand{\inynec}{\pripevmskipc{\inynen}}

\newcommand{\TsariuNebesnyj}{%
Царю Небесный, Утешителю, Душе истины, Иже везде сый и вся исполняяй, Сокровище благих и жизни Подателю, прииди и вселися в ны, и очисти ны от всякия скверны, и спаси, Блаже, души наша.}

\newcommand{\TrisviatoePoOtcheNash}{%
Святый Боже, Святый Крепкий, Святый Безсмертный, помилуй нас. \myemph{ (Tрижды)}

Слава Отцу и Сыну и Святому Духу, и ныне и присно и во веки веков. Аминь.

Пресвятая Троице, помилуй нас; Господи, очисти грехи наша; Владыко, прости беззакония наша; Святый, посети и исцели немощи наша, имене Твоего ради.

Господи, помилуй. \myemph{ (Трижды)}

Слава Отцу и Сыну и Святому Духу, и ныне и присно и во веки веков. Аминь.

Отче наш, Иже еси на небесех! Да святится имя Твое, да приидет Царствие Твое, да будет воля Твоя, яко на небеси и на земли. Хлеб наш насущный даждь нам днесь; и остави нам долги наша, якоже и мы оставляем должником нашим; и не введи нас во искушение, но избави нас от лукаваго.
}

\newcommand{\priiditepoklonimsia}{%
Приидите, поклонимся Цареви нашему Богу. \myemph{(Поклон)}

Приидите, поклонимся и припадем Христу, Цареви нашему Богу. \myemph{(Поклон)}

Приидите, поклонимся и припадем Самому Христу, Цареви и Богу нашему. \myemph{(Поклон)}}


\newcommand{\PsalmFifty}{%
Помилуй мя, Боже, по велицей милости Твоей, и по множеству щедрот Твоих очисти беззаконие мое. Наипаче омый мя от беззакония моего, и от греха моего очисти мя; яко беззаконие мое аз знаю, и грех мой предо мною есть выну. Тебе Единому согреших и лукавое пред Тобою сотворих, яко да оправдишися во словесех Твоих, и победиши внегда судити Ти. Се бо, в беззакониих зачат есмь, и во гресех роди мя мати моя. Се бо, истину возлюбил еси; безвестная и тайная премудрости Твоея явил ми еси. Окропиши мя иссопом, и очищуся; омыеши мя, и паче снега убелюся. Слуху моему даси радость и веселие; возрадуются кости смиренныя. Отврати лице Твое от грех моих и вся беззакония моя очисти. Сердце чисто созижди во мне, Боже, и дух прав обнови во утробе моей. Не отвержи мене от лица Твоего и Духа Твоего Святаго не отыми от мене. Воздаждь ми радость спасения Твоего и Духом владычним утверди мя. Научу беззаконыя путем Твоим, и нечестивии к Тебе обратятся. Избави мя от кровей, Боже, Боже спасения моего; возрадуется язык мой правде Твоей. Господи, устне мои отверзеши, и уста моя возвестят хвалу Твою. Яко аще бы восхотел еси жертвы, дал бых убо: всесожжения не благоволиши. Жертва Богу дух сокрушен; сердце сокрушенно и смиренно Бог не уничижит. Ублажи, Господи, благоволением Твоим Сиона, и да созиждутся стены Иерусалимския. Тогда благоволиши жертву правды, возношение и всесожегаемая; тогда возложат на oлтарь Твой тельцы.\par}

\newcommand{\Chestneyshuyu}{%
Достойно есть яко воистинну блажити Тя, Богородицу, Присноблаженную и Пренепорочную и Матерь Бога нашего. Честнейшую Херувим и славнейшую без сравнения Серафим, без истления Бога Слова рождшую, сущую Богородицу Тя величаем.}

\newcommand{\MolitvamiSviatyhOtecNashih}{%
Молитвами святых отец наших, Господи Иисусе Христе, Боже наш, помилуй нас. Аминь.}

\newcommand{\tolkopoblagosloveniyu}{%
{\centering\myemph{\normalfont Читаются только по благословению духовника}\par}}

\newcommand{\SymbolOfFaith}{%
  Верую во единаго Бога Отца, Вседержителя, Творца небу и земли, видимым же всем и невидимым.
  И во единаго Господа Иисуса Христа, Сына Божия, Единороднаго, Иже от Отца рожденнаго прежде всех век; Света от Света, Бога истинна от Бога истинна, рожденна, несотворенна, единосущна Отцу, Имже вся быша.
  Нас ради человек и нашего ради спасения сшедшаго с небес и воплотившагося от Духа Свята и Марии Девы и вочеловечшася.
  Распятаго же за ны при Понтийстем Пилате, и страдавша, и погребенна.
  И воскресшаго в третий день по Писанием.
  И возшедшаго на небеса, и седяща одесную Отца.
  И паки грядущаго со славою судити живым и мертвым, Егоже Царствию не будет конца.
  И в Духа Святаго, Господа, Животворящаго, Иже от Отца исходящего, Иже со Отцем и Сыном спокланяема и сславима, глаголавшаго пророки.
  Во едину Святую, Соборную и Апостольскую Церковь.
  Исповедую едино крещение во оставление грехов.
  Чаю воскресения мертвых, и жизни будущаго века. Аминь.}

\newcommand{\TroparPomilujNas}{Помилуй нас, Господи, помилуй нас; всякаго бо ответа недоумеюще, сию Ти молитву яко Владыце грешнии приносим: помилуй нас.

\slavan

 Господи, помилуй нас, на Тя бо уповахом; не прогневайся на ны зело, ниже помяни беззаконий наших, но призри и ныне яко благоутробен, и избави ны от враг наших; Ты бо еси Бог наш, и мы людие Твои, вси дела руку Твоею, и имя Твое призываем.

\inynen

 Милосердия двери отверзи нам, благословенная Богородице, надеющиися на Тя да не погибнем, но да избавимся Тобою от бед: Ты бо еси спасение рода христианскаго.}



