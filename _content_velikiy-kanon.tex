

\mypart{ВЕЛИКИЙ КАНОН (ТВОРЕНИЕ СВЯТОГО АНДРЕЯ КРИТСКОГО)}\label{_content_velikiy-kanon}
%http://www.molitvoslov.com/content/velikiy-kanon

 

%\mychapter{Весь Канон целиком на церковнославянском языке (pdf)}
%http://www.molitvoslov.com/content/Ves-Kanon-tselikom-na-tserkovnoslayavyanskom-yazyke-pdf 
 
\myfig{300px-Andreas_cretensis_0}

Первая седмица Великого Поста с древних времен называется «зарей воздержания» и «неделей чистой». В эту неделю Церковь убеждает своих чад выйти из того греховного состояния, в которое невоздержанием наших прародителей ниспал весь род человеческий, утратив райское блаженство, и которое каждый из нас умножает сам своими грехами,"--- выйти путем веры, молитвы, смирения и Богоугодного поста. Се время покаяния, говорит Церковь, се день спасительный, пощения вход: душе, бодрствуй, и страстей входа затвори, ко Господу взирающе  (из первой песни трипеснца на утрени понедельника первой седмицы Великого Поста).


Подобно ветхозаветной церкви, которая особенно святила первый и последний день некоторых великих праздников, православные христиане, приготовленные и воодушевленные матерними внушениями своей Церкви, издревле, согласно ее уставу, с особенным усердием и строгостью проводят первую и последнюю седмицу Великого Поста.


На первой седмице совершаются особенно продолжительные богослужения и подвиг телесного воздержания значительно более строгий, чем в последующие дни Святой Четыредесятницы. В первые четыре дня Великого Поста великое повечерие совершается с чтением Великого Покаянного Канона преп. Андрея Критского, который как бы задает «тон», определяет всю последующую тональность, «мелодию» Великого Поста. На первой седмице Поста Канон делится на четыре части. Дивное творение св. Андрея Критского полностью предлагается нашему вниманию и в четверг (точнее в среду вечером) пятой седмицы Св. Четыредесятницы для того, чтобы мы, видя приближающееся окончание Поста, не разленились к духовным подвигам, не сделались небрежными, не забылись и не перестали во всем строго следить за собой.


Каждый стих Великого Канона сопровождается псаломским припевом Помилуй мя, Боже, помилуй мя! К канону присоединяется несколько тропарей в честь самого автора "--- св. Андрея и преп. Марии Египетской. Еще при жизни св. Андрея Иерусалимская Церковь ввела у себя в употребление Великий Канон. Отправляясь в 680 году на Шестой Вселенский Собор в Константинополь, св. Андрей принес туда и сделал известным свое великое творение и житие преп. Марии Египетской, написанное его соотечественником и учителем патриархом Иерусалимским Софронием. Житие египетской подвижницы читается совместно с Великим Каноном на утрени в среду пятой седмицы Великого Поста.


Из всех молитвословий Великого Поста, больше всех других поражает душу Великий Канон. Великий Канон "--- это чудо церковной гимнографии, это тексты удивительной силы и поэтической красоты. Канон составлен в 7-м веке св. Андреем, Архиепископом Критским, составившим также много других канонов, которые Церковь использует в течение всего богослужебного года. Церковь именовала этот канон великим, не столько из-за его объема (в нем 250 тропарей или стихов), сколько по его внутреннему достоинству и силе.


Великий канон представляет собой беседу кающегося с собственной душой. Вот как он начинается:


Откуда начну плакати окаянного моего жития деяний? Кое ли положу начало, Христе, нынешнему рыданию? Но яко благоутробен даждь ми прегрешений оставление "--- с чего же начать мне каяться, ведь это так трудно.


Затем следует чудный тропарь:


Гряди, окаянная душе, с плотию твоею. Зиждителю всех исповеждься и останися прочее преждняго безсловесия, и принеси Богу в покаянии слезы.


Удивительные слова, тут и христианская антропология, и аскетика: плоть тоже должна участвовать в покаянии, как неотъемлемая часть человеческого естества.


Своего апогея эта беседа с душой, постоянные уговоры ее, призывы покаяться, достигают в кондаке, который поется после 6-й песни Канона:


Душе моя, душе моя, востани, что спиши? Конец приближается, и имаши смутитися; воспряни убо, да пощадит тя Христос Бог, везде сый, и вся исполняяй.


Эти слова произносит, обращаясь к себе, великий светильник Церкви, тот, к кому самому применимо было бы выражение, употребленное им относительно преп. Марии Египетской, которая действительно была «ангел во плоти». И вот он так к себе обратился, упрекая себя за то, что душа его спит. Если он видел себя таким, то какими должны видеть себя мы? Погруженные уже не только в беспробудный духовный сон, но в какое-то омертвение…


Когда мы внимаем словам кондака из канона святителя Андрея Критского, нужно спросить себя: что мне делать? Если бы человек как должно исполнял Божий закон, его жизнь была бы заполнена совсем другим содержанием. Вот для того Церковь и предлагает нам этот глубокий, проникновенный великопостный покаянный канон, чтобы мы заглянули поглубже в свою душу и посмотрели бы, что там. А душа-то спит… В этом и горе наше и беда наша.


В замечательной молитве преп. Ефрема Сирина, которую мы повторяем в продолжение всего Великого Поста, говорится: Господи Царю, даруй ми зрети моя прегрешения! "--- Я их не вижу, душа моя заснула, задремала и я этих грехов, как должно, даже и не вижу. Как же я буду каяться в них! И вот потому нужно в дни Великого Поста побольше сосредотачиваться на себе, оценивая свою жизнь и ее содержание евангельской мерой, а не какой-нибудь другой.


К основным особенностям Великого канона относится очень широкое использование образов и сюжетов из Священного Писания, как Ветхого, так и Нового Завета. Жаль, что мы плохо знакомы со Святой Библией. Многим из нас имена людей, упомянутых в Великом Каноне, ничего не говорят, потому что мы плохо знаем Библию.


А между тем Библия "--- это не только история израильского народа, но и грандиозная летопись души человеческой "--- души, которая падала и вставала перед лицом Бога, которая согрешала и каялась. Если мы посмотрим на жизнь людей, о которых говорится в Библии, то увидим, что каждый из них представлен не столько как исторический персонаж, не столько как личность, совершившая те или иные дела, сколько как человек, предстоящий перед лицом Живого Бога. Исторические и иные заслуги человека отходят на второй план, остается то, что всего важнее: сохранил человек верность Богу или нет. Если мы будем читать Библию и Великий Канон под таким углом зрения, то увидим, что многое из того, что говорится о древних праведниках и грешниках, является не чем иным, как летописью нашей души, наших падений и восстаний, наших грехов и покаяния.


Один церковный писатель по этому поводу очень кстати замечает: «Если в наши дни столь многие находят его (Великий Канон) скучным и не относящимся к нашей жизни, это происходит оттого, что вера их не питается из источника Священного Писания, которое для Отцов Церкви было именно источником их веры. Мы должны вновь научиться воспринимать мир таким, каким он открывается нам в Библии, научиться жить в этом библейском мире; и нет лучшего способа научиться этому, как именно через церковное богослужение, которое не только передает нам библейское учение, но и открывает нам библейский образ жизни» (Протопр. Александр Шмеман, «Великий Пост», стр. 97).


Итак, в Великом Каноне перед нами проходит в лицах и событиях вся ветхозаветная и новозаветная история. Автор указывает на грехопадение прародителей и растление первобытного мира, на добродетели Ноя и нераскаянность и ожесточение жителей Содома и Гоморры, воскрешает перед нами память благочестивых патриархов и доблестных мужей: Моисея, Иисуса Навина, Гедеона и Иефая, представляет нашему взору благочестие царя Давида, его падение и умиленное покаяние, указывает на нечестие Ахава и Иезавели и на великие образцы покаяния "--- неневитян, Манассию, блудницу и благоразумного разбойника, и в особенности на Марию Египетскую, неоднократно останавливает читателя у Креста и Гроба Господня "--- везде поучая покаянию, смирению, молитве, самоотвержению. На этих примерах постоянно происходит увещание души "--- вспомни этого праведника, он так угодил Богу, вспомни и этого праведника, он так угодил "--- ты ничего подобного не сделала.


Об одних перснонажах Библии говорится в положительном смысле, о других в отрицательном, кому-то нужно подражать, а кому-то нет.


Колесничник Илия, колесницею добродетелей вшед, яко на Небеса, ношашеся превыше иногда от земных. Сего убо, душе моя, восход помышляй "--- помышляй, душа моя, о восхождении ветхозаветных праведников.


Гиезиев подражала еси, окаянная, разум скверный всегда, душе, егоже сребролюбие отложи поне на старость, бегай геенскаго огня, отступивши злых твоих "--- хотя бы в старости отринь сребролюбие Гиезии, душа, и оставив свои злодеяния, избегай геенского огня.


Как видите, тексты довольно трудные, поэтому к восприятию Великого канона необходимо заранее готовиться.


В заключительной песни первого дня после всех воспоминаний следуют тропари удивительной силы:


Закон изнеможе, празднует Евангелие, писание же все в тебе небрежено бысть, пророцы изнемогоша, и все праведное слово: струпи твои, о душе, умножишася, не сущу врачу исцеляющему тя "--- нечего вспоминать из Ветхого Завета, все бесполезно. Буду приводить тебе примеры из Нового Завета, может быть, тогда ты покаешься:


Новаго привожду ти писания указания, вводящая тя, душе, ко умилению: праведным убо поревнуй, грешных же отвращайся, и умилостиви Христа молитвами же и пощеньми и чистотою и говением.


Наконец, духовный писатель, представив все ветхозаветное, восходит к Жизнодавцу, Спасителю душ наших, восклицая, как разбойник: Помяни мя!, взывая, как мытарь: Боже милостив буди мне грешному!, подражая в неотступности Хананеянке и слепцам на распутии: Помилуй мя, сыне Давидов!, источая слезы, вместо мира, на главу и ноги Христа, подобно блуднице, и горько плача над собою, как Марфа и Мария над Лазарем.


Далее в Каноне подчеркивается, что самые страшные грешники покаялись и придут в Царство Небесное прежде нас: Христос вочеловечися, призвав к покаянию разбойники и блудницы: душе, покайся, дверь отверзеся Царствия уже, и предвосхищают о фарисее и мытарии и прелюбодеи кающиися.


Когда же, в некоем духовном ужасе, следуя издали за чудесами Спасителя и умиляясь над каждым подвигом Его земной жизни, автор Канона доходит до страшного заколения Христа,"--- силы сердца его оскудевает и, вместе со всей тварью, он умолкает на трепещущей Голгофе, в последний раз воскликнув: Судие мой и ведче мой, хотяй паки приити со ангелы, судити миру всему, милостивным твоим оком тогда, видев мя пощади и ущедри мя, Иисусе, паче всякого естества человеча согрешивша.


Великий канон, всеми средствами подвигая нас к покаянию, в последних тропарях как бы открывает нам свою «методику»: как я с тобой беседовал, душа, и праведников ветхозаветных тебе напоминал, и новозаветные образы тебе в пример приводил, и все напрасно: ихже не поревновала еси, душе, ни деянием, ни житию: но горе тебе, внегда будеши судитися "--- горе тебе, когда предстанешь на суд!


Вслушиваясь в слова Великого Канона, всматриваясь в историю жизни людей, бежавших от Бога, но настигнутых Им, людей, которые оказывались в безднах, но которых Бог выводил оттуда, подумаем о том, как Бог выводит каждого из нас из бездны греха и отчаяния для того, чтобы мы принесли Ему плоды покаяния.


Не нужно думать, что покаяние заключается в том, чтобы копаться в собственных грехах, заниматься самобичеванием, стараться выявить в себе как можно больше злого и темного. Истинное покаяние "--- это когда мы обращаемся от тьмы к свету, от греха к праведности; когда мы понимаем, что прежняя наша жизнь была недостойна высокого призвания, когда перед лицом Бога мы сознаем, как ничтожны мы сами, и сознаем, что единственная наша надежда "--- Сам Бог. Истинное покаяние "--- это когда перед лицом Бога, по слову Апостола Петра «призвавшего нас из тьмы в чудный Свой свет» (1 Пет. 2: 9), мы понимаем, что жизнь дана нам для того, чтобы стать детьми Божиими, чтобы приобщиться к Божественному свету. Истинное покаяние "--- то, которое выражается не столько в словах, но и в делах, в готовности прийти на помощь людям, в открытости по отношению к ближним, а не в обращенности на себя. Истинное покаяние "--- это когда мы понимаем, что, хотя и не в наших силах стать настоящими христианами, Бог в силах сделать нас таковыми. Как говорится в Великом Каноне, идеже бо хощет Бог, побеждается естества чин. Другими словами там, где Бог хочет, происходят сверхъестественные события: Савл становится Павлом, Иона изводится из чрева кита, Моисей проходит через море по суше, умерший Лазарь воскресает, Мария Египетская из блудницы превращается в великую праведницу. Ибо, по словам Спасителя, «человекам это невозможно, Богу же все возможно» (Мф. 19: 26).


©Протоиерей Виктор Потапов


www.stjohndc.org


февраль 2001 г.
\mychapterending

\mychapter{В понедельник первой седмицы Великого поста}
%http://www.molitvoslov.com/text571.htm 
 
\mysubsubsection{Песнь 1}

\itshape Ирмос\normalfont{}: Помощник и Покровитель бысть мне во спасение, Сей мой Бог, и прославлю Его, Бог Отца моего, и вознесу Его: славно бо прославися. 

Помилуй мя, Боже, помилуй мя. 

Откуду начну плакати окаяннаго моего жития деяний? кое ли положу начало, Христе, нынешнему рыданию? но, яко благоутробен, даждь ми прегрешений оставление. 

Гряди, окаянная душе, с плотию твоею, Зиждителю всех исповеждься и останися прочее преждняго безсловесия, и принеси Богу в покаянии слезы. 

Первозданнаго Адама преступлению поревновав, познах себе обнажена от Бога и присносущнаго Царствия и сладости, грех ради моих. 

Увы мне, окаянная душе, что уподобилася еси первей Еве? видела бо еси зле, и уязвилася еси горце, и коснулася еси древа, и вкусила еси дерзостно безсловесныя снеди. 

Вместо Евы чувственныя мысленная ми бысть Ева, во плоти страстный помысл, показуяй сладкая и вкушаяй присно горькаго напоения. 

Достойно из Едема изгнан бысть, яко не сохранив едину Твою, Спасе, заповедь Адам: аз же что постражду, отметая всегда животная Твоя словеса? 

\itshape Слава\normalfont{}: Пресущная Троице, во Единице покланяемая, возьми бремя от мене тяжкое греховное и, яко благоутробна, даждь ми слезы умиления. 

\itshape И ныне\normalfont{}: Богородице, Надежде и Предстательство Тебе поющих, возьми бремя от мене тяжкое греховное, и, яко Владычица Чистая, кающася приими мя. 

\mysubsubsection{Песнь 2}

\itshape Ирмос\normalfont{}: Вонми, Небо, и возглаголю, и воспою Христа, от Девы плотию пришедшаго. 

Вонми, Небо, и возглаголю, земле, внушай глас, кающийся к Богу и воспевающий Его. 

Вонми ми, Боже, Спасе мой, милостивым Твоим оком и приими мое теплое исповедание. 

Согреших паче всех человек, един согреших Тебе; но ущедри, яко Бог, Спасе, творение Твое. 

Вообразив моих страстей безобразие, любосластными стремленьми погубих ума красоту. 

Буря мя злых обдержит, благоутробне Господи; но яко Петру и мне руку простри. 

Оскверних плоти моея ризу и окалях, еже по образу, Спасе, и по подобию. 

Омрачих душевную красоту страстей сластьми и всячески весь ум персть сотворих. 

Раздрах ныне одежду мою первую, юже ми изтка Зиждитель из начала, и оттуду лежу наг. 

Облекохся в раздранную ризу, юже изтка ми змий советом, и стыждуся. 

Слезы блудницы, Щедре, и аз предлагаю, очисти мя, Спасе, благоутробием Твоим. 

Воззрех на садовную красоту и прельстихся умом: и оттуду лежу наг и срамляюся. 

Делаша на хребте моем вси начальницы страстей, продолжающе на мя беззаконие их. 

\itshape Слава\normalfont{}: Единаго Тя в Триех Лицех, Бога всех пою, Отца и Сына и Духа Святаго. 

\itshape И ныне\normalfont{}: Пречистая Богородице Дево, Едина Всепетая, моли прилежно, во еже спастися нам. 

\mysubsubsection{Песнь 3}

\itshape Ирмос\normalfont{}: На недвижимом, Христе, камени заповедей Твоих утверди мое помышление. 

Огнь от Господа иногда Господь одождив, землю содомскую прежде попали. 

На горе спасайся, душе, якоже Лот оный, и в Сигор угонзай. 

Бегай запаления, о душе, бегай содомскаго горения, бегай тления Божественнаго пламене. 

Согреших Тебе един аз, согреших паче всех, Христе Спасе, да не презриши мене. 

Ты еси Пастырь добрый, взыщи мене, агнца, и заблуждшаго да не презриши мене. 

Ты еси сладкий Иисусе, Ты еси Создателю мой, в Тебе, Спасе, оправдаюся. 

Исповедаюся Тебе, Спасе, согреших, согреших Ти; но ослаби, остави ми, яко благоутробен. 

\itshape Слава\normalfont{}: О Троице Единице Боже, спаси нас от прелести, и искушений, и обстояний. 

\itshape И ныне\normalfont{}: Радуйся, Богоприятная утробо, радуйся, престоле Господень, радуйся, Мати Жизни нашея. 

\mysubsubsection{Песнь 4}

\itshape Ирмос\normalfont{}: Услыша пророк пришествие Твое, Господи, и убояся, яко хощеши от Девы родитися и человеком явитися, и глаголаше: услышах слух Твой и убояхся, слава силе Твоей, Господи. 

Дел Твоих да не презриши, создания Твоего да не оставиши, Правосуде. Аще и един согреших, яко человек, паче всякаго человека, Человеколюбче; но имаши, яко Господь всех, власть оставляти грехи. 

Приближается, душе, конец, приближается, и нерадиши, ни готовишися, время сокращается: востани, близ при дверех Судия есть. Яко соние, яко цвет, время жития течет: что всуе мятемся? 

Воспряни, о душе моя, деяния твоя, яже соделала еси, помышляй, и сия пред лице твое принеси, и капли испусти слез твоих; рцы со дерзновением деяния и помышления Христу и оправдайся. 

Не бысть в житии греха, ни деяния, ни злобы, еяже аз, Спасе, не согреших, умом, и словом, и произволением, и предложением, и мыслию, и деянием согрешив, яко ин никтоже когда. 

Отсюду и осужден бых, отсюду препрен бых аз, окаянный, от своея совести, еяже ничтоже в мире нужнейше: Судие, Избавителю мой и Ведче, пощади, и избави, и спаси мя, раба Твоего. 

Лествица, юже виде древле великий в патриарсех, указание есть, душе моя, деятельнаго восхождения, разумнаго возшествия: аще хощеши убо деянием, и разумом, и зрением пожити, обновися. 

Зной дневный претерпе лишения ради патриарх и мраз нощный понесе, на всяк день снабдения творя, пасый, труждаяйся, работаяй, да две жене сочетает. 

Жены ми две разумей, деяние же и разум в зрении, Лию убо деяние, яко многочадную, Рахиль же разум, яко многотрудную; ибо кроме трудов ни деяние, ни зрение, душе, исправится. 

\itshape Слава\normalfont{}: Нераздельное Существом, Неслитное Лицы богословлю Тя, Троическое Едино Божество, яко Единоцарственное и Сопрестольное, вопию Ти песнь великую, в вышних трегубо песнословимую. 

\itshape И ныне\normalfont{}: И раждаеши, и девствуеши, и пребываеши обоюду естеством Дева, Рождейся обновляет законы естества, утроба же раждает нераждающая. Бог идеже хощет, побеждается естества чин: творит бо, елика хощет. 

\mysubsubsection{Песнь 5}

\itshape Ирмос\normalfont{}: От нощи утренююща, Человеколюбче, просвети, молюся, и настави и мене на повеления Твоя, и научи мя, Спасе, творити волю Твою. 

В нощи житие мое преидох присно, тьма бо бысть, и глубока мне мгла, нощь греха, но яко дне сына, Спасе, покажи мя. 

Рувима подражая, окаянный аз, содеях беззаконный и законопреступный совет на Бога Вышняго, осквернив ложе мое, яко отчее он. 

Исповедаюся Тебе, Христе Царю: согреших, согреших, яко прежде Иосифа братия продавшии, чистоты плод и целомудрия. 

От сродников праведная душа связася, продася в работу сладкий, во образ Господень: ты же вся, душе, продалася еси злыми твоими. 

Иосифа праведнаго и целомудреннаго ума подражай, окаянная и неискусная душе, и не оскверняйся безсловесными стремленьми, присно беззаконнующи. 

Аще и в рове поживе иногда Иосиф, Владыко Господи, но во образ погребения и востания Твоего: аз же что Тебе когда сицевое принесу? 

\itshape Слава\normalfont{}: Тя, Троице, славим Единаго Бога: Свят, Свят, Свят еси, Отче, Сыне и Душе, Простое Существо, Единице присно покланяемая. 

\itshape И ныне\normalfont{}: Из Тебе облечеся в мое смешение, нетленная, безмужная Мати Дево, Бог, создавый веки, и соедини Себе человеческое естество. 

\mysubsubsection{Песнь 6}

\itshape Ирмос\normalfont{}: Возопих всем сердцем моим к щедрому Богу, и услыша мя от ада преисподняго, и возведе от тли живот мой. 

Слезы, Спасе, очию моею и из глубины воздыхания чисте приношу, вопиющу сердцу: Боже, согреших Ти, очисти мя. 

Уклонилася еси, душе, от Господа твоего, якоже Дафан и Авирон, но пощади, воззови из ада преисподняго, да не пропасть земная тебе покрыет. 

Яко юница, душе, разсвирепевшая, уподобилася еси Ефрему, яко серна от тенет сохрани житие, вперивши деянием ум и зрением. 

Рука нас Моисеова да уверит, душе, како может Бог прокаженное житие убелити и очистити, и не отчайся сама себе, аще и прокаженна еси. 

\itshape Слава\normalfont{}: Троица есмь Проста, Нераздельна, раздельна Личне, и Единица есмь естеством соединена, Отец глаголет, и Сын, и Божественный Дух. 

\itshape И ныне\normalfont{}: Утроба Твоя Бога нам роди, воображена по нам: Егоже, яко Создателя всех, моли, Богородице, да молитвами Твоими оправдимся. 

Господи, помилуй. (Трижды.) 

\itshape Слава, и ныне\normalfont{}: 

\itshape Кондак, глас 6: 

\normalfont{}Душе моя, душе моя, востани, что спиши? конец приближается, и имаши смутитися: воспряни убо, да пощадит тя Христос Бог, везде сый и вся исполняяй. 

\mysubsubsection{Песнь 7}

\itshape Ирмос\normalfont{}: Согрешихом, беззаконновахом, неправдовахом пред Тобою, ниже соблюдохом, ниже сотворихом, якоже заповедал еси нам; но не предаждь нас до конца, отцев Боже. 

Согреших, беззаконновах и отвергох заповедь Твою, яко во гресех произведохся, и приложих язвам струпы себе; но Сам мя помилуй, яко благоутробен, отцев Боже. 

Тайная сердца моего исповедах Тебе, Судии моему, виждь мое смирение, виждь и скорбь мою, и вонми суду моему ныне, и Сам мя помилуй, яко благоутробен, отцев Боже. 

Саул иногда, яко погуби отца своего, душе, ослята, внезапу царство обрете к прослутию; но блюди, не забывай себе, скотския похоти твоя произволивши паче Царства Христова. 

Давид иногда Богоотец, аще и согреши сугубо, душе моя, стрелою убо устрелен быв прелюбодейства, копием же пленен быв убийства томлением; но ты сама тяжчайшими делы недугуеши, самохотными стремленьми. 

Совокупи убо Давид иногда беззаконию беззаконие, убийству же любодейство растворив, покаяние сугубое показа абие; но сама ты, лукавнейшая душе, соделала еси, не покаявшися Богу. 

Давид иногда вообрази, списав яко на иконе песнь, еюже деяние обличает, еже содея, зовый: помилуй мя, Тебе бо единому согреших всех Богу, Сам очисти мя. 

\itshape Слава\normalfont{}: Троице Простая, Нераздельная, Единосущная и Естество Едино, Светове и Свет, и Свята Три, и Едино Свято поется Бог Троица; но воспой, прослави Живот и Животы, душе, всех Бога. 

\itshape И ныне\normalfont{}: Поем Тя, благословим Тя, покланяемся Ти, Богородительнице, яко Нераздельныя Троицы породила еси Единаго Христа Бога, и Сама отверзла еси нам, сущим на земли, Небесная. 

\mysubsubsection{Песнь 8}

\itshape Ирмос\normalfont{}: Егоже воинства Небесная славят, и трепещут херувими и серафими, всяко дыхание и тварь, пойте, благословите и превозносите во вся веки. 

Согрешивша, Спасе, помилуй, воздвигни мой ум ко обращению, приими мя кающагося, ущедри вопиюща: согреших Ти, спаси, беззаконновах, помилуй мя. 

Колесничник Илия колесницею добродетелей вшед, яко на небеса, ношашеся превыше иногда от земных: сего убо, душе моя, восход помышляй. 

Елиссей иногда прием милоть Илиину, прият сугубую благодать от Бога; ты же, о душе моя, сея не причастилася еси благодати за невоздержание. 

Иорданова струя первее милотию Илииною Елиссеем ста сюду и сюду; ты же, о душе моя, сея не причастилася еси благодати за невоздержание. 

Соманитида иногда праведнаго учреди, о душе, нравом благим; ты же не ввела еси в дом ни странна, ни путника. Темже чертога изринешися вон, рыдающи. 

Гиезиев подражала еси, окаянная, разум скверный всегда, душе, егоже сребролюбие отложи поне на старость; бегай геенскаго огня, отступивши злых твоих. 

\itshape Слава\normalfont{}: Безначальне Отче, Сыне Собезначальне, Утешителю Благий, Душе Правый, Слова Божия Родителю, Отца Безначальна Слове, Душе Живый и Зиждяй, Троице Единице, помилуй мя. 

\itshape И ныне\normalfont{}: Яко от оброщения червленицы, Пречистая, умная багряница Еммануилева внутрь во чреве Твоем плоть исткася. Темже Богородицу воистинну Тя почитаем. 

\mysubsubsection{Песнь 9}

\itshape Ирмос\normalfont{}: Безсеменнаго зачатия Рождество несказанное, Матере безмужныя нетленен Плод, Божие бо Рождение обновляет естества. Темже Тя вси роди, яко Богоневестную Матерь, православно величаем. 

Ум острупися, тело оболезнися, недугует дух, слово изнеможе, житие умертвися, конец при дверех. Темже, моя окаянная душе, что сотвориши, егда приидет Судия испытати твоя? 

Моисеово приведох ти, душе, миробытие, и от того все заветное Писание, поведающее тебе праведныя и неправедныя: от нихже вторыя, о душе, подражала еси, а не первыя, в Бога согрешивши. 

Закон изнеможе, празднует Евангелие, Писание же все в тебе небрежено бысть, пророцы изнемогоша и все праведное слово; струпи твои, о душе, умножишася, не сущу врачу, исцеляющему тя. 

Новаго привожду ти Писания указания, вводящая тя, душе, ко умилению: праведным убо поревнуй, грешных же отвращайся и умилостиви Христа молитвами же, и пощеньми, и чистотою, и говением. 

Христос вочеловечися, призвав к покаянию разбойники и блудницы; душе, покайся, дверь отверзеся Царствия уже, и предвосхищают е фарисее и мытари и прелюбодеи кающиися. 

Христос вочеловечися, плоти приобщився ми, и вся елика суть естества хотением исполни греха кроме, подобие тебе, о душе, и образ предпоказуя Своего снисхождения. 

Христос волхвы спасе, пастыри созва, младенец множества показа мученики, старцы прослави и старыя вдовицы, ихже не поревновала еси, душе, ни деянием, ни житию, но горе тебе, внегда будеши судитися. 

Постився Господь дний четыредесять в пустыни, последи взалка, показуя человеческое; душе, да не разленишися, аще тебе приложится враг, молитвою же и постом от ног твоих да отразится. 

\itshape Слава\normalfont{}: Отца прославим, Сына превознесем, Божественному Духу верно поклонимся, Троице Нераздельней, Единице по существу, яко Свету и Светом, и Животу и Животом, животворящему и просвещающему концы. 

\itshape И ныне\normalfont{}: Град Твой сохраняй, Богородительнице Пречистая, в Тебе бо сей верно царствуяй, в Тебе и утверждается, и Тобою побеждаяй, побеждает всякое искушение, и пленяет ратники, и проходит послушание. 

\bfseries 

Преподобне отче Андрее, моли Бога о нас.

\normalfont{}

Андрее честный и отче треблаженнейший, пастырю Критский, не престай моляся о воспевающих тя: да избавимся вси гнева и скорби, и тления, и прегрешений безмерных, чтущии твою память верно. 

Таже оба лика вкупе поют\bfseries  \normalfont{}\itshape Ирмос: 

\normalfont{}Безсеменнаго зачатия Рождество несказанное, Матере безмужныя нетленен Плод, Божие бо Рождение обновляет естества. Темже Тя вси роди, яко Богоневестную Матерь, православно величаем. \mychapterending

\mychapter{Во вторник первой седмицы Великого Поста}
%http://www.molitvoslov.com/text572.htm 
 
\mysubsubsection{Песнь 1}

\itshape Ирмос\normalfont{}: Помощник и Покровитель бысть мне во спасение, Сей мой Бог, и прославлю Его, Бог отца моего, и вознесу Его: славно бо прославися. 

Каиново прешед убийство, произволением бых убийца совести душевней, оживив плоть и воевав на ню лукавыми моими деяньми. 

Авелеве, Иисусе, не уподобихся правде, дара Тебе приятна не принесох когда, ни деяния божественна, ни жертвы чистыя, ни жития непорочнаго. 

Яко Каин и мы, душе окаянная, всех Содетелю деяния скверная, и жертву порочную, и непотребное житие принесохом вкупе: темже и осудихомся. 

Брение Здатель живосоздав, вложил еси мне плоть и кости, и дыхание, и жизнь; но, о Творче мой, Избавителю мой и Судие, кающася приими мя. 

Извещаю Ти, Спасе, грехи, яже содеях, и души и тела моего язвы, яже внутрь убийственнии помыслы разбойнически на мя возложиша. 

Аще и согреших, Спасе, но вем, яко Человеколюбец еси, наказуеши милостивно и милосердствуеши тепле: слезяща зриши и притекаеши, яко отец, призывая блуднаго. 

\itshape Слава\normalfont{}: Пресущная Троице, во Единице покланяемая, возьми бремя от мене тяжкое греховное и, яко благоутробна, даждь ми слезы умиления. 

\itshape И ныне\normalfont{}: Богородице, Надежде и Предстательство Тебе поющих, возьми бремя от мене тяжкое греховное и, яко Владычица Чистая, кающася приими мя. 

\mysubsubsection{Песнь 2}

\itshape Ирмос\normalfont{}: Вонми, Небо, и возглаголю, и воспою Христа, от Девы плотию пришедшаго. 

Сшиваше кожныя ризы грех мне, обнаживый мя первыя боготканныя одежды. 

Обложен есмь одеянием студа, якоже листвием смоковным, во обличение моих самовластных страстей. 

Одеяхся в срамную ризу и окровавленную студно течением страстнаго и любосластнаго живота. 

Впадох в страстную пагубу и в вещественную тлю, и оттоле до ныне враг мне досаждает. 

Любовещное и любоименное житие невоздержанием, Спасе, предпочет ныне, тяжким бременем обложен есмь. 

Украсих плотский образ скверных помышлений различным обложением и осуждаюся. 

Внешним прилежно благоукрашением единем попекохся, внутреннюю презрев Богообразную скинию. 

Погребох перваго образа доброту, Спасе, страстьми, юже, яко иногда драхму, взыскав, обрящи. 

Согреших, якоже блудница, вопию Ти: един согреших Тебе, яко миро, приими, Спасе, и моя слезы. 

Очисти, якоже мытарь, вопию Ти, Спасе, очисти мя: никтоже бо сущих из Адама, якоже аз, согреших Тебе. 

\itshape Слава\normalfont{}: Единаго Тя в Триех Лицех, Бога всех пою, Отца и Сына и Духа Святаго. 

\itshape И ныне\normalfont{}: Пречистая Богородице Дево, Едина Всепетая, моли прилежно, во еже спастися нам. 

\mysubsubsection{Песнь 3}

\itshape Ирмос\normalfont{}: Утверди, Господи, на камени заповедей Твоих подвигшееся сердце мое, яко Един Свят еси и Господь. 

Источник живота стяжах Тебе, смерти Низложителя, и вопию Ти от сердца моего прежде конца: согреших, очисти и спаси мя. 

Согреших, Господи, согреших Тебе, очисти мя: несть бо иже кто согреши в человецех, егоже не превзыдох прегрешеньми. 

При Нои, Спасе, блудствовавшия подражах, онех наследствовах осуждение в потопе погружения. 

Хама онаго, душе, отцеубийца подражавши, срама не покрыла еси искренняго, вспять зря возвратившися. 

Запаления, якоже Лот, бегай, душе моя, греха: бегай Содомы и Гоморры, бегай пламене всякаго безсловеснаго желания. 

Помилуй, Господи, помилуй мя, вопию Ти, егда приидеши со ангелы Твоими воздати всем по достоянию деяний. 

\itshape Слава\normalfont{}: Троице Простая, Несозданная, Безначальное Естество, в Троице певаемая Ипостасей, спаси ны, верою покланяющияся державе Твоей. 

\itshape И ныне\normalfont{}: От Отца безлетна Сына в лето, Богородительнице, неискусомужно родила еси, странное чудо, пребывши Дева доящи. 

\mysubsubsection{Песнь 4}

\itshape Ирмос\normalfont{}: Услыша пророк пришествие Твое, Господи, и убояся, яко хощеши от Девы родитися и человеком явитися, и глаголаше: услышах слух Твой и убояхся, слава силе Твоей, Господи. 

Бди, о душе моя, изрядствуй, якоже древле великий в патриарсех, да стяжеши деяние с разумом, да будеши ум, зряй Бога, и достигнеши незаходящий мрак в видении, и будеши великий купец. 

Дванадесять патриархов великий в патриарсех детотворив, тайно утверди тебе лествицу деятельнаго, душе моя, восхождения: дети, яко основания, степени, яко восхождения, премудренно подложив. 

Исава возненавиденнаго подражала еси, душе, отдала еси прелестнику твоему первыя доброты первенство и отеческия молитвы отпала еси, и дважды поползнулася еси, окаянная, деянием и разумом: темже ныне покайся. 

Едом Исав наречеся, крайняго ради женонеистовнаго смешения: невоздержанием бо присно разжигаем и сластьми оскверняем, Едом именовася, еже глаголется разжжение души любогреховныя. 

Иова на гноищи слышавши, о душе моя, оправдавшагося, того мужеству не поревновала еси, твердаго не имела еси предложения во всех, яже веси, и имиже искусилася еси, но явилася еси нетерпелива. 

Иже первее на престоле, наг ныне на гноище гноен, многий в чадех и славный, безчаден и бездомок напрасно: палату убо гноище и бисерие струпы вменяше. 

\itshape Слава\normalfont{}: Нераздельное Существом, Неслитное Лицы богословлю Тя, Троическое Едино Божество, яко Единоцарственное и Сопрестольное, вопию Ти песнь великую, в вышних трегубо песнословимую. 

\itshape И ныне\normalfont{}: И раждаеши, и девствуеши, и пребываеши обоюду естеством Дева, Рождейся обновляет законы естества, утроба же раждает нераждающая. Бог идеже хощет, побеждается естества чин: творит бо, елика хощет. 

\mysubsubsection{Песнь 5}

\itshape Ирмос\normalfont{}: От нощи утренююща, Человеколюбче, просвети, молюся, и настави и мене на повеления Твоя, и научи мя, Спасе, творити волю Твою. 

Моисеов слышала еси ковчежец, душе, водами, волнами носим речными, яко в чертозе древле бегающий дела, горькаго совета фараонитска. 

Аще бабы слышала еси, убивающия иногда безвозрастное мужеское, душе окаянная, целомудрия деяние, ныне, яко великий Моисей, сси премудрость. 

Яко Моисей великий египтянина, ума, уязвивши, окаянная, не убила еси, душе; и како вселишися, глаголи, в пустыню страстей покаянием? 

В пустыню вселися великий Моисей; гряди убо, подражай того житие, да и в купине Богоявления, душе, в видении будеши. 

Моисеов жезл воображай, душе, ударяющий море и огустевающий глубину, во образ Креста Божественнаго: имже можеши и ты великая совершити. 

Аарон приношаше огнь Богу непорочный, нелестный; но Офни и Финеес, яко ты, душе, приношаху чуждее Богу, оскверненное житие. 

\itshape Слава\normalfont{}: Тя, Троице, славим Единаго Бога: Свят, Свят, Свят еси, Отче, Сыне и Душе, Простое Существо, Единице присно покланяемая. 

\itshape И ныне\normalfont{}: Из Тебе облечеся в мое смешение, нетленная, безмужная Мати Дево, Бог, создавый веки, и соедини Себе человеческое естество. 

\mysubsubsection{Песнь 6}

\itshape Ирмос\normalfont{}: Возопих всем сердцем моим к щедрому Богу, и услыша мя от ада преисподняго, и возведе от тли живот мой. 

Волны, Спасе, прегрешений моих, яко в мори Чермнем возвращающеся, покрыша мя внезапу, яко египтяны иногда и тристаты. 

Неразумное, душе, произволение имела еси, яко прежде Израиль: Божественныя бо манны предсудила еси безсловесно любосластное страстей объядение. 

Кладенцы, душе, предпочла еси хананейских мыслей паче жилы камене, из негоже премудрости река, яко чаша, проливает токи богословия. 

Свиная мяса и котлы и египетскую пищу, паче Небесныя, предсудила еси, душе моя, якоже древле неразумнии людие в пустыни. 

Яко удари Моисей, раб Твой, жезлом камень, образно животворивая ребра Твоя прообразоваше, из нихже вси питие жизни, Спасе, почерпаем. 

Испытай, душе, и смотряй, якоже Иисус Навин, обетования землю, какова есть, и вселися в ню благозаконием. 

\itshape Слава\normalfont{}: Троица есмь Проста, Нераздельна, раздельна Личне и Единица есмь естеством соединена, Отец глаголет, и Сын, и Божественный Дух. 

\itshape И ныне\normalfont{}: Утроба Твоя Бога нам роди, воображена по нам: Егоже, яко Создателя всех, моли, Богородице, да молитвами Твоими оправдимся. 

Господи, помилуй. (Трижды.) 

\itshape Слава, и ныне\normalfont{}: 

\mysubsubsection{Кондак, глас 6: }

Душе моя, душе моя, востани, что спиши? конец приближается, и имаши смутитися: воспряни убо, да пощадит тя Христос Бог, везде сый и вся исполняяй. 

\mysubsubsection{Песнь7}

\itshape Ирмос\normalfont{}: Согрешихом, беззаконновахом, неправдовахом пред Тобою, ниже соблюдохом, ниже сотворихом, якоже заповедал еси нам; но не предаждь нас до конца, отцев Боже. 

Кивот яко ношашеся на колеснице, Зан оный, егда превращшуся тельцу, точию коснуся, Божиим искусися гневом; но того дерзновения убежавши, душе, почитай Божественная честне. 

Слышала еси Авессалома, како на естество воста, познала еси того скверная деяния, имиже оскверни ложе Давида отца; но ты подражала еси того страстная и любосластная стремления. 

Покорила еси неработное твое достоинство телу твоему, иного бо Ахитофела обретше врага, душе, снизшла еси сего советом; но сия разсыпа Сам Христос, да ты всяко спасешися. 

Соломон чудный и благодати премудрости исполненный, сей лукавое иногда пред Богом сотворив, отступи от Него; емуже ты проклятым твоим житием, душе, уподобилася еси. 

Сластьми влеком страстей своих, оскверняшеся, увы мне, рачитель премудрости, рачитель блудных жен и странен от Бога: егоже ты подражала еси умом, о душе, сладострастьми скверными. 

Ровоаму поревновала еси, не послушавшему совета отча, купно же и злейшему рабу Иеровоаму, прежнему отступнику, душе, но бегай подражания и зови Богу: согреших, ущедри мя. 

\itshape Слава\normalfont{}: Троице Простая, Нераздельная, Единосущная и Естество Едино, Светове и Свет, и Свята Три, и Едино Свято поется Бог Троица; но воспой, прослави Живот и Животы, душе, всех Бога. 

\itshape И ныне\normalfont{}: Поем Тя, благословим Тя, покланяемся Ти, Богородительнице, яко Неразлучныя Троицы породила еси Единаго Христа Бога и Сама отверзла еси нам, сущим на земли, Небесная. 

\mysubsubsection{Песнь 8}

\itshape Ирмос\normalfont{}: Егоже воинства Небесная славят, и трепещут херувими и серафими, всяко дыхание и тварь, пойте, благословите и превозносите во вся веки. 

Ты Озии, душе, поревновавши, сего прокажение в себе стяжала еси сугубо: безместная бо мыслиши, беззаконная же дееши; остави, яже имаши, и притецы к покаянию. 

Ниневитяны, душе, слышала еси кающияся Богу, вретищем и пепелом, сих не подражала еси, но явилася еси злейшая всех, прежде закона и по законе прегрешивших. 

В рове блата слышала еси Иеремию, душе, града Сионя рыданьми вопиюща и слез ищуща: подражай сего плачевное житие и спасешися. 

Иона в Фарсис побеже, проразумев обращение ниневитянов, разуме бо, яко пророк, Божие благоутробие: темже ревноваше пророчеству не солгатися. 

Даниила в рове слышала еси, како загради уста, о душе, зверей; уведела еси, како отроцы, иже о Азарии, погасиша верою пещи пламень горящий. 

Ветхаго Завета вся приведох ти, душе, к подобию; подражай праведных боголюбивая деяния, избегни же паки лукавых грехов. 

\itshape Слава\normalfont{}: Безначальне Отче, Сыне Собезначальне, Утешителю Благий, Душе Правый, Слова Божия Родителю, Отца Безначальна Слове, Душе Живый и Зиждяй, Троице Единице, помилуй мя. 

\itshape И ныне\normalfont{}: Яко от оброщения червленицы, Пречистая, умная багряница Еммануилева внутрь во чреве Твоем плоть исткася. Темже Богородицу воистинну Тя почитаем. 

\mysubsubsection{Песнь 9}

\itshape Ирмос\normalfont{}: Безсеменнаго зачатия Рождество несказанное, Матере безмужныя нетленен Плод, Божие бо Рождение обновляет естества. Темже Тя вси роди, яко Богоневестную Матерь, православно величаем. 

Христос искушашеся, диавол искушаше, показуя камение, да хлеби будут, на гору возведе видети вся царствия мира во мгновении; убойся, о душе, ловления, трезвися, молися на всякий час Богу. 

Горлица пустыннолюбная, глас вопиющаго возгласи, Христов светильник, проповедуяй покаяние, Ирод беззаконнова со Иродиадою. Зри, душе моя, да не увязнеши в беззаконныя сети, но облобызай покаяние. 

В пустыню вселися благодати Предтеча, и Иудея вся и Самария, слышавше, течаху и исповедаху грехи своя, крещающеся усердно: ихже ты не подражала еси, душе. 

Брак убо честный и ложе нескверно, обоя бо Христос прежде благослови, плотию ядый и в Кане же на браце воду в вино совершая, и показуя первое чудо, да ты изменишися, о душе. 

Разслабленнаго стягну Христос, одр вземша, и юношу умерша воздвиже, вдовиче рождение, и сотнича отрока, и самаряныне явися, в дусе службу тебе, душе, предживописа. 

Кровоточивую исцели прикосновением края ризна Господь, прокаженныя очисти, слепыя и хромыя просветив, исправи, глухия же, и немыя, и ничащия низу исцели словом: да ты спасешися, окаянная душе. 

\itshape Слава\normalfont{}: Отца прославим, Сына превознесем, Божественному Духу верно поклонимся, Троице Нераздельней, Единице по существу, яко Свету и Светом, и Животу и Животом, Животворящему и Просвещающему концы. 

\itshape И ныне\normalfont{}: Град Твой сохраняй, Богородительнице Пречистая, в Тебе бо сей верно царствуяй, в Тебе и утверждается, и Тобою побеждаяй, побеждает всякое искушение, и пленяет ратники, и проходит послушание. 

Преподобне отче Андрее, моли Бога о нас. 

Андрее честный и отче треблаженнейший, пастырю Критский, не престай моляся о воспевающих тя, да избавимся вси гнева, и скорби, и тления, и прегрешений безмерных, чтущии твою память верно. 

Таже оба лика вкупе поют \itshape Ирмос\normalfont{}: 

Безсеменнаго зачатия Рождество несказанное, Матере безмужныя нетленен Плод, Божие бо Рождение обновляет естества. Темже Тя вси роди, яко Богоневестную Матерь, православно величаем. \mychapterending

\mychapter{В среду первой седмицы Великого Поста}
%http://www.molitvoslov.com/text573.htm 
 
\mysubsubsection{Песнь 1}

\itshape Ирмос\normalfont{}: Помощник и Покровитель бысть мне во спасение, Сей мой Бог, и прославлю Его, Бог отца моего, и вознесу Его: славно бо прославися. 

От юности, Христе, заповеди Твоя преступих, всестрастно небрегий, унынием преидох житие. Темже зову Ти, Спасе: поне на конец спаси мя. 

Повержена мя, Спасе, пред враты Твоими, поне на старость не отрини мене во ад тща, но прежде конца, яко Человеколюбец, даждь ми прегрешений оставление. 

Богатство мое, Спасе, изнурив в блуде, пуст есмь плодов благочестивых, алчен же зову: Отче щедрот, предварив, Ты мя ущедри. 

В разбойники впадый аз есмь помышленьми моими, весь от них уязвихся ныне и исполнихся ран, но, Сам ми представ, Христе Спасе, исцели. 

Священник, мя предвидев, мимо иде, и левит, видя в лютых нага, презре, но из Марии возсиявый Иисусе, Ты, представ, ущедри мя. 

Преподобная мати Марие, моли Бога о нас. 

Ты ми даждь светозарную благодать от Божественнаго свыше промышления избежати страстей омрачения и пети усердно Твоего, Марие, жития красная исправления. 

\itshape Слава\normalfont{}: Пресущная Троице, во Единице покланяемая, возьми бремя от мене тяжкое греховное и, яко благоутробна, даждь ми слезы умиления. 

\itshape И ныне\normalfont{}: Богородице, Надежде и Предстательство Тебе поющих, возьми бремя от мене тяжкое греховное и, яко Владычица Чистая, кающася приими мя. 

\mysubsubsection{Песнь 2}

\itshape Ирмос\normalfont{}: Вонми, Небо, и возглаголю, и воспою Христа, от Девы плотию пришедшаго. 

Поползохся, яко Давид, блудно и осквернихся, но омый и мене, Спасе, слезами. 

Ни слез, ниже покаяния имам, ниже умиления. Сам ми сия, Спасе, яко Бог, даруй. 

Погубих первозданную доброту и благолепие мое и ныне лежу наг и стыждуся. 

Дверь Твою не затвори мне тогда, Господи, Господи, но отверзи ми сию, кающемуся Тебе. 

Внуши воздыхания души моея и очию моею приими капли, Спасе, и спаси мя. 

Человеколюбче, хотяй всем спастися, Ты воззови мя и приими, яко благ, кающагося. 

Пресвятая Богородице, спаси нас. 

Пречистая Богородице Дево, Едина Всепетая, моли прилежно, во еже спастися нам. 

Иный. \itshape Ирмос\normalfont{}: Видите, видите, яко Аз есмь Бог, манну одождивый и воду из камене источивый древле в пустыни людем Моим, десницею единою и крепостию Моею. 

Видите, видите, яко Аз есмь Бог, внушай, душе моя, Господа вопиюща, и удалися прежняго греха, и бойся, яко неумытнаго и яко Судии и Бога. 

Кому уподобилася еси, многогрешная душе? токмо первому Каину и Ламеху оному, каменовавшая тело злодействы и убившая ум безсловесными стремленьми. 

Вся прежде закона претекши, о душе, Сифу не уподобилася еси, ни Еноса подражала еси, ни Еноха преложением, ни Ноя, но явилася еси убога праведных жизни. 

Едина отверзла еси хляби гнева Бога Твоего, душе моя, и потопила еси всю, якоже землю, плоть, и деяния, и житие, и пребыла еси вне спасительнаго ковчега. 

Преподобная мати Марие, моли Бога о нас. 

Всем усердием и любовию притекла еси Христу, первый греха путь отвращши, и в пустынях непроходимых питающися, и Того чисте совершающи Божественныя заповеди. 

\itshape Слава\normalfont{}: Безначальная, Несозданная Троице, Нераздельная Единице, кающася мя приими, согрешивша спаси, Твое есмь создание, не презри, но пощади и избави огненнаго мя осуждения. 

\itshape И ныне\normalfont{}: Пречистая Владычице, Богородительнице, Надеждо к Тебе притекающих и пристанище сущих в бури, Милостиваго и Создателя и Сына Твоего умилостиви и мне молитвами Твоими.

\mysubsubsection{Песнь 3}

\itshape Ирмос\normalfont{}: Утверди, Господи, на камени заповедей Твоих подвигшееся сердце мое, яко Един Свят еси и Господь. 

Благословения Симова не наследовала еси, душе окаянная, ни пространное одержание, якоже Иафеф, имела еси на земли оставления. 

От земли Харран изыди от греха, душе моя, гряди в землю, точащую присноживотное нетление, еже Авраам наследствова. 

Авраама слышала еси, душе моя, древле оставльша землю отечества и бывша пришельца, сего произволению подражай. 

У дуба Мамврийскаго учредив патриарх ангелы, наследствова по старости обетования ловитву. 

Исаака, окаянная душе моя, разумевши новую жертву, тайно всесожженную Господеви, подражай его произволению. 

Исмаила слышала еси, трезвися, душе моя, изгнана, яко рабынино отрождение, виждь, да не како подобно что постраждеши, ласкосердствующи. 

Преподобная мати Марие, моли Бога о нас. 

Содержим есмь бурею и треволнением согрешений, но сама мя, мати, ныне спаси и к пристанищу Божественнаго покаяния возведи. 

Преподобная мати Марие, моли Бога о нас. 

Рабское моление \itshape И ныне\normalfont{}, преподобная, принесши ко благоутробней молитвами твоими Богородице, отверзи ми Божественныя входы. 

\itshape Слава\normalfont{}: Троице Простая, Несозданная, Безначальное Естество, в Троице певаемая Ипостасей, спаси ны, верою покланяющияся державе Твоей. 

\itshape И ныне\normalfont{}: От Отца безлетна Сына в лето, Богородительнице, неискусомужно родила еси, странное чудо, пребывши Дева доящи. 

\mysubsubsection{Песнь 4}

\itshape Ирмос\normalfont{}: Услыша пророк пришествие Твое, Господи, и убояся, яко хощеши от Девы родитися и человеком явитися, и глаголаше: услышах слух Твой и убояхся, слава силе Твоей, Господи. 

Тело осквернися, дух окаляся, весь острупихся, но яко врач, Христе, обоя покаянием моим уврачуй, омый, очисти, покажи, Спасе мой, паче снега чистейша. 

Тело Твое и кровь, распинаемый о всех, положил еси, Слове: тело убо, да мя обновиши, кровь, да омыеши мя. Дух же предал еси, да мя приведеши, Христе, Твоему Родителю. 

Соделал еси спасение посреде земли, Щедре, да спасемся. Волею на древе распялся еси, Едем затворенный отверзеся, горняя и дольняя тварь, языцы вси, спасени, покланяются Тебе. 

Да будет ми купель кровь из ребр Твоих, вкупе и питие, источившее воду оставления, да обоюду очищаюся, помазуяся и пия, яко помазание и питие, Слове, животочная Твоя словеса. 

Чашу Церковь стяжа, ребра Твоя живоносная, из нихже сугубыя нам источи токи оставления и разума во образ древняго и новаго, двоих вкупе заветов, Спасе наш. 

Наг есмь чертога, наг есмь и брака, купно и вечери; светильник угасе, яко безъелейный, чертог заключися мне спящу, вечеря снедеся, аз же по руку и ногу связан, вон низвержен есмь. 

Слава: Нераздельное Существом, Неслитное Лицы богословлю Тя, Троическое Едино Божество, яко Единоцарственное и Сопрестольное, вопию Ти песнь великую, в вышних трегубо песнословимую. 

\itshape И ныне\normalfont{}: И раждаеши, и девствуеши, и пребываеши обоюду естеством Дева, Рождейся обновляет законы естества, утроба же раждает нераждающая. Бог идеже хощет, побеждается естества чин: творит бо, елика хощет. 

\mysubsubsection{Песнь 5}

\itshape Ирмос\normalfont{}: От нощи утренююща, Человеколюбче, просвети, молюся, и настави и мене на повеления Твоя, и научи мя, Спасе, творити волю Твою. 

Яко тяжкий нравом, фараону горькому бых, Владыко, Ианни и Иамври, душею и телом, и погружен умом, но помози ми. 

Калом смесихся, окаянный, умом, омый мя, Владыко, банею моих слез, молю Тя, плоти моея одежду убелив, яко снег. 

Аще испытаю моя дела, Спасе, всякаго человека превозшедша грехами себе зрю, яко разумом мудрствуяй, согреших не неведением. 

Пощади, пощади, Господи, создание Твое, согреших, ослаби ми, яко естеством чистый Сам сый Един, и ин разве Тебе никтоже есть кроме скверны. 

Мене ради Бог сый, вообразился еси в мя, показал еси чудеса, исцелив прокаженныя и разслабленнаго стягнув, кровоточивыя ток уставил еси, Спасе, прикосновением риз. 

Преподобная мати Марие, моли Бога о нас. 

Струи Иорданския прешедши, обрела еси покой безболезненный, плоти сласти избежавши, еяже и нас изми твоими молитвами, преподобная. 

Слава: Тя, Троице, славим Единаго Бога: Свят, Свят, Свят еси, Отче, Сыне и Душе, Простое Существо, Единице присно покланяемая. 

\itshape И ныне\normalfont{}: Из Тебе облечеся в мое смешение, нетленная, безмужная Мати Дево, Бог, создавый веки, и соедини Себе человеческое естество.

\mysubsubsection{Песнь 6}

\itshape Ирмос\normalfont{}: Возопих всем сердцем моим к щедрому Богу, и услыша мя от ада преисподняго, и возведе от тли живот мой. 

Востани и побори, яко Иисус Амалика, плотския страсти, и гаваониты, лестныя помыслы, присно побеждающи. 

Преиди, времене текущее естество, яко прежде ковчег, и земли оныя буди во одержании обетования, душе, Бог повелевает. 

Яко спасл еси Петра, возопивша, спаси, предварив мя, Спасе, от зверя избави, простер Твою руку, и возведи из глубины греховныя. 

Пристанище Тя вем утишное, Владыко, Владыко Христе, но от незаходимых глубин греха и отчаяния мя, предварив, избави. 

\itshape Слава\normalfont{}: Троица есмь Проста, Нераздельна, раздельна Личне, и Единица есмь естеством соединена, Отец глаголет, и Сын, и Божественный Дух. 

\itshape И ныне\normalfont{}: Утроба Твоя Бога нам роди, воображена по нам: Егоже, яко Создателя всех, моли, Богородице, да молитвами Твоими оправдимся. 

Господи, помилуй. (Трижды.) 

\itshape Слава, и ныне\normalfont{}: 

\itshape Кондак, глас 6:

\normalfont{}

Душе моя, душе моя, востани, что спиши? конец приближается, и имаши смутитися: воспряни убо, да пощадит тя Христос Бог, везде сый и вся исполняяй. 

\mysubsubsection{Песнь 7}

\itshape Ирмос\normalfont{}: Согрешихом, беззаконновахом, неправдовахом пред Тобою, ниже соблюдохом, ниже сотворихом, якоже заповедал еси нам; но не предаждь нас до конца, отцев Боже. 

Манассиева собрала еси согрешения изволением, поставльши яко мерзости страсти и умноживши, душе, негодования, но того покаянию ревнующи тепле, стяжи умиление. 

Ахаавовым поревновала еси сквернам, душе моя, увы мне, была еси плотских скверн пребывалище и сосуд срамлен страстей, но из глубины твоея воздохни и глаголи Богу грехи твоя. 

Заключися тебе небо, душе, и глад Божий постиже тя, егда Илии Фесвитянина, якоже Ахаав, не покорися словесем иногда, но Сараффии уподобився, напитай пророчу душу. 

Попали Илия иногда дващи пятьдесят Иезавелиных, егда студныя пророки погуби, во обличение Ахаавово, но бегай подражания двою, душе, и укрепляйся. 

\itshape Слава\normalfont{}: Троице Простая, Нераздельная, Единосущная, и Естество Едино, Светове и Свет, и Свята Три, и Едино Свято поется Бог Троица; но воспой, прослави Живот и Животы, душе, всех Бога. 

\itshape И ныне\normalfont{}: Поем Тя, благословим Тя, покланяемся Ти, Богородительнице, яко Нераздельныя Троицы породила еси Единаго Христа Бога и Сама отверзла еси нам, сущим на земли, Небесная. 

\mysubsubsection{Песнь 8}

\itshape Ирмос\normalfont{}: Егоже воинства Небесная славят, и трепещут херувими и серафими, всяко дыхание и тварь, пойте, благословите и превозносите во вся веки. 

Правосуде Спасе, помилуй и избави мя огня и прещения, еже имам на суде праведно претерпети; ослаби ми прежде конца добродетелию и покаянием. 

Яко разбойник, вопию Ти: помяни мя; яко Петр, плачу горце: ослаби ми, Спасе; зову, яко мытарь, слезю, яко блудница; приими мое рыдание, якоже иногда хананеино. 

Гноение, Спасе, исцели смиренныя моея души, Едине Врачу, пластырь мне наложи, и елей, и вино, дела покаяния, умиление со слезами. 

Хананею и аз подражая, помилуй мя, вопию, Сыне Давидов; касаюся края ризы, яко кровоточивая, плачу, яко Марфа и Мария над Лазарем. 

\itshape Слава\normalfont{}: Безначальне Отче, Сыне Собезначальне, Утешителю Благий, Душе Правый, Слова Божия Родителю, Отца Безначальна Слове, Душе Живый и Зиждяй, Троице Единице, помилуй мя. 

\itshape И ныне\normalfont{}: Яко от оброщения червленицы, Пречистая, умная багряница Еммануилева внутрь во чреве Твоем плоть исткася. Темже Богородицу воистинну Тя почитаем. 

\mysubsubsection{Песнь 9}

\itshape Ирмос\normalfont{}: Безсеменнаго зачатия Рождество несказанное, Матере безмужныя нетленен Плод, Божие бо Рождение обновляет естества. Темже Тя вси роди, яко Богоневестную Матерь, православно величаем. 

Недуги исцеляя, нищим благовествоваше Христос Слово, вредныя уврачева, с мытари ядяше, со грешники беседоваше, Иаировы дщере душу предумершую возврати осязанием руки. 

Мытарь спасашеся, и блудница целомудрствоваше, и фарисей, хваляся, осуждашеся. Ов убо: очисти мя; ова же: помилуй мя; сей же величашеся вопия: Боже, благодарю Тя, и прочия безумныя глаголы. 

Закхей мытарь бе, но обаче спасашеся, и фарисей Симон соблажняшеся, и блудница приимаше оставительная разрешения от Имущаго крепость оставляти грехи, юже, душе, потщися подражати. 

Блуднице, о окаянная душе моя, не поревновала еси, яже приимши мира алавастр, со слезами мазаше нозе Спасове, отре же власы, древних согрешений рукописание Раздирающаго ея. 

Грады, имже даде Христос благовестие, душе моя, уведала еси, како прокляти быша. Убойся указания, да не будеши якоже оны, ихже содомляном Владыка уподобив, даже до ада осуди. 

Да не горшая, о душе моя, явишися отчаянием, хананеи веру слышавшая, еяже дщи словом Божиим исцелися; Сыне Давидов, спаси и мене, воззови из глубины сердца, якоже она Христу. 

Слава: Отца прославим, Сына превознесем, Божественному Духу верно поклонимся, Троице Нераздельней, Единице по Существу, яко Свету и Светом, и Животу и Животом, Животворящему и Просвещающему концы. 

\itshape И ныне\normalfont{}: Град Твой сохраняй, Богородительнице Пречистая, в Тебе бо сей верно царствуяй, в Тебе и утверждается, и Тобою побеждаяй, побеждает всякое искушение, и пленяет ратники, и проходит послушание. 

\bfseries 

Преподобне отче Андрее, моли Бога о нас.\normalfont{}

Андрее честный и отче треблаженнейший, пастырю Критский, не престай моляся о воспевающих тя, да избавимся вси гнева, и скорби, и тления, и прегрешений безмерных, чтущии твою память верно. 

\itshape 

Таже оба лика вкупе поют Ирмос\normalfont{}:

Безсеменнаго зачатия Рождество несказанное, Матере безмужныя нетленен Плод, Божие бо Рождение обновляет естества. Темже Тя вси роди, яко Богоневестную Матерь, православно величаем.

 

\mychapterending

\mychapter{В четверг первой седмицы Великого Поста}
%http://www.molitvoslov.com/text574.htm 
 


\mysubsubsection{Песнь 1}

\itshape Ирмос\normalfont{}: Помощник и Покровитель бысть мне во спасение, Сей мой Бог, и прославлю Его, Бог Отца моего, и вознесу Его: славно бо прославися. 

Агнче Божий, вземляй грехи всех, возьми бремя от мене тяжкое греховное, и, яко благоутробен, даждь ми слезы умиления. 

Тебе припадаю, Иисусе, согреших Ти, очисти мя, возьми бремя от мене тяжкое греховное и, яко благоутробен, даждь ми слезы умиления. 

Не вниди со мною в суд, нося моя деяния, словеса изыскуя и исправляя стремления. Но в щедротах Твоих презирая моя лютая, спаси мя, Всесильне. 

Покаяния время, прихожду Ти, Создателю моему: возьми бремя от мене тяжкое греховное и, яко благоутробен, даждь ми слезы умиления. 

Богатство душевное иждив грехом, пуст есмь добродетелей благочестивых, гладствуя же зову: милости подателю Господи, спаси мя. 

\bfseries 

Преподобная Мати Марие, моли Бога о нас.

\normalfont{}

Приклоньшися Христовым Божественным законом, к сему приступила еси, сладостей неудержимая стремления оставивши, и всякую добродетель всеблагоговейно, яко едину, исправила еси. 

\itshape Слава\normalfont{}: Пресущественная Троице, во Единице покланяемая, возьми бремя от мене тяжкое греховное и, яко благоутробна, даждь ми слезы умиления. 

\itshape И ныне\normalfont{}: Богородице, Надежде и Предстательство Тебе поющих, возьми бремя от мене тяжкое греховное и, яко Владычица Чистая, кающася приими мя. 


\mysubsubsection{Песнь 2}

\itshape Ирмос\normalfont{}: Видите, видите, яко Аз есмь Бог, манну одождивый и воду из камене источивый древле в пустыни людем Моим, десницею единою и крепостию Моею. 

Мужа убих, глаголет, в язву мне и юношу в струп, Ламех, рыдая, вопияше; ты же не трепещеши, о душе моя, окалявши плоть и ум осквернивши. 

Столп умудрила еси создати, о душе, и утверждение водрузити твоими похотьми, аще не бы Зиждитель удержал советы твоя и низвергл на землю ухищрения твоя. 

О како поревновах Ламеху, первому убийце, душу, яко мужа, ум, яко юношу, яко брата же моего, тело убив, яко Каин убийца, любосластными стремленьми. 

Одожди Господь от Господа огнь иногда на беззаконие гневающее, сожег содомляны; ты же огнь вжегла еси геенский, в немже имаши, о душе, сожещися. 

Уязвихся, уранихся, се стрелы вражия, уязвившия мою душу и тело; се струпи, гноения, омрачения вопиют, раны самовольных моих страстей. 

\bfseries 

Преподобная Мати Марие, моли Бога о нас.

\normalfont{}

Простерла еси руце твои к щедрому Богу, Марие, в бездне зол погружаемая; и якоже Петру человеколюбно руку Божественную простре твое обращение всячески Иский. 

\itshape Слава\normalfont{}: Безначальная, Несозданная Троице, Нераздельная Единице, кающася мя приими, согрешивша спаси, Твое есмь создание, не презри, но пощади и избави мя огненнаго осуждения. 

\itshape И ныне\normalfont{}: Пречистая Владычице, Богородительнице, Надеждо к Тебе притекающих и пристанище сущих в бури, Милостиваго и Создателя и Сына Твоего умилостиви и мне молитвами Твоими.


\mysubsubsection{Песнь 3}

\itshape Ирмос\normalfont{}: Утверди, Господи, на камени заповедей Твоих подвигшееся сердце мое, яко Един Свят еси и Господь. 

Агаре древле, душе, египтяныне уподобилася еси, поработившися произволением и рождши новаго Исмаила, презорство. 

Иаковлю лествицу разумела еси, душе моя, являемую от земли к Небесем: почто не имела еси восхода тверда, благочестия. 

Священника Божия и царя уединена, Христово подобие в мире жития, в человецех подражай. 

Обратися, постени, душе окаянная, прежде даже не приимет конец жития торжество, прежде даже дверь не заключит чертога Господь. 

Не буди столп сланый, душе, возвратившися вспять, образ да устрашит тя содомский, горе в Сигор спасайся. 

Моления, Владыко, Тебе поющих не отвержи, но ущедри, Человеколюбче, и подаждь верою просящим оставление. 

\itshape Слава\normalfont{}: Троица Простая, Несозданная, Безначальное Естество, в Троице певаемая Ипостасей, спаси ны, верою покланяющияся державе Твоей. 

\itshape И ныне\normalfont{}: От Отца безлетна Сына в лето, Богородительнице, неискусомужно родила еси, странное чудо, пребывши Дева доящи.


\mysubsubsection{Песнь 4}

\itshape Ирмос\normalfont{}: Услыша пророк пришествие Твое, Господи, и убояся, яко хощеши от Девы родитися и человеком явитися, и глаголаше: услышах слух Твой и убояхся, слава силе Твоей, Господи. 

Время живота моего мало и исполнено болезней и лукавства, но в покаянии мя приими и в разум призови, да не буду стяжание ни брашно чуждему, Спасе, Сам мя ущедри. 

Царским достоинством, венцем и багряницею одеян, многоименный человек и праведный, богатством кипя и стады, внезапу богатства, славы царства, обнищав, лишися. 

Аще праведен бяше он и непорочен паче всех, и не убеже ловления льстиваго и сети; ты же, грехолюбива сущи, окаянная душе, что сотвориши, аще чесому от недоведомых случится наити тебе? 

Высокоглаголив ныне есмь, жесток же и сердцем, вотще и всуе, да не с фарисеем осудиши мя. Паче же мытарево смирение подаждь ми, Едине Щедре, Правосуде, и сему мя сочисли. 

Согреших, досадив сосуду плоти моея, вем, Щедре, но в покаянии мя приими и в разум призови, да не буду стяжание ни брашно чуждему, Спасе, Сам мя ущедри. 

Самоистукан бых страстьми, душу мою вредя, Щедре, но в покаянии мя приими и в разум призови, да не буду стяжание ни брашно чуждему, Спасе, Сам мя ущедри. 

Не послушах гласа Твоего, преслушах Писание Твое, Законоположника, но в покаянии мя приими и в разум призови, да не буду стяжание ни брашно чуждему, Спасе, Сам мя ущедри. 

\bfseries 

Преподобная Мати Марие, моли Бога о нас.

\normalfont{}

Великих безместий во глубину низведшися, неодержима была еси; но востекла еси помыслом лучшим к крайней деяньми яве добродетели преславно, ангельское естество, Марие, удививши. 

\itshape Слава\normalfont{}: Нераздельное Существом, Неслитное Лицы богословлю Тя, Троическое Едино Божество, яко Единоцарственное и Сопрестольное, вопию Ти песнь великую, в вышних трегубо песнословимую. 

\itshape И ныне\normalfont{}: И раждаеши, и девствуеши, и пребываеши обоюду естеством Дева, Рождейся обновляет законы естества, утроба же раждает нераждающая. Бог идеже хощет, побеждается естества чин: творит бо, елика хощет.


\mysubsubsection{Песнь 5}

\itshape Ирмос\normalfont{}: От нощи утренююща, Человеколюбче, просвети, молюся, и настави и мене на повеления Твоя, и научи мя, Спасе, творити волю Твою. 

Низу сничащую подражай, о душе, прииди, припади к ногама Иисусовыма, да тя исправит, и да ходиши право стези Господни. 

Аще и кладязь еси глубокий, Владыко, источи ми воду из пречистых Твоих жил, да, яко самаряныня, не ктому, пияй, жажду: жизни бо струи источаеши. 

Силоам да будут ми слезы моя, Владыко Господи, да умыю и аз зеницы сердца, и вижду Тя, умна Света превечна. 

\bfseries 

Преподобная Мати Марие, моли Бога о нас.

\normalfont{}

Несравненным желанием, всебогатая, древу возжелевши поклонитися животному, сподобилася еси желания, сподоби убо и мене улучити вышния славы. 

\itshape Слава\normalfont{}: Тя, Троице, славим Единаго Бога: Свят, Свят, Свят еси, Отче, Сыне и Душе, Простое Существо, Единице присно покланяемая. 

\itshape И ныне\normalfont{}: Из Тебе облечеся в мое смешение, нетленная, безмужная Мати Дево, Бог, создавый веки, и соедини Себе человеческое естество.


\mysubsubsection{Песнь 6}

\itshape Ирмос\normalfont{}: Возопих всем сердцем моим к щедрому Богу, и услыша мя от ада преисподняго, и возведе от тли живот мой. 

Аз есмь, Спасе, юже погубил еси древле царскую драхму; но вжег светильник, Предтечу Твоего, Слове, взыщи и обрящи Твой образ. 

Востани и побори, яко Иисус Амалика, плотския страсти, и гаваониты, лестныя помыслы, присно побеждающи. 

\bfseries 

Преподобная Мати Марие, моли Бога о нас.

\normalfont{}

Да страстей пламень угасиши, слез капли источала еси присно, Марие, душею распаляема, ихже благодать подаждь и мне, твоему рабу. 

\bfseries 

Преподобная Мати Марие, моли Бога о нас.

\normalfont{}

Безстрастие Небесное стяжала еси крайним на земли житием, мати. Темже тебе поющим страстей избавитися молитвами твоими молися. 

\itshape Слава\normalfont{}: Троица есмь Проста, Нераздельна, раздельна Личне и Единица есмь естеством соединена, Отец глаголет, и Сын, и Божественный Дух. 

\itshape И ныне\normalfont{}: Утроба Твоя Бога нам роди, воображена по нам: Егоже, яко Создателя всех, моли, Богородице, да Твоими молитвами оправдимся. 

Господи, помилуй. (Трижды.) 

\itshape Слава, и ныне\normalfont{}:


\mysubsubsection{Кондак, глас 6:}

Душе моя, душе моя, востани, что спиши? конец приближается, и имаши смутитися: воспряни убо, да пощадит тя Христос Бог, везде сый и вся исполняяй.


\mysubsubsection{Песнь 7}

\itshape Ирмос\normalfont{}: Согрешихом, беззаконновахом, неправдовахом пред Тобою, ниже соблюдохом, ниже сотворихом, якоже заповедал еси нам; но не предаждь нас до конца, отцев Боже. 

Исчезоша дние мои, яко соние востающаго; темже, яко Езекия, слезю на ложи моем, приложитися мне летом живота. Но кий Исаия предстанет тебе, душе, аще не всех Бог? 

Припадаю Ти и приношу Тебе, якоже слезы, глаголы моя: согреших, яко не согреши блудница, и беззаконновах, яко иный никтоже на земли. Но ущедри, Владыко, творение Твое и воззови мя. 

Погребох образ Твой и растлих заповедь Твою, вся помрачися доброта, и страстьми угасися, Спасе, свеща. Но ущедрив, воздаждь ми, якоже поет Давид, радование. 

Обратися, покайся, открый сокровенная, глаголи Богу, вся ведущему: Ты веси моя тайная, Едине Спасе. Но Сам мя помилуй, якоже поет Давид, по милости Твоей. 

\bfseries 

Преподобная Мати Марие, моли Бога о нас.

\normalfont{}

Возопивши к Пречистей Богоматери, первее отринула еси неистовство страстей, нужно стужающих, и посрамила еси врага запеншаго. Но даждь ныне помощь от скорби и мне, рабу твоему. 

\bfseries 

Преподобная Мати Марие, моли Бога о нас.

\normalfont{}

Егоже возлюбила еси, Егоже возжелела еси, Егоже ради плоть изнурила еси, преподобная, моли ныне Христа о рабех: яко да милостив быв всем нам, мирное состояние дарует почитающим Его. 

\itshape Слава\normalfont{}: Троице Простая, Нераздельная, Единосущная и Естество Едино, Светове и Свет, и Свята Три, и Едино Свято поется Бог Троица; но воспой, прослави Живот и Животы, душе, всех Бога. 

\itshape И ныне\normalfont{}: Поем Тя, благословим Тя, покланяемся Ти, Богородительнице, яко Неразлучныя Троицы породила еси Единаго Христа Бога и Сама отверзла еси нам, сущим на земли, Небесная.


\mysubsubsection{Песнь 8}

\itshape Ирмос\normalfont{}: Егоже воинства Небесная славят, и трепещут херувими и серафими, всяко дыхание и тварь, пойте, благословите и превозносите во вся веки. 

Слезную, Спасе, сткляницу яко миро истощавая на главу, зову Ти, якоже блудница, милости ищущая, мольбу приношу и оставление прошу прияти. 

Аще и никтоже, якоже аз, согреши Тебе, но обаче приими и мене, благоутробне Спасе, страхом кающася и любовию зовуща: согреших Тебе Единому, помилуй мя, Милостиве. 

Пощади, Спасе, Твое создание и взыщи, яко Пастырь, погибшее, предвари заблуждшаго, восхити от волка, сотвори мя овча на пастве Твоих овец. 

Егда, Судие, сядеши, яко благоутробен, и покажеши страшную славу Твою, Спасе, о каковый страх тогда, пещи горящей, всем боящимся нестерпимаго судища Твоего. 

\bfseries 

Преподобная Мати Марие, моли Бога о нас.

\normalfont{}

Света незаходимаго Мати тя просветивши, от омрачения страстей разреши. Темже вшедши в духовную благодать, просвети, Марие, тя верно восхваляющия. 

\bfseries 

Преподобная Мати Марие, моли Бога о нас.

\normalfont{}

Чудо ново видев, ужасашеся божественный в тебе воистинну, мати, Зосима: ангела бо зряше во плоти и ужасом весь исполняшеся, Христа поя во веки. 

\itshape Слава\normalfont{}: Безначальне Отче, Сыне Собезначальне, Утешителю Благий, Душе Правый, Слова Божия Родителю, Отца Безначальна Слове, Душе Живый и Зиждяй, Троице Единице, помилуй мя. 

\itshape И ныне\normalfont{}: Яко от оброщения червленицы, Пречистая, умная багряница Еммануилева внутрь во чреве Твоем плоть исткася. Темже Богородицу воистинну Тя почитаем.


\mysubsubsection{Песнь 9}

\itshape Ирмос\normalfont{}: Безсеменнаго зачатия Рождество несказанное, Матере безмужныя нетленен Плод, Божие бо Рождение обновляет естества. Темже Тя вси роди, яко Богоневестную Матерь, православно величаем. 

Умилосердися, спаси мя, Сыне Давидов, помилуй, беснующияся словом исцеливый, глас же благоутробный, яко разбойнику, мне рцы: аминь, глаголю тебе, со Мною будеши в раи, егда прииду во славе Моей. 

Разбойник оглаголоваше Тя, разбойник богословяше Тя: оба бо на кресте свисяста. Но, о Благоутробне, яко верному разбойнику Твоему, познавшему Тя Бога, и мне отверзи дверь славнаго Царствия Твоего. 

Тварь содрогашеся, распинаема Тя видящи, горы и камения страхом распадахуся, и земля сотрясашеся, и ад обнажашеся, и соомрачашеся свет во дни, зря Тебе, Иисусе, пригвождена ко Кресту. 

Достойных покаяния плодов не истяжи от мене, ибо крепость моя во мне оскуде; сердце мне даруй присно сокрушенное, нищету же духовную: да сия Тебе принесу яко приятную жертву, Едине Спасе. 

Судие мой и Ведче мой, хотяй паки приити со ангелы, судити миру всему, милостивным Твоим оком тогда видев мя, пощади и ущедри мя, Иисусе, паче всякаго естества человеча согрешивша. 

\bfseries 

Преподобная Мати Марие, моли Бога о нас.

\normalfont{}

Удивила еси всех странным житием твоим, ангелов чины и человеков соборы, невещественно поживши и естество прешедши: имже, яко невещественныма ногама вшедши, Марие, Иордан прешла еси. 

\bfseries 

Преподобная Мати Марие, моли Бога о нас.

\normalfont{}

Умилостиви Создателя о хвалящих тя, преподобная мати, избавитися озлоблений и скорбей, окрест нападающих: да избавившеся от напастей, возвеличим непрестанно прославльшаго тя Господа. 

\bfseries 

Преподобне отче Андрее, моли Бога о нас.

\normalfont{}

Андрее честный и отче треблаженнейший, пастырю Критский, не престай моляся о воспевающих тя: да избавимся вси гнева, и скорби, и тления, и прегрешений безмерных, чтущии твою память верно. 

\itshape Слава\normalfont{}: Отца прославим, Сына превознесем, Божественному Духу верно поклонимся, Троице Нераздельней, Единице по Существу, яко Свету и Светом, и Животу и Животом, Животворящему и Просвещающему концы. 

\itshape И ныне\normalfont{}: Град Твой сохраняй, Богородительнице Пречистая, в Тебе бо сей верно царствуяй, в Тебе и утверждается, и Тобою побеждаяй, побеждает всякое искушение, и пленяет ратники, и проходит послушание. 

Таже оба лика вкупе поют \itshape Ирмос\normalfont{}: 

Безсеменнаго зачатия Рождество несказанное, Матере безмужныя нетленен Плод, Божие бо Рождение обновляет естества. Темже Тя вси роди, яко Богоневестную Матерь, православно величаем.



\mychapterending

\mychapter{В четверг пятой седмицы Великого Поста}
%http://www.molitvoslov.com/text575.htm 
 
\mysubsubsection{Песнь 1}


\itshape Ирмос\normalfont{}: Помощник и Покровитель бысть мне во спасение, Сей мой Бог, и прославлю Его, Бог Отца моего, и вознесу Его: славно бо прославися.


Откуду начну плакати окаяннаго моего жития деяний? кое ли положу начало, Христе, нынешнему рыданию? но яко благоутробен, даждь ми прегрешений оставление.


Гряди, окаянная душе, с плотию твоею, Зиждителю всех исповеждься, и останися прочее преждняго безсловесия, и принеси Богу в покаянии слезы.


Первозданнаго Адама преступлению поревновав, познах себе обнажена от Бога и присносущнаго Царствия и сладости, грех ради моих.


Увы мне, окаянная душе, что уподобилася еси первей Еве? видела бо еси зле, и уязвилася еси горце, и коснулася еси древа, и вкусила еси дерзостно безсловесныя снеди.


Вместо Евы чувственныя мысленная ми бысть Ева, во плоти страстный помысл, показуяй сладкая и вкушаяй присно горькаго напоения.


Достойно из Едема изгнан бысть, яко не сохранив едину Твою, Спасе, заповедь Адам: аз же что постражду, отметая всегда животная Твоя словеса?


Каиново прешед убийство, произволением бых убийца совести душевней, оживив плоть и воевав на ню лукавыми моими деяньми.


Авелеве, Иисусе, не уподобихся правде, дара Тебе приятна не принесох когда, ни деяния божественна, ни жертвы чистыя, ни жития непорочнаго.


Яко Каин и мы, душе окаянная, всех Содетелю деяния скверная, и жертву порочную, и непотребное житие принесохом вкупе: темже и осудихомся.


Брение Здатель живосоздав, вложил еси мне плоть, и кости, и дыхание, и жизнь; но, о Творче мой, Избавителю мой и Судие, кающася приими мя.


Извещаю Ти, Спасе, грехи, яже содеях, и души и тела моего язвы, яже внутрь убийственнии помыслы разбойнически на мя возложиша.


Аще и согреших, Спасе, но вем, яко Человеколюбец еси, наказуеши милостивно и милосердствуеши тепле: слезяща зриши и притекаеши, яко отец, призывая блуднаго.


Повержена мя, Спасе, пред враты Твоими, поне на старость не отрини мене во ад тща, но прежде конца, яко Человеколюбец, даждь ми прегрешений оставление.


В разбойники впадый аз есмь помышленьми моими, весь от них уязвихся ныне и исполнихся ран, но, Сам ми представ, Христе Спасе, исцели.


Священник, мя предвидев, мимо иде, и левит, видя в лютых нага, презре, но, из Марии возсиявый, Иисусе, Ты, представ, ущедри мя.


Агнче Божий, вземляй грехи всех, возьми бремя от мене тяжкое греховное и, яко благоутробен, даждь ми слезы умиления.


Покаяния время, прихожду Ти, Создателю моему: возьми бремя от мене тяжкое греховное и, яко благоутробен, даждь ми слезы умиления.


Не возгнушайся мене, Спасе, не отрини от Твоего лица, возьми бремя от мене тяжкое греховное и, яко благоутробен, даждь мне грехопадений оставление.


Вольная, Спасе, и невольная прегрешения моя, явленная и сокровенная и ведомая и неведомая, вся простив, яко Бог, очисти и спаси мя.


От юности, Христе, заповеди Твоя преступих, всестрастно небрегий, унынием преидох житие. Темже зову Ти, Спасе: поне на конец спаси мя.


Богатство мое, Спасе, изнурив в блуде, пуст есмь плодов благочестивых, алчен же зову: Отче щедрот, предварив, Ты мя ущедри.


Тебе припадаю, Иисусе, согреших Ти, очисти мя, возьми бремя от мене тяжкое греховное и, яко благоутробен, даждь ми слезы умиления.


Не вниди со мною в суд, нося моя деяния, словеса изыскуя и исправляя стремления. Но в щедротах Твоих презирая моя лютая, спаси мя, Всесильне.


\mysubsubsection{Иный канон преподобныя матере нашея Марии Египетския, глас 6:}


Преподобная мати Марие, моли Бога о нас.


Ты ми даждь светозарную благодать от Божественнаго свыше промышления избежати страстей омрачения и пети усердно твоего, Марие, жития красная исправления.


\mysubsubsection{Преподобная мати Марие, моли Бога о нас.}


Приклоньшися Христовым Божественным законом, к сему приступила еси, сладостей неудержимая стремления оставивши, и всякую добродетель всеблагоговейно, яко едину, исправила еси.


Преподобне отче Андрее, моли Бога о нас.


Молитвами твоими нас, Андрее, избави страстей безчестных и Царствия ныне Христова общники верою и любовию воспевающия тя, славне, покажи, молимся.


\itshape Слава\normalfont{}: Пресущественная Троице, во Единице покланяемая, возьми бремя от мене тяжкое греховное и, яко благоутробна, даждь ми слезы умиления.


\itshape И ныне\normalfont{}: Богородице, Надежде и Предстательство Тебе поющих, возьми бремя от мене тяжкое греховное и, яко Владычица Чистая, кающася приими мя. 


\mysubsubsection{Песнь 2}


\itshape Ирмос\normalfont{}: Вонми, Небо, и возглаголю, и воспою Христа, от Девы плотию пришедшаго.


Вонми, Небо, и возглаголю, земле, внушай глас, кающийся к Богу и воспевающий Его.


Вонми ми, Боже, Спасе мой, милостивным Твоим оком и приими мое теплое исповедание.


Согреших паче всех человек, един согреших Тебе; но ущедри, яко Бог, Спасе, творение Твое.


Буря мя злых обдержит, благоутробне Господи; но яко Петру и мне руку простри.


Слезы блудницы, Щедре, и аз предлагаю, очисти мя, Спасе, благоутробием Твоим.


Омрачих душевную красоту страстей сластьми и всячески весь ум персть сотворих.


Раздрах ныне одежду мою первую, юже ми изтка Зиждитель из начала, и оттуду лежу наг.


Облекохся в раздранную ризу, юже изтка ми змий советом, и стыждуся.


Воззрех на садовную красоту и прельстихся умом: и оттуду лежу наг и срамляюся.


Делаша на хребте моем вси начальницы страстей, продолжающе на мя беззаконие их.


Погубих первозданную доброту и благолепие мое и ныне лежу наг и стыждуся.


Сшиваше кожныя ризы грех мне, обнаживый мя первыя боготканныя одежды.


Обложен есмь одеянием студа, якоже листвием смоковным, во обличение моих самовластных страстей.


Одеяхся в срамную ризу и окровавленную студно течением страстнаго и любосластнаго живота.


Оскверних плоти моея ризу и окалях еже по образу, Спасе, и по подобию.


Впадох в страстную пагубу и в вещественную тлю, и оттоле до ныне враг мне досаждает.


Любовещное и любоименное житие, невоздержанием, Спасе, предпочет ныне, тяжким бременем обложен есмь.


Украсих плотский образ скверных помышлений различным обложением и осуждаюся.


Внешним прилежно благоукрашением единем попекохся, внутреннюю презрев Богообразную скинию.


Вообразив моих страстей безобразие, любосластными стремленьми погубих ума красоту.


Погребох перваго образа доброту, Спасе, страстьми, юже, яко иногда драхму, взыскав, обрящи.


Согреших, якоже блудница, вопию Ти: един согреших Тебе; яко миро, приими, Спасе, и моя слезы.


Поползохся, яко Давид, блудно и осквернихся, но омый и мене, Спасе, слезами.


Очисти, якоже мытарь, вопию Ти, Спасе, очисти мя: никтоже бо сущих из Адама, якоже аз, согреших Тебе.


Ни слез, ниже покаяния имам, ниже умиления. Сам ми сия, Спасе, яко Бог, даруй.


Дверь Твою не затвори мне тогда, Господи, Господи, но отверзи ми сию, кающемуся Тебе.


Человеколюбче, хотяй всем спастися, Ты воззови мя и приими, яко благ, кающагося.


Внуши воздыхания души моея и очию моею приими капли, Спасе, и спаси мя.


\bfseries Пресвятая Богородице, спаси нас.


\normalfont{}


Пречистая Богородице Дево, Едина Всепетая, моли прилежно, во еже спастися нам.


\itshape Иный. Ирмос\normalfont{}: Видите, видите, яко Аз есмь Бог, манну одождивый и воду из камене источивый древле в пустыни людем Моим, десницею единою и крепостию Моею.


Видите, видите, яко Аз есмь Бог, внушай, душе моя, Господа вопиюща, и удалися прежняго греха, и бойся, яко неумытнаго и яко Судии и Бога.


Кому уподобилася еси, многогрешная душе? токмо первому Каину и Ламеху оному, каменовавшая тело злодействы и убившая ум безсловесными стремленьми.


Вся прежде закона претекши, о душе, Сифу не уподобилася еси, ни Еноса подражала еси, ни Еноха преложением, ни Ноя, но явилася еси убога праведных жизни.


Едина отверзла еси хляби гнева Бога Твоего, душе моя, и потопила еси всю, якоже землю, плоть, и деяния, и житие, и пребыла еси вне спасительнаго ковчега.


Мужа убих, глаголет, в язву мне и юношу в струп, Ламех рыдая вопияше; ты же не трепещеши, о душе моя, окалявши плоть и ум осквернивши.


О како поревновах Ламеху, первому убийце, душу, яко мужа, ум, яко юношу, яко брата же моего, тело убив, яко Каин убийца, любосластными стремленьми.


Столп умудрила еси создати, о душе, и утверждение водрузити твоими похотьми, аще не бы Зиждитель удержал советы твоя и низвергл на землю ухищрения твоя.


Уязвихся, уранихся, се стрелы вражия, уязвившия мою душу и тело; се струпи, гноения, омрачения вопиют, раны самовольных моих страстей.


Одожди Господь от Господа огнь иногда на беззаконие гневающее, сожег содомляны; ты же огнь вжегла еси геенский, в немже имаши, о душе, сожещися.


Разумейте и видите, яко Аз есмь Бог, испытаяй сердца и умучаяй мысли, обличаяй деяния, и попаляяй грехи, и судяй сиру, и смирену, и нищу.


\bfseries Преподобная мати Марие, моли Бога о нас.


\normalfont{}Простерла еси руце твои к щедрому Богу, Марие, в бездне зол погружаемая, и якоже Петру человеколюбно руку Божественную простре твое обращение всячески Иский.


\mysubsubsection{Преподобная мати Марие, моли Бога о нас.}


Всем усердием и любовию притекла еси Христу, первый греха путь отвращши, и в пустынях непроходимых питающися, и Того чисте совершающи Божественныя заповеди.


\mysubsubsection{Преподобне отче Андрее, моли Бога о нас.}


Видим, видим человеколюбие, о душе, Бога и Владыки; сего ради прежде конца тому со слезами припадем вопиюще: Андрея молитвами, Спасе, помилуй нас.


\itshape Слава\normalfont{}: Безначальная, Несозданная Троице, Нераздельная Единице, кающася мя приими, согрешивша спаси, Твое есмь создание, не презри, но пощади и избави мя огненнаго осуждения.


\itshape И ныне\normalfont{}: Пречистая Владычице, Богородительнице, Надеждо к Тебе притекающих и пристанище сущих в бури, Милостиваго и Создателя и Сына Твоего умилостиви и мне молитвами Твоими. 

\mysubsubsection{Песнь 3}


\itshape Ирмос\normalfont{}: На недвижимом, Христе, камени заповедей Твоих утверди мое помышление.


Огнь от Господа иногда Господь одождив, землю содомскую прежде попали.


На горе спасайся, душе, якоже Лот оный, и в Сигор угонзай.


Бегай запаления, о душе, бегай содомскаго горения, бегай тления Божественнаго пламене.


Исповедаюся Тебе, Спасе, согреших, согреших Ти, но ослаби, остави ми, яко благоутробен.


Согреших Тебе един аз, согреших паче всех, Христе Спасе, да не презриши мене.


Ты еси Пастырь добрый, взыщи мене, агнца, и заблуждшаго да не презриши мене.

Ты еси сладкий Иисусе, Ты еси Создателю мой, в Тебе, Спасе, оправдаюся.


\mysubsubsection{Пресвятая Троице, Боже наш, слава Тебе.}


О Троице Единице Боже, спаси нас от прелести, и искушений, и обстояний.

\mysubsubsection{Пресвятая Богородице, спаси нас.}


Радуйся, Богоприятная утробо, радуйся, престоле Господень, радуйся, Мати Жизни нашея.


\itshape Иный. Ирмос\normalfont{}: Утверди, Господи, на камени заповедей Твоих подвигшееся сердце мое, яко Един Свят еси и Господь.


Источник живота стяжах Тебе, смерти Низложителя, и вопию Ти от сердца моего прежде конца: согреших, очисти, спаси мя.


При Нои, Спасе, блудствовавшия подражах, онех наследствовав осуждение в потопе погружения.


Согреших, Господи, согреших Тебе, очисти мя: несть бо иже кто согреши в человецех, егоже не превзыдох прегрешеньми.


Хама онаго, душе, отцеубийца подражавши, срама не покрыла еси искренняго, вспять зря возвратившися.


Благословения Симова не наследовала еси, душе окаянная, ни пространное одержание, якоже Иафеф, имела еси на земли оставления.


От земли Харран изыди от греха, душе моя, гряди в землю, точащую присноживотное нетление, еже Авраам наследствова.


Авраама слышала еси, душе моя, древле оставльша землю отечества и бывша пришельца, сего произволению подражай.


У дуба Мамврийскаго учредив патриарх ангелы, наследствова по старости обетования ловитву.


Исаака, окаянная душе моя, разумевши новую жертву, тайно всесожженную Господеви, подражай его произволению.


Исмаила слышала еси, трезвися, душе моя, изгнана, яко рабынино отрождение, виждь, да не како подобно что постраждеши, ласкосердствующи.


Агаре древле, душе, египтяныне уподобилася еси, поработившися произволением и рождши новаго Исмаила, презорство.


Иаковлю лествицу разумела еси, душе моя, являемую от земли к Небесем: почто не имела еси восхода тверда, благочестия.


Священника Божия и царя уединена, Христово подобие в мире жития, в человецех подражай.


Не буди столп сланый, душе, возвратившися вспять, образ да устрашит тя содомский, горе в Сигор спасайся.


Запаления, якоже Лот, бегай, душе моя, греха, бегай Содомы и Гоморры, бегай пламене всякаго безсловеснаго желания.


Помилуй, Господи, помилуй мя, вопию Ти, егда приидеши со ангелы Твоими воздати всем по достоянию деяний.


Моления, Владыко, Тебе поющих не отвержи, но ущедри, Человеколюбче, и подаждь верою просящим оставление.


\mysubsubsection{Преподобная мати Марие, моли Бога о нас.}


Содержим есмь бурею и треволнением согрешений, но сама мя, мати, ныне спаси и к пристанищу Божественнаго покаяния возведи.


\mysubsubsection{Преподобная мати Марие, моли Бога о нас.}


Рабское моление и ныне, преподобная, принесши ко благоутробней молитвами твоими Богородице, отверзи ми Божественныя входы.


\mysubsubsection{Преподобне отче Андрее, моли Бога о нас.}


Твоими молитвами даруй и мне оставление долгов, о Андрее, Критский председателю, покаяния бо ты таинник преизрядный.


\itshape Слава\normalfont{}: Троице Простая, Несозданная, Безначальное Естество, в Троице певаемая Ипостасей, спаси ны, верою покланяющияся державе Твоей.


\itshape И ныне\normalfont{}: От Отца безлетна Сына в лето, Богородительнице, неискусомужно родила еси, странное чудо, пребывши Дева доящи.


\itshape Ирмос\normalfont{}: Утверди, Господи, на камени заповедей Твоих подвигшееся сердце мое, яко Един Свят еси и Господь. 

\mysubsubsection{Седален, глас 8:}


Светила богозрачная, Спасовы апостоли, просветите нас во тьме жития, яко да во дни ныне благообразно ходим, светом воздержания нощных страстей отбегающе, и светлыя страсти Христовы узрим, радующеся. 

\mysubsubsection{Слава, другий седален, глас тойже:}


Апостольская двоенадесятице Богоизбранная, мольбу Христу ныне принеси, постное поприще всем прейти, совершающим во умилении молитвы, творящим усердно добродетели, яко да сице предварим видети Христа Бога славное Воскресение, славу и хвалу приносяще. 


\mysubsubsection{И ныне, Богородичен:}



Непостижимаго Бога, Сына и Слово, несказанно паче ума из Тебе рождшееся, моли, Богородице, со апостолы, мир вселенней чистый подати, и согрешений дати нам прежде конца прощение, и Царствия Небеснаго крайния ради благости сподобити рабы Твоя. 


\mysubsubsection{Таже трипеснец, глас 8:}


\mysubsubsection{Песнь 4}


\itshape Ирмос\normalfont{}: Услышах, Господи, смотрения Твоего таинство, разумех дела Твоя и прославих Твое Божество.


\mysubsubsection{Святии апостоли, молите Бога о нас.}


Воздержанием поживше, просвещеннии Христовы апостоли, воздержания время нам ходатайствы Божественными утишают.


\mysubsubsection{Святии апостоли, молите Бога о нас.}


Двоенадесятострунный орган песнь воспе спасительную, учеников лик Божественный, лукавая возмущая гласования.


\mysubsubsection{Святии апостоли, молите Бога о нас.}


Одождением духовным всю подсолнечную напоисте, сушу отгнавше многобожия, всеблаженнии.


\mysubsubsection{Пресвятая Богородице, спаси нас.}


Смирившася спаси мя, высокомудренно пожившаго, рождшая Вознесшаго смиренное естество, Дево Всечистая.


Иный. \itshape Ирмос\normalfont{}, глас тойже: Услышах, Господи, смотрения Твоего таинство, разумех дела Твоя и прославих Твое Божество.


\mysubsubsection{Святии апостоли, молите Бога о нас.}


Апостольское всечестное ликостояние, Зиждителя всех молящее, проси помиловати ны, восхваляющия тя.


\mysubsubsection{Святии апостоли, молите Бога о нас.}


Яко делателе суще, Христовы апостоли, во всем мире Божественным словом возделавшии, приносите плоды Ему всегда.


\mysubsubsection{Святии апостоли, молите Бога о нас.}


Виноград бысте Христов воистинну возлюбленный, вино бо духовное источисте миру, апостоли.


\mysubsubsection{Пресвятая Троице, Боже наш, слава Тебе.}


Преначальная, Сообразная, Всесильнейшая Троице Святая, Отче, Слове и Душе Святый, Боже, Свете и Животе, сохрани стадо Твое.


\mysubsubsection{Пресвятая Богородице, спаси нас.}


Радуйся, престоле огнезрачный, радуйся, светильниче свещеносный, радуйся, горо освящения, ковчеже Жизни, святых святая сене.


\itshape Великаго канона Ирмос\normalfont{}: Услыша пророк пришествие Твое, Господи, и убояся, яко хощеши от Девы родитися и человеком явитися, и глаголаше: услышах слух Твой и убояхся, слава силе Твоей, Господи.


Дел Твоих да не презриши, создания Твоего да не оставиши, Правосуде. Аще и един согреших, яко человек, паче всякаго человека, Человеколюбче; но имаши, яко Господь всех, власть оставляти грехи.


Приближается, душе, конец, приближается, и нерадиши, ни готовишися, время сокращается: востани, близ при дверех Судия есть. Яко соние, яко цвет, время жития течет: что всуе мятемся?


Воспряни, о душе моя, деяния твоя, яже соделала еси, помышляй, и сия пред лице твое принеси, и капли испусти слез твоих; рцы со дерзновением деяния и помышления Христу и оправдайся.


Не бысть в житии греха, ни деяния, ни злобы, еяже аз, Спасе, не согреших, умом, и словом, и произволением, и предложением, и мыслию, и деянием согрешив, яко ин никтоже когда.


Отсюду осужден бых, отсюду и препрен бых аз, окаянный, от своея совести, еяже ничтоже в мире нужнейше: Судие, Избавителю мой и Ведче, пощади, и избави, и спаси мя, раба Твоего.


Лествица, юже виде древле великий в патриарсех, указание есть, душе моя, деятельнаго восхождения, разумнаго возшествия: аще хощеши убо деянием, и разумом, и зрением пожити, обновися.


Зной дневный претерпе лишения ради патриарх и мраз нощный понесе, на всяк день снабдения творя, пасый, труждаяйся, работаяй, да две жене сочетает.


Жены ми две разумей, деяние же и разум в зрении, Лию убо деяние, яко многочадную, Рахиль же разум, яко многотрудную; ибо кроме трудов ни деяние, ни зрение, душе, исправится.


Бди, о душе моя, изрядствуй, якоже древле великий в патриарсех, да стяжеши деяние с разумом, да будеши ум, зряй Бога, и достигнеши незаходящий мрак в видении, и будеши великий купец.


Дванадесяте патриархов великий в патриарсех детотворив, тайно утверди тебе лествицу деятельнаго, душе моя, восхождения: дети, яко основания, степени, яко восхождения, премудренно подложив.


Исава возненавиденнаго подражала еси, душе, отдала еси прелестнику твоему первыя доброты первенство, и отеческия молитвы отпала еси, и дважды поползнулася еси, окаянная, деянием и разумом: темже ныне покайся.


Едом Исав наречеся, крайняго ради женонеистовнаго смешения: невоздержанием бо присно разжигаем и сластьми оскверняем, Едом именовася, еже глаголется разжжение души любогреховныя.


Иова на гноищи слышавши, о душе моя, оправдавшагося, того мужеству не поревновала еси, твердаго не имела еси предложения во всех, яже веси, и имиже искусилася еси, но явилася еси нетерпелива.


Иже первее на престоле, наг ныне на гноище гноен, многий в чадех и славный, безчаден и бездомок напрасно: палату убо гноище и бисерие струпы вменяше.


Царским достоинством, венцем и багряницею одеян, многоименный человек и праведный, богатством кипя и стады, внезапу богатства, славы царства, обнищав, лишися.


Аще праведен бяше он и непорочен паче всех, и не убеже ловления льстиваго и сети; ты же грехолюбива сущи, окаянная душе, что сотвориши, аще чесому от недоведомых случится наити тебе?


Тело осквернися, дух окаляся, весь острупихся, но яко врач, Христе, обоя покаянием моим уврачуй, омый, очисти, покажи, Спасе мой, паче снега чистейша.


Тело Твое и кровь, распинаемый о всех, положил еси, Слове: тело убо, да мя обновиши, кровь, да омыеши мя. Дух же предал еси, да мя приведеши, Христе, Твоему Родителю.


Соделал еси спасение посреде земли, Щедре, да спасемся. Волею на древе распялся еси, Едем затворенный отверзеся, горняя и дольняя тварь, языцы вси спасени покланяются Тебе.


Да будет ми купель кровь из ребр Твоих, вкупе и питие, источившее воду оставления, да обоюду очищаюся, помазуяся и пия, яко помазание и питие, Слове, животочная Твоя словеса.


Наг есмь чертога, наг есмь и брака, купно и вечери, светильник угасе, яко безъелейный, чертог заключися мне спящу, вечеря снедеся, аз же по руку и ногу связан, вон низвержен есмь.


Чашу Церковь стяжа, ребра Твоя живоносная, из нихже сугубыя нам источи токи оставления и разума, во образ древняго и новаго, двоих вкупе заветов, Спасе наш.


Время живота моего мало и исполнено болезней и лукавства, но в покаянии мя приими и в разум призови, да не буду стяжание, ни брашно чуждему, Спасе, Сам мя ущедри.


Высокоглаголив ныне есмь, жесток же и сердцем, вотще и всуе, да не с фарисеем осудиши мя. Паче же мытарево смирение подаждь ми, Едине Щедре, Правосуде, и сему мя сочисли.


Согреших, досадив сосуду плоти моея, вем, Щедре, но в покаянии мя приими и в разум призови, да не буду стяжание ни брашно чуждему, Спасе, Сам мя ущедри.


Самоистукан бых страстьми, душу мою вредя, Щедре, но в покаянии мя приими и в разум призови, да не буду стяжание ни брашно чуждему, Спасе, Сам мя ущедри.


Не послушах гласа Твоего, преслушах Писание Твое, Законоположника, но в покаянии мя приими и в разум призови, да не буду стяжание ни брашно чуждему, Спасе, Сам мя ущедри.


\bfseries Пресвятая Троице, Боже наш, слава Тебе.





Преподобная мати Марие, моли Бога о нас.\normalfont{}


Безплотных жительство в плоти преходящи, благодать, преподобная, к Богу велию воистинну прияла еси, верно о чтущих тя предстательствуй. Темже молим тя, от всяких напастей и нас молитвами твоими избави.


\mysubsubsection{Преподобная мати Марие, моли Бога о нас.}


Великих безместий во глубину низведшися, неодержима была еси, но востекла еси помыслом лучшим к крайней деяньми яве добродетели преславно, ангельское естество, Марие, удививши.


\mysubsubsection{Преподобне отче Андрее, моли Бога о нас.}


Андрее, отеческая похвало, молитвами твоими не престай, моляся, предстоя Троице Пребожественней, яко да избавимся мучения, любовию предстателя тя Божественнаго, всеблаженне, призывающии, Криту удобрение.


\itshape Слава\normalfont{}: Нераздельное Существом, Неслитное Лицы богословлю Тя, Троическое Едино Божество, яко Единоцарственное и Сопрестольное, вопию Ти песнь великую, в вышних трегубо песнословимую.


\itshape И ныне\normalfont{}: И раждаеши, и девствуеши, и пребываеши обоюду естеством Дева, Рождейся обновляет законы естества, утроба же раждает нераждающая. Бог идеже хощет, побеждается естества чин: творит бо, елика хощет. 


\mysubsubsection{Песнь 5}


\itshape Ирмос\normalfont{}: От нощи утренююща, Человеколюбче, просвети, молюся, и настави и мене на повеления Твоя, и научи мя, Спасе, творити волю Твою.


В нощи житие мое преидох присно, тьма бо бысть, и глубока мне мгла, нощь греха, но яко дне сына, Спасе, покажи мя.


Рувима подражая, окаянный аз содеях беззаконный и законопреступный совет на Бога Вышняго, осквернив ложе мое, яко отчее он.


Исповедаюся Тебе, Христе Царю: согреших, согреших, яко прежде Иосифа братия продавшии, чистоты плод и целомудрия.


От сродников праведная душа связася, продася в работу сладкий во образ Господень; ты же вся, душе, продалася еси злыми твоими.


Иосифа праведнаго и целомудреннаго ума подражай, окаянная и неискусная душе, и не оскверняйся безсловесными стремленьми, присно беззаконнующи.


Аще и в рове поживе иногда Иосиф, Владыко Господи, но во образ погребения и востания Твоего: аз же что Тебе когда сицевое принесу?


Моисеов слышала еси ковчежец, душе, водами, волнами носим речными, яко в чертозе древле бегающий дела, горькаго совета фараонитска.


Аще бабы слышала еси, убивающия иногда безвозрастное мужеское, душе окаянная, целомудрия деяние, ныне, яко великий Моисей, сси премудрость.


Яко Моисей великий египтянина, ума, уязвивши, окаянная, не убила еси, душе; и како вселишися, глаголи, в пустыню страстей покаянием?


В пустыню вселися великий Моисей; гряди убо, подражай того житие, да и в купине Богоявления, душе, в видении будеши.


Моисеов жезл воображай, душе, ударяющий море и огустевающий глубину во образ Креста Божественнаго: имже можеши и ты великая совершити.


Аарон приношаше огнь Богу непорочный, нелестный; но Офни и Финеес, яко ты, душе, приношаху чуждее Богу, оскверненное житие.


Яко тяжкий нравом, фараону горькому бых, Владыко, Ианни и Иамври, душею и телом, и погружен умом, но помози ми.


Калу примесихся, окаянный, умом, омый мя, Владыко, банею моих слез, молю Тя, плоти моея одежду убелив яко снег.


Аще испытаю моя дела, Спасе, всякаго человека превозшедша грехами себе зрю, яко разумом мудрствуяй, согреших не неведением.


Пощади, пощади, Господи, создание Твое, согреших, ослаби ми, яко естеством чистый Сам сый Един, и ин разве Тебе никтоже есть кроме скверны.


Мене ради Бог сый, вообразился еси в мя, показал еси чудеса, исцелив прокаженныя и разслабленнаго стягнув, кровоточивыя ток уставил еси, Спасе, прикосновением риз.


Кровоточивую подражай, окаянная душе, притецы, удержи ометы Христовы, да избавишися ран и услышиши от Него: вера твоя спасе тя.


Низу сничащую подражай, о душе, прииди, припади к ногама Иисусовыма, да тя исправит, и да ходиши право стези Господни.


Аще и кладязь еси глубокий, Владыко, источи ми воду из пречистых Твоих жил, да, яко самаряныня, не ктому, пияй, жажду: жизни бо струи источаеши.


Силоам да будут ми слезы моя, Владыко Господи, да умыю и аз зеницы сердца и вижду Тя умно, Света превечна.


\mysubsubsection{Преподобная мати Марие, моли Бога о нас.}


Несравненным желанием, всебогатая, древу возжелевши поклонитися животному, сподобилася еси желания, сподоби убо и мене улучити вышния славы.


\mysubsubsection{Преподобная мати Марие, моли Бога о нас.}


Струи Иорданския прешедши, обрела еси покой безболезненный, плоти сласти избежавши, еяже и нас изми твоими молитвами, преподобная.


Преподобный отче Андрее, моли Бога о нас.


Яко пастырей изряднейша, Андрее премудре, избранна суща тя, любовию велиею и страхом молю, твоими молитвами спасение улучити и жизнь вечную.


\itshape Слава\normalfont{}: Тя, Троице, славим, Единаго Бога: Свят, Свят, Свят еси, Отче, Сыне и Душе, Простое Существо, Единице присно покланяемая.


\itshape И ныне\normalfont{}: Из Тебе облечеся в мое смешение, нетленная, безмужная Мати Дево, Бог, создавый веки, и соедини Себе человеческое естество. 


\mysubsubsection{Песнь 6}


\itshape Ирмос\normalfont{}: Возопих всем сердцем моим к щедрому Богу, и услыша мя от ада преисподняго, и возведе от тли живот мой.


Слезы, Спасе, очию моею и из глубины воздыхания чисте приношу, вопиющу сердцу: Боже, согреших Ти, очисти мя.


Уклонилася еси, душе, от Господа твоего, якоже Дафан и Авирон, но пощади, воззови из ада преисподняго, да не пропасть земная тебе покрыет.


Яко юница, душе, разсвирепевшая, уподобилася еси Ефрему, яко серна от тенет сохрани житие, вперивши деянием ум и зрением.


Рука нас Моисеова да уверит, душе, како может Бог прокаженное житие убелити и очистити, и не отчайся сама себе, аще и прокаженна еси.


Волны, Спасе, прегрешений моих, яко в мори Чермнем возвращающеся, покрыша мя внезапу, яко египтяны иногда и тристаты.


Неразумное, душе, произволение имела еси, яко прежде Израиль: Божественныя бо манны предсудила еси безсловесно любосластное страстей объядение.


Кладенцы, душе, предпочла еси хананейских мыслей паче жилы камене, из негоже премудрости река, яко чаша, проливает токи богословия.


Свиная мяса и котлы и египетскую пищу паче Небесныя предсудила еси, душе моя, якоже древле неразумнии людие в пустыни.


Яко удари Моисей, раб Твой, жезлом камень, образно животворивая ребра Твоя прообразоваше, из нихже вси питие жизни, Спасе, почерпаем.


Испытай, душе, и смотряй, якоже Иисус Навин, обетования землю, какова есть, и вселися в ню благозаконием.


Востани и побори, яко Иисус Амалика, плотския страсти, и гаваониты, лестныя помыслы, присно побеждающи.


Прейди времене текущее естество, яко прежде ковчег, и земли оныя буди во одержании обетования, душе, Бог повелевает.


Яко спасл еси Петра, возопивша, спаси, предварив мя, Спасе, от зверя избави, простер Твою руку, и возведи из глубины греховныя.


Пристанище Тя вем утишное, Владыко, Владыко Христе, но от незаходимых глубин греха и отчаяния мя, предварив, избави.


Аз есмь, Спасе, юже погубил еси древле царскую драхму; но вжег светильник, Предтечу Твоего, Слове, взыщи и обрящи Твой образ.


\mysubsubsection{Преподобная мати Марие, моли Бога о нас.}


Да страстей пламень угасиши, слез капли источала еси присно, Марие, душею распалаема, ихже благодать подаждь и мне, твоему рабу.


\mysubsubsection{Преподобная мати Марие, моли Бога о нас.}


Безстрастие Небесное стяжала еси крайним на земли житием, мати. Темже тебе поющим, от страстей избавитися молитвами твоими молися.


Преподобный отче Андрее, моли Бога о нас.


Критскаго тя пастыря и председателя и вселенныя молитвенника ведый, притекаю, Андрее, и вопию ти: изми мя, отче, из глубины греха.


\itshape Слава\normalfont{}: Троица есмь Проста, Нераздельна, раздельна Личне и Единица есмь естеством соединена, Отец глаголет, и Сын, и Божественный Дух.


\itshape И ныне\normalfont{}: Утроба Твоя Бога нам роди, воображена по нам: Егоже, яко Создателя всех, моли, Богородице, да молитвами Твоими оправдимся.


\itshape Ирмос\normalfont{}: Возопих всем сердцем моим к щедрому Богу, и услыша мя от ада преисподняго, и возведе от тли живот мой. 

\mysubsubsection{Кондак, глас 6:}


;Душе моя, душе моя, востани, что спиши? конец приближается, и имаши смутитися: воспряни убо, да пощадит тя Христос Бог, везде сый и вся исполняяй. 


\mysubsubsection{Икос:}


Христово врачевство видя отверсто и от сего Адаму истекающее здравие, пострада, уязвися диавол и, яко бедствуя, рыдаше и своим другом возопи: что сотворю Сыну Мариину, убивает мя Вифлеемлянин, Иже везде сый и вся исполняяй. 

\mysubsubsection{Таже блаженны, глас 6:}


Во Царствии Твоем помяни нас, Господи.


Разбойника, Христе, рая жителя сотворил еси, на кресте Тебе возопивша: помяни мя; того покаянию сподоби и мене, недостойнаго.


Блажени нищии духом, яко тех есть Царство Небесное.


Маноя слышавши древле, душе моя, Бога в явлении бывша и из неплодове тогда приемша плод обетования, того благочестие подражай.


Блажени плачущии, яко тии утешатся.


Сампсоновой поревновавши лености, главу остригла еси, душе, дел твоих, предавши иноплеменником любосластием целомудренную жизнь и блаженную.


Блажени кротции, яко тии наследят землю.


Прежде челюстию ослею победивый иноплеменники, ныне пленение ласкосердству страстному обретеся; но избегни, душе моя, подражания, деяния и слабости.


Блажени алчущии и жаждущии правды, яко тии насытятся.


Варак и Иеффай военачальницы, судии Израилевы предпочтени быша, с нимиже Деворра мужеумная; тех доблестьми, душе, вмужившися, укрепися.


Блажени милостивии, яко тии помиловани будут.


Иаилино храбрство познала еси, душе моя, Сисара древле прободшую и спасение соделавшую древом острым, слышиши, имже тебе крест образуется.


Блажени чистии сердцем, яко тии Бога узрят.


Пожри, душе, жертву похвальную, деяние, яко дщерь, принеси от Иеффаевы чистейшую и заколи, яко жертву, страсти плотския Господеви твоему.


Блажени миротворцы, яко тии сынове Божии нарекутся.


Гедеоново руно помышляй, душе моя, с небесе росу подыми и приникни, якоже пес, и пий воду, от закона текущую, изгнетением письменным.


Блажени изгнани правды ради, яко тех есть Царство Небесное.


Илии священника осуждение, душе моя, восприяла еси, лишением ума приобретши страсти себе, якоже он чада, делати беззаконная.


Блажени есте, егда поносят вам и изженут и рекут всяк зол глагол на вы лжуще, Мене ради.


В судиях левит небрежением свою жену дванадесятим коленом раздели, душе моя, да скверну обличит от Вениамина беззаконную.


Радуйтеся и веселитеся, яко мзда ваша многа на Н ебесех.


Любомудренная Анна молящися, устне убо двизаше ко хвалению, глас же ея не слышашеся, но обаче неплодна сущи, сына молитвы раждает достойна.


Помяни нас, Господи, егда приидеши во Царствии Твоем.


В судиях спричтеся Аннино порождение, великий Самуил, егоже воспитала Армафема в дому Господни; тому поревнуй, душе моя, и суди прежде инех дела твоя.


Помяни нас, Владыко, егда приидеши во Царствии Твоем.


Давид на царство избран, царски помазася рогом Божественного мира; ты убо, душе моя, аще хощеши вышняго Царствия, миром помажися слезами.


Помяни нас, Святый, егда приидеши во Царствии Твоем.


Помилуй создание Твое, Милостиве, ущедри руку Твоею творение и пощади вся согрешившия, и мене паче всех, Твоих презревшаго повелений.


\itshape Слава\normalfont{}: Безначальну и рождению же и происхождению Отцу покланяюся рождшему, Сына славлю рожденнаго, пою сопросиявшаго Отцу же и Сыну Духа Святаго.


\itshape И ныне\normalfont{}: Преестественному Рождеству Твоему покланяемся, по естеству славы Младенца Твоего не разделяюще, Богородительнице: Иже бо Един Лицем, сугубыми исповедуется естествы. 


\mysubsubsection{Песнь 7}


\itshape Ирмос\normalfont{}: Согрешихом, беззаконновахом, неправдовахом пред Тобою, ниже соблюдохом, ниже сотворихом, якоже заповедал еси нам; но не предаждь нас до конца, отцев Боже.


Согреших, беззаконновах и отвергох заповедь Твою, яко во гресех произведохся, и приложих язвам струпы себе; но Сам мя помилуй, яко благоутробен, отцев Боже.


Тайная сердца моего исповедах Тебе, Судии моему, виждь мое смирение, виждь и скорбь мою, и вонми суду моему ныне, и Сам мя помилуй, яко благоутробен, отцев Боже.


Саул иногда, яко погуби отца своего, душе, ослята, внезапу царство обрете к прослутию; но блюди, не забывай себе, скотския похоти твоя произволивши паче Царства Христова.


Давид иногда Богоотец, аще и согреши сугубо, душе моя, стрелою убо устрелен быв прелюбодейства, копием же пленен быв убийства томлением; но ты сама тяжчайшими делы недугуеши, самохотными стремленьми.


Совокупи убо Давид иногда беззаконию беззаконие, убийству же любодейство растворив, покаяние сугубое показа абие; но сама ты, лукавнейшая душе, соделала еси, не покаявшися Богу.


Давид иногда вообрази, списав яко на иконе песнь, еюже деяние обличает, еже содея, зовый: помилуй мя, Тебе бо Единому согреших всех Богу, Сам очисти мя.


Кивот яко ношашеся на колеснице, Зан оный, егда превращшуся тельцу, точию коснуся, Божиим искусися гневом; но того дерзновения убежавши, душе, почитай Божественная честне.


Слышала еси Авессалома, како на естество воста, познала еси того скверная деяния, имиже оскверни ложе Давида отца; но ты подражала еси того страстная и любосластная стремления.


Покорила еси неработное твое достоинство телу твоему, иного бо Ахитофела обретши врага, душе, снизшла еси сего советом; но сия разсыпа Сам Христос, да ты всяко спасешися.


Соломон чудный и благодати премудрости исполненный, сей лукавое иногда пред Богом сотворив, отступи от Него; емуже ты проклятым твоим житием, душе, уподобилася еси.


Сластьми влеком страстей своих, оскверняшеся, увы мне, рачитель премудрости, рачитель блудных жен, и странен от Бога; егоже ты подражала еси умом, о душе, сладострастьми скверными.


Ровоаму поревновала еси, не послушавшему совета отча, купно же и злейшему врагу Иеровоаму, прежнему отступнику, душе, но бегай подражания и зови Богу: согреших, ущедри мя.


Ахаавовым поревновала еси сквернам, душе моя, увы мне, была еси плотских скверн пребывалище и сосуд срамлен страстей, но из глубины твоея воздохни и глаголи Богу грехи твоя.


Попали Илия иногда дващи пятьдесят Иезавелиных, егда студныя пророки погуби во обличение Ахаавово, но бегай подражания двою, душе, и укрепляйся.


Заключися тебе небо, душе, и глад Божий постиже тя, егда Илии Фесвитянина, якоже Ахаав, не покорися словесем иногда, но Сараффии уподобися, напитай пророчу душу.


Манассиева собрала еси согрешения изволением, поставльши, яко мерзости, страсти, и умноживши, душе, негодование, но того покаянию ревнующи тепле, стяжи умиление.


Припадаю Ти и приношу Тебе, якоже слезы, глаголы моя: согреших, яко не согреши блудница, и беззаконновах, яко иный никтоже на земли. Но ущедри, Владыко, творение Твое и воззови мя.


Погребох образ Твой и растлих заповедь Твою, вся помрачися доброта, и страстьми угасися, Спасе, свеща. Но, ущедрив, воздаждь ми, якоже поет Давид, радование.


Обратися, покайся, открый сокровенная, глаголи Богу, вся ведущему: Ты веси моя тайная, едине Спасе. Но Сам мя помилуй, якоже поет Давид, по милости Твоей.


Исчезоша дние мои, яко соние востающаго; темже, яко Езекия, слезю на ложи моем, приложитися мне летом живота. Но кий Исаия предстанет тебе, душе, аще не всех Бог?


\mysubsubsection{Преподобная мати Марие, моли Бога о нас.}


Возопивши к Пречистей Богоматери, первее отринула еси неистовство страстей, нужно стужающих, и посрамила еси врага запеншаго. Но даждь ныне помощь от скорби и мне, рабу твоему.


\mysubsubsection{Преподобная мати Марие, моли Бога о нас.}


Егоже возлюбила еси, Егоже возжелела еси, Егоже ради плоть изнурила еси, преподобная, моли ныне Христа о рабех: яко да милостив быв всем нам, мирное состояние дарует почитающим Его.


\mysubsubsection{Преподобне отче Андрее, моли Бога о нас.}


На камени мя веры молитвами твоими утверди, отче, страхом мя Божественным ограждая, и покаяние, Андрее, подаждь ми, молюся ти, и избави мя от сети врагов, ищущих мя.


\itshape Слава\normalfont{}: Троице Простая, Нераздельная, Единосущная и Естество Едино, Светове и Свет, и Свята Три, и Едино Свято поется Бог Троица; но воспой, прослави Живот и Животы, душе, всех Бога.


\itshape И ныне\normalfont{}: Поем Тя, благословим Тя, покланяемся Ти, Богородительнице, яко Неразлучныя Троицы породила еси Единаго Христа Бога и Сама отверзла еси нам, сущим на земли, Небесная. 


\medskip\bfseries Трипеснец, глас 8:

\normalfont{}


\mysubsubsection{Песнь 8}


\itshape Ирмос\normalfont{}: Безначальнаго Царя славы, Егоже трепещут Небесныя силы, пойте, священницы, людие, превозносите во вся веки.


\mysubsubsection{Святии апостоли, молите Бога о нас.}


Яко углие невещественнаго огня, попалите вещественныя страсти моя, возжизающе ныне во мне желание Божественныя любве, апостоли.


\mysubsubsection{Святии апостоли, молите Бога о нас.}


Трубы благогласныя Слова почтим, имиже падоша стены неутверждены вражия и богоразумия утвердишася забрала.


\mysubsubsection{Святии апостоли, молите Бога о нас.}


Кумиры страстныя души моея сокрушите, иже храмы и столпы сокрушисте врага, апостоли Господни, храмове освященнии.


Пресвятая Богородице, спаси нас.


Вместила еси Невместимаго естеством, носила еси Носящаго вся, доила еси, Чистая, питающаго тварь Христа Жизнодавца.


Иный трипеснец. \itshape Ирмос\normalfont{}: Безначальнаго Царя:


\mysubsubsection{Святии апостоли, молите Бога о нас.}


Духа началохитростием создавше всю Церковь, апостоли Христовы, в ней благословите Христа во веки.


\mysubsubsection{Святии апостоли, молите Бога о нас.}


Вострубивше трубою учений, низвергоша апостоли всю лесть идольскую, Христа превозносяща во вся веки.


\mysubsubsection{Святии апостоли, молите Бога о нас.}


Апостоли, доброе преселение, назирателие мира и Небеснии жителие, вас присно восхваляющия избавите от бед.


\mysubsubsection{Пресвятая Троице, Боже наш, слава Тебе.}


Трисолнечное Всесветлое Богоначалие, Единославное и Единопрестольное Естеству, Отче Вседетелю, Сыне и Божественный Душе, пою Тя во веки.


\mysubsubsection{Пресвятая Богородице, спаси нас.}


Яко честный и превышший престол, воспоим Божию Матерь непрестанно, людие, Едину по рождестве Матерь и Деву.


Великаго канона \itshape Ирмос\normalfont{}: Егоже воинства Небесная славят, и трепещут херувими и серафими, всяко дыхание и тварь, пойте, благословите и превозносите во вся веки.


Согрешивша, Спасе, помилуй, воздвигни мой ум ко обращению, приими мя кающагося, ущедри вопиюща: согреших Ти, спаси, беззаконновах, помилуй мя.


Колесничник Илия колесницею добродетелей вшед, яко на небеса, ношашеся превыше иногда от земных; сего убо, душе моя, восход помышляй.

Иорданова струя первее милотию Илииною Елиссеем ста сюду и сюду; ты же, о душе моя, сея не причастилася еси благодати за невоздержание.


Елиссей иногда прием милоть Илиину, прият сугубую благодать от Бога; ты же, о душе моя, сея не причастилася еси благодати за невоздержание.


Соманитида иногда праведнаго учреди, о душе, нравом благим; ты же не ввела еси в дом ни странна, ни путника. Темже чертога изринешися вон, рыдающи.


Гиезиев подражала еси, окаянная, разум скверный всегда, душе, егоже сребролюбие отложи поне на старость; бегай геенскаго огня, отступивши злых твоих.


Ты Озии, душе, поревновавши, сего прокажение в себе стяжала еси сугубо: безместная бо мыслиши, беззаконная же дееши; остави, яже имаши, и притецы к покаянию.


Ниневитяны, душе, слышала еси кающияся Богу, вретищем и пепелом, сих не подражала еси, но явилася еси злейшая всех, прежде закона и по законе прегрешивших.


В рове блата слышала еси Иеремию, душе, града Сионя рыданьми вопиюща и слез ищуща; подражай сего плачевное житие и спасешися.


Иона в Фарсис побеже, проразумев обращение ниневитянов, разуме бо, яко пророк, Божие благоутробие: темже ревноваше пророчеству не солгатися.


Даниила в рове слышала еси, како загради уста, о душе, зверей; уведела еси, како отроцы, иже о Азарии, погасиша верою пещи пламень горящий.


Ветхаго Завета вся приведох ти, душе, к подобию; подражай праведных боголюбивая деяния, избегни же паки лукавых грехов.


Правосуде Спасе, помилуй и избави мя огня и прещения, еже имам на суде праведно претерпети; ослаби ми прежде конца, добродетелию и покаянием.


Яко разбойник, вопию Ти: помяни мя; яко Петр, плачу горце: ослаби ми, Спасе; зову, яко мытарь, слезю, яко блудница; приими мое рыдание, якоже иногда хананеино.


Гноение, Спасе, исцели смиренныя моея души, Едине Врачу, пластырь мне наложи, и елей, и вино, дела покаяния, умиление со слезами.


Хананею и аз подражая, помилуй мя, вопию, Сыне Давидов; касаюся края ризы, яко кровоточивая, плачу, яко Марфа и Мария над Лазарем.


Слезную, Спасе, сткляницу яко миро истощавая на главу, зову Ти, якоже блудница, милости ищущая, мольбу приношу и оставление прошу прияти.


Аще и никтоже, якоже аз, согреши Тебе, но обаче приими и мене, благоутробне Спасе, страхом кающася и любовию зовуща: согреших Тебе Единому, помилуй мя, Милостиве.


Пощади, Спасе, Твое создание и взыщи, яко Пастырь, погибшее, предвари заблуждшаго, восхити от волка, сотвори мя овча на пастве Твоих овец.

Егда, Судие, сядеши, яко благоутробен, и покажеши страшную славу Твою, Спасе, о каковый страх тогда, пещи горящей, всем боящимся нестерпимаго судища Твоего.


\mysubsubsection{Преподобная мати Марие, моли Бога о нас.}


Света незаходимаго Мати тя просветивши, от омрачения страстей разреши. Темже вшедши в духовную благодать, просвети, Марие, тя верно восхваляющия.


\mysubsubsection{Преподобная мати Марие, моли Бога о нас.}


Чудо ново видев, ужасашеся божественный в тебе воистинну, мати, Зосима: ангела бо зряше во плоти и ужасом весь исполняшеся, Христа поя во веки.


\mysubsubsection{Преподобне отче Андрее, моли Бога о нас.}


Яко дерзновение имый ко Господу, Андрее Критский, честная похвало, молю, молися разрешение от уз беззакония ныне обрести мне молитвами твоими, яко покаяния учитель и преподобных слава.


Благословим Отца и Сына и Святаго Духа Господа.

Безначальне Отче, Сыне Собезначальне, Утешителю Благий, Душе Правый, Слова Божия Родителю, Отца Безначальна Слове, Душе Живый и Зиждяй, Троице Единице, помилуй мя.


\itshape И ныне\normalfont{} и присно и во веки веков.


Яко от оброщения червленицы, Пречистая, умная багряница Еммануилева внутрь во чреве Твоем плоть исткася. Темже Богородицу воистинну Тя почитаем.


Хвалим, благословим, покланяемся Господеви, поюще и превозносяще во вся веки.


\itshape Ирмос\normalfont{}: Егоже воинства Небесная славят, и трепещут херувими и серафими, всяко дыхание и тварь, пойте, благословите и превозносите во вся веки. 

\medskip\bfseries Поем Честнейшую:


Трипеснец, глас 8:\normalfont{}


\mysubsubsection{Песнь 9}


\itshape Ирмос\normalfont{}: Воистинну Богородицу Тя исповедуем, спасеннии Тобою, Дево чистая, с безплотными лики Тя величающе.


\mysubsubsection{Святии апостоли, молите Бога о нас.}


Источницы спасительныя воды явльшеся апостоли, истаявшую душу мою греховною жаждою оросите.


\mysubsubsection{Святии апостоли, молите Бога о нас.}


Плавающаго в пучине погибели и в погружении уже бывша Твоею десницею, якоже Петра, Господи, спаси мя.


\mysubsubsection{Святии апостоли, молите Бога о нас.}


Яко соли, вкусных суще учений, гнильство ума моего изсушите и неведения тьму отжените.


\mysubsubsection{Пресвятая Богородице, спаси нас.}


Радость яко родившая, плач мне подаждь, имже Божественное утешение, Владычице, в будущем дни обрести возмогу.


Иный. \itshape Ирмос\normalfont{}: Тя, Небесе и земли Ходатаицу:

\mysubsubsection{Святии апостоли, молите Бога о нас.}


Тя, благославное апостольское собрание, песньми величаем: вселенней бо светила светлая явистеся, прелесть отгоняще.


\mysubsubsection{Святии апостоли, молите Бога о нас.}


Благовестною мрежею вашею словесныя рыбы уловивше, сия приносите всегда снедь Христу, апостоли блаженнии.

\mysubsubsection{Святии апостоли, молите Бога о нас.}


К Богу вашим прошением помяните нас, апостоли, от всякаго избавитися искушения, молимся, любовию воспевающия вас.


\mysubsubsection{Пресвятая Троице, Боже наш, слава Тебе.}

Тя, Триипостасную Единицу, Отче, Сыне со Духом, Единаго Бога Единосущна пою, Троицу Единосильную Безначальную.


\mysubsubsection{Пресвятая Богородице, спаси нас.}


Тя, Детородительницу и Деву, вси роди ублажаем, яко Тобою избавльшеся от клятвы: радость бо нам родила еси, Господа.


Великаго канона \itshape Ирмос\normalfont{}: Безсеменнаго зачатия Рождество несказанное, Матере безмужныя нетленен Плод, Божие бо Рождение обновляет естества. Темже Тя вси роди, яко Богоневестную Матерь, православно величаем.


Ум острупися, тело оболезнися, недугует дух, слово изнеможе, житие умертвися, конец при дверех. Темже, моя окаянная душе, что сотвориши, егда приидет Судия испытати твоя?


Моисеово приведох ти, душе, миробытие и от того все заветное Писание, поведающее тебе праведныя и неправедныя; от нихже вторыя, о душе, подражала еси, а не первыя, в Бога согрешивши.


Закон изнеможе, празднует Евангелие, Писание же все в тебе небрежено бысть, пророцы изнемогоша, и все праведное слово; струпи твои, о душе, умножишася, не сущу врачу, исцеляющему тя.


Новаго привожду ти Писания указания, вводящая тя, душе, ко умилению: праведным убо поревнуй, грешных же отвращайся и умилостиви Христа молитвами же, и пощеньми, и чистотою, и говением.


Христос вочеловечися, призвав к покаянию разбойники и блудницы; душе, покайся, дверь отверзеся Царствия уже, и предвосхищают е фарисее, и мытари, и прелюбодеи кающиися.


Христос вочеловечися, плоти приобщився ми, и вся елика суть естества хотением исполни греха кроме, подобие тебе, о душе, и образ предпоказуя Своего снисхождения.


Христос волхвы спасе, пастыри созва, младенец множества показа мученики, старцы прослави и старыя вдовицы, ихже не поревновала еси, душе, ни деянием, ни житию, но горе тебе, внегда будеши судитися.


Постився Господь дний четыредесять в пустыни, последи взалка, показуя человеческое; душе, да не разленишися, аще тебе приложится враг, молитвою же и постом от ног твоих да отразится.


Христос искушашеся, диавол искушаше, показуя камение, да хлеби будут, на гору возведе видети вся царствия мира во мгновении; убойся, о душе, ловления, трезвися, молися на всякий час Богу.


Горлица пустыннолюбная, глас вопиющаго возгласи, Христов светильник, проповедуяй покаяние, Ирод беззаконнова со Иродиадою. Зри, душе моя, да не увязнеши в беззаконныя сети, но облобызай покаяние.


В пустыню вселися благодати Предтеча, и Иудея вся и Самария слышавше течаху и исповедаху грехи своя, крещающеся усердно: ихже ты не подражала еси, душе.


Брак убо честный и ложе нескверно, обоя бо Христос прежде благослови, плотию ядый, и в Кане же на браце воду в вино совершая, и показуя первое чудо, да ты изменишися, о душе.


Разслабленнаго стягну Христос, одр вземша, и юношу умерша воздвиже, вдовиче рождение, и сотнича отрока, и самаряныне явися, в дусе службу тебе, душе, предживописа.


Кровоточивую исцели прикосновением края ризна Господь, прокаженныя очисти, слепыя и хромыя просветив, исправи, глухия же и немыя и ничащия низу исцели словом: да ты спасешися, окаянная душе.


Недуги исцеляя, нищим благовествоваше Христос Слово, вредныя уврачева, с мытари ядяше, со грешники беседоваше, Иаировы дщере душу предумершую возврати осязанием руки.


Мытарь спасашеся, и блудница целомудрствоваше, и фарисей, хваляся, осуждашеся. Ов убо: очисти мя; ова же: помилуй мя; сей же величашеся вопия: Боже, благодарю Тя, и прочия безумныя глаголы.


Закхей мытарь бе, но обаче спасашеся, и фарисей Симон соблажняшеся, и блудница приимаше оставительная разрешения от Имущаго крепость оставляти грехи, юже, душе, потщися подражати.


Блуднице, о окаянная душе моя, не поревновала еси, яже приимши мира алавастр, со слезами мазаше нозе Спасове, отре же власы, древних согрешений рукописание Раздирающаго ея.


Грады, имже даде Христос благовестие, душе моя, уведала еси, како прокляти быша. Убойся указания, да не будеши якоже оны, ихже содомляном Владыка уподобив, даже до ада осуди.


Да не горшая, о душе моя, явишися отчаянием, хананеи веру слышавшая, еяже дщи словом Божиим исцелися; Сыне Давидов, спаси и мене, воззови из глубины сердца, якоже она Христу.


Умилосердися, спаси мя, Сыне Давидов, помилуй, беснующияся словом исцеливый, глас же благоутробный, яко разбойнику, мне рцы: аминь, глаголю тебе, со Мною будеши в раи, егда прииду во славе Моей.


Разбойник оглаголоваше Тя, разбойник богословяше Тя: оба бо на кресте свисяста. Но, о Благоутробне, яко верному разбойнику Твоему, познавшему Тя Бога, и мне отверзи дверь славнаго Царствия Твоего.


Тварь содрогашеся, распинаема Тя видящи, горы и камения страхом распадахуся, и земля сотрясашеся, и ад обнажашеся, и соомрачашеся свет во дне, зря Тебе, Иисусе, пригвождена ко Кресту.


Достойных покаяния плодов не истяжи от мене, ибо крепость моя во мне оскуде; сердце мне даруй присно сокрушенное, нищету же духовную: да сия Тебе принесу яко приятную жертву, едине Спасе.


Судие мой и Ведче мой, хотяй паки приити со ангелы судити миру всему, милостивным Твоим оком тогда видев мя, пощади и ущедри мя, Иисусе, паче всякаго естества человеча согрешивша.


\mysubsubsection{Преподобная мати Марие, моли Бога о нас.}


Удивила еси всех странным житием твоим, ангелов чины и человеков соборы, невещественно поживши и естество прешедши; имже, яко невещественныма ногама вшедши, Марие, Иордан прешла еси.


\mysubsubsection{Преподобная мати Марие, моли Бога о нас.}


Умилостиви Создателя о хвалящих тя, преподобная мати, избавитися озлоблений и скорбей, окрест нападающих: да избавившеся от напастей, возвеличим непрестанно прославльшаго тя Господа.


\mysubsubsection{Преподобне отче Андрее, моли Бога о нас.}


Андрее честный и отче треблаженнейший, пастырю Критский, не престай моляся о воспевающих тя: да избавимся вси гнева, и скорби, и тления, и прегрешений безмерных, чтущии твою память верно.


\itshape Слава\normalfont{}: Троице Единосущная, Единице Триипостасная, Тя воспеваем, Отца славяще, Сына величающе и Духу покланяющеся, Единому Естеству воистинну Богу, Жизни же и живущему Царству безконечному.


\itshape И ныне\normalfont{}: Град Твой сохраняй, Богородительнице Пречистая, в Тебе бо сей верно царствуяй, в Тебе и утверждается, и Тобою побеждаяй, побеждает всякое искушение, и пленяет ратники, и проходит послушание.


Таже оба лика вкупе поют \itshape Ирмос\normalfont{}:


Безсеменнаго зачатия Рождество несказанное, Матере безмужныя нетленен Плод, Божие бо Рождение обновляет естества. Темже Тя вси роди, яко Богоневестную Матерь, православно величаем.\mychapterending
