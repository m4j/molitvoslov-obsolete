

\mypart{ВЕЛИКИЙ КАНОН (ТВОРЕНИЕ СВЯТОГО АНДРЕЯ КРИТСКОГО)}\label{_content_velikiy-kanon}
%http://www.molitvoslov.com/content/velikiy-kanon

\mychapter{В начале Великопостного пути "--- Великий Канон св. Андрея Критского}\begin{mymulticols}
%http://www.molitvoslov.com/content/Ves-Kanon-tselikom-na-tserkovnoslayavyanskom-yazyke-pdf 
 
\myfigure{300px-Andreas_cretensis_0}

Первая седмица Великого Поста с древних времен называется «зарей воздержания» и «неделей чистой». В эту неделю Церковь убеждает своих чад выйти из того греховного состояния, в которое невоздержанием наших прародителей ниспал весь род человеческий, утратив райское блаженство, и которое каждый из нас умножает сам своими грехами,"--- выйти путем веры, молитвы, смирения и Богоугодного поста. Се время покаяния, говорит Церковь, се день спасительный, пощения вход: душе, бодрствуй, и страстей входа затвори, ко Господу взирающе (из первой песни трипеснца на утрени понедельника первой седмицы Великого Поста).

Подобно ветхозаветной церкви, которая особенно святила первый и последний день некоторых великих праздников, православные христиане, приготовленные и воодушевленные матерними внушениями своей Церкви, издревле, согласно ее уставу, с особенным усердием и строгостью проводят первую и последнюю седмицу Великого Поста.

На первой седмице совершаются особенно продолжительные богослужения и подвиг телесного воздержания значительно более строгий, чем в последующие дни Святой Четыредесятницы. В первые четыре дня Великого Поста великое повечерие совершается с чтением Великого Покаянного Канона преп. Андрея Критского, который как бы задает «тон», определяет всю последующую тональность, «мелодию» Великого Поста. На первой седмице Поста Канон делится на четыре части. Дивное творение св. Андрея Критского полностью предлагается нашему вниманию и в четверг (точнее в среду вечером) пятой седмицы Св. Четыредесятницы для того, чтобы мы, видя приближающееся окончание Поста, не разленились к духовным подвигам, не сделались небрежными, не забылись и не перестали во всем строго следить за собой.

Каждый стих Великого Канона сопровождается псаломским припевом Помилуй мя, Боже, помилуй мя! К канону присоединяется несколько тропарей в честь самого автора "--- св. Андрея и преп. Марии Египетской. Еще при жизни св. Андрея Иерусалимская Церковь ввела у себя в употребление Великий Канон. Отправляясь в 680 году на Шестой Вселенский Собор в Константинополь, св. Андрей принес туда и сделал известным свое великое творение и житие преп. Марии Египетской, написанное его соотечественником и учителем патриархом Иерусалимским Софронием. Житие египетской подвижницы читается совместно с Великим Каноном на утрени в среду пятой седмицы Великого Поста.

Из всех молитвословий Великого Поста, больше всех других поражает душу Великий Канон. Великий Канон "--- это чудо церковной гимнографии, это тексты удивительной силы и поэтической красоты. Канон составлен в 7-м веке св. Андреем, Архиепископом Критским, составившим также много других канонов, которые Церковь использует в течение всего богослужебного года. Церковь именовала этот канон великим, не столько из-за его объема (в нем 250 тропарей или стихов), сколько по его внутреннему достоинству и силе.

Великий канон представляет собой беседу кающегося с собственной душой. Вот как он начинается:

Откуда начну плакати окаянного моего жития деяний? Кое ли положу начало, Христе, нынешнему рыданию? Но яко благоутробен даждь ми прегрешений оставление "--- с чего же начать мне каяться, ведь это так трудно.

Затем следует чудный тропарь:

Гряди, окаянная душе, с плотию твоею. Зиждителю всех исповеждься и останися прочее преждняго безсловесия, и принеси Богу в покаянии слезы.

Удивительные слова, тут и христианская антропология, и аскетика: плоть тоже должна участвовать в покаянии, как неотъемлемая часть человеческого естества.

Своего апогея эта беседа с душой, постоянные уговоры ее, призывы покаяться, достигают в кондаке, который поется после 6-й песни Канона:

Душе моя, душе моя, востани, что спиши? Конец приближается, и имаши смутитися; воспряни убо, да пощадит тя Христос Бог, везде сый, и вся исполняяй.

Эти слова произносит, обращаясь к себе, великий светильник Церкви, тот, к кому самому применимо было бы выражение, употребленное им относительно преп. Марии Египетской, которая действительно была «ангел во плоти». И вот он так к себе обратился, упрекая себя за то, что душа его спит. Если он видел себя таким, то какими должны видеть себя мы? Погруженные уже не только в беспробудный духовный сон, но в какое-то омертвение…

Когда мы внимаем словам кондака из канона святителя Андрея Критского, нужно спросить себя: что мне делать? Если бы человек как должно исполнял Божий закон, его жизнь была бы заполнена совсем другим содержанием. Вот для того Церковь и предлагает нам этот глубокий, проникновенный великопостный покаянный канон, чтобы мы заглянули поглубже в свою душу и посмотрели бы, что там. А душа-то спит… В этом и горе наше и беда наша.

В замечательной молитве преп. Ефрема Сирина, которую мы повторяем в продолжение всего Великого Поста, говорится: Господи Царю, даруй ми зрети моя прегрешения! "--- Я их не вижу, душа моя заснула, задремала и я этих грехов, как должно, даже и не вижу. Как же я буду каяться в них! И вот потому нужно в дни Великого Поста побольше сосредотачиваться на себе, оценивая свою жизнь и ее содержание евангельской мерой, а не какой-нибудь другой.

К основным особенностям Великого канона относится очень широкое использование образов и сюжетов из Священного Писания, как Ветхого, так и Нового Завета. Жаль, что мы плохо знакомы со Святой Библией. Многим из нас имена людей, упомянутых в Великом Каноне, ничего не говорят, потому что мы плохо знаем Библию.

А между тем Библия "--- это не только история израильского народа, но и грандиозная летопись души человеческой "--- души, которая падала и вставала перед лицом Бога, которая согрешала и каялась. Если мы посмотрим на жизнь людей, о которых говорится в Библии, то увидим, что каждый из них представлен не столько как исторический персонаж, не столько как личность, совершившая те или иные дела, сколько как человек, предстоящий перед лицом Живого Бога. Исторические и иные заслуги человека отходят на второй план, остается то, что всего важнее: сохранил человек верность Богу или нет. Если мы будем читать Библию и Великий Канон под таким углом зрения, то увидим, что многое из того, что говорится о древних праведниках и грешниках, является не чем иным, как летописью нашей души, наших падений и восстаний, наших грехов и покаяния.

Один церковный писатель по этому поводу очень кстати замечает: «Если в наши дни столь многие находят его (Великий Канон) скучным и не относящимся к нашей жизни, это происходит оттого, что вера их не питается из источника Священного Писания, которое для Отцов Церкви было именно источником их веры. Мы должны вновь научиться воспринимать мир таким, каким он открывается нам в Библии, научиться жить в этом библейском мире; и нет лучшего способа научиться этому, как именно через церковное богослужение, которое не только передает нам библейское учение, но и открывает нам библейский образ жизни» (Протопр. Александр Шмеман, «Великий Пост», стр. 97).

Итак, в Великом Каноне перед нами проходит в лицах и событиях вся ветхозаветная и новозаветная история. Автор указывает на грехопадение прародителей и растление первобытного мира, на добродетели Ноя и нераскаянность и ожесточение жителей Содома и Гоморры, воскрешает перед нами память благочестивых патриархов и доблестных мужей: Моисея, Иисуса Навина, Гедеона и Иефая, представляет нашему взору благочестие царя Давида, его падение и умиленное покаяние, указывает на нечестие Ахава и Иезавели и на великие образцы покаяния "--- неневитян, Манассию, блудницу и благоразумного разбойника, и в особенности на Марию Египетскую, неоднократно останавливает читателя у Креста и Гроба Господня "--- везде поучая покаянию, смирению, молитве, самоотвержению. На этих примерах постоянно происходит увещание души "--- вспомни этого праведника, он так угодил Богу, вспомни и этого праведника, он так угодил "--- ты ничего подобного не сделала.

Об одних перснонажах Библии говорится в положительном смысле, о других в отрицательном, кому-то нужно подражать, а кому-то нет.

Колесничник Илия, колесницею добродетелей вшед, яко на Небеса, ношашеся превыше иногда от земных. Сего убо, душе моя, восход помышляй "--- помышляй, душа моя, о восхождении ветхозаветных праведников.

Гиезиев подражала еси, окаянная, разум скверный всегда, душе, егоже сребролюбие отложи поне на старость, бегай геенскаго огня, отступивши злых твоих "--- хотя бы в старости отринь сребролюбие Гиезии, душа, и оставив свои злодеяния, избегай геенского огня.

Как видите, тексты довольно трудные, поэтому к восприятию Великого канона необходимо заранее готовиться.

В заключительной песни первого дня после всех воспоминаний следуют тропари удивительной силы:

Закон изнеможе, празднует Евангелие, писание же все в тебе небрежено бысть, пророцы изнемогоша, и все праведное слово: струпи твои, о душе, умножишася, не сущу врачу исцеляющему тя "--- нечего вспоминать из Ветхого Завета, все бесполезно. Буду приводить тебе примеры из Нового Завета, может быть, тогда ты покаешься:

Новаго привожду ти писания указания, вводящая тя, душе, ко умилению: праведным убо поревнуй, грешных же отвращайся, и умилостиви Христа молитвами же и пощеньми и чистотою и говением.

Наконец, духовный писатель, представив все ветхозаветное, восходит к Жизнодавцу, Спасителю душ наших, восклицая, как разбойник: Помяни мя!, взывая, как мытарь: Боже милостив буди мне грешному!, подражая в неотступности Хананеянке и слепцам на распутии: Помилуй мя, сыне Давидов!, источая слезы, вместо мира, на главу и ноги Христа, подобно блуднице, и горько плача над собою, как Марфа и Мария над Лазарем.

Далее в Каноне подчеркивается, что самые страшные грешники покаялись и придут в Царство Небесное прежде нас: Христос вочеловечися, призвав к покаянию разбойники и блудницы: душе, покайся, дверь отверзеся Царствия уже, и предвосхищают о фарисее и мытарии и прелюбодеи кающиися.

Когда же, в некоем духовном ужасе, следуя издали за чудесами Спасителя и умиляясь над каждым подвигом Его земной жизни, автор Канона доходит до страшного заколения Христа,"--- силы сердца его оскудевает и, вместе со всей тварью, он умолкает на трепещущей Голгофе, в последний раз воскликнув: Судие мой и ведче мой, хотяй паки приити со ангелы, судити миру всему, милостивным твоим оком тогда, видев мя пощади и ущедри мя, Иисусе, паче всякого естества человеча согрешивша.

Великий канон, всеми средствами подвигая нас к покаянию, в последних тропарях как бы открывает нам свою «методику»: как я с тобой беседовал, душа, и праведников ветхозаветных тебе напоминал, и новозаветные образы тебе в пример приводил, и все напрасно: ихже не поревновала еси, душе, ни деянием, ни житию: но горе тебе, внегда будеши судитися "--- горе тебе, когда предстанешь на суд!

Вслушиваясь в слова Великого Канона, всматриваясь в историю жизни людей, бежавших от Бога, но настигнутых Им, людей, которые оказывались в безднах, но которых Бог выводил оттуда, подумаем о том, как Бог выводит каждого из нас из бездны греха и отчаяния для того, чтобы мы принесли Ему плоды покаяния.

Не нужно думать, что покаяние заключается в том, чтобы копаться в собственных грехах, заниматься самобичеванием, стараться выявить в себе как можно больше злого и темного. Истинное покаяние "--- это когда мы обращаемся от тьмы к свету, от греха к праведности; когда мы понимаем, что прежняя наша жизнь была недостойна высокого призвания, когда перед лицом Бога мы сознаем, как ничтожны мы сами, и сознаем, что единственная наша надежда "--- Сам Бог. Истинное покаяние "--- это когда перед лицом Бога, по слову Апостола Петра «призвавшего нас из тьмы в чудный Свой свет» (\mbox{1 Пет. 2:9}), мы понимаем, что жизнь дана нам для того, чтобы стать детьми Божиими, чтобы приобщиться к Божественному свету. Истинное покаяние "--- то, которое выражается не столько в словах, но и в делах, в готовности прийти на помощь людям, в открытости по отношению к ближним, а не в обращенности на себя. Истинное покаяние "--- это когда мы понимаем, что, хотя и не в наших силах стать настоящими христианами, Бог в силах сделать нас таковыми. Как говорится в Великом Каноне, идеже бо хощет Бог, побеждается естества чин. Другими словами там, где Бог хочет, происходят сверхъестественные события: Савл становится Павлом, Иона изводится из чрева кита, Моисей проходит через море по суше, умерший Лазарь воскресает, Мария Египетская из блудницы превращается в великую праведницу. Ибо, по словам Спасителя, «человекам это невозможно, Богу же все возможно» (\mbox{Мф. 19:26}).

©Протоиерей Виктор Потапов

\url{http://www.stjohndc.org/Russian/feasts/fasts/grlent/r_post_akritsk.htm}

февраль 2001 г.

\end{mymulticols}

\mychapterending

\noparindent

\mychapter{В понедельник первой седмицы Великого Поста}
%http://www.molitvoslov.com/text571.htm 
 
\begin{Parallel}{}{}

\mysubsection{Славянский текст с русским переводом и ссылками на текст Библии}

\mysubsubsection{Песнь 1}

\pripevc{\myemph{Ирм\'{о}с:}}

\minicolumns{\firstletter{П}омощник и Покровитель бысть мне во спасение, Сей мой Бог, и прославлю Его, Бог Отца моего, и вознесу Его: славно бо прославися.}{Помощник и Покровитель явился мне ко спасению, Он Бог мой, и прославлю Его, Бога отца моего, и превознесу Его, ибо Он торжественно прославился.
\myemph{\footnotesize \mbox{Исх. 15:1--2}}}

\pripevpomiluj

\minicolumns{\firstletter{О}ткуду начну плакати окаяннаго моего жития деяний? кое ли положу начало, Христе, нынешнему рыданию? но, яко благоутробен, даждь ми прегрешений оставление.}{С чего начну я оплакивать деяния злосчастной моей жизни? Какое начало положу, Христе, я нынешнему моему сетованию? Но Ты, как милосердный, даруй мне оставление прегрешений.}

\pripevpomiluj

\minicolumns{\firstletter{Г}ряди, окаянная душе, с плотию твоею, Зиждителю всех исповеждься и останися прочее преждняго безсловесия, и принеси Богу в покаянии слезы.}{Прииди, несчастная душа, с плотию своею, исповедайся Создателю всего, воздержись, наконец, от прежнего безрассудства и с раскаянием принеси Богу слезы.}

\pripevpomiluj

\minicolumns{\firstletter{П}ервозданнаго Адама преступлению поревновав, познах себе обнажена от Бога и присносущнаго Царствия и сладости, грех ради моих.}{Подражая в преступлении первозданному Адаму, я сознаю себя лишенным Бога, вечного Царства и блаженства за мои грехи.
\myemph{\footnotesize \mbox{Быт. 3:6--7}}}

\pripevpomiluj

\minicolumns{\firstletter{У}вы мне, окаянная душе, что уподобилася еси первей Еве? видела бо еси зле, и уязвилася еси горце, и коснулася еси древа, и вкусила еси дерзостно безсловесныя снеди.}{Горе мне, моя несчастная душа, для чего ты уподобилась первосозданной Еве? Не с добром ты посмотрела и уязвилась жестоко, прикоснулась к дереву и дерзостно вкусила бессмысленного плода.
\myemph{\footnotesize \mbox{Быт. 3:6}}}

\pripevpomiluj

\minicolumns{\firstletter{В}место Евы чувственныя мысленная ми бысть Ева, во плоти страстный помысл, показуяй сладкая и вкушаяй присно горькаго напоения.}{Вместо чувственной Евы восстала во мне Ева мысленная "--- плотский страстный помысел, обольщающий приятным, но при вкушении всегда напояющий горечью.}

\pripevpomiluj

\minicolumns{\firstletter{Д}остойно из Едема изгнан бысть, яко не сохранив едину Твою, Спасе, заповедь Адам: аз же что постражду, отметая всегда животная Твоя словеса.}{Достойно был изгнан из Едема Адам, как не сохранивший одной Твоей заповеди, Спаситель. Что же должен претерпеть я, всегда отвергающий Твои животворные повеления?}

\slavac

\minicolumns{\firstletter{П}ресущная Тр\'{о}ице, во Ед\'{и}нице покланяемая, возьми бремя от мене тяжкое греховное и, яко благоутробна, даждь ми слезы умиления.}{Пресущественная Троица, Которой мы поклоняемся в одном Существе, сними с меня тяжкое бремя греховное и по Своему милосердию, даруй мне слезы умиления.}

\inynec

\minicolumns{\firstletter{Б}огородице, Надежде и Предстательство Тебе поющих, возьми бремя от мене тяжкое греховное, и, яко Владычица Чистая, кающася приими мя.}{Богородице, надежда и защита воспевающих Тебя, сними с меня тяжкое бремя греховное и, как Владычица Чистая, приими меня кающегося.}

\mysubsubsection{Песнь 2}

\pripevc{\myemph{Ирм\'{о}с:}}

\minicolumns{\firstletter{В}онми, Небо, и возглаголю, и воспою Христа, от Девы плотию пришедшаго.}{Внемли, небо, я буду возвещать и воспевать Христа, пришедшего во плоти от Девы.}

\pripevpomiluj

\minicolumns{\firstletter{В}онми, Небо, и возглаголю, земле, внушай глас, кающийся к Богу и воспевающий Его.}{Внемли, небо, я буду возвещать: земля, услышь голос, кающийся Богу и прославляющий Его.}

\pripevpomiluj

\minicolumns{\firstletter{В}онми ми, Боже, Спасе мой, милостивым Твоим оком и приими мое теплое исповедание.}{Воззри на меня Боже, Спаситель мой, милостивым Твоим оком и прими мою пламенную исповедь.}

\pripevpomiluj

\minicolumns{\firstletter{С}огреших паче всех человек, един согреших Тебе; но ущедри, яко Бог, Спасе, творение Твое.}{Согрешил я более всех людей, один я согрешил пред Тобою; но, как Бог, сжалься, Спаситель, над Твоим созданием.
\myemph{\footnotesize \mbox{1 Тим. 1:15}}}

\pripevpomiluj

\minicolumns{\firstletter{В}ообразив моих страстей безобразие, любосластными стремленьми погубих ума красоту.}{Отобразив в себе безобразие моих страстей, сластолюбивыми стремлениями исказил я красоту ума.}

\pripevpomiluj

\minicolumns{\firstletter{Б}уря мя злых обдержит, благоутробне Господи; но яко Петру и мне руку простри.}{Буря зла окружает меня, Милосердный Господи, но, как Петру, и мне Ты простри руку.
\myemph{\footnotesize \mbox{Мф. 14:31}}}

\pripevpomiluj

\minicolumns{\firstletter{О}скверних плоти моея ризу и окалях, еже по образу, Спасе, и по подобию.}{Осквернил я одежду плоти моей и очернил в себе, Спаситель, то, что было создано по Твоему образу и подобию.}

\pripevpomiluj

\minicolumns{\firstletter{О}мрачих душевную красоту страстей сластьми и всячески весь ум персть сотворих.}{Помрачил я красоту души страстными удовольствиями и весь ум совершенно превратил в прах.}

\pripevpomiluj

\minicolumns{\firstletter{Р}аздрах ныне одежду мою первую, юже ми изтка Зиждитель из начала, и оттуду лежу наг.}{Разодрал я первую одежду мою, которую вначале соткал мне Создатель, и оттого лежу обнаженным.}

\pripevpomiluj

\minicolumns{\firstletter{О}блекохся в раздранную ризу, юже изтка ми змий советом, и стыждуся.}{Облекся я в разодранную одежду, которую соткал мне змий коварством, и стыжусь.
\myemph{\footnotesize \mbox{Быт. 3:21}}}

\pripevpomiluj

\minicolumns{\firstletter{С}лезы блудницы, Щедре, и аз предлагаю, очисти мя, Спасе, благоутробием Твоим.}{Как блудница, и я проливаю слезы, Милосердный; смилуйся надо мною, Спаситель, по благоснисхождению Твоему.
\myemph{\footnotesize \mbox{Лк. 7:38}}}

\pripevpomiluj

\minicolumns{\firstletter{В}оззрех на садовную красоту и прельстихся умом: и оттуду лежу наг и срамляюся.}{Взглянул я на красоту дерева и прельстился в уме; оттого лежу обнаженным и стыжусь.}

\pripevpomiluj

\minicolumns{\firstletter{Д}елаша на хребте моем вси начальницы страстей, продолжающе на мя беззаконие их.}{На хребте моем пахали все вожди страстей, проводя вдоль по мне беззаконие свое.
\myemph{\footnotesize \mbox{Пс. 128:3}}}

\slavac

\minicolumns{\firstletter{Е}динаго Тя в Триех Лицех, Бога всех пою, Отца и Сына и Духа Святаго.}{Воспеваю Тебя, Единого в трех Лицах, Бога всех, Отца, Сына и Святого Духа.}

\inynec

\minicolumns{\firstletter{П}речистая Богородице Дево, Едина Всепетая, моли прилежно, во еже спастися нам.}{Пречистая Богородица Дева, Ты Одна, всеми воспеваемая, усердно моли о нашем спасении.}

\mysubsubsection{Песнь 3}

\pripevc{\myemph{Ирм\'{о}с:}}

\minicolumns{\firstletter{Н}а недвижимом, Христе, камени заповедей Твоих утверди мое помышление.}{На неподвижном камне заповедей Твоих, Христе, утверди мое помышление.}

\pripevpomiluj

\minicolumns{\firstletter{О}гнь от Господа иногда Господь одождив, землю содомскую прежде попали.}{Пролив дождем огонь от Господа, Господь попалил некогда землю содомлян.
\myemph{\footnotesize \mbox{Быт. 19:24}}}

\pripevpomiluj

\minicolumns{\firstletter{Н}а горе спасайся, душе, якоже Лот оный, и в Сигор угонзай.}{Спасайся на горе, душа, как праведный Лот и спеши укрыться в Сигор.
\myemph{\footnotesize \mbox{Быт. 19:22--23}}}

\pripevpomiluj

\minicolumns{\firstletter{Б}егай запаления, о душе, бегай содомскаго горения, бегай тления Божественнаго пламене.}{Беги, душа, от пламени, беги от горящего Содома, беги от истребления Божественным огнем.}

\pripevpomiluj

\minicolumns{\firstletter{С}огреших Тебе един аз, согреших паче всех, Христе Спасе, да не презриши мене.}{Согрешил я один пред Тобою, согрешил более всех, Христос Спаситель "--- не презирай меня.}

\pripevpomiluj

\minicolumns{\firstletter{Т}ы еси Пастырь добрый, взыщи мене, агнца, и заблуждшаго да не презриши мене.}{Ты "--- Пастырь Добрый, отыщи меня "--- агнца, и не презирай меня, заблудившегося.
\myemph{\footnotesize \mbox{Ин. 10:11--14}}}

\pripevpomiluj

\minicolumns{\firstletter{Т}ы еси сладкий Иисусе, Ты еси Создателю мой, в Тебе, Спасе, оправдаюся.}{Ты "--- вожделенный Иисус; Ты "--- Создатель мой, Спаситель, Тобою я оправдаюсь.}

\pripevpomiluj

\minicolumns{\firstletter{И}споведаюся Тебе, Спасе, согреших, согреших Ти; но ослаби, остави ми, яко благоутробен.}{Исповедуюсь Тебе, Спаситель; согрешил я, согрешил пред Тобою, но отпусти, прости меня, как Милосердный.}

\slavac

\minicolumns{\firstletter{О} Тр\'{о}ице Ед\'{и}нице Боже, спаси нас от прелести, и искушений, и обстояний.}{О, Троица, Единица, Боже, спаси нас от обольщений, от искушений и опасностей.}

\inynec

\minicolumns{\firstletter{Р}адуйся, Богоприятная утробо, радуйся, престоле Господень, радуйся, Мати Жизни нашея.}{Радуйся, чрево, вместившее Бога; радуйся, Престол Господень; радуйся, Матерь Жизни нашей.}

\mysubsubsection{Песнь 4}

\pripevc{\myemph{Ирм\'{о}с:}}

\minicolumns{\firstletter{У}слыша пророк пришествие Твое, Господи, и убояся, яко хощеши от Девы родитися и человеком явитися, и глаголаше: услышах слух Твой и убояхся, слава силе Твоей, Господи.}{Услышал пророк о пришествии Твоем, Господи, и устрашился, что Тебе угодно родиться от Девы и явиться людям, и сказал: услышал я весть о Тебе и устрашился; слава силе Твоей, Господи.}

\pripevpomiluj

\minicolumns{\firstletter{Д}ел Твоих да не презриши, создания Твоего да не оставиши, Правосуде. Аще и един согреших, яко человек, паче всякаго человека, Человеколюбче; но имаши, яко Господь всех, власть оставляти грехи.}{Не презри творений Твоих, не оставь создания Твоего, Праведный Судия, ибо хотя я, как человек, один согрешил более всякого человека, но Ты, Человеколюбец, как Господь всего мира, имеешь власть отпускать грехи.
\myemph{\footnotesize \mbox{Мф. 9:6}; \mbox{Мк. 2:10}}}

\pripevpomiluj

\minicolumns{\firstletter{П}риближается, душе, конец, приближается, и нерадиши, ни готовишися, время сокращается: востани, близ при дверех Судия есть. Яко соние, яко цвет, время жития течет: что всуе мятемся?}{Конец приближается, душа, приближается, и ты не заботишься, не готовишься; время сокращается "--- восстань: Судия уже близко "--- при дверях; время жизни проходит, как сновиденье, как цвет. Для чего мы напрасно суетимся?
\myemph{\footnotesize \mbox{Мф. 24:33}; \mbox{Мк. 13:29}; \mbox{Лк. 21:31}}}

\pripevpomiluj

\minicolumns{\firstletter{В}оспряни, о душе моя, деяния твоя, яже соделала еси, помышляй, и сия пред лице твое принеси, и капли испусти слез твоих; рцы со дерзновением деяния и помышления Христу и оправдайся.}{Пробудись, душа моя, размысли о делах своих, которые ты сделала, представь их пред своими очами, и пролей капли слез твоих, безбоязненно открой Христу дела и помышления твои и оправдайся.}

\pripevpomiluj

\minicolumns{\firstletter{Н}е бысть в житии греха, ни деяния, ни злобы, еяже аз, Спасе, не согреших, умом, и словом, и произволением, и предложением, и мыслию, и деянием согрешив, яко ин никтоже когда.}{Нет в жизни ни греха, ни деяния, ни зла, в которых я не был бы виновен, Спаситель, умом, и словом, и произволением, согрешив и намерением, и мыслью, и делом так, как никто другой никогда.}

\pripevpomiluj

\minicolumns{\firstletter{О}тсюду и осужден бых, отсюду препрен бых аз, окаянный, от своея совести, еяже ничтоже в мире нужнейше: Судие, Избавителю мой и Ведче, пощади, и избави, и спаси мя, раба Твоего.}{Потому и обвиняюсь, потому и осуждаюсь я, несчастный, своею совестью, строже которой нет ничего в мире; Судия, Искупитель мой и Испытатель, пощади, избавь и спаси меня, раба Твоего.}

\pripevpomiluj

\minicolumns{\firstletter{Л}ествица, юже виде древле великий в патриарсех, указание есть, душе моя, деятельнаго восхождения, разумнаго возшествия: аще хощеши убо деянием, и разумом, и зрением пожити, обновися.}{Лестница, которую в древности видел великий из патриархов, служит указанием, душа моя, на восхождение делами, на возвышение разумом; поэтому, если хочешь жить в деятельности и в разумении и созерцании, то обновляйся.
\myemph{\footnotesize \mbox{Быт. 28:12}}}

\pripevpomiluj

\minicolumns{\firstletter{З}ной дневный претерпе лишения ради патриарх и мраз нощный понесе, на всяк день снабдения творя, пасый, труждаяйся, работаяй, да две жене сочетает.}{Патриарх по нужде терпел дневной зной и переносил ночной холод, ежедневно сокращая время, пася стада, трудясь и служа, чтобы получить себе две жены.
\myemph{\footnotesize \mbox{Быт. 31:7}, \mbox{Быт. 31:40}}}

\pripevpomiluj

\minicolumns{\firstletter{Ж}ены ми две разумей, деяние же и разум в зрении, Лию убо деяние, яко многочадную, Рахиль же разум, яко многотрудную; ибо кроме трудов ни деяние, ни зрение, душе, исправится.}{Под двумя женами понимай деятельность и разумение в созерцании: под Лиею, как многочадною, "--- деятельность, а под Рахилью, как полученной через многие труды, "--- разумение, ибо без трудов, душа, ни деятельность, ни созерцание не усовершенствуются.}

\slavac

\minicolumns{\firstletter{Н}ераздельное Существом, Неслитное Лицы богословлю Тя, Троическое Едино Божество, яко Единоцарственное и Сопрестольное, вопию Ти песнь великую, в вышних трегубо песнословимую.}{Нераздельным по существу, неслиянным по Лицам богословски исповедую Тебя, Троичное Единое Божество, Соцарственное и Сопрестольное; возглашаю Тебе великую песнь, в небесных обителях троекратно воспеваемую.
\myemph{\footnotesize \mbox{Ис. 6:1--3}}}

\inynec

\minicolumns{\firstletter{И} раждаеши, и девствуеши, и пребываеши обоюду естеством Дева, Рождейся обновляет законы естества, утроба же раждает нераждающая. Бог идеже хощет, побеждается естества чин: творит бо, елика хощет.}{И рождаешь Ты, и остаешься Девою, в обоих случаях сохраняя по естеству девство. Рожденный Тобою обновляет закон природы, а девственное чрево рождает; когда пожелает Бог, то нарушается порядок природы, ибо Он творит, что хочет.}

\mysubsubsection{Песнь 5}

\pripevc{\myemph{Ирм\'{о}с:}}

\minicolumns{\firstletter{О}т нощи утренююща, Человеколюбче, просвети, молюся, и настави и мене на повеления Твоя, и научи мя, Спасе, творити волю Твою.}{От ночи бодрствующего, просвети меня, молю, Человеколюбец, путеводи меня в повелениях Твоих и научи меня, Спаситель, исполнять Твою волю.
\myemph{\footnotesize \mbox{Пс. 62:2}; \mbox{Пс. 118:35}}}

\pripevpomiluj

\minicolumns{\firstletter{В} нощи житие мое преидох присно, тьма бо бысть, и глубока мне мгла, нощь греха, но яко дне сына, Спасе, покажи мя.}{Жизнь свою я постоянно проводил в ночи, ибо мраком и глубокою мглою была для меня ночь греха; но покажи меня сыном дня, Спаситель.
\myemph{\footnotesize \mbox{Еф. 5:8}}}

\pripevpomiluj

\minicolumns{\firstletter{Р}увима подражая, окаянный аз, содеях беззаконный и законопреступный совет на Бога Вышняго, осквернив ложе мое, яко отчее он.}{Подобно Рувиму я, несчастный, совершил преступное и беззаконное дело пред Всевышним Богом, осквернив ложе мое, как тот "--- отчее.
\myemph{\footnotesize \mbox{Быт. 35:22}; \mbox{Быт. 49:3--4}}}

\pripevpomiluj

\minicolumns{\firstletter{И}споведаюся Тебе, Христе Царю: согреших, согреших, яко прежде Иосифа братия продавшии, чистоты плод и целомудрия.}{Исповедаюсь Тебе, Христос-Царь: согрешил я, согрешил, как некогда братья, продавшие Иосифа, "--- плод чистоты и целомудрия.
\myemph{\footnotesize \mbox{Быт. 37:28}}}

\pripevpomiluj

\minicolumns{\firstletter{О}т сродников праведная душа связася, продася в работу сладкий, во образ Господень: ты же вся, душе, продалася еси злыми твоими.}{Сродниками предана была душа праведная; возлюбленный продан в рабство, прообразуя Господа; ты же, душа, сама всю продала себя своим порокам.}

\pripevpomiluj

\minicolumns{\firstletter{И}осифа праведнаго и целомудреннаго ума подражай, окаянная и неискусная душе, и не оскверняйся безсловесными стремленьми, присно беззаконнующи.}{Подражай праведному Иосифу и уму его целомудренному, несчастная и невоздержанная душа, не оскверняйся и не беззаконствуй всегда безрассудными стремлениями.}

\pripevpomiluj

\minicolumns{\firstletter{А}ще и в рове поживе иногда Иосиф, Владыко Господи, но во образ погребения и востания Твоего: аз же что Тебе когда сицевое принесу?}{Владыко Господи, Иосиф был некогда во рву, но в прообраз Твоего погребения и воскресения; принесу ли когда-либо что подобное Тебе я?}

\slavac

\minicolumns{\firstletter{Т}я, Тр\'{о}ице, славим Единаго Бога: Свят, Свят, Свят еси, Отче, Сыне и Душе, Пр\'{о}стое Существо, Ед\'{и}нице присно покланяемая.}{Тебя, Пресвятая Троица, прославляем за Единого Бога: Свят, Свят, Свят Ты, Отец, Сын и Дух "--- простая Сущность, Единица вечно поклоняемая.}

\inynec

\minicolumns{\firstletter{И}з Тебе облечеся в мое смешение, нетленная, безмужная Мати Дево, Бог, создавый веки, и соедини Себе человеческое естество.}{В Тебе, нетленная, не познавшая мужа Матерь-Дево, сотворивший мир Бог облекся в мой состав и соединил с Собою человеческую природу.}

\mysubsubsection{Песнь 6}

\pripevc{\myemph{Ирм\'{о}с:}}

\minicolumns{\firstletter{В}озопих всем сердцем моим к щедрому Богу, и услыша мя от ада преисподняго, и возведе от тли живот мой.}{От всего сердца моего я воззвал к милосердному Богу, и Он услышал меня из ада преисподнего и воззвал жизнь мою от погибели.}

\pripevpomiluj

\minicolumns{\firstletter{С}лезы, Спасе, очию моею и из глубины воздыхания чисте приношу, вопиющу сердцу: Боже, согреших Ти, очисти мя.}{Искренно приношу Тебе, Спаситель, слезы очей моих и воздыхания из глубины сердца, взывающего: Боже, согрешил я пред Тобою, умилосердись надо мною.}

\pripevpomiluj

\minicolumns{\firstletter{У}клонилася еси, душе, от Господа твоего, якоже Дафан и Авирон, но пощади, воззови из ада преисподняго, да не пропасть земная тебе покрыет.}{Уклонилась ты, душа, от Господа своего, как Дафан и Авирон, но воззови из ада преисподнего: пощади!, чтобы пропасть земная не поглотила тебя.
\myemph{\footnotesize \mbox{Чис. 16:32}}}

\pripevpomiluj

\minicolumns{\firstletter{Я}ко юница, душе, разсвирепевшая, уподобилася еси Ефрему, яко серна от тенет сохрани житие, вперивши деянием ум и зрением.}{Рассвирепев, как телица, ты, душа, уподобилась Ефрему, но как серна спасай от тенет свою жизнь, окрылив ум деятельностью и созерцанием.
\myemph{\footnotesize \mbox{Иер. 31:18}; \mbox{Ос. 10:11}}}

\pripevpomiluj

\minicolumns{\firstletter{Р}ука нас Моисеова да уверит, душе, како может Бог прокаженное житие убелити и очистити, и не отчайся сама себе, аще и прокаженна еси.}{Моисеева рука да убедит нас, душа, как Бог может убелить и очистить прокаженную жизнь, и не отчаивайся сама за себя, хотя ты и поражена проказою.
\myemph{\footnotesize \mbox{Исх. 4:6--7}}}

\slavac

\minicolumns{\firstletter{Т}р\'{о}ица есмь Пр\'{о}ста, Нераздельна, раздельна Личне, и Ед\'{и}ница есмь естеством соединена, Отец глаголет, и Сын, и Божественный Дух.}{Я "--- Троица несоставная, нераздельная, раздельная в лицах, и Единица, соединенная по существу; свидетельствует Отец, Сын и Божественный Дух.}

\inynec

\minicolumns{\firstletter{У}троба Твоя Бога нам роди, воображена по нам: Егоже, яко Создателя всех, моли, Богородице, да молитвами Твоими оправдимся.}{Чрево Твое родило нам Бога, принявшего наш образ; Его, как Создателя всего мира, моли, Богородица, чтобы по молитвам Твоим нам оправдаться.}

\mysubsubsection{Кондак, глас 6:}

\minicolumns{\firstletter{Д}уше моя, душе моя, востани, что спиши? конец приближается, и имаши смутитися: воспряни убо, да пощадит тя Христос Бог, везде сый и вся исполняяй.}{Душа моя, душа моя, восстань, что ты спишь? Конец приближается, и ты смутишься; пробудись же, чтобы пощадил тебя Христос Бог, Вездесущий и все наполняющий.}

\mysubsubsection{Песнь 7}

\pripevc{\myemph{Ирм\'{о}с:}}

\minicolumns{\firstletter{С}огрешихом, беззаконновахом, неправдовахом пред Тобою, ниже соблюдохом, ниже сотворихом, якоже заповедал еси нам; но не предаждь нас до конца, отцев Боже.}{Мы согрешили, жили беззаконно, неправо поступали пред Тобою, не сохранили, не исполнили, что Ты заповедал нам; но не оставь нас до конца, Боже отцов.
\myemph{\footnotesize \mbox{Дан. 9:5--6}}}

\pripevpomiluj

\minicolumns{\firstletter{С}огреших, беззаконновах и отвергох заповедь Твою, яко во гресех произведохся, и приложих язвам струпы себе; но Сам мя помилуй, яко благоутробен, отцев Боже.}{Я согрешил, жил в беззакониях и нарушил заповедь Твою, ибо я рожден в грехах и к язвам своим приложил еще раны, но Сам Ты помилуй меня, как Милосердный Боже отцов.}

\pripevpomiluj

\minicolumns{\firstletter{Т}айная сердца моего исповедах Тебе, Судии моему, виждь мое смирение, виждь и скорбь мою, и вонми суду моему ныне, и Сам мя помилуй, яко благоутробен, отцев Боже.}{Тайны сердца моего я открыл пред Тобою, Судьей моим; воззри на смирение мое, воззри и на скорбь мою, обрати внимание на мое ныне осуждение и Сам помилуй меня, как Милосердный, Боже отцов.
\myemph{\footnotesize \mbox{Пс. 37:19}; \mbox{Пс. 24:18}; \mbox{Пс. 34:23}}}

\pripevpomiluj

\minicolumns{\firstletter{С}аул иногда, яко погуби отца своего, душе, ослята, внезапу царство обрете к прослутию; но блюди, не забывай себе, скотския похоти твоя произволивши паче Царства Христова.}{Саул, некогда потеряв ослиц своего отца, неожиданно с известием о них получил царство; душа, не забывайся, предпочитая свои скотские стремления Христову Царству.
\myemph{\footnotesize \mbox{1 Цар. 9:1--27}; \mbox{1 Цар. 10:1}}}

\pripevpomiluj

\minicolumns{\firstletter{Д}авид иногда Богоотец, аще и согреши сугубо, душе моя, стрелою убо устрелен быв прелюбодейства, копием же пленен быв убийства томлением; но ты сама тяжчайшими делы недугуеши, самохотными стремленьми.}{Если богоотец Давид некогда и вдвойне согрешил, будучи уязвлен стрелою прелюбодеяния, сражен был копьем мщения за убийства; но ты, душа моя, сама страдаешь более тяжко, нежели этими делами, произвольными стремлениями.
\myemph{\footnotesize \mbox{2 Цар. 11:14--15}}}

\pripevpomiluj

\minicolumns{\firstletter{С}овокупи убо Давид иногда беззаконию беззаконие, убийству же любодейство растворив, покаяние сугубое показа абие; но сама ты, лукавнейшая душе, соделала еси, не покаявшися Богу.}{Давид некогда присовокупил беззаконие к беззаконию, ибо с убийством соединил прелюбодеяние, но скоро принес и усиленное покаяние, а ты, коварнейшая душа, совершив бОльшие грехи, не раскаялась пред Богом.}

\pripevpomiluj

\minicolumns{\firstletter{Д}авид иногда вообрази, списав яко на иконе песнь, еюже деяние обличает, еже содея, зовый: помилуй мя, Тебе бо единому согреших всех Богу, Сам очисти мя.}{Давид некогда, изображая как бы на картине, начертал песнь, которой обличает совершенный им проступок, взывая: помилуй мя, ибо согрешил я пред Тобою, Одним, Богом всех; Сам очисти меня.
\myemph{\footnotesize \mbox{Пс. 50:3--6}}}

\slavac

\minicolumns{\firstletter{Т}р\'{о}ице Пр\'{о}стая, Нераздельная, Единосущная и Естество Едино, Светове и Свет, и Свята Три, и Едино Свято поется Бог Тр\'{о}ица; но воспой, прослави Живот и Животы, душе, всех Бога.}{Троица простая, нераздельная, единосущная, и одно Божество, Светы и Свет, три Святы и одно лицо Свято, Бог-Троица, воспеваемая в песнопениях; воспой же и ты, душа, прославь Жизнь и Жизни "--- Бога всех.}

\inynec

\minicolumns{\firstletter{П}оем Тя, благословим Тя, покланяемся Ти, Богородительнице, яко Нераздельныя Тр\'{о}ицы породила еси Единаго Христа Бога, и Сама отверзла еси нам, сущим на земли, Небесная.}{Воспеваем Тебя, благословляем Тебя, поклоняемся Тебе, Богородительница, ибо Ты родила Одного из Нераздельной Троицы, Христа Бога, и Сама открыла для нас, живущих на земле, небесные обители.}

\mysubsubsection{Песнь 8}

\pripevc{\myemph{Ирм\'{о}с:}}

\minicolumns{\firstletter{Е}гоже воинства Небесная славят, и трепещут херувими и серафими, всяко дыхание и тварь, пойте, благословите и превозносите во вся веки.}{Кого прославляют воинства небесные и пред Кем трепещут Херувимы и Серафимы, Того, все существа и творения, воспевайте, благословляйте и превозносите во все века.}

\pripevpomiluj

\minicolumns{\firstletter{С}огрешивша, Спасе, помилуй, воздвигни мой ум ко обращению, приими мя кающагося, ущедри вопиюща: согреших Ти, спаси, беззаконновах, помилуй мя.}{Помилуй меня, грешника, Спаситель, пробуди мой ум к обращению, приими кающегося, умилосердись над взывающим: я согрешил пред Тобою, спаси; я жил в беззакониях, помилуй меня.}

\pripevpomiluj

\minicolumns{\firstletter{К}олесничник Илия колесницею добродетелей вшед, яко на небеса, ношашеся превыше иногда от земных: сего убо, душе моя, восход помышляй.}{Везомый на колеснице Илия, взойдя на колесницу добродетелей, некогда вознесся как бы на небеса, превыше всего земного; помышляй, душа моя, об его восходе.
\myemph{\footnotesize \mbox{4 Цар. 2:11}}}

\pripevpomiluj

\minicolumns{\firstletter{Е}лиссей иногда прием милоть Илиину, прият сугубую благодать от Бога; ты же, о душе моя, сея не причастилася еси благодати за невоздержание.}{Некогда Елисей, приняв милоть (плащ) Илии, получил сугубую благодать от Господа; но ты, душа моя, не получила этой благодати за невоздержание.
\myemph{\footnotesize \mbox{4 Цар. 2:9},4 Цар. 12--13}}

\pripevpomiluj

\minicolumns{\firstletter{И}орданова струя первее милотию Илииною Елиссеем ста сюду и сюду; ты же, о душе моя, сея не причастилася еси благодати за невоздержание.}{Елисея милотию Илии некогда разделил поток Иордана на ту и другую сторону; но ты, душа моя, не получила этой благодати за невоздержание.
\myemph{\footnotesize \mbox{4 Цар. 2:14}}}

\pripevpomiluj

\minicolumns{\firstletter{С}оманитида иногда праведнаго учреди, о душе, нравом благим; ты же не ввела еси в дом ни странна, ни путника. Темже чертога изринешися вон, рыдающи.}{Соманитянка некогда угостила праведника с добрым усердием; а ты, душа, не приняла в свой дом ни странника, ни пришельца; за то будешь извержена вон из брачного чертога с рыданием.
\myemph{\footnotesize \mbox{4 Цар. 4:8}}}

\pripevpomiluj

\minicolumns{\firstletter{Г}иезиев подражала еси, окаянная, разум скверный всегда, душе, егоже сребролюбие отложи поне на старость; бегай геенскаго огня, отступивши злых твоих.}{Ты, несчастная душа, непрестанно подражала нечистому нраву Гиезия; хотя в старости отвергни его сребролюбие и, оставив свои злодеяния, избегни огня геенского.
\myemph{\footnotesize \mbox{4 Цар. 5:20--27}}}

\slavac

\minicolumns{\firstletter{Б}езначальне Отче, Сыне Собезначальне, Утешителю Благий, Душе Правый, Слова Божия Родителю, Отца Безначальна Слове, Душе Живый и Зиждяй, Тр\'{о}ице Ед\'{и}нице, помилуй мя.}{Безначальный Отче, Собезначальный Сын, Утешитель Благий, Дух Правый, Родитель Слова Божия, Безначальное Слово Отца, Дух, Животворящий и Созидающий, Троица Единая, помилуй меня.}

\inynec

\minicolumns{\firstletter{Я}ко от оброщения червленицы, Пречистая, умная багряница Еммануилева внутрь во чреве Твоем плоть исткася. Темже Богородицу воистинну Тя почитаем.}{Мысленная порфира "--- плоть Еммануила соткалась внутри Твоего чрева, Пречистая, как бы из вещества пурпурного; потому мы почитаем Тебя, Истинную Богородицу.}

\mysubsubsection{Песнь 9}

\pripevc{\myemph{Ирм\'{о}с:}}

\minicolumns{\firstletter{Б}езсеменнаго зачатия Рождество несказанное, Матере безмужныя нетленен Плод, Божие бо Рождение обновляет естества. Темже Тя вси роди, яко Богоневестную Матерь, православно величаем.}{Рождество от бессеменного зачатия неизъяснимо, безмужной Матери нетленен Плод, ибо рождение Бога обновляет природу. Поэтому Тебя, как Богоневесту-Матерь мы, все роды, православно величаем.}

\pripevpomiluj

\minicolumns{\firstletter{У}м острупися, тело оболезнися, недугует дух, слово изнеможе, житие умертвися, конец при дверех. Темже, моя окаянная душе, что сотвориши, егда приидет Судия испытати твоя.}{Ум изранился, тело расслабилось, дух болезнует, слово потеряло силу, жизнь замерла, конец при дверях. Что же сделаешь ты, несчастная душа, когда придет Судия исследовать дела твои?}

\pripevpomiluj

\minicolumns{\firstletter{М}оисеово приведох ти, душе, миробытие, и от того все заветное Писание, поведающее тебе праведныя и неправедныя: от нихже вторыя, о душе, подражала еси, а не первыя, в Бога согрешивши.}{Я воспроизвел пред тобою, душа, сказание Моисея о бытии мира и затем все Заветное Писание, повествующее о праведных и неправедных; из них ты, душа, подражала последним, а не первым, согрешая пред Богом.}

\pripevpomiluj

\minicolumns{\firstletter{З}акон изнеможе, празднует Евангелие, Писание же все в тебе небрежено бысть, пророцы изнемогоша и все праведное слово; струпи твои, о душе, умножишася, не сущу врачу, исцеляющему тя.}{Ослабел закон, не воздействует Евангелие, пренебрежено все Писание тобою, пророки и всякое слово праведника потеряли силу; язвы твои, душа, умножились, без Врача, исцеляющего тебя.}

\pripevpomiluj

\minicolumns{\firstletter{Н}оваго привожду ти Писания указания, вводящая тя, душе, ко умилению: праведным убо поревнуй, грешных же отвращайся и умилостиви Христа молитвами же, и пощеньми, и чистотою, и говением.}{Из Новозаветного Писания привожу тебе примеры, душа, возбуждающие в тебе умиление; так подражай праведным и отвращайся примера грешных и умилостивляй Христа молитвою, постом, чистотою и непорочностью.}

\pripevpomiluj

\minicolumns{\firstletter{Х}ристос вочеловечися, призвав к покаянию разбойники и блудницы; душе, покайся, дверь отверзеся Царствия уже, и предвосхищают е фарисее и мытари и прелюбодеи кающиися.}{Христос, сделавшись человеком, призвал к покаянию разбойников и блудниц; покайся, душа, дверь Царства уже открылась, и прежде тебя входят в нее кающиеся фарисеи, мытари и прелюбодеи.
\myemph{\footnotesize \mbox{Мф. 11:12}; \mbox{Мф. 21:31}; \mbox{Лк. 16:16}}}

\pripevpomiluj

\minicolumns{\firstletter{Х}ристос вочеловечися, плоти приобщився ми, и вся елика суть естества хотением исполни греха кроме, подобие тебе, о душе, и образ предпоказуя Своего снисхождения.}{Христос, сделался человеком, приобщившись ко мне плотию, и добровольно испытал все, что свойственно природе, за исключением греха, показывая тебе, душа, пример и образец Своего снисхождения.}

\pripevpomiluj

\minicolumns{\firstletter{Х}ристос волхвы спасе, пастыри созва, младенец множества показа мученики, старцы прослави и старыя вдовицы, ихже не поревновала еси, душе, ни деянием, ни житию, но горе тебе, внегда будеши судитися.}{Христос спас волхвов, призвал к Себе пастухов, множество младенцев сделал мучениками, прославил старца и престарелую вдовицу\footnote{Здесь имеются в виду Симеон Богоприимец и Анна-пророчица.}; их деяниям и жизни ты не подражала, душа, но горе тебе, когда будешь судима!
\myemph{\footnotesize \mbox{Мф. 2:1}, 16; \mbox{Лк. 2:4--8} и след.; \mbox{Лк. 2:25--26} и след.; \mbox{Лк. 2:36--38}}}

\pripevpomiluj

\minicolumns{\firstletter{П}остився Господь дний четыредесять в пустыни, последи взалка, показуя человеческое; душе, да не разленишися, аще тебе приложится враг, молитвою же и постом от ног твоих да отразится.}{Господь, постившись сорок дней в пустыне, наконец взалкал, обнаруживая в Себе человеческую природу. Не унывай, душа, если враг устремится на тебя, но да отразится он от ног твоих молитвами и постом.
\myemph{\footnotesize \mbox{Исх. 34:28}; \mbox{Мф. 4:2}; \mbox{Лк. 4:2}; \mbox{Мк. 1:13}}}

\slavac

\minicolumns{\firstletter{О}тца прославим, Сына превознесем, Божественному Духу верно поклонимся, Тр\'{о}ице Нераздельней, Ед\'{и}нице по существу, яко Свету и Светом, и Животу и Животом, животворящему и просвещающему концы.}{Прославим Отца, превознесем Сына, с верою поклонимся Божественному Духу, Нераздельной Троице, Единой по существу, как Свету и Светам, Жизни и Жизням, животворящему и просвещающему пределы вселенной.}

\inynec

\minicolumns{\firstletter{Г}рад Твой сохраняй, Богородительнице Пречистая, в Тебе бо сей верно царствуяй, в Тебе и утверждается, и Тобою побеждаяй, побеждает всякое искушение, и пленяет ратники, и проходит послушание.}{Сохраняй град Свой, Пречистая Богородительница. Под Твоею защитою он царствует с верою, и от Тебя получает крепость, и при Твоем содействии неотразимо побеждает всякое бедствие, берет в плен врагов и держит их в подчинении.}

\pripevmskipc{\pripev{\firstletter{П}реподобне отче Андрее, моли Бога о нас.}}

\minicolumns{\firstletter{А}ндрее честный и отче треблаженнейший, пастырю Критский, не престай моляся о воспевающих тя: да избавимся вси гнева и скорби, и тления, и прегрешений безмерных, чтущии твою память верно.}{Андрей досточтимый, отец преблаженный, пастырь Критский, не переставай молиться за воспевающих тебя, чтобы избавиться от гнева, скорби, погибели и бесчисленных прегрешений нам всем, искренно почитающим память твою.}

\pripevmskipc{\myemph{\firstletter{Т}аже оба лика вкупе поют Ирм\'{о}с:}}

\minicolumns{\firstletter{Б}езсеменнаго зачатия Рождество несказанное, Матере безмужныя нетленен Плод, Божие бо Рождение обновляет естества. Темже Тя вси роди, яко Богоневестную Матерь, православно величаем.}{Рождество от бессеменного зачатия неизъяснимо, безмужной Матери нетленен Плод, ибо рождение Бога обновляет природу. Поэтому Тебя, как Богоневесту-Матерь мы, все роды, православно величаем.}

\end{Parallel}

\mychapterending

\mychapter{Во вторник первой седмицы Великого Поста}
%http://www.molitvoslov.com/text572.htm 

\begin{Parallel}{}{}

\mysubsection{Славянский текст с русским переводом и ссылками на текст Библии}
 
\mysubsubsection{Песнь 1}

\pripevc{\myemph{Ирм\'{о}с:}}

\minicolumns{\firstletter{П}омощник и Покровитель бысть мне во спасение, Сей мой Бог, и прославлю Его, Бог отца моего, и вознесу Его: славно бо прославися.}{
Помощник и Покровитель явился мне ко спасению, Он Бог мой, и прославлю Его, Бога отца моего, и превознесу Его, ибо Он торжественно прославился.
\myemph{\footnotesize \mbox{Исх. 15:1--2}}
}

\pripevpomiluj

\minicolumns{\firstletter{К}аиново прешед убийство, произволением бых убийца совести душевней, оживив плоть и воевав на ню лукавыми моими деяньми.}{
Превзойдя Каиново убийство, сознательным произволением, через оживление греховной плоти, я сделался убийцею души, вооружившись против нее злыми моими делами.
\myemph{\footnotesize \mbox{Быт. 4:8}}
}

\pripevpomiluj

\minicolumns{\firstletter{А}велеве, Иисусе, не уподобихся правде, дара Тебе приятна не принесох когда, ни деяния божественна, ни жертвы чистыя, ни жития непорочнаго.}{
Авелевой праведности, Иисусе, я не подражал, никогда не приносил Тебе приятных даров, ни дел богоугодных, ни жертвы чистой, ни жизни непорочной.
\myemph{\footnotesize \mbox{Быт. 4:4}}
}

\pripevpomiluj

\minicolumns{\firstletter{Я}ко Каин и мы, душе окаянная, всех Содетелю деяния скверная, и жертву порочную, и непотребное житие принесохом вкупе: темже и осудихомся.}{
Как Каин, так и мы, несчастная душа, принесли Создателю всего жертву порочную "--- дела нечестивые и жизнь невоздержанную: поэтому мы и осуждены.
\myemph{\footnotesize \mbox{Быт. 4:3--5}}
}

\pripevpomiluj

\minicolumns{\firstletter{Б}рение Здатель живосоздав, вложил еси мне плоть и кости, и дыхание, и жизнь; но, о Творче мой, Избавителю мой и Судие, кающася приими мя.}{
Оживотворивший земной прах, Скудельник, Ты даровал мне плоть и кости, дыхание и жизнь; но, Творец мой, Искупитель мой и Судия, приими меня кающегося!
\myemph{\footnotesize \mbox{Быт. 2:7}}
}

\pripevpomiluj

\minicolumns{\firstletter{И}звещаю Ти, Спасе, грехи, яже содеях, и души и тела моего язвы, яже внутрь убийственнии помыслы разбойнически на мя возложиша.}{
Пред Тобою, Спаситель, открываю грехи, сделанные мною, и раны души и тела моего, которые разбойнически нанесли мне внутренние убийственные помыслы.
\myemph{\footnotesize \mbox{Лк. 10:30}}
}

\pripevpomiluj

\minicolumns{\firstletter{А}ще и согреших, Спасе, но вем, яко Человеколюбец еси, наказуеши милостивно и милосердствуеши тепле: слезяща зриши и притекаеши, яко отец, призывая блуднаго.}{
Хотя я и согрешил, Спаситель, но знаю, что Ты человеколюбив; наказываешь с состраданием и милуешь с любовью, взираешь на плачущего и спешишь, как Отец, призвать блудного.
\myemph{\footnotesize \mbox{Лк. 15:20}}
}

\slavac

\minicolumns{\firstletter{П}ресущная Тр\'{о}ице, во Ед\'{и}нице покланяемая, возьми бремя от мене тяжкое греховное и, яко благоутробна, даждь ми слезы умиления.}{
Пресущественная Троица, Которой мы поклоняемся как Единому существу, сними с меня тяжелое бремя греховное и, как Милосердная, даруй мне слезы умиления.
}

\inynec

\minicolumns{\firstletter{Б}огородице, Надежде и Предстательство Тебе поющих, возьми бремя от мене тяжкое греховное и, яко Владычица Чистая, кающася приими мя.}{
Богородице, Надежда и Помощь всем воспевающих Тебя, сними с меня тяжелое бремя греховное и, как Владычица Непорочная, прими меня кающегося.
}

\mysubsubsection{Песнь 2}

\pripevc{\myemph{Ирм\'{о}с:}}

\minicolumns{\firstletter{В}онми, Небо, и возглаголю, и воспою Христа, от Девы плотию пришедшаго.}{
Внемли, небо, я буду возвещать и воспевать Христа, пришедшего во плоти от Девы.
}

\pripevpomiluj

\minicolumns{\firstletter{С}шиваше кожныя ризы грех мне, обнаживый мя первыя боготканныя одежды.}{
«Кожаные ризы» сшил мне грех, сняв с меня прежнюю Богом сотканную одежду.
}

\pripevpomiluj

\minicolumns{\firstletter{О}бложен есмь одеянием студа, якоже листвием смоковным, во обличение моих самовластных страстей.}{
Облекся я одеянием стыда, как листьями смоковницы, во обличение самовольных страстей моих.
\myemph{\footnotesize \mbox{Быт. 3:7}}
}

\pripevpomiluj

\minicolumns{\firstletter{О}деяхся в срамную ризу и окровавленную студно течением страстнаго и любосластнаго живота.}{
Оделся я в одежду, постыдно запятнанную и окровавленную нечистотой страстной и сластолюбивой жизни.
}

\pripevpomiluj

\minicolumns{\firstletter{В}падох в страстную пагубу и в вещественную тлю, и оттоле до ныне враг мне досаждает.}{
Подвергся я мучению страстей и вещественному тлению, и оттого ныне враг угнетает меня.
}

\pripevpomiluj

\minicolumns{\firstletter{Л}юбовещное и любоименное житие невоздержанием, Спасе, предпочет ныне, тяжким бременем обложен есмь.}{
Предпочтя нестяжательности жизнь, привязанную к земным вещам и любостяжательную, Спаситель, я теперь нахожусь под тяжким бременем.
}

\pripevpomiluj

\minicolumns{\firstletter{У}красих плотский образ скверных помышлений различным обложением и осуждаюся.}{
Украсил я кумир плоти разноцветным одеянием гнусных помыслов и подвергаюсь осуждению.
}

\pripevpomiluj

\minicolumns{\firstletter{В}нешним прилежно благоукрашением единем попекохся, внутреннюю презрев Богообразную скинию.}{
Усердно заботясь об одном внешнем благолепии, я пренебрег внутренней скинией, устроенной по образу Божию.
}

\pripevpomiluj

\minicolumns{\firstletter{П}огребох перваго образа доброту, Спасе, страстьми, юже, яко иногда драхму, взыскав, обрящи.}{
Засыпал страстями красоту первобытного образа, Спаситель; ее, как некогда драхму, Ты взыщи и найди.
\myemph{\footnotesize \mbox{Лк. 15:8}}
}

\pripevpomiluj

\minicolumns{\firstletter{С}огреших, якоже блудница, вопию Ти: един согреших Тебе, яко миро, приими, Спасе, и моя слезы.}{
Согрешил, и, как блудница, взываю к Тебе: один я согрешил пред Тобою, приими, Спаситель, и от меня слезы вместо мира.
\myemph{\footnotesize \mbox{Лк. 7:37--38}}
}

\pripevpomiluj

\minicolumns{\firstletter{О}чисти, якоже мытарь, вопию Ти, Спасе, очисти мя: никтоже бо сущих из Адама, якоже аз, согреших Тебе.}{
Умилостивись, как мытарь, взываю к Тебе, Спаситель, смилуйся надо мною: ибо как никто из потомков Адамовых я согрешил пред Тобою.
\myemph{\footnotesize \mbox{Лк. 18:13}}
}

\slavac

\minicolumns{\firstletter{Е}динаго Тя в Триех Лицех, Бога всех пою, Отца и Сына и Духа Святаго.}{
Воспеваю Тебя, Одного в трех Лицах Бога всех, Отца, Сына и Святого Духа.
}

\inynec

\minicolumns{\firstletter{П}речистая Богородице Дево, Едина Всепетая, моли прилежно, во еже спастися нам.}{
Пречистая Богородице Дево, Ты Одна всеми воспеваемая, усердно моли о нашем спасении.
}

\mysubsubsection{Песнь 3}

\pripevc{\myemph{Ирм\'{о}с:}}

\minicolumns{\firstletter{У}тверди, Господи, на камени заповедей Твоих подвигшееся сердце мое, яко Един Свят еси и Господь.}{
Утверди, Господи, на камне Твоих заповедей поколебавшееся сердце мое, ибо Ты один свят и Господь.
}

\pripevpomiluj

\minicolumns{\firstletter{И}сточник живота стяжах Тебе, смерти Низложителя, и вопию Ти от сердца моего прежде конца: согреших, очисти и спаси мя.}{
Источник жизни нашел я в Тебе, Разрушитель смерти, и прежде кончины взываю к Тебе от сердца моего: согрешил я, умилостивись, спаси меня.
}

\pripevpomiluj

\minicolumns{\firstletter{С}огреших, Господи, согреших Тебе, очисти мя: несть бо иже кто согреши в человецех, егоже не превзыдох прегрешеньми.}{
Согрешил я, Господи, согрешил пред Тобою, смилуйся надо мною, ибо нет грешника между людьми, которого я не превзошел бы прегрешениями.
}

\pripevpomiluj

\minicolumns{\firstletter{П}ри Нои, Спасе, блудствовавшия подражах, онех наследствовах осуждение в потопе погружения.}{
Я подражал, Спаситель, развращенным современникам Ноя и наследовал осуждение их на потопление в потопе.\\
\myemph{\footnotesize \mbox{Быт. 6:1--17}}
}

\pripevpomiluj

\minicolumns{\firstletter{Х}ама онаго, душе, отцеубийца подражавши, срама не покрыла еси искренняго, вспять зря возвратившися.}{
Подражая отцеубийце Хаму, ты, душа, не прикрыла срамоты ближнего с лицом, обращенным назад.
\myemph{\footnotesize \mbox{Быт. 9:22--23}}
}

\pripevpomiluj

\minicolumns{\firstletter{З}апаления, якоже Лот, бегай, душе моя, греха: бегай Содомы и Гоморры, бегай пламене всякаго безсловеснаго желания.}{
Беги, душа моя, от пламени греха; как Лот; беги от Содома и Гоморры; беги от огня всякого безрассудного пожелания.
\myemph{\footnotesize \mbox{Быт. 19:15--17}}
}

\pripevpomiluj

\minicolumns{\firstletter{П}омилуй, Господи, помилуй мя, вопию Ти, егда приидеши со ангелы Твоими воздати всем по достоянию деяний.}{
Помилуй, Господи, взываю к Тебе, помилуй меня, когда придешь с Ангелами Своими воздать всем по достоинству их дел.
}

\slavac

\minicolumns{\firstletter{Т}р\'{о}ице Пр\'{о}стая, Несозданная, Безначальное Естество, в Тр\'{о}ице певаемая Ипостасей, спаси ны, верою покланяющияся державе Твоей.}{
Троица несоставная, несозданная, Существо Безначальная, в троичности лиц воспеваемая, спаси нас, с верою поклоняющихся силе Твоей.
}

\inynec

\minicolumns{\firstletter{О}т Отца безлетна Сына в лето, Богородительнице, неискусомужно родила еси, странное чудо, пребывши Дева доящи.}{
Ты, Богородительница, не испытавши мужа, во времени родила Сына от Отца вне времени и "--- дивное чудо: питая молоком, пребыла Девою.
}

\mysubsubsection{Песнь 4}

\pripevc{\myemph{Ирм\'{о}с:}}

\minicolumns{\firstletter{У}слыша пророк пришествие Твое, Господи, и убояся, яко хощеши от Девы родитися и человеком явитися, и глаголаше: услышах слух Твой и убояхся, слава силе Твоей, Господи.}{
Услышал пророк о пришествии Твоем, Господи, и устрашился, что Тебе угодно родиться от Девы и явиться людям, и сказал: услышал я весть о Тебе и устрашился; слава силе Твоей, Господи.
}

\pripevpomiluj

\minicolumns{\firstletter{Б}ди, о душе моя, изрядствуй, якоже древле великий в патриарсех, да стяжеши деяние с разумом, да будеши ум, зряй Бога, и достигнеши незаходящий мрак в видении, и будеши великий купец.}{
Бодрствуй, душа моя, будь мужественна, как великий из патриархов, чтобы приобрести себе дело по разуму, чтобы обогатиться умом, видящим Бога, и проникнуть в неприступный мрак в созерцании и получить великое сокровище.
\myemph{\footnotesize \mbox{Быт. 32:28}}
}

\pripevpomiluj

\minicolumns{\firstletter{Д}ванадесять патриархов великий в патриарсех детотворив, тайно утверди тебе лествицу деятельнаго, душе моя, восхождения: дети, яко основания, степени, яко восхождения, премудренно подложив.}{
Великий из патриархов, родив двенадцать патриархов, таинственно представил тебе, душа моя, лестницу деятельного восхождения, премудро расположив детей как ступени, а свои шаги, как восхождения вверх.
}

\pripevpomiluj

\minicolumns{\firstletter{И}сава возненавиденнаго подражала еси, душе, отдала еси прелестнику твоему первыя доброты первенство и отеческия молитвы отпала еси, и дважды поползнулася еси, окаянная, деянием и разумом: темже ныне покайся.}{
Подражая ненавиденному Исаву, душа, ты отдала соблазнителю своему первенство первоначальной красоты и лишилась отеческого благословения и, несчастная, пала дважды, деятельностью и разумением, поэтому ныне покайся.
\myemph{\footnotesize \mbox{Быт. 25:32}; \mbox{Быт. 27:37}; \mbox{Мал. 1:2--3}}
}

\pripevpomiluj

\minicolumns{\firstletter{Е}дом Исав наречеся, крайняго ради женонеистовнаго смешения: невоздержанием бо присно разжигаем и сластьми оскверняем, Едом именовася, еже глаголется разжжение души любогреховныя.}{
Исав был назван Едомом за крайнее пристрастие к женолюбию; он непрестанно разжигаясь невоздержанием и оскверняясь любострастием, назван Едомом, что значит "--- «распаление души грехолюбивой».
\myemph{\footnotesize \mbox{Быт. 25:30}}
}

\pripevpomiluj

\minicolumns{\firstletter{И}ова на гноищи слышавши, о душе моя, оправдавшагося, того мужеству не поревновала еси, твердаго не имела еси предложения во всех, яже веси, и имиже искусилася еси, но явилася еси нетерпелива.}{
Слышав об Иове, сидевшем на гноище, ты, душа моя, не подражала ему в мужестве, не имела твердой воли во всем, что узнала, что видела, что испытала, но оказалась нетерпеливою.
\myemph{\footnotesize \mbox{Иов. 1:1--22}}
}

\pripevpomiluj

\minicolumns{\firstletter{И}же первее на престоле, наг ныне на гноище гноен, многий в чадех и славный, безчаден и бездомок напрасно: палату убо гноище и бисерие струпы вменяше.}{
Бывший прежде на престоле, теперь "--- на гноище, обнаженный и изъязвленный; имевший многих детей и знаменитый, внезапно стал бездетным и бездомным; гноище считал он своим чертогом и язвы "--- драгоценными камнями.
\myemph{\footnotesize \mbox{Иов. 2:11--13}}
}

\slavac

\minicolumns{\firstletter{Н}ераздельное Существом, Неслитное Лицы богословлю Тя, Троическое Едино Божество, яко Единоцарственное и Сопрестольное, вопию Ти песнь великую, в вышних трегубо песнословимую.}{
Нераздельным по существу, неслиянным в Лицах богословски исповедую Тебя, Троичное Единое Божество, Соцарственное и Сопрестольное; возглашаю Тебе великую песнь, в небесных обителях троекратно воспеваемую.
\myemph{\footnotesize \mbox{Ис. 6:1--3}}
}

\inynec

\minicolumns{\firstletter{И} раждаеши, и девствуеши, и пребываеши обоюду естеством Дева, Рождейся обновляет законы естества, утроба же раждает нераждающая. Бог идеже хощет, побеждается естества чин: творит бо, елика хощет.}{
И рождаешь Ты, и остаешься Девою, в обоих случаях сохраняя по естеству девство. Рожденный Тобою обновляет законы природы, а девственное чрево рождает; когда хочет Бог, то нарушается порядок природы, ибо Он творит, что хочет.
}

\mysubsubsection{Песнь 5}

\pripevc{\myemph{Ирм\'{о}с:}}

\minicolumns{\firstletter{О}т нощи утренююща, Человеколюбче, просвети, молюся, и настави и мене на повеления Твоя, и научи мя, Спасе, творити волю Твою.}{
От ночи бодрствующего, просвети меня, молю, Человеколюбец, путеводи меня в повелениях Твоих и научи меня, Спаситель, исполнять Твою волю.
\myemph{\footnotesize \mbox{Пс. 62:2}; \mbox{Пс. 118:35}}
}

\pripevpomiluj

\minicolumns{\firstletter{М}оисеов слышала еси ковчежец, душе, водами, волнами носим речными, яко в чертозе древле бегающий дела, горькаго совета фараонитска.}{
Ты слышала, душа, о корзинке с Моисеем, в древности носимом водами в волнах реки, как в чертоге, избегшем горестного последствия замысла фараонова.
\myemph{\footnotesize \mbox{Исх. 2:3}}
}

\pripevpomiluj

\minicolumns{\firstletter{А}ще бабы слышала еси, убивающия иногда безвозрастное мужеское, душе окаянная, целомудрия деяние, ныне, яко великий Моисей, сси премудрость.}{
Если ты слышала, несчастная душа, о повивальных бабках, некогда умерщвлявших новорожденных младенцев мужского пола, то теперь, подобно Моисею, млекопитайся мудростью.
\myemph{\footnotesize \mbox{Исх. 1:8--22}}
}

\pripevpomiluj

\minicolumns{\firstletter{Я}ко Моисей великий египтянина, ума, уязвивши, окаянная, не убила еси, душе; и како вселишися, глаголи, в пустыню страстей покаянием.}{
Подобно великому Моисею, поразившему египтянина, ты, не умертвила, несчастная душа, гордого ума; как же, скажи, вселишься ты в пустыню от страстей через покаяние?
\myemph{\footnotesize \mbox{Исх. 2:11--12}}
}

\pripevpomiluj

\minicolumns{\firstletter{В} пустыню вселися великий Моисей; гряди убо, подражай того житие, да и в купине Богоявления, душе, в видении будеши.}{
Великий Моисей поселился в пустыне; иди и ты, душа, подражай его жизни, чтобы и тебе увидеть в терновом кусте явление Бога.
\myemph{\footnotesize \mbox{Исх. 3:2--3}}
}

\pripevpomiluj

\minicolumns{\firstletter{М}оисеов жезл воображай, душе, ударяющий море и огустевающий глубину, во образ Креста Божественнаго: имже можеши и ты великая совершити.}{
Изобрази, душа, Моисеев жезл, поражающий море и огустевающий глубину, в знамение Божественного Креста, которым и ты можешь совершить великое.
\myemph{\footnotesize \mbox{Исх. 14:21--22}}
}

\pripevpomiluj

\minicolumns{\firstletter{А}арон приношаше огнь Богу непорочный, нелестный; но Офни и Финеес, яко ты, душе, приношаху чуждее Богу, оскверненное житие.}{
Аарон приносил Богу огонь чистый, беспримесный, но Офни и Финеес принесли, как ты, душа, отчужденную от Бога нечистую жизнь.
\myemph{\footnotesize \mbox{1 Цар. 2:12--13}}
}

\slavac

\minicolumns{\firstletter{Т}я, Тр\'{о}ице, славим Единаго Бога: Свят, Свят, Свят еси, Отче, Сыне и Душе, Пр\'{о}стое Существо, Ед\'{и}нице присно покланяемая.}{
Тебя, Пресвятая Троица, прославляем за Единого Бога: Свят, Свят, Свят Отец, Сын и Дух, Простое Существо, Единица вечно поклоняемая.
}

\inynec

\minicolumns{\firstletter{И}з Тебе облечеся в мое смешение, нетленная, безмужная Мати Дево, Бог, создавый веки, и соедини Себе человеческое естество.}{
В Тебе, Нетленная, не познавшая мужа Матерь-Дево, облекся в мой состав сотворивший мир Бог и соединил с Собою человеческую природу.
}

\mysubsubsection{Песнь 6}

\pripevc{\myemph{Ирм\'{о}с:}}

\minicolumns{\firstletter{В}озопих всем сердцем моим к щедрому Богу, и услыша мя от ада преисподняго, и возведе от тли живот мой.}{
От всего сердца моего я воззвал к милосердному Богу, и Он услышал меня из ада преисподнего и воззвал жизнь мою от погибели.
}

\pripevpomiluj

\minicolumns{\firstletter{В}олны, Спасе, прегрешений моих, яко в мори Чермнем возвращающеся, покрыша мя внезапу, яко египтяны иногда и тристаты.}{
Волны грехов моих, Спаситель, обратившись, как в Чермном море, внезапно покрыли меня, как некогда египтян и их всадников.
\myemph{\footnotesize \mbox{Исх. 14:26--28}; \mbox{Исх. 15:4--5}}
}

\pripevpomiluj

\minicolumns{\firstletter{Н}еразумное, душе, произволение имела еси, яко прежде Израиль: Божественныя бо манны предсудила еси безсловесно любосластное страстей объядение.}{
Нерассудителен твой выбор, душа, как у древнего Израиля, ибо ты безрассудно предпочла Божественной манне сластолюбивое пресыщение страстями.
\myemph{\footnotesize \mbox{Чис. 21:5}}
}

\pripevpomiluj

\minicolumns{\firstletter{К}ладенцы, душе, предпочла еси хананейских мыслей паче жилы камене, из негоже премудрости река, яко чаша, проливает токи богословия.}{
Колодцы хананейских помыслов ты, душа, предпочла камню с источником, из которого река премудрости, как чаша, изливает струи богословия.
\myemph{\footnotesize \mbox{Быт. 21:25}; \mbox{Исх. 17:6}}
}

\pripevpomiluj

\minicolumns{\firstletter{С}виная мяса и котлы и египетскую пищу, паче Небесныя, предсудила еси, душе моя, якоже древле неразумнии людие в пустыни.}{
Свиное мясо, котлы и египетскую пищу ты предпочла пище небесной, душа моя, как древний безрассудный народ в пустыне.
\myemph{\footnotesize \mbox{Исх. 16:3}}
}

\pripevpomiluj

\minicolumns{\firstletter{Я}ко удари Моисей, раб Твой, жезлом камень, образно животворивая ребра Твоя прообразоваше, из нихже вси питие жизни, Спасе, почерпаем.}{
Как Моисей, раб Твой, ударив жезлом о камень, таинственно предызобразил животворное ребро Твое, Спаситель, из которого все мы почерпаем питие жизни.
}

\pripevpomiluj

\minicolumns{\firstletter{И}спытай, душе, и смотряй, якоже Иисус Навин, обетования землю, какова есть, и вселися в ню благозаконием.}{
Исследуй, душа, подобно Иисусу Навину, и обозри обещанную землю, какова она, и поселись в ней путем исполнения закона.
}

\slavac

\minicolumns{\firstletter{Т}р\'{о}ица есмь Пр\'{о}ста, Нераздельна, раздельна Личне и Ед\'{и}ница есмь естеством соединена, Отец глаголет, и Сын, и Божественный Дух.}{
Я "--- Троица Простая, Нераздельная, раздельная в Лицах и Единица, соединенная по существу; свидетельствует Отец, Сын и Божественный Дух.
}

\inynec

\minicolumns{\firstletter{У}троба Твоя Бога нам роди, воображена по нам: Егоже, яко Создателя всех, моли, Богородице, да молитвами Твоими оправдимся.}{
Чрево Твое родило нам Бога, принявшего наш образ; Его, как Создателя всего мира, моли, Богородица, чтобы по молитвам Твоим нам оправдаться.
}

\pripevmskipc{\firstletter{Г}осподи, помилуй. \myemph{Трижды.}}

\pripevmskipc{\slavainynen}

\mysubsubsection{Кондак, глас 6: }

\minicolumns{\firstletter{Д}уше моя, душе моя, востани, что спиши? конец приближается, и имаши смутитися: воспряни убо, да пощадит тя Христос Бог, везде сый и вся исполняяй.}{
Душа моя, душа моя, восстань, что ты спишь? Конец приближается, и ты смутишься; пробудись же, чтобы пощадил тебя Христос Бог, Вездесущий и все наполняющий.
}

\mysubsubsection{Песнь 7}

\pripevc{\myemph{Ирм\'{о}с:}}

\minicolumns{\firstletter{С}огрешихом, беззаконновахом, неправдовахом пред Тобою, ниже соблюдохом, ниже сотворихом, якоже заповедал еси нам; но не предаждь нас до конца, отцев Боже.}{
Мы согрешили, жили беззаконно, неправо поступали пред Тобою, не сохранили, не исполнили, что Ты заповедал нам; но не оставь нас до конца, Боже отцов.
\myemph{\footnotesize \mbox{Дан. 9:5--6}}
}

\pripevpomiluj

\minicolumns{\firstletter{К}ивот яко ношашеся на колеснице, Зан оный, егда превращшуся тельцу, точию коснуся, Божиим искусися гневом; но того дерзновения убежавши, душе, почитай Божественная честне.}{
Когда ковчег везли на колеснице, то Оза, когда вол свернул в сторону, лишь только прикоснулся, испытал на себе гнев Божий, но, душа, избегая его дерзости, благоговейно почитай Божественное.
\myemph{\footnotesize \mbox{2 Цар. 6:6--7}}
}

\pripevpomiluj

\minicolumns{\firstletter{С}лышала еси Авессалома, како на естество воста, познала еси того скверная деяния, имиже оскверни ложе Давида отца; но ты подражала еси того страстная и любосластная стремления.}{
Ты слышала об Авессаломе, как он восстал на самую природу, знаешь гнусные его деяния, которыми он обесчестил ложе отца "--- Давида; но ты сама подражала его страстным и сластолюбивым порывам.
\myemph{\footnotesize \mbox{2 Цар. 15:1--37}; \mbox{2 Цар. 16:21}}
}

\pripevpomiluj

\minicolumns{\firstletter{П}окорила еси неработное твое достоинство телу твоему, иного бо Ахитофела обретше врага, душе, снизшла еси сего советом; но сия разсыпа Сам Христос, да ты всяко спасешися.}{
Свободное свое достоинство ты, душа, подчинила своему телу, ибо, нашедши другого Ахитофела-врага, ты склонилась на его советы, но их рассеял Сам Христос, чтобы ты спасена была.
\myemph{\footnotesize \mbox{2 Цар. 16:20--21}}
}

\pripevpomiluj

\minicolumns{\firstletter{С}оломон чудный и благодати премудрости исполненный, сей лукавое иногда пред Богом сотворив, отступи от Него; емуже ты проклятым твоим житием, душе, уподобилася еси.}{
Чудный Соломон, будучи преисполнен дара премудрости, некогда, сотворив злое пред Богом, отступил от Него; ему ты уподобилась, душа, своей жизнью, достойной проклятия.
\myemph{\footnotesize \mbox{3 Цар. 3:12}; \mbox{3 Цар. 11:4--6}}
}

\pripevpomiluj

\minicolumns{\firstletter{С}ластьми влеком страстей своих, оскверняшеся, увы мне, рачитель премудрости, рачитель блудных жен и странен от Бога: егоже ты подражала еси умом, о душе, сладострастьми скверными.}{
Увлекшись сластолюбивыми страстями, осквернился, увы, ревнитель премудрости, возлюбив нечестивых женщин и отчуждившись от Бога; ему, душа, ты сама подражала в уме постыдным сладострастием.
\myemph{\footnotesize \mbox{3 Цар. 11:6--8}}
}

\pripevpomiluj

\minicolumns{\firstletter{Р}овоаму поревновала еси, не послушавшему совета отча, купно же и злейшему рабу Иеровоаму, прежнему отступнику, душе, но бегай подражания и зови Богу: согреших, ущедри мя.}{
Ты поревновала, душа, Ровоаму, не послушавшему совета отеческого, и вместе злейшему рабу Иеровоаму, древнему мятежнику; избегай подражание им и взывай к Богу: согрешила я, умилосердись надо мною.
\myemph{\footnotesize \mbox{3 Цар. 12:13--14}, \mbox{3 Цар. 12:20}}
}

\slavac

\minicolumns{\firstletter{Т}р\'{о}ице Пр\'{о}стая, Нераздельная, Единосущная и Естество Едино, Светове и Свет, и Свята Три, и Едино Свято поется Бог Тр\'{о}ица; но воспой, прослави Живот и Животы, душе, всех Бога.}{
Троица Простая, Нераздельная, Единосущная, Единая Естеством, Светы и Свет и Три Святы и Едино (Лицо) Свято, Бог-Троица, воспеваемая в песнопениях; воспой же и ты, душа, прославь Жизнь и Жизни "--- Бога всех.
}

\inynec

\minicolumns{\firstletter{П}оем Тя, благословим Тя, покланяемся Ти, Богородительнице, яко Неразлучныя Тр\'{о}ицы породила еси Единаго Христа Бога и Сама отверзла еси нам, сущим на земли, Небесная.}{
Воспеваем Тебя, благословляем Тебя, поклоняемся Тебе, Богородительница, ибо Ты родила Единого из Нераздельной Троицы Христа Бога и Сама открыла для нас, живущих на земле, небесные обители.
}

\mysubsubsection{Песнь 8}

\pripevc{\myemph{Ирм\'{о}с:}}

\minicolumns{\firstletter{Е}гоже воинства Небесная славят, и трепещут херувими и серафими, всяко дыхание и тварь, пойте, благословите и превозносите во вся веки.}{
Кого прославляют воинства небесные и пред Кем трепещут Херувимы и Серафимы, Того, все существа и творения, воспевайте, благословляйте и превозносите во все века.
}

\pripevpomiluj

\minicolumns{\firstletter{Т}ы Озии, душе, поревновавши, сего прокажение в себе стяжала еси сугубо: безместная бо мыслиши, беззаконная же дееши; остави, яже имаши, и притецы к покаянию.}{
Соревновав Озии, душа, ты получила себе вдвойне его проказу, ибо помышляешь недолжное и делаешь беззаконное; оставь, что у тебя есть и приступи к покаянию.
\myemph{\footnotesize \mbox{4 Цар. 15:5}; \mbox{2 Пар. 26:19}}
}

\pripevpomiluj

\minicolumns{\firstletter{Н}иневитяны, душе, слышала еси кающияся Богу, вретищем и пепелом, сих не подражала еси, но явилася еси злейшая всех, прежде закона и по законе прегрешивших.}{
Ты слышала, душа, о ниневитянах, в рубище и пепле каявшихся Богу; им ты не подражала, но оказалась упорнейшею всех, согрешивших до закона и после закона.
\myemph{\footnotesize \mbox{Иона 3:5}}
}

\pripevpomiluj

\minicolumns{\firstletter{В} рове блата слышала еси Иеремию, душе, града Сионя рыданьми вопиюща и слез ищуща: подражай сего плачевное житие и спасешися.}{
Ты слышала, душа, как Иеремия, в нечистом рве с рыданиями взывал к городу Сиону и искал слез; подражай плачевной его жизни и спасешься.
\myemph{\footnotesize \mbox{Иер. 38:6}}
}

\pripevpomiluj

\minicolumns{\firstletter{И}она в Фарсис побеже, проразумев обращение ниневитянов, разуме бо, яко пророк, Божие благоутробие: темже ревноваше пророчеству не солгатися.}{
Иона побежал в Фарсис, предвидя обращение ниневитян, ибо он, как пророк, знал милосердие Божие и вместе ревновал, чтобы пророчество не оказалось ложным.
\myemph{\footnotesize \mbox{Иона 1:3}}
}

\pripevpomiluj

\minicolumns{\firstletter{Д}аниила в рове слышала еси, како загради уста, о душе, зверей; уведела еси, како отроцы, иже о Азарии, погасиша верою пещи пламень горящий.}{
Ты слышала, душа, как Даниил во рве заградил уста зверей; ты узнала, как юноши, бывшие с Азариею, верою угасили разожженный пламень печи.
\myemph{\footnotesize \mbox{Дан. 14:31}; \mbox{Дан. 3:24}}
}

\pripevpomiluj

\minicolumns{\firstletter{В}етхаго Завета вся приведох ти, душе, к подобию; подражай праведных боголюбивая деяния, избегни же паки лукавых грехов.}{
Из Ветхого Завета всех я привел тебе в пример, душа; подражай богоугодным деяниям праведных, и избегай грехов людей порочных.
}

\slavac

\minicolumns{\firstletter{Б}езначальне Отче, Сыне Собезначальне, Утешителю Благий, Душе Правый, Слова Божия Родителю, Отца Безначальна Слове, Душе Живый и Зиждяй, Тр\'{о}ице Ед\'{и}нице, помилуй мя.}{
Безначальный Отче, Собезначальный Сын, Утешитель благий, Дух правды, Родитель Бога Слова, Слово Безначальное Отца, Дух Животворящий и Созидающий, Троица Единая, помилуй меня.
}

\inynec

\minicolumns{\firstletter{Я}ко от оброщения червленицы, Пречистая, умная багряница Еммануилева внутрь во чреве Твоем плоть исткася. Темже Богородицу воистинну Тя почитаем.}{
Мысленная порфира "--- плоть Еммануила соткалась внутри Твоего чрева, Пречистая, как бы из вещества пурпурного; потому мы почитаем Тебя, Истинную Богородицу.
}

\mysubsubsection{Песнь 9}

\pripevc{\myemph{Ирм\'{о}с:}}

\minicolumns{\firstletter{Б}езсеменнаго зачатия Рождество несказанное, Матере безмужныя нетленен Плод, Божие бо Рождение обновляет естества. Темже Тя вси роди, яко Богоневестную Матерь, православно величаем.}{
Рождество от бессеменного зачатия неизъяснимо, безмужной Матери нетленен Плод, ибо рождение Бога обновляет природу. Поэтому Тебя, как Богоневесту-Матерь мы, все роды, православно величаем.
}

\pripevpomiluj

\minicolumns{\firstletter{Х}ристос искушашеся, диавол искушаше, показуя камение, да хлеби будут, на гору возведе видети вся царствия мира во мгновении; убойся, о душе, ловления, трезвися, молися на всякий час Богу.}{
Христос был искушаем; диавол искушал, показывая камни, чтобы они обратились в хлебы; возвел Его на гору, чтобы видеть все царства мира в одно мгновение; бойся, душа, этого обольщения, бодрствуй и ежечасно молись Богу.
\myemph{\footnotesize \mbox{Мф. 4:1--9}; Мк. 1; 12--13; \mbox{Лк. 4:1--12}}
}

\pripevpomiluj

\minicolumns{\firstletter{Г}орлица пустыннолюбная, глас вопиющаго возгласи, Христов светильник, проповедуяй покаяние, Ирод беззаконнова со Иродиадою. Зри, душе моя, да не увязнеши в беззаконныя сети, но облобызай покаяние.}{
Пустыннолюбивая горлица, голос вопиющего, Христов светильник взывал, проповедуя покаяние, а Ирод беззаконствовал с Иродиадою; смотри, душа моя, чтобы не впасть тебе в сети беззаконных, но возлюби покаяние.
\myemph{\footnotesize \mbox{Песн. 2:12}; \mbox{Ис. 40:3}; \mbox{Мф. 3:8}; \mbox{Мф. 14:3}; \mbox{Мк. 6:17}; \mbox{Лк. 3:19--20}}
}

\pripevpomiluj

\minicolumns{\firstletter{В} пустыню вселися благодати Предтеча, и Иудея вся и Самария, слышавше, течаху и исповедаху грехи своя, крещающеся усердно: ихже ты не подражала еси, душе.}{
Предтеча благодати обитал в пустыне и все иудеи и самаряне стекались слушать его и исповедовали грехи свои, с усердием принимая крещение. Но ты, душа, не подражала им.
\myemph{\footnotesize \mbox{Мф. 3:1--6}; \mbox{Мк. 1:3--6}}
}

\pripevpomiluj

\minicolumns{\firstletter{Б}рак убо честный и ложе нескверно, обоя бо Христос прежде благослови, плотию ядый и в Кане же на браце воду в вино совершая, и показуя первое чудо, да ты изменишися, о душе.}{
Брак честен и ложе непорочно, ибо Христос благословил их некогда, в Кане на браке вкушая пищу плотию и претворяя воду в вино, совершая первое чудо, чтобы ты, душа, изменилась.
\myemph{\footnotesize \mbox{Евр. 13:4}; \mbox{Ин. 2:1--11}}
}

\pripevpomiluj

\minicolumns{\firstletter{Р}азслабленнаго стягну Христос, одр вземша, и юношу умерша воздвиже, вдовиче рождение, и сотнича отрока, и самаряныне явися, в дусе службу тебе, душе, предживописа.}{
Христос укрепил расслабленного, взявшего постель свою; воскресил умершего юного сына вдовы, исцелил слугу сотника и, открыв Себя самарянке, предначертал тебе, душа, служение Богу духом.
\myemph{\footnotesize \mbox{Мф. 9:6}; \mbox{Мф. 8:13}; \mbox{Лк. 7:14}; \mbox{Ин. 4:7--24}}
}

\pripevpomiluj

\minicolumns{\firstletter{К}ровоточивую исцели прикосновением края ризна Господь, прокаженныя очисти, слепыя и хромыя просветив, исправи, глухия же, и немыя, и ничащия низу исцели словом: да ты спасешися, окаянная душе.}{
Господь исцелил кровоточивую через прикосновение к одежде Его, очистил прокаженных, дал прозрение слепым, исправил хромых, глухих, немых и уврачевал словом скорченную, чтобы ты спаслась, несчастная душа.
\myemph{\footnotesize \mbox{Мф. 9:20}; \mbox{Мф. 11:5}; \mbox{Лк. 13:11--13}}
}

\slavac

\minicolumns{\firstletter{О}тца прославим, Сына превознесем, Божественному Духу верно поклонимся, Тр\'{о}ице Нераздельней, Ед\'{и}нице по существу, яко Свету и Светом, и Животу и Животом, Животворящему и Просвещающему концы.}{
Прославим Отца, превознесем Сына, с верою поклонимся Божественному Духу, Нераздельной Троице, Единой по существу, как Свету и Светам, Жизни и Жизням, животворящему и просвещающему пределы вселенной.
}

\inynec

\minicolumns{\firstletter{Г}рад Твой сохраняй, Богородительнице Пречистая, в Тебе бо сей верно царствуяй, в Тебе и утверждается, и Тобою побеждаяй, побеждает всякое искушение, и пленяет ратники, и проходит послушание.}{Сохраняй град Свой, Пречистая Богородительница. Под Твоею защитою он царствует с верою, и от Тебя получает крепость, и при Твоем содействии неотразимо побеждает всякое бедствие, берет в плен врагов и держит их в подчинении.}

\pripevmskipc{\pripev{\firstletter{П}реподобне отче Андрее, моли Бога о нас.}}

\minicolumns{\firstletter{А}ндрее честный и отче треблаженнейший, пастырю Критский, не престай моляся о воспевающих тя, да избавимся вси гнева, и скорби, и тления, и прегрешений безмерных, чтущии твою память верно.}{
Андрей досточтимый, отец преблаженный, пастырь Критский, не переставай молиться за воспевающих тебя, чтобы избавиться от гнева, скорби, погибели и бесчисленных прегрешений нам всем, искренно почитающим память твою.}

\pripevmskipc{\myemph{\firstletter{Т}аже оба лика вкупе поют Ирм\'{о}с:}}

\minicolumns{\firstletter{Б}езсеменнаго зачатия Рождество несказанное, Матере безмужныя нетленен Плод, Божие бо Рождение обновляет естества. Темже Тя вси роди, яко Богоневестную Матерь, православно величаем.}{Рождество от бессеменного зачатия неизъяснимо, безмужной Матери нетленен Плод, ибо рождение Бога обновляет природу. Поэтому Тебя, как Богоневесту-Матерь мы, все роды, православно величаем.}

\end{Parallel}

\mychapterending

\mychapter{В среду первой седмицы Великого Поста}
 
\begin{Parallel}{}{}

\mysubsection{Славянский текст с русским переводом и ссылками на текст Библии}

\mysubsubsection{Песнь 1}

\pripevc{\myemph{Ирм\'{о}с:}}

\minicolumns{\firstletter{П}омощник и Покровитель бысть мне во спасение, Сей мой Бог, и прославлю Его, Бог отца моего, и вознесу Его: славно бо прославися.}{
Помощник и Покровитель явился мне ко спасению, Он Бог мой, и прославлю Его, Бога отца моего, и превознесу Его, ибо Он торжественно прославился.
\myemph{\footnotesize \mbox{Исх. 15:1--2}}
}

\pripevpomiluj

\minicolumns{\firstletter{О}т юности, Христе, заповеди Твоя преступих, всестрастно небрегий, унынием преидох житие. Темже зову Ти, Спасе: поне на конец спаси мя.}{
С юности, Христе, я пренебрегал Твоими заповедями, всю жизнь провел в страстях, беспечности и нерадении. Поэтому и взываю к Тебе, Спаситель: хотя при кончине спаси меня.
}

\pripevpomiluj

\minicolumns{\firstletter{П}овержена мя, Спасе, пред враты Твоими, поне на старость не отрини мене во ад тща, но прежде конца, яко Человеколюбец, даждь ми прегрешений оставление.}{
Поверженного пред вратами Твоими, Спаситель, хотя в старости, не низринь меня в ад, как невоздержанного, но прежде кончины, как Человеколюбец, даруй мне оставление прегрешений.
}

\pripevpomiluj

\minicolumns{\firstletter{Б}огатство мое, Спасе, изнурив в блуде, пуст есмь плодов благочестивых, алчен же зову: Отче щедрот, предварив, Ты мя ущедри.}{
Расточив богатство мое в распутстве, Спаситель, я чужд плодов благочестия, но, чувствуя голод, взываю: Отец Милосердный, поспеши и умилосердись надо мною.
}

\pripevpomiluj

\minicolumns{\firstletter{В} разбойники впадый аз есмь помышленьми моими, весь от них уязвихся ныне и исполнихся ран, но, Сам ми представ, Христе Спасе, исцели.}{
По помыслам моим я человек, попавшийся разбойникам; теперь я весь изранен ими, покрыт язвами, но Ты Сам, Христос Спаситель, приди и исцели меня.
\myemph{\footnotesize \mbox{Лк. 10:30}}
}

\pripevpomiluj

\minicolumns{\firstletter{С}вященник, мя предвидев, мимо иде, и левит, видя в лютых нага, презре, но из Марии возсиявый Иисусе, Ты, представ, ущедри мя.}{
Священник, заметив меня, прошел мимо, и левит, видя меня в беде обнаженного, презрел; но Ты, воссиявший от Марии Иисусе, прииди и умилосердись надо мною.
\myemph{\footnotesize \mbox{Лк. 10:31--32}}
}

\pripevmskipc{\pripev{\firstletter{П}реподобная мати Марие, моли Бога о нас.}}

\minicolumns{\firstletter{Т}ы ми даждь светозарную благодать от Божественнаго свыше промышления избежати страстей омрачения и пети усердно Твоего, Марие, жития красная исправления.}{
Даруй мне, Мария, ниспосланную тебе свыше Божественным Промыслом светозарную благодать "--- избежать мрака страстей и усердно воспеть прекрасные подвиги твоей жизни.
}

\slavac

\minicolumns{\firstletter{П}ресущная Тр\'{о}ице, во Ед\'{и}нице покланяемая, возьми бремя от мене тяжкое греховное и, яко благоутробна, даждь ми слезы умиления.}{
Пресущественная Троица, Которой мы поклоняемся как Единому Существу, сними с меня тяжелое бремя греховное и, как Милосердная, даруй мне слезы умиления.
}

\inynec

\minicolumns{\firstletter{Б}огородице, Надежде и Предстательство Тебе поющих, возьми бремя от мене тяжкое греховное и, яко Владычица Чистая, кающася приими мя.}{
Богородице, Надежда и Помощь всем воспевающих Тебя, сними с меня тяжелое бремя греховное и, как Владычица Непорочная, прими меня кающегося.
}

\mysubsubsection{Песнь 2}

\pripevc{\myemph{Ирм\'{о}с:}}

\minicolumns{\firstletter{В}онми, Небо, и возглаголю, и воспою Христа, от Девы плотию пришедшаго.}{
Внемли, небо, я буду возвещать и воспевать Христа, пришедшего во плоти от Девы.
}

\pripevpomiluj

\minicolumns{\firstletter{П}оползохся, яко Давид, блудно и осквернихся, но омый и мене, Спасе, слезами.}{
От невоздержания, как Давид, я пал и осквернился, но омой и меня, Спаситель, слезами.
\myemph{\footnotesize \mbox{2 Цар. 11:4}}
}

\pripevpomiluj

\minicolumns{\firstletter{Н}и слез, ниже покаяния имам, ниже умиления. Сам ми сия, Спасе, яко Бог, даруй.}{
Ни слез, ни покаяния, ни умиления нет у меня; Сам Ты, Спаситель, как Бог, даруй мне это.
}

\pripevpomiluj

\minicolumns{\firstletter{П}огубих первозданную доброту и благолепие мое и ныне лежу наг и стыждуся.}{
Погубил я первозданную красоту и благообразие мое и теперь лежу обнаженным и стыжусь.
}

\pripevpomiluj

\minicolumns{\firstletter{Д}верь Твою не затвори мне тогда, Господи, Господи, но отверзи ми сию, кающемуся Тебе.}{
Не затвори предо мною теперь дверь Твою, Господи, Господи, но отвори ее для меня, кающегося Тебе.
\myemph{\footnotesize \mbox{Мф. 7:21--23}; \mbox{Мф. 25:11}}
}

\pripevpomiluj

\minicolumns{\firstletter{В}нуши воздыхания души моея и очию моею приими капли, Спасе, и спаси мя.}{
Внемли, Спаситель, стенаниям души моей, прими слезы очей моих и спаси меня.
}

\pripevpomiluj

\minicolumns{\firstletter{Ч}еловеколюбче, хотяй всем спастися, Ты воззови мя и приими, яко благ, кающагося.}{
Человеколюбец, желающий всем спасения, Ты призови меня и прими, как Благий, кающегося.
\myemph{\footnotesize \mbox{1 Тим. 2:4}}
}

\pripevmskipc{\pripev{\firstletter{П}ресвятая Богородице, спаси нас.}}

\minicolumns{\firstletter{П}речистая Богородице Дево, Едина Всепетая, моли прилежно, во еже спастися нам.}{
Пречистая Богородице Дева, Ты Одна, всеми воспеваемая, усердно моли о нашем спасении.
}

\pripevmskipc{\myemph{Иный Ирм\'{о}с:}}

\minicolumns{\firstletter{В}идите, видите, яко Аз есмь Бог, манну одождивый и воду из камене источивый древле в пустыни людем Моим, десницею единою и крепостию Моею.}{
Видите, видите, что Я "--- Бог, в древности ниспославший манну и источивший воду из камня народу Моему в пустыне "--- одним Своим всемогуществом.
\myemph{\footnotesize \mbox{Исх. 16:14}; \mbox{Исх. 17:6}}
}

\pripevpomiluj

\minicolumns{\firstletter{В}идите, видите, яко Аз есмь Бог, внушай, душе моя, Господа вопиюща, и удалися прежняго греха, и бойся, яко неумытнаго и яко Судии и Бога.}{
Видите, видите, что Я "--- Бог. Внимай, душа моя, взывающему Господу, оставь прежний грех и убойся как праведного Судию и Бога.
}

\pripevpomiluj

\minicolumns{\firstletter{К}ому уподобилася еси, многогрешная душе? токмо первому Каину и Ламеху оному, каменовавшая тело злодействы и убившая ум безсловесными стремленьми.}{
Кому уподобилась ты, многогрешная душа, как не первому Каину и тому Ламеху, жестоко окаменив тело злодеяниями и убив ум безрассудными стремлениями.
\myemph{\footnotesize \mbox{Быт. 4:1--26}}
}

\pripevpomiluj

\minicolumns{\firstletter{В}ся прежде закона претекши, о душе, Сифу не уподобилася еси, ни Еноса подражала еси, ни Еноха преложением, ни Ноя, но явилася еси убога праведных жизни.}{
Имея в виду всех, живших до закона, о душа, не уподобилась ты Сифу, не подражала ни Еносу, ни Еноху через преселение духовное, ни Ною, но оказалась чуждой жизни праведников.
\myemph{\footnotesize \mbox{Быт. 5:1--32}}
}

\pripevpomiluj

\minicolumns{\firstletter{Е}дина отверзла еси хляби гнева \mbox{Бога} Твоего, душе моя, и потопила еси всю, якоже землю, плоть, и деяния, и житие, и пребыла еси вне спасительнаго ковчега.}{
Ты одна, душа моя, открыла бездны гнева Бога своего и потопила, как землю, всю плоть, и дела, и жизнь, и осталась вне спасительного ковчега.
\myemph{\footnotesize \mbox{Быт. 7:1--24}}
}

\pripevmskipc{\pripev{\firstletter{П}реподобная мати Марие, моли Бога о нас.}}

\minicolumns{\firstletter{В}сем усердием и любовию притекла еси Христу, первый греха путь отвращши, и в пустынях непроходимых питающися, и Того чисте совершающи Божественныя заповеди.}{
Оставив прежний путь греха, ты с всем усердием и любовью прибегла ко Христу, живя в непроходимых пустынях и в чистоте исполняя Божественные Его заповеди.
}

\slavac

\minicolumns{\firstletter{Б}езначальная, Несозданная Тр\'{о}ице, Нераздельная Ед\'{и}нице, кающася мя приими, согрешивша спаси, Твое есмь создание, не презри, но пощади и избави огненнаго мя осуждения.}{
Безначальная Несозданная Троица, Нераздельная Единица, прими меня кающегося, спаси согрешившего, я "--- Твое создание, не презри, но пощади и избавь меня от осуждения в огонь.
}

\inynec

\minicolumns{\firstletter{П}речистая Владычице, Богородительнице, Надеждо к Тебе притекающих и пристанище сущих в бури, Милостиваго и Создателя и Сына Твоего умилостиви и мне молитвами Твоими.}{
Пречистая Владычица, Богородительница, Надежда прибегающих к Тебе и пристанище для застигнутых бурей, Твоими молитвами приклони на милость и ко мне Милостивого Творца и Сына Твоего.
}

\mysubsubsection{Песнь 3}

\pripevc{\myemph{Ирм\'{о}с:}}

\minicolumns{\firstletter{У}тверди, Господи, на камени заповедей Твоих подвигшееся сердце мое, яко Един Свят еси и Господь.}{
Утверди, Господи, на камне Твоих заповедей поколебавшееся сердце мое, ибо Ты один Свят и Господь.
}

\pripevpomiluj

\minicolumns{\firstletter{Б}лагословения Симова не наследовала еси, душе окаянная, ни пространное одержание, якоже Иафеф, имела еси на земли оставления.}{
Симова благословения не наследовала ты, несчастная душа, и не получила, подобно Иафету, обширного владения на земле "--- отпущения грехов.
}

\pripevpomiluj

\minicolumns{\firstletter{О}т земли Харран изыди от греха, душе моя, гряди в землю, точащую присноживотное нетление, еже Авраам наследствова.}{
Удались, душа моя, от земли Харран "--- от греха; иди в землю, источающую вечно живое нетление, которую наследовал Авраам.
\myemph{\footnotesize \mbox{Быт. 12:1--7}}
}

\pripevpomiluj

\minicolumns{\firstletter{А}враама слышала еси, душе моя, древле оставльша землю отечества и бывша пришельца, сего произволению подражай.}{
Ты слышала, душа моя, как в древности Авраам оставил землю отеческую и сделался странником; подражай его решимости.
\myemph{\footnotesize \mbox{Быт. 12:1--7}}
}

\pripevpomiluj

\minicolumns{\firstletter{У} дуба Мамврийскаго учредив патриарх ангелы, наследствова по старости обетования ловитву.}{
Угостив Ангелов под дубом Маврийским, патриарх на старости получил, как добычу, обещанное.
\myemph{\footnotesize \mbox{Быт. 18:1}}
}

\pripevpomiluj

\minicolumns{\firstletter{И}саака, окаянная душе моя, разумевши новую жертву, тайно всесожженную Господеви, подражай его произволению.}{
Зная, бедная душа моя, как Исаак принесен таинственно в новую жертву всесожжения Господу, подражай его решимости.
\myemph{\footnotesize \mbox{Быт. 22:2}}
}

\pripevpomiluj

\minicolumns{\firstletter{И}смаила слышала еси, трезвися, душе моя, изгнана, яко рабынино отрождение, виждь, да не како подобно что постраждеши, ласкосердствующи.}{
Ты слышала, душа моя, что Измаил был изгнан, как рожденный рабыней, бодрствуй, смотри, чтобы и тебе не потерпеть бы чего-либо подобного за сладострастие.
\myemph{\footnotesize \mbox{Быт. 21:10--11}}
}

\pripevmskipc{\pripev{\firstletter{П}реподобная мати Марие, моли Бога о нас.}}

\minicolumns{\firstletter{С}одержим есмь бурею и треволнением согрешений, но сама мя, мати, ныне спаси и к пристанищу Божественнаго покаяния возведи.}{
Окружен я, матерь, бурей и сильным волнением согрешений, но ты сама ныне спаси меня и приведи к пристанищу Божественного покаяния.
}

\pripevmskipc{\pripev{\firstletter{П}реподобная мати Марие, моли Бога о нас.}}

\minicolumns{\firstletter{Р}абское моление и ныне, преподобная, принесши ко благоутробней молитвами твоими Богородице, отверзи ми Божественныя входы.}{
Усердное моление и ныне, преподобная, принеся к умилостивленной твоими молитвами Богородице, открой и для меня Божественные входы.
}

\slavac

\minicolumns{\firstletter{Т}р\'{о}ице Пр\'{о}стая, Несозданная, Безначальное Естество, в Тр\'{о}ице певаемая Ипостасей, спаси ны, верою покланяющияся державе Твоей.}{
Троица Несоставная, Несозданная, Существо Безначальная, в троичности Лиц воспеваемая, спаси нас, с верою поклоняющихся силе Твоей.
}

\inynec

\minicolumns{\firstletter{О}т Отца безлетна Сына в лето, Богородительнице, неискусомужно родила еси, странное чудо, пребывши Дева доящи.}{
Ты, Богородительница, не испытавши мужа, во времени родила Сына от Отца вне времени и "--- дивное чудо: питая молоком, пребыла Девою.
}

\mysubsubsection{Песнь 4}

\pripevc{\myemph{Ирм\'{о}с:}}

\minicolumns{\firstletter{У}слыша пророк пришествие Твое, Господи, и убояся, яко хощеши от Девы родитися и человеком явитися, и глаголаше: услышах слух Твой и убояхся, слава силе Твоей, Господи.}{
Услышал пророк о пришествии Твоем, Господи, и устрашился, что Тебе угодно родиться от Девы и явиться людям, и сказал: услышал я весть о Тебе и устрашился; слава силе Твоей, Господи.
}

\pripevpomiluj

\minicolumns{\firstletter{Т}ело осквернися, дух окаляся, весь острупихся, но яко врач, Христе, обоя покаянием моим уврачуй, омый, очисти, покажи, Спасе мой, паче снега чистейша.}{
Тело мое осквернено, дух грязен, весь я покрыт струпами, но Ты, Христе, как врач, уврачуй и то и другое моим покаянием, омой, очисти, яви меня чище снега, Спаситель мой.
}

\pripevpomiluj

\minicolumns{\firstletter{Т}ело Твое и кровь, распинаемый о всех, положил еси, Слове: тело убо, да мя обновиши, кровь, да омыеши мя. Дух же предал еси, да мя приведеши, Христе, Твоему Родителю.}{
Твое тело и Кровь, Слово, Ты принес в жертву за всех при распятии; Тело "--- чтобы воссоздать меня, Кровь "--- чтобы омыть меня, и Дух Ты, Христе, предал, чтобы привести меня к Твоему Отцу.
}

\pripevpomiluj

\minicolumns{\firstletter{С}оделал еси спасение посреде земли, Щедре, да спасемся. Волею на древе распялся еси, Едем затворенный отверзеся, горняя и дольняя тварь, языцы вси, спасени, покланяются Тебе.}{
Посреди земли Ты устроил спасение, Милосердный, чтобы мы спаслись; Ты добровольно распялся на древе; Едем затворенный открылся; Тебе поклоняются небесные и земные и все спасенные Тобою народы.
\myemph{\footnotesize \mbox{Пс. 73:12}}
}

\pripevpomiluj

\minicolumns{\firstletter{Д}а будет ми купель кровь из ребр Твоих, вкупе и питие, источившее воду оставления, да обоюду очищаюся, помазуяся и пия, яко помазание и питие, Слове, животочная Твоя словеса.}{
Да будет мне омовением Кровь из ребр Твоих и вместе питием, источившая оставление грехов, чтобы мне и тем и другим очищаться, Слове, помазуясь и напояясь животворными Твоими словами, как мазью и питием.
\myemph{\footnotesize \mbox{Ин. 19:34}}
}

\pripevpomiluj

\minicolumns{\firstletter{Ч}ашу Церковь стяжа, ребра Твоя живоносная, из нихже сугубыя нам источи токи оставления и разума во образ древняго и новаго, двоих вкупе заветов, Спасе наш.}{
Церковь приобрела себе Чашу в живоносном ребре Твоем, из которого проистек нам двойной поток оставления грехов и разумения, Спаситель наш, в образ обоих Заветов, Ветхого и Нового.
}

\pripevpomiluj

\minicolumns{\firstletter{Н}аг есмь чертога, наг есмь и брака, купно и вечери; светильник угасе, яко безъелейный, чертог заключися мне спящу, вечеря снедеся, аз же по руку и ногу связан, вон низвержен есмь.}{
Я лишен брачного чертога, лишен и брака, и вечери; светильник, как без елея, погас; чертог закрылся во время моего сна, вечеря окончена, а я, связанный по рукам и ногам, извержен вон.
\myemph{\footnotesize \mbox{Мф. 25:1--13}; \mbox{Лк. 12:35--37}; \mbox{Лк. 13:24--27}; \mbox{Лк. 14:7--24}}
}

\slavac

\minicolumns{\firstletter{Н}ераздельное Существом, Неслитное Лицы богословлю Тя, Троическое Едино Божество, яко Единоцарственное и Сопрестольное, вопию Ти песнь великую, в вышних трегубо песнословимую.}{
Нераздельным по существу, неслиянным в Лицах богословски исповедую Тебя, Троичное Единое Божество, Соцарственное и Сопрестольное; возглашаю Тебе великую песнь, в небесных обителях троекратно воспеваемую.
\myemph{\footnotesize \mbox{Ис. 6:1--3}}
}

\inynec

\minicolumns{\firstletter{И} раждаеши, и девствуеши, и пребываеши обоюду естеством Дева, Рождейся обновляет законы естества, утроба же раждает нераждающая. Бог идеже хощет, побеждается естества чин: творит бо, елика хощет.}{
И рождаешь Ты, и остаешься Девою, в обоих случаях сохраняя по естеству девство. Рожденный Тобою обновляет законы природы, а девственное чрево рождает; когда хочет Бог, то нарушается порядок природы, ибо Он творит, что хочет.
}

\mysubsubsection{Песнь 5}

\pripevc{\myemph{Ирм\'{о}с:}}

\minicolumns{\firstletter{О}т нощи утренююща, Человеколюбче, просвети, молюся, и настави и мене на повеления Твоя, и научи мя, Спасе, творити волю Твою.}{
От ночи бодрствующего, просвети меня, молю, Человеколюбец, путеводи меня в повелениях Твоих и научи меня, Спаситель, исполнять Твою волю.
\myemph{\footnotesize \mbox{Пс. 62:2}; \mbox{Пс. 118:35}}
}

\pripevpomiluj

\minicolumns{\firstletter{Я}ко тяжкий нравом, фараону горькому бых, Владыко, Ианни и Иамври, душею и телом, и погружен умом, но помози ми.}{
По упорству я стал как жестокий нравом фараон, Владыко, по душе и телу я "--- Ианний и Иамврий, и по уму погрязший, но помоги мне.
\myemph{\footnotesize \mbox{Исх. 7:11}; \mbox{2 Тим. 3:8}}
}

\pripevpomiluj

\minicolumns{\firstletter{К}алом смесихся, окаянный, умом, омый мя, Владыко, банею моих слез, молю Тя, плоти моея одежду убелив, яко снег.}{
Загрязнил я, несчастный, свой ум, но омой меня, Владыко, в купели слез моих молю Тебя, и убели, как снег, одежду плоти моей.
}

\pripevpomiluj

\minicolumns{\firstletter{А}ще испытаю моя дела, Спасе, всякаго человека превозшедша грехами себе зрю, яко разумом мудрствуяй, согреших не неведением.}{
Когда исследую свои дела, Спаситель, то вижу, что превзошел я грехами всех людей, ибо я грешил с разумным сознанием, а не по неведению.
}

\pripevpomiluj

\minicolumns{\firstletter{П}ощади, пощади, Господи, создание Твое, согреших, ослаби ми, яко естеством чистый Сам сый Един, и ин разве Тебе никтоже есть кроме скверны.}{
Пощади, Господи, пощади, создание Твое: я согрешил, прости мне, ибо только Ты один чист по природе, и никто, кроме Тебя, не чужд нечистоты.
}

\pripevpomiluj

\minicolumns{\firstletter{М}ене ради Бог сый, вообразился еси в мя, показал еси чудеса, исцелив прокаженныя и разслабленнаго стягнув, кровоточивыя ток уставил еси, Спасе, прикосновением риз.}{
Ради меня, будучи Богом, Ты принял мой образ, Спаситель, и, совершая чудеса, исцелял прокаженных, укреплял расслабленных, остановил кровотечение у кровоточивой прикосновением одежды.
\myemph{\footnotesize \mbox{Мф. 9:20}; \mbox{Мк. 5:25--27}; \mbox{Лк. 8:43--44}}
}

\pripevmskipc{\pripev{\firstletter{П}реподобная мати Марие, моли Бога о нас.}}

\minicolumns{\firstletter{С}труи Иорданския прешедши, обрела еси покой безболезненный, плоти сласти избежавши, еяже и нас изми твоими молитвами, преподобная.}{
Ты перешла поток Иорданский и приобрела покой безболезненный, оставив плотское удовольствие, от которого избавь и нас твоими молитвами, преподобная.
}

\slavac

\minicolumns{\firstletter{Т}я, Тр\'{о}ице, славим Единаго Бога: Свят, Свят, Свят еси, Отче, Сыне и Душе, Пр\'{о}стое Существо, Ед\'{и}нице присно покланяемая.}{
Тебя, Пресвятая Троица, прославляем за Единого Бога: Свят, Свят, Свят Отец, Сын и Дух, Простое Существо, Единица вечно поклоняемая.
}

\inynec

\minicolumns{\firstletter{И}з Тебе облечеся в мое смешение, нетленная, безмужная Мати Дево, Бог, создавый веки, и соедини Себе человеческое естество.}{
В Тебе, Нетленная, не познавшая мужа Матерь-Дево, облекся в мой состав сотворивший мир Бог и соединил с Собою человеческую природу.
}

\mysubsubsection{Песнь 6}

\pripevc{\myemph{Ирм\'{о}с:}}

\minicolumns{\firstletter{В}озопих всем сердцем моим к щедрому Богу, и услыша мя от ада преисподняго, и возведе от тли живот мой.}{
От всего сердца моего я воззвал к милосердному Богу, и Он услышал меня из ада преисподнего и воззвал жизнь мою от погибели.
}

\pripevpomiluj

\minicolumns{\firstletter{В}остани и побори, яко Иисус Амалика, плотския страсти, и гаваониты, лестныя помыслы, присно побеждающи.}{
Восстань и побеждай плотские страсти, как Иисус Амалика, всегда побеждая и гаваонитян "--- обольстительные помыслы.
\myemph{\footnotesize \mbox{Исх. 17:8}; \mbox{Нав. 8:21}}
}

\pripevpomiluj

\minicolumns{\firstletter{П}реиди, времене текущее естество, яко прежде ковчег, и земли оныя буди во одержании обетования, душе, Бог повелевает.}{
Душа, Бог повелевает: перейди, как некогда ковчег Иордан, текущее по своему существу время и сделайся обладательницею обещанной земли.
\myemph{\footnotesize \mbox{Нав. 3:17}}
}

\pripevpomiluj

\minicolumns{\firstletter{Я}ко спасл еси Петра, возопивша, спаси, предварив мя, Спасе, от зверя избави, простер Твою руку, и возведи из глубины греховныя.}{
Подобно тому как Ты спас Петра, воззвавшего, поспеши, Спаситель, спасти и меня, избавь меня от чудовища, простерши Свою руку, и выведи из глубины греха.
\myemph{\footnotesize \mbox{Мф. 14:31}}
}

\pripevpomiluj

\minicolumns{\firstletter{П}ристанище Тя вем утишное, Владыко, Владыко Христе, но от незаходимых глубин греха и отчаяния мя, предварив, избави.}{
Тихое пристанище вижу в Тебе, Владыка, Владыка Христе, поспеши же избавить меня от непроходимых глубин греха и отчаяния.
}

\slavac

\minicolumns{\firstletter{Т}р\'{о}ица есмь Пр\'{о}ста, Нераздельна, раздельна Личне, и Ед\'{и}ница есмь естеством соединена, Отец глаголет, и Сын, и Божественный Дух.}{
Я "--- Троица Несоставная, Нераздельная, раздельная в лицах, и Единица, соединенная по существу; свидетельствует Отец, Сын и Божественный Дух.
}

\inynec

\minicolumns{\firstletter{У}троба Твоя Бога нам роди, воображена по нам: Егоже, яко Создателя всех, моли, Богородице, да молитвами Твоими оправдимся.}{
Чрево Твое родило нам Бога, принявшего наш образ; Его, как Создателя всего мира, моли, Богородица, чтобы по молитвам Твоим нам оправдаться.
}

\pripevmskipc{\firstletter{Г}осподи, помилуй. \myemph{Трижды.}}

\pripevmskipc{\slavainynen}

\minicolumns{\firstletter{Д}уше моя, душе моя, востани, что спиши? конец приближается, и имаши смутитися: воспряни убо, да пощадит тя Христос Бог, везде сый и вся исполняяй.}{
Душа моя, душа моя, восстань, что ты спишь? Конец приближается, и ты смутишься; пробудись же, чтобы пощадил тебя Христос Бог, Вездесущий и все наполняющий.
}

\mysubsubsection{Песнь 7}

\pripevc{\myemph{Ирм\'{о}с:}}

\minicolumns{\firstletter{С}огрешихом, беззаконновахом, не\-прав\-до\-ва\-хом пред Тобою, ниже соблюдохом, ниже сотворихом, якоже заповедал еси нам; но не предаждь нас до конца, отцев Боже.}{
Мы согрешили, жили беззаконно, неправо поступали пред Тобою, не сохранили, не исполнили, что Ты заповедал нам; но не оставь нас до конца, Боже отцов.
\myemph{\footnotesize \mbox{Дан. 9:5--6}}
}

\pripevpomiluj

\minicolumns{\firstletter{М}анассиева собрала еси согрешения изволением, поставльши яко мерзости страсти и умноживши, душе, негодования, но того покаянию ревнующи тепле, стяжи умиление.}{
Ты, душа, добровольно вместила преступления Манассии, поставив вместо идолов страсти и умножив мерзости; но усердно подражай и его покаянию с чувством умиления.
\myemph{\footnotesize \mbox{4 Цар. 21:1--2}}
}

\pripevpomiluj

\minicolumns{\firstletter{А}хаавовым поревновала еси сквернам, душе моя, увы мне, была еси плотских скверн пребывалище и сосуд срамлен страстей, но из глубины твоея воздохни и глаголи Богу грехи твоя.}{
Ты подражала Ахаву в мерзостях, душа моя; увы, ты сделалась жилищем плотских нечистот и постыдным сосудом страстей; но воздохни из глубины своей и поведай Богу грехи свои.
\myemph{\footnotesize \mbox{3 Цар. 16:30}}
}

\pripevpomiluj

\minicolumns{\firstletter{З}аключися тебе небо, душе, и глад Божий постиже тя, егда Илии Фесвитянина, якоже Ахаав, не покорися словесем иногда, но Сараффии уподобився, напитай пророчу душу.}{
Заключилось небо для тебя, душа, и голод от Бога послан на тебя, как некогда на Ахава за то, что он не послушал слов Илии Фесфитянина; но ты подражай вдове Сарептской, напитай душу пророка.
\myemph{\footnotesize \mbox{3 Цар. 17:8--9}}
}

\pripevpomiluj

\minicolumns{\firstletter{П}опали Илия иногда дващи пятьдесят Иезавелиных, егда студныя пророки погуби, во обличение Ахаавово, но бегай подражания двою, душе, и укрепляйся.}{
Илия попалил некогда дважды по пятьдесят служителей Иезавели, когда истреблял гнусных пророков ее в обличение Ахава; но ты, душа, избегай подражания обоим им и крепись в воздержании.
\myemph{\footnotesize \mbox{4 Цар. 1:10--15}}
}

\slavac

\minicolumns{\firstletter{Т}р\'{о}ице Пр\'{о}стая, Нераздельная, Единосущная, и Естество Едино, Светове и Свет, и Свята Три, и Едино Свято поется Бог Тр\'{о}ица; но воспой, прослави Живот и Животы, душе, всех Бога.}{
Троица Простая, Нераздельная, Единосущная, и Одно Божество, Светы и Свет, Три Святы и Одно Лицо Свято, Бог-Троица, воспеваемая в песнопениях; воспой же и ты, душа, прославь Жизнь и Жизни "--- Бога всех.
}

\inynec

\minicolumns{\firstletter{П}оем Тя, благословим Тя, покланяемся Ти, Богородительнице, яко Нераздельныя Тр\'{о}ицы породила еси Единаго Христа Бога и Сама отверзла еси нам, сущим на земли, Небесная.}{
Воспеваем Тебя, благословляем Тебя, поклоняемся Тебе, Богородительница, ибо Ты родила Одного из Нераздельной Троицы, Христа Бога, и Сама открыла для нас, живущих на земле, небесные обители.
}

\mysubsubsection{Песнь 8}

\pripevc{\myemph{Ирм\'{о}с:}}

\minicolumns{\firstletter{Е}гоже воинства Небесная славят, и трепещут херувими и серафими, всяко дыхание и тварь, пойте, благословите и превозносите во вся веки.}{
Кого прославляют воинства небесные и пред Кем трепещут Херувимы и Серафимы, Того, все существа и творения, воспевайте, благословляйте и превозносите во все века.
}

\pripevpomiluj

\minicolumns{\firstletter{П}равосуде Спасе, помилуй и избави мя огня и прещения, еже имам на суде праведно претерпети; ослаби ми прежде конца добродетелию и покаянием.}{
Правосудный Спаситель, помилуй и избавь меня от огня и наказания, которому я должен справедливо подвергнуться на суде; прости меня прежде кончины, дав мне добродетель и покаяние.
}

\pripevpomiluj

\minicolumns{\firstletter{Я}ко разбойник, вопию Ти: помяни мя; яко Петр, плачу горце: ослаби ми, Спасе; зову, яко мытарь, слезю, яко блудница; приими мое рыдание, якоже иногда хананеино.}{
Как разбойник взываю к Тебе: вспомни меня; как Петр, горько плачу, Спаситель; как мытарь, издаю вопль: будь милостив ко мне; проливаю слезы, как блудница; прими мое рыдание, как некогда от жены Хананейской.
\myemph{\footnotesize \mbox{Лк. 7:37--38}; \mbox{Лк. 18:13}; \mbox{Лк. 23:42};}
 \mbox{Лк. 22:62}; \mbox{Мф. 15:22}
}

\pripevpomiluj

\minicolumns{\firstletter{Г}ноение, Спасе, исцели смиренныя моея души, Едине Врачу, пластырь мне наложи, и елей, и вино, дела покаяния, умиление со слезами.}{
Один Врач "--- Спаситель, исцели гниение моей смиренной души; приложи мне пластырь, елей и вино "--- дела покаяния, умиление со слезами.
}

\pripevpomiluj

\minicolumns{\firstletter{Х}ананею и аз подражая, помилуй мя, вопию, Сыне Давидов; касаюся края ризы, яко кровоточивая, плачу, яко Марфа и Мария над Лазарем.}{
Подражая жене Хананейской, и я взываю к Сыну Давидову: помилуй меня; касаюсь одежды Его, как кровоточивая, плачу, как Марфа и Мария над Лазарем.
\myemph{\footnotesize \mbox{Мф. 9:20}; \mbox{Мф. 15:22}; \mbox{Ин. 11:33}}
}

\slavac

\minicolumns{\firstletter{Б}езначальне Отче, Сыне Собезначальне, Утешителю Благий, Душе Правый, Слова Божия Родителю, Отца Безначальна Слове, Душе Живый и Зиждяй, Тр\'{о}ице Ед\'{и}нице, помилуй мя.}{
Безначальный Отче, Собезначальный Сын, Утешитель Благий, Дух Правый, Родитель Слова Божия, Безначальное Слово Отца, Дух, Животворящий и Созидающий, Троица Единая, помилуй меня.
}

\inynec

\minicolumns{\firstletter{Я}ко от оброщения червленицы, Пречистая, умная багряница Еммануилева внутрь во чреве Твоем плоть исткася. Темже Богородицу воистинну Тя почитаем.}{
Мысленная порфира "--- плоть Еммануила соткалась внутри Твоего чрева. Пречистая, как бы из вещества пурпурного; потому мы почитаем Тебя, Истинную Богородицу.
}

\mysubsubsection{Песнь 9}

\pripevc{\myemph{Ирм\'{о}с:}}

\minicolumns{\firstletter{Б}езсеменнаго зачатия Рождество несказанное, Матере безмужныя нетленен Плод, Божие бо Рождение обновляет естества. Темже Тя вси роди, яко Богоневестную Матерь, православно величаем.}{
Рождество от бессеменного зачатия неизъяснимо, безмужной Матери нетленен Плод, ибо рождение Бога обновляет природу. Поэтому Тебя, как Богоневесту-Матерь мы, все роды, православно величаем.
}

\pripevpomiluj

\minicolumns{\firstletter{Н}едуги исцеляя, нищим благовествоваше Христос Слово, вредныя уврачева, с мытари ядяше, со грешники беседоваше, Иаировы дщере душу предумершую возврати осязанием руки.}{
Врачуя болезни, Христос-Слово, благовествовал нищим, исцелял увечных, вкушал с мытарями, беседовал с грешниками и прикосновением руки возвратил вышедшую из тела душу Иаировой дочери.
\myemph{\footnotesize \mbox{Мф. 4:23}; \mbox{Мф. 9:10--11}; \mbox{Мк. 5:41--42}}
}

\pripevpomiluj

\minicolumns{\firstletter{М}ытарь спасашеся, и блудница целомудрствоваше, и фарисей, хваляся, осуждашеся. Ов убо: очисти мя; ова же: помилуй мя; сей же величашеся вопия: Боже, благодарю Тя, и прочия безумныя глаголы.}{
Мытарь спасся и блудница сделалась целомудренною, а гордый фарисей подвергся осуждению, ибо первый взывал: «Будь милостив ко мне»; другая: «Помилуй меня»; а последний тщеславно возглашал: «Боже, благодарю Тебя…» и прочие безумные речи.
\myemph{\footnotesize \mbox{Лк. 7:46--47}; \mbox{Лк. 18:14}}
}

\pripevpomiluj

\minicolumns{\firstletter{З}акхей мытарь бе, но обаче спасашеся, и фарисей Симон соблажняшеся, и блудница приимаше оставительная разрешения от Имущаго крепость оставляти грехи, юже, душе, потщися подражати.}{
Закхей был мытарь, однако спасся; Симон фарисей соблазнялся, а блудница получила решительное прощение от Имеющего власть отпускать грехи; спеши, душа, и ты подражать ей.
\myemph{\footnotesize \mbox{Лк. 7:39}; \mbox{Лк. 19:9}; \mbox{Ин. 8:3--11}}
}

\pripevpomiluj

\minicolumns{\firstletter{Б}луднице, о окаянная душе моя, не поревновала еси, яже приимши мира алавастр, со слезами мазаше нозе Спасове, отре же власы, древних согрешений рукописание Раздирающаго ея.}{
Бедная душа моя, ты не подражала блуднице, которая, взяв сосуд с миром, мазала со слезами и отирала волосами ноги Спасителя, разорвавшего запись прежних ее прегрешений.
\myemph{\footnotesize \mbox{Лк. 7:37--38}}
}

\pripevpomiluj

\minicolumns{\firstletter{Г}рады, имже даде Христос благовестие, душе моя, уведала еси, како прокляти быша. Убойся указания, да не будеши якоже оны, ихже содомляном Владыка уподобив, даже до ада осуди.}{
Ты знаешь, душа моя, как прокляты города, которым Христос благовестил Евангелие; страшись этого примера, чтобы и тебе не быть, как они, ибо Владыка, уподобив их содомлянам, присудил их к аду.
\myemph{\footnotesize \mbox{Лк. 10:12--15}}
}

\pripevpomiluj

\minicolumns{\firstletter{Д}а не горшая, о душе моя, явишися отчаянием, хананеи веру слышавшая, еяже дщи словом Божиим исцелися; Сыне Давидов, спаси и мене, воззови из глубины сердца, якоже она Христу.}{
Не окажись, душа моя, по отчаянию хуже хананеянки, слышавшей о вере, по которой Божиим словом исцелена дочь ее; взывай, как она, Христу из глубины сердца: «Сын Давидов, спаси и меня».
\myemph{\footnotesize \mbox{Мф. 15:22}}
}

\slavac

\minicolumns{\firstletter{О}тца прославим, Сына превознесем, Божественному Духу верно поклонимся, Тр\'{о}ице Нераздельней, Ед\'{и}нице по Существу, яко Свету и Светом, и Животу и Животом, Животворящему и Просвещающему концы.}{
Прославим Отца, превознесем Сына, с верою поклонимся Божественному Духу, Нераздельной Троице, Единой по существу, как Свету и Светам, Жизни и Жизням, Животворящему и Просвещающему пределы вселенной.
}

\inynec

\minicolumns{\firstletter{Г}рад Твой сохраняй, Богородительнице Пречистая, в Тебе бо сей верно царствуяй, в Тебе и утверждается, и Тобою побеждаяй, побеждает всякое искушение, и пленяет ратники, и проходит послушание.}{
Сохраняй град Свой, Пречистая Богородительница. Под Твоею защитою он царствует с верою, и от Тебя получает крепость, и при Твоем содействии неотразимо побеждает всякое бедствие, берет в плен врагов и держит их в подчинении.
}

\pripevmskipc{\pripev{\firstletter{П}реподобне отче Андрее, моли Бога о нас.}}

\minicolumns{\firstletter{А}ндрее честный и отче треблаженнейший, пастырю Критский, не престай моляся о воспевающих тя, да избавимся вси гнева, и скорби, и тления, и прегрешений безмерных, чтущии твою память верно.}{
Андрей досточтимый, отец преблаженный, пастырь Критский, не переставай молиться за воспевающих тебя, чтобы избавиться от гнева, скорби, погибели и бесчисленных прегрешений нам всем, искренно почитающим память твою.}

\pripevmskipc{\myemph{\firstletter{Т}аже оба лика вкупе поют Ирм\'{о}с:}}

\minicolumns{\firstletter{Б}езсеменнаго зачатия Рождество несказанное, Матере безмужныя нетленен Плод, Божие бо Рождение обновляет естества. Темже Тя вси роди, яко Богоневестную Матерь, православно величаем.}{Рождество от бессеменного зачатия неизъяснимо, безмужной Матери нетленен Плод, ибо рождение Бога обновляет природу. Поэтому Тебя, как Богоневесту-Матерь мы, все роды, православно величаем.}

\end{Parallel}

\mychapterending

\mychapter{В четверг первой седмицы Великого Поста}
%http://www.molitvoslov.com/text574.htm 
 
\begin{Parallel}{}{}

\mysubsection{Славянский текст с русским переводом и ссылками на текст Библии}

\mysubsubsection{Песнь 1}

\pripevc{\myemph{Ирм\'{о}с:}}

\minicolumns{\firstletter{П}омощник и Покровитель бысть мне во спасение, Сей мой Бог, и прославлю Его, Бог отца моего, и вознесу Его: славно бо прославися. }{Помощник и Покровитель явился мне ко спасению, Он Бог мой, и прославлю Его, Бога отца моего, и превознесу Его, ибо Он торжественно прославился.
\myemph{\footnotesize \mbox{Исх. 15:1--2}}}

\pripevpomiluj

\minicolumns{\firstletter{А}гнче Божий, вземляй грехи всех, возьми бремя от мене тяжкое греховное, и, яко благоутробен, даждь ми слезы умиления.}{Агнец Божий, взявший грехи всех, сними с меня тяжкое бремя греховное и, как Милосердный, даруй мне слезы умиления.
\myemph{\footnotesize \mbox{Ин. 1:29}}}

\pripevpomiluj

\minicolumns{\firstletter{Т}ебе припадаю, Иисусе, согреших Ти, очисти мя, возьми бремя от мене тяжкое греховное и, яко благоутробен, даждь ми слезы умиления.}{К Тебе припадаю, Иисусе, согрешил я пред Тобою, умилосердись надо мною, сними с меня тяжкое бремя греховное и, как Милосердный, даруй мне слезы умиления.}

\pripevpomiluj

\minicolumns{\firstletter{Н}е вниди со мною в суд, нося моя деяния, словеса изыскуя и исправляя стремления. Но в щедротах Твоих презирая моя лютая, спаси мя, Всесильне.}{Не входи со мною в суд, взвешивая мои дела, исследуя слова и обличая стремления, но по Твоим щедротам презирая мои злодеяния, спаси меня, Всесильный.}

\pripevpomiluj

\minicolumns{\firstletter{П}окаяния время, прихожду Ти, Создателю моему: возьми бремя от мене тяжкое греховное и, яко благоутробен, даждь ми слезы умиления.}{Время покаяния: к Тебе прихожу, моему Создателю, сними с меня тяжкое бремя греховное и, как Милосердный, даруй мне слезы умиления.}

\pripevpomiluj

\minicolumns{\firstletter{Б}огатство душевное иждив грехом, пуст есмь добродетелей благочестивых, гладствуя же зову: милости подателю Господи, спаси мя.}{Расточив в грехе духовное богатство, я чужд святых добродетелей, но, испытывая голод, взываю: Источник милости, Господи, спаси меня.}

\pripevmskipc{\pripev{\firstletter{П}реподобная мати Марие, моли Бога о нас.}}

\minicolumns{\firstletter{П}риклоньшися Христовым Божественным законом, к сему приступила еси, сладостей неудержимая стремления оставивши, и всякую добродетель всеблагоговейно, яко едину, исправила еси.}{Покорившись перед Божественными заповедями Христа, ты предалась Ему, оставив необузданные стремления к удовольствиям, и все добродетели, как одну, исполнила со всем благоговением.}

\slavac

\minicolumns{\firstletter{П}ресущественная Тр\'{о}\-ице, во Ед\'{и}нице покланяемая, возьми бремя от мене тяжкое греховное и, яко благоутробна, даждь ми слезы умиления.}{Пресущественная Троица, Которой мы поклоняемся как Единому Существу, сними с меня тяжелое бремя греховное и, как Милосердная, даруй мне слезы умиления.}

\inynec

\minicolumns{\firstletter{Б}огородице, Надежде и Предстательство Тебе поющих, возьми бремя от мене тяжкое греховное и, яко Владычица Чистая, кающася приими мя.}{Богородице, Надежда и Помощь всем воспевающих Тебя, сними с меня тяжелое бремя греховное и, как Владычица Непорочная, прими меня кающегося.}

\mysubsubsection{Песнь 2}

\pripevc{\myemph{Ирм\'{о}с:}}

\minicolumns{\firstletter{В}идите, видите, яко Аз есмь Бог, манну одождивый и воду из камене источивый древле в пустыни людем Моим, десницею единою и крепостию Моею.}{Видите, видите, что Я "--- Бог, в древности ниспославший манну и источивший воду из камня народу Моему в пустыне "--- одним Своим всемогуществом.
\myemph{\footnotesize \mbox{Исх. 16:14}; \mbox{Исх. 17:6}}}

\pripevpomiluj

\minicolumns{\firstletter{М}ужа убих, глаголет, в язву мне и юношу в струп, Ламех, рыдая, вопияше; ты же не трепещеши, о душе моя, окалявши плоть и ум осквернивши.}{Мужа убил я, сказал Ламех, в язву себе, и юношу "--- в рану себе, взывал он, рыдая; ты же, душа моя, не трепещешь, осквернив тело и помрачив ум.
\myemph{\footnotesize \mbox{Быт. 4:23}}}

\pripevpomiluj

\minicolumns{\firstletter{С}толп умудрила еси создати, о душе, и утверждение водрузити твоими похотьми, аще не бы Зиждитель удержал советы твоя и низвергл на землю ухищрения твоя.}{Ты умудрилась, душа, устроить столп и воздвигнуть твердыню своими вожделениями, но Творец обуздал замыслы твои и поверг на землю твои построения.
\myemph{\footnotesize \mbox{Быт. 11:3--4}}}

\pripevpomiluj

\minicolumns{\firstletter{О} како поревновах Ламеху, первому убийце, душу, яко мужа, ум, яко юношу, яко брата же моего, тело убив, яко Каин убийца, любосластными стремленьми.}{О, как уподобился я древнему убийце Ламеху, убив душу, как мужа, ум "--- как юношу, и подобно убийце Каину "--- тело мое, как брата, сластолюбивыми стремлениями.}

\pripevpomiluj

\minicolumns{\firstletter{О}дожди Господь от Господа огнь иногда на беззаконие гневающее, сожег содомляны; ты же огнь вжегла еси геенский, в немже имаши, о душе, сожещися.}{Господь некогда пролил дождем огонь от Господа, попалив неистовое беззаконие содомлян; ты же, душа, разожгла огонь геенский, в котором должна будешь гореть.
\myemph{\footnotesize \mbox{Быт. 19:24}}}

\pripevpomiluj

\minicolumns{\firstletter{У}язвихся, уранихся, се стрелы вражия, уязвившия мою душу и тело; се струпи, гноения, омрачения вопиют, раны самовольных моих страстей.}{Изранен я, изъявлен; вот стрелы врага, пронзившие душу мою и тело; вот раны, язвы и струпы вопиют об ударах самопроизвольных моих страстей.}

\pripevmskipc{\pripev{\firstletter{П}реподобная мати Марие, моли Бога о нас.}}

\minicolumns{\firstletter{П}ростерла еси руце твои к щедрому Богу, Марие, в бездне зол погружаемая; и якоже Петру человеколюбно руку Божественную простре твое обращение всячески Иский.}{Утопая в бездне зла, ты простерла, Мария, руки свои к Милосердному Богу, и Он, всячески ища твоего обращения, человеколюбиво подал тебе, как Петру, Божественную руку.
\myemph{\footnotesize \mbox{Мф. 14:31}}}

\slavac

\minicolumns{\firstletter{Б}езначальная, Несозданная Тр\'{о}ице, Нераздельная Ед\'{и}нице, кающася мя приими, согрешивша спаси, Твое есмь создание, не презри, но пощади и избави мя огненнаго осуждения.}{Безначальная Несозданная Троица, Нераздельная Единица, прими меня кающегося, спаси согрешившего, я "--- Твое создание, не презри, но пощади и избавь меня от осуждения в огонь.}

\inynec

\minicolumns{\firstletter{П}речистая Владычице, Богородительнице, Надеждо к Тебе притекающих и пристанище сущих в бури, Милостиваго и Создателя и Сына Твоего умилостиви и мне молитвами Твоими.}{Пречистая Владычица, Богородительница, Надежда прибегающих к Тебе и пристанище для застигнутых бурей, Твоими молитвами приклони на милость и ко мне Милостивого Творца и Сына Твоего.}

\mysubsubsection{Песнь 3}

\pripevc{\myemph{Ирм\'{о}с:}}

\minicolumns{\firstletter{У}тверди, Господи, на камени заповедей Твоих подвигшееся сердце мое, яко Един Свят еси и Господь.}{Утверди, Господи, на камне Твоих заповедей поколебавшееся сердце мое, ибо Ты Один Свят и Господь.}

\pripevpomiluj

\minicolumns{\firstletter{А}гаре древле, душе, египтяныне уподобилася еси, поработившися произволением и рождши новаго Исмаила, презорство.}{Древней Агари египтянке уподобилась ты, душа, порабощенная своим произволом и родив нового Измаила "--- дерзость.
\myemph{\footnotesize \mbox{Быт. 16:16}}}

\pripevpomiluj

\minicolumns{\firstletter{И}аковлю лествицу разумела еси, душе моя, являемую от земли к Небесем: почто не имела еси восхода тверда, благочестия.}{Ты знаешь, душа моя, о лестнице с земли до небес, показанной Иакову; почему же ты не избрала безопасного восхода "--- благочестия?
\myemph{\footnotesize \mbox{Быт. 28:12}}}

\pripevpomiluj

\minicolumns{\firstletter{С}вященника Божия и царя уединена, Христово подобие в мире жития, в человецех подражай.}{Подражай священнику Божию и царю одинокому Мелхиседеку, образу жизни Христа среди людей в мире.
\myemph{\footnotesize \mbox{Быт. 14:18}; \mbox{Евр. 7:1--3}}}

\pripevpomiluj

\minicolumns{\firstletter{О}братися, постени, душе окаянная, прежде даже не приимет конец жития торжество, прежде даже дверь не заключит чертога Господь.}{Обратись и воздыхай, несчастная душа, прежде нежели кончится торжество жизни, прежде чем Господь затворит дверь брачного чертога.}

\pripevpomiluj

\minicolumns{\firstletter{Н}е буди столп сланый, душе, возвратившися вспять, образ да устрашит тя содомский, горе в Сигор спасайся.}{Не сделайся соляным столпом, душа, обратившись назад, да устрашит тебя пример содомлян; спасайся на гору в Сигор.
\myemph{\footnotesize \mbox{Быт. 19:19--23}; \mbox{Быт. 19:26}}}

\pripevpomiluj

\minicolumns{\firstletter{М}оления, Владыко, Тебе поющих не отвержи, но ущедри, Человеколюбче, и подаждь верою просящим оставление.}{Не отвергни, Владыко, моления воспевающих Тебя, но умилосердись Человеколюбец, и просящим с верою даруй прощение.}

\slavac

\minicolumns{\firstletter{Т}р\'{о}ица Пр\'{о}стая, Несозданная, Безначальное Естество, в Тр\'{о}ице певаемая Ипостасей, спаси ны, верою покланяющияся державе Твоей.}{Троица Несоставная, Несозданная, Существо Безначальная, в троичности Лиц воспеваемая, спаси нас, с верою поклоняющихся силе Твоей.}

\inynec

\minicolumns{\firstletter{О}т Отца безлетна Сына в лето, Богородительнице, неискусомужно родила еси, странное чудо, пребывши Дева доящи.}{Ты, Богородительница, не испытавши мужа, во времени родила Сына от Отца вне времени и "--- дивное чудо: питая молоком, пребыла Девою.}

\mysubsubsection{Песнь 4}

\pripevc{\myemph{Ирм\'{о}с:}}

\minicolumns{\firstletter{У}слыша пророк пришествие Твое, Господи, и убояся, яко хощеши от Девы родитися и человеком явитися, и глаголаше: услышах слух Твой и убояхся, слава силе Твоей, Господи.}{Услышал пророк о пришествии Твоем, Господи, и устрашился, что Тебе угодно родиться от Девы и явиться людям, и сказал: услышал я весть о Тебе и устрашился; слава силе Твоей, Господи.}

\pripevpomiluj

\minicolumns{\firstletter{В}ремя живота моего мало и исполнено болезней и лукавства, но в покаянии мя приими и в разум призови, да не буду стяжание ни брашно чуждему, Спасе, Сам мя ущедри.}{Время жизни моей кратко и исполнено огорчений и пороков, но прими меня в покаянии и призови к познанию истины, чтобы не сделаться мне добычею и пищею врага, Спаситель, умилосердись надо мною.
\myemph{\footnotesize \mbox{Быт. 47:9}}}

\pripevpomiluj

\minicolumns{\firstletter{Ц}арским достоинством, венцем и багряницею одеян, многоименный человек и праведный, богатством кипя и стады, внезапу богатства, славы царства, обнищав, лишися.}{Человек, облеченный царским достоинством, венцом и багряницею, много имевший и праведный, изобиловавший богатством и стадами, внезапно обнищав, лишился богатства, славы и царства.
\myemph{\footnotesize \mbox{Иов. 1:1--22}}}

\pripevpomiluj

\minicolumns{\firstletter{А}ще праведен бяше он и непорочен паче всех, и не убеже ловления льстиваго и сети; ты же, грехолюбива сущи, окаянная душе, что сотвориши, аще чесому от недоведомых случится наити тебе?}{Если он, будучи праведным и безукоризненным более всех, не избежал козней и сетей обольстителя диавола, то что сделаешь, ты, грехолюбивая несчастная душа, если что-нибудь неожиданное постигнет тебя?}

\pripevpomiluj

\minicolumns{\firstletter{В}ысокоглаголив ныне есмь, жесток же и сердцем, вотще и всуе, да не с фарисеем осудиши мя. Паче же мытарево смирение подаждь ми, Едине Щедре, Правосуде, и сему мя сочисли.}{Высокомерен я ныне на словах, дерзок и в сердце, напрасно и тщетно; не осуди меня с фарисеем, но даруй мне смирение мытаря и к нему причисли, Один Милосердный и Правосудный.}

\pripevpomiluj

\minicolumns{\firstletter{С}огреших, досадив сосуду плоти моея, вем, Щедре, но в покаянии мя приими и в разум призови, да не буду стяжание ни брашно чуждему, Спасе, Сам мя ущедри.}{Знаю, Милосердный, согрешил я, осквернив сосуд моей плоти, но прими меня в покаянии и призови к познанию истины, чтобы не сделаться мне добычею и пищею врага; Сам, Ты, Спаситель, умилосердись надо мною.}

\pripevpomiluj

\minicolumns{\firstletter{С}амоистукан бых страстьми, душу мою вредя, Щедре, но в покаянии мя приими и в разум призови, да не буду стяжание ни брашно чуждему, Спасе, Сам мя ущедри.}{Истуканом я сделал сам себя, исказив душу свою страстями, Милосердный; но прими меня в покаянии и призови к познанию истины, чтобы не сделаться мне добычею и пищею врага; Сам, Ты, Спаситель, умилосердись надо мною.}

\pripevpomiluj

\minicolumns{\firstletter{Н}е послушах гласа Твоего, преслушах Писание Твое, Законоположника, но в покаянии мя приими и в разум призови, да не буду стяжание ни брашно чуждему, Спасе, Сам мя ущедри.}{Не послушал я голоса Твоего, нарушил Писание Твое, Законодатель; но прими меня в покаянии и призови к познанию истины, чтобы не сделаться мне добычею и пищею врага; Сам, Ты, Спаситель, умилосердись надо мною.}

\pripevmskipc{\pripev{\firstletter{П}реподобная мати Марие, моли Бога о нас.}}

\minicolumns{\firstletter{В}еликих безместий во глубину низведшися, неодержима была еси; но востекла еси помыслом лучшим к крайней деяньми яве добродетели преславно, ангельское естество, Марие, удививши.}{Увлекшись в глубину великих пороков, ты, Мария, не погрязла в ней, но высшим помыслом через деятельность явно поднялась до совершенной добродетели, дивно изумив ангельскую природу.}

\slavac

\minicolumns{\firstletter{Н}ераздельное Существом, Неслитное Лицы богословлю Тя, Троическое Едино Божество, яко Единоцарственное и Сопрестольное, вопию Ти песнь великую, в вышних трегубо песнословимую.}{Нераздельным по существу, неслиянным в Лицах богословски исповедую Тебя, Троичное Единое Божество, Соцарственное и Сопрестольное; возглашаю Тебе великую песнь, в небесных обителях троекратно воспеваемую.
\myemph{\footnotesize \mbox{Ис. 6:1--3}}}

\inynec

\minicolumns{\firstletter{И} раждаеши, и девствуеши, и пребываеши обоюду естеством Дева, Рождейся обновляет законы естества, утроба же раждает нераждающая. Бог идеже хощет, побеждается естества чин: творит бо, елика хощет.}{И рождаешь Ты, и остаешься Девою, в обоих случаях сохраняя по естеству девство. Рожденный Тобою обновляет законы природы, а девственное чрево рождает; когда хочет Бог, то нарушается порядок природы, ибо Он творит, что хочет.}

\mysubsubsection{Песнь 5}

\pripevc{\myemph{Ирм\'{о}с:}}

\minicolumns{\firstletter{О}т нощи утренююща, Человеколюбче, просвети, молюся, и настави и мене на повеления Твоя, и научи мя, Спасе, творити волю Твою.}{От ночи бодрствующего, просвети меня, молю, Человеколюбец, путеводи меня в повелениях Твоих и научи меня, Спаситель, исполнять Твою волю.
\myemph{\footnotesize \mbox{Пс. 62:2}; \mbox{Пс. 118:35}}}

\pripevpomiluj

\minicolumns{\firstletter{Н}изу сничащую подражай, о душе, прииди, припади к ногама Иисусовыма, да тя исправит, и да ходиши право стези Господни.}{Подражай, душа, скорченной жене, приди, припади к ногам Иисуса, чтобы Он исправил тебя и ты могла ходить прямо по стезям Господним.
\myemph{\footnotesize \mbox{Лк. 13:11--13}}}

\pripevpomiluj

\minicolumns{\firstletter{А}ще и кладязь еси глубокий, Владыко, источи ми воду из пречистых Твоих жил, да, яко самаряныня, не ктому, пияй, жажду: жизни бо струи источаеши.}{Если Ты "--- и глубокий колодец, Владыко, то источи мне струи из пречистых ребр Своих, чтобы я, как самарянка, испив, уже не жаждал, ибо Ты источаешь потоки жизни.
\myemph{\footnotesize \mbox{Ин. 4:11--15}}}

\pripevpomiluj

\minicolumns{\firstletter{С}илоам да будут ми слезы моя, Владыко Господи, да умыю и аз зеницы сердца, и вижду Тя, умна Света превечна.}{Силоамом да будут мне слезы мои, Владыко Господи, чтобы и мне омыть очи сердца и умственно созерцать Тебя, Предвечный Свет.
\myemph{\footnotesize \mbox{Ин. 9:7}}}

\pripevmskipc{\pripev{\firstletter{П}реподобная мати Марие, моли Бога о нас.}}

\minicolumns{\firstletter{Н}есравненным желанием, всебогатая, древу возжелевши поклонитися животному, сподобилася еси желания, сподоби убо и мене улучити вышния славы.}{С чистой любовию возжелав поклониться Древу Жизни, всеблаженная, ты удостоилась желаемого; удостой же и меня достигнуть высшей славы.}

\slavac

\minicolumns{\firstletter{Т}я, Тр\'{о}ице, славим Единаго Бога: Свят, Свят, Свят еси, Отче, Сыне и Душе, Пр\'{о}стое Существо, Ед\'{и}нице присно покланяемая.}{Тебя, Пресвятая Троица, прославляем за Единого Бога: Свят, Свят, Свят Отец, Сын и Дух, Простое Существо, Единица вечно поклоняемая.}

\inynec

\minicolumns{\firstletter{И}з Тебе облечеся в мое смешение, нетленная, безмужная Мати Дево, Бог, создавый веки, и соедини Себе человеческое естество.}{В Тебе, Нетленная, не познавшая мужа Матерь-Дево, облекся в мой состав сотворивший мир Бог и соединил с Собою человеческую природу.}

\mysubsubsection{Песнь 6}

\pripevc{\myemph{Ирм\'{о}с:}}

\minicolumns{\firstletter{В}озопих всем сердцем моим к щедрому Богу, и услыша мя от ада преисподняго, и возведе от тли живот мой.}{От всего сердца моего я воззвал к милосердному Богу, и Он услышал меня из ада преисподнего и воззвал жизнь мою от погибели.}

\pripevpomiluj

\minicolumns{\firstletter{А}з есмь, Спасе, юже погубил еси древле царскую драхму; но вжег светильник, Предтечу Твоего, Слове, взыщи и обрящи Твой образ.}{Я "--- та драхма с царским изображением, которая с древности потеряна у Тебя, Спаситель, но, засветив светильник "--- Предтечу Своего, Слове, поищи и найди Свой образ.}

\pripevpomiluj

\minicolumns{\firstletter{В}остани и побори, яко Иисус Амалика, плотския страсти, и гаваониты, лестныя помыслы, присно побеждающи.}{Восстань и низложи плотские страсти, как Иисус Амалика, всегда побеждая и гаваонитян "--- обольстительные помыслы.
\myemph{\footnotesize \mbox{Исх. 17:8}; \mbox{Нав. 8:21}}}

\pripevmskipc{\pripev{\firstletter{П}реподобная мати Марие, моли Бога о нас.}}

\minicolumns{\firstletter{Д}а страстей пламень угасиши, слез капли источала еси присно, Марие, душею распаляема, ихже благодать подаждь и мне, твоему рабу.}{Чтобы угасить пламень страстей, ты, Мария, пылая душой, непрестанно проливала потоки слез, преизобилие которых даруй и мне, рабу твоему.}

\pripevmskipc{\pripev{\firstletter{П}реподобная мати Марие, моли Бога о нас.}}

\minicolumns{\firstletter{Б}езстрастие Небесное стяжала еси крайним на земли житием, мати. Темже тебе поющим страстей избавитися молитвами твоими молися.}{Возвышеннейшим образом жизни на земле, ты, матерь, приобрела небесное бесстрастие; поэтому ходатайствуй, чтобы воспевающие тебя избавились от страстей по твоим молитвам.}

\slavac

\minicolumns{\firstletter{Т}р\'{о}ица есмь Пр\'{о}ста, Нераздельна, раздельна Личне и Ед\'{и}ница есмь естеством соединена, Отец глаголет, и Сын, и Божественный Дух.}{Я "--- Троица Несоставная, Нераздельная, раздельная в Лицах, и Единица, соединенная по существу; свидетельствует Отец, Сын и Божественный Дух.}

\inynec

\minicolumns{\firstletter{У}троба Твоя Бога нам роди, воображена по нам: Егоже, яко Создателя всех, моли, Богородице, да Твоими молитвами оправдимся.}{Чрево Твое родило нам Бога, принявшего наш образ; Его, как Создателя всего мира, моли, Богородица, чтобы по молитвам Твоим нам оправдаться.}

\pripevmskipc{\firstletter{Г}осподи, помилуй. \myemph{Трижды.}}

\pripevmskipc{\slavainynen}

\mysubsubsection{Кондак, глас 6:}

\minicolumns{\firstletter{Д}уше моя, душе моя, востани, что спиши? конец приближается, и имаши смутитися: воспряни убо, да пощадит тя Христос Бог, везде сый и вся исполняяй.}{Душа моя, душа моя, восстань, что ты спишь? Конец приближается, и ты смутишься; пробудись же, чтобы пощадил тебя Христос Бог, Вездесущий и все наполняющий.}

\mysubsubsection{Песнь 7}

\pripevc{\myemph{Ирм\'{о}с:}}

\minicolumns{\firstletter{С}огрешихом, беззаконновахом, неправдовахом пред Тобою, ниже соблюдохом, ниже сотворихом, якоже заповедал еси нам; но не предаждь нас до конца, отцев Боже.}{Мы согрешили, жили беззаконно, неправо поступали пред Тобою, не сохранили, не исполнили, что Ты заповедал нам; но не оставь нас до конца, Боже отцов.
\myemph{\footnotesize \mbox{Дан. 9:5--6}}}

\pripevpomiluj

\minicolumns{\firstletter{И}счезоша дние мои, яко соние востающаго; темже, яко Езекия, слезю на ложи моем, приложитися мне летом живота. Но кий Исаия предстанет тебе, душе, аще не всех Бог?}{Дни мои прошли как сновидение пробуждающегося; поэтому, подобно Езекии, я плачу на ложе моем, чтобы продлились годы жизни моей; но какой Исаия посетит тебя, душа, если не Бог всех?
\myemph{\footnotesize \mbox{4 Цар. 20:3}; \mbox{Ис. 38:2--6}}}

\pripevpomiluj

\minicolumns{\firstletter{П}рипадаю Ти и приношу Тебе, якоже слезы, глаголы моя: согреших, яко не согреши блудница, и беззаконновах, яко иный никтоже на земли. Но ущедри, Владыко, творение Твое и воззови мя.}{Припадаю к Тебе и приношу Тебе со слезами слова мои: согрешил я, как не согрешила блудница, и жил в беззакониях, как никто другой на земле; но умилосердись, Владыка, над созданием Своим и восстанови меня.}

\pripevpomiluj

\minicolumns{\firstletter{П}огребох образ Твой и растлих заповедь Твою, вся помрачися доброта, и страстьми угасися, Спасе, свеща. Но ущедрив, воздаждь ми, якоже поет Давид, радование.}{Затмил я образ Твой и нарушил заповедь Твою; вся красота помрачилась во мне, и светильник погас от страстей; но умилосердись, Спаситель, и возврати мне, как поет Давид, веселие.
\myemph{\footnotesize \mbox{Пс. 50:14}}}

\pripevpomiluj

\minicolumns{\firstletter{О}братися, покайся, открый сокровенная, глаголи Богу, вся ведущему: Ты веси моя тайная, Едине Спасе. Но Сам мя помилуй, якоже поет Давид, по милости Твоей.}{Обратись, покайся, открой сокровенное, скажи Богу Всеведущему: Спаситель, Ты Один знаешь мои тайны, но Сам помилуй меня, как поет Давид, по Твоей милости.
\myemph{\footnotesize \mbox{Пс. 50:3}}}

\pripevmskipc{\pripev{\firstletter{П}реподобная мати Марие, моли Бога о нас.}}

\minicolumns{\firstletter{В}озопивши к Пречистей Богоматери, первее отринула еси неистовство страстей, нужно стужающих, и посрамила еси врага запеншаго. Но даждь ныне помощь от скорби и мне, рабу твоему.}{Воззвавши к Пречистой Богоматери, ты обуздала неистовство страстей, прежде жестоко свирепствовавших, и посрамила врага-обольстителя; даруй же ныне помощь в скорби и мне, рабу твоему.
\myemph{\footnotesize \mbox{Пс. 59:13}}}

\pripevmskipc{\pripev{\firstletter{П}реподобная мати Марие, моли Бога о нас.}}

\minicolumns{\firstletter{Е}гоже возлюбила еси, Егоже возжелела еси, Егоже ради плоть изнурила еси, преподобная, моли ныне Христа о рабех: яко да милостив быв всем нам, мирное состояние дарует почитающим Его.}{Кого ты возлюбила, Кого избрала, для Кого изнуряла плоть, Преподобная, моли ныне Христа о рабах твоих, чтобы Он по Своей милости ко всем даровал мирное состояние почитающим Его.}

\slavac

\minicolumns{\firstletter{Т}р\'{о}ице Пр\'{о}стая, Нераздельная, Единосущная и Естество Едино, Светове и Свет, и Свята Три, и Едино Свято поется Бог Тр\'{о}ица; но воспой, прослави Живот и Животы, душе, всех Бога.}{Троица Простая, Нераздельная, Единосущная, и Одно Божество, Светы и Свет, Три Святы и Одно Лицо Свято, Бог-Троица, воспеваемая в песнопениях; воспой же и ты, душа, прославь Жизнь и Жизни "--- Бога всех.}

\inynec

\minicolumns{\firstletter{П}оем Тя, благословим Тя, покланяемся Ти, Богородительнице, яко Неразлучныя Тр\'{о}ицы породила еси Единаго Христа Бога и Сама отверзла еси нам, сущим на земли, Небесная.}{Воспеваем Тебя, благословляем Тебя, поклоняемся Тебе, Богородительница, ибо Ты родила Одного из Нераздельной Троицы, Христа Бога, и Сама открыла для нас, живущих на земле, небесные обители.}

\mysubsubsection{Песнь 8}

\pripevc{\myemph{Ирм\'{о}с:}}

\minicolumns{\firstletter{Е}гоже воинства Небесная славят, и трепещут херувими и серафими, всяко дыхание и тварь, пойте, благословите и превозносите во вся веки.}{Кого прославляют воинства небесные и пред Кем трепещут Херувимы и Серафимы, Того, все существа и творения, воспевайте, благословляйте и превозносите во все века.}

\pripevpomiluj

\minicolumns{\firstletter{С}лезную, Спасе, сткляницу яко миро истощавая на главу, зову Ти, якоже блудница, милости ищущая, мольбу приношу и оставление прошу прияти.}{Изливая сосуд слез, как миро на голову, Спаситель, взываю к Тебе, как ищущая милости блудница, приношу моление и прошу о получении мне прощения.
\myemph{\footnotesize \mbox{Мф. 26:6--7}; \mbox{Мк. 14:3}; \mbox{Лк. 7:37--38}}}

\pripevpomiluj

\minicolumns{\firstletter{А}ще и никтоже, якоже аз, согреши Тебе, но обаче приими и мене, благоутробне Спасе, страхом кающася и любовию зовуща: согреших Тебе Единому, помилуй мя, Милостиве.}{Хотя никто не согрешил пред Тобою, как я, но, Милосердный Спаситель, приими меня, кающегося со страхом и с любовию взывающего: я согрешил пред Тобою Одним, помилуй меня, Милосердный!}

\pripevpomiluj

\minicolumns{\firstletter{П}ощади, Спасе, Твое создание и взыщи, яко Пастырь, погибшее, предвари заблуждшаго, восхити от волка, сотвори мя овча на пастве Твоих овец.}{Пощади, Спаситель, создание Свое и, как Пастырь, отыщи потерянного, возврати заблудшего, отними у волка и сделай меня агнцем на пастбище Твоих овец.
\myemph{\footnotesize \mbox{Пс. 118:176}}}

\pripevpomiluj

\minicolumns{\firstletter{Е}гда, Судие, сядеши, яко благоутробен, и покажеши страшную славу Твою, Спасе, о каковый страх тогда, пещи горящей, всем боящимся нестерпимаго судища Твоего.}{Когда Ты, Милосердный, воссядешь, как Судия и откроешь грозное величие Свое, Спаситель, о, какой ужас тогда: печь будет гореть, и все трепетать пред неумолимым судом Твоим.
\myemph{\footnotesize \mbox{Мф. 25:31}; \mbox{Мф. 25:41}; \mbox{Мф. 25:47}}}

\pripevmskipc{\pripev{\firstletter{П}реподобная мати Марие, моли Бога о нас.}}

\minicolumns{\firstletter{С}вета незаходимаго Мати тя просветивши, от омрачения страстей разреши. Темже вшедши в духовную благодать, просвети, Марие, тя верно восхваляющия.}{Матерь незаходимаго Света "--- Христа, просветив тебя, освободила от мрака страстей; поэтому, приняв благодать Духа, просвети, Мария, искренно прославляющих тебя.}

\pripevmskipc{\pripev{\firstletter{П}реподобная мати Марие, моли Бога о нас.}}

\minicolumns{\firstletter{Ч}удо ново видев, ужасашеся божественный в тебе воистинну, мати, Зосима: ангела бо зряше во плоти и ужасом весь исполняшеся, Христа поя во веки.}{Увидев в тебе, матерь, поистине новое чудо, святой Зосима удивился, ибо он увидел Ангела во плоти, и весь преисполнился изумлением, воспевая Христа вовеки.}

\slavac

\minicolumns{\firstletter{Б}езначальне Отче, Сыне Собезначальне, Утешителю Благий, Душе Правый, Слова Божия Родителю, Отца Безначальна Слове, Душе Живый и Зиждяй, Тр\'{о}ице Ед\'{и}нице, помилуй мя.}{Безначальный Отче, Собезначальный Сын, Утешитель Благий, Дух Правый, Родитель Слова Божия, Безначальное Слово Отца, Дух, Животворящий и Созидающий, Троица Единая, помилуй меня.}

\inynec

\minicolumns{\firstletter{Я}ко от оброщения червленицы, Пречистая, умная багряница Еммануилева внутрь во чреве Твоем плоть исткася. Темже Богородицу воистинну Тя почитаем.}{Мысленная порфира "--- плоть Еммануила соткалась внутри Твоего чрева. Пречистая, как бы из вещества пурпурного; потому мы почитаем Тебя, Истинную Богородицу.}

\mysubsubsection{Песнь 9}

\pripevc{\myemph{Ирм\'{о}с:}}

\minicolumns{\firstletter{Б}езсеменнаго зачатия Рождество несказанное, Матере безмужныя нетленен Плод, Божие бо Рождение обновляет естества. Темже Тя вси роди, яко Богоневестную Матерь, православно величаем.}{Рождество от бессеменного зачатия неизъяснимо, безмужной Матери нетленен Плод, ибо рождение Бога обновляет природу. Поэтому Тебя, как Богоневесту-Матерь мы, все роды, православно величаем.}

\pripevpomiluj

\minicolumns{\firstletter{У}милосердися, спаси мя, Сыне Давидов, помилуй, беснующияся словом исцеливый, глас же благоутробный, яко разбойнику, мне рцы: аминь, глаголю тебе, со Мною будеши в раи, егда прииду во славе Моей.}{Умилосердись, спаси и помилуй меня, Сын Давидов, словом исцелявший беснующихся, и скажи, как разбойнику, милостивые слова: истинно говорю тебе, со Мною будешь в раю, когда приду Я в славе Моей.
\myemph{\footnotesize \mbox{Лк. 23:43}}}

\pripevpomiluj

\minicolumns{\firstletter{Р}азбойник оглаголоваше Тя, разбойник богословяше Тя: оба бо на кресте свисяста. Но, о Благоутробне, яко верному разбойнику Твоему, познавшему Тя Бога, и мне отверзи дверь славнаго Царствия Твоего.}{Разбойник поносил Тебя, разбойник же и Богом исповедал Тебя, вися оба на кресте; но, Милосердный, как уверовавшему разбойнику, познавшему в Тебе Бога, открой и мне, дверь славного Твоего Царства.}

\pripevpomiluj

\minicolumns{\firstletter{Т}варь содрогашеся, распинаема Тя видящи, горы и камения страхом распадахуся, и земля сотрясашеся, и ад обнажашеся, и соомрачашеся свет во дни, зря Тебе, Иисусе, пригвождена ко Кресту.}{Тварь содрогалась, видя Тебя распинаемым, горы и камни от ужаса распадались и колебалась земля, преисподняя пустела, и свет среди дня помрачался, взирая на Тебя, Иисус, плотию ко кресту пригвожденного.
\myemph{\footnotesize \mbox{Мф. 27:51--52}; \mbox{Мк. 15:38}; \mbox{Лк. 23:45}}}

\pripevpomiluj

\minicolumns{\firstletter{Д}остойных покаяния плодов не истяжи от мене, ибо крепость моя во мне оскуде; сердце мне даруй присно сокрушенное, нищету же духовную: да сия Тебе принесу яко приятную жертву, Едине Спасе.}{Достойных плодов покаяния не требуй от меня, Единый Спаситель, ибо сила моя истощилась во мне; даруй мне всегда сокрушенное сердце и духовную нищету, чтобы я принес их Тебе, как благоприятную жертву.}

\pripevpomiluj

\minicolumns{\firstletter{С}удие мой и Ведче мой, хотяй паки приити со ангелы, судити миру всему, милостивным Твоим оком тогда видев мя, пощади и ущедри мя, Иисусе, паче всякаго естества человеча согрешивша.}{Судия мой, знающий меня, когда опять придешь Ты с Ангелами, чтобы судить весь мир, тогда, обратив на меня милостивый взор, пощади, Иисусе, и помилуй меня, согрешившего более всего человеческого рода.}

\pripevmskipc{\pripev{\firstletter{П}реподобная мати Марие, моли Бога о нас.}}

\minicolumns{\firstletter{У}дивила еси всех странным житием твоим, ангелов чины и человеков соборы, невещественно поживши и естество прешедши: имже, яко невещественныма ногама вшедши, Марие, Иордан прешла еси.}{Ты удивила необычайною своею жизнью всех, как чины ангельские, так и человеческие сонмы, духовно пожив и превзошедши природу; поэтому, Мария, ты, как бесплотная, шествуя стопами, перешла Иордан.}

\pripevmskipc{\pripev{\firstletter{П}реподобная мати Марие, моли Бога о нас.}}

\minicolumns{\firstletter{У}милостиви Создателя о хвалящих тя, преподобная мати, избавитися озлоблений и скорбей, окрест нападающих: да избавившеся от напастей, возвеличим непрестанно прославльшаго тя Господа.}{Склони Творца на милость к восхваляющим тебя, преподобная матерь, чтобы нам избавиться от огорчений и скорбей, отовсюду нападающих на нас, чтобы, избавившись от искушений, мы непрестанно величали прославившего тебя Господа.}

\pripevmskipc{\pripev{\firstletter{П}реподобне отче Андрее, моли Бога о нас.}}

\minicolumns{\firstletter{А}ндрее честный и отче треблаженнейший, пастырю Критский, не престай моляся о воспевающих тя: да избавимся вси гнева, и скорби, и тления, и прегрешений безмерных, чтущии твою память верно.}{Андрей досточтимый, отец преблаженный, пастырь Критский, не переставай молиться за воспевающих тебя, чтобы избавиться от гнева, скорби, погибели и бесчисленных прегрешений нам всем, искренно почитающим память твою.}

\slavac

\minicolumns{\firstletter{О}тца прославим, Сына превознесем, Божественному Духу верно поклонимся, Тр\'{о}ице Нераздельней, Ед\'{и}нице по Существу, яко Свету и Светом, и Животу и Животом, Животворящему и Просвещающему концы.}{Прославим Отца, превознесем Сына, с верою поклонимся Божественному Духу, Нераздельной Троице, Единой по существу, как Свету и Светам, Жизни и Жизням, Животворящему и Просвещающему пределы вселенной.}

\inynec

\minicolumns{\firstletter{Г}рад Твой сохраняй, Богородительнице Пречистая, в Тебе бо сей верно царствуяй, в Тебе и утверждается, и Тобою побеждаяй, побеждает всякое искушение, и пленяет ратники, и проходит послушание.}{Сохраняй град Свой, Пречистая Богородительница. Под Твоею защитою он царствует с верою, и от Тебя получает крепость, и при Твоем содействии неотразимо побеждает всякое бедствие, берет в плен врагов и держит их в подчинении.}

\pripevmskipc{\myemph{\firstletter{Т}аже оба лика вкупе поют Ирм\'{о}с:}}

\minicolumns{\firstletter{Б}езсеменнаго зачатия Рождество несказанное, Матере безмужныя нетленен Плод, Божие бо Рождение обновляет естества. Темже Тя вси роди, яко Богоневестную Матерь, православно величаем.}{Рождество от бессеменного зачатия неизъяснимо, безмужной Матери нетленен Плод, ибо рождение Бога обновляет природу. Поэтому Тебя, как Богоневесту-Матерь мы, все роды, православно величаем.}

\end{Parallel}

\mychapterending

\mychapter{В четверг пятой седмицы Великого Поста}
%http://www.molitvoslov.com/text575.htm 
 
\begin{Parallel}{}{}

\mysubsection{Славянский текст с русским переводом и ссылками на текст Библии}

\mysubsubsection{Песнь 1}

\pripevc{\myemph{Ирм\'{о}с:}}

\minicolumns{\firstletter{П}омощник и Покровитель бысть мне во спасение, Сей мой Бог, и прославлю Его, Бог Отца моего, и вознесу Его: славно бо прославися.}{\firstletter{П}омощник и Покровитель явился мне ко спасению, Он Бог мой, и прославлю Его, Бога отца моего, и превознесу Его, ибо Он торжественно прославился.
\myemph{\footnotesize \mbox{Исх. 15:1--2}}}

\pripevpomiluj

\minicolumns{\firstletter{О}ткуду начну плакати окаяннаго моего жития деяний? Кое ли положу начало, Христе, нынешнему рыданию? Но, яко благоутробен, даждь ми прегрешений оставление.}{\firstletter{С} чего начну я оплакивать деяния злосчастной моей жизни? Какое начало положу, Христе, я нынешнему моему сетованию? Но Ты, как милосердный, даруй мне оставление прегрешений.}

\pripevpomiluj

\minicolumns{\firstletter{Г}ряди, окаянная душе, с плотию твоею, Зиждителю всех исповеждься, и останися прочее преждняго безсловесия, и принеси Богу в покаянии слезы.}{\firstletter{П}рииди, несчастная душа, с плотию своею, исповедайся Создателю всего, воздержись, наконец, от прежнего безрассудства и с раскаянием принеси Богу слезы.}

\pripevpomiluj

\minicolumns{\firstletter{П}ервозданнаго Адама преступлению поревновав, познах себе обнажена от Бога и присносущнаго Царствия и сладости, грех ради моих.}{\firstletter{П}одражая в преступлении первозданному Адаму, я сознаю себя лишенным Бога, вечного Царства и блаженства за мои грехи.
\myemph{\footnotesize \mbox{Быт. 3:6--7}}}

\pripevpomiluj

\minicolumns{\firstletter{У}вы мне, окаянная душе, что уподобилася еси первей Еве? Видела бо еси зле, и уязвилася еси горце, и коснулася еси древа, и вкусила еси дерзостно безсловесныя снеди.}{\firstletter{Г}оре мне, моя несчастная душа, для чего ты уподобилась первосозданной Еве? Не с добром ты посмотрела и уязвилась жестоко, прикоснулась к дереву и дерзостно вкусила бессмысленного плода.
\myemph{\footnotesize \mbox{Быт. 3:6}}}

\pripevpomiluj

\minicolumns{\firstletter{В}место Евы чувственныя, мысленная ми бысть Ева, во плоти страстный помысл, показуяй сладкая и вкушаяй присно горькаго напоения.}{\firstletter{В}место чувственной Евы восстала во мне Ева мысленная "--- плотский страстный помысел, обольщающий приятным, но при вкушении всегда напояющий горечью.}

\pripevpomiluj

\minicolumns{\firstletter{Д}остойно из Едема изгнан бысть, яко не сохранив едину Твою, Спасе, заповедь Адам; аз же что постражду, отметая всегда животная Твоя словеса?}{\firstletter{Д}остойно был изгнан из Едема Адам, как не сохранивший одной Твоей заповеди, Спаситель. Что же должен претерпеть я, всегда отвергающий Твои животворные повеления?
\myemph{\footnotesize \mbox{Быт. 3:23}}}

\pripevpomiluj

\minicolumns{\firstletter{К}аиново прешед убийство, произволением бых убийца совести душевней, оживив плоть и воевав на ню лукавыми моими деяньми.}{\firstletter{П}ревзойдя Каиново убийство, сознательным произволением, через оживление греховной плоти, я сделался убийцею души, вооружившись против нее злыми моими делами.
\myemph{\footnotesize \mbox{Быт. 4:8}}}

\pripevpomiluj

\minicolumns{\firstletter{А}велеве, Иисусе, не уподобихся правде, дара Тебе приятна не принесох когда, ни деяния Божественна, ни жертвы чистыя, ни жития непорочнаго.}{\firstletter{А}велевой праведности, Иисусе, я не подражал, никогда не приносил Тебе приятных даров, ни дел богоугодных, ни жертвы чистой, ни жизни непорочной.
\myemph{\footnotesize \mbox{Быт. 4:3--4}}}

\pripevpomiluj

\minicolumns{\firstletter{Я}ко Каин и мы, душе окаянная, всех Содетелю деяния скверная, и жертву порочную, и непотребное житие принесохом вкупе: темже и осудихомся.}{\firstletter{К}ак Каин, так и мы, несчастная душа, принесли Создателю всего жертву порочную "--- дела нечестивые и жизнь невоздержанную: поэтому мы и осуждены.
\myemph{\footnotesize \mbox{Быт. 4:5}}}

\pripevpomiluj

\minicolumns{\firstletter{Б}рение Здатель живосоздав, вложил еси мне плоть и кости, и дыхание, и жизнь; но, о Творче мой, Избавителю мой и Судие, кающася приими мя.}{\firstletter{О}животворивший земной прах, Скудельник, Ты даровал мне плоть и кости, дыхание и жизнь; но, Творец мой, Искупитель мой и Судия, приими меня кающегося!
\myemph{\footnotesize \mbox{Быт. 2:7}}}

\pripevpomiluj

\minicolumns{\firstletter{И}звещаю Ти, Спасе, грехи, яже содеях, и души и тела моего язвы, яже внутрь убийственнии помыслы разбойнически на мя возложиша.}{\firstletter{П}ред Тобою, Спаситель, открываю грехи, сделанные мною, и раны души и тела моего, которые разбойнически нанесли мне внутренние убийственные помыслы.
\myemph{\footnotesize \mbox{Лк. 10:30}}}

\pripevpomiluj

\minicolumns{\firstletter{А}ще и согреших, Спасе, но вем, яко Человеколюбец еси, наказуеши милостивно и милосердствуеши тепле, слезяща зриши и притекаеши, яко отец, призывая блуднаго.}{\firstletter{Х}отя я и согрешил, Спаситель, но знаю, что Ты человеколюбив; наказываешь с состраданием и милуешь с любовью, взираешь на плачущего и спешишь, как Отец, призвать блудного.
\myemph{\footnotesize \mbox{Лк. 15:20}}}

\pripevpomiluj

\minicolumns{\firstletter{П}овержена мя, Спасе, пред враты Твоими, поне на старость не отрини мене во ад тща, но прежде конца, яко Человеколюбец, даждь ми прегрешений оставление.}{\firstletter{П}оверженного пред вратами Твоими, Спаситель, хотя в старости, не низринь меня в ад, как невоздержанного, но прежде кончины, как Человеколюбец, даруй мне оставление прегрешений.}

\pripevpomiluj

\minicolumns{\firstletter{В} разбойники впадый аз есмь помышленьми моими, весь от них уязвихся ныне и исполнихся ран, но, Сам ми представ, Христе Спасе, исцели.}{\firstletter{П}о помыслам моим я человек, попавшийся разбойникам; теперь я весь изранен ими, покрыт язвами, но Ты Сам, Христос Спаситель, приди и исцели меня.
\myemph{\footnotesize \mbox{Лк. 10:30}}}

\pripevpomiluj

\minicolumns{\firstletter{С}вященник, мя предвидев мимо иде, и левит, видев в лютых нага, презре, но из Марии возсиявый Иисусе, Ты, представ, ущедри мя.}{\firstletter{С}вященник, заметив меня, прошел мимо, и левит, видя меня в беде обнаженного, презрел; но Ты, воссиявший от Марии Иисусе, прииди и умилосердись надо мною.
\myemph{\footnotesize \mbox{Лк. 10:31--32}}}

\pripevpomiluj

\minicolumns{\firstletter{А}гнче Божий, вземляй грехи всех, возми бремя от мене тяжкое греховное, и яко благоутробен, даждь ми слезы умиления.}{\firstletter{А}гнец Божий, взявший грехи всех, сними с меня тяжкое бремя греховное и, как Милосердный, даруй мне слезы умиления.
\myemph{\footnotesize \mbox{Ин. 1:29}}}

\pripevpomiluj

\minicolumns{\firstletter{П}окаяния время, прихожду Ти, Создателю моему: возми бремя от мене тяжкое греховное и яко благоутробен, даждь ми слезы умиления.}{\firstletter{В}ремя покаяния: к Тебе прихожу, моему Создателю, сними с меня тяжкое бремя греховное и, как Милосердный, даруй мне слезы умиления.}

\pripevpomiluj

\minicolumns{\firstletter{Н}е возгнушайся мене, Спасе, не отрини от Твоего лица, возьми бремя от мене тяжкое греховное и, яко благоутробен, даждь мне грехопадений оставление.}{\firstletter{Н}е погнушайся мной, Спаситель, не отвергни от Твоего лица, сними с меня тяжкое бремя греховное и, как благосердный, даруй мне освобождение от грехопадений.}

\pripevpomiluj

\minicolumns{\firstletter{В}ольная, Спасе, и невольная прегрешения моя, явленная и сокровенная и ведомая и неведомая, вся простив, яко Бог, очисти и спаси мя.}{\firstletter{С}отворенные по своей воле и невольные прегрешения мои, Спаситель, явные и скрытые, те, о которых знаю и о которых не знаю, все простив, как Бог, изгладь и спаси меня.}

\pripevpomiluj

\minicolumns{\firstletter{О}т юности, Христе, заповеди Твоя преступих, всестрастно небрегий, унынием преидох житие. Темже зову Ти, Спасе: поне на конец спаси мя.}{\firstletter{С} юности, Христе, я пренебрегал Твоими заповедями, всю жизнь провел в страстях, беспечности и нерадении. Поэтому и взываю к Тебе, Спаситель: хотя при кончине спаси меня.}

\pripevpomiluj

\minicolumns{\firstletter{Б}огатство мое, Спасе, изнурив в блуде, пуст есмь плодов благочестивых, алчен же зову: Отче щедрот, предварив Ты мя ущедри.}{\firstletter{Р}асточив богатство мое в распутстве, Спаситель, я чужд плодов благочестия, но, чувствуя голод, взываю: Отец Милосердный, поспеши и умилосердись надо мною.
\myemph{\footnotesize \mbox{Лк. 15:11--14}; \mbox{Лк. 15:17--18}}}

\pripevpomiluj

\minicolumns{\firstletter{Т}ебе припадаю, Иисусе, согреших Ти, очисти мя, возми бремя от мене тяжкое греховное и яко благоутробен, даждь ми слезы умиления.}{\firstletter{К} Тебе припадаю, Иисусе, согрешил я пред Тобою, умилосердись надо мною, сними с меня тяжкое бремя греховное и, как Милосердный, даруй мне слезы умиления.}

\pripevpomiluj

\minicolumns{\firstletter{Н}е вниди со мною в суд, нося моя деяния, словеса изыскуя и исправляя стремления. Но в щедротах Твоих презирая моя лютая, спаси мя, Всесильне.}{\firstletter{Н}е входи со мною в суд, взвешивая мои дела, исследуя слова и обличая стремления, но по Твоим щедротам презирая мои злодеяния, спаси меня, Всесильный.}

\mysubsubsection{Иный канон преподобныя матере нашея Марии Египетския, глас 6:}

\pripevmskipc{\pripev{\firstletter{П}реподобная мати Марие, моли Бога о нас.}}

\minicolumns{\firstletter{Т}ы ми даждь светозарную благодать от Божественнаго свыше Промышления избежати страстей омрачения и пети усердно Твоего, Марие, жития красная исправления.}{\firstletter{Д}аруй мне, Мария, ниспосланную тебе свыше Божественным Промыслом светозарную благодать "--- избежать мрака страстей и усердно воспеть прекрасные подвиги твоей жизни.}

\pripevmskipc{\pripev{\firstletter{П}реподобная мати Марие, моли Бога о нас.}}

\minicolumns{\firstletter{П}риклоньшися Христовым Божественным законом, к Сему приступила еси, сладостей неудержимая стремления оставивши, и всякую добродетель всеблагоговейно, яко едину, исправила еси.}{\firstletter{П}окорившись перед Божественными заповедями Христа, ты предалась Ему, оставив необузданные стремления к удовольствиям, и все добродетели, как одну, исполнила со всем благоговением.}

\pripevmskipc{\pripev{\firstletter{П}реподобне отче Андрее, моли Бога о нас.}}

\minicolumns{\firstletter{М}олитвами твоими нас, Андрее, избави страстей безчестных и Царствия ныне Христова общники верою и любовию воспевающия тя, славне, покажи, молимся.}{\firstletter{М}олитвами твоими, Андрей, избавь нас от позорных страстей и сделай, молимся, ныне участниками Царства Христова воспевающих с верой и любовию тебя, прославленный.}

\slavac

\minicolumns{\firstletter{П}ресущественная Троице, во Единице покланяемая, возми бремя от мене тяжкое греховное и, яко благоутробна, даждь ми слезы умиления.}{\firstletter{П}ресущественная Троица, Которой мы поклоняемся как Единому Существу, сними с меня тяжелое бремя греховное и, как Милосердная, даруй мне слезы умиления.}

\inynec

\minicolumns{\firstletter{Б}огородице, Надежде и Предстательство Тебе поющих, возми бремя от мене тяжкое греховное и, яко Владычица Чистая, кающася приими мя.}{\firstletter{Б}огородице, Надежда и Помощь всем воспевающих Тебя, сними с меня тяжелое бремя греховное и, как Владычица Непорочная, прими меня кающегося.}

\mysubsubsection{Песнь 2}

\pripevc{\myemph{Ирм\'{о}с:}}

\minicolumns{\firstletter{В}онми, небо, и возглаголю, и воспою Христа, от Девы плотию пришедшаго.}{\firstletter{В}немли, небо, я буду возвещать и воспевать Христа, пришедшего во плоти от Девы.}

\pripevpomiluj

\minicolumns{\firstletter{В}онми, небо, и возглаголю, земле внушай глас, кающийся к Богу и воспевающий Его.}{\firstletter{В}немли, небо, я буду возвещать: земля, услышь голос, кающийся Богу и прославляющий Его.}

\pripevpomiluj

\minicolumns{\firstletter{В}онми ми, Боже, Спасе мой, милостивым Твоим оком, и приими мое теплое исповедание.}{\firstletter{В}оззри на меня Боже, Спаситель мой, милостивым Твоим оком и прими мою пламенную исповедь.}

\pripevpomiluj

\minicolumns{\firstletter{С}огреших паче всех человек, един согреших Тебе; но ущедри яко Бог, Спасе, творение Твое.}{\firstletter{С}огрешил я более всех людей, один я согрешил пред Тобою; но, как Бог, сжалься, Спаситель, над Твоим созданием.
\myemph{\footnotesize \mbox{1 Тим. 1:15}}}

\pripevpomiluj

\minicolumns{\firstletter{Б}уря мя злых обдержит, благоутробне Господи; но яко Петру, и мне руку простри.}{\firstletter{Б}уря зла окружает меня, Милосердный Господи, но, как Петру, и мне Ты простри руку.
\myemph{\footnotesize \mbox{Мф. 14:31}}}

\pripevpomiluj

\minicolumns{\firstletter{С}лезы блудницы, Щедре, и аз предлагаю, очисти мя, Спасе, благоутробием Твоим.}{\firstletter{К}ак блудница, и я проливаю слезы, Милосердный; смилуйся надо мною, Спаситель, по благоснисхождению Твоему.
\myemph{\footnotesize \mbox{Лк. 7:37--38}}}

\pripevpomiluj

\minicolumns{\firstletter{О}мрачих душевную красоту страстей сластьми, и всячески весь ум персть сотворих.}{\firstletter{П}омрачил я красоту души страстными удовольствиями и весь ум совершенно превратил в прах.}

\pripevpomiluj

\minicolumns{\firstletter{Р}аздрах ныне одежду мою первую, юже ми изтка Зиждитель изначала, и оттуду лежу наг.}{\firstletter{Р}азодрал я первую одежду мою, которую вначале соткал мне Создатель, и оттого лежу обнаженным.}

\pripevpomiluj

\minicolumns{\firstletter{О}блекохся в раздранную ризу, юже изтка ми змий советом, и стыждуся.}{\firstletter{О}блекся я в разодранную одежду, которую соткал мне змий коварством, и стыжусь.
\myemph{\footnotesize \mbox{Быт. 3:7}}}

\pripevpomiluj

\minicolumns{\firstletter{В}оззрех на садовную красоту, и прельстихся умом; и оттуду лежу наг и срамляюся.}{\firstletter{В}зглянул я на красоту дерева и прельстился в уме; оттого лежу обнаженным и стыжусь.}

\pripevpomiluj

\minicolumns{\firstletter{Д}елаша на хребте моем вси начальницы страстей, продолжающе на мя беззаконие их.}{\firstletter{Н}а хребте моем пахали все вожди страстей, проводя вдоль по мне беззаконие свое.
\myemph{\footnotesize \mbox{Пс. 128:3}}}

\pripevpomiluj

\minicolumns{\firstletter{П}огубих первозданную доброту и благолепие мое и ныне лежу наг и стыждуся.}{\firstletter{П}огубил я первозданную красоту и благообразие мое и теперь лежу обнаженным и стыжусь.}

\pripevpomiluj

\minicolumns{\firstletter{С}шиваше кожныя ризы грех мне, обнаживый мя первыя боготканныя одежды.}{\firstletter{«}Кожаные ризы» сшил мне грех, сняв с меня прежнюю Богом сотканную одежду.
\myemph{\footnotesize \mbox{Быт. 3:11}; \mbox{Быт. 3:21}}}

\pripevpomiluj

\minicolumns{\firstletter{О}бложен есмь одеянием студа, якоже листвием смоковным, во обличение моих самовластных страстей.}{\firstletter{О}блекся я одеянием стыда, как листьями смоковницы, во обличение самовольных страстей моих.
\myemph{\footnotesize \mbox{Быт. 3:7}}}

\pripevpomiluj

\minicolumns{\firstletter{О}деяхся в срамную ризу и окровавленную студно течением страстнаго и любосластнаго живота.}{\firstletter{О}делся я в одежду, постыдно запятнанную и окровавленную нечистотой страстной и сластолюбивой жизни.}

\pripevpomiluj

\minicolumns{\firstletter{О}скверних плоти моея ризу, и окалях еже по образу, Спасе, и по подобию.}{\firstletter{О}сквернил я одежду плоти моей и очернил в себе, Спаситель, то, что было создано по Твоему образу и подобию..
\myemph{\footnotesize \mbox{Быт. 3:21}}}

\pripevpomiluj

\minicolumns{\firstletter{В}падох в страстную пагубу и в вещественную тлю, и оттоле до ныне враг мне досаждает.}{\firstletter{П}одвергся я мучению страстей и вещественному тлению, и оттого ныне враг угнетает меня.}

\pripevpomiluj

\minicolumns{\firstletter{Л}юбовещное и любоименное житие невоздержанием, Спасе, предпочет, ныне тяжким бременем обложен есмь.}{\firstletter{П}редпочтя нестяжательности жизнь, привязанную к земным вещам и любостяжательную, Спаситель, я теперь нахожусь под тяжким бременем.}

\pripevpomiluj

\minicolumns{\firstletter{У}красих плотский образ скверных помышлений различным обложением, и осуждаюся.}{\firstletter{У}красил я кумир плоти разноцветным одеянием гнусных помыслов и подвергаюсь осуждению.}

\pripevpomiluj

\minicolumns{\firstletter{В}нешним прилежно благоукрашением единем попекохся, внутреннюю презрев Богообразную скинию.}{\firstletter{У}сердно заботясь об одном внешнем благолепии, я пренебрег внутренней скинией, устроенной по образу Божию.
\myemph{\footnotesize \mbox{Исх. 25:8--9}}}

\pripevpomiluj

\minicolumns{\firstletter{В}ообразив моих страстей безобразие, любосластными стремленьми, погубих ума красоту.}{\firstletter{О}тобразив в себе безобразие моих страстей, сластолюбивыми стремлениями исказил я красоту ума.}

\pripevpomiluj

\minicolumns{\firstletter{П}огребох перваго образа доброту, Спасе, страстьми, юже яко иногда драхму взыскав, обрящи.}{\firstletter{З}асыпал страстями красоту первобытного образа, Спаситель; ее, как некогда драхму, Ты взыщи и найди.
\myemph{\footnotesize \mbox{Лк. 15:8}}}

\pripevpomiluj

\minicolumns{\firstletter{С}огреших, якоже блудница вопию Ти: един согреших Тебе, яко миро, приими, Спасе, и моя слезы.}{\firstletter{С}огрешил, и, как блудница, взываю к Тебе: один я согрешил пред Тобою, приими, Спаситель, и от меня слезы вместо мира.
\myemph{\footnotesize \mbox{Лк. 7:37--38}}}

\pripevpomiluj

\minicolumns{\firstletter{П}оползохся, яко Давид, блудно и осквернихся, но омый и мене, Спасе, слезами.}{\firstletter{О}т невоздержания, как Давид, я пал и осквернился, но омой и меня, Спаситель, слезами.
\myemph{\footnotesize \mbox{2 Цар. 11:4}}}

\pripevpomiluj

\minicolumns{\firstletter{О}чисти, якоже мытарь вопию Ти, Спасе, очисти мя: никтоже бо сущих из Адама, якоже аз, согреших Тебе.}{\firstletter{У}милостивись, как мытарь, взываю к Тебе, Спаситель, смилуйся надо мною: ибо как никто из потомков Адамовых я согрешил пред Тобою.
\myemph{\footnotesize \mbox{Лк. 18:13}}}

\pripevpomiluj

\minicolumns{\firstletter{Н}и слез, ниже покаяния имам, ниже умиления. Сам ми сия, Спасе, яко Бог, даруй.}{\firstletter{Н}и слез, ни покаяния, ни умиления нет у меня; Сам Ты, Спаситель, как Бог, даруй мне это.}

\pripevpomiluj

\minicolumns{\firstletter{Д}верь Твою не затвори мне тогда, Господи, Господи, но отверзи ми сию, кающемуся Тебе.}{\firstletter{Н}е затвори предо мною теперь дверь Твою, Господи, Господи, но отвори ее для меня, кающегося Тебе.
\myemph{\footnotesize \mbox{Мф. 7:21--23}; \mbox{Мф. 25:10--12}}}

\pripevpomiluj

\minicolumns{\firstletter{Ч}еловеколюбче, хотяй всем спастися, Ты воззови мя и приими яко благ кающагося.}{\firstletter{Ч}еловеколюбец, желающий всем спасения, Ты призови меня и прими, как Благий, кающегося.
\myemph{\footnotesize \mbox{1 Тим. 2:4}}}

\pripevpomiluj

\minicolumns{\firstletter{В}нуши воздыхания души моея и очию моею приими капли, Спасе, и спаси мя.}{\firstletter{В}немли, Спаситель, стенаниям души моей, прими слезы очей моих и спаси меня.}

\pripevmskipc{\pripev{\firstletter{П}ресвятая Богородице, спаси нас.}}

\minicolumns{\firstletter{П}речистая Богородице Дево, Едина Всепетая, моли прилежно во еже спастися нам.}{\firstletter{П}речистая Богородице Дева, Ты Одна, всеми воспеваемая, усердно моли о нашем спасении.}

\pripevmskipc{\myemph{Иный Ирм\'{о}с:}}

\minicolumns{\firstletter{В}идите, видите, яко Аз есмь Бог, манну одождивый и воду из камене источивый древле в пустыни людем Моим, десницею единою и крепостию Моею.}{\firstletter{В}идите, видите, что Я "--- Бог, в древности ниспославший манну и источивший воду из камня народу Моему в пустыне "--- одним Своим всемогуществом.
\myemph{\footnotesize \mbox{Исх. 16:14}; \mbox{Исх. 17:6}}}

\pripevpomiluj

\minicolumns{\firstletter{В}идите, видите, яко Аз есмь Бог, внушай, душе моя, Господа вопиюща, и удалися прежняго греха, и бойся яко неумытнаго и яко Судии и Бога.}{\firstletter{В}идите, видите, что Я "--- Бог. Внимай, душа моя, взывающему Господу, оставь прежний грех и убойся как праведного Судию и Бога.}

\pripevpomiluj

\minicolumns{\firstletter{К}ому уподобилася еси, многогрешная душе? Токмо первому Каину и Ламеху оному, каменовавшая тело злодействы и убившая ум безсловесными стремленьми.}{\firstletter{К}ому уподобилась ты, многогрешная душа, как не первому Каину и тому Ламеху, жестоко окаменив тело злодеяниями и убив ум безрассудными стремлениями.
\myemph{\footnotesize \mbox{Быт. 4:1--26}}}

\pripevpomiluj

\minicolumns{\firstletter{В}ся прежде закона претекши, о душе, Сифу не уподобилася еси, ни Еноса подражала еси, ни Еноха преложением, ни Ноя, но явилася еси убога праведных жизни.}{\firstletter{И}мея в виду всех, живших до закона, о душа, не уподобилась ты Сифу, не подражала ни Еносу, ни Еноху через преселение духовное, ни Ною, но оказалась чуждой жизни праведников.
\myemph{\footnotesize \mbox{Быт. 5:1--32}}}

\pripevpomiluj

\minicolumns{\firstletter{Е}дина отверзла еси хляби гнева Бога твоего, душе моя, и потопила еси всю, якоже землю, плоть, и деяния, и житие, и пребыла еси вне спасительнаго ковчега.}{\firstletter{Т}ы одна, душа моя, открыла бездны гнева Бога своего и потопила, как землю, всю плоть, и дела, и жизнь, и осталась вне спасительного ковчега.
\myemph{\footnotesize \mbox{Быт. 7:1--24}}}

\pripevpomiluj

\minicolumns{\firstletter{М}ужа убих, глаголет, в язву мне и юношу в струп, Ламех, рыдая вопияше; ты же не трепещеши, о душе моя, окалявши плоть и ум осквернивши.}{\firstletter{М}ужа убил я, сказал Ламех, в язву себе, и юношу "--- в рану себе, взывал он, рыдая; ты же, душа моя, не трепещешь, осквернив тело и помрачив ум.
\myemph{\footnotesize \mbox{Быт. 4:23}}}

\pripevpomiluj

\minicolumns{\firstletter{О}, како поревновах Ламеху, первому убийце, душу яко мужа, ум яко юношу, яко брата же моего, тело убив, яко Каин убийца, любосластными стремленьми.}{\firstletter{О}, как уподобился я древнему убийце Ламеху, убив душу, как мужа, ум "--- как юношу, и подобно убийце Каину "--- тело мое, как брата, сластолюбивыми стремлениями.
\myemph{\footnotesize \mbox{Быт. 4:8}}}

\pripevpomiluj

\minicolumns{\firstletter{С}толп умудрила еси создати, о душе, и утверждение водрузити твоими похотьми, аще не бы Зиждитель удержал советы твоя и низвергл на землю ухищрения твоя.}{\firstletter{Т}ы умудрилась, душа, устроить столп и воздвигнуть твердыню своими вожделениями, но Творец обуздал замыслы твои и поверг на землю твои построения.
\myemph{\footnotesize \mbox{Быт. 11:3--4}}}

\pripevpomiluj

\minicolumns{\firstletter{У}язвихся, уранихся, се стрелы вражия уязвившыя мою душу и тело, се струпи, гноения, омрачения вопиют, раны самовольных моих страстей.}{\firstletter{И}зранен я, изъявлен; вот стрелы врага, пронзившие душу мою и тело; вот раны, язвы и струпы вопиют об ударах самопроизвольных моих страстей.}

\pripevpomiluj

\minicolumns{\firstletter{О}дожди Господь от Господа огнь иногда на беззаконие гневающее, сожег содомляны; ты же огнь вжегла еси геенский, в немже имаши, о душе, сожещися.}{\firstletter{Г}осподь некогда пролил дождем огонь от Господа, попалив неистовое беззаконие содомлян; ты же, душа, разожгла огонь геенский, в котором должна будешь гореть.
\myemph{\footnotesize \mbox{Быт. 19:24}}}

\pripevpomiluj

\minicolumns{\firstletter{Р}азумейте и видите, яко Аз есмь Бог, испытаяй сердца и умучаяй мысли, обличаяй деяния, и попаляяй грехи, и судяй сиру, и смирену, и нищу.}{\firstletter{П}ознайте и увидьте, что Я "--- Бог, испытующий сердца и подвергающий наказанию мысли, обличающий деяния и огнем сжигающий грехи, и творящий праведный суд сироте, и уничиженному, и нищему.}

\pripevmskipc{\pripev{\firstletter{П}реподобная мати Марие, моли Бога о нас.}}

\minicolumns{\firstletter{П}ростерла еси руце твои к щедрому Богу, Марие, в бездне зол погружаемая; и якоже Петру человеколюбно руку Божественную простре, твое обращение всячески Иский.}{\firstletter{У}топая в бездне зла, ты простерла, Мария, руки свои к Милосердному Богу, и Он, всячески ища твоего обращения, человеколюбиво подал тебе, как Петру, Божественную руку.
\myemph{\footnotesize \mbox{Мф. 14:30--31}}}

\pripevmskipc{\pripev{\firstletter{П}реподобная мати Марие, моли Бога о нас.}}

\minicolumns{\firstletter{В}сем усердием и любовию притекла еси Христу, первый греха путь отвращши, и в пустынях непроходимых питающися, и Того чисте совершающи Божественныя заповеди.}{\firstletter{О}ставив прежний путь греха, ты с всем усердием и любовью прибегла ко Христу, живя в непроходимых пустынях и в чистоте исполняя Божественные Его заповеди.}

\pripevmskipc{\pripev{\firstletter{П}реподобне отче Андрее, моли Бога о нас.}}

\minicolumns{\firstletter{В}идим, видим человеколюбие, о душе, Бога и Владыки; сего ради прежде конца тому со слезами припадем вопиюще: Андрея молитвами, Спасе, помилуй нас.}{\firstletter{В}идим, видим, душа моя, человеколюбие Бога и Владыки; поэтому прежде кончины припадем к Нему со слезами, взывая: "По молитвам Андрея, Спаситель, помилуй нас".}

\slavac

\minicolumns{\firstletter{Б}езначальная, Несозданная Троице, Нераздельная Единице, кающася мя приими, согрешивша спаси, Твое есмь создание, не презри, но пощади и избави мя огненнаго осуждения.}{\firstletter{Б}езначальная Несозданная Троица, Нераздельная Единица, прими меня кающегося, спаси согрешившего, я "--- Твое создание, не презри, но пощади и избавь меня от осуждения в огонь.}

\inynec

\minicolumns{\firstletter{П}речистая Владычице, Богородительнице, Надеждо к Тебе притекающих и пристанище сущих в бури, Милостиваго и Создателя и Сына Твоего умилостиви и мне молитвами Твоими.}{\firstletter{П}речистая Владычица, Богородительница, Надежда прибегающих к Тебе и пристанище для застигнутых бурей, Твоими молитвами приклони на милость и ко мне Милостивого Творца и Сына Твоего.}

\mysubsubsection{Песнь 3}

\pripevc{\myemph{Ирм\'{о}с:}}

\minicolumns{\firstletter{Н}а недвижимом, Христе, камени заповедей Твоих утверди мое помышление.}{\firstletter{Н}а неподвижном камне заповедей Твоих, Христе, утверди мое помышление.}

\pripevpomiluj

\minicolumns{\firstletter{О}гнь от Господа иногда Господь одождив, землю содомскую прежде попали.}{\firstletter{П}ролив дождем огонь от Господа, Господь попалил некогда землю содомлян.
\myemph{\footnotesize \mbox{Быт. 19:24}}}

\pripevpomiluj

\minicolumns{\firstletter{Н}а горе спасайся душе, якоже Лот оный, и в Сигор угонзай.}{\firstletter{С}пасайся на горе, душа, как праведный Лот и спеши укрыться в Сигор.
\myemph{\footnotesize \mbox{Быт. 19:22--23}}}

\pripevpomiluj

\minicolumns{\firstletter{Б}егай запаления, о душе, бегай содомскаго горения, бегай тления Божественнаго пламене.}{\firstletter{Б}еги, душа, от пламени, беги от горящего Содома, беги от истребления Божественным огнем.}

\pripevpomiluj

\minicolumns{\firstletter{И}споведаюся Тебе, Спасе, согреших, согреших Ти, но ослаби, остави ми, яко благоутробен.}{\firstletter{И}споведуюсь Тебе, Спаситель; согрешил я, согрешил пред Тобою, но отпусти, прости меня, как Милосердный.}

\pripevpomiluj

\minicolumns{\firstletter{С}огреших Тебе един аз, согреших паче всех, Христе Спасе, да не презриши мене.}{\firstletter{С}огрешил я один пред Тобою, согрешил более всех, Христос Спаситель "--- не презирай меня.}

\pripevpomiluj

\minicolumns{\firstletter{Т}ы еси Пастырь Добрый, взыщи мене агнца, и заблуждшаго да не презриши мене.}{\firstletter{Т}ы "--- Пастырь Добрый, отыщи меня "--- агнца, и не презирай меня, заблудившегося.
\myemph{\footnotesize \mbox{Ин. 10:11--14}}}

\pripevpomiluj

\minicolumns{\firstletter{Т}ы еси сладкий Иисусе, Ты еси Создателю мой, в Тебе, Спасе, оправдаюся.}{\firstletter{Т}ы "--- вожделенный Иисус; Ты "--- Создатель мой, Спаситель, Тобою я оправдаюсь.}

\slavac

\minicolumns{\firstletter{О} Троице Единице Боже, спаси нас от прелести, и искушений, и обстояний.}{\firstletter{О}, Троица, Единица, Боже, спаси нас от обольщений, от искушений и опасностей.}

\inynec

\minicolumns{\firstletter{Р}адуйся, Богоприятная утробо, радуйся, престоле Господень, радуйся, Мати Жизни нашея.}{\firstletter{Р}адуйся, чрево, вместившее Бога; радуйся, Престол Господень; радуйся, Матерь Жизни нашей.}

\pripevmskipc{\myemph{Иный Ирм\'{о}с:}}

\minicolumns{\firstletter{У}тверди, Господи, на камени заповедей Твоих подвигшееся сердце мое, яко Един Свят еси и Господь.}{\firstletter{У}тверди, Господи, на камне Твоих заповедей поколебавшееся сердце мое, ибо Ты один свят и Господь.}

\pripevpomiluj

\minicolumns{\firstletter{И}сточник живота стяжах Тебе, смерти Низложителя, и вопию Ти от сердца моего прежде конца: согреших, очисти и спаси мя.}{\firstletter{И}сточник жизни нашел я в Тебе, Разрушитель смерти, и прежде кончины взываю к Тебе от сердца моего: согрешил я, умилостивись, спаси меня}

\pripevpomiluj

\minicolumns{\firstletter{П}ри Нои, Спасе, блудствовавшыя подражах, онех наследовах осуждение в потопе погружения.}{\firstletter{Я} подражал, Спаситель, развращенным современникам Ноя и наследовал осуждение их на потопление в потопе.\\
\myemph{\footnotesize \mbox{Быт. 6:1--17}}}

\pripevpomiluj

\minicolumns{\firstletter{С}огреших, Господи, согреших Тебе, очисти мя: несть бо иже кто согреши в человецех, егоже не превзыдох прегрешеньми.}{\firstletter{С}огрешил я, Господи, согрешил пред Тобою, смилуйся надо мною, ибо нет грешника между людьми, которого я не превзошел бы прегрешениями.}

\pripevpomiluj

\minicolumns{\firstletter{Х}ама онаго душе, отцеубийца подражавши, срама не покрыла еси искренняго, вспять зря возвратившися.}{\firstletter{П}одражая отцеубийце Хаму, ты, душа, не прикрыла срамоты ближнего с лицом, обращенным назад.
\myemph{\footnotesize \mbox{Быт. 9:22--23}}}

\pripevpomiluj

\minicolumns{\firstletter{Б}лагословения Симова не наследовала еси, душе окаянная, ни пространное одержание, якоже Иафеф, имела еси на земли оставления.}{\firstletter{С}имова благословения не наследовала ты, несчастная душа, и не получила, подобно Иафету, обширного владения на земле "--- отпущения грехов.
\myemph{\footnotesize \mbox{Быт. 9:26--27}}}

\pripevpomiluj

\minicolumns{\firstletter{О}т земли Харран изыди от греха, душе моя, гряди в землю, точащую присноживотное нетление, еже Авраам наследствова.}{\firstletter{У}дались, душа моя, от земли Харран "--- от греха; иди в землю, источающую вечно живое нетление, которую наследовал Авраам.
\myemph{\footnotesize \mbox{Быт. 12:1--7}}}

\pripevpomiluj

\minicolumns{\firstletter{А}враама слышала еси, душе моя, древле оставльша землю отечества и бывша пришельца, сего произволению подражай.}{\firstletter{Т}ы слышала, душа моя, как в древности Авраам оставил землю отеческую и сделался странником; подражай его решимости.
\myemph{\footnotesize \mbox{Быт. 12:1--7}}}

\pripevpomiluj

\minicolumns{\firstletter{У} дуба Мамврийскаго учредив патриарх Ангелы, наследствова по старости обетования ловитву.}{\firstletter{У}гостив Ангелов под дубом Маврийским, патриарх на старости получил, как добычу, обещанное.
\myemph{\footnotesize \mbox{Быт. 18:1--5}}}

\pripevpomiluj

\minicolumns{\firstletter{И}саака, окаянная душе моя, разумевши новую жертву, тайно всесожженную Господеви, подражай его произволению.}{\firstletter{З}ная, бедная душа моя, как Исаак принесен таинственно в новую жертву всесожжения Господу, подражай его решимости.
\myemph{\footnotesize \mbox{Быт. 22:2}}}

\pripevpomiluj

\minicolumns{\firstletter{И}смаила слышала еси, трезвися, душе моя, изгнана, яко рабынино отрождение, виждь, да не како подобно что постраждеши, ласкосердствующи.}{\firstletter{Т}ы слышала, душа моя, что Измаил был изгнан, как рожденный рабыней, бодрствуй, смотри, чтобы и тебе не потерпеть бы чего-либо подобного за сладострастие.
\myemph{\footnotesize \mbox{Быт. 21:10--11}}}

\pripevpomiluj

\minicolumns{\firstletter{А}гаре древле, душе, египтяныне уподобилася еси, поработившися произволением и рождши новаго Исмаила, презорство.}{\firstletter{Д}ревней Агари египтянке уподобилась ты, душа, порабощенная своим произволом и родив нового Измаила "--- дерзость.
\myemph{\footnotesize \mbox{Быт. 16:16}}}

\pripevpomiluj

\minicolumns{\firstletter{И}аковлю лествицу разумела еси, душе моя, являемую от земли к небесем: почто не имела еси восхода тверда, благочестия.}{\firstletter{Т}ы знаешь, душа моя, о лестнице с земли до небес, показанной Иакову; почему же ты не избрала безопасного восхода "--- благочестия?
\myemph{\footnotesize \mbox{Быт. 28:12}}}

\pripevpomiluj

\minicolumns{\firstletter{С}вященника Божия и царя уединена, Христово подобие в мире жития, в человецех подражай.}{\firstletter{П}одражай священнику Божию и царю одинокому Мелхиседеку, образу жизни Христа среди людей в мире.
\myemph{\footnotesize \mbox{Быт. 14:18}; \mbox{Евр. 7:1--3}}}

\pripevpomiluj

\minicolumns{\firstletter{Н}е буди столп сланый, душе, возвратившися вспять, образ да устрашит тя содомский, горе в Сигор спасайся.}{\firstletter{Н}е сделайся соляным столпом, душа, обратившись назад, да устрашит тебя пример содомлян; спасайся на гору в Сигор.
\myemph{\footnotesize \mbox{Быт. 19:19--23}; \mbox{Быт. 19:26}}}

\pripevpomiluj

\minicolumns{\firstletter{З}апаления, якоже Лот, бегай, душе моя, греха бегай Содомы и Гоморры, бегай пламене всякаго безсловеснаго желания.}{\firstletter{Б}еги, душа моя, от пламени греха; как Лот; беги от Содома и Гоморры; беги от огня всякого безрассудного пожелания.
\myemph{\footnotesize \mbox{Быт. 19:15--17}}}

\pripevpomiluj

\minicolumns{\firstletter{П}омилуй, Господи, помилуй мя, вопию Ти, егда приидеши со Ангелы Твоими воздати всем по достоянию деяний.}{\firstletter{П}омилуй, Господи, взываю к Тебе, помилуй меня, когда придешь с Ангелами Своими воздать всем по достоинству их дел.}

\pripevpomiluj

\minicolumns{\firstletter{М}оление, Владыко, Тебе поющих не отвержи, но ущедри, Человеколюбче, и подаждь верою просящим оставление.}{\firstletter{Н}е отвергни, Владыко, моления воспевающих Тебя, но умилосердись Человеколюбец, и просящим с верою даруй прощение.}

\pripevmskipc{\pripev{\firstletter{П}реподобная мати Марие, моли Бога о нас.}}

\minicolumns{\firstletter{С}одержимь есмь бурею и треволнением согрешений, но сама мя, мати, ныне спаси и к пристанищу Божественнаго покаяния возведи.}{\firstletter{О}кружен я, матерь, бурей и сильным волнением согрешений, но ты сама ныне спаси меня и приведи к пристанищу Божественного покаяния.}

\pripevmskipc{\pripev{\firstletter{П}реподобная мати Марие, моли Бога о нас.}}

\minicolumns{\firstletter{Р}абское моление и ныне, преподобная, принесши ко благоутробней молитвами твоими Богородице, отверзи ми Божественныя входы.}{\firstletter{У}сердное моление и ныне, преподобная, принеся к умилостивленной твоими молитвами Богородице, открой и для меня Божественные входы.}

\pripevmskipc{\pripev{\firstletter{П}реподобне отче Андрее, моли Бога о нас.}}

\minicolumns{\firstletter{Т}воими молитвами даруй и мне оставление долгов, о Андрее, Критский председателю, покаяния бо ты таинник преизрядный.}{\firstletter{Т}воими молитвами, О Андрей, глава (епископ) Крита, даруй и мне прощение долгов, ибо ты лучше других знаешь тайны покаяния.}

\slavac

\minicolumns{\firstletter{Т}роица Простая, Несозданная, Безначальное Естество, в Троице певаемая Ипостасей, спаси ны, верою покланяющыяся державе Твоей.}{\firstletter{Т}роица Несоставная, Несозданная, Существо Безначальная, в троичности Лиц воспеваемая, спаси нас, с верою поклоняющихся силе Твоей.}

\inynec

\minicolumns{\firstletter{О}т Отца безлетна Сына в лето, Богородительнице, неискусомужно родила еси, странное чудо, пребывши Дева доящи.}{\firstletter{Т}ы, Богородительница, не испытавши мужа, во времени родила Сына от Отца вне времени и "--- дивное чудо: питая молоком, пребыла Девою.}

\pripevmskipc{\myemph{Катавасия:}}

\minicolumns{\firstletter{У}тверди, Господи, на камени заповедей Твоих подвигшееся сердце мое, яко Един Свят еси и Господь.}{\firstletter{У}тверди, Господи, на камне Твоих заповедей поколебавшееся сердце мое, ибо Ты один свят и Господь.}

\pripevmskipc{\myemph{Седален, господина Иосифа, глас 8.}}

\pripevc{\myemph{Подобен:} Воскресл еси от гроба:}

\minicolumns{\firstletter{С}ветила богозрачная, Спасовы апостоли, просветите нас во тьме жития, яко да во дни ныне благообразно ходим, светом воздержания нощных страстей отбегающе, и светлыя страсти Христовы узрим, радующеся.}{
\myemph{\footnotesize \mbox{Рим. 12:13}}}

\slavac

\pripevmskipc{\myemph{Другий седален, глас 8.}}

\pripevc{\myemph{Подобен:} Повеленное тайно:}

\minicolumns{\firstletter{А}постольская двоенадесятице Богоизбранная, мольбу Христу ныне принеси, постное поприще всем прейти, совершающим во умилении молитвы, творящим усердно добродетели, яко да сице предварим видети Христа Бога славное Воскресение, славу и хвалу приносяще.}{}

\inynec

\minicolumns{\firstletter{Н}епостижимаго Бога, Сына и Слово, несказанно паче ума из Тебе рождшееся, моли, Богородице, со апостолы, мир вселенней чистый подати, и согрешений дати нам прежде конца прощение, и Царствия Небеснаго крайния ради благости сподобити рабы Твоя.}{}

\mysubsubsection{Песнь 4}

\mysubsubsection{Трипеснец, без поклонов, глас 8:}

\pripevc{\myemph{Ирм\'{о}с:}}

\minicolumns{\firstletter{У}слышах, Господи, смотрения Твоего таинство, разумех дела Твоя и прославих Твое Божество.}{}

\pripevmskipc{\pripev{\firstletter{С}вятии апостоли, молите Бога о нас.}}

\minicolumns{\firstletter{В}оздержанием поживше, просвещеннии Христовы апостоли, воздержания время нам ходатайствы Божественными утишают.}{}

\pripevmskipc{\pripev{\firstletter{С}вятии апостоли, молите Бога о нас.}}

\minicolumns{\firstletter{Д}военадесятострунный орган песнь воспе спасительную, учеников лик Божественный, лукавая возмущая гласования.}{}

\pripevmskipc{\pripev{\firstletter{С}вятии апостоли, молите Бога о нас.}}

\minicolumns{\firstletter{О}дождением духовным всю подсолнечную напоисте, сушу отгнавше многобожия, всеблаженнии.}{}

\pripevmskipc{\pripev{\firstletter{П}ресвятая Богородице, спаси нас.}}

\minicolumns{\firstletter{С}мирившася спаси мя, высокомудренно пожившаго, рождшая Вознесшаго смиренное естество, Дево Всечистая.}{}

\pripevmskipc{\myemph{Иный трипеснец. Ирмос, глас тойже:}}

\minicolumns{\firstletter{У}слышах, Господи, смотрения Твоего таинство, разумех дела Твоя и прославих Твое Божество.}{}

\pripevmskipc{\pripev{\firstletter{С}вятии апостоли, молите Бога о нас.}}

\minicolumns{\firstletter{А}постольское всечестное ликостояние, Зиждителя всех молящее, проси помиловати ны, восхваляющия тя.}{}

\pripevmskipc{\pripev{\firstletter{С}вятии апостоли, молите Бога о нас.}}

\minicolumns{\firstletter{Я}ко делателе суще, Христовы апостоли, во всем мире Божественным словом возделавшии, приносите плоды Ему всегда.}{}

\pripevmskipc{\pripev{\firstletter{С}вятии апостоли, молите Бога о нас.}}

\minicolumns{\firstletter{В}иноград бысте Христов воистинну возлюбленный, вино бо духовное источисте миру, апостоли.}{}

\pripevmskipc{\pripev{\firstletter{П}ресвятая Троице, Боже наш, слава Тебе.}}

\minicolumns{\firstletter{П}реначальная, Сообразная, Всесильнейшая Троице Святая, Отче, Слове и Душе Святый, Боже, Свете и Животе, сохрани стадо Твое.}{}

\pripevmskipc{\pripev{\firstletter{П}ресвятая Богородице, спаси нас.}}

\minicolumns{\firstletter{Р}адуйся, престоле огнезрачный, радуйся, светильниче свещеносный, радуйся, горо освящения, ковчеже Жизни, святых святая сене.}{}

\pripevmskipc{\myemph{Великаго канона Ирм\'{о}с:}}

\minicolumns{\firstletter{У}слыша пророк пришествие Твое, Господи, и убояся, яко хощеши от Девы родитися и человеком явитися, и глаголаше: услышах слух Твой и убояхся, слава силе Твоей, Господи.}{\firstletter{У}слышал пророк о пришествии Твоем, Господи, и устрашился, что Тебе угодно родиться от Девы и явиться людям, и сказал: услышал я весть о Тебе и устрашился; слава силе Твоей, Господи.
\myemph{\footnotesize \mbox{Авв. 3:1--3}}}

\pripevpomiluj

\minicolumns{\firstletter{Д}ел Твоих да не презриши, создания Твоего да не оставиши, Правосуде. Аще и един согреших яко человек, паче всякаго человека, Человеколюбче; но имаши, яко Господь всех, власть оставляти грехи.}{\firstletter{Н}е презри творений Твоих, не оставь создания Твоего, Праведный Судия, ибо хотя я, как человек, один согрешил более всякого человека, но Ты, Человеколюбец, как Господь всего мира, имеешь власть отпускать грехи.
\myemph{\footnotesize \mbox{Мф. 9:6}; \mbox{Мк. 2:10--11}}}

\pripevpomiluj

\minicolumns{\firstletter{П}риближается, душе, конец, приближается, и нерадиши, ни готовишися, время сокращается, востани, близ при дверех Судия есть. Яко соние, яко цвет, время жития течет: что всуе мятемся?}{\firstletter{К}онец приближается, душа, приближается, и ты не заботишься, не готовишься; время сокращается "--- восстань: Судия уже близко "--- при дверях; время жизни проходит, как сновиденье, как цвет. Для чего мы напрасно суетимся?
\myemph{\footnotesize \mbox{Мф. 24:33}; \mbox{Мк. 13:29}; \mbox{Лк. 21:31}}}

\pripevpomiluj

\minicolumns{\firstletter{В}оспряни, о душе моя, деяния твоя яже соделала еси помышляй, и сия пред лице твое принеси, и капли испусти слез твоих; рцы со дерзновением деяния и помышления Христу, и оправдайся.}{\firstletter{П}робудись, душа моя, размысли о делах своих, которые ты сделала, представь их пред своими очами, и пролей капли слез твоих, безбоязненно открой Христу дела и помышления твои и оправдайся.}

\pripevpomiluj

\minicolumns{\firstletter{Н}е бысть в житии греха, ни деяния, ни злобы, еяже аз, Спасе, не согреших, умом и словом, и произволением, и предложением, и мыслию, и деянием согрешив, яко ин никтоже когда.}{\firstletter{Н}ет в жизни ни греха, ни деяния, ни зла, в которых я не был бы виновен, Спаситель, умом, и словом, и произволением, согрешив и намерением, и мыслью, и делом так, как никто другой никогда.}

\pripevpomiluj

\minicolumns{\firstletter{О}тсюду и осужден бых, отсюду препрен бых аз окаянный от своея совести, еяже ничтоже в мире нужнейше; Судие, Избавителю мой и Ведче, пощади и избави, и спаси мя раба Твоего.}{\firstletter{П}отому и обвиняюсь, потому и осуждаюсь я, несчастный, своею совестью, строже которой нет ничего в мире; Судия, Искупитель мой и Испытатель, пощади, избавь и спаси меня, раба Твоего.}

\pripevpomiluj

\minicolumns{\firstletter{Л}ествица, юже виде древле великий в патриарсех, указание есть, душе моя, деятельнаго восхождения, разумнаго возшествия; аще хощеши убо деянием, и разумом, и зрением пожити, обновися.}{\firstletter{Л}естница, которую в древности видел великий из патриархов, служит указанием, душа моя, на восхождение делами, на возвышение разумом; поэтому, если хочешь жить в деятельности и в разумении и созерцании, то обновляйся.
\myemph{\footnotesize \mbox{Быт. 28:12}}}

\pripevpomiluj

\minicolumns{\firstletter{З}ной дневный претерпе лишения ради патриарх, и мраз нощный понесе, на всяк день снабдения творя, пасый, труждаяйся, работаяй, да две жене сочетает.}{\firstletter{П}атриарх по нужде терпел дневной зной и переносил ночной холод, ежедневно сокращая время, пася стада, трудясь и служа, чтобы получить себе две жены.
\myemph{\footnotesize Быт. 31; 7, 40; Быт. 29; 18--27}}

\pripevpomiluj

\minicolumns{\firstletter{Ж}ены ми две разумей, деяние же и разум в зрении, Лию убо деяние, яко многочадную, Рахиль же разум, яко многотрудную; ибо кроме трудов, ни деяние, ни зрение, душе, исправится.}{\firstletter{П}од двумя женами понимай деятельность и разумение в созерцании: под Лиею, как многочадною, "--- деятельность, а под Рахилью, как полученной через многие труды, "--- разумение, ибо без трудов, душа, ни деятельность, ни созерцание не усовершенствуются.}

\pripevpomiluj

\minicolumns{\firstletter{Б}ди, о душе моя, изрядствуй якоже древле великий в патриарсех, да стяжеши деяние с разумом, да будеши ум, зряй Бога, и достигнеши незаходящий мрак в видении, и будеши великий купец.}{\firstletter{Б}одрствуй, душа моя, будь мужественна, как великий из патриархов, чтобы приобрести себе дело по разуму, чтобы обогатиться умом, видящим Бога, и проникнуть в неприступный мрак в созерцании и получить великое сокровище.
\myemph{\footnotesize \mbox{Мф. 13:45--46}}}

\pripevpomiluj

\minicolumns{\firstletter{Д}ванадесять патриархов великий в патриарсех детотворив, тайно утверди тебе лествицу деятельнаго, душе моя, восхождения: дети, яко основания, степени яко восхождения, премудренно подложив.}{\firstletter{В}еликий из патриархов, родив двенадцать патриархов, таинственно представил тебе, душа моя, лестницу деятельного восхождения, премудро расположив детей как ступени, а свои шаги, как восхождения вверх.}

\pripevpomiluj

\minicolumns{\firstletter{И}сава возненавиденнаго подражала еси, душе, отдала еси прелестнику твоему первыя доброты первенство и отеческия молитвы отпала еси, и дважды поползнулася еси, окаянная, деянием и разумом: темже ныне покайся.}{\firstletter{П}одражая ненавиденному Исаву, душа, ты отдала соблазнителю своему первенство первоначальной красоты и лишилась отеческого благословения и, несчастная, пала дважды, деятельностью и разумением, поэтому ныне покайся.
\myemph{\footnotesize \mbox{Быт. 25:32}; \mbox{Быт. 27:37}; \mbox{Мал. 1:2--3}}}

\pripevpomiluj

\minicolumns{\firstletter{Е}дом Исав наречеся, крайняго ради женонеистовнаго смешения: невоздержанием бо присно разжигаемь и сластьми оскверняемь, Едом именовася, еже глаголется разжжение души любогреховныя.}{\firstletter{И}сав был назван Едомом за крайнее пристрастие к женолюбию; он непрестанно разжигаясь невоздержанием и оскверняясь любострастием, назван Едомом, что значит "--- «распаление души грехолюбивой».
\myemph{\footnotesize \mbox{Быт. 25:30}}}

\pripevpomiluj

\minicolumns{\firstletter{И}ова на гноищи слышавши, о душе моя, оправдавшагося, того мужеству не поревновала еси, твердаго не имела еси предложения во всех, яже веси, и имиже искусилася еси, но явилася еси нетерпелива.}{\firstletter{С}лышав об Иове, сидевшем на гноище, ты, душа моя, не подражала ему в мужестве, не имела твердой воли во всем, что узнала, что видела, что испытала, но оказалась нетерпеливою.
\myemph{\footnotesize \mbox{Иов. 1:1--22}}}

\pripevpomiluj

\minicolumns{\firstletter{И}же первее на престоле, наг ныне на гноище гноен, многий в чадех и славный, безчаден и бездомок напрасно: палату убо гноище и бисерие струпы вменяше.}{\firstletter{Б}ывший прежде на престоле, теперь "--- на гноище, обнаженный и изъязвленный; имевший многих детей и знаменитый, внезапно стал бездетным и бездомным; гноище считал он своим чертогом и язвы "--- драгоценными камнями.
\myemph{\footnotesize \mbox{Иов. 2:1--13}}}

\pripevpomiluj

\minicolumns{\firstletter{Ц}арским достоинством, венцем и багряницею одеян, многоименный человек и праведный, богатством кипя и стады, внезапу богаства, славы царства, обнищав, лишися.}{\firstletter{Ч}еловек, облеченный царским достоинством, венцом и багряницею, много имевший и праведный, изобиловавший богатством и стадами, внезапно обнищав, лишился богатства, славы и царства.
\myemph{\footnotesize \mbox{Иов. 1:1--22}}}

\pripevpomiluj

\minicolumns{\firstletter{А}ще праведен бяше он и непорочен паче всех, и не убеже ловления льстиваго и сети; ты же, грехолюбива сущи, окаянная душе, что сотвориши, аще чесому от недоведомых случится наити тебе?}{\firstletter{Е}сли он, будучи праведным и безукоризненным более всех, не избежал козней и сетей обольстителя диавола, то что сделаешь, ты, грехолюбивая несчастная душа, если что-нибудь неожиданное постигнет тебя?}

\pripevpomiluj

\minicolumns{\firstletter{Т}ело осквернися, дух окаляся, весь острупихся, но яко врач, Христе, обоя покаянием моим уврачуй, омый, очисти, покажи, Спасе мой, паче снега чистейша.}{\firstletter{Т}ело мое осквернено, дух грязен, весь я покрыт струпами, но Ты, Христе, как врач, уврачуй и то и другое моим покаянием, омой, очисти, яви меня чище снега, Спаситель мой.}

\pripevpomiluj

\minicolumns{\firstletter{Т}ело Твое и кровь Распинаемый о всех положил еси, Слове: тело убо, да мя обновиши, кровь, да омыеши мя. Дух же предал еси, да мя приведеши, Христе, Твоему Родителю.}{\firstletter{Т}вое тело и Кровь, Слово, Ты принес в жертву за всех при распятии; Тело "--- чтобы воссоздать меня, Кровь "--- чтобы омыть меня, и Дух Ты, Христе, предал, чтобы привести меня к Твоему Отцу.}

\pripevpomiluj

\minicolumns{\firstletter{С}оделал еси спасение посреде земли, Щедре, да спасемся. Волею на древе распялся еси, Едем затворенный отверзеся, горняя и дольняя тварь, языцы вси, спасени, покланяются Тебе.}{\firstletter{П}осреди земли Ты устроил спасение, Милосердный, чтобы мы спаслись; Ты добровольно распялся на древе; Едем затворенный открылся; Тебе поклоняются небесные и земные и все спасенные Тобою народы.
\myemph{\footnotesize \mbox{Пс. 73:12}}}

\pripevpomiluj

\minicolumns{\firstletter{Д}а будет ми купель кровь из ребр Твоих, вкупе и питие, источившее воду оставления, да обоюду очищаюся, помазуяся и пия, яко помазание и питие, Слове, животочная Твоя словеса.}{\firstletter{Д}а будет мне омовением Кровь из ребр Твоих и вместе питием, источившая оставление грехов, чтобы мне и тем и другим очищаться, Слове, помазуясь и напояясь животворными Твоими словами, как мазью и питием.
\myemph{\footnotesize \mbox{Ин. 19:33--34}}}

\pripevpomiluj

\minicolumns{\firstletter{Н}аг есмь чертога, наг есмь и брака, купно и вечери; светильник угасе, яко безъелейный, чертог заключися мне спящу, вечеря снедеся, аз же по руку и ногу связан, вон низвержен есмь.}{\firstletter{Я} лишен брачного чертога, лишен и брака, и вечери; светильник, как без елея, погас; чертог закрылся во время моего сна, вечеря окончена, а я, связанный по рукам и ногам, извержен вон.
\myemph{\footnotesize \mbox{Мф. 25:1--13}; \mbox{Мф. 22:11--13}; \mbox{Лк. 12:35--37}; \mbox{Лк. 13:24--27}; \mbox{Лк. 14:7--24}}}

\pripevpomiluj

\minicolumns{\firstletter{Ч}ашу Церковь стяжа, ребра Твоя живоносная, из нихже сугубыя нам источи токи оставления и разума во образ Древняго и Новаго, двоих вкупе Заветов, Спасе наш.}{\firstletter{Ц}ерковь приобрела себе Чашу в живоносном ребре Твоем, из которого проистек нам двойной поток оставления грехов и разумения, Спаситель наш, в образ обоих Заветов, Ветхого и Нового.}

\pripevpomiluj

\minicolumns{\firstletter{В}ремя живота моего мало и исполнено болезней и лукавства, но в покаянии мя приими и в разум призови, да не буду стяжание ни брашно чуждему, Спасе, Сам мя ущедри.}{\firstletter{В}ремя жизни моей кратко и исполнено огорчений и пороков, но прими меня в покаянии и призови к познанию истины, чтобы не сделаться мне добычею и пищею врага, Спаситель, умилосердись надо мною.
\myemph{\footnotesize \mbox{Быт. 47:9}}}

\pripevpomiluj

\minicolumns{\firstletter{В}ысокоглаголив ныне есмь, жесток же и сердцем, вотще и всуе, да не с фарисеем осудиши мя. Паче же мытарево смирение подаждь ми, Едине Щедре, Правосуде, и сему мя сочисли.}{\firstletter{В}ысокомерен я ныне на словах, дерзок и в сердце, напрасно и тщетно; не осуди меня с фарисеем, но даруй мне смирение мытаря и к нему причисли, Один Милосердный и Правосудный.}

\pripevpomiluj

\minicolumns{\firstletter{С}огреших, досадив сосуду плоти моея, вем, Щедре, но в покаянии мя приими и в разум призови, да не буду стяжание ни брашно чуждему, Спасе, Сам мя ущедри.}{\firstletter{З}наю, Милосердный, согрешил я, осквернив сосуд моей плоти, но прими меня в покаянии и призови к познанию истины, чтобы не сделаться мне добычею и пищею врага; Сам, Ты, Спаситель, умилосердись надо мною.}

\pripevpomiluj

\minicolumns{\firstletter{С}амоистукан бых страстьми, душу мою вредя, Щедре, но в покаянии мя приими и в разум призови, да не буду стяжание ни брашно чуждему, Спасе, Сам мя ущедри.}{\firstletter{И}стуканом я сделал сам себя, исказив душу свою страстями, Милосердный; но прими меня в покаянии и призови к познанию истины, чтобы не сделаться мне добычею и пищею врага; Сам, Ты, Спаситель, умилосердись надо мною.}

\pripevpomiluj

\minicolumns{\firstletter{Н}е послушах гласа Твоего, преслушах Писание Твое, Законоположника, но в покаянии мя приими и в разум призови, да не буду стяжание ни брашно чуждему, Спасе, Сам мя ущедри.}{\firstletter{Н}е послушал я голоса Твоего, нарушил Писание Твое, Законодатель; но прими меня в покаянии и призови к познанию истины, чтобы не сделаться мне добычею и пищею врага; Сам, Ты, Спаситель, умилосердись надо мною.}

\pripevmskipc{\pripev{\firstletter{П}реподобная мати Марие, моли Бога о нас.}}

\minicolumns{\firstletter{Б}езплотных жительство в плоти преходящи, благодать, преподобная, к Богу велию воистинну прияла еси, верно о чтущих тя предстательствуй. Темже молим тя, от всяких напастей и нас молитвами твоими избави.}{}

\pripevmskipc{\pripev{\firstletter{П}реподобная мати Марие, моли Бога о нас.}}

\minicolumns{\firstletter{В}еликих безместий во глубину низведшися, неодержима была еси; но востекла еси помыслом лучшим к крайней деяньми яве добродетели преславно, ангельское естество, Марие, удививши.}{\firstletter{У}влекшись в глубину великих пороков, ты, Мария, не погрязла в ней, но высшим помыслом через деятельность явно поднялась до совершенной добродетели, дивно изумив ангельскую природу.}

\pripevmskipc{\pripev{\firstletter{П}реподобне отче Андрее, моли Бога о нас.}}

\minicolumns{\firstletter{А}ндрее, отеческая похвало, молитвами твоими не престай, моляся, предстоя Троице Пребожественней, яко да избавимся мучения, любовию предстателя тя Божественнаго, всеблаженне, призывающии, Криту удобрение.}{}

\slavac

\minicolumns{\firstletter{Н}ераздельное существом, неслитное Лицы богословлю Тя, Троическое Едино Божество, яко Единоцарственное и Сопрестольное, вопию Ти песнь великую, в вышних трегубо песнословимую.}{\firstletter{Н}ераздельным по существу, неслиянным в Лицах богословски исповедую Тебя, Троичное Единое Божество, Соцарственное и Сопрестольное; возглашаю Тебе великую песнь, в небесных обителях троекратно воспеваемую.
\myemph{\footnotesize \mbox{Ис. 6:1--3}}}

\inynec

\minicolumns{\firstletter{И} раждаеши, и девствуеши, и пребываеши обоюду естеством Дева, Рождейся обновляет законы естества, утроба же раждает нераждающая. Бог идеже хощет, побеждается естества чин: творит бо елика хощет.}{\firstletter{И} рождаешь Ты, и остаешься Девою, в обоих случаях сохраняя по естеству девство. Рожденный Тобою обновляет законы природы, а девственное чрево рождает; когда хочет Бог, то нарушается порядок природы, ибо Он творит, что хочет.}

\mysubsubsection{Песнь 5}

\pripevc{\myemph{Ирм\'{о}с:}}

\minicolumns{\firstletter{О}т нощи утренююща, Человеколюбче, просвети, молюся, и настави и мене на повеления Твоя, и научи мя Спасе, творити волю Твою.}{\firstletter{О}т ночи бодрствующего, просвети меня, молю, Человеколюбец, путеводи меня в повелениях Твоих и научи меня, Спаситель, исполнять Твою волю.
\myemph{\footnotesize \mbox{Пс. 62:2}; \mbox{Пс. 118:35}}}

\pripevpomiluj

\minicolumns{\firstletter{В} нощи житие мое преидох присно, тьма бо бысть, и глубока мне мгла, нощь греха, но яко дне сына, Спасе, покажи мя.}{\firstletter{Ж}изнь свою я постоянно проводил в ночи, ибо мраком и глубокою мглою была для меня ночь греха; но покажи меня сыном дня, Спаситель.
\myemph{\footnotesize \mbox{Еф. 5:8}; \mbox{1 Фес. 5:5}}}

\pripevpomiluj

\minicolumns{\firstletter{Р}увима подражая окаянный аз, содеях беззаконный и законопреступный совет на Бога Вышняго, осквернив ложе мое, яко отчее он.}{\firstletter{П}одобно Рувиму я, несчастный, совершил преступное и беззаконное дело пред Всевышним Богом, осквернив ложе мое, как тот "--- отчее.
\myemph{\footnotesize \mbox{Быт. 35:22}; \mbox{Быт. 49:3--4}}}

\pripevpomiluj

\minicolumns{\firstletter{И}споведаюся Тебе Христе Царю: согреших, согреших, яко прежде Иосифа братия продавшии, чистоты плод и целомудрия.}{\firstletter{И}споведаюсь Тебе, Христос-Царь: согрешил я, согрешил, как некогда братья, продавшие Иосифа, "--- плод чистоты и целомудрия.
\myemph{\footnotesize \mbox{Быт. 37:28}}}

\pripevpomiluj

\minicolumns{\firstletter{О}т сродников праведная душа связася, продася в работу сладкий, во образ Господень; ты же вся, душе, продалася еси злыми твоими.}{\firstletter{С}родниками предана была душа праведная; возлюбленный продан в рабство, прообразуя Господа; ты же, душа, сама всю продала себя своим порокам.}

\pripevpomiluj

\minicolumns{\firstletter{И}осифа праведнаго и целомудреннаго ума подражай, окаянная и неискусная душе, и не оскверняйся безсловесными стремленьми, присно беззаконнующи.}{\firstletter{П}одражай праведному Иосифу и уму его целомудренному, несчастная и невоздержанная душа, не оскверняйся и не беззаконствуй всегда безрассудными стремлениями.}

\pripevpomiluj

\minicolumns{\firstletter{А}ще и в рове поживе иногда Иосиф, Владыко Господи, но во образ погребения и востания Твоего, аз же что Тебе когда сицевое принесу?}{\firstletter{В}ладыко Господи, Иосиф был некогда во рву, но в прообраз Твоего погребения и воскресения; принесу ли когда-либо что подобное Тебе я?}

\pripevpomiluj

\minicolumns{\firstletter{М}оисеов слышала еси ковчежец, душе, водами, волнами носимь речными, яко в чертозе древле бегающий дела, горькаго совета фараонитска.}{\firstletter{Т}ы слышала, душа, о корзинке с Моисеем, в древности носимом водами в волнах реки, как в чертоге, избегшем горестного последствия замысла фараонова.
\myemph{\footnotesize \mbox{Исх. 2:3}}}

\pripevpomiluj

\minicolumns{\firstletter{А}ще бабы слышала еси, убивающыя иногда безвозрастное мужеское, душе окаянная, целомудрия деяние, ныне, яко великий Моисей, сси премудрость.}{\firstletter{Е}сли ты слышала, несчастная душа, о повивальных бабках, некогда умерщвлявших новорожденных младенцев мужского пола, то теперь, подобно Моисею, млекопитайся мудростью.
\myemph{\footnotesize \mbox{Исх. 1:8--22}}}

\pripevpomiluj

\minicolumns{\firstletter{Я}ко Моисей великий египтянина, ума уязвивши окаянная, не убила еси, душе; и како вселишися, глаголи, в пустыню страстей покаянием?}{\firstletter{П}одобно великому Моисею, поразившему египтянина, ты, не умертвила, несчастная душа, гордого ума; как же, скажи, вселишься ты в пустыню от страстей через покаяние?
\myemph{\footnotesize \mbox{Исх. 2:11--12}}}

\pripevpomiluj

\minicolumns{\firstletter{В} пустыню вселися великий Моисей; гряди убо, подражай того житие, да и в купине Богоявления, душе, в видении будеши.}{\firstletter{В}еликий Моисей поселился в пустыне; иди и ты, душа, подражай его жизни, чтобы и тебе увидеть в терновом кусте явление Бога.
\myemph{\footnotesize \mbox{Исх. 3:1--3}}}

\pripevpomiluj

\minicolumns{\firstletter{М}оисеов жезл воображай, душе, ударяющий море и огустевающий глубину, во образ Креста Божественнаго: имже можеши и ты великая совершити.}{\firstletter{И}зобрази, душа, Моисеев жезл, поражающий море и огустевающий глубину, в знамение Божественного Креста, которым и ты можешь совершить великое.
\myemph{\footnotesize \mbox{Исх. 14:21--22}}}

\pripevpomiluj

\minicolumns{\firstletter{А}арон приношаше огнь Богу непорочный, нелестный; но Офни и Финеес, яко ты душе, приношаху чуждее Богу, оскверненное житие.}{\firstletter{А}арон приносил Богу огонь чистый, беспримесный, но Офни и Финеес принесли, как ты, душа, отчужденную от Бога нечистую жизнь.
\myemph{\footnotesize \mbox{1 Цар. 2:12--13}}}

\pripevpomiluj

\minicolumns{\firstletter{Я}ко тяжкий нравом, фараону горькому бых, Владыко, Ианни и Иамври, душею и телом, и погружен умом, но помози ми.}{\firstletter{П}о упорству я стал как жестокий нравом фараон, Владыко, по душе и телу я "--- Ианний и Иамврий, и по уму погрязший, но помоги мне.
\myemph{\footnotesize \mbox{Исх. 7:11}; \mbox{2 Тим. 3:8}}}

\pripevpomiluj

\minicolumns{\firstletter{К}алу примесихся, окаянный, умом, омый мя, Владыко, банею моих слез, молю Тя, плоти моея одежду убелив яко снег.}{\firstletter{З}агрязнил я, несчастный, свой ум, но омой меня, Владыко, в купели слез моих молю Тебя, и убели, как снег, одежду плоти моей.}

\pripevpomiluj

\minicolumns{\firstletter{А}ще испытаю моя дела, Спасе, всякаго человека превозшедша грехами себе зрю, яко разумом мудрствуяй согреших, не неведением.}{\firstletter{К}огда исследую свои дела, Спаситель, то вижу, что превзошел я грехами всех людей, ибо я грешил с разумным сознанием, а не по неведению.}

\pripevpomiluj

\minicolumns{\firstletter{П}ощади, пощади, Господи, создание Твое, согреших, ослаби ми, яко естеством чистый Сам сый Един, и ин разве Тебе никтоже есть кроме скверны.}{\firstletter{П}ощади, Господи, пощади, создание Твое: я согрешил, прости мне, ибо только Ты один чист по природе, и никто, кроме Тебя, не чужд нечистоты.}

\pripevpomiluj

\minicolumns{\firstletter{М}ене ради Бог сый, вообразился еси в мя, показал еси чудеса, исцелив прокаженныя и разслабленнаго стягнув, кровоточивыя ток уставил еси, Спасе, прикосновением риз.}{\firstletter{Р}ади меня, будучи Богом, Ты принял мой образ, Спаситель, и, совершая чудеса, исцелял прокаженных, укреплял расслабленных, остановил кровотечение у кровоточивой прикосновением одежды.
\myemph{\footnotesize \mbox{Мф. 9:20}; \mbox{Мк. 5:25--27}; \mbox{Лк. 8:43--44}}}

\pripevpomiluj

\minicolumns{\firstletter{К}ровоточивую исцели прикосновением края ризна Господь, прокаженная очисти, слепыя и хромыя просветив исправи, глухия же и немыя и ничащия низу исцели словом, да ты спасешися, окаянная душе.}{\firstletter{Г}осподь исцелил кровоточивую через прикосновение к одежде Его, очистил прокаженных, дал прозрение слепым, исправил хромых, глухих, немых и уврачевал словом скорченную, чтобы ты спаслась, несчастная душа.
\myemph{\footnotesize \mbox{Мф. 9:20--22}; \mbox{Мф. 11:4--5}; \mbox{Лк. 13:10--13}}}

\pripevpomiluj

\minicolumns{\firstletter{Н}изу сничащую подражай, о душе, прииди, припади к ногама Иисусовыма, да тя исправит, и да ходиши право стези Господни.}{\firstletter{П}одражай, душа, скорченной жене, приди, припади к ногам Иисуса, чтобы Он исправил тебя и ты могла ходить прямо по стезям Господним.
\myemph{\footnotesize \mbox{Лк. 13:11--13}}}

\pripevpomiluj

\minicolumns{\firstletter{А}ще и кладязь еси глубокий, Владыко, источи ми воду из пречистых Твоих жил, да, яко самаряныня, не ктому пияй, жажду жизни бо струи источаеши.}{\firstletter{Е}сли Ты "--- и глубокий колодец, Владыко, то источи мне струи из пречистых ребр Своих, чтобы я, как самарянка, испив, уже не жаждал, ибо Ты источаешь потоки жизни.
\myemph{\footnotesize \mbox{Ин. 4:11--15}}}

\pripevpomiluj

\minicolumns{\firstletter{С}илоам да будут ми слезы моя, Владыко Господи, да умыю и аз зеницы сердца, и вижду Тя, умна Света превечна.}{\firstletter{С}илоамом да будут мне слезы мои, Владыко Господи, чтобы и мне омыть очи сердца и умственно созерцать Тебя, Предвечный Свет.
\myemph{\footnotesize \mbox{Ин. 9:7}}}

\pripevmskipc{\pripev{\firstletter{П}реподобная мати Марие, моли Бога о нас.}}

\minicolumns{\firstletter{Н}есравненным желанием, всебогатая, древу возжелевши поклонитися животному, сподобилася еси желания, сподоби убо и мене улучити вышния славы.}{\firstletter{С} чистой любовию возжелав поклониться Древу Жизни, всеблаженная, ты удостоилась желаемого; удостой же и меня достигнуть высшей славы.}

\pripevmskipc{\pripev{\firstletter{П}реподобная мати Марие, моли Бога о нас.}}

\minicolumns{\firstletter{С}труи Иорданския прешедши, обрела еси покой безболезненный, плоти сласти избежавши, еяже и нас изми твоими молитвами, преподобная.}{\firstletter{Т}ы перешла поток Иорданский и приобрела покой безболезненный, оставив плотское удовольствие, от которого избавь и нас твоими молитвами, преподобная.}

\pripevmskipc{\pripev{\firstletter{П}реподобне отче Андрее, моли Бога о нас.}}

\minicolumns{\firstletter{Я}ко пастырей изряднейша, Андрее премудре, избранна суща тя, любовию велиею и страхом молю, твоими молитвами спасение улучити и жизнь вечную.}{\firstletter{К}ак превосходнейшего из пастырей, премудрый Андрей, как избранного молю тебя с великой любовью и благоговением, чтобы мне, по молитвам твоим, получить спасение и жизнь вечную.}

\slavac

\minicolumns{\firstletter{Т}я, Троице, славим Единаго Бога: Свят, Свят, Свят еси, Отче, Сыне и Душе, Простое Существо, Единице присно покланяемая.}{\firstletter{Т}ебя, Пресвятая Троица, прославляем за Единого Бога: Свят, Свят, Свят Отец, Сын и Дух, Простое Существо, Единица вечно поклоняемая.}

\inynec

\minicolumns{\firstletter{И}з Тебе облечеся в мое смешение, нетленная, безмужная Мати Дево, Бог, создавый веки, и соедини Себе человеческое естество.}{\firstletter{В} Тебе, Нетленная, не познавшая мужа Матерь-Дево, облекся в мой состав сотворивший мир Бог и соединил с Собою человеческую природу.}

\mysubsubsection{Песнь 6}

\pripevc{\myemph{Ирм\'{о}с:}}

\minicolumns{\firstletter{В}озопих всем сердцем моим к щедрому Богу, и услыша мя от ада преисподняго, и возведе от тли живот мой.}{\firstletter{О}т всего сердца моего я воззвал к милосердному Богу, и Он услышал меня из ада преисподнего и воззвал жизнь мою от погибели.
\myemph{\footnotesize \mbox{Иона 2:3}}}

\pripevpomiluj

\minicolumns{\firstletter{С}лезы, Спасе, очию моею и из глубины воздыхания чисте приношу, вопиющу сердцу: Боже, согреших Ти, очисти мя.}{\firstletter{И}скренно приношу Тебе, Спаситель, слезы очей моих и воздыхания из глубины сердца, взывающего: Боже, согрешил я пред Тобою, умилосердись надо мною.}

\pripevpomiluj

\minicolumns{\firstletter{У}клонилася еси, душе, от Господа твоего, якоже Дафан и Авирон, но пощади, воззови из ада преисподняго, да не пропасть земная тебе покрыет.}{\firstletter{У}клонилась ты, душа, от Господа своего, как Дафан и Авирон, но воззови из ада преисподнего: пощади!, чтобы пропасть земная не поглотила тебя
\myemph{\footnotesize \mbox{Чис. 16:1--3}; \mbox{Чис. 16:28--31}}}

\pripevpomiluj

\minicolumns{\firstletter{Я}ко юница, душе, разсвирепевшая, уподобилася еси Ефрему, яко серна от тенет сохрани житие, вперивши деянием ум и зрением.}{\firstletter{Р}ассвирепев, как телица, ты, душа, уподобилась Ефрему, но как серна спасай от тенет свою жизнь, окрылив ум деятельностью и созерцанием.
\myemph{\footnotesize \mbox{Иер. 31:18}; \mbox{Ос. 10:11}}}

\pripevpomiluj

\minicolumns{\firstletter{Р}ука нас Моисеова да уверит душе, како может Бог прокаженное житие убелити и очистити, и не отчайся сама себе, аще и прокаженна еси.}{\firstletter{М}оисеева рука да убедит нас, душа, как Бог может убелить и очистить прокаженную жизнь, и не отчаивайся сама за себя, хотя ты и поражена проказою.
\myemph{\footnotesize \mbox{Исх. 4:6--7}}}

\pripevpomiluj

\minicolumns{\firstletter{В}олны, Спасе, прегрешений моих, яко в мори Чермнем возвращающеся, покрыша мя внезапу, яко египтяны иногда и тристаты.}{\firstletter{В}олны грехов моих, Спаситель, обратившись, как в Чермном море, внезапно покрыли меня, как некогда египтян и их всадников.
\myemph{\footnotesize \mbox{Исх. 14:26--28}; \mbox{Исх. 15:4--5}}}

\pripevpomiluj

\minicolumns{\firstletter{Н}еразумное, душе, произволение имела еси, яко прежде Израиль: Божественныя бо манны предсудила еси безсловесно любосластное страстей объядение.}{\firstletter{Н}ерассудителен твой выбор, душа, как у древнего Израиля, ибо ты безрассудно предпочла Божественной манне сластолюбивое пресыщение страстями.
\myemph{\footnotesize \mbox{Чис. 21:5}}}

\pripevpomiluj

\minicolumns{\firstletter{К}ладенцы, душе, предпочла еси хананейских мыслей паче жилы камене, из негоже премудрости река, яко чаша проливает токи богословия.}{\firstletter{К}олодцы хананейских помыслов ты, душа, предпочла камню с источником, из которого река премудрости, как чаша, изливает струи богословия.
\myemph{\footnotesize \mbox{Быт. 21:25}; \mbox{Исх. 17:3}; \mbox{Исх. 17:6}}}

\pripevpomiluj

\minicolumns{\firstletter{С}виная мяса и котлы и египетскую пищу, паче Небесныя, предсудила еси, душе моя, якоже древле неразумнии людие в пустыни.}{\firstletter{С}виное мясо, котлы и египетскую пищу ты предпочла пище небесной, душа моя, как древний безрассудный народ в пустыне.
\myemph{\footnotesize \mbox{Исх. 16:3}}}

\pripevpomiluj

\minicolumns{\firstletter{Я}ко удари Моисей, раб Твой, жезлом камень, образно животворивая ребра Твоя прообразоваше, из нихже вси питие жизни, Спасе, почерпаем.}{\firstletter{К}ак Моисей, раб Твой, ударив жезлом о камень, таинственно предызобразил животворное ребро Твое, Спаситель, из которого все мы почерпаем питие жизни.}

\pripevpomiluj

\minicolumns{\firstletter{И}спытай, душе, и смотряй, якоже Иисус Навин, обетования землю, какова есть, и вселися в ню благозаконием.}{\firstletter{И}сследуй, душа, подобно Иисусу Навину, и обозри обещанную землю, какова она, и поселись в ней путем исполнения закона.}

\pripevpomiluj

\minicolumns{\firstletter{В}остани и побори, яко Иисус Амалика, плотския страсти, и гаваониты, лестныя помыслы, присно побеждающи.}{\firstletter{В}осстань и побеждай плотские страсти, как Иисус Амалика, всегда побеждая и гаваонитян "--- обольстительные помыслы.
\myemph{\footnotesize \mbox{Исх. 17:8--9}; \mbox{Исх. 17:13}; \mbox{Нав. 8:21}}}

\pripevpomiluj

\minicolumns{\firstletter{П}рейди, времене текущее естество, яко прежде ковчег, и земли оныя буди во одержании обетования, душе, Бог повелевает.}{\firstletter{Д}уша, Бог повелевает: перейди, как некогда ковчег Иордан, текущее по своему существу время и сделайся обладательницею обещанной земли.
\myemph{\footnotesize \mbox{Нав. 3:17}}}

\pripevpomiluj

\minicolumns{\firstletter{Я}ко спасл еси Петра, возопивша спаси, предварив мя, Спасе, от зверя избави, простер Твою руку, и возведи из глубины греховныя.}{\firstletter{П}одобно тому как Ты спас Петра, воззвавшего, поспеши, Спаситель, спасти и меня, избавь меня от чудовища, простерши Свою руку, и выведи из глубины греха.
\myemph{\footnotesize \mbox{Мф. 14:28--31}}}

\pripevpomiluj

\minicolumns{\firstletter{П}ристанище Тя вем утишное, Владыко, Владыко Христе, но от незаходимых глубин греха и отчаяния мя, предварив, избави.}{\firstletter{Т}ихое пристанище вижу в Тебе, Владыка, Владыка Христе, поспеши же избавить меня от непроходимых глубин греха и отчаяния.}

\pripevpomiluj

\minicolumns{\firstletter{А}з есмь, Спасе, юже погубил еси древле царскую драхму; но вжег светильник, Предтечу Твоего, Слове, взыщи и обрящи Твой образ.}{\firstletter{Я} "--- та драхма с царским изображением, которая с древности потеряна у Тебя, Спаситель, но, засветив светильник "--- Предтечу Своего, Слове, поищи и найди Свой образ.
\myemph{\footnotesize \mbox{Лк. 15:8--9}}}

\pripevmskipc{\pripev{\firstletter{П}реподобная мати Марие, моли Бога о нас.}}

\minicolumns{\firstletter{Д}а страстей пламень угасиши, слез капли источила еси присно, Марие, душею распаляема, ихже благодать подаждь и мне, твоему рабу.}{\firstletter{Ч}тобы угасить пламень страстей, ты, Мария, пылая душой, непрестанно проливала потоки слез, преизобилие которых даруй и мне, рабу твоему.}

\pripevmskipc{\pripev{\firstletter{П}реподобная мати Марие, моли Бога о нас.}}

\minicolumns{\firstletter{Б}езстрастие небесное стяжала еси крайним на земли житием, Мати. Темже тебе поющым страстей избавитися молитвами твоими молися.}{\firstletter{В}озвышеннейшим образом жизни на земле, ты, матерь, приобрела небесное бесстрастие; поэтому ходатайствуй, чтобы воспевающие тебя избавились от страстей по твоим молитвам.}

\pripevmskipc{\pripev{\firstletter{П}реподобне отче Андрее, моли Бога о нас.}}

\minicolumns{\firstletter{К}ритскаго тя пастыря и председателя и вселенныя молитвенника ведый, притекаю, Андрее, и вопию ти: изми мя, отче, из глубины греха.}{\firstletter{Т}ебя, критского пастыря и главу, и молитвенника за всю вселенную зная, прибегаю к тебе, Андрей, и взываю: «Выведи меня отче, из глубины греха!»}

\slavac

\minicolumns{\firstletter{Т}роица есмь Проста, Нераздельна, раздельна Личне и Единица есмь естеством соединена, Отец глаголет, и Сын, и Божественный Дух.}{\firstletter{Я} "--- Троица Несоставная, Нераздельная, раздельная в Лицах, и Единица, соединенная по существу; свидетельствует Отец, Сын и Божественный Дух.}

\inynec

\minicolumns{\firstletter{У}троба Твоя Бога нам роди, воображенна по нам: Егоже, яко Создателя всех, моли, Богородице, да Твоими молитвами оправдимся.}{\firstletter{Ч}рево Твое родило нам Бога, принявшего наш образ; Его, как Создателя всего мира, моли, Богородица, чтобы по молитвам Твоим нам оправдаться.}

\pripevmskipc{\myemph{Катавасия:}}

\minicolumns{\firstletter{В}озопих всем сердцем моим к щедрому Богу, и услыша мя от ада преисподняго, и возведе от тли живот мой.}{\firstletter{О}т всего сердца моего я воззвал к милосердному Богу, и Он услышал меня из ада преисподнего и воззвал жизнь мою от погибели.
\myemph{\footnotesize \mbox{Иона 2:3}}}

\mysubsubsection{Кондак, глас 6:}

\minicolumns{\firstletter{Д}уше моя, душе моя, востани, что спиши? Конец приближается, и имаши смутитися; воспряни убо, да пощадит тя Христос Бог, везде сый и вся исполняяй.}{\firstletter{Д}уша моя, душа моя, восстань, что ты спишь? Конец приближается, и ты смутишься; пробудись же, чтобы пощадил тебя Христос Бог, Вездесущий и все наполняющий.}

\pripevmskipc{\myemph{Икос:}}

\minicolumns{\firstletter{Х}ристово врачевство видя отверсто и от сего Адаму истекающее здравие, пострада, уязвися диавол и, яко бедствуя, рыдаше и своим другом возопи: что сотворю Сыну Мариину, убивает мя Вифлеемлянин, Иже везде сый и вся исполняяй.}{}

\mysubsubsection{Блаженны, с поклонами, глас 6:}

\minicolumns{\firstletter{В}о Царствии Твоем помяни нас, Господи.}{}

\minicolumns{\firstletter{Р}азбойника, Христе, рая жителя сотворил еси, на кресте Тебе возопивша: помяни мя; того покаянию сподоби и мене, недостойнаго.}{}

\minicolumns{\firstletter{Б}лажени нищии духом, яко тех есть Царство Небесное.}{}

\minicolumns{\firstletter{М}аноя слышавши древле, душе моя, Бога в явлении бывша и из неплодове тогда приемша плод обетования, того благочестие подражай.}{
\myemph{\footnotesize \mbox{Суд. 13:2--24}}}

\minicolumns{\firstletter{Б}лажени плачущии, яко тии утешатся.}{}

\minicolumns{\firstletter{С}ампсоновой поревновавши лености, главу остригла еси, душе, дел твоих, предавши иноплеменником любосластием целомудренную жизнь и блаженную.}{
\myemph{\footnotesize \mbox{Суд. 16:4--21}}}

\minicolumns{\firstletter{Б}лажени кротции, яко тии наследят землю.}{}

\minicolumns{\firstletter{П}режде челюстию ослею победивый иноплеменники, ныне пленение ласкосердству страстному обретеся; но избегни, душе моя, подражания, деяния и слабости.}{
\myemph{\footnotesize \mbox{Суд. 15:15}}}

\minicolumns{\firstletter{Б}лажени алчущии и жаждущии правды, яко тии насытятся.}{}

\minicolumns{\firstletter{В}арак и Иеффай военачальницы, судии Израилевы предпочтени быша, с нимиже Деворра мужеумная; тех доблестьми, душе, вмужившися, укрепися.}{
\myemph{\footnotesize \mbox{Суд. 4:4--16}; \mbox{Суд. 11:9--10}; \mbox{Суд. 11:32--33}; \mbox{Суд. 12:4--7}}}

\minicolumns{\firstletter{Б}лажени милостивии, яко тии помиловани будут.}{}

\minicolumns{\firstletter{И}аилино храбрство познала еси, душе моя, Сисара древле прободшую и спасение соделавшую древом острым, слышиши, имже тебе крест образуется.}{
\myemph{\footnotesize \mbox{Суд. 4:17--22}}}

\minicolumns{\firstletter{Б}лажени чистии сердцем, яко тии Бога узрят.}{}

\minicolumns{\firstletter{П}ожри, душе, жертву похвальную, деяние, яко дщерь, принеси от Иеффаевы чистейшую и заколи, яко жертву, страсти плотския Господеви твоему.}{
\myemph{\footnotesize \mbox{Суд. 11:30--40}}}

\minicolumns{\firstletter{Б}лажени миротворцы, яко тии сынове Божии нарекутся.}{}

\minicolumns{\firstletter{Г}едеоново руно помышляй, душе моя, с небесе росу подыми и приникни, якоже пес, и пий воду, от закона текущую, изгнетением письменным.}{
\myemph{\footnotesize \mbox{Суд. 6:37--38}; \mbox{Суд. 7:4--7}}}

\minicolumns{\firstletter{Б}лажени изгнани правды ради, яко тех есть Царство Небесное.}{}

\minicolumns{\firstletter{И}лии священника осуждение, душе моя, восприяла еси, лишением ума приобретши страсти себе, якоже он чада, делати беззаконная.}{
\myemph{\footnotesize \mbox{1 Цар. 2:22}; \mbox{1 Цар. 2:31--34}}}

\minicolumns{\firstletter{Б}лажени есте, егда поносят вам и изженут и рекут всяк зол глагол на вы лжуще, Мене ради.}{}

\minicolumns{\firstletter{В} судиях левит небрежением свою жену дванадесятим коленом раздели, душе моя, да скверну обличит от Вениамина беззаконную.}{
\myemph{\footnotesize \mbox{Суд. 19:27--29}}}

\minicolumns{\firstletter{Р}адуйтеся и веселитеся, яко мзда ваша многа на Небесех.}{}

\minicolumns{\firstletter{Л}юбомудренная Анна молящися, устне убо двизаше ко хвалению, глас же ея не слышашеся, но обаче неплодна сущи, сына молитвы раждает достойна.}{
\myemph{\footnotesize \mbox{1 Цар. 1:9--13}; \mbox{1 Цар. 1:19--20}}}

\minicolumns{\firstletter{П}омяни нас, Господи, егда приидеши во Царствии Твоем.}{}

\minicolumns{\firstletter{В} судиях спричтеся Аннино порождение, великий Самуил, егоже воспитала Армафема в дому Господни; тому поревнуй, душе моя, и суди прежде инех дела твоя.}{
\myemph{\footnotesize \mbox{1 Цар. 1:25--28}}}

\minicolumns{\firstletter{П}омяни нас, Владыко, егда приидеши во Царствии Твоем.}{}

\minicolumns{\firstletter{Д}авид на царство избран, царски помазася рогом Божественного мира; ты убо, душе моя, аще хощеши вышняго Царствия, миром помажися слезами.}{
\myemph{\footnotesize \mbox{1 Цар. 16:1--13}}}

\minicolumns{\firstletter{П}омяни нас, Святый, егда приидеши во Царствии Твоем.}{}

\minicolumns{\firstletter{П}омилуй создание Твое, Милостиве, ущедри руку Твоею творение и пощади вся согрешившия, и мене паче всех, Твоих презревшаго повелений.}{}

\slavac

\minicolumns{\firstletter{Б}езначальну и рождению же и происхождению Отцу покланяюся рождшему, Сына славлю рожденнаго, пою сопросиявшаго Отцу же и Сыну Духа Святаго.}{}

\inynec

\minicolumns{\firstletter{П}реестественному Рождеству Твоему покланяемся, по естеству славы Младенца Твоего не разделяюще, Богородительнице: Иже бо Един Лицем, сугубыми исповедуется естествы.}{}

\mysubsubsection{Песнь 7}

\pripevc{\myemph{Ирм\'{о}с:}}

\minicolumns{\firstletter{С}огрешихом, беззаконновахом, неправдовахом пред Тобою, ниже соблюдохом, ниже сотворихом, якоже заповедал еси нам; но не предаждь нас до конца, отцев Боже.}{\firstletter{М}ы согрешили, жили беззаконно, неправо поступали пред Тобою, не сохранили, не исполнили, что Ты заповедал нам; но не оставь нас до конца, Боже отцов.
\myemph{\footnotesize \mbox{Дан. 9:5--6}}}

\pripevpomiluj

\minicolumns{\firstletter{С}огреших, беззаконновах и отвергох заповедь Твою, яко во гресех произведохся, и приложих язвам струпы себе; но Сам мя помилуй, яко благоутробен, отцев Боже.}{\firstletter{Я} согрешил, жил в беззакониях и нарушил заповедь Твою, ибо я рожден в грехах и к язвам своим приложил еще раны, но Сам Ты помилуй меня, как Милосердный Боже отцов.}

\pripevpomiluj

\minicolumns{\firstletter{Т}айная сердца моего исповедах Тебе, Судии моему, виждь мое смирение, виждь и скорбь мою, и вонми суду моему ныне, и Сам мя помилуй, яко благоутробен, отцев Боже.}{\firstletter{Т}айны сердца моего я открыл пред Тобою, Судьей моим; воззри на смирение мое, воззри и на скорбь мою, обрати внимание на мое ныне осуждение и Сам помилуй меня, как Милосердный, Боже отцов.
\myemph{\footnotesize \mbox{Пс. 37:19}; \mbox{Пс. 24:18}; \mbox{Пс. 34:23}}}

\pripevpomiluj

\minicolumns{\firstletter{С}аул иногда, яко погуби отца своего, душе, ослята, внезапу царство обрете к прослутию; но блюди, не забывай себе, скотския похоти твоя произволивши паче Царства Христова.}{\firstletter{С}аул, некогда потеряв ослиц своего отца, неожиданно с известием о них получил царство; душа, не забывайся, предпочитая свои скотские стремления Христову Царству.
\myemph{\footnotesize \mbox{1 Цар. 9:1--27}; \mbox{1 Цар. 10:1}}}

\pripevpomiluj

\minicolumns{\firstletter{Д}авид иногда Богоотец, аще и согреши сугубо, душе моя, стрелою убо устрелен быв прелюбодейства, копием же пленен быв убийства томлением; но ты сама тяжчайшими делы недугуеши, самохотными стремленьми.}{\firstletter{Е}сли богоотец Давид некогда и вдвойне согрешил, будучи уязвлен стрелою прелюбодеяния, сражен был копьем мщения за убийства; но ты, душа моя, сама страдаешь более тяжко, нежели этими делами, произвольными стремлениями.
\myemph{\footnotesize \mbox{2 Цар. 11:2--6}; \mbox{2 Цар. 11:14--15}}}

\pripevpomiluj

\minicolumns{\firstletter{С}овокупи убо Давид иногда беззаконию беззаконие, убийству же любодейство растворив, покаяние сугубое показа абие; но сама ты, лукавнейшая душе, соделала еси, не покаявшися Богу.}{\firstletter{Д}авид некогда присовокупил беззаконие к беззаконию, ибо с убийством соединил прелюбодеяние, но скоро принес и усиленное покаяние, а ты, коварнейшая душа, совершив бОльшие грехи, не раскаялась пред Богом.}

\pripevpomiluj

\minicolumns{\firstletter{Д}авид иногда вообрази, списав яко на иконе песнь, еюже деяние обличает, еже содея, зовый: помилуй мя, Тебе бо Единому согреших всех Богу, Сам очисти мя.}{\firstletter{Д}авид некогда, изображая как бы на картине, начертал песнь, которой обличает совершенный им проступок, взывая: помилуй мя, ибо согрешил я пред Тобою, Одним, Богом всех; Сам очисти меня.
\myemph{\footnotesize \mbox{Пс. 50:3--6}}}

\pripevpomiluj

\minicolumns{\firstletter{К}ивот яко ношашеся на колеснице, Зан оный, егда превращшуся тельцу, точию коснуся, Божиим искусися гневом; но того дерзновения убежавши, душе, почитай Божественная честне.}{\firstletter{К}огда ковчег везли на колеснице, то Оза, когда вол свернул в сторону, лишь только прикоснулся, испытал на себе гнев Божий, но, душа, избегая его дерзости, благоговейно почитай Божественное.
\myemph{\footnotesize \mbox{2 Цар. 6:6--7}}}

\pripevpomiluj

\minicolumns{\firstletter{С}лышала еси Авессалома, како на естество воста, познала еси того скверная деяния, имиже оскверни ложе Давида отца; но ты подражала еси того страстная и любосластная стремления.}{\firstletter{Т}ы слышала об Авессаломе, как он восстал на самую природу, знаешь гнусные его деяния, которыми он обесчестил ложе отца "--- Давида; но ты сама подражала его страстным и сластолюбивым порывам.
\myemph{\footnotesize \mbox{2 Цар. 15:1--37}; \mbox{2 Цар. 16:20--22}}}

\pripevpomiluj

\minicolumns{\firstletter{П}окорила еси неработное твое достоинство телу твоему, иного бо Ахитофела обретше врага, душе, снизшла еси сего советом; но сия разсыпа Сам Христос, да ты всяко спасешися.}{\firstletter{С}вободное свое достоинство ты, душа, подчинила своему телу, ибо, нашедши другого Ахитофела-врага, ты склонилась на его советы, но их рассеял Сам Христос, чтобы ты спасена была.
\myemph{\footnotesize \mbox{2 Цар. 16:20--21}}}

\pripevpomiluj

\minicolumns{\firstletter{С}оломон чудный и благодати премудрости исполненный, сей лукавое иногда пред Богом сотворив, отступи от Него; емуже ты проклятым твоим житием, душе, уподобилася еси.}{\firstletter{Ч}удный Соломон, будучи преисполнен дара премудрости, некогда, сотворив злое пред Богом, отступил от Него; ему ты уподобилась, душа, своей жизнью, достойной проклятия.
\myemph{\footnotesize \mbox{3 Цар. 3:12}; \mbox{3 Цар. 11:4--6}}}

\pripevpomiluj

\minicolumns{\firstletter{С}ластьми влеком страстей своих, оскверняшеся, увы мне, рачитель премудрости, рачитель блудных жен, и странен от Бога; егоже ты подражала еси умом, о душе, сладострастьми скверными.}{\firstletter{У}влекшись сластолюбивыми страстями, осквернился, увы, ревнитель премудрости, возлюбив нечестивых женщин и отчуждившись от Бога; ему, душа, ты сама подражала в уме постыдным сладострастием.
\myemph{\footnotesize \mbox{3 Цар. 11:6--8}}}

\pripevpomiluj

\minicolumns{\firstletter{Р}овоаму поревновала еси, не послушавшему совета отча, купно же и злейшему рабу Иеровоаму, прежнему отступнику, душе, но бегай подражания и зови Богу: согреших, ущедри мя.}{\firstletter{Т}ы поревновала, душа, Ровоаму, не послушавшему совета отеческого, и вместе злейшему рабу Иеровоаму, древнему мятежнику; избегай подражание им и взывай к Богу: согрешила я, умилосердись надо мною.
\myemph{\footnotesize \mbox{3 Цар. 12:13--14}; 3 Цар. 20}}

\pripevpomiluj

\minicolumns{\firstletter{А}хаавовым поревновала еси сквернам, душе моя, увы мне, была еси плотских скверн пребывалище и сосуд срамлен страстей, но из глубины твоея воздохни и глаголи Богу грехи твоя.}{\firstletter{Т}ы подражала Ахаву в мерзостях, душа моя; увы, ты сделалась жилищем плотских нечистот и постыдным сосудом страстей; но воздохни из глубины своей и поведай Богу грехи свои.
\myemph{\footnotesize \mbox{3 Цар. 16:29--31}}}

\pripevpomiluj

\minicolumns{\firstletter{П}опали Илиа иногда дващи пятьдесят Иезавелиных, егда студныя пророки погуби, во обличение Ахаавово, но бегай подражания двою, душе, и укрепляйся.}{\firstletter{И}лия попалил некогда дважды по пятьдесят служителей Иезавели, когда истреблял гнусных пророков ее в обличение Ахава; но ты, душа, избегай подражания обоим им и крепись в воздержании.
\myemph{\footnotesize \mbox{3 Цар. 18:40}; \mbox{4 Цар. 1:9--15}}}

\pripevpomiluj

\minicolumns{\firstletter{З}аключися тебе небо, душе, и глад Божий постиже тя, егда Илии Фесвитянина, якоже Ахаав, не покорися словесем иногда, но Сараффии уподобився, напитай пророчу душу.}{\firstletter{З}аключилось небо для тебя, душа, и голод от Бога послан на тебя, как некогда на Ахава за то, что он не послушал слов Илии Фесфитянина; но ты подражай вдове Сарептской, напитай душу пророка.
\myemph{\footnotesize \mbox{3 Цар. 17:1}; \mbox{3 Цар. 17:8--9}}}

\pripevpomiluj

\minicolumns{\firstletter{М}анассиева собрала еси согрешения изволением, поставльши яко мерзости страсти и умноживши, душе, негодования, но того покаянию ревнующи тепле, стяжи умиление.}{\firstletter{Т}ы, душа, добровольно вместила преступления Манассии, поставив вместо идолов страсти и умножив мерзости; но усердно подражай и его покаянию с чувством умиления
\myemph{\footnotesize \mbox{4 Цар. 21:1--2}}}

\pripevpomiluj

\minicolumns{\firstletter{П}рипадаю Ти и приношу Тебе, якоже слезы, глаголы моя: согреших, яко не согреши блудница, и беззаконновах, яко иный никтоже на земли. Но ущедри, Владыко, творение Твое и воззови мя.}{\firstletter{П}рипадаю к Тебе и приношу Тебе со слезами слова мои: согрешил я, как не согрешила блудница, и жил в беззакониях, как никто другой на земле; но умилосердись, Владыка, над созданием Своим и восстанови меня.}

\pripevpomiluj

\minicolumns{\firstletter{П}огребох образ Твой и растлих заповедь Твою, вся помрачися доброта, и страстьми угасися, Спасе, свеща. Но ущедрив, воздаждь ми, якоже поет Давид, радование.}{\firstletter{З}атмил я образ Твой и нарушил заповедь Твою; вся красота помрачилась во мне, и светильник погас от страстей; но умилосердись, Спаситель, и возврати мне, как поет Давид, веселие.
\myemph{\footnotesize \mbox{Пс. 50:14}}}

\pripevpomiluj

\minicolumns{\firstletter{О}братися, покайся, открый сокровенная, глаголи Богу, вся ведущему: Ты веси моя тайная, Едине Спасе. Но Сам мя помилуй, якоже поет Давид, по милости Твоей.}{\firstletter{О}братись, покайся, открой сокровенное, скажи Богу Всеведущему: Спаситель, Ты Один знаешь мои тайны, но Сам помилуй меня, как поет Давид, по Твоей милости.
\myemph{\footnotesize \mbox{Пс. 50:3}}}

\pripevpomiluj

\minicolumns{\firstletter{И}счезоша дние мои, яко соние востающаго; темже, яко Езекиа, слезю на ложи моем, приложитися мне летом живота. Но кий Исаия предстанет тебе, душе, аще не всех Бог?}{\firstletter{Д}ни мои прошли как сновидение пробуждающегося; поэтому, подобно Езекии, я плачу на ложе моем, чтобы продлились годы жизни моей; но какой Исаия посетит тебя, душа, если не Бог всех?
\myemph{\footnotesize \mbox{4 Цар. 20:1--6}; \mbox{Ис. 38:1--6}}}

\pripevmskipc{\pripev{\firstletter{П}реподобная мати Марие, моли Бога о нас.}}

\minicolumns{\firstletter{В}озопивши к Пречистей Богоматери, первее отринула еси неистовство страстей, нужно стужающих, и посрамила еси врага запеншаго. Но даждь ныне помощь от скорби и мне, рабу твоему.}{\firstletter{В}оззвавши к Пречистой Богоматери, ты обуздала неистовство страстей, прежде жестоко свирепствовавших, и посрамила врага-обольстителя; даруй же ныне помощь в скорби и мне, рабу твоему.
\myemph{\footnotesize \mbox{Пс. 59:13}}}

\pripevmskipc{\pripev{\firstletter{П}реподобная мати Марие, моли Бога о нас.}}

\minicolumns{\firstletter{Е}гоже возлюбила еси, Егоже возжелела еси, Егоже ради плоть изнурила еси, преподобная, моли ныне Христа о рабех: яко да милостив быв всем нам, мирное состояние дарует почитающим Его.}{\firstletter{К}ого ты возлюбила, Кого избрала, для Кого изнуряла плоть, Преподобная, моли ныне Христа о рабах твоих, чтобы Он по Своей милости ко всем даровал мирное состояние почитающим Его.}

\pripevmskipc{\pripev{\firstletter{П}реподобне отче Андрее, моли Бога о нас.}}

\minicolumns{\firstletter{Н}а камени мя веры молитвами твоими утверди, отче, страхом мя Божественным ограждая, и покаяние, Андрее, подаждь ми, молюся ти, и избави мя от сети врагов, ищущих мя.}{}

\slavac

\minicolumns{\firstletter{Т}роице Простая, Нераздельная, Единосущная и Естество Едино, Светове и Свет, и Свята Три, и Едино Свято поется Бог Троица; но воспой, прослави Живот и Животы, душе, всех Бога.}{\firstletter{Т}роица Простая, Нераздельная, Единосущная, и Одно Божество, Светы и Свет, Три Святы и Одно Лицо Свято, Бог-Троица, воспеваемая в песнопениях; воспой же и ты, душа, прославь Жизнь и Жизни "--- Бога всех.}

\inynec

\minicolumns{\firstletter{П}оем Тя, благословим Тя, покланяемся Ти, Богородительнице, яко Неразлучныя Троицы породила еси Единаго Христа Бога и Сама отверзла еси нам, сущим на земли, Небесная.}{\firstletter{В}оспеваем Тебя, благословляем Тебя, поклоняемся Тебе, Богородительница, ибо Ты родила Одного из Нераздельной Троицы, Христа Бога, и Сама открыла для нас, живущих на земле, небесные обители.}

\newpage\mysubsubsection{Песнь 8}

\mysubsubsection{Трипеснец, глас 8:}

\pripevc{\myemph{Ирм\'{о}с:}}

\minicolumns{\firstletter{Б}езначальнаго Царя славы, Егоже трепещут Небесныя силы, пойте, священницы, людие, превозносите во вся веки.}{}

\pripevmskipc{\pripev{\firstletter{С}вятии апостоли, молите Бога о нас.}}

\minicolumns{\firstletter{Я}ко углие невещественнаго огня, попалите вещественныя страсти моя, возжизающе ныне во мне желание Божественныя любве, апостоли.}{}

\pripevmskipc{\pripev{\firstletter{С}вятии апостоли, молите Бога о нас.}}

\minicolumns{\firstletter{Т}рубы благогласныя Слова почтим, имиже падоша стены неутверждены вражия и богоразумия утвердишася забрала.}{}

\pripevmskipc{\pripev{\firstletter{С}вятии апостоли, молите Бога о нас.}}

\minicolumns{\firstletter{К}умиры страстныя души моея сокрушите, иже храмы и столпы сокрушисте врага, апостоли Господни, храмове освященнии.}{}

\pripevmskipc{\pripev{\firstletter{П}ресвятая Богородице, спаси нас.}}

\minicolumns{\firstletter{В}местила еси Невместимаго естеством, носила еси Носящаго вся, доила еси, Чистая, питающаго тварь Христа Жизнодавца.}{}

\pripevmskipc{\myemph{Иный трипеснец. Ирм\'{о}с:}}

\minicolumns{\firstletter{Б}езначальнаго Царя славы, Егоже трепещут Небесныя силы, пойте, священницы, людие, превозносите во вся веки.}{}

\pripevmskipc{\pripev{\firstletter{С}вятии апостоли, молите Бога о нас.}}

\minicolumns{\firstletter{Д}уха началохитростием создавше всю Церковь, апостоли Христовы, в ней благословите Христа во веки.}{}

\pripevmskipc{\pripev{\firstletter{С}вятии апостоли, молите Бога о нас.}}

\minicolumns{\firstletter{В}острубивше трубою учений, низвергоша апостоли всю лесть идольскую, Христа превозносяща во вся веки.}{}

\pripevmskipc{\pripev{\firstletter{С}вятии апостоли, молите Бога о нас.}}

\minicolumns{\firstletter{А}постоли, доброе преселение, назирателие мира и Небеснии жителие, вас присно восхваляющия избавите от бед.}{}

\pripevmskipc{\pripev{\firstletter{П}ресвятая Троице, Боже наш, слава Тебе.}}

\minicolumns{\firstletter{Т}рисолнечное Всесветлое Богоначалие, Единославное и Единопрестольное Естеству, Отче Вседетелю, Сыне и Божественный Душе, пою Тя во веки.}{}

\pripevmskipc{\pripev{\firstletter{П}ресвятая Богородице, спаси нас.}}

\minicolumns{\firstletter{Я}ко честный и превышший престол, воспоим Божию Матерь непрестанно, людие, Едину по рождестве Матерь и Деву.}{}

\pripevmskipc{\myemph{Великаго канона Ирм\'{о}с:}}

\minicolumns{\firstletter{Е}гоже воинства небесная славят, и трепещут Херувими и Серафими, всяко дыхание и тварь, пойте, благословите и превозносите во вся веки.}{\firstletter{К}ого прославляют воинства небесные и пред Кем трепещут Херувимы и Серафимы, Того, все существа и творения, воспевайте, благословляйте и превозносите во все века.}

\pripevpomiluj

\minicolumns{\firstletter{С}огрешивша, Спасе, помилуй, воздвигни мой ум ко обращению, приими мя кающагося, ущедри вопиюща: согреших Ти, спаси, беззаконновах, помилуй мя.}{\firstletter{П}омилуй меня, грешника, Спаситель, пробуди мой ум к обращению, приими кающегося, умилосердись над взывающим: я согрешил пред Тобою, спаси; я жил в беззакониях, помилуй меня.}

\pripevpomiluj

\minicolumns{\firstletter{К}олесничник Илиа колесницею добродетелей вшед, яко на небеса, ношашеся превыше иногда от земных; сего убо, душе моя, восход помышляй.}{\firstletter{В}езомый на колеснице Илия, взойдя на колесницу добродетелей, некогда вознесся как бы на небеса, превыше всего земного; помышляй, душа моя, об его восходе.
\myemph{\footnotesize \mbox{4 Цар. 2:11--13}}}

\pripevpomiluj

\minicolumns{\firstletter{И}орданова струя первее милотию Илииною Елиссеем ста сюду и сюду; ты же, о душе моя, сея не причастилася еси благодати за невоздержание.}{\firstletter{Е}лисея милотию Илии некогда разделил поток Иордана на ту и другую сторону; но ты, душа моя, не получила этой благодати за невоздержание.
\myemph{\footnotesize \mbox{4 Цар. 2:14}}}

\pripevpomiluj

\minicolumns{\firstletter{Е}лиссей иногда прием милоть Илиину, прият сугубую благодать от Бога; ты же, о душе моя, сея не причастилася еси благодати за невоздержание.}{\firstletter{Н}екогда Елисей, приняв милоть (плащ) Илии, получил сугубую благодать от Господа; но ты, душа моя, не получила этой благодати за невоздержание.
\myemph{\footnotesize \mbox{4 Цар. 2:9}; 4 Цар. 12--13}}

\pripevpomiluj

\minicolumns{\firstletter{С}оманитида иногда праведнаго учреди о душе, нравом благим; ты же не ввела еси в дом ни странна, ни путника. Темже чертога изринешися вон, рыдающи.}{\firstletter{С}оманитянка некогда угостила праведника с добрым усердием; а ты, душа, не приняла в свой дом ни странника, ни пришельца; за то будешь извержена вон из брачного чертога с рыданием.
\myemph{\footnotesize \mbox{4 Цар. 4:8}}}

\pripevpomiluj

\minicolumns{\firstletter{Г}иезиев подражала еси окаянная разум скверный всегда, душе, егоже сребролюбие отложи поне на старость; бегай геенскаго огня, отступивши злых твоих.}{\firstletter{Т}ы, несчастная душа, непрестанно подражала нечистому нраву Гиезия; хотя в старости отвергни его сребролюбие и, оставив свои злодеяния, избегни огня геенского.
\myemph{\footnotesize \mbox{4 Цар. 5:20--27}}}

\minicolumns{\firstletter{Т}ы Озии, душе, поревновавши, сего прокажение в себе стяжала еси сугубо: безместная бо мыслиши, беззаконная же дееши; остави, яже имаши, и притецы к покаянию.}{\firstletter{С}оревновав Озии, душа, ты получила себе вдвойне его проказу, ибо помышляешь недолжное и делаешь беззаконное; оставь, что у тебя есть и приступи к покаянию.
\myemph{\footnotesize \mbox{4 Цар. 15:5}; \mbox{2 Пар. 26:19}}}

\pripevpomiluj

\minicolumns{\firstletter{Н}иневитяны, душе, слышала еси кающыяся Богу, вреищем и пепелом, сих не подражала еси, но явилася еси злейшая всех, прежде закона и по законе прегрешивших.}{\firstletter{Т}ы слышала, душа, о ниневитянах, в рубище и пепле каявшихся Богу; им ты не подражала, но оказалась упорнейшею всех, согрешивших до закона и после закона.
\myemph{\footnotesize \mbox{Иона 3:5}; \mbox{Иона 3:8}}}

\pripevpomiluj

\minicolumns{\firstletter{В} рове блата слышала еси Иеремию, душе, града Сионя рыданьми вопиюща и слез ищуща: подражай сего плачевное житие и спасешися.}{\firstletter{Т}ы слышала, душа, как Иеремия, в нечистом рве с рыданиями взывал к городу Сиону и искал слез; подражай плачевной его жизни и спасешься.
\myemph{\footnotesize \mbox{Иер. 38:6}}}

\pripevpomiluj

\minicolumns{\firstletter{И}она в Фарсис побеже, проразумев обращение ниневитянов, разуме бо, яко пророк, Божие благоутробие: темже ревноваше пророчеству не солгатися.}{\firstletter{И}она побежал в Фарсис, предвидя обращение ниневитян, ибо он, как пророк, знал милосердие Божие и вместе ревновал, чтобы пророчество не оказалось ложным.
\myemph{\footnotesize \mbox{Иона 1:3}}}

\pripevpomiluj

\minicolumns{\firstletter{Д}аниила в рове слышала еси, како загради уста, о душе, зверей; уведела еси, како отроцы, иже о Азарии, погасиша верою пещи пламень горящий.}{\firstletter{Т}ы слышала,, душа, как Даниил во рве заградил уста зверей; ты узнала, как юноши, бывшие с Азариею, верою угасили разожженный пламень печи.
\myemph{\footnotesize \mbox{Дан. 14:31}; \mbox{Дан 3:24}}}

\pripevpomiluj

\minicolumns{\firstletter{В}етхаго Завета вся приведох ти, душе, к подобию; подражай праведных боголюбивая деяния, избегни же паки лукавых грехов.}{\firstletter{И}з Ветхого Завета всех я привел тебе в пример, душа; подражай богоугодным деяниям праведных, и избегай грехов людей порочных.}

\pripevpomiluj

\minicolumns{\firstletter{П}равосуде Спасе, помилуй и избави мя огня и прещения, еже имам на суде праведно претерпети; ослаби ми прежде конца добродетелию и покаянием.}{\firstletter{П}равосудный Спаситель, помилуй и избавь меня от огня и наказания, которому я должен справедливо подвергнуться на суде; прости меня прежде кончины, дав мне добродетель и покаяние.}

\pripevpomiluj

\minicolumns{\firstletter{Я}ко разбойник вопию Ти: помяни мя; яко Петр, плачу горце: ослаби ми, Спасе; зову, яко мытарь, слезю, яко блудница; приими мое рыдание, якоже иногда хананеино.}{\firstletter{К}ак разбойник взываю к Тебе: вспомни меня; как Петр, горько плачу, Спаситель; как мытарь, издаю вопль: будь милостив ко мне; проливаю слезы, как блудница; прими мое рыдание, как некогда от жены Хананейской.
\myemph{\footnotesize \mbox{Лк. 7:37--38}; \mbox{Лк. 18:13}; \mbox{Лк. 23:42}; \mbox{Лк. 22:61--62}; \mbox{Мф. 15:22}}}

\pripevpomiluj

\minicolumns{\firstletter{Г}ноение, Спасе, исцели смиренныя моея души, Едине Врачу, пластырь мне наложи, и елей, и вино, дела покаяния, умиление со слезами.}{\firstletter{О}дин Врач "--- Спаситель, исцели гниение моей смиренной души; приложи мне пластырь, елей и вино "--- дела покаяния, умиление со слезами.}

\pripevpomiluj

\minicolumns{\firstletter{Х}ананею и аз подражая, помилуй мя, вопию, Сыне Давидов; касаюся края ризы, яко кровоточивая, плачу, яко Марфа и Мария над Лазарем.}{\firstletter{П}одражая жене Хананейской, и я взываю к Сыну Давидову: помилуй меня; касаюсь одежды Его, как кровоточивая, плачу, как Марфа и Мария над Лазарем.
\myemph{\footnotesize \mbox{Мф. 9:20}; \mbox{Мф. 15:22}; \mbox{Ин. 11:33}}}

\pripevpomiluj

\minicolumns{\firstletter{С}лезную, Спасе, сткляницу яко миро истощавая на главу, зову Ти, якоже блудница, милости ищущая, мольбу приношу и оставление прошу прияти.}{\firstletter{И}зливая сосуд слез, как миро на голову, Спаситель, взываю к Тебе, как ищущая милости блудница, приношу моление и прошу о получении мне прощения.
\myemph{\footnotesize \mbox{Мф. 26:6--7}; \mbox{Мк. 14:3}; \mbox{Лк. 7:37--38}}}

\pripevpomiluj

\minicolumns{\firstletter{А}ще и никтоже, якоже аз, согреши Тебе, но обаче приими и мене, благоутробне Спасе, страхом кающася и любовию зовуща: согреших Тебе Единому, помилуй мя, Милостиве.}{\firstletter{Х}отя никто не согрешил пред Тобою, как я, но, Милосердный Спаситель, приими меня, кающегося со страхом и с любовию взывающего: я согрешил пред Тобою Одним, помилуй меня, Милосердный!}

\pripevpomiluj

\minicolumns{\firstletter{П}ощади, Спасе, Твое создание и взыщи, яко Пастырь, погибшее, предвари заблуждшаго, восхити от волка, сотвори мя овча на пастве Твоих овец.}{\firstletter{П}ощади, Спаситель, создание Свое и, как Пастырь, отыщи потерянного, возврати заблудшего, отними у волка и сделай меня агнцем на пастбище Твоих овец.
\myemph{\footnotesize \mbox{Пс. 118:176}}}

\pripevpomiluj

\minicolumns{\firstletter{Е}гда Судие сядеши, яко благоутробен, и покажеши страшную славу Твою, Спасе, о каковый страх тогда пещи горящей, всем боящимся нестерпимаго судища Твоего.}{\firstletter{К}огда Ты, Милосердный, воссядешь, как Судия и откроешь грозное величие Свое, Спаситель, о, какой ужас тогда: печь будет гореть, и все трепетать пред неумолимым судом Твоим.
\myemph{\footnotesize \mbox{Мф. 25:31}; \mbox{Мф. 41:47}}}

\pripevmskipc{\pripev{\firstletter{П}реподобная мати Марие, моли Бога о нас.}}

\minicolumns{\firstletter{С}вета незаходимаго Мати, тя просветивши, от омрачения страстей разреши. Темже вшедши в духовную благодать, просвети, Марие, тя верно восхваляющыя.}{\firstletter{М}атерь незаходимаго Света "--- Христа, просветив тебя, освободила от мрака страстей; поэтому, приняв благодать Духа, просвети, Мария, искренно прославляющих тебя.}

\pripevmskipc{\pripev{\firstletter{П}реподобная мати Марие, моли Бога о нас.}}

\minicolumns{\firstletter{Ч}удо ново видев, ужасашеся божественный в тебе воистинну, мати, Зосима: ангела бо зряше во плоти и ужасом весь исполняшеся, Христа поя во веки.}{\firstletter{У}видев в тебе, матерь, поистине новое чудо, святой Зосима удивился, ибо он увидел Ангела во плоти, и весь преисполнился изумлением, воспевая Христа вовеки .}

\pripevmskipc{\pripev{\firstletter{П}реподобне отче Андрее, моли Бога о нас.}}

\minicolumns{\firstletter{Я}ко дерзновение имый ко Господу, Андрее Критский, честная похвало, молю, молися разрешение от уз беззакония ныне обрести мне молитвами твоими, яко покаяния учитель и преподобных слава.}{}

\pripevmskipc{\firstletter{Б}лагословим Отца и Сына и Святаго Духа Господа.}

\minicolumns{\firstletter{Б}езначальне Отче, Сыне Собезначальне, Утешителю Благий, Душе Правый, Слова Божия Родителю, Отца Безначальна Слове, Душе Живый и Зиждяй, Троице Единице, помилуй мя.}{\firstletter{Б}езначальный Отче, Собезначальный Сын, Утешитель Благий, Дух Правый, Родитель Слова Божия, Безначальное Слово Отца, Дух, Животворящий и Созидающий, Троица Единая, помилуй меня.}

\inynec

\minicolumns{\firstletter{Я}ко от оброщения червленицы, Пречистая, умная багряница Еммануилева внутрь во чреве Твоем плоть исткася. Темже Богородицу воистинну Тя почитаем.}{\firstletter{М}ысленная порфира "--- плоть Еммануила соткалась внутри Твоего чрева. Пречистая, как бы из вещества пурпурного; потому мы почитаем Тебя, Истинную Богородицу.}

\pripevmskipc{\firstletter{Х}валим, благословим, покланяемся Господеви, поюще и превозносяще во вся веки.}

\pripevmskipc{\myemph{Катавасия:}}

\minicolumns{\firstletter{Е}гоже воинства небесная славят, и трепещут Херувими и Серафими, всяко дыхание и тварь, пойте, благословите и превозносите во вся веки.}{\firstletter{К}ого прославляют воинства небесные и пред Кем трепещут Херувимы и Серафимы, Того, все существа и творения, воспевайте, благословляйте и превозносите во все века.}

\mysubsubsection{Песнь 9}

\mysubsubsection{Трипеснец, глас 8:}

\pripevc{\myemph{Ирм\'{о}с:}}

\minicolumns{\firstletter{В}оистинну Богородицу Тя исповедуем, спасеннии Тобою, Дево чистая, с безплотными лики Тя величающе.}{}

\pripevmskipc{\pripev{\firstletter{С}вятии апостоли, молите Бога о нас.}}

\minicolumns{\firstletter{И}сточницы спасительныя воды явльшеся апостоли, истаявшую душу мою греховною жаждою оросите.}{}

\pripevmskipc{\pripev{\firstletter{С}вятии апостоли, молите Бога о нас.}}

\minicolumns{\firstletter{П}лавающаго в пучине погибели и в погружении уже бывша Твоею десницею, якоже Петра, Господи, спаси мя.}{}

\pripevmskipc{\pripev{\firstletter{С}вятии апостоли, молите Бога о нас.}}

\minicolumns{\firstletter{Я}ко соли, вкусных суще учений, гнильство ума моего изсушите и неведения тьму отжените.}{}

\pripevmskipc{\pripev{\firstletter{П}ресвятая Богородице, спаси нас.}}

\minicolumns{\firstletter{Р}адость яко родившая, плач мне подаждь, имже Божественное утешение, Владычице, в будущем дни обрести возмогу.}{}

\pripevmskipc{\myemph{Иный трипеснец. Ирм\'{о}с:}}

\minicolumns{\firstletter{Т}я, Небесе и земли Ходатаицу вси роди ублажаем: плотски бо вселися в Тя исполнение, Дево, Божества.}{}

\pripevmskipc{\pripev{\firstletter{С}вятии апостоли, молите Бога о нас.}}

\minicolumns{\firstletter{Т}я, благославное апостольское собрание, песньми величаем: вселенней бо светила светлая явистеся, прелесть отгоняще.}{}

\pripevmskipc{\pripev{\firstletter{С}вятии апостоли, молите Бога о нас.}}

\minicolumns{\firstletter{Б}лаговестною мрежею вашею словесныя рыбы уловивше, сия приносите всегда снедь Христу, апостоли блаженнии.}{}

\pripevmskipc{\pripev{\firstletter{С}вятии апостоли, молите Бога о нас.}}

\minicolumns{\firstletter{К} Богу вашим прошением помяните нас, апостоли, от всякаго избавитися искушения, молимся, любовию воспевающия вас.}{}

\pripevmskipc{\pripev{\firstletter{П}ресвятая Троице, Боже наш, слава Тебе.}}

\minicolumns{\firstletter{Т}я, Триипостасную Единицу, Отче, Сыне со Духом, Единаго Бога Единосущна пою, Троицу Единосильную Безначальную.}{}

\pripevmskipc{\pripev{\firstletter{П}ресвятая Богородице, спаси нас.}}

\minicolumns{\firstletter{Т}я, Детородительницу и Деву, вси роди ублажаем, яко Тобою избавльшеся от клятвы: радость бо нам родила еси, Господа.}{}

\pripevmskipc{\myemph{Великаго канона Ирм\'{о}с:}}

\minicolumns{\firstletter{Б}езсеменнаго зачатия рождество несказанное, Матере безмужныя нетленен Плод, Божие бо Рождение обновляет естества. Темже Тя вси роди, яко Богоневестную Матерь, православно величаем.}{\firstletter{Р}ождество от бессеменного зачатия неизъяснимо, безмужной Матери нетленен Плод, ибо рождение Бога обновляет природу. Поэтому Тебя, как Богоневесту-Матерь мы, все роды, православно величаем.}

\pripevpomiluj

\minicolumns{\firstletter{У}м острупися, тело оболезнися, недугует дух, слово изнеможе, житие умертвися, конец при дверех. Темже, моя окаянная душе, что сотвориши, егда приидет Судия испытати твоя?}{\firstletter{У}м изранился, тело расслабилось, дух болезнует, слово потеряло силу, жизнь замерла, конец при дверях. Что же сделаешь ты, несчастная душа, когда придет Судия исследовать дела твои?}

\pripevpomiluj

\minicolumns{\firstletter{М}оисеово приведох ти, душе, миробытие, и от того все Заветное Писание, поведающее тебе праведныя и неправедныя; от нихже вторыя, о душе, подражала еси, а не первыя, в Бога согрешивши.}{\firstletter{Я} воспроизвел пред тобою, душа, сказание Моисея о бытии мира и затем все Заветное Писание, повествующее о праведных и неправедных; из них ты, душа, подражала последним, а не первым, согрешая пред Богом.}

\pripevpomiluj

\minicolumns{\firstletter{З}акон изнеможе, празднует Евангелие, Писание же все в тебе небрежено бысть, пророцы изнемогоша и все праведное слово; струпи твои, о душе, умножишася, не сущу врачу, исцеляющему тя.}{\firstletter{О}слабел закон, не воздействует Евангелие, пренебрежено все Писание тобою, пророки и всякое слово праведника потеряли силу; язвы твои, душа, умножились, без Врача, исцеляющего тебя.}

\pripevpomiluj

\minicolumns{\firstletter{Н}оваго привожду ти Писания указания, вводящая тя, душе, ко умилению; праведным убо поревнуй, грешных же отвращайся и умилостиви Христа молитвами же и пощеньми, и чистотою, и говением.}{\firstletter{И}з Новозаветного Писания привожу тебе примеры, душа, возбуждающие в тебе умиление; так подражай праведным и отвращайся примера грешных и умилостивляй Христа молитвою, постом, чистотою и непорочностью.}

\pripevpomiluj

\minicolumns{\firstletter{Х}ристос вочеловечися, призвав к покаянию разбойники и блудницы; душе, покайся, дверь отверзеся Царствия уже, и предвосхищают е фарисее и мытари и прелюбодеи кающиися}{\firstletter{Х}ристос, сделавшись человеком, призвал к покаянию разбойников и блудниц; покайся, душа, дверь Царства уже открылась, и прежде тебя входят в нее кающиеся фарисеи, мытари и прелюбодеи.
\myemph{\footnotesize \mbox{Мф. 11:12}; \mbox{Мф. 21:31}; \mbox{Лк. 16:16}}}

\pripevpomiluj

\minicolumns{\firstletter{Х}ристос вочеловечися, плоти приобщився ми и вся елика суть естества хотением исполни, греха кроме, подобие тебе, о душе, и образ предпоказуя Своего снисхождения.}{\firstletter{Х}ристос, сделался человеком, приобщившись ко мне плотию, и добровольно испытал все, что свойственно природе, за исключением греха, показывая тебе, душа, пример и образец Своего снисхождения.}

\pripevpomiluj

\minicolumns{\firstletter{Х}ристос волхвы спасе, пастыри созва, младенец множества показа мученики, старцы прослави и старыя вдовицы, ихже не поревновала еси, душе, ни деянием, ни житию, но горе тебе внегда будеши судитися.}{\firstletter{Х}ристос спас волхвов, призвал к Себе пастухов, множество младенцев сделал мучениками, прославил старца и престарелую вдовицу; их деяниям и жизни ты не подражала, душа, но горе тебе, когда будешь судима!
\myemph{\footnotesize \mbox{Мф. 2:1--2}; \mbox{Мф. 2:16}; \mbox{Лк. 2:4--8}; \mbox{Лк. 2:25--26}; \mbox{Лк. 2:36--38}}}

\pripevpomiluj

\minicolumns{\firstletter{П}остився Господь дний четыредесять в пустыни, последи взалка, показуя человеческое; душе, да не разленишися, аще тебе приложится враг, молитвою же и постом от ног твоих да отразится.}{\firstletter{Г}осподь, постившись сорок дней в пустыне, наконец взалкал, обнаруживая в Себе человеческую природу. Не унывай, душа, если враг устремится на тебя, но да отразится он от ног твоих молитвами и постом.
\myemph{\footnotesize \mbox{Исх. 34:28}; \mbox{Мф. 4:2}; \mbox{Лк. 4:2}; \mbox{Мк. 1:13}}}

\pripevpomiluj

\minicolumns{\firstletter{Х}ристос искушашеся, диавол искушаше, показуя камение, да хлеби будут, на гору возведе видети вся царствия мира во мгновении; убойся, о душе, ловления, трезвися, молися на всякий час Богу.}{\firstletter{Х}ристос был искушаем; диавол искушал, показывая камни, чтобы они обратились в хлебы; возвел Его на гору, чтобы видеть все царства мира в одно мгновение; бойся, душа, этого обольщения, бодрствуй и ежечасно молись Богу.
\myemph{\footnotesize \mbox{Мф. 4:1--10}; Мк. 1; 12--13; \mbox{Лк. 4:1--12}}}

\pripevpomiluj

\minicolumns{\firstletter{Г}орлица пустыннолюбная, глас вопиющаго возгласи, Христов светильник, проповедуяй покаяние, Ирод беззаконнова со Иродиадою. Зри, душе моя, да не увязнеши в беззаконныя сети, но облобызай покаяние.}{\firstletter{П}устыннолюбивая горлица, голос вопиющего, Христов светильник взывал, проповедуя покаяние, а Ирод беззаконствовал с Иродиадою; смотри, душа моя, чтобы не впасть тебе в сети беззаконных, но возлюби покаяние.
\myemph{\footnotesize \mbox{Песн. 2:12}; \mbox{Ис. 40:3}; \mbox{Мф. 3:1--8}; \mbox{Мф. 14:3--4}; \mbox{Мк. 6:17}; \mbox{Лк. 3:19} -20}}

\pripevpomiluj

\minicolumns{\firstletter{В} пустыню вселися благодати Предтеча, и Иудея вся и Самария, слышавше, течаху и исповедаху грехи своя, крещающеся усердно: ихже ты не подражала еси, душе.}{\firstletter{П}редтеча благодати обитал в пустыне и все иудеи и самаряне стекались слушать его и исповедовали грехи свои, с усердием принимая крещение. Но ты, душа, не подражала им.
\myemph{\footnotesize \mbox{Мф. 3:1--6}; \mbox{Мк. 1:3--6}}}

\pripevpomiluj

\minicolumns{\firstletter{Б}рак убо честный и ложе нескверно, обоя бо Христос прежде благослови, плотию ядый, и в Кане же на браце воду в вино совершая, и показуя первое чудо, да ты изменишися, о душе.}{\firstletter{Б}рак честен и ложе непорочно, ибо Христос благословил их некогда, в Кане на браке вкушая пищу плотию и претворяя воду в вино, совершая первое чудо, чтобы ты, душа, изменилась.
\myemph{\footnotesize \mbox{Евр. 13:4}; \mbox{Ин. 2:1--11}}}

\pripevpomiluj

\minicolumns{\firstletter{Р}азслабленнаго стягну Христос, одр вземша, и юношу умерша воздвиже, вдовиче рождение, и сотнича отрока, и самаряныне явися, в дусе службу тебе, душе, предживописа.}{\firstletter{Х}ристос укрепил расслабленного, взявшего постель свою; воскресил умершего юного сына вдовы, исцелил слугу сотника и, открыв Себя самарянке, предначертал тебе, душа, служение Богу духом.
\myemph{\footnotesize \mbox{Мф. 9:6}; \mbox{Мф. 8:13}; \mbox{Лк. 7:12--15}; \mbox{Ин. 4:7--26}}}

\pripevpomiluj

\minicolumns{\firstletter{К}ровоточивую исцели прикосновением края ризна Господь, прокаженныя очисти, слепыя и хромыя просветив исправи, глухия же, и немыя, и ничащия низу исцели словом: да ты спасешися, окаянная душе.}{\firstletter{Г}осподь исцелил кровоточивую через прикосновение к одежде Его, очистил прокаженных, дал прозрение слепым, исправил хромых, глухих, немых и уврачевал словом скорченную, чтобы ты спаслась, несчастная душа.
\raggedright\myemph{\footnotesize \mbox{Мф. 9:20--22}; \mbox{Мф. 11:4--5}; \mbox{Лк. 13:10--13}}}

\pripevpomiluj

\minicolumns{\firstletter{Н}едуги исцеляя, нищим благовествоваше Христос Слово, вредныя уврачева, с мытари ядяше, со грешники беседоваше, Иаировы дщере душу предумершую возврати осязанием руки.}{\firstletter{В}рачуя болезни, Христос-Слово, благовествовал нищим, исцелял увечных, вкушал с мытарями, беседовал с грешниками и прикосновением руки возвратил вышедшую из тела душу Иаировой дочери.
\myemph{\footnotesize \mbox{Мф. 4:23}; \mbox{Мф. 9:10--11}; \mbox{Мк. 5:41--42}}}

\pripevpomiluj

\minicolumns{\firstletter{М}ытарь спасашеся, и блудница целомудрствоваше, и фарисей, хваляся, осуждашеся. Ов убо, очисти мя; ова же, помилуй мя; сей же величашеся вопия: Боже, благодарю Тя, и прочыя безумныя глаголы.}{\firstletter{М}ытарь спасся и блудница сделалась целомудренною, а гордый фарисей подвергся осуждению, ибо первый взывал: «Будь милостив ко мне»; другая: «Помилуй меня»; а последний тщеславно возглашал: «Боже, благодарю Тебя...» и прочие безумные речи.
\myemph{\footnotesize \mbox{Лк. 7:37--38}; \mbox{Лк. 7:46--47}; \mbox{Лк. 18:11--14}}}

\pripevpomiluj

\minicolumns{\firstletter{З}акхей мытарь бе, но обаче спасашеся, и фарисей Симон соблажняшеся, и блудница приимаше оставительная разрешения от Имущаго крепость оставляти грехи, юже, душе, потщися подражати.}{\firstletter{З}акхей был мытарь, однако спасся; Симон фарисей соблазнялся, а блудница получила прощение от Имеющего власть отпускать грехи; спеши, душа, и ты подражать ей.
\myemph{\footnotesize \mbox{Лк. 7:39}; \mbox{Лк. 19:9}; \mbox{Ин. 8:3--11}}}

\pripevpomiluj

\minicolumns{\firstletter{Б}луднице, о окаянная душе моя, не поревновала еси, яже приимши мира алавастр, со слезами мазаше нозе Спасове, отре же власы, древних согрешений рукописание Раздирающаго ея.}{\firstletter{Б}едная душа моя, ты не подражала блуднице, которая, взяв сосуд с миром, мазала со слезами и отирала волосами ноги Спасителя, разорвавшего запись прежних ее прегрешений.
\myemph{\footnotesize \mbox{Лк. 7:37--38}}}

\pripevpomiluj

\minicolumns{\firstletter{Г}рады, имже даде Христос благовестие, душе моя, уведала еси, како прокляти быша. Убойся указания, да не будеши якоже оны, ихже содомляном Владыка уподобив, даже до ада осуди.}{\firstletter{Т}ы знаешь, душа моя, как прокляты города, которым Христос благовестил Евангелие; страшись этого примера, чтобы и тебе не быть, как они, ибо Владыка, уподобив их содомлянам, присудил их к аду.
\myemph{\footnotesize \mbox{Лк. 10:12--15}}}

\pripevpomiluj

\minicolumns{\firstletter{Д}а не горшая, о душе моя, явишися отчаянием, хананеи веру слышавшая, еяже дщи словом Божиим исцелися; Сыне Давидов, спаси и мене, воззови из глубины сердца, якоже она Христу.}{\firstletter{Н}е окажись, душа моя, по отчаянию хуже хананеянки, слышавшей о вере, по которой Божиим словом исцелена дочь ее; взывай, как она, Христу из глубины сердца: «Сын Давидов, спаси и меня».
\myemph{\footnotesize \mbox{Мф. 15:22}}}

\pripevpomiluj

\minicolumns{\firstletter{У}милосердися, спаси мя, Сыне Давидов, помилуй, беснующыяся словом исцеливый, глас же благоутробный, яко разбойнику, мне рцы: аминь, глаголю тебе, со Мною будеши в раи, егда прииду во славе Моей.}{\firstletter{У}милосердись, спаси и помилуй меня, Сын Давидов, словом исцелявший беснующихся, и скажи, как разбойнику, милостивые слова: истинно говорю тебе, со Мною будешь в раю, когда приду Я в славе Моей.
\myemph{\footnotesize \mbox{Лк. 23:42--43}}}

\pripevpomiluj

\minicolumns{\firstletter{Р}азбойник оглаголоваше Тя, разбойник богословяше Тя, оба бо на кресте свисяста. Но, о Благоутробне, яко верному разбойнику Твоему, познавшему Тя Бога, и мне отверзи дверь славнаго Царствия Твоего.}{\firstletter{Р}азбойник поносил Тебя, разбойник же и Богом исповедал Тебя, вися оба на кресте; но, Милосердный, как уверовавшему разбойнику, познавшему в Тебе Бога, открой и мне, дверь славного Твоего Царства.}

\pripevpomiluj

\minicolumns{\firstletter{Т}варь содрогашеся, распинаема Тя видящи, горы и камения страхом распадахуся, и земля сотрясашеся, и ад обнажашеся, и соомрачашеся свет во дни, зря Тебе, Иисусе, пригвождена ко Кресту.}{\firstletter{Т}варь содрогалась, видя Тебя распинаемым, горы и камни от ужаса распадались и колебалась земля, преисподняя пустела, и свет среди дня помрачался, взирая на Тебя, Иисус, плотию ко кресту пригвожденного.
\myemph{\footnotesize \mbox{Мф. 27:51--52}; \mbox{Мк. 15:38}; \mbox{Лк. 23:45}}}

\pripevpomiluj

\minicolumns{\firstletter{Д}остойных покаяния плодов не истяжи от мене, ибо крепость моя во мне оскуде; сердце мне даруй присно сокрушенное, нищету же духовную: да сия Тебе принесу, яко приятную жертву, Едине Спасе.}{\firstletter{Д}остойных плодов покаяния не требуй от меня, Единый Спаситель, ибо сила моя истощилась во мне; даруй мне всегда сокрушенное сердце и духовную нищету, чтобы я принес их Тебе, как благоприятную жертву.}

\pripevpomiluj

\minicolumns{\firstletter{С}удие мой и Ведче мой, хотяй паки приити со ангелы, судити миру всему, милостивным Твоим оком тогда видев мя, пощади и ущедри мя, Иисусе, паче всякаго естества человеча согрешивша.}{\firstletter{С}удия мой, знающий меня, когда опять придешь Ты с Ангелами, чтобы судить весь мир, тогда, обратив на меня милостивый взор, пощади, Иисусе, и помилуй меня, согрешившего более всего человеческого рода.}

\pripevmskipc{\pripev{\firstletter{П}реподобная мати Марие, моли Бога о нас.}}

\minicolumns{\firstletter{У}дивила еси всех странным житием твоим, ангелов чины и человеков соборы, невещественно поживши и естество прешедши: имже, яко невещественныма ногама вшедши, Марие, Иордан прешла еси.}{\firstletter{Т}ы удивила необычайною своею жизнью всех, как чины ангельские, так и человеческие сонмы, духовно пожив и превзошедши природу; поэтому, Мария, ты, как бесплотная, шествуя стопами, перешла Иордан.}

\pripevmskipc{\pripev{\firstletter{П}реподобная мати Марие, моли Бога о нас.}}

\minicolumns{\firstletter{У}милостиви Создателя о хвалящих тя, преподобная мати, избавитися озлоблений и скорбей, окрест нападающих, да избавившеся от напастей, возвеличим непрестанно прославльшаго тя Господа.}{\firstletter{С}клони Творца на милость к восхваляющим тебя, преподобная матерь, чтобы нам избавиться от огорчений и скорбей, отовсюду нападающих на нас, чтобы, избавившись от искушений, мы непрестанно величали прославившего тебя Господа.}

\pripevmskipc{\pripev{\firstletter{П}реподобне отче Андрее, моли Бога о нас.}}

\minicolumns{\firstletter{А}ндрее честный и отче треблаженнейший, пастырю Критский, не престай моляся о воспевающих тя: да избавимся вси гнева и скорби, и тления, и прегрешений безмерных, чтущии твою память верно.}{\firstletter{А}ндрей досточтимый, отец преблаженный, пастырь Критский, не переставай молиться за воспевающих тебя, чтобы избавиться от гнева, скорби, погибели и бесчисленных прегрешений нам всем, искренно почитающим память твою.}

\slavac

\minicolumns{\firstletter{Т}роице Единосущная, Единице Триипостасная, Тя воспеваем, Отца славяще, Сына величающе и Духу покланяющеся, Единому Естеству воистинну Богу, Жизни же и живущему Царству безконечному.}{}

\inynec

\minicolumns{\firstletter{Г}рад Твой сохраняй, Богородительнице Пречистая, в Тебе бо сей верно царствуяй, в Тебе и утверждается, и Тобою побеждаяй, побеждает всякое искушение, и пленяет ратники, и проходит послушание.}{\firstletter{С}охраняй град Свой, Пречистая Богородительница. Под Твоею защитою он царствует с верою, и от Тебя получает крепость, и при Твоем содействии неотразимо побеждает всякое бедствие, берет в плен врагов и держит их в подчинении.}

\pripevmskipc{\myemph{Таже оба лика вкупе поют Ирм\'{о}с:}}

\minicolumns{\firstletter{Б}езсеменнаго зачатия рождество несказанное, Матере безмужныя нетленен Плод, Божие бо Рождение обновляет естества. Темже Тя вси роди, яко Богоневестную Матерь, православно величаем.}{\firstletter{Р}ождество от бессеменного зачатия неизъяснимо, безмужной Матери нетленен Плод, ибо рождение Бога обновляет природу. Поэтому Тебя, как Богоневесту-Матерь мы, все роды, православно величаем.}

\end{Parallel}

\mychapterending

\restoreparindent


