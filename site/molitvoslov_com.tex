\documentclass[11pt,oneside]{book}
\usepackage[utf8]{inputenc}
\usepackage[T2A]{fontenc}
\usepackage[russian]{babel}
\usepackage{graphicx}
\graphicspath{{target/img/},{target/img/tall/},{target/img/wide/}}

\usepackage{wrapfig}
\usepackage{indentfirst}
\usepackage{url}
\usepackage{ifpdf}

\renewcommand\sectionmark[1]{}
\newcommand{\partornament}{cross}
\newcommand{\ornament}{uzor_begin_10}
\newcommand{\ornamentending}{uzor_end_3}

% calculate header offset from text body
\setlength\headheight{14pt}
\newlength\headoff
\addtolength\headoff{\headheight}
\addtolength\headoff{\headsep}
\addtolength\headoff{14pt}

\title{Молитвослов на всякую потребу}
\author{www.molitvoslov.com}
\date{2011-10-13}
\setcounter{secnumdepth}{-3}
%\setcounter{tocdepth}{-1}

\newcommand{\mypart}[1]{
    \part{#1}
}

\newcommand{\mychapter}[1]{
    \renewcommand{\thechapter}{}
    \chapter{#1}
}

\newcommand{\mychapterz}[2]{
    \renewcommand{\thechapter}{}
    \chapter[#1]{#2}
}

\newcommand\mysubtitle[1]{
    \subsubsection{#1}
}

\newcommand\mysubsection[1]{
    \subsection{#1}
}

\newcommand\myparagraph[1]{
    \paragraph{#1}
}

\newenvironment{mymulticols}[1]{}{}

% the idea of identity environment is to provide a mechanism
% to assign some id to the element it surrounds. Its main use is
% with htlatex configs
\newcommand\myid{}
\newenvironment{identity}{}{}

\newenvironment{Parallel}{}{}

\newenvironment{centericon}{}{}

\newcommand{\myfig}[2][0.43]{}

\newcommand{\myfigh}[3][0.43]{}

\newcommand{\myfigr}[2][0.43]{}

\newcommand{\myfigrh}[3][0.43]{}

\newcommand{\myfigure}[2][0.43]{\myfig[#1]{#2}}

\newcommand{\mychapterending}[1][2.5]{}

\newcommand{\bukvaending}{\mychapterending}

\newcommand\longpage[1][1]{}
\newcommand\shortpage[1][1]{}

% save the value of current parindent to be used
% later in minipage environments
\newlength\myparindent
\setlength\myparindent{\parindent}
\newcommand\noparindent{\setlength\parindent{0pt}}
\newcommand\restoreparindent{\setlength\parindent{\myparindent}}

\long\def\symbolfootnote[#1]#2{\begingroup%
\def\thefootnote{\fnsymbol{footnote}}\footnote[#1]{#2}\endgroup}

\newcommand{\firstletter}[1]{#1}%{\myheadingcolor #1}}

\newcommand\pripev[2][Припев:]{{\small\myemph{#1} #2}}
\newcommand\irmos[1]{\pripev[Ирм\'{о}с:]{#1}}
\newcommand\Bogorodichen[1]{\myemph{Богородичен:} #1}
\newcommand\slavan{Слава Отцу и Сыну и Святому Духу.}
\newcommand\slava{{\small\slavan}}
\newcommand\inynen{И ныне и присно и во веки веков. Аминь.}
\newcommand\inyne{{\small\inynen}}
\newcommand\slavainynen{Слава Отцу и Сыну и Святому Духу. И ныне и присно и во веки веков. Аминь.}
\newcommand\slavainyne{{\small\slavainynen}}

\newcommand{\pripevc}[1]{{\small \centerline{#1} \nopagebreak}}
\newcommand{\pripevmskipc}[1]{\medskip\pripevc{#1}}
\newcommand{\pripevpomiluj}{\pripevmskipc{\pripev{\firstletter{П}омилуй мя, Боже, помилуй мя.}}}
\newcommand{\slavac}{\pripevmskipc{\slavan}}
\newcommand{\inynec}{\pripevmskipc{\inynen}}

\newcommand{\TsariuNebesnyj}{%
Царю Небесный, Утешителю, Душе истины, Иже везде сый и вся исполняяй, Сокровище благих и жизни Подателю, прииди и вселися в ны, и очисти ны от всякия скверны, и спаси, Блаже, души наша.}

\newcommand{\TrisviatoePoOtcheNash}{%
Святый Боже, Святый Крепкий, Святый Безсмертный, помилуй нас. \myemph{ (Tрижды)}

Слава Отцу и Сыну и Святому Духу, и ныне и присно и во веки веков. Аминь.

Пресвятая Троице, помилуй нас; Господи, очисти грехи наша; Владыко, прости беззакония наша; Святый, посети и исцели немощи наша, имене Твоего ради.

Господи, помилуй. \myemph{ (Трижды)}

Слава Отцу и Сыну и Святому Духу, и ныне и присно и во веки веков. Аминь.

Отче наш, Иже еси на небесех! Да святится имя Твое, да приидет Царствие Твое, да будет воля Твоя, яко на небеси и на земли. Хлеб наш насущный даждь нам днесь; и остави нам долги наша, якоже и мы оставляем должником нашим; и не введи нас во искушение, но избави нас от лукаваго.
}

\newcommand{\priiditepoklonimsia}{%
Приидите, поклонимся Цареви нашему Богу. \myemph{(Поклон)}

Приидите, поклонимся и припадем Христу, Цареви нашему Богу. \myemph{(Поклон)}

Приидите, поклонимся и припадем Самому Христу, Цареви и Богу нашему. \myemph{(Поклон)}}


\newcommand{\PsalmFifty}{%
Помилуй мя, Боже, по велицей милости Твоей, и по множеству щедрот Твоих очисти беззаконие мое. Наипаче омый мя от беззакония моего, и от греха моего очисти мя; яко беззаконие мое аз знаю, и грех мой предо мною есть выну. Тебе Единому согреших и лукавое пред Тобою сотворих, яко да оправдишися во словесех Твоих, и победиши внегда судити Ти. Се бо, в беззакониих зачат есмь, и во гресех роди мя мати моя. Се бо, истину возлюбил еси; безвестная и тайная премудрости Твоея явил ми еси. Окропиши мя иссопом, и очищуся; омыеши мя, и паче снега убелюся. Слуху моему даси радость и веселие; возрадуются кости смиренныя. Отврати лице Твое от грех моих и вся беззакония моя очисти. Сердце чисто созижди во мне, Боже, и дух прав обнови во утробе моей. Не отвержи мене от лица Твоего и Духа Твоего Святаго не отыми от мене. Воздаждь ми радость спасения Твоего и Духом владычним утверди мя. Научу беззаконыя путем Твоим, и нечестивии к Тебе обратятся. Избави мя от кровей, Боже, Боже спасения моего; возрадуется язык мой правде Твоей. Господи, устне мои отверзеши, и уста моя возвестят хвалу Твою. Яко аще бы восхотел еси жертвы, дал бых убо: всесожжения не благоволиши. Жертва Богу дух сокрушен; сердце сокрушенно и смиренно Бог не уничижит. Ублажи, Господи, благоволением Твоим Сиона, и да созиждутся стены Иерусалимския. Тогда благоволиши жертву правды, возношение и всесожегаемая; тогда возложат на oлтарь Твой тельцы.\par}

\newcommand{\Chestneyshuyu}{%
Достойно есть яко воистинну блажити Тя, Богородицу, Присноблаженную и Пренепорочную и Матерь Бога нашего. Честнейшую Херувим и славнейшую без сравнения Серафим, без истления Бога Слова рождшую, сущую Богородицу Тя величаем.}

\newcommand{\MolitvamiSviatyhOtecNashih}{%
Молитвами святых отец наших, Господи Иисусе Христе, Боже наш, помилуй нас. Аминь.}

\newcommand{\tolkopoblagosloveniyu}{%
{\centering\myemph{\normalfont Читаются только по благословению духовника}\par}}

\newcommand{\SymbolOfFaith}{%
  Верую во единаго Бога Отца, Вседержителя, Творца небу и земли, видимым же всем и невидимым.
  И во единаго Господа Иисуса Христа, Сына Божия, Единороднаго, Иже от Отца рожденнаго прежде всех век; Света от Света, Бога истинна от Бога истинна, рожденна, несотворенна, единосущна Отцу, Имже вся быша.
  Нас ради человек и нашего ради спасения сшедшаго с небес и воплотившагося от Духа Свята и Марии Девы и вочеловечшася.
  Распятаго же за ны при Понтийстем Пилате, и страдавша, и погребенна.
  И воскресшаго в третий день по Писанием.
  И возшедшаго на небеса, и седяща одесную Отца.
  И паки грядущаго со славою судити живым и мертвым, Егоже Царствию не будет конца.
  И в Духа Святаго, Господа, Животворящаго, Иже от Отца исходящего, Иже со Отцем и Сыном спокланяема и сславима, глаголавшаго пророки.
  Во едину Святую, Соборную и Апостольскую Церковь.
  Исповедую едино крещение во оставление грехов.
  Чаю воскресения мертвых, и жизни будущаго века. Аминь.}

\newcommand{\TroparPomilujNas}{Помилуй нас, Господи, помилуй нас; всякаго бо ответа недоумеюще, сию Ти молитву яко Владыце грешнии приносим: помилуй нас.

\slavan

 Господи, помилуй нас, на Тя бо уповахом; не прогневайся на ны зело, ниже помяни беззаконий наших, но призри и ныне яко благоутробен, и избави ны от враг наших; Ты бо еси Бог наш, и мы людие Твои, вси дела руку Твоею, и имя Твое призываем.

\inynen

 Милосердия двери отверзи нам, благословенная Богородице, надеющиися на Тя да не погибнем, но да избавимся Тобою от бед: Ты бо еси спасение рода христианскаго.}



\renewcommand{\pripevc}[1]{#1}

\begin{document}

%

\mypart{МОЛИТВЫ ЗА БОЛЯЩИХ}
%http://www.molitvoslov.com/content/obolyachih

 

\mychapter{Канон за болящего, глас 3-й}
%http://www.molitvoslov.com/text577.htm 
 


\bfseries Песнь 1\normalfont{}


\itshape Ирмос: \normalfont{}Пресекаемое море жезлом древле, Израиль пройде яко по пустыни, и крестообразно яве предуготовляет стези. Сего ради поим во хвалении чудному Богу нашему, яко прославися.

\itshape Припев: \normalfont{}\bfseries Милостиве Господи, услыши молитву раб Твоих, молящихся Тебе.

\normalfont{}В день печали, нашедшия на ны, к Тебе, Христе Спасе, припадающе, Твоея милости просим. Облегчи болезнь раба Твоего, изреки нам яко и сотнику: иди, се здрав есть отрок твой.

\itshape Припев: \normalfont{}\bfseries Милостиве Господи, услыши молитву раб Твоих, молящихся Тебе.

\normalfont{}Мольбы и моления, с воздыханием к Тебе вопием, Сыне Божий, помилуй нас. Воздвигни со одра лежащаго, яко разслабленнаго словом: возьми одр твой, глаголя, отпущаются ти греси твои.

\itshape Слава: \normalfont{}Твоего, Христе, образа подобию поклоняющеся, верою целуем, и болящему здравия просим, подражающе кровоточивей, яже подольца риз Твоих коснуся, и исцеление недуга прият.

\itshape И ныне: \normalfont{}Пречистая Госпоже Богородице, всем известная Помощница, не презри нас к Тебе припадающих, моли яко блага Твоего Сына и Бога нашего, дати здравие болящему, да Тя с нами прославляет.


\medskip


\bfseries Песнь 3\normalfont{}


\itshape Ирмос: \normalfont{}Иже от не сущих вся приведый, словом созидаемая, совершаемая духом, Вседержителю Вышний, в любви Твоей утверди мене.

\itshape Припев: \normalfont{}\bfseries Милостиве Господи, услыши молитву раб Твоих, молящихся Тебе.\normalfont{}

Иже от тяжких болезней на земли повержен, к Тебе, Христе, с нами вопиет, подаждь здравие телеси его, яко же Иезекии плакавшуся к Тебе.

\itshape Припев: \normalfont{}\bfseries Милостиве Господи, услыши молитву раб Твоих, молящихся Тебе.\normalfont{}

Призри, Господи, на наше смирение, и не помяни беззаконий наших, но веры ради болящаго, яко прокаженнаго словом исцели ему болезнь, да Твое, Христе Боже, славится имя.

\itshape Слава: \normalfont{}Церковь, юже еси освятил, на ту, Христе, не даждь поношения, но воздвигни невидимо в недузе на одре лежащаго, в нейже молим Ти ся: да не рекут невернии, где есть Бог их.

\itshape И ныне: \normalfont{}К Твоему пречистому, Богородице, образу руце воздеюще вопием, услыши раб Твоих молитву и спаси в болезни лежащаго, да от болезни востав, воздаст обеты, яже в печали глаголаша уста его.


\medskip


\bfseries Седален, глас 8-й:\normalfont{}


На одре греховней лежаща мя, и страстьми уязвена, и якоже воздвигл еси Петрову тещу и спасе разслабленнаго носима со одром, тако и ныне посети, Милостиве, болящаго, понесый недуги рода нашего. Тебе Единаго вемы, терпелива и милосерда, милостива Врача душам и телесем, Христа Бога нашего, наводяща недуги и паки исцеляюща, подавающа прощение кающимся о согрешениих, Единаго Милосердаго и Милостиваго.

\itshape Слава: \normalfont{}Аз грешный плачуся на одре своем лежа, прощение ми подаждь, Христе Боже, и от болезни сея воздвигни мя, и яже есмь от юности грехи сотворил, избыти ми сих молитвами Богородицы даруй.

\itshape И ныне: \normalfont{}Умилосердися и спаси мя, воздвигни мя от одра болезненнаго, мощь бо моя во мне изнеможе и весь нечаянием одержим есмь, Мати Божия Пречистая, исцели недугующа люте, Ты бо еси Помощница Христианом.


\medskip


\bfseries Песнь 4\normalfont{}


\itshape Ирмос: \normalfont{}Положил еси к нам твердую любовь, Господи: Единороднаго бо Твоего Сына за ны на смерть дал еси, темже Ти зовем благодаряще: слава силе Твоей, Господи.

\itshape Припев: \normalfont{}\bfseries Милостиве Господи, услыши молитву раб Твоих, молящихся Тебе.

\normalfont{}Уже отчаянна лютым недугом и к смерти приближившася, возврати, Христе, на живот и даждь плачущимся утеху, да вси прославляем Твоя святая чудеса.

\itshape Припев: \normalfont{}\bfseries Милостиве Господи, услыши молитву раб Твоих, молящихся Тебе.

\normalfont{}К Тебе, Творче, каемся в своих гресех, яко не хощеши смерти грешничи, оживи, уздрави болящаго, да востав послужит Ти, исповедуя с нами Твою благость.

\itshape Слава: \normalfont{}Слезы Манассиины, Ниневитян покаяние, Давидово исповедание приим, вскоре спасл еси, и наши молитвы ныне приими, даждь здравие болящему, о нем же Тя молим.

\itshape И ныне: \normalfont{}Подаждь нам Твою милость, Госпоже, всегда на Тя надеющимся, испроси здравие болящему, врачебнии Твои руце с Предтечею, Богородице, ко Господу Богу простирающи.


\medskip


\bfseries Песнь 5\normalfont{}


\itshape Ирмос: \normalfont{}На земли Невидимый явился еси, и человеком волею сожил еси Непостижимый, и к Тебе утренююще, воспеваем Тя, Человеколюбче.

\itshape Припев: \normalfont{}\bfseries Милостиве Господи, услыши молитву раб Твоих, молящихся Тебе.

\normalfont{}Дщерь Иаирову уже умершу, яко Бог оживил еси, и ныне возведи, Христе Боже, от врат смертных болящаго, Ты бо еси Путь и Живот всем.

\itshape Припев: \normalfont{}\bfseries Милостиве Господи, услыши молитву раб Твоих, молящихся Тебе.

\normalfont{}Сына вдовича оживив Спасе, и тоя слезы преложив на радость, спаси от недуга тлеющаго раба Твоего, да и наша скорбь и болезнь на радость приидет.

\itshape Слава: \normalfont{}Огненну болезнь тещи Петрове прикосновением Твоим исцелив, и ныне возстави болящаго раба Твоего, да востав яко Иона, послужит Ти.

\itshape И ныне: \normalfont{}Скорбнии, смиреннии, грешнии, не имущии дерзновения к Тебе, Пречистая Богомати, вопием, умоли Сына Твоего Христа дати болящему здравие телеси.


\medskip


\bfseries Песнь 6\normalfont{}


\itshape Ирмос: \normalfont{}Бездна последняя грехов обыде мя, и исчезает дух мой: но прострый, Владыко, высокую Твою мышцу, яко Петра мя, Управителю, спаси.

\itshape Припев: \normalfont{}\bfseries Милостиве Господи, услыши молитву раб Твоих, молящихся Тебе.

\normalfont{}Бездну милосердия и милости имея, Христе Боже, услыши моления раб Твоих. Ты бо еси Петром Тавифу воскресил, Ты и ныне в болезни лежащаго воздвигни, послушав церковных молитвенник.

\itshape Припев: \normalfont{}\bfseries Милостиве Господи, услыши молитву раб Твоих, молящихся Тебе.

\normalfont{}Врачу душам и телесем нашим, понесый недуги всего мира, Христе, и Енея Павлом исцеливый, Ты и ныне болящаго исцели святых апостол молитвами.

\itshape Слава: \normalfont{}Преложи, Христе, на радость рыдание о недужнем скорбящем, да Твою получивше милость, внидут в дом Твой со обетными дарми, славяще Тя в Троице Единаго Бога.

\itshape И ныне: \normalfont{}Приидите, о друзи, поклонимся молящеся о болящем Божией Матери. Та бо имеет власть недужныя исцеляти, со безмездникома вкупе, духовным невидимо помазующе маслом.


\medskip


\bfseries Кондак, глас 3-й:\normalfont{}


Душу мою, Господи, во гресех всяческих, безместными деяньми разслаблену воздвигни божественным Твоим человеколюбием, яко же и разслабленнаго воздвигл еси древле, да зову Ти спасаем: щедре Христе, даждь ми изцеление.


\medskip


\bfseries Икос:\normalfont{}


Иже руки Своей горстию содержай концы, Иисусе Боже, Иже Отцу собезначальный и Духу Святому совладычествуя, яко плотию явился еси недуги исцеляя и страсти очистил еси, слепыя просветил еси, и разслабленнаго словом Божественным совоставил еси, сего право ходяща сотворив и одр повелел еси на раму взяти. Темже вси с ним воспеваем и поем: щедре Христе, даждь ми иcцеление.


\medskip


\bfseries Песнь 7\normalfont{}


\itshape Ирмос: \normalfont{}Прежде образу златому, персидскому чтилищу отроцы не поклонишася, трие поюще посреде пещи: отцев Боже, благословен еси.

\itshape Припев: \normalfont{}\bfseries Милостиве Господи, услыши молитву раб Твоих, молящихся Тебе.

\normalfont{}О, Кресте Пресвятый Христов, честное Древо Животное. Тобою смерть погибе и мертвии воскресоша, и ныне болящаго изцели и оживи, якоже при Елене умершую девицу.

\itshape Припев: \normalfont{}\bfseries Милостиве Господи, услыши молитву раб Твоих, молящихся Тебе.

\normalfont{}Долгую и лютую болезнь Иовлю на гноищи и в червех, того молящася, словом изцелил еси, Господи. И ныне же в церкви о болящем молим Тя: яко благ, изцели невидимо молитвами святых Твоих.

\itshape Слава: \normalfont{}Вси вемы, яко умрети нам есть, Тебе Богу тако изволившу, но на мало время, Милостиве, здравия просим болящему, премени от смерти на живот, даждь скорбящим утеху.

\itshape И ныне: \normalfont{}Пособствуй и помогай нашему сиротству, Богородице, Ты бо веси время и час, когда умолити Сына Твоего и Бога нашего, дати болящему здравие и прощение от всех грех.


\medskip


\bfseries Песнь 8\normalfont{}


\itshape Ирмос: \normalfont{}Служити Живому Богу, в Вавилоне отроцы претерпевше, о мусикийских органех нерадиша, и посреде пламене стояще, боголепную песнь воспеваху, глаголюще: благословите вся дела Господня Господа.

\itshape Припев: \normalfont{}\bfseries Милостиве Господи, услыши молитву раб Твоих, молящихся Тебе.

\normalfont{}Милость покажи, Владыко, в болезни раба Твоего, и скоро изцели, милостиве Христе Боже, и аще смерти не предаждь судней, да Ти покаяние воздаст. Сам бо рекл еси: не хощу смерти грешничи.

\itshape Припев: \normalfont{}\bfseries Милостиве Господи, услыши молитву раб Твоих, молящихся Тебе.

\normalfont{}Господи милостиве, Твоя преславная чудеса да и до нас ныне достигнут: бесы прожени, недуги погуби, раны изцели, болезни уврачуй, и потворы и чародеяния и всякия язи избави ны.

\itshape Слава: \normalfont{}Запретивый, Христе, морским ветром, и страх ученик на радость преложивый, и ныне запрети тяжким болезнем, труждающим раба Твоего, да вси возвеселимся хваляща Тя во веки.

\itshape И ныне: \normalfont{}Избави, Богородице, от обышедших ны печалей, различных недуг, отравы же и чародейства, и бесовскаго мечтания, и от навета злых человек и от напрасныя смерти, молим Ти ся.


\medskip


\bfseries Песнь 9\normalfont{}


\itshape Ирмос: \normalfont{}На Синайстей горе виде Тя в купине Моисей, неопально огнь Божества заченшую во чреве: Даниил же Тя виде гору несекомую, жезл прозябший, Исайа взываше, от корене Давидова.

\itshape Припев: \normalfont{}\bfseries Милостиве Господи, услыши молитву раб Твоих, молящихся Тебе.

\normalfont{}Источниче жизни, подателю, Христе, милости, не отврати лица Твоего от нас. Облегчи болезнь утруженому болезнию, и воздвигни яко Фаддеом Авгаря, да присно Тебе славит со Отцем и Святым Духом.

\itshape Припев: \normalfont{}\bfseries Милостиве Господи, услыши молитву раб Твоих, молящихся Тебе.

\normalfont{}Евангельскому верующе гласу, Твоего ищем обета, Христе: просите бо, рече, и дастся вам. Тем и ныне предстояще молим Тя, возстави со одра здрава лютым повержена недугом, да Тя с нами вкупе величает.

\itshape Слава: \normalfont{}Мучимый недугом, внутрь невидимыми ранами, Тебе, Христе, с нами вопиет, не нам, Господи, не нас ради, вси бо мы исполнены грехов, но Матерними Твоими и Предтечевыми молитвами даждь исцеление болящему, да Тя вси возвеличим.

\itshape И ныне: \normalfont{}Божия Мати Пречистая, со всеми святыми призываем Тя, со ангелы и архангелы, с пророки и патриархи,

со апостолы, с преподобными и праведными молися Христу Богу нашему дати здравие болящему, да Тя вси величаем.


\medskip


\bfseries Молитва:\normalfont{}


Боже сильный, милостию строяй вся на спасение роду человеческому, посети раба Твоего сего \itshape (имя), \normalfont{}нарицающа имя Христа Твоего, исцели его от всякого недуга плотскаго: и отпусти грех и греховныя соблазны, и всяку напасть, и всяко нашествие неприязнено далече сотвори от раба Твоего. И воздвигни от одра греховнаго, и устрой его во святую Твою Церковь здрава душею и телом, и делы добрыми славящаго со всеми людьми имя Христа Твоего, яко Тебе славу возсылаем, со Безначальным Ти Сыном и со Святым Духом, ныне и присно и во веки веков. Аминь.


\mychapterending

\mychapter{Молитва до и после чтения Евангелия}
%http://www.molitvoslov.com/text553.htm 
 


\itshape Каждый день нужно читать по одной главе Евангелия, а перед и после главы "--- эту молитву:\normalfont{}




Спаси, Господи, и помилуй раба Твоего (\itshape имя\normalfont{}) словами Божественнаго Евангелия, чтомыми о спасении раба Твоего. 

Попали, Господи, терние всех его согрешений, и да вселится в него благодать Твоя, опаляющая, очищающая, освящающая всего человека во имя Отца и Сына и Святаго Духа. Аминь.





\mychapterending

\mychapter{Молитва болящей}
%http://www.molitvoslov.com/text552.htm 
 


Господи, видишь Ты мою болезнь. Ты знаешь, как я грешна и немощна; помоги мне терпеть и благодарить Твою Благость. Господи, сотвори, чтобы болезнь эта была в очищение многих моих грехов. Владыко Господи, я в руках Твоих, помилуй меня по воле Твоей и, если мне полезно, исцели меня вскоре. Достойное по делам моим приемлю; помяни мя, Господи, во Царствии Твоем! Слава Богу за все!





\mychapterending

\mychapter{Молитва благодарственная, святого Иоанна Кронштадского, читаемая после исцеления от болезни}
%http://www.molitvoslov.com/text556.htm 
 


Слава Тебе, Господи, Иисусе Христе, Сыне Единородный Безначальнаго Отца, едине изцеляяй всяк недуг и всяку язю в людех, яко помиловал мя еси грешнаго и избавил еси мя от болезни моей, не попустив ей развиться и умертвить меня по грехам моим. Даруй мне отныне, Владыко, силу твердо творить волю Твою во спасение души моея окаянныя и во славу Твою со Безначальным Твоим Отцем и Единосущным Твоим Духом, ныне и присно и во веки веков. Аминь.





\mychapterending

\mychapter{Молитва в болезни}
%http://www.molitvoslov.com/text551.htm 
 


Господи Боже, Владыко жизни моей, Ты по благости Твоей сказал: не хочу смерти грешника, но чтоб он обратился и жив был. Я знаю, что эта болезнь, которою я страдаю, есть наказание Твое за мои грехи и беззакония; знаю, что по делам моим я заслужил тягчайшее наказание, но, Человеколюбче, поступай со мною не по злобе моей, а по беспредельному милосердию Твоему. Не пожелай смерти моей, но дай мне силы, чтобы я терпеливо сносил болезнь, как заслуженное мною испытание, и по исцелении от нее обратился всем сердцем, всею душою и всеми моими чувствами к Тебе, Господу Богу, Создателю моему, и жив был для исполнения святых Твоих заповедей, для спокойствия моих родных и для моего благополучия. Аминь. 





\mychapterending

\mychapter{Молитва во время эпидемии}
%http://www.molitvoslov.com/text555.htm 
 




Господи Боже наш! Услышь с высоты святаго Твоего Престола нас, грешных и недостойных рабов Твоих, благость Твою грехами своими прогневавших и милосердие Твое удаливших, и не взыскивай с рабов Твоих, но отврати страшный гнев Твой, справедливо нас постигший, прекрати пагубное наказание, удали страшный меч Твой, невидимо и безвременно нас поражающий, и пощади несчастных и слабых рабов Твоих, и не обрекай на смерть души наши, в покаянии прибегающие с истомленным сердцем и со слезами к Тебе, Богу Милосердому, мольбам нашим внимающему и перемену подающему. Ибо Тебе (одному только) принадлежит милость и спасение, Боже наш, и Тебе славословие приносим, Отцу и Сыну и Святому Духу, ныне и присно и во веки веков. Аминь.





\mychapterending

\mychapter{Молитва на всякую немощь}
%http://www.molitvoslov.com/text550.htm 
 


Владыко Вседержителю, Врачу душ и телес, смиряяй и возносяй, наказуяй и паки исцеляяй, брата нашего (\itshape имя\normalfont{}) немощствующа посети милостию Твоею, простри мышцу Твою, исполнену исцеления и врачбы, и исцели его, возставляй от одра и немощи, запрети духу немощи, остави от него всяку язву, всяку болезнь, всяку рану, всяку огневицу и трясавицу. И аще есть в нем согрешение или беззаконие, ослаби, остави, прости, Твоего ради человеколюбия.





\mychapterending

\mychapter{Молитва за немощного и неспящего}
%http://www.molitvoslov.com/text554.htm 
 


Боже Великий, Хвальный и Непостижимый, и Неисповедимый, создавый человека рукою Твоею, персть взем от земли и образом Твоим почтивый его, явися на рабе Твоем \itshape (имя)\normalfont{}  и даждь ему сон успокоения, сон телесный, здравия и спасения живота, и крепость душевную и телесную. Сам убо, Человеколюбче Царю, предстани и ныне наитием Святаго Твоего Духа, и посети раба Твоего \itshape (имя)\normalfont{}, даруй ему здравие, крепость и благомощие Твоею благостию: яко от Тебе есть всяко даяние благо и всяк дар совершен. Ты бо еси Врач душ наших, и Тебе славу, и благодарение, и поклонение возсылаем со Безначальным Твоим Отцем и с Пресвятым и Благим и Животворящим Твоим Духом, ныне и присно и во веки веков. Аминь. 


\itshape О том же молятся святым семи отрокам и Ангелу Хранителю болящего.\normalfont{}





\mychapterending

\mychapter{Молитва о том, чтобы с любовью ухаживать за болящими}
%http://www.molitvoslov.com/text549.htm 
 


Господи, Иисусе Христе, Сыне Бога Живаго, Агнче Божий, вземляй грехи мира, Пастырю добрый, положивый душу Твою за овцы Твоя, Небесный Врачу душ и телес наших, исцеляяй всякий недуг и всякую язву в людех Твоих! Тебе припадаю, помози мне, недостойной рабе Твоей. Призри, Многомилостиве, на делание и служение мое, даждь ми быти верною в мале; послужити болящим, Тебе ради, носити немощи немощных, и не себе, но Тебе Единому угождати во вся дни живота моего. Ты бо рекл еси, о, Сладчайший Иисусе: "Понеже сотвористе единому от сих братий Моих меньших, Мне сотвористе". Ей, Господи, суди мне, грешной, по сему глаголу Твоему, яко да сподоблюся творити благую Твою волю во отраду и утешение искушаемых, недугующих раб Твоих, ихже искупил еси честною Твоею Кровию. Ниспосли ми благодать Твою, попаляющую во мне страстей терние, призвавый мя, грешную, на дело служения о Имени Твоем; без Тебе не можем творити ничесоже: посети убо нощию и искуси сердце мое, внегда предстояти ми у возглавия болящих и низверженных; уязви душу мою Твоею любовию, вся терпящею и николиже отпадающею. Тогда возмогу, Тобою укрепляема, подвигом добрым подвизатися и веру соблюсти, даже до последнего моего издыхания. Ты бо еси Источник исцелений душевных же и телесных, Христе Боже наш, и Тебе, яко Спасителю человеков и Жениху душ, грядущему в полунощи, славу и благодарение и поклонение возсылаем, ныне и присно и во веки веков. Аминь. 





\mychapterending

\mychapter{Молитва Пресвятой Богородице за болящего}
%http://www.molitvoslov.com/content/Molitva-Presvyatoi-Bogoroditse-za-bolyashchego 
 


Пресвятая Богородица, всесильным заступлением Твоим помоги мне умолить Сына Твоего, Бога моего, об исцелении раба Божия (\itshape имя\normalfont{})


\mychapterending

\mychapter{Молитва всем святым и ангелам за болящего}
%http://www.molitvoslov.com/content/Molitva-vsem-svyatym-i-angelam-za-bolyashchego 
 


Все святые и ангелы Господни, молите Бога о больном рабе Его (\itshape имя\normalfont{}). Аминь.


\mychapterending
\chapter*{Список изменений}

Этот раздел содержит информацию об изменениях и исправлениях на сайте \url{www.molitvoslov.com}, вошедших в эту и предыдущие публикации. Информация о новых версиях располагается в начале.
 
%\begin{center}
%\begin{tabular}{lp{0.7\textwidth}l}
%\toprule
%Версия & Изменения\\
%\midrule
%20110606 &
%Обновление текстов с сайта от 06 мая 2011 г.
%\\
%20110503.02 & Содержимое сайта от 03 мая 2011 г. \\
%\bottomrule
%\end{tabular}
%\end{center}

\small

\section*{Версия 20110606}

Обновление текстов с сайта от 06 июня 2011 г.
\begin{itemize}

\item В разделе «НАПУТСТВИЕ ХРИСТИАНИНА ПЕРЕД СМЕРТЬЮ И ЗАУПОКОЙНЫЕ МОЛИТВЫ»
\begin{itemize}
\item Исправлен «Чин литии, совершаемой мирянином дома и на кладбище» "--- убрана «Разрешительная молитва, читаемая при смерти\ldots»
\item Исправлен тропарь прп. Паисию Великому в молитве «Об ослаблении вечных мук умерших без покаяния», а также в разделе «КАНОН ПРЕПОДОБНОМУ ПАИСИЮ ВЕЛИКОМУ ОБ ИЗБАВЛЕНИИ ОТ МУК УМЕРШИХ БЕЗ ПОКАЯНИЯ».
\item Исправлен «Канон молебный ко Господу Иисусу Христу и Пречистой Богородице Матери Господни при разлучении души от тела всякого правоверного».
\end{itemize}

\item В разделе «МОЛИТВЫ СВЯТЫМ»
\begin{itemize}
\item Убраны дубли молитв прп. Александру Свирскому и прп. Серафиму Саровскому.
\item Добавлены молитвы св. первомученику архидиакону Стефану и св. царице Грузинской Тамаре.
\end{itemize}

\item В разделе «МОЛИТВЫ В СКОРБЯХ И ИСКУШЕНИЯХ ТВОРИМЫЕ»
\begin{itemize}
\item Добавлена «Молитва преследуемого человеками (свт. Игнатия Брянчанинова)».
\end{itemize}

\item В разделе «МОЛИТВЫ ЗА БОЛЯЩИХ»
\begin{itemize}
\item Исправлена опечатка в «Каноне за болящего, глас 3-й», Песнь 9.
\end{itemize}


\end{itemize}


\section*{Версия 20110503.02}
Содержимое сайта от 03 мая 2011 г.

\normalfont


\end{document}
