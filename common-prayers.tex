\newcommand\pripev[2][Припев:]{{\small\myemph{#1} #2}}
\newcommand\irmos[1]{\pripev[Ирм\'{о}с:]{#1}\smallskip}
\newcommand\Bogorodichen[1]{\myemph{Богородичен:} #1}
\newcommand\slavan{Слава Отцу и Сыну и Святому Духу.}
\newcommand\slava{{\small\slavan}}
\newcommand\inynen{И ныне и присно и во веки веков. Аминь.}
\newcommand\inyne{{\small\inynen}}
\newcommand\slavainynen{Слава Отцу и Сыну и Святому Духу. И ныне и присно и во веки веков. Аминь.}
\newcommand\slavainyne{{\small\slavainynen}}

\newcommand{\pripevc}[1]{{\small \centerline{#1} \nopagebreak}}
\newcommand{\pripevmskipc}[1]{\medskip\pripevc{#1}}
\newcommand{\pripevpomiluj}{\pripevmskipc{\firstletter{П}омилуй мя, Боже, помилуй мя.}}
\newcommand{\slavac}{\pripevmskipc{\slavan}}
\newcommand{\inynec}{\pripevmskipc{\inynen}}

\newcommand{\TrisviatoePoOtcheNash}{%
Святый Боже, Святый Крепкий, Святый Безсмертный, помилуй нас. \myemph{ (Tрижды)}

Слава Отцу и Сыну и Святому Духу, и ныне и присно и во веки веков. Аминь.

Пресвятая Троице, помилуй нас; Господи, очисти грехи наша; Владыко, прости беззакония наша; Святый, посети и исцели немощи наша, имене Твоего ради.

Господи, помилуй. \myemph{ (Трижды)}

Слава Отцу и Сыну и Святому Духу, и ныне и присно и во веки веков. Аминь.

Отче наш, Иже еси на небесех! Да святится имя Твое, да приидет Царствие Твое, да будет воля Твоя, яко на небеси и на земли. Хлеб наш насущный даждь нам днесь; и остави нам долги наша, якоже и мы оставляем должником нашим; и не введи нас во искушение, но избави нас от лукаваго.
}

\newcommand{\priiditepoklonimsia}{%
Приидите, поклонимся Цареви нашему Богу. \myemph{(Поклон)}

Приидите, поклонимся и припадем Христу, Цареви нашему Богу. \myemph{(Поклон)}

Приидите, поклонимся и припадем Самому Христу, Цареви и Богу нашему. \myemph{(Поклон)}


\newcommand{\PsalmFifty}{%
Помилуй мя, Боже, по велицей милости Твоей, и по множеству щедрот Твоих очисти беззаконие мое. Наипаче омый мя от беззакония моего, и от греха моего очисти мя; яко беззаконие мое аз знаю, и грех мой предо мною есть выну. Тебе Единому согреших и лукавое пред Тобою сотворих, яко да оправдишися во словесех Твоих, и победиши внегда судити Ти. Се бо, в беззакониих зачат есмь, и во гресех роди мя мати моя. Се бо, истину возлюбил еси; безвестная и тайная премудрости Твоея явил ми еси. Окропиши мя иссопом, и очищуся; омыеши мя, и паче снега убелюся. Слуху моему даси радость и веселие; возрадуются кости смиренныя. Отврати лице Твое от грех моих и вся беззакония моя очисти. Сердце чисто созижди во мне, Боже, и дух прав обнови во утробе моей. Не отвержи мене от лица Твоего и Духа Твоего Святаго не отыми от мене. Воздаждь ми радость спасения Твоего и Духом владычним утверди мя. Научу беззаконыя путем Твоим, и нечестивии к Тебе обратятся. Избави мя от кровей, Боже, Боже спасения моего; возрадуется язык мой правде Твоей. Господи, устне мои отверзеши, и уста моя возвестят хвалу Твою. Яко аще бы восхотел еси жертвы, дал бых убо: всесожжения не благоволиши. Жертва Богу дух сокрушен; сердце сокрушенно и смиренно Бог не уничижит. Ублажи, Господи, благоволением Твоим Сиона, и да созиждутся стены Иерусалимския. Тогда благоволиши жертву правды, возношение и всесожегаемая; тогда возложат на oлтарь Твой тельцы.\par}

\newcommand{\Chestneyshuyu}{%
Достойно есть яко воистинну блажити Тя, Богородицу, Присноблаженную и Пренепорочную и Матерь Бога нашего. Честнейшую Херувим и славнейшую без сравнения Серафим, без истления Бога Слова рождшую, сущую Богородицу Тя величаем.}
