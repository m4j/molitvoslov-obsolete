\chapter{Словарь и термины}
\markright{Словарь и термины}
%http://www.molitvoslov.com/slovar.php 
 
\begin{mymulticols}{2}\footnotesize

\bukva{А}


\noindent\textbf{Абие} "--- немедленно, тотчас. 




\noindent\textbf{Авва} "--- отец. 




\noindent\textbf{Аввадон} "--- евр. «Погубитель»; имя ангела бездны. 




\noindent\textbf{Авраамово недро, лоно} "--- иносказательно: рай, место вечного блаженства. 




\noindent\textbf{Агаряне} "--- потомки Исмаила, сына Агари, наложницы Авраама, иносказательно "--- кочевые восточные племена. 




\noindent\textbf{Агиасма} "--- освященная по церковному чину вода. Освященная на празднике Богоявления вода называется Великой агиасмой. 




\noindent\textbf{Агиос} "--- надписание на древних иконах; греч. «святой». 




\noindent\textbf{Агкира} (читается «анкира») "--- якорь. 




\noindent\textbf{Агнец} "--- ягненок; чистое, кроткое существо; изымаемая на проскомидии часть просфоры для Евхаристии; мн. ч. "--- « \noindent\textbf{агнцы}» "--- иногда значит «христиане». 




\noindent\textbf{Агница} "--- овечка. 




\noindent\textbf{Агня} "--- ягненок. 




\noindent\textbf{Ад} "--- место нахождения душ умерших до освобождения их Господом Иисусом Христом; место вечного мучения грешников; жилище диавола. 




\noindent\textbf{Адамант} "--- алмаз; бриллиант; драгоценный камень. 




\noindent\textbf{Адамантовый} "--- твердый; крепкий; драгоценный. 




\noindent\textbf{Адов} "--- адский. 




\noindent\textbf{Адонаи} "--- евр. «Господь мой». 




\noindent\textbf{Аер} "--- покровец, полагаемый сверху священных сосудов на Литургии. 




\noindent\textbf{Аермонский} "--- связанный с горой Аермон. 




\noindent\textbf{Аз} "--- я. 




\noindent\textbf{Аиромантия} "--- воздуховолхвование, т.~е. суеверное гадание на основании атмосферных явлений. 




\noindent\textbf{Акафист} "--- греч. «неседальное»; церковная служба, во время которой возбраняется сидеть. 




\noindent\textbf{Аки} "--- как будто, как бы. 




\noindent\textbf{Акриды} "--- пища Иоанна Крестителя; по мнению одних "--- род съедобной саранчи, или кузнечиков; по мнению других "--- какое-то растение. 




\noindent\textbf{Аксиос} "--- греч. «достоин». 




\noindent\textbf{Алавастр} "--- каменный сосуд. 




\noindent\textbf{Алектор} "--- петух. 




\noindent\textbf{Алкати} "--- голодать; хотеть есть, сильно желать чего-либо. 




\noindent\textbf{Алкота} "--- голод. 




\noindent\textbf{Аллилуия} "--- евр. «хвалите Бога»; «слава Богу!» 




\noindent\textbf{Аллилуия красная} "--- пение «аллилуйя» на особый умилительный распев. См. Триодь постную. 




\noindent\textbf{Аллилуиарий, аллилуиар} "--- стих, возглашаемый чтецом после чтения Апостола на Литургии. При этом возглашении на клиросах поют «аллилуия». 




\noindent\textbf{Алой, алое} "--- сок благовонного дерева, употреблявшийся для каждения и бальзамирования. 




\noindent\textbf{Алтабас} "--- самая лучшая старинная парча. 




\noindent\textbf{Амалик} "--- народ, живший между Палестиною и Египтом. В церковной поэзии это имя часто прилагается к диаволу. 




\noindent\textbf{Амвон} "--- возвышенная часть храма перед царскими вратами. 




\noindent\textbf{Амвросия} "--- неистлеваемая пища. 




\noindent\textbf{Амигдал} "--- миндаль. 




\noindent\textbf{Аминь} "--- евр. «да будет так»; «истинно»; «подлинно»; «да». 




\noindent\textbf{Амо, аможе} "--- куда. 




\noindent\textbf{Аможе аще} "--- куда бы ни. 




\noindent\textbf{Аналой} (правильнее \noindent\textbf{аналогий}) "--- возвышенный стол, на который полагаются церковные книги при чтении и иконы. 




\noindent\textbf{Анафема} "--- отлучение от общины верных и предание суду Божию; тот, кто подвергся такому отлучению. 




\noindent\textbf{Анафематствовати} "--- предавать анафеме. 




\noindent\textbf{Анахорет} "--- отшельник. 




\noindent\textbf{Ангел} "--- вестник. 




\noindent\textbf{Ангеловидный} "--- внешне напоминающий Ангела. 




\noindent\textbf{Ангелозрачный} "--- внешне напоминающий Ангела. 




\noindent\textbf{Ангелоименитый} "--- знаменитый, почитаемый в лике ангельском; носящий имя какого-либо Ангела. 




\noindent\textbf{Ангелолепный} "--- приличный Ангелу. 




\noindent\textbf{Ангеломудренный} "--- имеющий мудрость Ангела. 




\noindent\textbf{Ангельское житие, ангельский образ} "--- высшая степень монашеского совершенства; греч. «схима». 




\noindent\textbf{Анепсий} "--- племянник, родственник. 




\noindent\textbf{Антидор} "--- благословенный хлеб, т.~е. остатки той просфоры, из которой на проскомидии был изъят Агнец. 




\noindent\textbf{Антиминс} "--- греч. «вместопрестолие», освященный плат с изображением Иисуса Христа во гробе и вшитыми св. мощами. Только на антиминсе может быть совершаема Литургия. 




\noindent\textbf{Антифон} "--- греч. «противугласник»; песнопение, которое должно быть пето попеременно на обоих клиросах. 




\noindent\textbf{Антихрист} "--- греч. «противник Христа». 




\noindent\textbf{Антологион} "--- греч. «Цветослов»; название «Минеи праздничной». 




\noindent\textbf{Анфипат} "--- наместник, проконсул. 




\noindent\textbf{Анфракс} "--- яхонт. 




\noindent\textbf{Апокалипсис} "--- греч. «откровение». 




\noindent\textbf{Аполлион} "--- греч. «Погубитель»; имя ангела бездны. 




\noindent\textbf{Апостол} "--- греч. «посланник». 




\noindent\textbf{Апостасис} "--- отступничество. 




\noindent\textbf{Апостата} "--- отступник. 




\noindent\textbf{Априллий} "--- апрель. 




\noindent\textbf{Ариил} "--- горн у жертвенника всесожжения в храме Иерусалимском. 




\noindent\textbf{Армония} "--- гармония. 




\noindent\textbf{Ароматы} "--- душистая мазь. 




\noindent\textbf{Артос} "--- греч.хлеб квасной; он освящается с особой молитвой в день св. Пасхи. 




\noindent\textbf{Архангел} "--- начальствующий у Ангелов, название одного из чинов ангельских. 




\noindent\textbf{Архиерей} "--- первосвященник, епископ. 




\noindent\textbf{Архимагир} "--- главный повар. 




\noindent\textbf{Архипастырь} "--- первенствующий епископ. 




\noindent\textbf{Архисинагог} "--- начальник синагоги. 




\noindent\textbf{Архистратиг} "--- военачальник, полководец. 




\noindent\textbf{Архитектон} "--- архитектор, художник-строитель; главный строитель. 




\noindent\textbf{Архитриклин} "--- распорядитель пира. 




\noindent\textbf{Асмодей, Азмодеос} "--- «губитель», имя бесовское. 




\noindent\textbf{Аспид} "--- ядовитая змея. 




\noindent\textbf{Аспид парящий} "--- летающий ящер. 




\noindent\textbf{Ассарий} "--- мелкая медная монетка. 




\noindent\textbf{Астерикс} "--- звездица, поставляемая на дискосе при совершении Литургии. 




\noindent\textbf{Афарим} "--- соглядатаи; лазутчики. 




\noindent\textbf{Афедрон} "--- задний проход (Мф. 15, 17). 




\noindent\textbf{Афинеи} "--- афиняне. 




\noindent\textbf{Африкия} "--- Африка. 




\noindent\textbf{Аще} "--- если; хотя; или; ли. 




\noindent\textbf{Аще убо} "--- поскольку; потому что. 




\bukvaending

\bukva{Б} 





\noindent\textbf{Баальник} "--- волшебник. 




\noindent\textbf{Баба} "--- повивальная женщина. 




\noindent\textbf{Бабити} "--- помогать при родах. 




\noindent\textbf{Багряница} "--- ткань темно-красного цвета; порфира, пурпурная одежда высокопоставленных особ. 




\noindent\textbf{Балия} "--- колдунья; волшебница. 




\noindent\textbf{Баня пакибытия} "--- таинство св. Крещения. 




\noindent\textbf{Баснословити} "--- рассказывать небылицы; лгать. 




\noindent\textbf{Баснь} "--- ложное и бесполезное учение. 




\noindent\textbf{Бдение} "--- бодрствование; продолжительное ночное богослужение. 




\noindent\textbf{Бденно} "--- неусыпно, бодрственно. 




\noindent\textbf{Бденный} "--- неусыпный. 




\noindent\textbf{Бдети} "--- бодрствовать; не спать. 




\noindent\textbf{Бедне} "--- трудно; несносно; тяжело. 




\noindent\textbf{Бесоватися} "--- бесноваться, неистоваться. 




\noindent\textbf{Бедник} "--- калека; увечный. 




\noindent\textbf{Бедный} "--- иногда: увечный; калека. 




\noindent\textbf{Безведрие} "--- ненастье. 




\noindent\textbf{Безвидный} "--- не имеющий вида, образа. 




\noindent\textbf{Безвиновный} "--- не имеющий начала или причины для своего бытия. Одно из Божественных определений. 




\noindent\textbf{Безвозрастное} "--- младенец. 




\noindent\textbf{Безвременне} "--- некстати; неблаговременно. 




\noindent\textbf{Безгласие} "--- немота; молчание. 




\noindent\textbf{Безгодие} "--- бедствие; несчастье; тяжелый период в жизни. 




\noindent\textbf{Безквасный} "--- пресный; не кислый. 




\noindent\textbf{Безкнижный} "--- неученый. 




\noindent\textbf{Безлетно} "--- бесконечно; вечно; прежде всех времен. 




\noindent\textbf{Безматерен} "--- не имеющий матери. 




\noindent\textbf{Безмездник} "--- не принимающий мзды, платы. 




\noindent\textbf{Безмилостивный} "--- не чувствующий или не оказывающий милости, жалости. 




\noindent\textbf{Безмолвник} "--- пустынножитель; отшельник. 




\noindent\textbf{Безмолвный} "--- иногда значит: безопасный; спокойный. 




\noindent\textbf{Безневестный} "--- безбрачный; девственный. 




\noindent\textbf{Безочство} "--- нахальство; бесстыдство; наглость. 




\noindent\textbf{Безпрестани} "--- всегда; непрерывно. 




\noindent\textbf{Безпреткновенный} "--- не имеющий претыкания, соблазна, препятствия. 




\noindent\textbf{Безпутие} "--- совращение с пути; развращение. 




\noindent\textbf{Безсквернен} "--- не имеющий скверны или порока. 




\noindent\textbf{Безсловеснство} "--- скотство; глупость; безумие. 




\noindent\textbf{Безсловесные} "--- животные, скоты. 




\noindent\textbf{Безсребреник} "--- человек, трудящийся даром, бесплатно. 




\noindent\textbf{Безстудие} "--- бесстыдство. 




\noindent\textbf{Безцарный} "--- не имеющий над собой царя. 




\noindent\textbf{Безчадие} "--- неимение, лишение детей. 




\noindent\textbf{Безчаствовати} "--- лишать положенной части; обделять. 




\noindent\textbf{Безчестен} "--- обесчещенный. 




\noindent\textbf{Безчиние} "--- беспорядок; неустройство; смешение. 




\noindent\textbf{Безчинновати} "--- вести беспорядочную жизнь. 




\noindent\textbf{Безчисльство} "--- бесчисленное множество. 




\noindent\textbf{Бервенный} "--- деревянный. 




\noindent\textbf{Бесный} "--- бесноватый. 




\noindent\textbf{Бийца} "--- драчун. 




\noindent\textbf{Било} "--- колотушка, при помощи которой в монастырях созывают на молитву. 




\noindent\textbf{Бисер} "--- жемчуг. 




\noindent\textbf{Благий} "--- хороший; добрый. 




\noindent\textbf{Благовест} "--- удары колокола, созывающие христиан на молитву в храм. От \textbf{«звона»} отличается тем, что благовестят в один колокол, а звонят во многие. 



\noindent\textbf{Благовестити} "--- возвещать доброе; проповедовать. 




\noindent\textbf{Благоверный} "--- исповедующий правую веру; православный. 




\noindent\textbf{Благовещение} "--- добрая весть. 




\noindent\textbf{Благоволити} "--- хорошо относиться к кому-нибудь; принимать участие в ком-либо. 




\noindent\textbf{Благовоние} "--- благоухание, хороший запах. 




\noindent\textbf{Благовременне} "--- в удобное время. 




\noindent\textbf{Благогласник} "--- проповедник слова Божия. 




\noindent\textbf{Благодатный} "--- преисполненный Божественной благодати. 




\noindent\textbf{Благоделие} "--- доброе, богоугодное дело. 




\noindent\textbf{Благодушествовати} "--- радоваться. 




\noindent\textbf{Благоискусный} "--- имеющий знание в полезных вещах. 




\noindent\textbf{Благокласный} "--- приносящий обильную жатву. 




\noindent\textbf{Благоключимый} "--- случившийся вовремя. 




\noindent\textbf{Благокрасный} "--- очень красивый. 




\noindent\textbf{Благолепие} "--- красота; великолепие; богатое убранство. 




\noindent\textbf{Благолепно} "--- красиво; прилично. 




\noindent\textbf{Благолозный} "--- приносящий обильные, хорошие плоды. 




\noindent\textbf{Благолюбец} "--- склонный к добру. 




\noindent\textbf{Благомилостивый} "--- милосердный. 




\noindent\textbf{Благомощие} "--- крепость; сила. 




\noindent\textbf{Благомужство} "--- благоразумная храбрость, доблесть. 




\noindent\textbf{Благонаказательный} "--- направляющий к благонравию. 




\noindent\textbf{Благоодеждный} "--- украшенный изящной одеждой. 




\noindent\textbf{Благоотдатливый} "--- воздающий добром за зло. 




\noindent\textbf{Благопитание} "--- сладкая, вкусная пища. 




\noindent\textbf{Благопослушливый} "--- слушающий со вниманием; послушный. 




\noindent\textbf{Благопослушный} "--- внимательный; легко, хорошо слышимый. 




\noindent\textbf{Благопотребный} "--- хорошо устроенный; угодный; необходимый. 




\noindent\textbf{Благоразтворение} "--- очищение; оздоровление; прояснение. 




\noindent\textbf{Благоразтворити} "--- очищать; оздоровлять. 




\noindent\textbf{Благорасленный} "--- хорошо растущий. 




\noindent\textbf{Благорозгный} "--- ветвистый. 




\noindent\textbf{Благосеннолиственный} "--- тенистый. 




\noindent\textbf{Благосенный} "--- производящий обильную тень. 




\noindent\textbf{Благословенный} "--- прославляемый; восхваляемый; превозносимый. 




\noindent\textbf{Благословити} "--- посвятить Богу; желать добра; хвалить; помолиться о ниспослании Божией благодати на кого-либо; дозволить; пожелать добра. 




\noindent\textbf{Благословная вина} "--- уважительная причина. 




\noindent\textbf{Благостояние} "--- твердость, крепость (в добре, против зла). 




\noindent\textbf{Благостыня} "--- благодеяние; милосердие; добродетель, доброе дело. 




\noindent\textbf{Благость} "--- доброта. 




\noindent\textbf{Благотещи} "--- быстро идти. 




\noindent\textbf{Благоуветие} "--- снисхождение. 




\noindent\textbf{Благоуветливый} "--- снисходительный. 




\noindent\textbf{Благоутишие} "--- тихая, ясная погода. 




\noindent\textbf{Благоутробие} "--- милосердие. 




\noindent\textbf{Благохваление} "--- откровенная похвала. 




\noindent\textbf{Благоцветный} "--- испещренный; изобилующий цветами. 




\noindent\textbf{Благочествовати} "--- благоговеть; благоговейно почитать кого-либо. 




\noindent\textbf{Благочестие} "--- истинное Богопочитание. 




\noindent\textbf{Благочестивый, благочестный} "--- богобоязненный; благоговейный; почитающий Бога. 




\noindent\textbf{Блаженный} "--- счастливый. 




\noindent\textbf{Блажити} "--- ублажать; прославлять. 




\noindent\textbf{Блазнити} "--- соблазнять. 




\noindent\textbf{Блед, бледый} "--- бледный. 




\noindent\textbf{Блещатися} "--- блистать; сиять; светить. 




\noindent\textbf{Близна} "--- рубец; морщина; складка. 




\noindent\textbf{Блистание} "--- сверкание; излияние света, блеска. 




\noindent\textbf{Блудилище} "--- непотребный дом. 




\noindent\textbf{Блудодей} "--- нарушитель супружества. 




\noindent\textbf{Блудопитие} "--- побуждающая к блуду попойка. 




\noindent\textbf{Блужение} "--- неверность Богу истинному, служение идолам (Исх. 34, 15; Суд. 8, 33). Как нарушение брачного союза есть блудодеяние, так в духовном смысле и нарушение союза с Богом есть служение идолам, хождение во след богов иных, то есть блужение, тем более что некоторые виды идолослужения сопровождались блудом в собственном смысле слова. 




\noindent\textbf{Блюдомый} "--- сохраняемый. 




\noindent\textbf{Блюсти} "--- хранить; беречь; соблюдать. 




\noindent\textbf{Блядение} "--- суесловие; ложные слова; вранье. 




\noindent\textbf{Бо} "--- потому что; так как; ибо; поскольку. 




\noindent\textbf{Богатити} "--- обогащать. 




\noindent\textbf{Богоглаголивый} "--- говорящий по внушению от Бога или от Его Имени. 




\noindent\textbf{Богодельне} "--- по действию Бога. 




\noindent\textbf{Боголепно} "--- так, как прилично Богу. 




\noindent\textbf{Боголепный} "--- имеющий Божественную красоту, достоинство. 




\noindent\textbf{Богомужный} "--- Богочеловеческий. 




\noindent\textbf{Богоначальный} "--- имеющий в Боге свое начало. 




\noindent\textbf{Богоотец} "--- это название в церковных книгах усвояется Давиду, от рода которого родился Христос. 




\noindent\textbf{Бодренно} "--- бдительно; неусыпно. 




\noindent\textbf{Болезновати} "--- терпеть боль; страдать. 




\noindent\textbf{Болий} "--- больший. 




\noindent\textbf{Борзе} "--- скоро. 




\noindent\textbf{Борзитися} "--- спешить. 




\noindent\textbf{Боритель} "--- противник. 




\noindent\textbf{Брада} "--- борода. 




\noindent\textbf{Брадатый} "--- бородатый. 




\noindent\textbf{Бразда} "--- борозда. 




\noindent\textbf{Бракоокрадованная} "--- лишенная целомудрия, девственности. 




\noindent\textbf{Бранити} "--- запрещать; оборонять; препятствовать. 




\noindent\textbf{Брань} "--- война; битва. 




\noindent\textbf{Братися} "--- бороться; воевать. 




\noindent\textbf{Брашно} "--- пища; еда. 




\noindent\textbf{Бремя} "--- ноша; тяжесть. 




\noindent\textbf{Брение} "--- глина; грязь. 




\noindent\textbf{Бренный} "--- взятый из земли; слабый; непрочный. 




\noindent\textbf{Брещи} "--- стеречь; хранить. 




\noindent\textbf{Брозда} "--- удила (часть конской сбруи). 




\noindent\textbf{Бряцати} "--- звенеть. 




\noindent\textbf{Будильник} "--- один из монахов в обители, будящий на молитву братию. 




\noindent\textbf{Буесловие} "--- глупые речи; вранье. 




\noindent\textbf{Буесловити} "--- говорить глупые речи. 




\noindent\textbf{Буй, (буий)} "--- безумный; сумасшедший; глупый. 




\noindent\textbf{Буйство} "--- глупость; безумие; сумасшествие. 




\noindent\textbf{Былие} "--- трава. 




\bukvaending

\bukva{В} 





\noindent\textbf{Вавилонское семя} "--- племя нечестивцев. 




\noindent\textbf{Вага} "--- весы; тяжесть. 




\noindent\textbf{Вадити} "--- делать ложный донос; клеветать; обвинять; приманивать; привлекать. 




\noindent\textbf{Ваия} "--- ветви; листья. 




\noindent\textbf{Вайный} "--- состоящий из ваий. 




\noindent\textbf{Валсамный} "--- благовонный; пахучий; ароматический. 




\noindent\textbf{Вап (а)} "--- краска. 




\noindent\textbf{Вар} "--- зной; жара; кипяток. 




\noindent\textbf{Варити} "--- предварять; упреждать; опереживать, предостерегать. 




\noindent\textbf{Василиск} "--- большая ядовитая змея. 




\noindent\textbf{Вборзе} "--- скоро. 




\noindent\textbf{Ввергати} "--- вбрасывать. 




\noindent\textbf{Вдавати} "--- поручать; передавать; доверять. 




\noindent\textbf{Веглас} "--- знающий; искусный. 




\noindent\textbf{Ведети} "--- знать. 




\noindent\textbf{Ведунство} "--- волхвование; ворожба; чародейство. 




\noindent\textbf{Веельзевул} "--- «повелитель мух»; начальник злых духов; одно из имен сатаны. 




\noindent\textbf{Вежди} "--- веки; ресницы. 




\noindent\textbf{Веие} "--- ветвь; сучок. 




\noindent\textbf{Велелепие} "--- красота; великолепие; украшение. 




\noindent\textbf{Велелепота} "--- красота; великолепие; украшение. 




\noindent\textbf{Велемудренно} "--- высокомудренно. 




\noindent\textbf{Веление} "--- указ; повеление; заповедь; учение. 




\noindent\textbf{Велеречивый} "--- многословный; хвастливый. 




\noindent\textbf{Велеречити} "--- много говорить; хвастать; гордиться. 




\noindent\textbf{Велиар} (или \textbf{Велиал}) "--- одно из имен диавола. 




\noindent\textbf{Велий} "--- великий; сильный. 




\noindent\textbf{Великовыйный} "--- гордый. 




\noindent\textbf{Великое} "--- самый большой, главный колокол. 




\noindent\textbf{Великодарный, великодаровный, великодаровитый} "--- щедро награждающий. 




\noindent\textbf{Величатися} "--- гордиться; хвалиться; кичиться. 




\noindent\textbf{Велми} "--- весьма; очень. 




\noindent\textbf{Вельблуд, велбуд} "--- верблюд; толстый канат. 




\noindent\textbf{Вельблуждь} "--- верблюжий. 




\noindent\textbf{Вено} "--- плата жениха за невесту. 




\noindent\textbf{Венчати} "--- возлагать венок или венец; удостаивать; сподоблять. 




\noindent\textbf{Вепрь} "--- дикий кабан. 




\noindent\textbf{Вербие} "--- ива; лоза. 




\noindent\textbf{Вервица} "--- четки. 




\noindent\textbf{Верея} "--- дверь; перекладина; столб у ворот. 




\noindent\textbf{Вержение} "--- кидание; метание; бросание. 




\noindent\textbf{Вержение камени} "--- расстояние, равное тому, на какое можно бросить камень. 




\noindent\textbf{Верзити} "--- кинуть. 




\noindent\textbf{Верзитися} "--- упасть. 




\noindent\textbf{Вериги} "--- цепи; оковы. 




\noindent\textbf{Верт, вертоград} "--- сад. 




\noindent\textbf{Вертеп} "--- пещера. 




\noindent\textbf{Вертоградарь} "--- садовник. 




\noindent\textbf{Верху} "--- на; над; сверху. 




\noindent\textbf{Весь} "--- селение, деревушка. 




\noindent\textbf{Ветия} "--- оратор; ритор. 




\noindent\textbf{Ветрило} "--- парус. 




\noindent\textbf{Ветхий деньми} "--- имя Божие в Дан. 7, 9. На основании этого пророческого видения в иконографической традиции новозаветной Церкви образ Бога Отца изображается в виде старца. 




\noindent\textbf{Вечеря} "--- ужин; пир. 




\noindent\textbf{Вечеряти} "--- ужинать. 




\noindent\textbf{Вещенеистовный} "--- пристрастившийся до безумия к тленным благам. 




\noindent\textbf{Вещь} "--- дело; событие. 




\noindent\textbf{Взаим} "--- в долг; взаймы. 




\noindent\textbf{Взиматися} "--- подниматься. 




\noindent\textbf{Взыгратися} "--- играть; скакать; веселиться. 




\noindent\textbf{Взыскати} "--- стремиться, искать. 




\noindent\textbf{Взятися} "--- взяться; отвориться; совершиться. 




\noindent\textbf{Вина} "--- причина; обвинение; извинение. 




\noindent\textbf{Винарь} "--- виноградарь. 




\noindent\textbf{Винничина} "--- виноградная лоза. 




\noindent\textbf{Винопийца} "--- пьяница. 




\noindent\textbf{Винопитие} "--- употребление вина. 




\noindent\textbf{Висети} "--- висеть; держаться на чем-либо. 




\noindent\textbf{Виссон} "--- драгоценная тонкая пряжа желтоватого цвета или одежда из этой ткани. 




\noindent\textbf{Виталище} "--- место жительства; жилище. 




\noindent\textbf{Виталница} "--- комната; гостиница; постоялый двор; ночлег. 




\noindent\textbf{Витати} "--- обитать; пребывать; проживать; ночевать. 




\noindent\textbf{Вкупе} "--- вместе. 




\noindent\textbf{Влагалище} "--- мешок; карман; ларец. 




\noindent\textbf{Владычный} "--- господский или Господний. 




\noindent\textbf{Владящий} "--- обладающий; господствующий. 




\noindent\textbf{Власти} "--- имя одного из чинов ангельских. 




\noindent\textbf{Власяница} "--- одежда из жесткого, колючего волоса. 




\noindent\textbf{Влаятися} "--- мыкаться; колебаться; волноваться; носиться по волнам. 




\noindent\textbf{Влещи} "--- тащить. 




\noindent\textbf{Влещися} "--- брести; медленно идти; тащиться. 




\noindent\textbf{Вмале} "--- вскоре; немного спустя; почти; едва. 




\noindent\textbf{Вне(уду)} "--- извне; снаружи. 




\noindent\textbf{Внегда} "--- когда. 




\noindent\textbf{Внезапу} "--- вдруг; неожиданно. 




\noindent\textbf{Внемшийся} "--- загоревшийся. 




\noindent\textbf{Внитие} "--- вхождение; явление; пришествие. 




\noindent\textbf{Внове} "--- недавно. 




\noindent\textbf{Внутрь (уду)} "--- внутри. 




\noindent\textbf{Вняти} "--- обратить внимание; услышать. 




\noindent\textbf{Вняти от} "--- остерегаться. 




\noindent\textbf{Внятися} "--- загореться. 




\noindent\textbf{Воврещи} "--- бросить во что-либо; ввергнуть; внести. 




\noindent\textbf{Водрузити} "--- утвердить; укрепить. 




\noindent\textbf{Во еже} "--- чтобы; ради; для. 




\noindent\textbf{Вожделети} "--- сильно желать. 




\noindent\textbf{Возбеситися} "--- сделаться неистовым. 




\noindent\textbf{Возбранный, взбранный} "--- военный; храбрый в бранях; победоносный. 




\noindent\textbf{Возбраняти} "--- препятствовать; удерживать. 




\noindent\textbf{Возбряцати} "--- воспеть; хвалить в песнях. 




\noindent\textbf{Возвлачити} "--- затащить наверх. 




\noindent\textbf{Возврещи, возвергати} "--- возложить, возлагать. 




\noindent\textbf{Возглавие} "--- подушка; изголовье. 




\noindent\textbf{Возглас} "--- окончательные слова молитвы, тайно творимой священником. 




\noindent\textbf{Возглашение} "--- громкое пение или чтение; Возглас. 




\noindent\textbf{Возглядати} "--- взирать; смотреть. 




\noindent\textbf{Возгнещати} "--- разводить огонь. 




\noindent\textbf{Воздвигнути} "--- поднять; возвысить. 




\noindent\textbf{Воздвижение} "--- поднятие, возвышение. 




\noindent\textbf{Воздвизатися} "--- иногда: отправляться в путь. 




\noindent\textbf{Воздвизати} "--- поднимать; возвышать. 




\noindent\textbf{Воздеяние} "--- поднятие, возвышение. 




\noindent\textbf{Воздух} "--- покровец, полагаемый сверху священных сосудов на Литургии. 




\noindent\textbf{Возлежати} "--- лежать облокотившись; полулежать. 




\noindent\textbf{Возмущение} "--- смятение; бунт. 




\noindent\textbf{Возмятати} "--- возмущать; производить раздор. 




\noindent\textbf{Возниспослати} "--- послать свыше; наградить. 




\noindent\textbf{Возничати} "--- поднять голову. 




\noindent\textbf{Возобразитися} "--- принять образ; олицетвориться; вселиться в видимый образ; вновь быть изображенным. 




\noindent\textbf{Возраст} "--- возраст (число лет); рост. 




\noindent\textbf{Возрастити} "--- вырастить; увеличить. 




\noindent\textbf{Возреяти} "--- поколебать; потрясти. 




\noindent\textbf{Возставити} "--- восстановить; поставить на прежнее место. 




\noindent\textbf{Волити} "--- хотеть; желать; требовать. 




\noindent\textbf{Волна} "--- шерсть; руно; овчина. 




\noindent\textbf{Волхв} "--- мудрец; звездочет; чародей; предсказатель. 




\noindent\textbf{Волчец} "--- колючая трава. 




\noindent\textbf{Вонь} "--- в него. 




\noindent\textbf{Воня} "--- запах; курение. 




\noindent\textbf{Воня злая} "--- смрад. 




\noindent\textbf{Вопити, вопияти} "--- громко кричать; взывать. 




\noindent\textbf{Ворожити} "--- колдовать; предсказывать будущее. 




\noindent\textbf{Ворожея} "--- волшебник; колдун; отравитель. 




\noindent\textbf{Воскликновение} "--- хоровое пение. 




\noindent\textbf{Восклонятися} "--- выпрямляться; подниматься; разгибаться. 




\noindent\textbf{Воскресати} "--- восставать; оживать; возвращаться к жизни. 




\noindent\textbf{Воскресение} "--- восстание из мертвых. 




\noindent\textbf{Воскрешати} "--- оживлять. 




\noindent\textbf{Воскрилие} "--- подол; край одежды; пола верхней одежды. 




\noindent\textbf{Восперяти} "--- оперять; окрылять (надеждой). 




\noindent\textbf{Восписовати} "--- изображать письменно; изъявлять. 




\noindent\textbf{Воспросити} "--- попросить. 




\noindent\textbf{Воспрянути} "--- вскочить; вспрыгнуть; приходить в себя. 




\noindent\textbf{Востерзати} "--- извлекать; выдергивать. 




\noindent\textbf{Восторгати} "--- рвать; щипать; полоть. 




\noindent\textbf{Востягнути} "--- подтянуть; укрепить; подтащить. 




\noindent\textbf{Востязати, востязовати} "--- исследовать; испытывать; интересоваться. 




\noindent\textbf{Восхитити} "--- изловить; поймать; не законно захватить; похитить; увлечь в высоту; привести в восторг. 




\noindent\textbf{Вотще} "--- понапрасну; впустую; даром; тщетно. 




\noindent\textbf{Воутрие, воутрий} "--- на другой день. 




\noindent\textbf{Впасти} "--- упасть; попасть; ввалиться; подвергнуться; подпасть. 




\noindent\textbf{Вперити} "--- возвысить; поднять; устремить вверх наподобие пера. 




\noindent\textbf{Вперитися} "--- воспарить; взлететь. 




\noindent\textbf{Вперсити} "--- внутрь себя принять. 




\noindent\textbf{Впрямо} "--- прямо; напротив. 




\noindent\textbf{Врабий} "--- воробей. 




\noindent\textbf{Вран} "--- ворон. 




\noindent\textbf{Врата красныя} "--- западные церковные двери. 




\noindent\textbf{Вратарь, вратник} "--- сторож у ворот. 




\noindent\textbf{Врачба} "--- лекарство; врачевание. 




\noindent\textbf{Врачебница} "--- больница. 




\noindent\textbf{Вреды} "--- кожное заболевание. 




\noindent\textbf{Вресноту} "--- вправду; по достоинству; пристойно. 




\noindent\textbf{Вретище} "--- плохая, грубая одежда; дерюга; скорбное одеяние. 




\noindent\textbf{Вреяти} "--- кипеть; пениться; разгорячаться; бить ключом; выкипать. 




\noindent\textbf{Вреяти} "--- ввергать; вметать; вталкивать. 




\noindent\textbf{Всеблаголепный} "--- великолепный. 




\noindent\textbf{Всевидно} "--- всенародно. 




\noindent\textbf{Вседетельный} "--- все создавший. 




\noindent\textbf{Всеконечне} "--- совершенно. 




\noindent\textbf{Всекрасный} "--- самый красивый. 




\noindent\textbf{Вселукавый} "--- самый коварный, т.~е. диавол. 




\noindent\textbf{Всеоружие} "--- полное вооружение. 




\noindent\textbf{Всепетый} "--- препрославленный; всеми или всюду восхваляемый. 




\noindent\textbf{Всесожжение} "--- жертвоприношение, при котором жертва сжигалась целиком. 




\noindent\textbf{Всечреждение} "--- богатое угощение. 




\noindent\textbf{Всеядец} "--- тот, кто всех поедает, т.~е. ад, или смерть. 




\noindent\textbf{Всеяти} "--- посеять. 




\noindent\textbf{Вскрай} "--- по краю; близ; около. 




\noindent\textbf{Вскую} "--- почему? из-за чего? за что? 




\noindent\textbf{Вспять} "--- назад. 




\noindent\textbf{Всуе} "--- напрасно. 




\noindent\textbf{Всяко} "--- совершенно; совсем; вовсе. 




\noindent\textbf{Вторицею} "--- вторично; усиленно. 




\noindent\textbf{Вчиняти} "--- учреждать; узаконять. 




\noindent\textbf{Выну} "--- всегда. 




\noindent\textbf{Выспренний} "--- высокий; гордый. 




\noindent\textbf{Выспрь} "--- вверх. 




\noindent\textbf{Высоковыйный} "--- гордый; надменный; кичливый. 




\noindent\textbf{Высокомудрствовати} "--- высокоумствовать; гордиться. 




\noindent\textbf{Вышелетный} "--- предвечный. 




\noindent\textbf{Выя} "--- шея. 




\noindent\textbf{Вящший} "--- больший. 




\noindent\textbf{Вящше} "--- больше. 




\bukvaending

\bukva{Г} 





\noindent\textbf{Гаггрена} (читается «гангрена») "--- гангрена; антонов огонь; рак. 




\noindent\textbf{Гадание} "--- загадка; неясность. 




\noindent\textbf{Гади} (мн. ч.) "--- пресмыкающиеся. 




\noindent\textbf{Гаждение} "--- поношение; бесчестие; ругань. 




\noindent\textbf{Газофилакия} "--- казнохранилище во храме Иерусалимском. 




\noindent\textbf{Ганание} "--- загадка; притча. 




\noindent\textbf{Гастримаргия} "--- чревобесие; обжорство. 




\noindent\textbf{Гащи} "--- штаны; нижнее мужское белье. 




\noindent\textbf{Геенна} "--- долина Гинном около Иерусалима, где идолопоклонствующие иудеи при царе Ахазе сжигали своих детей в честь идола Молоха. Иносказательно: место будущих мучений, загробных наказаний. 




\noindent\textbf{Генварь} "--- январь. 




\noindent\textbf{Гибель} "--- трата; расход. 




\noindent\textbf{Главизна} "--- глава; начало; причина. 




\noindent\textbf{Главотяж} "--- головная повязка у иудеев. 




\noindent\textbf{Глагол} "--- слово; речь. 




\noindent\textbf{Глаголати} "--- говорить; рассказывать. 




\noindent\textbf{Глаголемый} "--- называемый; так называемый. 




\noindent\textbf{Глас} "--- голос; напев. 




\noindent\textbf{Глезна} "--- голень; ступня. 




\noindent\textbf{Глоба} "--- казнь; наказание. 




\noindent\textbf{Глумец} "--- кощун; пересмешник. 




\noindent\textbf{Глумилище} "--- место для скачек, плясок, маскарадов и т. п. 




\noindent\textbf{Глумитися} "--- забавляться; тешиться; получать удовольствие. 




\noindent\textbf{Глумы} (мн. ч.) "--- шутки; смех; игры. 




\noindent\textbf{Глядати} "--- смотреть; глядеть. 




\noindent\textbf{Гнати} "--- гнать; преследовать; идти; следовать за кем или чем-либо. 




\noindent\textbf{Гной, гноище} "--- навозная куча; раны. 




\noindent\textbf{Гнушатися} "--- считать гнусным; презирать. 




\noindent\textbf{Гобзование} "--- изобилие, довольство. 




\noindent\textbf{Гобзовати} "--- изобиловать; избыточествовать; быть богатым. 




\noindent\textbf{Гобзующий} "--- живущий в довольстве. 




\noindent\textbf{Говение} "--- почитание (например, поста). 




\noindent\textbf{Говети} "--- чтить; почитать (например, пост). 




\noindent\textbf{Говядо} "--- рогатый скот. 




\noindent\textbf{Годе} "--- угодно; приятно; подходяще. 




\noindent\textbf{Година, год} "--- час; время; пора. 




\noindent\textbf{Голоть} "--- гололедица, лед. 




\noindent\textbf{Гомола} "--- ком; комок; катыш; кусок. 




\noindent\textbf{Гонзати} "--- убегать; спасаться бегством. 




\noindent\textbf{Гонзнутие} "--- избежание. 




\noindent\textbf{Гонитель} "--- преследователь. 




\noindent\textbf{Горе} "--- ввысь; вверх. 




\noindent\textbf{Горее} "--- хуже; бедственнее. 




\noindent\textbf{Горлица, горличишь} "--- дикий голубь. 




\noindent\textbf{Горнец} "--- горшок; котелок; жаровня. 




\noindent\textbf{Горнило} "--- кузнечный горн; плавильня; место для плавки или очищения огнем. 




\noindent\textbf{Горница} "--- верхняя комната; столовая. 




\noindent\textbf{Горний} "--- высокий; вышний; небесный. 




\noindent\textbf{Горохищный} "--- пасущийся; блуждающий по горам; украденный диким зверем с горного пастбища. 




\noindent\textbf{Гортанобесие} "--- пристрастие к лакомствам. 




\noindent\textbf{Горушный} "--- горчичный. 




\noindent\textbf{Горший} "--- злейший; худший. 




\noindent\textbf{Господствия} "--- один из чинов ангельских. 




\noindent\textbf{Господыня} "--- госпожа. 




\noindent\textbf{Гостинник} "--- корчмарь; содержатель постоялого двора. 




\noindent\textbf{Градарь} "--- садовник; огородник. 




\noindent\textbf{Градеж} "--- оплот; забор. 




\noindent\textbf{Грезн} "--- гроздь виноградная. 




\noindent\textbf{Гривна} "--- ожерелье; носимая на шее цепь. 




\noindent\textbf{Гроздие} "--- кисть плодов; ветвь (винограда). 




\noindent\textbf{Грясти} "--- идти; шествовать. 




\noindent\textbf{Гугнивый} "--- гнусавый; заика; косноязычный; картавый; говорящий в нос. 




\noindent\textbf{Гудение} "--- игра на гуслях или арфе. 




\noindent\textbf{Гудец} "--- гуслист; музыкант. 




\bukvaending

\bukva{Д} 





\noindent\textbf{Да} "--- пусть; чтобы. 




\noindent\textbf{Дабы} "--- чтобы. 




\noindent\textbf{Далечайше} "--- гораздо далее. 




\noindent\textbf{Далечен} "--- далекий; трудный. 




\noindent\textbf{Даннословие} "--- обещание; обязательство. 




\noindent\textbf{Двакраты} "--- дважды. 




\noindent\textbf{Дващи} "--- дважды. 




\noindent\textbf{Дверь адова} "--- смерть. 




\noindent\textbf{Двизати} "--- двигать; шевелить. 




\noindent\textbf{Двоедушный} "--- нетвердый в вере. 




\noindent\textbf{Двоица} "--- пара. 




\noindent\textbf{Дворище} "--- небольшой или запустелый дом. 




\noindent\textbf{Дебельство} "--- тучность; полнота; дородность. 




\noindent\textbf{Дебрь} "--- долина; ложбина; овраг; ущелье. 




\noindent\textbf{Девствовати} "--- хранить девство, целомудрие. 




\noindent\textbf{Действо} "--- действие; представление. 




\noindent\textbf{Декемврий} "--- декабрь. 




\noindent\textbf{Делатилище} "--- купеческая лавка; орудие в чьих-либо руках. 




\noindent\textbf{Делва} "--- бочка; кадка. 




\noindent\textbf{Делма} "--- для. 




\noindent\textbf{Деля} "--- для; ради. 




\noindent\textbf{Демественник} "--- певчий. 




\noindent\textbf{Демоноговение} "--- почитание бесов. 




\noindent\textbf{Демоночтец} "--- идолопоклонник. 




\noindent\textbf{Денница} "--- утренняя заря, утренняя звезда; отпадший ангел. 




\noindent\textbf{Денносветлый} "--- подобный дневному свету. 




\noindent\textbf{Держава} "--- сила; крепость; власть; государство. 




\noindent\textbf{Державно} "--- властно; могущественно. 




\noindent\textbf{Дерзать} "--- осмеливаться; полагаться. 




\noindent\textbf{Дерзновение} "--- смелость. 




\noindent\textbf{Дерзословие} "--- наглая речь. 




\noindent\textbf{Дерзостник} "--- наглец; нахал. 




\noindent\textbf{Дерзый} "--- смелый; бесстыдный; дерзкий. 




\noindent\textbf{Десница} "--- правая рука. 




\noindent\textbf{Десный} "--- правый; находящийся с правой стороны. 




\noindent\textbf{Десятина} "--- десятая часть. 




\noindent\textbf{Десятословие} "--- десять заповедей Божиих, данных через Моисея. 




\noindent\textbf{Детищ} "--- младенец; дитя; отроча. 




\noindent\textbf{Детосаждение} "--- зачатие во утробе младенца. 




\noindent\textbf{Диадима} "--- венец; диадема. 




\noindent\textbf{Дивий} "--- дикий; лесной. 




\noindent\textbf{Дивьячитися} "--- зверствовать. 




\noindent\textbf{Дидрахма} "--- греч.двойная драхма", древнегреч. серебряная монета. 




\noindent\textbf{Динарий} "--- монета. 




\noindent\textbf{Длань} "--- ладонь. 




\noindent\textbf{Дмение} "--- гордость. 




\noindent\textbf{Дмитися} "--- гордиться; кичиться. 




\noindent\textbf{Дне} "--- относящийся к числу песнопений из Октоиха, а в дни пения Триоди "--- из этой книги. 




\noindent\textbf{Дненощно} "--- в течение целых суток. 




\noindent\textbf{Днесь} "--- сегодня, ныне; теперь. 




\noindent\textbf{Днешний} "--- нынешний; сегодняшний. 




\noindent\textbf{Доблий, добльственный, доблестный} "--- крепкий в добре; твердый в добродетели. 




\noindent\textbf{Доброзрачие} "--- красота; благообразие. 




\noindent\textbf{Доброкласный} "--- приносящий обильную жатву. 




\noindent\textbf{Добропобедный} "--- прославленный победами. 




\noindent\textbf{Доброта} "--- красота. 




\noindent\textbf{Доброхвальный} "--- заслуживающий похвалы; похвальный. 




\noindent\textbf{Довлесотворити} "--- удовлетворить. 




\noindent\textbf{Довлети} "--- доставать; быть достаточным; хватать. 




\noindent\textbf{Доволний} "--- достаточный. 




\noindent\textbf{Догмат} "--- греч. одно из основных положений веры. 




\noindent\textbf{Дождити} "--- посылать дождь; кропить; орошать. 




\noindent\textbf{Дозде} "--- доселе; до сего дня; досюда. 




\noindent\textbf{Дозела} "--- чрезвычайно. 




\noindent\textbf{Доилица} "--- кормилица; мамка. 




\noindent\textbf{Доити} "--- кормить грудью. 




\noindent\textbf{Доколе} "--- до какого времени? долго ли? 




\noindent\textbf{Долний} "--- нижний; земной (как противоп. «небесный, горний»). 




\noindent\textbf{Долу, доле} "--- внизу; вниз. 




\noindent\textbf{Долувлекущий} "--- тянущий вниз. 




\noindent\textbf{Дондеже} "--- пока. 




\noindent\textbf{Донележе} "--- пока. 




\noindent\textbf{Дориносити} "--- сопровождать кого-либо в качестве стражи, свиты. 




\noindent\textbf{Досаждение} "--- делание неугодного; нечестие; оскорбление. 




\noindent\textbf{Достижно} "--- понятно. 




\noindent\textbf{Достояние} "--- имение; наследство; власть. 




\noindent\textbf{Драхма} "--- древнегреч. серебряная монета. 




\noindent\textbf{Драчие} "--- сорная трава. 




\noindent\textbf{Древле} "--- давно. 




\noindent\textbf{Древодель} "--- плотник; столяр. 




\noindent\textbf{Дреколие} "--- колья. 




\noindent\textbf{Дрождие} "--- дрожжи; отстой. 




\noindent\textbf{Другиня} "--- подруга. 




\noindent\textbf{Дружина} "--- общество (товарищей, сверстников). 




\noindent\textbf{Дручити} "--- удручать; томить; изнурять. 




\noindent\textbf{Дряселовати} "--- быть пасмурным, мрачным, печальным. 




\noindent\textbf{Дряхлование} "--- печаль. 




\noindent\textbf{Дряхлый} "--- печальный. 




\noindent\textbf{Дска, дщица} "--- доска; дощечка. 




\noindent\textbf{Дуга} "--- радуга. 




\noindent\textbf{Дхнути} "--- дохнуть; дунуть. 




\noindent\textbf{Дщи, дщерь} "--- дочь. 




\bukvaending

\bukva{Е}





\noindent\textbf{Евнух} "--- скопец; сторож при гареме; придворный. 




\noindent\textbf{Егда} "--- когда. 




\noindent\textbf{Егов} "--- его (притяжательный падеж от местоимения «он»). 




\noindent\textbf{Еда} "--- разве? неужели? 




\noindent\textbf{Едем} "--- Эдем; рай земной. 




\noindent\textbf{Единако} "--- согласно; одинаково. 




\noindent\textbf{Единаче} "--- одинаково; равно; еще. 




\noindent\textbf{Единаче ли} "--- неужели еще? 




\noindent\textbf{Единовидный} "--- одновидный; однообразный. 




\noindent\textbf{Единою} "--- однажды. 




\noindent\textbf{Еже} "--- что; кое. 




\noindent\textbf{Езеро} "--- озеро. 




\noindent\textbf{Ей} "--- да; истинно; верно. 




\noindent\textbf{Ексапсалмы} "--- шестопсалмие. 




\noindent\textbf{Ектения} "--- усиленное моление; прошение. 




\noindent\textbf{Елей} "--- оливковое, деревянное масло. 




\noindent\textbf{Елень} "--- олень; лань. 




\noindent\textbf{Елеонский} "--- оливковый. 




\noindent\textbf{Елижды аще} "--- когда бы ни. 




\noindent\textbf{Елижды, еликожды} "--- всегда как; всякий раз, когда. 




\noindent\textbf{Еликий} "--- кто; который. 




\noindent\textbf{Елико} "--- сколько. 




\noindent\textbf{Елико-елико} "--- через короткое время; очень скоро. 




\noindent\textbf{Еликомощно} "--- по возможности; сколько дозволяют силы. 




\noindent\textbf{Еллин} "--- грек; язычник; прозелит иудаизма. 




\noindent\textbf{Елма} "--- поскольку; насколько. 




\noindent\textbf{Епендит} "--- верхнее платье. 




\noindent\textbf{Епистолия} "--- письмо; послание. 




\noindent\textbf{Еродий} "--- цапля. 




\noindent\textbf{Есмирнисменный} "--- смешанный вместе со смирной. 




\noindent\textbf{Ехидна} "--- ядовитая змея. 




\bukvaending

\bukva{Ж} 





\noindent\textbf{Жаждати} "--- хотеть пить; сильно желать чего-либо. 




\noindent\textbf{Жалость} "--- ревность; рвение. 




\noindent\textbf{Жатель} "--- жнец. 




\noindent\textbf{Жегомый} "--- тот, кого жгут огнем; больной огнем; больной огневицей, горячкой. 




\noindent\textbf{Жезл} "--- посох; трость; палка. 




\noindent\textbf{Женитва} "--- бракосочетание; супружество; брак. 




\noindent\textbf{Женонеистовый} "--- похотливый; блудный; сластолюбивый. 




\noindent\textbf{Жестоковыйный} "--- бесчувственный; упрямый. 




\noindent\textbf{Живити} "--- животворить; давать жизнь; оживлять. 




\noindent\textbf{Живодавец} "--- податель жизни. 




\noindent\textbf{Живоначалие} "--- начало; причина жизни. 




\noindent\textbf{Живот} "--- жизнь. 




\noindent\textbf{Животный} "--- живущий; одушевленный. 




\noindent\textbf{Жребя} "--- жеребенок. 




\noindent\textbf{Жрети} "--- заколать; приносить жертвоприношение. 




\noindent\textbf{Жупел} "--- горячая сера. 




\bukvaending

\bukva{З} 





\noindent\textbf{Забавати} "--- заговаривать; заколдовывать. 




\noindent\textbf{Забавление} "--- промедление; мешкание; ожидание. 




\noindent\textbf{Забавляти} "--- удерживать; замедлять. 




\noindent\textbf{Забобоны} "--- самовольная служба, бесчиние. 




\noindent\textbf{Забрало} "--- стена; забор. 




\noindent\textbf{Завет} "--- союз; договор; условие. 




\noindent\textbf{Завида} "--- зависть. 




\noindent\textbf{Завистно} "--- мало; недостаточно. 




\noindent\textbf{За еже} "--- для того, чтобы. 




\noindent\textbf{Заздати} "--- загородить. 




\noindent\textbf{Зазрети} "--- заглянуть; заметить; осудить; упрекнуть. 




\noindent\textbf{Заимование} "--- заем; долг. 




\noindent\textbf{Заимовати} "--- занимать; заимствовать. 




\noindent\textbf{Заклание} "--- жертвоприношение. 




\noindent\textbf{Заклеп} "--- запор; замок; задвижка. 




\noindent\textbf{Заколение} "--- жертвоприношение. 




\noindent\textbf{Законописец} "--- составитель законов. 




\noindent\textbf{Законополагати} "--- давать закон. 




\noindent\textbf{Закров} "--- место для укрытия. 




\noindent\textbf{Залещи} "--- быть в засаде; скрываться. 




\noindent\textbf{Заматорети} "--- устареть; зачерстветь; состариться. 




\noindent\textbf{Замреженый} "--- пойманный в сети. 




\noindent\textbf{Зане} "--- так как; потому что. 




\noindent\textbf{Занеже} "--- поскольку. 




\noindent\textbf{Зань} "--- за него. 




\noindent\textbf{Запаление} "--- загорание; пожар. 




\noindent\textbf{Запев} "--- краткий стих, предваряющий стихиры (на «Господи, воззвах», хвалитны, стиховны) или тропари канона. 




\noindent\textbf{Запечатствовати} "--- запечатать; утвердить; связать; скрепить. 




\noindent\textbf{Запинание} "--- враждебное действие. 




\noindent\textbf{Запойство} "--- пьянство. 




\noindent\textbf{Запона} "--- завеса. 




\noindent\textbf{Запрение} "--- отрицание; запирание. 




\noindent\textbf{Запретити} "--- запретить; опечалиться; скорбеть. 




\noindent\textbf{Запустение} "--- опустение; пустыня. 




\noindent\textbf{Запустети} "--- придти в запущение или запустение, запустеть. 




\noindent\textbf{Запяти, запнути} "--- остановить; задержать; обольститься. 




\noindent\textbf{Запятие} "--- препинание; препятствие; преткновение. 




\noindent\textbf{Заревидный} "--- подобный заре. 




\noindent\textbf{Зарелучный} "--- лучезарный. 




\noindent\textbf{Застояти} "--- останавливать на дороге; удерживать; наскучивать; утруждать. 




\noindent\textbf{За ся} "--- за себя. 




\noindent\textbf{Затвор} "--- замок; запор; место молитвенного подвига некоторых иноков, давших обет не исходить из своей келлии. 




\noindent\textbf{Заточаемый} "--- обуреваемый ветром; носимый; гонимый. 




\noindent\textbf{Затулити} "--- закрыть; спрятать; укрыть. 




\noindent\textbf{Затуне} "--- даром; без причины. 




\noindent\textbf{Зауститися} "--- закрыть уста; замолчать. 




\noindent\textbf{Заутра} "--- до восхода солнца; поутру; рано; завтра. 




\noindent\textbf{Заутрие} "--- завтрашний день. 




\noindent\textbf{Заушение} "--- пощечина; удар рукой по лицу. 




\noindent\textbf{Заушати} "--- заграждать уста; запрещать говорить. 




\noindent\textbf{Захленутися} "--- погрузиться. 




\noindent\textbf{Заходный} "--- западный. 




\noindent\textbf{Зачало} "--- начало; название отрезков текста в книгах Священного Писания Нового Завета. 




\noindent\textbf{Заяти} "--- взять взаймы; занять. 




\noindent\textbf{Звездоблюститель} "--- астроном. 




\noindent\textbf{Звездоволхвовати} "--- гадать по звездам; заниматься астрологией. 




\noindent\textbf{Звездозаконие} "--- астрономия. 




\noindent\textbf{Звездослов} "--- астролог. 




\noindent\textbf{Звездословие} "--- астрология. 




\noindent\textbf{Звездословити} "--- заниматься астрологией. 




\noindent\textbf{Звероядина} "--- скот, поврежденный хищным зверем. 




\noindent\textbf{Звиздание} "--- свист; посвист. 




\noindent\textbf{Звиздати} "--- свистеть. 




\noindent\textbf{Звонец} "--- колокольчик. 




\noindent\textbf{Звонница} "--- колокольня. 




\noindent\textbf{Звяцати} "--- звенеть; бренчать. 




\noindent\textbf{Здати} "--- строить. 




\noindent\textbf{Зде} "--- здесь. 




\noindent\textbf{Здо} "--- здание; стена; крыша. 




\noindent\textbf{Зелейник} "--- знахарь, лечащий травами и заговором. 




\noindent\textbf{Зелейничество} "--- напоение отравой. 




\noindent\textbf{Зелейный} "--- состоящий из зелия, т.~е. травы или других растений. 




\noindent\textbf{Зелие} "--- трава; растение. 




\noindent\textbf{Зело, зельне} "--- весьма; очень сильно. 




\noindent\textbf{Зельный} "--- сильный; великий. 




\noindent\textbf{Земен} "--- земной. 




\noindent\textbf{Земстий} "--- земной. 




\noindent\textbf{Зеница} "--- зрачок в глазе. 




\noindent\textbf{Зепь} "--- карман; мешок. 




\noindent\textbf{Зерцало} "--- зеркало. 




\noindent\textbf{Зиждитель} "--- создатель; творец. 




\noindent\textbf{Зиждити} "--- строить. 




\noindent\textbf{Зима} "--- зима; холод; плохая погода. 




\noindent\textbf{Злак} "--- растение; зелень; овощ. 




\noindent\textbf{Златарь} "--- золотых дел мастер. 




\noindent\textbf{Златица, златница} "--- золотая монета. 




\noindent\textbf{Златозарный} "--- яркоблестящий. 




\noindent\textbf{Злато} "--- золото. 




\noindent\textbf{Златокованный} "--- отчеканенный из золота. 




\noindent\textbf{Златокровный} "--- имеющий позлащенную крышу. 




\noindent\textbf{Злачный} "--- травный; богатый растительностью, злаками. 




\noindent\textbf{Зле} "--- зло; жестоко; худо. 




\noindent\textbf{Злоба} "--- забота. 




\noindent\textbf{Злокозненный} "--- исполненный злобы; лукавства. 




\noindent\textbf{Злокоман} "--- злодей; зложелатель; враг. 




\noindent\textbf{Злонравие} "--- развратный или дурной нрав. 




\noindent\textbf{Злообстояние} "--- беда ;несчастье. 




\noindent\textbf{Злопомнение} "--- злопамятство. 




\noindent\textbf{Злоречети} "--- бранить; ругать; злословить; поносить. 




\noindent\textbf{Злосердный} "--- безжалостный. 




\noindent\textbf{Злосмрадие} "--- зловоние. 




\noindent\textbf{Злосоветие} "--- злой умысел. 




\noindent\textbf{Злострастие} "--- сильные и порочные страсти. 




\noindent\textbf{Злостужати} "--- сильно досаждать. 




\noindent\textbf{Злостужный} "--- причиняющий большое беспокойство, мучение. 




\noindent\textbf{Злотечение} "--- развратные или злые поступки. 




\noindent\textbf{Злоумерший} "--- претерпевший тяжелую смерть. 




\noindent\textbf{Злоухищряти} "--- замышлять зло. 




\noindent\textbf{Злохитренный} "--- коварный. 




\noindent\textbf{Злохудожный} "--- лукавый; злобный; беззаконный. 




\noindent\textbf{Злый} "--- злой; плохой; негодный; худой; жестокий. 




\noindent\textbf{Знаемый} "--- знакомый, близкий человек. 




\noindent\textbf{Знаменательне} "--- прообразовательно. 




\noindent\textbf{Знаменательный} "--- прообразовательный; обозначающий нечто. 




\noindent\textbf{Знаменати} "--- обозначать знаком; помечать; изображать; показывать; являть. 




\noindent\textbf{Знамение} "--- знак; признак; явление; чудо. 




\noindent\textbf{Знаменоносец} "--- чудотворец. 




\noindent\textbf{Знаменоносный} "--- чудотворный. 




\noindent\textbf{Зобати} "--- наполнять зоб; клевать; есть; поглощать. 




\noindent\textbf{Зрак} "--- лицо; вид; образ. 




\noindent\textbf{Зрети} "--- смотреть. 




\noindent\textbf{Зрети к смерти} "--- находиться при последнем издыхании. 




\noindent\textbf{Зыбати} "--- шевелить; двигать; качать. 




\bukvaending

\bukva{И}





\noindent\textbf{И} "--- его. 




\noindent\textbf{Игемон} "--- вождь; начальник; правитель. 




\noindent\textbf{Иго} "--- ярмо; ноша. 




\noindent\textbf{Игралище} "--- место для представления. 




\noindent\textbf{Игрище} "--- смешное или непристойное представление. 




\noindent\textbf{Идеже} "--- где; когда. 




\noindent\textbf{Идолобесие} "--- неистовое идолопоклонство. 




\noindent\textbf{Иерей} "--- священник. 




\noindent\textbf{Иждивати} "--- проживать; тратить; издерживать. 




\noindent\textbf{Иже} "--- который. 




\noindent\textbf{Изблистати} "--- осиять; облистать; излить свет. 




\noindent\textbf{Избодати} "--- пропороть; поразить; пронзить; проколоть; выколоть. 




\noindent\textbf{Изборение} "--- поражение. 




\noindent\textbf{Изборати} "--- побеждать; поражать. 




\noindent\textbf{Избременяти} "--- облегчать; освобождать от бремени; выгружать. 




\noindent\textbf{Избутелый} "--- согнивший; испортившийся. 




\noindent\textbf{Избыти} "--- остаться в избытке, излишестве; изобиловать; освободиться. 




\noindent\textbf{Избыток} "--- довольство; изобилие. 




\noindent\textbf{Изваяние} "--- идол; кумир. 




\noindent\textbf{Извержение} "--- исключение из церковного клира или лишение сана. 




\noindent\textbf{Извесити} "--- свесить; вывесить. 




\noindent\textbf{Извествовати} "--- объявлять; оглашать; удостоверять. 




\noindent\textbf{Известно} "--- точно; тщательно. 




\noindent\textbf{Извет} "--- донос; извещение. 




\noindent\textbf{Извещен} "--- уверен. 




\noindent\textbf{Извещение} "--- удостоверение. 




\noindent\textbf{Извитие словес} "--- красноречие; витийство. 




\noindent\textbf{Извитийствовати} "--- красноречиво рассказать. 




\noindent\textbf{Извлачитися} "--- раздеться; разоблачиться. 




\noindent\textbf{Изволение} "--- воля; желание. 




\noindent\textbf{Изволити} "--- дозволять; захотеть; пожелать. 




\noindent\textbf{Извращати} "--- выворачивать; изменять; превращать. 




\noindent\textbf{Изврещи} "--- выбросить; вымести. 




\noindent\textbf{Извыкати} "--- научиться, познавать. 




\noindent\textbf{Изгвоздити} "--- выдернуть, вынуть гвозди. 




\noindent\textbf{Изгибающий} "--- погибающий, пропадающий. 




\noindent\textbf{Изгибнути} "--- погибнуть; пропасть; потеряться. 




\noindent\textbf{Изглаждати} "--- исключать; уничтожать. 




\noindent\textbf{Издетска} "--- с детства. 




\noindent\textbf{Издревле} "--- издавна; исстари. 




\noindent\textbf{Издручитися} "--- изнурить себя. 




\noindent\textbf{Изженяти} "--- изгонять; выгонять. 




\noindent\textbf{Излазити} "--- выходить; сходить (например, с корабля). 




\noindent\textbf{Излиха} "--- чрезмерно; еще более. 




\noindent\textbf{Излишше} "--- до излишества; паче меры. 




\noindent\textbf{Изляцати} "--- протягивать; простирать. 




\noindent\textbf{Измерети} "--- умереть. 




\noindent\textbf{Изметати} "--- извергать; выкидывать; выбрасывать. 




\noindent\textbf{Измена} "--- замена; перемена; выкуп. 




\noindent\textbf{Изменяти} "--- заменять; переменять. 




\noindent\textbf{Изменяти лице} "--- притворяться. 




\noindent\textbf{Измлада} "--- смолоду. 




\noindent\textbf{Измовение} "--- омытие; очищение. 




\noindent\textbf{Измолкати} "--- перестать говорить; замолкать. 




\noindent\textbf{Изнесение, изношение} "--- вынос. 




\noindent\textbf{Изницати} "--- возникать; появляться. 




\noindent\textbf{Износити} "--- выносить; произносить; производить; произращать; приносить. 




\noindent\textbf{Изнуждати} "--- выводить из нужды. 




\noindent\textbf{Изобнажати} "--- обнаруживать; являть; открывать. 




\noindent\textbf{Изостати} "--- остаться где-либо. 




\noindent\textbf{Изощряти} "--- обострить; наточить. 




\noindent\textbf{Изращение} "--- вырощение; произведение; порождение. 




\noindent\textbf{Изриновение} "--- выбрасывание; извержение; исключение. 




\noindent\textbf{Изриновенный} "--- изверженный; выкинутый; прогнанный. 




\noindent\textbf{Изринути} "--- столкнуть; опрокинуть; повалить; погубить. 




\noindent\textbf{Изрок} "--- изречение; осуждение. 




\noindent\textbf{Изрыти} "--- вырыть; выкопать. 




\noindent\textbf{Изрядно} "--- особенно; преимущественно. 




\noindent\textbf{Изсунути} "--- вынуть; исторгнуть; вырвать; изъять. 




\noindent\textbf{Изступление} "--- изумление; восторг. 




\noindent\textbf{Изуведети} "--- уразуметь; познать. 




\noindent\textbf{Изуздитися} "--- освободиться; получить волю. 




\noindent\textbf{Изумевати} "--- недоумевать; не понимать. 




\noindent\textbf{Изумителен} "--- буйствующий; беснующийся. 




\noindent\textbf{Изумитися} "--- сойти с ума; обезуметь. 




\noindent\textbf{Изути} "--- разуть; снять обувь. 




\noindent\textbf{Изчленити} "--- лишить членов; сокрушить члены; изуродовать. 




\noindent\textbf{Изъядати} "--- проматывать; растрачивать. 




\noindent\textbf{Иконом} "--- домоправитель. 




\noindent\textbf{Иконоратный} "--- иконоборственный. 




\noindent\textbf{Икос} "--- пространная песнь, написанная в похвалу святого или праздника. 




\noindent\textbf{Имати} "--- брать. 




\noindent\textbf{Иматисма} "--- верхнее платье, плащ. 




\noindent\textbf{Именный} "--- сокровищный; касающийся имения. 




\noindent\textbf{Имуществительно} "--- преимущественно. 




\noindent\textbf{Ин} "--- иной; другой. 




\noindent\textbf{Инамо} "--- в ином месте. 




\noindent\textbf{Иноковати} "--- жить по-иночески. 




\noindent\textbf{Инуде, инде} "--- в ином месте, в иное место. 




\noindent\textbf{Ипакои} "--- песнопение, положенное по малой ектении после полиелея на воскресной утрени. 




\noindent\textbf{Ипарх} "--- начальник области; градоначальник; наместник. 




\noindent\textbf{Ипостась} "--- лицо. 




\noindent\textbf{Ирмос} "--- песнопение, стоящее в начале каждой из песен канона. 




\noindent\textbf{Иродианы} "--- сторонники Ирода. 




\noindent\textbf{Ирой} "--- греч. миф. герой. 




\noindent\textbf{Исказити} "--- испортить; оскопить. 




\noindent\textbf{Искапати} "--- источать; испускать каплями; истечь. 




\noindent\textbf{Исковати} "--- выковать. 




\noindent\textbf{Искони} "--- изначала; вначале; всегда. 




\noindent\textbf{Исконный} "--- бывший искони; всегдашний. 




\noindent\textbf{Искренний} "--- ближний. 




\noindent\textbf{Искус} "--- испытание; искушение; проверка. 




\noindent\textbf{Исперва} "--- сначала; искони. 




\noindent\textbf{Исплевити} "--- выполоть; вырвать; исторгнуть; выдернуть; собрать. 




\noindent\textbf{Исплести} "--- сплести; сложить; составить. 




\noindent\textbf{Исповедатися} "--- признаваться; открыто выражать свою веру. 




\noindent\textbf{Исповедник} "--- человек, подвергавшийся страданиям или гонению за веру Христову. 




\noindent\textbf{Исполнение} "--- полнота; наполнение; совершение. 




\noindent\textbf{Исполнь} "--- наполненный; исполненный. 




\noindent\textbf{Исполняти} "--- наполнять; совершать. 




\noindent\textbf{Исполу} "--- вполовину; пополам; частию. 




\noindent\textbf{Исправити} "--- выпрямить; исправить; направить; укрепить. 




\noindent\textbf{Исправление} "--- восстановление; правый образ жизни. 




\noindent\textbf{Испразднити} "--- ниспровергнуть; уничтожить; умалить. 




\noindent\textbf{Испрати, исперити} "--- вытоптать; вымыть. 




\noindent\textbf{Испытно} "--- тщательно. 




\noindent\textbf{Испытовати} "--- выведывать. 




\noindent\textbf{Иссоп} "--- растение, употребляемое в пучках для кропления. 




\noindent\textbf{Истее} "--- точнее; яснее. 




\noindent\textbf{Истесы} "--- чресла, лядвеи. 




\noindent\textbf{Истицание} "--- истечение; истечение семени; поллюция. 




\noindent\textbf{Истаевати} "--- растаять; исчезать. 




\noindent\textbf{Исторгнути} "--- вырвать; вывести. 




\noindent\textbf{Истощание} "--- изнурение; унижение; снисхождение. 




\noindent\textbf{Истрезвлятися} "--- протрезвляться. 




\noindent\textbf{Истукан} "--- статуя; болван; идол. 




\noindent\textbf{Истый, истовый} "--- точный; подлинный; истинный. 




\noindent\textbf{Истязати} "--- вытягивать; получать; допрашивать. 




\noindent\textbf{Исходище} "--- место выхода; исток; начало. 




\noindent\textbf{Исходище вод(ное)} "--- ручей; поток; река. 




\noindent\textbf{Исходище путей} "--- распутье; перекресток. 




\noindent\textbf{Исчадие} "--- детище; плод; род; потомки. 




\noindent\textbf{Иулий} "--- июль. 




\noindent\textbf{Иуний} "--- июнь. 




\bukvaending

\bukva{К}





\noindent\textbf{Кадило} "--- возносимое во славу Божию благовонное курение. 




\noindent\textbf{Кадильница} "--- сосуд, в котором на горящие угли возлагается фимиам для совершения каждения. 




\noindent\textbf{Кадь} "--- кадка, ушат. 




\noindent\textbf{Каженик} "--- скопец; сторож при гареме; придворный. 




\noindent\textbf{Казатель} "--- учитель, наставник. 




\noindent\textbf{Казати} "--- наставлять; поучать. 




\noindent\textbf{Казити} "--- искажать; повреждать. 




\noindent\textbf{Како} "--- как. 




\noindent\textbf{Камара} "--- шатер; скиния; горница; покои. 




\noindent\textbf{Камо} "--- куда? 




\noindent\textbf{Кампан} "--- колокол. 




\noindent\textbf{Камы, камык} "--- камень. 




\noindent\textbf{Камык горящ} "--- сера. 




\noindent\textbf{Кандило} "--- лампада. 




\noindent\textbf{Кандиловжигатель} "--- пономарь. 




\noindent\textbf{Кандия} "--- небольшая чаша. 




\noindent\textbf{Капище} "--- идольский храм. 




\noindent\textbf{Катапетасма} "--- завеса. 




\noindent\textbf{Кафисма} "--- один из 20 разделов, на которые разделена Псалтирь. 




\noindent\textbf{Кацея} "--- кадильница не на цепочках, а на ручке. 




\noindent\textbf{Кацы} "--- каковые; которые; какие. 




\noindent\textbf{Квас} "--- закваска; дрожжи. 




\noindent\textbf{Квасный} "--- приготовленный на дрожжах. 




\noindent\textbf{Келарня, келарница} "--- помещение в монастыре для сохранения вещей, необходимых келарю. 




\noindent\textbf{Келарь} "--- старшая хозяйственная должность в монастыре. 




\noindent\textbf{Кивот} "--- ящик для икон. 




\noindent\textbf{Кидар} "--- головной убор ветхозаветного первосвященника. 




\noindent\textbf{Кимвал} "--- музыкальный инструмент. 




\noindent\textbf{Кимин} "--- тмин. 




\noindent\textbf{Киновия} "--- общежительный монастырь. 




\noindent\textbf{Кинсон} "--- дань; подать; ценз. 




\noindent\textbf{Кириопасха} "--- название праздника Пасхи, пришедшегося на день Благовещения Пресвятой Богородицы 25 марта. 




\noindent\textbf{Кичение} "--- гордость. 




\noindent\textbf{Клада} "--- колода (орудие пытки). 




\noindent\textbf{Кладенец} "--- яма; клад. 




\noindent\textbf{Кладязь} "--- колодец. 




\noindent\textbf{Клас} "--- колос. 




\noindent\textbf{Клеврет} "--- товарищ; собрат. 




\noindent\textbf{Клепало} "--- колотушка, при помощи которой в монастырях созывают на молитву. 




\noindent\textbf{Клепати} "--- звонить; стучать или бить в клепало. 




\noindent\textbf{Клеть} "--- изба; покои; кладовая; комната. 




\noindent\textbf{Клирос} "--- возвышение в храме, на котором располагаются певчие. 




\noindent\textbf{Клич} "--- крик, гам. 




\noindent\textbf{Клобук} "--- покрывало, носимое монашествующими поверх камилавки. 




\noindent\textbf{Ключимый} "--- годный; хороший; случившийся кстати; полезный. 




\noindent\textbf{Ключитися} "--- приключиться; случиться. 




\noindent\textbf{Книгочий} "--- судья; приставник. 




\noindent\textbf{Книжник} "--- ученый. 




\noindent\textbf{Ков} "--- умысел; заговор. 




\noindent\textbf{Ковчег} "--- кованый ящик: сундук; ларец. 




\noindent\textbf{Кодрант} "--- мелкая римская монета. 




\noindent\textbf{Козлогласование} "--- бесчинные крики на пиршестве. 




\noindent\textbf{Козни} "--- лукавство; хитрость. 




\noindent\textbf{Кокош} "--- наседка. 




\noindent\textbf{Колено} "--- род; поколение. 




\noindent\textbf{Колесницегонитель} "--- возница; преследователь на колеснице. 




\noindent\textbf{Коливо} "--- вареная пшеница с медом, приносимая для благословения в церковь на праздники. 




\noindent\textbf{Колиждо} "--- когда; как. 




\noindent\textbf{Колико} "--- сколько. 




\noindent\textbf{Колия} "--- яма; ров. 




\noindent\textbf{Колми} "--- сколько. 




\noindent\textbf{Колми паче} "--- тем более; особенно. 




\noindent\textbf{Коло} "--- колесо. 




\noindent\textbf{Колобродити} "--- ходить вокруг; уклоняться. 




\noindent\textbf{Коль} "--- сколько; насколько; как. 




\noindent\textbf{Колькраты} "--- сколько раз; как часто. 




\noindent\textbf{Комбоста} "--- сырая капуста. 




\noindent\textbf{Кондак} "--- короткая песнь в честь святого или праздника. 




\noindent\textbf{Коноб} "--- котел; горшок; умывальница. 




\noindent\textbf{Конура} "--- небольшой мешочек, носимый суеверными людьми вместе с кореньями или другими амулетами. 




\noindent\textbf{Копр} "--- укроп; анис. 




\noindent\textbf{Кораблец} "--- небольшой корабль. 




\noindent\textbf{Корван} "--- дар; жертва Богу. 




\noindent\textbf{Корвана} "--- казнохранилище при храме Иерусалимском. 




\noindent\textbf{Кормило} "--- руль. 




\noindent\textbf{Кормильствовати} "--- править; руководить. 




\noindent\textbf{Кормление} "--- правление; управление. 




\noindent\textbf{Корчаг} "--- лохань. 




\noindent\textbf{Корчемница} "--- корчма;кабак. 




\noindent\textbf{Коснити} "--- медлить. 




\noindent\textbf{Косноязычный} "--- медленноязычный; заика. 




\noindent\textbf{Косный} "--- медленный; нерешительный; упорно остающийся в одном и том же состоянии. 




\noindent\textbf{Котва} "--- якорь. 




\noindent\textbf{Кош} "--- кошель; корзина. 




\noindent\textbf{Кошница} "--- кошель, корзина. 




\noindent\textbf{Кощунник} "--- шут, балагур. 




\noindent\textbf{Кощунница} "--- актриса; танцовщица. 




\noindent\textbf{Кощуны} "--- смехотворство. 




\noindent\textbf{Крабица} "--- коробочка; ящичек; ковчежец; ларчик. 




\noindent\textbf{Крава} "--- корова. 




\noindent\textbf{Краегранесие, краестрочие} "--- акростих, т.~е. поэтическое произведение, в котором начальные буквы каждой строчки составляют слово, фразу или следуют порядку алфавита. 




\noindent\textbf{Крамола} "--- смута; заговор; бунт. 




\noindent\textbf{Красный} "--- красивый; прекрасный; непорочный. 




\noindent\textbf{Красовул} "--- мерная чаша в монастырях, вмещающая более 200 г . 




\noindent\textbf{Крастель} "--- перепел. 




\noindent\textbf{Крата} "--- раз. 




\noindent\textbf{Крепкий} "--- сильный; крепкий. 




\noindent\textbf{Креплий} "--- крепчайший, сильнейший. 




\noindent\textbf{Кресати} "--- извлекать; высекать огонь; оживлять. 




\noindent\textbf{Крин} "--- лилия. 




\noindent\textbf{Кроме} "--- вне; извне; отдельно; кроме. 




\noindent\textbf{Кромешный} "--- внешний; запредельный; отдаленный; лишенный. 




\noindent\textbf{Кропило} "--- кисть для окропления освященной водой. 




\noindent\textbf{Ктитор} "--- создатель; строитель или снабдитель храма или монастыря; церковный староста. 




\noindent\textbf{Ктому} "--- впредь; затем; еще; уже; более. 




\noindent\textbf{Купа} "--- кипа; груда; куча; ворох. 




\noindent\textbf{Купель} "--- озеро; пруд; садок; сосуд для совершения Таинства Крещения. 




\noindent\textbf{Купина} "--- соединение нескольких однородных предметов: куст, сноп; терновый куст. 




\noindent\textbf{Купно} "--- вместе. 




\noindent\textbf{Купный} "--- совместный. 




\noindent\textbf{Кустодия} "--- стража; караул; охрана; 




\noindent\textbf{Кутия} "--- вареная пшеница с медом, приносимая в церковь на поминовение усопших христиан. 




\noindent\textbf{Куща} "--- шатер; палатка; шалаш. 




\noindent\textbf{Кущник} "--- человек, делающий палатки или живущий в шалаше. 




\bukvaending

\bukva{Л}





\noindent\textbf{Ладия} "--- небольшое судно; кораблик; ладья. 




\noindent\textbf{Ладан} "--- благоуханная смола, влагаемая в кадильницу на горящие угли для благовонного курения. 




\noindent\textbf{Лазарома} "--- гробная одежда; повой; плащаница, в которую повивали усопших у иудеев. 




\noindent\textbf{Лазня} "--- баня. 




\noindent\textbf{Лай} "--- хула; поношение. 




\noindent\textbf{Лакать} "--- евр. мера длины. 




\noindent\textbf{Ланита} "--- щека. 




\noindent\textbf{Лаятель} "--- ругатель; хулитель; седящий в засаде. 




\noindent\textbf{Лвичищ} "--- львенок. 




\noindent\textbf{Левиафан} "--- крокодил. 




\noindent\textbf{Легеон} "--- полк; толпа; множество. 




\noindent\textbf{Лежание} "--- лежание; опочивание. 




\noindent\textbf{Лемаргия} "--- гортанобесие, т.~е. гурманство. 




\noindent\textbf{Лентион, лентий} "--- полотенце. 




\noindent\textbf{Лепо} "--- красиво. 




\noindent\textbf{Лепоподобно} "--- благопристойно; по достоинству. 




\noindent\textbf{Лепота} "--- красота; изящество. 




\noindent\textbf{Лепта} "--- мелкая монетка. 




\noindent\textbf{Лествица} "--- лестница. 




\noindent\textbf{Лестчий} "--- льстивый, ложный. 




\noindent\textbf{Лесть} "--- обман; хитрость; коварность. 




\noindent\textbf{Лето} "--- год; время. 




\noindent\textbf{Леторасль} "--- выросшее за год, годовой побег дерева. 




\noindent\textbf{Леть} "--- льзя; можно. 




\noindent\textbf{Леха} "--- гряда, ряд. 




\noindent\textbf{Лечба} "--- лекарство; врачевство. 




\noindent\textbf{Лечец} "--- лекарь; врач. 




\noindent\textbf{Лжа} "--- ложь. 




\noindent\textbf{Лжесловесие} "--- лживые речи. 




\noindent\textbf{Лив} "--- полдень; юг; юго-западный ветер. 




\noindent\textbf{Ливан} "--- иногда значит то же, что и Ладан. 




\noindent\textbf{Лик} "--- собрание; хор. 




\noindent\textbf{Ликование} "--- многолюдное пение; пляска; танцы. 




\noindent\textbf{Ликоватися} "--- приветствовать чрез соприкосновение правой щекой. 




\noindent\textbf{Ликовне} "--- с ликованием. 




\noindent\textbf{Ликостояние} "--- бдение на молитве церковной. 




\noindent\textbf{Лития} "--- исхождение из церкви на молитву. 




\noindent\textbf{Литра} "--- мера веса. 




\noindent\textbf{Литургисати} "--- совершать Литургию. 




\noindent\textbf{Лихва} "--- прибыль; проценты. 




\noindent\textbf{Лихоимец} "--- ростовщик; сребролюбец. 




\noindent\textbf{Лице} "--- лицо; вид; человек. 




\noindent\textbf{Личина} "--- маскарадная или шутовская маска. 




\noindent\textbf{Лишатися} "--- нуждаться. 




\noindent\textbf{Лишше} "--- больше, сверх того. 




\noindent\textbf{Лобзание} "--- устное целование. 




\noindent\textbf{Ловитва} "--- ловля; охота; сети; добыча; грабеж. 




\noindent\textbf{Ловительство} "--- засада, ловушка. 




\noindent\textbf{Ложе} "--- постель, одр. 




\noindent\textbf{Ложесна} "--- утроба женщины. 




\noindent\textbf{Лоза} "--- виноград. 




\noindent\textbf{Ломимый} "--- преломляемый. 




\noindent\textbf{Лоно} "--- пазуха; грудь; колени. 




\noindent\textbf{Луновение} "--- месячный цикл у женщин. 




\noindent\textbf{Лысто} "--- голень; икры; лытка. 




\noindent\textbf{Льстивый} "--- обманчивый. 




\noindent\textbf{Льщение} "--- обман; коварство; лесть. 




\noindent\textbf{Любо} "--- либо, или. 




\noindent\textbf{Любомудрие} "--- философия. 




\noindent\textbf{Любомятежный} "--- склонный к мятежу. 




\noindent\textbf{Любоначалие} "--- властолюбие. 




\noindent\textbf{Любопразднственный} "--- любящий празднствовать. 




\noindent\textbf{Любопрение} "--- любовь состязаться, спорить. 




\noindent\textbf{Любосластие} "--- сластолюбие; любовь к плотским утехам. 




\noindent\textbf{Любочестие} "--- почитание; чествование. 




\noindent\textbf{Любочестный} "--- достойный похвалы, чести. 




\noindent\textbf{Любы} "--- любовь. 




\noindent\textbf{Люте} "--- жестоко; тяжко. 




\noindent\textbf{Лютый} "--- свирепый; жестокий; злой; мучительный. 




\noindent\textbf{Лядвея} "--- ляжка; верхняя половина ноги; промежность. 




\noindent\textbf{Лярва} "--- маска; личина. 




\bukvaending

\bukva{М}





\noindent\textbf{Маание} "--- знак рукой, головою, глазами или иного рода, содержащий приказание; повеление; воля. 




\noindent\textbf{Маий} "--- май. 




\noindent\textbf{Малакия} "--- грех рукоблудия. 




\noindent\textbf{Малимый} "--- уменьшаемый. 




\noindent\textbf{Малобрещи} "--- нерадеть о чем-либо. 




\noindent\textbf{Мамона} "--- богатство; имение. 




\noindent\textbf{Мандра} "--- ограда. 




\noindent\textbf{Мание} "--- знак рукой, головою, глазами или иного рода, содержащий приказание; повеление; воля. 




\noindent\textbf{Манна} "--- небесный хлеб, данный израильтянам в пустыне. 




\noindent\textbf{Манноприемный} "--- содержащий манну. 




\noindent\textbf{Маслина} "--- олива; оливковое дерево. 




\noindent\textbf{Масличный} "--- оливковый. 




\noindent\textbf{Мастити} "--- намазывать. 




\noindent\textbf{Маститый} "--- обильный; тучный; заслуженный. 




\noindent\textbf{Масть} "--- мазь; масло. 




\noindent\textbf{Матеродевственный} "--- одновременно относящийся и к матери, и к деве. 




\noindent\textbf{Матеролепне} "--- по-матерински. 




\noindent\textbf{Матерский} "--- материнский. 




\noindent\textbf{Матерь градовом} "--- столица; первопрестольный град. 




\noindent\textbf{Мгляный} "--- окруженный или покрытый мглой. 




\noindent\textbf{Медвен} "--- медовый. 




\noindent\textbf{Медленоязычный} "--- косноязычный; заика. 




\noindent\textbf{Медница} "--- медная монетка. 




\noindent\textbf{Медовина} "--- вареный мед с хмелем. 




\noindent\textbf{Медоточный} "--- источающий, изливающий мед. 




\noindent\textbf{Медоязычный} "--- сладкословесный. 




\noindent\textbf{Междорамие} "--- пространство между плечами. 




\noindent\textbf{Мездник} "--- наемник. 




\noindent\textbf{Мерзость} "--- скверна; гнусность; беззаконие;нечестие; иногда "--- идол. 




\noindent\textbf{Мерило} "--- мера; весы. 




\noindent\textbf{Меск} "--- полуосел; мул; лошак. 




\noindent\textbf{Мессия} "--- евр. помазанник. 




\noindent\textbf{Метание} "--- поясной поклон. 




\noindent\textbf{Мех} "--- кожаный мешок для сохранения и перевоза жидкостей. 




\noindent\textbf{Мжа} "--- мигание; прищур. 




\noindent\textbf{Мжати} "--- жмурить глаза; щуриться; плохо видеть. 




\noindent\textbf{Мзда} "--- награда; плата. 




\noindent\textbf{Мздовоздаятель} "--- оплачивающий работу, дающий награду. 




\noindent\textbf{Мздоимание} "--- взяточничество. 




\noindent\textbf{Мила ся деяти} "--- низко припадать к земле; просить сжалиться над собой. 




\noindent\textbf{Милоть} "--- овчина; грубый шерстяной плащ из овечьей шерсти. 




\noindent\textbf{Милый} "--- жалкий; заслуживающий сожаления. 




\noindent\textbf{Мимотещи} "--- идти, проходить мимо, не останавливаясь. 




\noindent\textbf{Мирная} "--- название великой ектении. 




\noindent\textbf{Миро} "--- благовонная жидкость или мазь. 




\noindent\textbf{Мироподательне} "--- подавая мир. 




\noindent\textbf{Мироточец} "--- источающий чудотворное миро. 




\noindent\textbf{Мироявленный} "--- явленный, открытый миру. 




\noindent\textbf{Мирсина} "--- название красивого дерева. 




\noindent\textbf{Младодеяти, младодействовати} "--- принимать образ младенца; облекаться в плоть. 




\noindent\textbf{Младоумие} "--- незрелость ума. 




\noindent\textbf{Млат} "--- молот. 




\noindent\textbf{Млеко} "--- молоко. 




\noindent\textbf{Мнас} "--- мина, древнегреч. серебряная монета. 




\noindent\textbf{Мнее} "--- менее. 




\noindent\textbf{Мнети, мнити} "--- думать; предполагать; казаться. 




\noindent\textbf{Мний} "--- меньший. 




\noindent\textbf{Мних} "--- монах. 




\noindent\textbf{Многажды, множицею} "--- часто; много раз. 




\noindent\textbf{Многобезсловесие} "--- невежество. 




\noindent\textbf{Многобогатый} "--- изобилующий во всем. 




\noindent\textbf{Многоболезненный} "--- подъявший многие труды, подвиги, беды, страдания. 




\noindent\textbf{Многоборимый} "--- подвергаемый сильным искушениям, нападениям. 




\noindent\textbf{Многобурный} "--- тревожный. 




\noindent\textbf{Многогобзенный} "--- весьма обильный. 




\noindent\textbf{Многогубо} "--- многократно. 




\noindent\textbf{Многокласный} "--- колосистый. 




\noindent\textbf{Многомятущий} "--- преисполненный суетою. 




\noindent\textbf{Многонарочитый} "--- весьма знаменитый. 




\noindent\textbf{Многообразне} "--- во многих видах; различно. 




\noindent\textbf{Многооранный} "--- многократно возделанный. 




\noindent\textbf{Многоочитый} "--- имеющий множество глаз. 




\noindent\textbf{Многоплодие} "--- плодоносие; многочадие. 




\noindent\textbf{Многоплотие} "--- тучность. 




\noindent\textbf{Многопрелестный} "--- исполненный прелестей и соблазнов. 




\noindent\textbf{Многосветлый} "--- радостный; торжественный. 




\noindent\textbf{Многослезный} "--- исполненный печали и горя. 




\noindent\textbf{Многоснедный} "--- изобилующий многообразием пищи. 




\noindent\textbf{Многосугубый} "--- усугубленный; умноженный; усиленный. 




\noindent\textbf{Многосуетный} "--- совершенно пустой, бесполезный. 




\noindent\textbf{Многоуветливый} "--- очень снисходительный. 




\noindent\textbf{Многоцелебный} "--- подающий многие исцеления. 




\noindent\textbf{Многочастне} "--- много раз. 




\noindent\textbf{Многочудесный} "--- источающий многие чудеса; прославленный чудотворениями. 




\noindent\textbf{Многоязычный} "--- состоящий из множества племен. 




\noindent\textbf{Молва} "--- говор; ропот; слух; забота; волнение. 




\noindent\textbf{Молвити} "--- заботиться; суетиться; волноваться; роптать. 




\noindent\textbf{Молие} "--- моль. 




\noindent\textbf{Молниезрачный} "--- напоминающий молнию. 




\noindent\textbf{Мочащийся к стене} "--- пес. 




\noindent\textbf{Мощи} "--- нетленное тело угодника Божия. 




\noindent\textbf{Мравий} "--- муравей. 




\noindent\textbf{Мраз} "--- мороз. 




\noindent\textbf{Мрежа} "--- рыболовная сеть. 




\noindent\textbf{Мужатая} "--- замужняя. 




\noindent\textbf{Мужатица} "--- замужняя женщина. 




\noindent\textbf{Муженеискусная} "--- не познавшая мужа; не причастная браку. 




\noindent\textbf{Мурин} "--- эфиоп; арап; негр; чернокожий; дух тьмы; бес. 




\noindent\textbf{Мусийский, мусикийский} "--- музыкальный. 




\noindent\textbf{Мусикия} "--- музыка. 




\noindent\textbf{Мшела} "--- взятка. 




\noindent\textbf{Мшелоимство} "--- корыстолюбие. 




\noindent\textbf{Мшица} "--- мошка; мошкара. 




\noindent\textbf{Мытарь} "--- сборщик подати. 




\noindent\textbf{Мытница} "--- таможня; дом или двор для сбора пошлин. 




\noindent\textbf{Мыто} "--- пошлина; сбор; налог. 




\noindent\textbf{Мышца} "--- рука; плечо; сила. 




\noindent\textbf{Мясопуст} "--- последний день вкушения мясной пищи. 




\noindent\textbf{Мясоястие, мясоед} "--- время, когда Устав разрешает вкушение мяса. 




\noindent\textbf{Мятва} "--- мята. 




\bukvaending

\bukva{Н}





\noindent\textbf{Набдевати} "--- снабжать; наделять; хранить. 




\noindent\textbf{Наваждати} "--- научать; подстрекать. 




\noindent\textbf{Навет} "--- наговор; клевета; козни. 




\noindent\textbf{Навклир} "--- хозяин корабля. 




\noindent\textbf{Навыкнути} "--- приучиться; привыкнуть. 




\noindent\textbf{Наготовати} "--- ходить без одежды. 




\noindent\textbf{Нагствовати} "--- см. Наготовати. 




\noindent\textbf{Надходити} "--- внезапно постигнуть, случиться. 




\noindent\textbf{Наздати} "--- надстроить; укрепить; утвердить. 




\noindent\textbf{Назирати} "--- примечать; наблюдать. 




\noindent\textbf{Назнаменовати} "--- назначать; обозначать; осенять Крестом. 




\noindent\textbf{Наипаче} "--- особенно; преимущественно. 




\noindent\textbf{Наитие} "--- нисшествие; нашествие; сошествие. 




\noindent\textbf{Наказание} "--- иногда: учение. 




\noindent\textbf{Наляцати} "--- натянуть. 




\noindent\textbf{На мале} "--- малое время; дешево. 




\noindent\textbf{Намащати} "--- намазывать; втирать. 




\noindent\textbf{На мнозе} "--- на долгое время; дорого. 




\noindent\textbf{Наопак} "--- наоборот; вопреки. 




\noindent\textbf{Напаствуемый} "--- находящийся в напасти. 




\noindent\textbf{Наперсник} "--- друг, доверенное лицо. 




\noindent\textbf{Напоследок} "--- недавно. 




\noindent\textbf{Нард} "--- колосистое ароматическое растение. 




\noindent\textbf{Нарекованный} "--- предопределенный; предуставленный; назначенный. 




\noindent\textbf{Нарицати} "--- называть. 




\noindent\textbf{Нарок} "--- определенное или назначенное время. 




\noindent\textbf{Нарочитый} "--- особый; славный. 




\noindent\textbf{Наругатися} "--- насмеяться; пренебречь; опозорить. 




\noindent\textbf{Насмертник} "--- осужденный на смерть 




\noindent\textbf{Насущный} "--- настоящий; нынешний; существенный; необходимый. 




\noindent\textbf{На толице} "--- в такое время; за такую цену, за столько. 




\noindent\textbf{Началозлобный} "--- виновник зла. 




\noindent\textbf{Начаток} "--- начало; первый плод. 




\noindent\textbf{Начертавати} "--- изобразить. 




\noindent\textbf{Наясне} "--- наружу; открыто. 




\noindent\textbf{Наяти} "--- нанять. 




\noindent\textbf{Неблазненный} "--- безопасный; непогрешимый. 




\noindent\textbf{Неблазный} "--- непрельщаемый. 




\noindent\textbf{Небрещи} "--- нерадеть; пренебрегать. 




\noindent\textbf{Невеглас} "--- невежда; простак; неученый. 




\noindent\textbf{Невеститель} "--- снабжающий бедных невест приданым. 




\noindent\textbf{Невестоукрасити} "--- украсить как невесту. 




\noindent\textbf{Невечерний} "--- непомрачаемый; светлый. 




\noindent\textbf{Невиновный} "--- беспричинный; самобытный. 




\noindent\textbf{Невозбранно} "--- беспрепятственно. 




\noindent\textbf{Невозносительно} "--- смиренно. 




\noindent\textbf{Негли} "--- неужели; может быть; авось. 




\noindent\textbf{Неделя} "--- церковное название воскресного дня. 




\noindent\textbf{Недремлющий} "--- неусыпный. 




\noindent\textbf{Недристый} "--- имеющий широкую грудь. 




\noindent\textbf{Недро} "--- нутро; утроба; грудь; внутренность; залив. 




\noindent\textbf{Недуг} "--- болезнь. 




\noindent\textbf{Неже} "--- нежели; чем. 




\noindent\textbf{Независтный} "--- неиспорченный; невредимый; довольный; обильный. 




\noindent\textbf{Неиждиваемый} "--- не могущий быть истрачен или использован до конца. 




\noindent\textbf{Неизводимый} "--- непрекращаемый. 




\noindent\textbf{Неизгиблемый} "--- не подлежащий тлению или времени. 




\noindent\textbf{Неизреченный} "--- невыразимый. 




\noindent\textbf{Неискусобрачный} "--- не испытавший брака. 




\noindent\textbf{Неискусомужная} "--- не познавшая мужа. 




\noindent\textbf{Неиспытный} "--- сокровенный; тайный. 




\noindent\textbf{Неистовно} "--- с ожесточением; с яростию. 




\noindent\textbf{Неистовый} "--- вышедший из себя; находящийся не в должном состоянии. 




\noindent\textbf{Неисследимый} "--- непостижимый. 




\noindent\textbf{Неключимый} "--- бесполезный; негодный. 




\noindent\textbf{Некосненно} "--- немедленно. 




\noindent\textbf{Не ктому} "--- более не; еще не; уже не. 




\noindent\textbf{Нелестный} "--- необманчивый; нелукавый. 




\noindent\textbf{Нелеть} "--- нельзя. 




\noindent\textbf{Неможение} "--- болезнь; немощь; бессилие. 




\noindent\textbf{Немокренно} "--- по суху. 




\noindent\textbf{Немощствующий} "--- больной. 




\noindent\textbf{Необименный} "--- необъятный. 




\noindent\textbf{Не обинутися} "--- поступать смело. 




\noindent\textbf{Необинуяся} "--- смело; дерзновенно. 




\noindent\textbf{Неопальный} "--- несгораемый. 




\noindent\textbf{Неописанне} "--- изобразимо. 




\noindent\textbf{Неопределенный} "--- беспредельный. 




\noindent\textbf{Неоранный} "--- непаханный; невозделанный; нетронутый. 




\noindent\textbf{Неотметный} "--- неотчужденный. 




\noindent\textbf{Неплоды, неплодовь} "--- бесплодная женщина. 




\noindent\textbf{Неподобный} "--- непристойный. 




\noindent\textbf{Непорочны} "--- название 17-й кафизмы псалма 118. 




\noindent\textbf{Непорочный} "--- беспорочный; святой; чистый. 




\noindent\textbf{Неправдовати} "--- поступать нечестиво. 




\noindent\textbf{Непраздная} "--- беременная. 




\noindent\textbf{Непревратный} "--- непременный; неизменяемый. 




\noindent\textbf{Непреложно} "--- неизменно; без изменения. 




\noindent\textbf{Непременный} "--- неизменяемый. 




\noindent\textbf{Непреоборимый} "--- неодолимый; непобедимый. 




\noindent\textbf{Непщевание} "--- мнение; подлог; выдумка. 




\noindent\textbf{Непщевати} "--- думать; придумывать; считать. 




\noindent\textbf{Неразседный} "--- неразрушаемый. 




\noindent\textbf{Нерешимый} "--- несокрушимый; неразвязываемый. 




\noindent\textbf{Неседальное} "--- церковная служба, во время которой возбраняется сидеть. 




\noindent\textbf{Несланый} "--- несоленый. 




\noindent\textbf{Неслиянне} "--- неслитно. 




\noindent\textbf{Несмесне} "--- не смешиваясь. 




\noindent\textbf{Нестареемый} "--- вечный; неизменный. 




\noindent\textbf{Несть} "--- нет. 




\noindent\textbf{Нестояние} "--- непостоянство; смущение. 




\noindent\textbf{Несумненный} "--- несомненный; надежный; беспристрастный. 




\noindent\textbf{Несущий} "--- не имеющий бытия. 




\noindent\textbf{Нетление} "--- неуничтожимость; вечность; несокрушимость. 




\noindent\textbf{Нетребе} "--- не нужно. 




\noindent\textbf{Нетреный} "--- непротертый; непроходимый. 




\noindent\textbf{Нетесноместно} "--- удобовместительно. 




\noindent\textbf{Нетяжестне} "--- без труда. 




\noindent\textbf{Не у} "--- еще не. 




\noindent\textbf{Неудобоприятный} "--- невместимый; непонятный; непостижимый. 




\noindent\textbf{Неудобь} "--- неудобно; трудно. 




\noindent\textbf{Неумытный} "--- неподкупный. 




\noindent\textbf{Нечаяние} "--- неожиданность; беспечность. 




\noindent\textbf{Неясыть} "--- пеликан. 




\noindent\textbf{Ниже} "--- тем более не…; ни даже…; и не… 




\noindent\textbf{Николиже} "--- никогда. 




\noindent\textbf{Ни ли} "--- разве не? неужели? или не? 




\noindent\textbf{Ниц} "--- вниз; лицем на землю. 




\noindent\textbf{Нищетный} "--- нищенский; униженный; бедный. 




\noindent\textbf{Новемврий} "--- ноябрь. 




\noindent\textbf{Новина} "--- новость. 




\noindent\textbf{Новозданный} "--- вновь построенный. 




\noindent\textbf{Новопросвещенный} "--- недавно крещеный. 




\noindent\textbf{Новосаждение} "--- почки; отпрыски 




\noindent\textbf{Ноемврий} "--- ноябрь. 




\noindent\textbf{Ножница} "--- ножны. 




\noindent\textbf{Нощный вран} "--- филин; сова. 




\noindent\textbf{Нудитися} "--- неволиться; принуждаться; достигаться с усилием. 




\noindent\textbf{Нудить} "--- пытать. 




\noindent\textbf{Нудма} "--- насильно. 




\noindent\textbf{Нуждник} "--- употребляющий усилие. 




\noindent\textbf{Нырище} "--- развалины; руины; нежилое место. 




\noindent\textbf{Ню} "--- ее. 




\bukvaending

\bukva{О}





\noindent\textbf{Обавание} "--- ворожба; нашептывание; волхвование; колдовство. 




\noindent\textbf{Обаватель} "--- обаятель; чародей; ворожея. 




\noindent\textbf{Обавати} "--- обаять; очаровывать; ворожить; колдовать; заговаривать. 




\noindent\textbf{Обада} "--- оболгание; оклеветание. 




\noindent\textbf{Обажаемый} "--- оклеветаемый. 




\noindent\textbf{Обанадесять} "--- двенадцать. 




\noindent\textbf{Обапо} "--- с обеих сторон; по обеим сторонам. 




\noindent\textbf{Обаче} "--- однако; впрочем; но. 




\noindent\textbf{Обвеселити} "--- обрадовать. 




\noindent\textbf{Обвечеряти} "--- ночевать; переночевать. 




\noindent\textbf{Обглядати} "--- смотреть; оглядывать. 




\noindent\textbf{Обдержание} "--- сдерживание; управление; стеснение; грусть; впадение. 




\noindent\textbf{Обдесноручный} "--- человек, свободно владеющий как правой, так и левой рукой. 




\noindent\textbf{Обезвинити} "--- остаться без наказания; не знать за собой вины. 




\noindent\textbf{Обезжилити} "--- лишить сил, крепости. 




\noindent\textbf{Обезплодствити} "--- лишить плода, успеха. 




\noindent\textbf{Обезтлити} "--- сделать нетленным. 




\noindent\textbf{Обесити} "--- повесить на чем-либо. 




\noindent\textbf{Обет, обетование} "--- обещание. 




\noindent\textbf{Обетшати} "--- придти в ветхость; состариться; сделаться негодным; ослабеть; сокрушиться. 




\noindent\textbf{Обещник} "--- сообщник; товарищ. 




\noindent\textbf{Обжадати} "--- доносить; клеветать. 




\noindent\textbf{Обзорище} "--- высокая башня для наблюдения за местностью. 




\noindent\textbf{Обидитель} "--- обидчик. 




\noindent\textbf{Обиматель} "--- собиратель винограда. 




\noindent\textbf{Обиноватися} "--- колебаться; сомневаться; робеть; говорить непрямо, намеками. 




\noindent\textbf{Обиновение} "--- отступление. 




\noindent\textbf{Обиталище} "--- жилище. 




\noindent\textbf{Обитель} "--- гостиница. 




\noindent\textbf{Облагати} "--- ублажать; говорить ласково. 




\noindent\textbf{Облагодатити} "--- ниспослать благодать. 




\noindent\textbf{Облагоухати} "--- исполнить благовонием. 




\noindent\textbf{Облазнити} "--- направить по ложному следу; ввести в заблуждение. 




\noindent\textbf{Облазнитися} "--- впасть в заблуждение. 




\noindent\textbf{Область} "--- власть; сила; господство. 




\noindent\textbf{Облачити} "--- одеть. 




\noindent\textbf{Облещи} "--- облечь; одеть; лечь вокруг; окружить; сделать привал; остановиться; остаться. 




\noindent\textbf{Облистание} "--- озарение; яркий свет. 




\noindent\textbf{Облистати} "--- осветить; озарить. 




\noindent\textbf{Обличати} "--- показывать чье-либо подлинное лицо; выказывать; обнаруживать. 




\noindent\textbf{Обложити} "--- окружить. 




\noindent\textbf{Обноществовати} "--- ночевать; препроводить ночь. 




\noindent\textbf{Обнощь} "--- всю ночь. 




\noindent\textbf{Обожати} "--- обоготворять; чествовать как Бога; делать причастным Божественной благодати. 




\noindent\textbf{Оболгати} "--- обмануть. 




\noindent\textbf{Обон пол} "--- по ту сторону; за. 




\noindent\textbf{Обочие} "--- висок. 




\noindent\textbf{Обоюду} "--- по обе стороны; с обеих сторон. 




\noindent\textbf{Обрадованный} "--- приветствованный. 




\noindent\textbf{Образовати} "--- изображать; приобретать образ. 




\noindent\textbf{Обращати} "--- поворачивать; перевертывать; перемещать; вращать. 




\noindent\textbf{Обрести} "--- найти. 




\noindent\textbf{Обретаемый} "--- находимый. 




\noindent\textbf{Обретение} "--- находка; открытие. 




\noindent\textbf{Оброк} "--- плата за службу. 




\noindent\textbf{Обручник} "--- жених, помолвленный с невестой, но еще не вступивший с ней в брак. 




\noindent\textbf{Обсолонь} "--- против солнца. 




\noindent\textbf{Обстояние} "--- осада; беда; напасть. 




\noindent\textbf{Обушие} "--- мочка у уха. 




\noindent\textbf{Обуяти} "--- обезуметь; испортиться; обессилить. 




\noindent\textbf{Объюродити} "--- обезуметь; поглупеть. 




\noindent\textbf{Ов} "--- иной; один. 




\noindent\textbf{Овамо} "--- там; туда. 




\noindent\textbf{Овен} "--- баран. 




\noindent\textbf{Ово} "--- или; либо. 




\noindent\textbf{Овогда} "--- иногда. 




\noindent\textbf{Овоуду} "--- с другой стороны; оттуда. 




\noindent\textbf{Огласити} "--- объявить всенародно; научить; просветить. 




\noindent\textbf{Оглохновение} "--- глухота. 




\noindent\textbf{Огневица} "--- горячка. 




\noindent\textbf{Огненосный} "--- носимый в вихрях огня. 




\noindent\textbf{Огнепальный} "--- пылающий; горящий; палящий. 




\noindent\textbf{Огребатися} "--- удаляться; остерегаться. 




\noindent\textbf{Огустети} "--- сгустить; сделать густым; свернуться (о молоке). 




\noindent\textbf{Одебелети} "--- растолстеть; огрубеть. 




\noindent\textbf{Одесную} "--- справа; по правую руку. 




\noindent\textbf{Одесятствовати} "--- выделять десятую часть. 




\noindent\textbf{Одигитрия} "--- путеводительница. 




\noindent\textbf{Одождити} "--- окропить; оросить; послать в виде дождя; в большом количестве. 




\noindent\textbf{Одр} "--- постель; кровать. 




\noindent\textbf{Ожестети} "--- сделаться жестким; засохнуть. 




\noindent\textbf{Озимение} "--- зимовка. 




\noindent\textbf{Озлобление} "--- несчастье; гнев. 




\noindent\textbf{Озлобляти} "--- причинять несчастье; гневить; распалять гневом. 




\noindent\textbf{Озобати} "--- пожирать. 




\noindent\textbf{Окаивати} "--- признавать отверженным. 




\noindent\textbf{Окаляти} "--- пачкать; осквернять; марать. 




\noindent\textbf{Окаменяти} "--- делать каменным. 




\noindent\textbf{Окаянный} "--- достойный проклятия; нечестивый; грешник. 




\noindent\textbf{Окаянство} "--- преступность; богоборчество; грех. 




\noindent\textbf{Око} "--- глаз. 




\noindent\textbf{Окованный} "--- обложенный оковами. 




\noindent\textbf{Окормитель} "--- кормчий; правитель. 




\noindent\textbf{Окормляти} "--- направлять; руководить; править. 




\noindent\textbf{Окоявленне} "--- очевидно; откровенно. 




\noindent\textbf{Окрастовети} "--- покрыться коростою. 




\noindent\textbf{Окрест} "--- кругом; около. 




\noindent\textbf{Окриляемый} "--- ограждаемый крыльями. 




\noindent\textbf{Оле} "--- О! 




\noindent\textbf{Оловина} "--- любое хмельное питие, отличное от виноградного вина. 




\noindent\textbf{Олтарь} "--- алтарь, жертвенник. 




\noindent\textbf{Оляденети} "--- зарасти тернием, сорняками. 




\noindent\textbf{Омакати} "--- обливать. 




\noindent\textbf{Ометы} "--- полы; края одежды. 




\noindent\textbf{Она} "--- они (двое). 




\noindent\textbf{Онагр} "--- дикий осел. 




\noindent\textbf{Онамо, онуду} "--- там; туда. 




\noindent\textbf{Онде} "--- в ином месте; там. 




\noindent\textbf{Онема} "--- им (двоим). 




\noindent\textbf{Он пол} "--- противоположный берег. 




\noindent\textbf{Онсица} "--- такой-то. 




\noindent\textbf{Опасно} "--- осмотрительно; тщательно; осторожно; опасно. 




\noindent\textbf{Оплазивый} "--- любопытный; пустословный; лазутчик. 




\noindent\textbf{Оплазнство} "--- ухищрение; пустословие. 




\noindent\textbf{Оплот} "--- ограда; забор; тын. 




\noindent\textbf{Ополчатися} "--- готовиться к сражению. 




\noindent\textbf{Оправдание} "--- заповедь; устав; закон. 




\noindent\textbf{Опреснок} "--- пресный хлеб, испеченный без использования дрожжей. 




\noindent\textbf{Орало} "--- плуг; соха. 




\noindent\textbf{Оранный} "--- распаханный. 




\noindent\textbf{Оратай} "--- пахарь. 




\noindent\textbf{Орати} "--- пахать. 




\noindent\textbf{Орган} "--- орган, музыкальный инструмент. 




\noindent\textbf{Осанна} "--- молитвенное восклицание у евреев-«спасение (от Бога)». 




\noindent\textbf{Оселский} "--- ослиный. 




\noindent\textbf{Жернов оселский} "--- верхний большой жернов в мельнице, приводимый в движение ослом. 




\noindent\textbf{Осенити} "--- покрыть тенью. 




\noindent\textbf{Осклабитися} "--- усмехнуться; улыбнуться. 




\noindent\textbf{Оскорбети} "--- опечалиться; соскучиться. 




\noindent\textbf{Оскорд} "--- топор. 




\noindent\textbf{Ослаба} "--- облегчение; льгота. 




\noindent\textbf{Осля} "--- молодой осел. 




\noindent\textbf{Осмица} "--- восемь. 




\noindent\textbf{Осмоктати} "--- обсосать; облизать. 




\noindent\textbf{Оставити} "--- оставить; простить; позволить. 




\noindent\textbf{Остенити} "--- огородить стеной, защитить. 




\noindent\textbf{Острастший} "--- обидящий. 




\noindent\textbf{Острог} "--- земляной вал. 




\noindent\textbf{Острупити} "--- поразить проказой. 




\noindent\textbf{Осуществовати} "--- осуществлять; давать бытие. 




\noindent\textbf{Осьмерицею} "--- восемь раз. 




\noindent\textbf{Отай} "--- тайно; скрытно. 




\noindent\textbf{Отверзати} "--- открывать; отворять. 




\noindent\textbf{Отвнеуду} "--- снаружи. 




\noindent\textbf{Отдати} "--- иногда: простить. 




\noindent\textbf{Отдоенное} "--- грудной младенец. 




\noindent\textbf{Отдоитися} "--- воскормить грудью. 




\noindent\textbf{Отерпати} "--- делаться твердым (терпким); деревенеть; отвердевать; неметь. 




\noindent\textbf{Отити в путь всея земли} "--- умереть. 




\noindent\textbf{Откосненно} "--- наискось. 




\noindent\textbf{Откровение} "--- открытие; просветление; просвещение. 




\noindent\textbf{Отлог} "--- ущерб; урон. 




\noindent\textbf{Отложение} "--- отвержение; отступление. 




\noindent\textbf{Отметатися} "--- отрекаться; не признавать; отвергаться; отпадать. 




\noindent\textbf{Отметный} "--- отвергнутый; запрещенный. 




\noindent\textbf{Отнелиже} "--- с тех пор как; с того времени как. 




\noindent\textbf{Отнюд} "--- совершенно; отнюдь. 




\noindent\textbf{Отнюдуже, отонюдуже} "--- откуда; почему. 




\noindent\textbf{Отобоюду} "--- с той и с другой стороны. 




\noindent\textbf{Отонуду} "--- с другой стороны. 




\noindent\textbf{Отполу} "--- от половины; с середины. 




\noindent\textbf{Отре} "--- сор; мякина; кожура. 




\noindent\textbf{Отребить} "--- очистить; ощипать. 




\noindent\textbf{Отрешати} "--- отвязывать; освобождать. 




\noindent\textbf{Отрешатися} "--- разлучаться. 




\noindent\textbf{Отреяти} "--- отбрасывать; отвергать. 




\noindent\textbf{Отрицатися} "--- отвергать; отметать. 




\noindent\textbf{Отрождение} "--- возрождение. 




\noindent\textbf{Отрок} "--- раб; служитель; мальчик до двенадцати лет; ученик; воин. 




\noindent\textbf{Отроковица} "--- девица до двенадцати лет. 




\noindent\textbf{Отроча} "--- дитя; младенец. 




\noindent\textbf{Отрыгнути} "--- извергнуть. 




\noindent\textbf{Отрыгнуть слово} "--- произнести. 




\noindent\textbf{Оттоле} "--- с того времени. 




\noindent\textbf{Отторгати} "--- открывать; отталкивать. 




\noindent\textbf{Оцеждати} "--- процеживать. 




\noindent\textbf{Оцет} "--- уксус. 




\noindent\textbf{Отщетевати} "--- отнимать; удалять. 




\noindent\textbf{Отщетити} "--- потерять; погубить. 




\noindent\textbf{Очепие} "--- ошейник. 




\noindent\textbf{Очеса} "--- очи, глаза. 




\noindent\textbf{Ошаяватися} "--- устраняться, удаляться. 




\noindent\textbf{Ошиб} "--- хвост. 




\noindent\textbf{Ошуюю} "--- слева; по левую руку. 




\bukvaending

\bukva{П}





\noindent\textbf{Павечерня, павечерница} "--- малая вечерня. 




\noindent\textbf{Паволока} "--- покрывало; чехол; пелена; покров. 




\noindent\textbf{Пагуба} "--- гибель; моровая язва. 




\noindent\textbf{Пажить} "--- луг; нива; пастбище; поле; корм для скота. 




\noindent\textbf{Пазнокти} (мн. ч.) "--- копыта; когти; ногти. 




\noindent\textbf{Паки} "--- опять; еще; снова. 




\noindent\textbf{Пакибытие} "--- духовное обновление. 




\noindent\textbf{Пакости деяти} "--- бить руками; ударять по щеке; оскорблять; вредить. 




\noindent\textbf{Пакостник} "--- причинитель зла, вреда; болезнь; боль; жало. 




\noindent\textbf{Пакость} "--- гадость; нечистота; мерзость. 




\noindent\textbf{Палата} "--- дворец. 




\noindent\textbf{Иже в палате суть} "--- правительство. 




\noindent\textbf{Палестра} "--- место для соревнований. 




\noindent\textbf{Палителище} "--- сильный огонь. 




\noindent\textbf{Палительный} "--- сожигающий. 




\noindent\textbf{Палица} "--- трость; дубина; палка. 




\noindent\textbf{Паличник} "--- ликтор; телохранитель; полицейский пристав. 




\noindent\textbf{Памятозлобие} "--- злопамятство. 




\noindent\textbf{Панфирь} "--- пантера или лев. 




\noindent\textbf{Пара} "--- пар; мгла; дым. 




\noindent\textbf{Параекклесиарх} "--- кандиловжигатель; пономарь. 




\noindent\textbf{Параклис} "--- усердная молитва. 




\noindent\textbf{Параклит} "--- утешитель. 




\noindent\textbf{Паримия} "--- притча; чтения из Священного Писания на вечерне или царских часах. 




\noindent\textbf{Парити} "--- лететь; висеть в воздухе (подобно пару). 




\noindent\textbf{Парусия} "--- торжественное шествие; второе славное пришествие Господа нашего Иисуса Христа; торжественное архиерейское богослужение. 




\noindent\textbf{Пасомый} "--- пасущийся; находящийся в ведении пастыря. 




\noindent\textbf{Паствити} "--- пасти. 




\noindent\textbf{Паствуемый} "--- имеющий пастыря. 




\noindent\textbf{Пастися} "--- согрешить (особенно против седьмой заповеди). 




\noindent\textbf{Пастыреначальник} "--- начальник над пастырями. 




\noindent\textbf{Пастырь} "--- пастух. 




\noindent\textbf{Паучина} "--- паутина. 




\noindent\textbf{Паче} "--- лучше; больше. 




\noindent\textbf{Паче естества} "--- сверхъестественно. 




\noindent\textbf{Паче слова} "--- невыразимо. 




\noindent\textbf{Паче ума} "--- непостижимо. 




\noindent\textbf{Певк, певг} "--- хвойное дерево. 




\noindent\textbf{Педагогон} "--- детородный член. 




\noindent\textbf{Пекло} "--- горючая сера, смола; неперестающий огонь. 




\noindent\textbf{Пентикостарий} "--- название «Триоди цветной». 




\noindent\textbf{Пентикостия} "--- Пятидесятница. 




\noindent\textbf{Пеняжник} "--- меняла. 




\noindent\textbf{Пенязь} "--- мелкая монета. 




\noindent\textbf{Первее} "--- прежде; сперва; вначале; наперед. 




\noindent\textbf{Первоверховный} "--- первый из верховных. 




\noindent\textbf{Первовозлежание} "--- возлежание, восседание на первых, почетных местах в собраниях. 




\noindent\textbf{Первоначаток} "--- первородное животное или первоснятый плод. 




\noindent\textbf{Первостоятель} "--- первенствующий священнослужитель. 




\noindent\textbf{Пернатый} "--- имеющий перья. 




\noindent\textbf{Перси} (мн. ч.) "--- грудь; передняя часть тела. 




\noindent\textbf{Перст} "--- палец. 




\noindent\textbf{Перст возложити на уста} "--- замолчать. 




\noindent\textbf{Перстный} "--- земляной; сделанный из земли. 




\noindent\textbf{Перстосозданный} "--- сотворенный из персти. 




\noindent\textbf{Персть} "--- прах; земля; пыль. 




\noindent\textbf{Песнопети} "--- прославлять в песнях. 




\noindent\textbf{Песнословити} "--- см. Песнопети. 




\noindent\textbf{Пестовати} "--- нянчить; воспитывать. 




\noindent\textbf{Пестротный} "--- разноцветный; нарядный. 




\noindent\textbf{Пестун} "--- воспитатель; педагог; дядька. 




\noindent\textbf{Петель} "--- петух. 




\noindent\textbf{Петлоглашение} "--- пение петуха; раннее утро; время от 12 до 3 часов ночи, по народному счету времени у иудеев. 




\noindent\textbf{Печаловати} (ся) "--- сетовать, тужить; печалиться. 




\noindent\textbf{Печаловник} "--- опекун. 




\noindent\textbf{Печатствовати} "--- запечатывать; утверждать; сокрывать. 




\noindent\textbf{Печать} "--- перстень. 




\noindent\textbf{Пешешествовати} "--- идти пешком. 




\noindent\textbf{Пещися} "--- заботиться; иметь попечение. 




\noindent\textbf{Пивный} "--- то, что можно выпить. 




\noindent\textbf{Пиво} "--- питие; напиток. 




\noindent\textbf{Пиган} "--- рута, трава. 




\noindent\textbf{Пира} "--- сума; котомка. 




\noindent\textbf{Пирга} "--- башня; столп. 




\noindent\textbf{Писало} "--- остроконечная трость для писания на вощаной дощечке. 




\noindent\textbf{Писание ставильное} "--- ставленная грамота, даваемая архиереем новопосвященному пресвитеру или диакону. 




\noindent\textbf{Пискати} "--- играть на свирели. 




\noindent\textbf{Писмя} "--- буква; графический знак; буквальный смысл. 




\noindent\textbf{Пистикия} "--- чистый; беспримесный. 




\noindent\textbf{Питенный} "--- возлелеянный; выращенный в неге. 




\noindent\textbf{Питомый} "--- откормленный; дебелый. 




\noindent\textbf{Пищный} "--- содержащий обильную пищу; питательный. 




\noindent\textbf{Пиянство} "--- пьянство. 




\noindent\textbf{Плавы} "--- нивы. 




\noindent\textbf{Плавый} "--- зрелый; спелый, соломенного цвета. 




\noindent\textbf{Плат} "--- лоскут; заплатка. 




\noindent\textbf{Плащаница} "--- погребальные пелены; покрывало; полотно; плащ. 




\noindent\textbf{Плевел} "--- сорняк; негодная трава. 




\noindent\textbf{Плежити} "--- ползать на чреве; пресмыкаться. 




\noindent\textbf{Плежущий} "--- пресмыкающийся. 




\noindent\textbf{Пленица} "--- косичка; цепочка; ожерелье; корзина; цепь; оковы; узы. 




\noindent\textbf{Плескати} "--- бить в ладоши; аплодировать. 




\noindent\textbf{Плесна} "--- стопа; ступня. 




\noindent\textbf{Плесница} "--- обувь типа сандалий. 




\noindent\textbf{Плещи} "--- плечи. 




\noindent\textbf{Плещущий} "--- ударяющий в ладоши. 




\noindent\textbf{Плинфа} "--- кирпич. 




\noindent\textbf{Плинфоделание} "--- обжигание кирпичей. 




\noindent\textbf{Плищ} "--- крик; шум. 




\noindent\textbf{Плодствовати} "--- приносить плоды. 




\noindent\textbf{Плод устен} "--- слово; голос. 




\noindent\textbf{Плод чрева} "--- ребенок; дети. 




\noindent\textbf{Плотолюбие} "--- забота о теле. 




\noindent\textbf{Плотски} "--- плотью; телесно. 




\noindent\textbf{Плотский} "--- плотский; чувственный; телесный. 




\noindent\textbf{Плоть} "--- тело; человек; немощь или слабость человека; страсть. 




\noindent\textbf{Плюновение} "--- слюна. 




\noindent\textbf{Плясалище} "--- балаган. 




\noindent\textbf{Плясица, плясавица} "--- танцовщица; актриса. 




\noindent\textbf{Победительно} "--- торжественно; победоносно. 




\noindent\textbf{Поболети} "--- тужить; сожалеть. 




\noindent\textbf{Поборник} "--- защитник. 




\noindent\textbf{Повапленный} "--- покрашенный; побеленный. 




\noindent\textbf{Повергнути} "--- бросить; опрокинуть. 




\noindent\textbf{Повесть} "--- рассказ. 




\noindent\textbf{Повити} "--- принять роды или обвить пеленами. 




\noindent\textbf{Повои} "--- повязка; пелена. 




\noindent\textbf{Поглумитися} "--- рассуждать; размышлять; подумать. 




\noindent\textbf{Подвигнуться} "--- трепетать; двинуться. 




\noindent\textbf{Подвизати} "--- побуждать; поощрять. 




\noindent\textbf{Подвизатися} "--- совершать подвиги; трудиться. 




\noindent\textbf{Подвои} "--- косяки дверей. 




\noindent\textbf{Подникати} "--- наклоняться; нагибаться. 




\noindent\textbf{Подобитися} "--- напоминать что-либо. 




\noindent\textbf{Подобник} "--- подражатель. 




\noindent\textbf{Подобозрачен} "--- внешне похожий. 




\noindent\textbf{Подточилие} "--- сосуд для собирания выжатого сока. 




\noindent\textbf{Подъяремник} "--- находящийся под ярмом (например, осел). 




\noindent\textbf{Подъяремничий} "--- принадлежащий подъяремнику. 




\noindent\textbf{Поелику} "--- поскольку; потому что; так как; насколько. 




\noindent\textbf{Поелику аще} "--- сколько бы ни. 




\noindent\textbf{Пожрети} "--- принести в жертву. 




\noindent\textbf{Позде} "--- поздно; не рано. 




\noindent\textbf{Позобати} "--- склевать. 




\noindent\textbf{Позорище, позор} "--- многолюдное зрелище. 




\noindent\textbf{Поимати} "--- брать. 




\noindent\textbf{Полма} "--- пополам; надвое. 




\noindent\textbf{Польский} "--- полевой. 




\noindent\textbf{Помавати, поманути} "--- делать знаки; изъясняться без слов. 




\noindent\textbf{Помале} "--- вскоре; немного погодя. 




\noindent\textbf{Поматы} "--- скрижали на мантиях архиерейских. 




\noindent\textbf{Пометати} "--- мести; выметати; бросать. 




\noindent\textbf{Помизати} "--- мигать. 




\noindent\textbf{Поне} "--- хотя; по крайней мере; так как. 




\noindent\textbf{Понеже} "--- потому что; так как. 




\noindent\textbf{Понос, поношение} "--- позор; бесславие. 




\noindent\textbf{Понт} "--- море; большое озеро. 




\noindent\textbf{Понява, понявица} "--- полотенце. 




\noindent\textbf{Пооблещися} "--- надеть сверху другую одежду. 




\noindent\textbf{Поострити} "--- наточить. 




\noindent\textbf{Поползнутися} "--- поскользнуться; совратиться; соблазниться. 




\noindent\textbf{Поприще} "--- мера длины, равная тысяче шагов или суточному переходу. 




\noindent\textbf{Пореватися} "--- порываться; стремиться; двигаться. 




\noindent\textbf{Поречение} "--- обвинение; жалоба; упрек; попрек. 




\noindent\textbf{Порещи} "--- обвинить; укорить; осудить. 




\noindent\textbf{Поругание} "--- бесчестие; поношение; воспаление; язва. 




\noindent\textbf{Поругати} "--- обесчестить. 




\noindent\textbf{Порфира} "--- ткань темно-красного цвета; порфира, пурпурная одежда высокопоставленных особ. 




\noindent\textbf{Порча} "--- яд; отрава. 




\noindent\textbf{Поряду} "--- по порядку. 




\noindent\textbf{Поскору} "--- скоро; бегло; без пения (о службе). 




\noindent\textbf{Последи} "--- затем; в конце концов. 




\noindent\textbf{Последний} "--- остальной; конечный; окончательный. 




\noindent\textbf{Последование} "--- изложение молитвословий только одного рода, т.~е. или изменяемых, или неизменяемых. 




\noindent\textbf{Последовати} "--- исследовать; следовать. 




\noindent\textbf{Послушествовати} "--- свидетельствовать; давать показания. 




\noindent\textbf{Посолонь} "--- по-солнечному; как солнце; от востока на запад. 




\noindent\textbf{Поспешествовати} "--- помогать; пособлять. 




\noindent\textbf{Поспешник} "--- пособник; помощник. 




\noindent\textbf{Посреде} "--- посередине. 




\noindent\textbf{Поставление} "--- посвящение в сан. 




\noindent\textbf{Постриг} "--- пострижение в монашество. 




\noindent\textbf{Посупление} "--- наклонение головы в печали; печаль; грусть; сетование. 




\noindent\textbf{Посягати} "--- вступать в брак. 




\noindent\textbf{Потворник} "--- угодник; льстец; чародей; колдун. 




\noindent\textbf{Потворы} "--- чародейство; колдовство. 




\noindent\textbf{Поткнутися} "--- споткнуться. 




\noindent\textbf{Потреба} "--- потребность; необходимость; случай. 




\noindent\textbf{Потребник} "--- блин; лепешка. 




\noindent\textbf{Потщитися} "--- поспешить; постараться. 




\noindent\textbf{Поуститель} "--- подстрекатель. 




\noindent\textbf{Поущати} "--- поощрять; побуждать; наставлять; поучать. 




\noindent\textbf{Похотствовати} "--- иметь вожделение, похоть. 




\noindent\textbf{Почерпало} "--- бадья; кошель; ведро. 




\noindent\textbf{Починатися} "--- начинаться. 




\noindent\textbf{Почити} "--- успокоиться. 




\noindent\textbf{Пояти} "--- взять. 




\noindent\textbf{Правый} "--- прямой; истинный; правильный; праведный. 




\noindent\textbf{Праг} "--- порог. 

\noindent\textbf{Празднословие} "--- пустой; вздорный разговор. 



\noindent\textbf{Праздный, празден} "--- беспредельный; ленивый; пустой; незанятый. 




\noindent\textbf{Прати} "--- попирать; давить. 




\noindent\textbf{Превзятися} "--- превознестись; возгордиться. 




\noindent\textbf{Превитати} "--- странствовать. 




\noindent\textbf{Превозвышенное око} "--- высокоумие; гордость. 




\noindent\textbf{Превратити} "--- изменить; поворотить; разрушить. 




\noindent\textbf{Превременный} "--- предвечный, существовавший до начала времени. 




\noindent\textbf{Предвзыграти} "--- предвозвестить радостью. 




\noindent\textbf{Предвозгласити} "--- начать пение; предвозвестить. 




\noindent\textbf{Предградие} "--- пригород; оплот; защита; ограда. 




\noindent\textbf{Преддворие} "--- передний, внешний двор в восточном доме. 




\noindent\textbf{Предзаклатися} "--- прежде других вкусить смерть, принести себя в жертву. 




\noindent\textbf{Преди} "--- впереди. 




\noindent\textbf{Предитещи} "--- бежать впереди. 




\noindent\textbf{Предложение} "--- жертвенник; то место в алтаре, где стоит жертвенник и хранятся священные сосуды. 




\noindent\textbf{Предначинательный псалом} "--- название псалма 103, поскольку им начинается вечерня. 




\noindent\textbf{Предний} "--- первый; вящий; изящный; старший. 




\noindent\textbf{Предпряда} "--- ткань темно-красного цвета; порфира, пурпурная одежда высокопоставленных особ. 




\noindent\textbf{Предстательство} "--- ходатайство; заступничество; усердная молитва. 




\noindent\textbf{Предстолпие} "--- укрепление. 




\noindent\textbf{Предстоятель} "--- настоятель. 




\noindent\textbf{Предтеча, предитеча} "--- идущий или бегущий впереди. 




\noindent\textbf{Предуставити} "--- предназначить. 




\noindent\textbf{Предусрести} "--- встретить заранее. 




\noindent\textbf{Предуведети} "--- предвидеть; знать заранее. 




\noindent\textbf{Предъявленне} "--- предображая. 




\noindent\textbf{Презорливый} "--- гордый; надменный. 




\noindent\textbf{Преизбыточествовати} "--- быть довольну; жить в изобилии. 




\noindent\textbf{Преизлиха} "--- сильно; очень; жестоко. 




\noindent\textbf{Преимение} "--- преимущество; превосходство. 




\noindent\textbf{Преиспещренный} "--- разукрашенный. 




\noindent\textbf{Преисподний} "--- самый низкий. 




\noindent\textbf{Преисподняя} "--- место нахождения душ умерших до освобождения их Господом Иисусом Христом; место вечного мучения грешников; жилище диавола. 




\noindent\textbf{Преитие} "--- превосхождение. 




\noindent\textbf{Прелагати пределы} "--- портить межи; нарушать границы. 




\noindent\textbf{Прелесть} "--- обман. 




\noindent\textbf{Прелюбы} "--- прелюбодейство. 




\noindent\textbf{Премудрость} "--- высшее знание; мудрость. 




\noindent\textbf{Преначальный} "--- доначальный; превышающий всякое начало. 




\noindent\textbf{Преогорчити} "--- противиться; быть непокорным; упрямиться. 




\noindent\textbf{Преодеян} "--- обильно украшен. 




\noindent\textbf{Преоруженный} "--- слишком вооруженный; гордый. 




\noindent\textbf{Препирати} "--- опровергать; отражать; отбивать; одолевать; увещевать. 




\noindent\textbf{Преподобие} "--- святость. 




\noindent\textbf{Преполовение} "--- половина; середина. 




\noindent\textbf{Преполовити} "--- переполовинить; разделить пополам; пройти половину пути. 




\noindent\textbf{Препона} "--- препятствие. 




\noindent\textbf{Препоясатися} "--- подпоясаться; приготовиться к чему-либо. 




\noindent\textbf{Препретельный} "--- спорный; убедительный. 




\noindent\textbf{Препростый} "--- неученый; невежда. 




\noindent\textbf{Препяти} "--- остановить. 




\noindent\textbf{Пререкаемый} "--- спорный. 




\noindent\textbf{Пререкати} "--- прекословить; говорить наперекор; перечить. 




\noindent\textbf{Пресецающий} "--- пересекающий; перерубающий. 




\noindent\textbf{Преслушание} "--- неповиновение. 




\noindent\textbf{Пресмыкаться} "--- ползти по земле. 




\noindent\textbf{Преспевати} "--- иметь успех. 




\noindent\textbf{Преставити} "--- переставить; переместить; переселить в вечность. 




\noindent\textbf{Престоли} "--- один из чинов ангельских. 




\noindent\textbf{Пресущественный} "--- предвечный; исконный. 




\noindent\textbf{Пресущный} "--- сверхъестественный. 




\noindent\textbf{Претися} "--- спорить; тягаться. 




\noindent\textbf{Претити} "--- запрещать; скорбеть; смущаться. 




\noindent\textbf{Преткновение} "--- помеха; соблазн; задержка; остановка. 




\noindent\textbf{Претор} "--- претория, резиденция представителя римской власти в Иерусалиме. 




\noindent\textbf{Претыкание} "--- помеха; соблазн; задержка; остановка. 




\noindent\textbf{Прещати} "--- грозить; устрашать. 




\noindent\textbf{Прещение} "--- угроза; страх; запрет. 




\noindent\textbf{Прибежище} "--- убежище; приют; покров; спасение. 




\noindent\textbf{Приведение} "--- доступ. 




\noindent\textbf{Привещевати} "--- приветствовать. 




\noindent\textbf{Привлещи} "--- притащить; позвать; призвать. 




\noindent\textbf{Привременный} "--- временный; непостоянный. 




\noindent\textbf{Придевати} "--- прицеплять; приближаться; подносить. 




\noindent\textbf{Придел} "--- небольшая церковь, пристроенная к главному храму. 




\noindent\textbf{Приделати} "--- прирастить; увеличить; принести. 




\noindent\textbf{Придеяти} "--- подносить; приносить. 




\noindent\textbf{Призрети} "--- милостиво посмотреть; принять; приютить. 




\noindent\textbf{Прииждивати} "--- расходовать; издерживать. 




\noindent\textbf{Приискренне} "--- точно так же; равно; точь-в-точь. 




\noindent\textbf{Прикровение} "--- прикрытие; предлог; выдуманная причина для сокрытия чего-либо. 




\noindent\textbf{Прикуп} "--- барыш; прибыль. 




\noindent\textbf{Прикупование} "--- купечество; торговля. 




\noindent\textbf{Прилог} "--- приложение; желание сделать зло; злоба; клевета. 




\noindent\textbf{Приложение} "--- заплатка; лоскут. 




\noindent\textbf{Приметати} "--- прибрасывать; отдавать; уступать. 




\noindent\textbf{Приметатися} "--- припадать; отдаваться; лежать у порога. 




\noindent\textbf{Примешатися} "--- присоединяться. 




\noindent\textbf{Приникнути} "--- пригнуться; наклониться; припасть; проникнуть. 




\noindent\textbf{Приобряща} "--- польза; плод; корысть. 




\noindent\textbf{Приразитися} "--- напасть; удариться. 




\noindent\textbf{Приревание} "--- устремление. 




\noindent\textbf{Приристати} "--- подбегать. 




\noindent\textbf{Пририщущий} "--- подбегающий. 




\noindent\textbf{Присвянути} "--- завянуть; засохнуть. 




\noindent\textbf{Приседение} "--- угнетение; окружение. 




\noindent\textbf{Приседети} "--- находиться около чего-либо; замышлять зло; нападать. 




\noindent\textbf{Присно} "--- непрестанно; всегда. 




\noindent\textbf{Присноживотный} "--- всегда живущий. 




\noindent\textbf{Присносущий} "--- вечный; всегдашний. 




\noindent\textbf{Присносущный} "--- всегда существующий. 




\noindent\textbf{Приснотекущий} "--- неиссякаемый. 




\noindent\textbf{Присный} "--- родной; близкий. 




\noindent\textbf{Приставление} "--- заплатка; назначение; управление; присмотр. 




\noindent\textbf{Пристанище} "--- приют; убежище; пристань. 




\noindent\textbf{Пристати} "--- прибегнуть; подбежать. 




\noindent\textbf{Пристрашен} "--- испуган. 




\noindent\textbf{Притвор} "--- вход в храм. 




\noindent\textbf{Прителный} "--- спорный. 




\noindent\textbf{Притча} "--- иносказание; загадка. 




\noindent\textbf{Причаститися} "--- стать участником. 




\noindent\textbf{Причастник} "--- участник. 




\noindent\textbf{Пришлец} "--- приезжий; пришелец. 




\noindent\textbf{Приятилище} "--- вместилище; поместилище; хранилище. 




\noindent\textbf{Пробавити} "--- продолжить; протянуть. 




\noindent\textbf{Продерзивый} "--- дерзкий. 




\noindent\textbf{Прозябение} "--- произрастание; росток. 




\noindent\textbf{Прозябнути} "--- расцвести; вырасти; произрастить. 




\noindent\textbf{Произникнути} "--- произойти; вырасти. 




\noindent\textbf{Пролитися стопам} "--- поскользнуться; иносказательно "--- согрешить. 




\noindent\textbf{Пронарещи} "--- предсказать; предназначить. 




\noindent\textbf{Проникнути} "--- вырасти; процвесть. 




\noindent\textbf{Проничение} "--- племя; род; стебель; росток. 




\noindent\textbf{Проповедати} "--- учить; провозглашать; проповедовать. 




\noindent\textbf{Прорещи} "--- предсказать. 




\noindent\textbf{Пророкованный} "--- предсказанный; предвозвещенный. 




\noindent\textbf{Пророковещательный} "--- говоримый пророком. 




\noindent\textbf{Проручествовати} "--- посвящать; рукополагать. 




\noindent\textbf{Просаждатися} "--- разрываться. 




\noindent\textbf{Просветительный} "--- светлый; просвещающий. 




\noindent\textbf{Просветити лице} "--- весело или милостивно взглянуть. 




\noindent\textbf{Проскомисати} "--- совершать проскомидию. 




\noindent\textbf{Прослутие} "--- притча; пословица; осмеяние. 




\noindent\textbf{Простый} "--- стоящий прямо; прямой; чистый; несмешанный. 




\noindent\textbf{Простыня} "--- сострадание. 




\noindent\textbf{Просядати} "--- разрываться; разваливаться; трескаться. 




\noindent\textbf{Протерзатися} "--- прорываться. 




\noindent\textbf{Противозрети} "--- смотреть прямо. 




\noindent\textbf{Противу, прямо} "--- против; напротив. 




\noindent\textbf{Проуведети} "--- узнать заранее; предвидеть. 




\noindent\textbf{Проявленне} "--- явно. 




\noindent\textbf{Пругло} "--- силок; петля; сеть. 




\noindent\textbf{Прудный} "--- неровный; каменистый. 




\noindent\textbf{Пружатися} "--- сопротивляться (отсюда "--- пружина); биться в припадке. 




\noindent\textbf{Пружие} "--- пища Иоанна Крестителя; по мнению одних "--- род съедобной саранчи, или кузнечиков; по мнению других "--- какое-то растение. 




\noindent\textbf{Пря} "--- спор; тяжба; беспорядок. 




\noindent\textbf{Пряжмо} "--- жареная пища. 




\noindent\textbf{Прямный} "--- находящийся напротив. 




\noindent\textbf{Пустити} "--- отпустить; развестись. 




\noindent\textbf{Пустыня} "--- уединенное, малообитаемое место. 




\noindent\textbf{Пустыня} "--- монастырь, расположенный вдалеке от населенных мест. 




\noindent\textbf{Путесотворити} "--- сохранять в пути; проложить дорогу. 




\noindent\textbf{Путы} "--- узы; кандалы; цепи; оковы. 




\noindent\textbf{Пучина} "--- водоворот; море. 




\noindent\textbf{Пучинородный} "--- морской; родившийся в море. 




\noindent\textbf{Пущеница} "--- разведенная с мужем женщина. 




\noindent\textbf{Птицеволхвование} "--- суеверие, состоящее в гадании по полету птиц или по их внутренностям. 




\noindent\textbf{Пядь, пядень} "--- мера длины, равная трем дланям, а каждая длань равна четырем перстам, а перст равен четырем граням или зернам. 




\noindent\textbf{Пясть} "--- кулак. 




\noindent\textbf{Пяток} "--- пятница. 




\bukvaending

\bukva{Р}





\noindent\textbf{Рабий} "--- рабский. 




\noindent\textbf{Работа} "--- рабство. 




\noindent\textbf{Работен} "--- покорен; порабощен. 




\noindent\textbf{Равви, раввуни} "--- учитель. 




\noindent\textbf{Равноангельно} "--- подобно Ангелам. 




\noindent\textbf{Равноапостольный} "--- сравниваемый с апостолами. 




\noindent\textbf{Равнодушный, равнодушевный} "--- единодушный; имеющий одинаковое усердие. 




\noindent\textbf{Равночестный} "--- достойный равного почитания. 




\noindent\textbf{Радованный} "--- радостный. 




\noindent\textbf{Радоватися} "--- радоваться; наслаждаться. 




\noindent\textbf{Радощи} "--- радости (мн. ч.); веселье. 




\noindent\textbf{Радуйся} "--- здравствуй; прощай. 




\noindent\textbf{Раждежение} "--- горение; воспламенение. 




\noindent\textbf{Разботети} "--- растолстеть; разбухнуть. 




\noindent\textbf{Разве} "--- кроме. 




\noindent\textbf{Развет} "--- мятеж; заговор. 




\noindent\textbf{Разврат} "--- волнение; возмущение. 




\noindent\textbf{Разгбенный} "--- разогнутый. 




\noindent\textbf{Разгнутие} "--- разгибание; раскрытие книги. 




\noindent\textbf{Раздолие} "--- долина. 




\noindent\textbf{Разжизати} "--- разжигать; раскалять; расплавлять. 




\noindent\textbf{Размыслити} "--- усомниться; задуматься; остановиться. 




\noindent\textbf{Разнство} "--- различие. 




\noindent\textbf{Разрешити} "--- развязать; освободить. 




\noindent\textbf{Разслабленный} "--- паралитик. 




\noindent\textbf{Разум} "--- ум; познание; разумение. 




\noindent\textbf{Разумети телом} "--- почувствовать. 




\noindent\textbf{Рака} "--- евр. дурак; пустой человек. 




\noindent\textbf{Рака} "--- гробница; ковчег с мощами святого угодника Божия. 




\noindent\textbf{Рало} "--- соха; плуг. 




\noindent\textbf{Рамо} "--- плечо. 




\noindent\textbf{Рамена} "--- плечи. 




\noindent\textbf{Расплощатися} "--- развертываться. 




\noindent\textbf{Распростирати} "--- расстеливать; разворачивать. 




\noindent\textbf{Распудити} "--- распугать; разогнать; рассеять. 




\noindent\textbf{Распутие} "--- перекресток. 




\noindent\textbf{Раст} "--- росток. 




\noindent\textbf{Растерзати} "--- разорвать. 




\noindent\textbf{Растнити} "--- рассечь. 




\noindent\textbf{Расточати} "--- рассеивать; рассыпать; проматывать; беспутно проживать. 




\noindent\textbf{Растренный} "--- перепиленный. 




\noindent\textbf{Расчинити} "--- расположить по порядку. 




\noindent\textbf{Ратай} "--- воин. 




\noindent\textbf{Ратовати} "--- воевать; сражаться; отстаивать. 




\noindent\textbf{Ратовище} "--- древко копья. 




\noindent\textbf{Рать} "--- война; воинство. 




\noindent\textbf{Рачитель} "--- попечитель; любитель. 




\noindent\textbf{Рачительный} "--- заботливый; достойный заботы. 




\noindent\textbf{Рвение, ревность} "--- ярость; страстное желание; страсть. 




\noindent\textbf{Ребра северова} "--- северный склон горы Сион. 




\noindent\textbf{Ревновати} "--- завидовать. 




\noindent\textbf{Рек} "--- ты, он сказал. 




\noindent\textbf{Рекла} "--- сказала. 




\noindent\textbf{Рекомый} "--- прозываемый. 




\noindent\textbf{Рекох} "--- я сказал. 




\noindent\textbf{Репие} "--- репейник; колючее растение. 




\noindent\textbf{Ресно} "--- ресницы; глаз. 




\noindent\textbf{Реснота} "--- действительность; истина. 




\noindent\textbf{Реть} "--- ссора; спор. 




\noindent\textbf{Рещи} "--- сказать; говорить. 




\noindent\textbf{Реяти} "--- отталкивать; отбрасывать. 




\noindent\textbf{Риза} "--- одежда; священное облачение. 




\noindent\textbf{Ризница} "--- помещение для сохранения риз. 




\noindent\textbf{Ризничий} "--- начальник над ризницей; хранитель церковной утвари. 




\noindent\textbf{Ристалище} "--- стадион; цирк. 




\noindent\textbf{Ристати} "--- рыскать; бегать. 




\noindent\textbf{Рог} "--- рок; иносказательно: сила; власть; защита. 




\noindent\textbf{Род} "--- происхождение; племя; поколение. 




\noindent\textbf{Родостама} "--- розовая вода, которой по обычаю в праздник Воздвижения производят омовение Честнаго и Животворящего Креста Господня при его воздвизании. 




\noindent\textbf{Рожаный} "--- роговой; напоминающий рог. 




\noindent\textbf{Рожец} "--- сладкий стручок. 




\noindent\textbf{Розга} "--- молодая ветвь; побег; отпрыск. 




\noindent\textbf{Росодательный} "--- росоносный; дающий росу. 




\noindent\textbf{Рота} "--- божба; клятва. 




\noindent\textbf{Ротитель} "--- клятвопреступник. 




\noindent\textbf{Ротитися} "--- клясться; божиться. 




\noindent\textbf{Ругатися} "--- насмехаться. 




\noindent\textbf{Рукописание} "--- список; письмо; письменный договор; свиток; расписка; обязательство. 




\noindent\textbf{Рукоять} "--- горсть; охапка. 




\noindent\textbf{Руно} "--- шерсть; овчина. 




\noindent\textbf{Ручка} "--- сосуд. 




\noindent\textbf{Рцем} "--- скажем (повел.наклонение). 




\noindent\textbf{Рцы} "--- скажи. 




\noindent\textbf{Рыбарь} "--- рыбак. 




\noindent\textbf{Рясно} "--- ожерелье; подвески. 




\bukvaending

\bukva{С}





\noindent\textbf{Самвик} "--- музыкальный инструмент. 




\noindent\textbf{Самовидец} "--- очевидец. 




\noindent\textbf{Самогласная стихира} "--- имеющая свой особый распев. 




\noindent\textbf{Самоохотие} "--- по собственному желанию. 




\noindent\textbf{Самоподобен} "--- стихира, имеющая свой особый распев. 




\noindent\textbf{Самочиние} "--- бесчиние; беспорядок. 




\noindent\textbf{Сата} "--- мера сыпучих тел. 




\noindent\textbf{Сбодати} "--- пронзить; заколоть. 




\noindent\textbf{Свара} "--- ссора; брань. 




\noindent\textbf{Сваритися} "--- ссориться. 




\noindent\textbf{Сведети} "--- ведать; знать. 




\noindent\textbf{Светильничное} "--- начало вечерни. 




\noindent\textbf{Светлопозлащен} "--- великолепно украшен. 




\noindent\textbf{Светлость} "--- светящаяся красота. 




\noindent\textbf{Светозарный} "--- озаряющий светом. 




\noindent\textbf{Светолитие} "--- сияние. 




\noindent\textbf{Светоначальник} "--- создатель светил. 




\noindent\textbf{Светоносец} "--- несущий свет. 




\noindent\textbf{Свечеряти} "--- совместно с кем-либо участвовать в пиру. 




\noindent\textbf{Свидение} "--- наставление; приказание. 




\noindent\textbf{Свирепоустие} "--- необузданность языка. 




\noindent\textbf{Свиток} "--- сверток; рукопись, намотанная на палочку. 




\noindent\textbf{Связание злата} "--- впрядение золотых нитей. 




\noindent\textbf{Связень} "--- узник; невольник. 




\noindent\textbf{Святилище} "--- алтарь; храм. 




\noindent\textbf{Святитель} "--- архиерей; епископ. 




\noindent\textbf{Святотатство} "--- похищение священных вещей. 




\noindent\textbf{Святцы} "--- месяцеслов (книга, содержащая имена святых, расположенных по дням года); икона «Всех святых». 




\noindent\textbf{Священнотаинник} "--- посвященный в Божественные тайны. 




\noindent\textbf{Се} "--- вот. 




\noindent\textbf{Седмерицею} "--- семикратно. 




\noindent\textbf{Седмица} "--- семь дней, которые в современном языке принято называть «неделя». 




\noindent\textbf{Седмичный} "--- относящийся к любому из дней седмицы, кроме Недели (воскресного дня); будничный. 




\noindent\textbf{Секира} "--- топор. 




\noindent\textbf{Секраты} "--- недавно; только что. 




\noindent\textbf{Секратый} "--- свежий; новый. 




\noindent\textbf{Селный} "--- полевой; дикий. 




\noindent\textbf{Село} "--- поле. 




\noindent\textbf{Семидал} "--- мелкая пшеничная мука; крупчатка. 




\noindent\textbf{Семо} "--- сюда. 




\noindent\textbf{Семя} "--- семя; потомки; род. 




\noindent\textbf{Сеннописание} "--- неясное изображение. 




\noindent\textbf{Сень} "--- тень; покров над престолом. 




\noindent\textbf{Септемврий} "--- сентябрь. 




\noindent\textbf{Серповидец} "--- наименование святого пророка Захарии. 




\noindent\textbf{Серядь} "--- монашеское рукоделие; пряжа. 




\noindent\textbf{Сеть} "--- западня. 




\noindent\textbf{Сечиво} "--- топор. 




\noindent\textbf{Сигклит} (читается «синклит») "--- собрание, сенат. 




\noindent\textbf{Сиесть} "--- то есть. 




\noindent\textbf{Сикарий} "--- убийца; разбойник. 




\noindent\textbf{Сикелия} "--- о. Сицилия. 




\noindent\textbf{Сикер} "--- хмельной напиток, изготовленный не из винограда. 




\noindent\textbf{Силы} "--- название одного из чинов ангельских; иногда значит чудеса. 




\noindent\textbf{Синаксарий} "--- сокращенное изложение житий святых или праздников. 




\noindent\textbf{Синедрион} "--- верховное судилище у иудеев. 




\noindent\textbf{Синфрог} "--- сопрестолие, т.~е. скамьи по обе стороны горнего места для сидения сослужащих архиерею священников. 




\noindent\textbf{Сиречь} "--- то есть; именно. 




\noindent\textbf{Сирини} "--- (в Ис. 13, 21) "--- страусы; сирены. 




\noindent\textbf{Сирт} "--- отмель; мель. 




\noindent\textbf{Сирый} "--- сиротливый; одинокий; беспомощный; бедный. 




\noindent\textbf{Сице} "--- так; таким образом. 




\noindent\textbf{Сицевый} "--- такой; таковой. 




\noindent\textbf{Скверна} "--- нечистота; грязь; порок. 




\noindent\textbf{Сквозе} "--- сквозь; через. 




\noindent\textbf{Скимен} "--- молодой лев; львенок. 




\noindent\textbf{Скиния} "--- палатка; шатер. 




\noindent\textbf{Скинотворец} "--- делатель палаток. 




\noindent\textbf{Скит} "--- маленький монастырь. 




\noindent\textbf{Склабитися} "--- улыбаться, усмехаться. 




\noindent\textbf{Скнипа} "--- вошь. 




\noindent\textbf{Сковник} "--- соучастник; сообщник. 




\noindent\textbf{Скоктание} "--- щекотание; подстрекательство. 




\noindent\textbf{Скопчий} "--- скопческий. 




\noindent\textbf{Скоротеча} "--- скороход; гонец. 




\noindent\textbf{Скорпия} "--- скорпион. 




\noindent\textbf{Скрания} "--- висок. 




\noindent\textbf{Скрижаль} "--- доска; таблица. 




\noindent\textbf{Скудель} "--- глина; то, что сделано из глины; кувшин; черепица. 




\noindent\textbf{Скудельник} "--- горшечник. 




\noindent\textbf{Скудный} "--- бедный; тощий. 




\noindent\textbf{Скураты} "--- маски; личины. 




\noindent\textbf{Славник} "--- молитвословие, положенное по уставу после «Славы». 




\noindent\textbf{Славословие} "--- прославление. 




\noindent\textbf{Сладковонный} "--- благоуханный. 




\noindent\textbf{Сладкогласие} "--- стройное пение. 




\noindent\textbf{Сладкопение} "--- тихое, умильное пение. 




\noindent\textbf{Слана} "--- гололедица; мороз; ледник; замерзший иней. 




\noindent\textbf{Сланость} "--- соленая морская вода; солончак, т.~е. сухая, пропитанная солью земля; гололед. 




\noindent\textbf{Сластотворный} "--- обольщающий плотскими удовольствиями. 




\noindent\textbf{Сликовствовати} "--- совместно играть; веселиться. 




\noindent\textbf{Словесный} "--- разумный. 




\noindent\textbf{Словоположение} "--- договор; условия. 




\noindent\textbf{Сложитися} "--- уговориться; определить. 




\noindent\textbf{Слота} "--- ненастье; дурная погода. 




\noindent\textbf{Слух} "--- слава; народная молва. 




\noindent\textbf{Слякий, слукий} "--- согнутый; скорченный; горбатый. 




\noindent\textbf{Сляцати} "--- сгибать; горбить. 




\noindent\textbf{Смарагд} "--- изумруд. 




\noindent\textbf{Смежити} "--- сблизить; соединять края, межи; закрывать. 




\noindent\textbf{Смерчие} "--- кедр. 




\noindent\textbf{Смеситися} "--- перемещаться; совокупиться плотски. 




\noindent\textbf{Смиряти} "--- унижать. 




\noindent\textbf{Смоква} "--- плод фигового дерева. 




\noindent\textbf{Смотрение} "--- промысел; попечение; забота. 




\noindent\textbf{Смясти} "--- привести в смятение; встревожить. 




\noindent\textbf{Снабдевати} "--- сберегать; сохранять. 




\noindent\textbf{Снедати} "--- съедать; разорять; сокрушать. 




\noindent\textbf{Снедь} "--- пища. 




\noindent\textbf{Сниматися} "--- сходиться; собираться. 




\noindent\textbf{Снисхождение} "--- снисшествие. 




\noindent\textbf{Снитися} "--- вступить в брак; сойтись. 




\noindent\textbf{Снуждею} "--- поневоле; насильно; по принуждению. 




\noindent\textbf{Собесити} "--- вместе повесить. 




\noindent\textbf{Соблюдение} "--- точное исполнение; темница. 




\noindent\textbf{С соблюдением} "--- видимым образом; явно. 




\noindent\textbf{Совет} "--- совет; решение; определение. 




\noindent\textbf{Советный} "--- рассудительный. 




\noindent\textbf{Совлачити} "--- разоблачить; снять. 




\noindent\textbf{Совлечение} "--- раздевание. 




\noindent\textbf{Совлещися} "--- раздеться. 




\noindent\textbf{Совозвести} "--- возвести вместе с собой. 




\noindent\textbf{Совоздыхати} "--- печалиться вместе. 




\noindent\textbf{Совопрошатися} "--- беседовать; состязаться в споре. 




\noindent\textbf{Совоспитанный} "--- воспитанный совместно с кем-либо. 




\noindent\textbf{Согласно} "--- единодушно. 




\noindent\textbf{Соглядатай} "--- разведчик; шпион. 




\noindent\textbf{Соглядати} "--- рассматривать; наблюдать; разведывать. 




\noindent\textbf{Сограждати} "--- сооружать; строить. 




\noindent\textbf{Содеватися} "--- сделаться. 




\noindent\textbf{Соделование} "--- дело; превращение. 




\noindent\textbf{Содетель} "--- творец. 




\noindent\textbf{Содетельный} "--- творческий. 




\noindent\textbf{Сокровище} "--- потаенное место; задняя комната; хранилище; клад; драгоценность; погреб. 




\noindent\textbf{Сокровиществовати} "--- собирать сокровища. 




\noindent\textbf{Сокрушение} "--- уничтожение. 




\noindent\textbf{Сокрушение сердца} "--- раскаяние. 




\noindent\textbf{Солило} "--- солонка; чаша; блюдо. 




\noindent\textbf{Соние} "--- сон; сновидение. 




\noindent\textbf{Сонм} "--- собрание; множество. 




\noindent\textbf{Сонмище} "--- синагога. 




\noindent\textbf{Сопель} "--- свирель, дудка. 




\noindent\textbf{Сопети} "--- играть на дудке. 




\noindent\textbf{Сопец} "--- сопельщик-музыкант, играющий на сопели, флейте (при похоронах у иудеев). 




\noindent\textbf{Сопретися} "--- ссориться; тягаться. 




\noindent\textbf{Соприбывати} "--- увеличиваться. 




\noindent\textbf{Соприсносущный} "--- совместно существующий в вечности. 




\noindent\textbf{Сопрягати} "--- соединять браком. 




\noindent\textbf{Сорокоустие} "--- пшеница, вино, фимиам, свечи и пр., приносимые в церковь на 40 дней поминовения усопших христиан. 




\noindent\textbf{Соскание} "--- шнурок; веревочка. 




\noindent\textbf{Сосканый} "--- витой; крученый 




\noindent\textbf{Соскутовати} "--- спеленать; окутать. 




\noindent\textbf{Состреляти} "--- поразить стрелой. 




\noindent\textbf{Сосудохранительница} "--- помещение для сохранения церковной утвари. 




\noindent\textbf{Сосуды смертные} "--- орудия смерти. 




\noindent\textbf{Сосцы} "--- иногда иносказательно так называются водные источники. 




\noindent\textbf{Сотница} "--- сотня; пение «Господи, помилуй» сто раз при воздвизании Честнаго Креста Господня. 




\noindent\textbf{Сотово тело} "--- мед. 




\noindent\textbf{Соуз} "--- союз; связь. 




\noindent\textbf{Сочетаватися} "--- вступать в союз, в брак. 




\noindent\textbf{Сочиво} "--- чечевица; вареная пшеница с медом. 




\noindent\textbf{Сочинение} "--- составление; собрание. 




\noindent\textbf{Спекулатор} "--- телохранитель. 




\noindent\textbf{Спира} "--- отряд; рота; полк. 




\noindent\textbf{Сплавати} "--- сопутствовать в плавании. 




\noindent\textbf{Споболети} "--- вместе печалиться; тужить. 




\noindent\textbf{Споборать} "--- вместе воевать. 




\noindent\textbf{Спод} "--- ряд; куча; отдел. 




\noindent\textbf{Спона} "--- препятствие. 




\noindent\textbf{Спослушествовати} "--- свидетельствовать; подтверждать. 




\noindent\textbf{Споспешник} "--- помощник. 




\noindent\textbf{Спостник} "--- вместе постящийся. 




\noindent\textbf{Спострадати} "--- вместе страдать. 




\noindent\textbf{Споющий} "--- вместе или одновременно поющий. 




\noindent\textbf{Спротяженный} "--- продолжительный. 




\noindent\textbf{Спуд} "--- сосуд; ведерко; мера сыпучих тел; покрышка; плита. 




\noindent\textbf{Спяти} "--- низвергнуть; опрокинуть 




\noindent\textbf{Срамословие} "--- сквернословие. 




\noindent\textbf{Срасленный} "--- сросшийся. 




\noindent\textbf{Срачица} "--- сорочка; рубаха. 




\noindent\textbf{Сребреник} "--- серебряная монета. 




\noindent\textbf{Сребропозлащенный} "--- позолоченный по серебру. 




\noindent\textbf{Средоградие} "--- перегородка; простенок; преграда. 




\noindent\textbf{Средостение} "--- перегородка; средняя стена. 




\noindent\textbf{Сретение} "--- встреча. 




\noindent\textbf{Сристатися} "--- стекаться; сбегаться. 




\noindent\textbf{Срящь} "--- неприятная встреча; нападение; зараза; мор; гаданье; приметы. 




\noindent\textbf{Ставленник} "--- человек, подготовляемый к посвящению в духовный сан. 




\noindent\textbf{Стадия} "--- мера длины, равная 100--125 шагам. 




\noindent\textbf{Стаинник} "--- сопричастный с кем-либо одной тайне. 




\noindent\textbf{Стакти} "--- благовонный сок. 




\noindent\textbf{Стамна} "--- сосуд; ведерко; кувшин. 




\noindent\textbf{Старей} "--- начальник; старший. 




\noindent\textbf{Статир} "--- серебряная или золотая монета. 




\noindent\textbf{Статия} "--- глава; подраздел. 




\noindent\textbf{Стегно} "--- верхняя половина ноги; бедро; ляжка. 




\noindent\textbf{Стезя} "--- тропинка; дорожка. 




\noindent\textbf{Стень, сень} "--- тень; отражение; образ. 




\noindent\textbf{Степени} "--- ступени. 




\noindent\textbf{Стерти} "--- стереть; разрушить. 




\noindent\textbf{Стихира} "--- песнопение. 




\noindent\textbf{Стих началу} "--- первый возглас священника при богослужении общественном или частном. 




\noindent\textbf{Стихологисати} "--- петь избранные стихи из Псалтири при богослужении. 




\noindent\textbf{Стихология} "--- чтение или пение Псалтири. 




\noindent\textbf{Стихословити} "--- петь избранные стихи из Псалтири при богослужении. 




\noindent\textbf{Стицатися} "--- стекаться; сходиться. 




\noindent\textbf{Сткляница} "--- стакан. 




\noindent\textbf{Сткляный} "--- стеклянный. 




\noindent\textbf{Столп} "--- башня; крепость. 




\noindent\textbf{Столпостена} "--- башня; крепость. 




\noindent\textbf{Стомах} "--- желудок. 




\noindent\textbf{Стопа} "--- ступня. 




\noindent\textbf{Сторицею} "--- во сто раз. 




\noindent\textbf{Стогна} "--- улица, дорога. 




\noindent\textbf{Страдальчество} "--- мученичество. 




\noindent\textbf{Стража} "--- караул; охрана; мера времени для ночи. 




\noindent\textbf{Страннолепный} "--- необычный. 




\noindent\textbf{Странный} "--- сторонний; чужой; прохожий; необычайный. 




\noindent\textbf{Странь} "--- напротив; против. 




\noindent\textbf{Страсть} "--- страдание; страсть; душевный порыв. 




\noindent\textbf{Стратиг} "--- военачальник. 




\noindent\textbf{Стратилат} "--- военачальник; воевода. 




\noindent\textbf{Страхование} "--- угроза; страх; ужас. 




\noindent\textbf{Стрекало} "--- спица; палочка с колючкой для управления скотом. 




\noindent\textbf{Стрещи} "--- стеречь. 




\noindent\textbf{Стрищи} "--- стричь; подстригать. 




\noindent\textbf{Стропотный} "--- кривой; извилистый; строптивый; упрямый; злой. 




\noindent\textbf{Стрыти} "--- стереть; сокрушить. 




\noindent\textbf{Студ} "--- стыд; срам. 




\noindent\textbf{Студенец} "--- колодец; родник; источник. 




\noindent\textbf{Студень} "--- холод; стужа; мороз. 




\noindent\textbf{Стужаемый} "--- беспокоимый. 




\noindent\textbf{Стужание} "--- стеснение; гонение; досаждение. 




\noindent\textbf{Стужати} "--- докучать; надоедать; теснить. 




\noindent\textbf{Стужатися} "--- скорбеть; печалиться. 




\noindent\textbf{Стужный} "--- тревожный. 




\noindent\textbf{Стягнути} "--- обвязать; собрать; исцелить. 




\noindent\textbf{Стязатися} "--- спорить; препираться. 




\noindent\textbf{Сугубый} "--- двойной; удвоенный; увеличенный; усиленный. 




\noindent\textbf{Сударь} "--- плат; пелена. 




\noindent\textbf{Судище} "--- суд; приговор. 




\noindent\textbf{Суемудренный} "--- софистический; пустословный. 




\noindent\textbf{Суеслов} "--- пустослов. 




\noindent\textbf{Суета} "--- пустота; ничтожность; мелочность; бессмысленность. 




\noindent\textbf{Суетие} "--- суетность; суета. 




\noindent\textbf{Сулица} "--- копье; кинжал; кортик. 




\noindent\textbf{Супостат} "--- противник; враг. 




\noindent\textbf{Супротивный} "--- противник. 




\noindent\textbf{Супруг} "--- пара; чета. 




\noindent\textbf{Супря} "--- спор; тяжба. 




\noindent\textbf{Суровый} "--- зеленый; свежий; сырой. 




\noindent\textbf{Сходник} "--- лазутчик; разведчик; шпион. 




\noindent\textbf{Счиневати} "--- соединять. 




\noindent\textbf{Сыноположение} "--- усыновление. 




\bukvaending

\bukva{Т}





\noindent\textbf{Таже} "--- потом; затем. 




\noindent\textbf{Тай} "--- тайно; скрытно. 




\noindent\textbf{Таинник} "--- посвященный в чьи-либо тайны. 




\noindent\textbf{Тайноядение} "--- тайное невоздержание от пищи в пост. 




\noindent\textbf{Таити} "--- скрывать. 




\noindent\textbf{Тако} "--- так. 




\noindent\textbf{Такожде} "--- равно; также. 




\noindent\textbf{Талант} "--- древнегреч. мера веса и монета. 




\noindent\textbf{Тамо} "--- там; туда. 




\noindent\textbf{Тартар} "--- место нахождения душ умерших до освобождения их Господом Иисусом Христом; место вечного мучения грешников; жилище диавола. 




\noindent\textbf{Татаур} "--- ремень для привешивания языка к колоколу; кожаный ремень, носимый монашествующими . 




\noindent\textbf{Тать} "--- вор. 




\noindent\textbf{Татьба} "--- кража; воровство. 




\noindent\textbf{Тафта} "--- тонкая шелковая материя. 




\noindent\textbf{Тацы} "--- таковы. 




\noindent\textbf{Таче} "--- для того; также; тогда. 




\noindent\textbf{Тварь} "--- творение; создание; произведение. 




\noindent\textbf{Твердыня} "--- крепость; цитадель; тюрьма. 




\noindent\textbf{Твердь} "--- основание; видимый небосклон, принимаемый глазом за твердую сферу, купол небес. 




\noindent\textbf{Твержа} "--- крепость; цитадель; тюрьма. 




\noindent\textbf{Тезоименитство} "--- одноименность; именины; день Ангела. 




\noindent\textbf{Тектон} "--- плотник; столяр. 




\noindent\textbf{Телец} "--- теленок; бычок. 




\noindent\textbf{Темже, темже убо} "--- поэтому; следовательно; итак. 




\noindent\textbf{Темник} "--- начальник над десятью тысячами человек. 




\noindent\textbf{Темнозрачный} "--- черный. 




\noindent\textbf{Темнонеистовство} "--- мракобесие; непросвещенность. 




\noindent\textbf{Теревинф} "--- дубрава; чаща; лес; большое ветвистое дерево с густой листвой. 




\noindent\textbf{Терние} "--- терновник; колючее растение. 




\noindent\textbf{Терноносный} "--- плодоносящий терние; иносказательно: не имеющий добрых дел. 




\noindent\textbf{Терпкий} "--- кислый; вяжущий; суровый. 




\noindent\textbf{Теснина} "--- узкий проход. 




\noindent\textbf{Теснота} "--- беда; напасть. 




\noindent\textbf{Тетива} "--- туго натянутая веревка. 




\noindent\textbf{Тещи} "--- быстро идти. 




\noindent\textbf{Тещити} "--- источать; испускать. 




\noindent\textbf{Тимение} "--- болото; топь; тина. 




\noindent\textbf{Тимпан} "--- литавра; бубен. 




\noindent\textbf{Тимпанница} "--- девушка, играющая на тимпане. 




\noindent\textbf{Тирон} "--- молодой воин, солдат. 




\noindent\textbf{Титло} "--- надпись; ярлык; знак для сокращения слова. 




\noindent\textbf{Тихообразно} "--- спокойно; кротко. 




\noindent\textbf{Тление, тля} "--- гниение; уничтожение; разрушение. 




\noindent\textbf{Тлети} "--- растлевать; гнить; разрушаться. 




\noindent\textbf{Тлити} "--- повреждать; губить. 




\noindent\textbf{Тма} "--- темнота; мрак; десять тысяч. 




\noindent\textbf{Тоболец} "--- мешок; котомка; сумка. 




\noindent\textbf{Ток} "--- течение. 




\noindent\textbf{Токмо} "--- только. 




\noindent\textbf{Толико} "--- столько. 




\noindent\textbf{Толк} "--- толкование; учение; особое мнение. 




\noindent\textbf{Толковник} "--- переводчик; истолкователь. 




\noindent\textbf{Толковый} "--- объясняющий; содержащий объяснения. 




\noindent\textbf{Толмач} "--- переводчик. 




\noindent\textbf{Толь} "--- столько. 




\noindent\textbf{Томитель} "--- мучитель. 




\noindent\textbf{Томити} "--- мучить; пытать. 




\noindent\textbf{Томление} "--- мучение; пытка. 




\noindent\textbf{Топазий} "--- топаз. 




\noindent\textbf{Торжище} "--- площадь; рынок. 




\noindent\textbf{Торжник} "--- меняла; торговец. 




\noindent\textbf{Торчаный} "--- растерзанный. 




\noindent\textbf{Точило} "--- пресс для выжимания виноградного сока. 




\noindent\textbf{Точию} "--- только. 




\noindent\textbf{Тощно} "--- усердно; точно. 




\noindent\textbf{Трапеза} "--- стол; кушание; столовая, трапезная; святой престол. 




\noindent\textbf{Требе} "--- потребно; надобно. 




\noindent\textbf{Требище} "--- жертвенник; языческий храм. 




\noindent\textbf{Треблаженный} "--- весьма блаженный. 




\noindent\textbf{Требование} "--- нужда; потребность. 




\noindent\textbf{Требовати} "--- нуждаться; иметь потребность. 




\noindent\textbf{Трегубо} "--- трояко; трижды. 




\noindent\textbf{Трекровник} "--- третий этаж. 




\noindent\textbf{Тресна} "--- украшение на одежде. 




\noindent\textbf{Третицею} "--- трижды; в третий раз. 




\noindent\textbf{Тридевять} "--- двадцать семь. 




\noindent\textbf{Трисиянный} "--- светящий от трех Светил. 




\noindent\textbf{Тристат} "--- военачальник. 




\noindent\textbf{Трищи} "--- трижды. 




\noindent\textbf{Тропарь} "--- краткое песнопение, выражающее характеристику праздника или события в жизни святого. 




\noindent\textbf{Трость} "--- тростник (использовавшийся в качестве пишущего инструмента). 




\noindent\textbf{Труд} "--- болезнь; недуг. 




\noindent\textbf{Труждатися} "--- трудиться; затрудняться. 




\noindent\textbf{Трус} "--- землетрясение. 




\noindent\textbf{Трыти} "--- тереть; омывать. 




\noindent\textbf{Ту} "--- тут; там; здесь. 




\noindent\textbf{Туга} "--- скорбь. 




\noindent\textbf{Тук} "--- жир; сало; богатство; пресыщение. 




\noindent\textbf{Тул} "--- колчан для стрел. 




\noindent\textbf{Туне} "--- напрасно; даром; впустую. 




\noindent\textbf{Тунегиблемый} "--- истрачиваемый noнапрасну. 




\noindent\textbf{Тщание} "--- усердие; старание. 




\noindent\textbf{Тщатися} "--- стараться; спешить. 




\noindent\textbf{Тщета} "--- урон; вред; убыток. 




\noindent\textbf{Тщий} "--- пустой; бесполезный; неудовлетворенный. 




\noindent\textbf{Тягота} "--- тяжесть; обременение. 




\noindent\textbf{Тяжание} "--- работа; дело; пашня; поле. 




\noindent\textbf{Тяжатель} "--- работник. 




\noindent\textbf{Тяжати} "--- работать. 




\noindent\textbf{Тяжкосердый} "--- бесчувственный. 




\bukvaending

\bukva{У}





\noindent\textbf{У} "--- еще; 




\noindent\textbf{не у} "--- еще не. 




\noindent\textbf{Убо} "--- а; же; вот; хотя; почему; поистине; подлинно. 




\noindent\textbf{Убрус} "--- плат; полотенце. 




\noindent\textbf{Убудитися} "--- пробудиться; очнуться. 




\noindent\textbf{Уведети} "--- узнать. 




\noindent\textbf{Увет} "--- увещание. 




\noindent\textbf{Увязение} "--- возложение на голову венца. 




\noindent\textbf{Увясло} "--- головная повязка. 




\noindent\textbf{Углебнути} "--- тонуть; погружаться; погрязать. 




\noindent\textbf{Угобзити} "--- обогатить, одарить. 




\noindent\textbf{Угонзнути} "--- убежать; ускользнуть; уйти. 




\noindent\textbf{Уготоватися} "--- приготовиться. 




\noindent\textbf{Угрызнути} "--- укусить зубами. 




\noindent\textbf{Уд} "--- телесный член. 




\noindent\textbf{Удава} "--- веревка. 




\noindent\textbf{Удица} "--- удочка. 




\noindent\textbf{Удобие} "--- удобнее. 




\noindent\textbf{Удобострастие} "--- склонность к угождению страстям. 




\noindent\textbf{Удобрение} "--- украшение. 




\noindent\textbf{Удобрити} "--- наполнить; украсить. 




\noindent\textbf{Удобь} "--- легко; удобно. 




\noindent\textbf{Удовлитися} "--- удовлетвориться. 




\noindent\textbf{Удолие, удоль, юдоль} "--- долина. 




\noindent\textbf{Удручати} "--- утомлять; оскорблять. 




\noindent\textbf{Уже} "--- веревка; цепь; узы. 




\noindent\textbf{Ужик} "--- родственник; родственница. 




\noindent\textbf{Узилище} "--- тюрьма. 




\noindent\textbf{Укорение} "--- бесславие; бесчестие. 




\noindent\textbf{Укрой} "--- повязка; пелена. 




\noindent\textbf{Укроп} "--- теплота, т.~е. горячая вода, вливаемая во святой потир на Литургии. 




\noindent\textbf{Укрух} "--- ломоть; кусок. 




\noindent\textbf{Улучити} "--- застать; найти; получить. 




\noindent\textbf{Умащати} "--- намазывать; натирать. 




\noindent\textbf{Умет} "--- помет; кал; сор. 




\noindent\textbf{Умовредие} "--- безумие. 




\noindent\textbf{Умучити} "--- укротить. 




\noindent\textbf{Уне} "--- лучше. 




\noindent\textbf{Унзнути} "--- воткнуть; вонзить. 




\noindent\textbf{Унше} "--- лучше; полезнее. 




\noindent\textbf{Упитанная} "--- откормленные животные. 




\noindent\textbf{Упование} "--- твердая надежда. 




\noindent\textbf{Упразднити} "--- уничтожить; отменить; исчезнуть. 




\noindent\textbf{Уранити} "--- встать рано утром. 




\noindent\textbf{Урок} "--- урок; подать; оброк. 




\noindent\textbf{Усекновение} "--- отсечение. 




\noindent\textbf{Усерязь} "--- серьга. 




\noindent\textbf{Усма} "--- выделанная кожа. 




\noindent\textbf{Усмарь} "--- кожевенник; скорняк. 




\noindent\textbf{Усмен} "--- кожаный. 




\noindent\textbf{Уста} "--- рот; губы; речь. 




\noindent\textbf{Устранити} "--- лишить; избежать; устранить. 




\noindent\textbf{Устудити} "--- охладить; остудить. 




\noindent\textbf{Усырити} "--- сделать сырым, твердым, мокрым. 




\noindent\textbf{Утварне} "--- по порядку; нарядно. 




\noindent\textbf{Утварь} "--- одежда; украшение; убранство. 




\noindent\textbf{Утверждение} "--- основание; опора. 




\noindent\textbf{Утешение} "--- угощение. 




\noindent\textbf{Утолити} "--- успокоить; утешить; умерить. 




\noindent\textbf{Утреневати} "--- рано вставать; совершать утреннюю молитву. 




\noindent\textbf{Утроба} "--- чрево; живот; сердце; душа. 




\noindent\textbf{Ухание} "--- обоняние; запах. 




\noindent\textbf{Ухлебити} "--- накормить. 




\noindent\textbf{Учреждати} "--- угостить. 




\noindent\textbf{Учреждение} "--- пир; обед; угощение. 




\noindent\textbf{Ушеса} "--- уши. 




\noindent\textbf{Ущедрити} "--- обогатить; помиловать; пожалеть. 




\bukvaending

\bukva{Ф}





\noindent\textbf{Факуд} "--- евр. начальник. 




\noindent\textbf{Фарос} "--- маяк. 




\noindent\textbf{Фаска} "--- Пасха. 




\noindent\textbf{Февруарий} "--- февраль. 




\noindent\textbf{Фелонь} "--- плащ; верхняя одежда; одно из священных облачений пресвитера. 




\noindent\textbf{Фиала} "--- чаша; бокал с широким дном. 




\noindent\textbf{Фимиам} "--- благовонная смола для воскурения при каждении. 




\noindent\textbf{Финикс} "--- пальма. 




\bukvaending

\bukva{Х}





\noindent\textbf{Халван} "--- благовонная смола. 




\noindent\textbf{Халколиван} "--- ливанская медь; янтарь. 




\noindent\textbf{Халуга} "--- плетень; забор; закоулок. 




\noindent\textbf{Харатейный} "--- написанный на прегаменте или папирусной бумаге. 




\noindent\textbf{Хартия} "--- пергамент или папирусная бумага; рукописный список. 




\noindent\textbf{Харя} "--- маска; личина. 




\noindent\textbf{Хврастие} "--- хворост. 




\noindent\textbf{Херет} "--- училище, темница. 




\noindent\textbf{Хитон} "--- нижняя одежда; рубашка. 




\noindent\textbf{Хитрец} "--- художник; ремесленник. 




\noindent\textbf{Хитрогласница} "--- риторика. 




\noindent\textbf{Хитрословесие} "--- риторика. 




\noindent\textbf{Хитрость} "--- художество; ремесло. 




\noindent\textbf{Хитрость, ухищренная вымыслом} "--- стенобитные и метательные военные машины. 




\noindent\textbf{Хищноблудие} "--- насильственное привлечение к блуду. 




\noindent\textbf{Хламида} "--- верхнее мужское платье; плащ; мантия. 




\noindent\textbf{Хлептати} "--- лакать. 




\noindent\textbf{Хлябь} "--- водопад; пропасть; бездна; простор; подъемная дверь. 




\noindent\textbf{Ходатай} "--- посредник, примиритель. 




\noindent\textbf{Хоругвь} "--- военное знамя. 




\noindent\textbf{Хотение} "--- воля. 




\noindent\textbf{Храмляти} "--- хромать. 




\noindent\textbf{Храм набдящий} "--- казнохранилище. 




\noindent\textbf{Храм, храмина} "--- дом; помещение; место богослужения. 




\noindent\textbf{Хранилище} "--- повязка на лбу или на руках со словами Закона. 




\noindent\textbf{Худогий} "--- искусный; умелый. 




\noindent\textbf{Худогласие} "--- косноязычие; заикание. 




\noindent\textbf{Художество} "--- наука; причуда; выкрутаса. 




\noindent\textbf{Худость} "--- скудость; недостоинство. 




\noindent\textbf{Хула} "--- злословие; нарекание. 




\bukvaending

\bukva{Ц}





\noindent\textbf{Цветник} "--- луг. 




\noindent\textbf{Цевница} "--- флейта; свирель. 




\noindent\textbf{Целование} "--- приветствие. 




\noindent\textbf{Целовати} "--- приветствовать. 




\noindent\textbf{Целомудрие} "--- благоразумие, непорочность и чистота телесная. 




\noindent\textbf{Целый} "--- здоровый, невредимый. 




\noindent\textbf{Цельбоносный} "--- врачебный; целительный. 




\bukvaending

\bukva{Ч}





\noindent\textbf{Чадо} "--- дитя. 




\noindent\textbf{Чадородие} "--- рождение детей. 




\noindent\textbf{Чадце} "--- деточка. 




\noindent\textbf{Чарование} "--- яд; отрава; волхвование; заговаривание. 




\noindent\textbf{Чаровник} "--- отравитель; волхв; колдун. 




\noindent\textbf{Чары} "--- волшебство; колдовство; заговаривание. 




\noindent\textbf{Часть} "--- часть; жребий; участь. 




\noindent\textbf{Чаяти} "--- надеяться; ждать. 




\noindent\textbf{Чван} "--- сосуд; штоф; кружка; фляжка. 




\noindent\textbf{Чванец} "--- сосуд; штоф; кружка; фляжка. 




\noindent\textbf{Чело} "--- лоб. 




\noindent\textbf{Челядь} "--- слуги; домочадцы. 




\noindent\textbf{Чепь} "--- цепь. 




\noindent\textbf{Червленица} "--- ткань темно-красного цвета; порфира, пурпурная одежда высокопоставленных особ. 




\noindent\textbf{Червленый} "--- багряный. 




\noindent\textbf{Чермноватися} "--- краснеть. 




\noindent\textbf{Чермный} "--- красный. 




\noindent\textbf{Чернец} "--- монах. 




\noindent\textbf{Черничие} "--- лесная смоква. 




\noindent\textbf{Чертог} "--- палата; покои. 




\noindent\textbf{Чесати} "--- собирать плоды. 




\noindent\textbf{Чесо} "--- чего; что. 




\noindent\textbf{Чести} "--- читать. 




\noindent\textbf{Честный} "--- уважаемый; прославляемый. 




\noindent\textbf{Четверовластник} "--- управляющий четвертою частью страны. 




\noindent\textbf{Чин} "--- порядок; полное изложение или указание всех молитвословий. 




\noindent\textbf{Чинити} "--- составлять; делать. 




\noindent\textbf{Чревобесие} "--- объядение; обжорство. 




\noindent\textbf{Чревоношение} "--- зачатие и ношение в утробе младенца. 




\noindent\textbf{Чреда} "--- порядок; очередь; черед. 




\noindent\textbf{Чреждение} "--- угощение. 




\noindent\textbf{Чресла} "--- поясница; бедра; пах. 




\noindent\textbf{Чтилище} "--- идол; кумир. 




\noindent\textbf{Чудитися} "--- удивляться. 




\noindent\textbf{Чудотворити} "--- творить чудеса. 




\noindent\textbf{Чути, чуяти} "--- чувствовать; слышать; ощущать. 




\bukvaending

\bukva{Ш}





\noindent\textbf{Шелом} "--- шлем; каска. 




\noindent\textbf{Шепотник} "--- наушник; клеветник. 




\noindent\textbf{Шептание} "--- клевета. 




\noindent\textbf{Шипок} "--- цветок шиповника. 




\noindent\textbf{Шуий} "--- левый. 




\noindent\textbf{Шуйца} "--- левая рука. 




\bukvaending

\bukva{Щ}





\noindent\textbf{Щедрота} "--- милость; великодушие; снисхождение. 




\noindent\textbf{Щедрый} "--- милостивый. 




\noindent\textbf{Щогла} "--- мачта; веха; жердь. 




\bukvaending

\bukva{Ю}





\noindent\textbf{Ю} "--- ее. 




\noindent\textbf{Юг} "--- зной; название южного ветра; иносказательно; несчастие. 




\noindent\textbf{Юдоль} "--- долина. 




\noindent\textbf{Юдоль плачевная} "--- мир сей. 




\noindent\textbf{Юдуже} "--- там, где. 




\noindent\textbf{Южик (а)} "--- родственник; родственница. 




\noindent\textbf{Юзник} "--- узник; заключенный. 




\noindent\textbf{Юзы} "--- кандалы; узы. 




\noindent\textbf{Юнец} "--- телок; молодой бычок. 




\noindent\textbf{Юница} "--- телка; молодая корова. 




\noindent\textbf{Юнота} "--- молодой человек. 




\noindent\textbf{Юродивый} "--- глупый; принявший духовный подвиг юродства. 




\bukvaending

\bukva{Я}





\noindent\textbf{Ягодичина} "--- фиговое дерево. 




\noindent\textbf{Ядрило} "--- мачта. 




\noindent\textbf{Ядца} "--- лакомка; гурман; обжора. 




\noindent\textbf{Ядь} "--- пища; еда. 




\noindent\textbf{Язвина} "--- нора. 




\noindent\textbf{Язвити} "--- жечь; ранить. 




\noindent\textbf{Язык} "--- народ; племя; орган речи. 




\noindent\textbf{Язя, язва} "--- рана; ожог. 




\noindent\textbf{Яко} "--- ибо; как; так как; потому что; когда. 




\noindent\textbf{Яковый} "--- каковой. 




\noindent\textbf{Якоже} "--- так чтобы; как; так как. 




\noindent\textbf{Яннуарий} "--- январь. 




\noindent\textbf{Ярем} "--- ярмо; груз; тяжесть; служение. 




\noindent\textbf{Ярина} "--- волна; шерсть. 




\noindent\textbf{Ясти} "--- есть; кушать. 




\noindent\textbf{Яти} "--- брать. 
\normalfont\end{mymulticols}
\longpage\mychapterending


 

