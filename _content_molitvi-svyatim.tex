

\mypart{МОЛИТВЫ СВЯТЫМ}\label{_content_molitvi-svyatim}
%http://www.molitvoslov.com/content/molitvi-svyatim

 

\mychapter{Собору двенадцати Апостолов}
%http://www.molitvoslov.com/text940.htm 
 



 О святии апостоли Христовы: Петре и  Андрее, Иакове и Иоанне, Филиппе и Варфоломее, Фомо и Матфее, Иакове и Иудо, Симоне и Матфие! Услышите наши молитвы и воздыхания, сердцем сокрушенным  ныне приносимые и помозите нам, рабам Божии\itshape  (имена\normalfont{}), вашим мощным пред Господем ходатайством, избавитися от всякаго зла и вражия лести, твердо же преданную вами веру православную сохраняти, в ней же вашим предстательством ни ранами, ни прещением, ни мором, ни коим гневом  от Создателя нашего умалени будем, но мирное зде поживем житие и сподобимся видети благая на земли живых, славяще Отца и Сына и Святаго Духа, Единаго в Троице славимаго и покланяемаго Бога, ныне и присно и во веки веков.
\mychapterending

\mychapter{Преподобному Пафнутию, игумену Боровскому}
%http://www.molitvoslov.com/text911.htm 
 



О священная главо, земный ангеле, небесный человече, великий чудотворче, преподобне oтче наш Пафнутие! К тебе с верою и любовию усердно прибегаем и умиленно твоего заступления просим. Не дерзаем, грех ради наших, с свободою чад Божиих Господа и Владыку нашего о помиловании и прощении просити. Но тебе молитвенника к Нему благоприятна предлагаем и молим: испроси нам у благости Eго дары благопотрeбныя и спаситeльныя душам нашим: веру правую, во благочестии крепкое стояние, грехов прощение, жития совершенное исправление, да обратившеся от злых дел к богоугождению, прочее не прогневляем Господа святых Его заповедей преступлением. Умоли, святче Божий, Всевышняго Творца даровати мир и благочестие стране нашей православной. Сохрани, угодниче Христов, святую обитель твою, тобою созданную, и вся живущия и подвизающияся в ней от всякаго зла ненаветны. Призри милостивно на люди, раце мощей твоих предстоящия, и вся прошeния их во благое исполни. Нам же всем здравие душевное и телесное, земли плодоносие, тихое и богоугодное житие, благую христианскую кончину и добрый ответ на Страшнем Суде у Всемилостиваго Бога исходатайствуй. Ей, oтче, вемы, яко много может молитва твоя пред лицем Вседержителя Господа, и ничтоже невозможно eсть ходатайству твоему, аще токмо восхощеши: сего ради крепце на тя уповаем, и на твоя святыя молитвы вельми надеемся, яко ты приведеши нас, предстательством твоим, в тихое пристанище спасения и наследники явиши ны светлаго Царствия Христова. Не посрами же упования нашего, чудотворче святый, и сподоби нас вкупе с тобою блаженства Райскаго наслаждатися, да славим, хвалим и величаем великую милость к нам Человеколюбца Бога, Отца и Сына и Святаго Духа, и твое благое oтеческое заступление во веки веков. Аминь.
\mychapterending

\mychapter{Св. первомученику архидиакону Стефану}
%http://www.molitvoslov.com/node/335 
 
{\noparindent\hspace{0.15\textwidth}\begin{minipage}{0.7\textwidth}
\restoreparindent\myfig{190_1}О святый первомучениче, апостоле архидиаконе Стефане! Преклоньше колена души и сердца нашего, тя молим: воздежи руце твои ко Господу и сподвижи всему роду человеческому Предстательницу и Ходатаицу к Сыну Своему, Пресвятую Матерь Деву, и о нас (имена), присно прогневляющих грехом и леностию, моли Благого Спаса нашего, да подаст нам благодать во благовременную помощь к покаянию, и твоим предстательством и заступлением Пресвятыя Матери Своея, в день он страшный и праведный, да не поставит пред нами грехи наша, но по щедротам Своим, тобою умолен быв, да речет душе нашей: спасение ваше есмь Аз во веки веков.
\end{minipage}\hspace{0.15\textwidth}}\mychapterending

\mychapter{Св. царице Грузинской Тамаре}
%http://www.molitvoslov.com/text652.htm 
 
О святая царице Тамаро! не забуди нас, но поминай во святых твоих молитвах раб Божиих \itshape (имена)\normalfont{}, моли за ны, царице святая. Не отступи от нас духом, сохраняй нас от стрел вражиих, от прелести бесовския и козней диавольских. Испроси нам время на покаяние и невозбранно прейти от земли на небо чрез мытарства горьких бесов, да твоим предстательством от вечныя муки избавльшеся, Небесное Царствие наследовати сподобимся со всеми праведными, от века угодившими Христу Господу нашему: емуже подобает всякая слава, честь и поклонение, ныне и в безконечныя веки веков.
\mychapterending

\mychapter{Молитва Святому Царю Страстотерпцу Николаю}
%http://www.molitvoslov.com/text560.htm 
 
\myfig{429}

\mysubsubsection{Молитва:}


О святый страстотерпче, царю мучениче Николае! Господь тя избра помазанника Своего, во еже милостивно и право судити людем твоим и хранителем церкви православныя быти. Сего ради со страхом Божиим царское служение и о душах попечение совершал еси. Господь же, испытуя тя, яко Иова многострадальнаго, попусти ти поношения, скорби горькия, измену, предательство, ближних отчуждение и в душевных муках земнаго царство оставление. Вся сия ради блага России, яко верный сын ея, претерпев и, яко истинный раб Христов, мученическую кончину прием, Небеснаго Царства достигл еси, идеже наслаждаешися вышния славы у престола всех Царя, купно со святою супружницею твоею, царицею Александрою, и царственными чады Алексием, Ольгою, Татияною, Мариею и Анастасиею. 

Ныне, имея дерзновение велие у Христа Царя, моли, да простит Господь грех отступления народа нашего, и подаст грехов прощение, и на всякую добродетель наставит нас, да стяжим смирение, кротость и любовь и сподобимся Небесного Царствия, идеже купно с тобою и всеми святыми новомученики и исповедники российскими прославим Отца и Сына и Святаго Духа, ныне и присно, и во веки веков. Аминь.

(М. Издательский Совет Русской Православной Церкви, 2003, по благословению Святейшего Патриарха Московского и всея Руси Алексия II).
\mychapterending
%\longpage{}\mychapterending

\mychapter{Тропарь Святому Благоверному Царю-Мученику Николаю о спасении России  (из Службы Святым Царственным Мученикам)}
%http://www.molitvoslov.com/text559.htm 
 


\mysubsubsection{Тропарь, глас 5.}


Царства земнаго лишение, узы и страдания многоразличныя кротко претерпел еси, свидетельствовав о Христе даже до смерти от богоборцев,страстотерпче великий Боговенчанный царю Николае, сего ради мученическим венцем на Небесех венча тя с царицею, и чады, и слуги твоими Христос Бог, Егоже моли помиловати страну Российскую и спасти души наша.
\mychapterending

\mychapter{Cвятому равноапостольному Великому Князю Владимиру}
%http://www.molitvoslov.com/text283.htm 
 
\myfig{175}Внук святой княгини Ольги, Владимир, поначалу, был язычником, но, чувствуя внутренне пустоту и ложь идолопоклонства, пережил обращение. В 988 году он принял Святое Крещение от греков и крестил свой народ. 


Св. Владимир строил церкви и школы, заботился о нищих и сиротах, являл собой пример милосердия и гостеприимства. 


Православная Церковь при св. Владимире становится духовным руководителем и просветителем русского народа, тем самым, именно князем Владимиром Святым был заложен несокрушимый духовный фундамент тысячелетней российской истории.


\mysubsubsection{Молитва первая}

О великий угодниче Божий, богоизбранный и богопрославленный, равноапостольный княже Владимире! Ты отринул еси зловерие и нечестие языческое, уверовал еси во Единаго Истиннаго Триипостаснаго Бога и, восприяв Святое Крещение, просветил еси светом Божественныя веры и благочестия всю страну Русскую. Славяще убо и благодаряще Премилосердаго Творца и Спасителя нашего, славим, благодарим и тя, просветителю наш и отче, яко тобою познахом спасительную веру Христову и крестихомся во Имя Пресвятыя и Пребожественныя Троицы: тою верою избавихомся от праведнаго осуждения Божия, вечнаго рабства диаволя и адова мучительства: тою верою восприяхом благодать всыновления Богу и надежду наследования Небеснаго блаженства. Ты еси первый вождь наш к Начальнику и Совершителю нашего вечнаго спасения Господу Иисусу Христу; ты еси теплый молитвенник и ходатай о стране Русской, о воинстве и о всех людех. Не может язык наш изобразити величие и высоту благодеяний, тобою излиянных на землю нашу, отцев и праотцев наших и на нас, недостойных. О всеблагий отче и просветителю наш! Призри на немощи наша и умоли премилосердаго Царя Небеснаго, да не прогневается на ны зело, яко по немощем нашим по вся дни согрешаем, да не погубит нас со беззаконьми нашими, но да помилует и спасет нас, по милости Своей, да всадит в сердце наше спасительный страх Свой, да просветит Своею благодатию ум наш, во еже разумети нам пути Господни, оставити стези нечестия и заблуждений, тщатися же во стезях спасения и истины, неуклоннаго исполнения заповедей Божиих и уставов Святыя Церкве. Моли, благосерде, Человеколюбца Господа, да пробавит нам великую милость Свою, да избавит нас от нашествия иноплеменных, от внутренних нестроений, мятежей и раздоров, от глада, смертоносных болезней и от всякаго зла, да подаст нам благорастворение воздуха и плодоносие земли, да даст пастырем ревность о спасении пасомых, всем же людем споспешение о еже усердно службы своя исправляти, любовь между собою и единомыслие имети, на благо же Отечества и Святыя Церкве верне подвизатися, да возсияет свет спасительныя веры в стране нашей во всех концех ея, да упразднятся вся ереси и расколы, да тако поживше в мире на земли, сподобимся с тобою вечнаго блаженства, хваляще и превозносяще Бога во веки веков. Аминь.


\mysubsubsection{Молитва вторая}

О великий угодниче Божий, равноапостольный княже Владимире! Призри на немощи наши и умоли Премилосердаго Царя Небеснаго, да не прогневается на ны зело и да не погубит нас со беззаконьми нашими, но да помилует и спасет нас по милости Своей, да всадит в сердца наша покаяние и спасительный страх Божий, да просветит Своею благодатию ум наш, во еже оставити нам стези нечестия и на путь спасения обратитися, неуклонно же заповеди Божия творити и уставы Святыя Церкве соблюдати. Моли, благосерде, Человеколюбца Бога, да явит нам великую милость Свою: да избавит нас от смертоносных болезней и от всякаго зла, да сохранит и спасет рабов Божиих (имена) от всех козней и наветов вражиих и да все мы сподобимся с тобою вечнаго блаженства, хваляще и превозносяще Бога во веки веков.\mychapterending

\mychapter{Святому блаженному Василию, Христа ради юродивому Московскому}
%http://www.molitvoslov.com/text288.htm 
 
Блаженный Василий родился в 1468 году в Подмосковье, в крестьянской семье. Отроком его отдали в обучение сапожному ремеслу, а в 16 лет бежал из дома, призванный Господом на труднейший подвиг юродства, который он и проходил 72 года, живя на многолюдных стогнах царствующего града Москвы. 


Блаженный изнурял свое тело постом, молитвой и всевозможными лишениями, он был прозорлив и совершал многие чудеса, обличал простых и знатных, невзирая на сан, но ведая внутреннее устроение человека. 


Василий Блаженный скончался в 1552 году, его св. мощи были положены в Москве, в Покровском соборе на Красной площади.




\mysubsubsection{Молитва}


О великий угодниче Христов, истинный друже и верный рабе Всетворца Господа Бога, преблаженне Василие! Услыши ны, многогрешныя, ныне воспевающия к тебе и призывающия имя твое святое, помилуй ны, припадающие днесь к пречистому образу твоему, приими малое наше и недостойное сие моление, умилосердися над убожеством нашим и молитвами твоими исцели всяк недуг и болезнь души и тела нашего грешнаго, и сподоби ны течение жизни сея невредимо от видимых и невидимых врагов безгрешно прейти, и христианскую кончину, непостыдну, мирну, безмятежну, и Небеснаго Царствия наследие получити со всеми святыми во веки веков. Аминь.\mychapterending

\mychapter{Mолитва новомученикам и исповедникам Российским}
%http://www.molitvoslov.com/text281.htm 
 



Святии новомученицы и исповедницы Церкве Российския, услышите усердную мольбу нашу! Вемы, яко нецыи от вас, еще отроцы суще, послушающе о древлих страстотерпцех, в сердце своем помыслиша, колико прелюбезно и доброхвально есть таковым подражати, ихже ни муки, ни смерть не разлучиша от любве Божия. 


Благо же вам, яко последовали есте вере и терпению тех, о нихже слышасте и ихже возлюбисте. А понеже на всякое время возможно есть найти на ны испытанием нечаянным, испросите от Господа нам мужества дар, иже толико благопотребен есть в житии человечестем. Вся концы отечества нашего страданьми своими освятивший, яко общий за вся ны молитвенницы, умолите Бога, избавити люди Своя от ига, ужаснейшаго паче всякаго инаго. И да отпустится нам и всему роду нашему грех, на народе российском тяготеющий: убиение царя, помазанника Божия, святителей же и пастырей с паствою, и страдания исповедников, и осквернение святынь наших. Да упразднятся расколы в Церкве нашей, да будут вей едино и да изведет Господь на жатву делатели Своя, сиесть да не оскудевает Церковь пастырьми добрыми, иже имут просвещати светом истинныя веры столь великое множество людей, вере ненаученных, или от веры отвратившихся. Недостойни есмы милости Божия, обаче страданий ради ваших Христос Бог наш да ущедрит и помилует всех нас, в помощь вас призывающих. Мы же Ему, Спасителю нашему, со Отцем и со Святым Духом сокрушение о гресех и благодарение за вся всегда да приносим, славяще Его во веки веков. Аминь.
\mychapterending

\mychapter{Святому преподобному Иосифу Волоцкому}
%http://www.molitvoslov.com/text286.htm 

\myfigh{178_0}{16}

Преподобный Иосиф родился в 1440 году, близ Волоколамска. Двадцати лет от роду принял монашество и подвизался сначала в Пафнутиевом Боровском монастыре, а потом основал свой монастырь на родине своей, в Волоколамске. 

Преподобный Иосиф непримиримо обличал ересь «жидовствующих» и написал против них книгу под названием «Просветитель», он был мудрым строиггелем монастырской жизни, руководствуясь во всем непогрешимыми правилами Святой Церкви. Скончался в 1515 году.


\mysubsubsection{Молитва}

О великий наставниче, ревнителю и учителю Православныя веры, святе премудре Иосифе! Приими моление, от нас, грешных, к тебе приносимое, и теплым предстательством умоли в Троице славимаго Бога, да ниспослет Он богатыя Своя милости нам, грешным: да утвердит во Святей Своей Православней Церкви правую веру и благочестие: пастырем ея да подаст святую ревность о спасении словеснаго стада, яко да верующих соблюдут, неверующих же и отпадших от истинныя веры вразумят и обратят. Всем же нам испроси вся благопотребная во временном сем житии и полезная к вечному спасению нашему. Помяни стадо свое, еже собрал еси, не забуди посещати чад своих и, яко чадолюбивый отец, не презри, ниже отрини моления наша, но воздежи руце свои молитвенней ко Господу Богу, да отложит праведный гнев Свой движимый на ны и избавит нас от враг видимых и невидимых, от глада, потопа, меча, смертоносныя язвы, нашествия иноплеменных и междоусобныя брани. Ей, премилосердый заступниче наш, преславный чудогворче, всех нас управи в мире и покаянии скончати живот наш и преселитися со упованием в блаженная недра Авраамова, идеже непрестанно славословится всепетое имя Отца и Сына и Святаго Духа. Аминь. (\itshape Из акафиста\normalfont{})
\longpage{}\mychapterending

\mychapter{Святому великомученику и Победоносцу Георгию}
%http://www.molitvoslov.com/text296.htm 
 
\myfig[0.33]{187_0}

Св. Георгий жил во времена императора Диоклетиана, был сыном богатых и благочестивых родителей-христиан. Поступив в военную службу, Георгий своею доблестью, умом и красотой скоро сделался любимцем Диоклетиана и заслужил звание тысяченачальника. 

Однажды, став свидетелем бесчеловечного судилища над христианами, он раздал свое имение бедным, обличил неправду царя и исповедовал свое христианство. 

Приняв жесточайшие страдания и победоносно преодолев их, св. Георгий принял мученическую кончину за Христа в 303 году, в Никомидии. 

Житие св. великомученика Георгия стало примером непременной победы Добра над силами тьмы. Георгий Победоносец, поражающий змея,"--- герб города Москвы и вместе с тем символ побеждающей зло России.


\mysubsubsection{Молитва}

О всехвальный святый великомучениче и чудотворче Георгие! Призри на ны скорою твоею помощию и умоли человеколюбца Бога, да не осудит нас грешных по беззакониям нашим, но да сотворит с нами по велицей Своей милости. Не презри моления нашего, но испроси нам у Христа Бога нашего тихое и богоугодное житие, здравие же душевное и телесное, земли плодородие и во всем изобилие, и да не во зло обратим благая, даруемая нам тобою от всещедраго Бога, но во славу святаго имене Его и в прославление крепкаго твоего заступления, да подаст Он православному народу нашему на супостаты одоление и да укрепит нас непременяемым миром и благословением. Изряднее же да оградит нас святых Ангел Своих ополчением, во еже избавитися нам, по исходе нашем из жития сего, от козней лукаваго и тяжких воздушных мытарств его, и неосужденным предстати престолу Господа славы. 

Услыши нас, страстотерпче Христов Георгие, и моли за ны непрестанно триипостасного Владыку всех Бога, да благодатию Его и человеколюбием, твоею же помощию и заступлением, обрящем милость со Ангелы и Архангелы и всеми святыми одесную правосуднаго Судии стати, и того выну славити со Отцем и Святым Духом, ныне и присно и во веки веков. Аминь.
\mychapterending

\mychapter{Преподобному Серафиму, Саровскому чудотворцу}
%http://www.molitvoslov.com/text291.htm 
 
\myfig[.39]{182}Преподобный Серафим Саровский происходил из благочестивой купеческой семьи города Курска; с юных лет обнаружил тягу к благочестию и монашеским подвигам, 17-ти лет оставил родительский дом; сначала подвизался в Киево-Печерской Лавре, а потом "--- в Саровской пустыни Тамбовской губернии. Тысячу дней и ночей провел в молитве, стоя на коленях на камне; за святую жизнь сподобился неоднократного посещения Божией Матери и святых. 


Преподобный Серафим был прозорлив, исцелял душевные и телесные недуги: его молитвенную помощь неоднократно испытывали на себе не только православные, но и люди других вероисповеданий. 


В начале XIX века именно преподобный Серафим был поистине всероссийским молитвенником и печальником за всю Русскую Землю и за каждого страждущего человека. 


Он скончался в 1833 году, 73 лет от роду, стоя на коленях во время молитвы. 


Его прославление состоялось летом 1903 года, при живом и горячем участии Государя Императора Николая Александровича и всей Царской Семьи.


\mysubsubsection{Тропарь, глас 4-й:}


От юности Христа возлюбил еси, блаженне, и Тому Единому работати пламенне вожделев, непрестанною молитвою и трудом в пустыни подвизался еси, умиленным же сердцем любовь Христову стяжав, избранник возлюблен Божия Матери явился еси. Сего ради вопием ти: спасай нас молитвами твоими, Серафиме, преподобне отче наш.



\mysubsubsection{Кондак, глас 2-й:}


Мира красоту и яже в нем тленная оставив, преподобне, в Саровскую обитель вселился еси; и тамо ангельски пожив, многим путь был еси ко спасению. Сего ради и Христос тебе, отче Серафиме, прослави, и даром исцелений и чудес обогати. Темже вопием ти: радуйся, Серафиме, преподобне отче наш.


\mysubsubsection{Молитва}


О пречудный отче Серафиме, великий Саровский чудотворче, всем прибегающим к тебе скоропослушный помощниче! Во дни земнаго жития твоего никтоже от тебе тощ и неутешен отъиде, но всем в сладость бысть видение лика твоего и благоуветливый глас словес твоих. К сим же и дар исцелений, дар прозрения, дар немощных душ врачевания обилен в тебе явися. Егда же призва тя Бог от земных трудов к небесному упокоению, николиже любовь твоя преста от нас, и невозможно есть исчислити чудеса твоя умножившаяся, яко звезды небесныя: се бопо всем концем земли нашея людем Божиим являешися и даруеши им исцеления. Темже и мы вопием ти: о претихий и кроткий угодниче Божий, дерзновенный к Нему молитвенниче, николиже призывающия тя отреваяй, вознеси о нас благомощную твою молитву ко Господу сил, да укрепит державу нашу, да дарует нам вся благопотребная в жизни сей и вся к душевному спасению полезная, да оградит нас от падений греховных и истинному покаянию научит нас, во еже безпреткновенно внити нам в вечное Небесное Царство, идеже ты ныне в незаходимей сияеши славе, и тамо воспевати со всеми святыми Живоначальную Троицу до скончания века. Аминь.


\mysubsubsection{Молитва:}


О преподобне отче Серафиме! Вознеси о нас, рабех Божиих (\itshape имена\normalfont{}), благомощную твою молитву ко Господу сил, да дарует нам вся благопотребная в жизни сей и вся к душевному спасению полезная, да оградит нас от падений греховных и истинному покаянию да научит нас, во еже безпреткновенно внити нам в вечное Небесное Царство, идеже ты ныне в незаходимей сияеши славе, и тамо воспевати со всеми святыми Живоначальную Троицу во веки веков.
\mychapterending

\mychapter{Святителю Алексию,  митрополиту Московскому, чудотворцу}
%http://www.molitvoslov.com/text301.htm 
 
\myfigh{191_0}{15}

\mysubsubsection{Тропарь, глас 8-й:}

Яко апостолам сопрестольна и врача предобра,  и служителя благоприятна,  к раце твоей честней притекающе. святителю Алексие богомудре, чудотворче,  сошедшеся любовию в память твою светло празднуем,  в песнех и пениих радующеся и Христа славяще,  таковую благодать тебе Даровавшаго исцелений / и граду твоему великое утверждение. 


\mysubsubsection{Кондак, глас 8-й:}

Божественнаго и пречестнаго святителя Христова, новаго чудотворца Алексия, верно вси поюще, людие, любовию да ублажим, яко пастыря великаго, служителя же и учителя премудра земли Российстей: днесь в память его притекше, радостно возопием песнь Богоносному, яко имея дерзновение к Богу, многообразных нас избави обстояний, да зовем ти: радуйся, утверждение граду нашему.


\mysubsubsection{Величание:}


Величаем тя, святителю отче Алексие, и чтим святую память твою, ты бо молиши за нас Христа Бога нашего.


\mysubsubsection{Молитва:}


О,  пречестная и священная  главо  и благодати Святаго Духа исполненная, Спасово со Отцем обиталище, великий архиерее, теплый наш заступниче, святителю Алексие! Предстоя у Престола всех Царя и наслаждаяся света Единосущныя Троицы и херувимски со ангелы возглашая песнь трисвятую, великое же и неизследованное дерзновение имея ко всемилостивому Владыце, молися паствы твоея спасти люди, единородный ти язык: мирно и безмятежно житие наше устрой: благосостояние святых церквей утверди; архиереи благолепием святительства украси; монашествующия к подвигом добраго течения укрепи; царствующий град (сей и святую обитель сию) и вся грады и страны добре сохрани, и веру святую непорочну соблюсти умоли; мир весь предстательством твоим умири, от глада и пагубы избави ны, и от нападения иноплеменных сохрани; старыя утеши, юныя настави, безумныя умудри, вдовицы помилуй, сироты заступи, младенцы возрасти, немощствующия исцели, и везде тепле призывающия тя и с верою притекающия к раце честных и многоцелебных мощей твоих, усердно припадающия и молящияся тебе, от всяких напастей и бед ходатайством твоим свободи, да зовем ти: радуйся, Богоизбранный пастырю, звездо всесветлая мысленныя тверди, тайнаго Сиона необоримый столпе, миродохновенный цвете райский, всезлатая уста Слова, московская похвало, всея России украшение! Моли о нас Всещедраго и Человеколюбиваго Христа Бога нашего да и в день Страшного Пришествия Его от шуияго стояния избавит нас, и радости святых причастники сотворит со всеми святыми во веки веков. Аминь.\mychapterending

\mychapter{Святому апостолу Андрею Первозванному}
%http://www.molitvoslov.com/text295.htm 
 
Брат св. апостола Петра, ученик св. Иоанна Крестителя, призванный первым на апостольское служение, св. апостол Андрей был в числе ближайших учеников Спасителя и проповедовал Евангелие во многих странах. По преданию, апостол Андрей был в России, поднялся по Днепру до Киевских холмов и, водрузив крест на горе, сказал своим ученикам: «На сем месте воссияет благодать Божия». 

За проповедь веры Христовой св. апостол Андрей принял мученическую кончину, распятый на кресте около 62 года по Рождению Xриста.


\mysubsubsection{Молитва}

\myfigr{186}

Первозванне Апостоле Бога и Спаса нашего Иисуса Христа, Церкве последователю, всехвальный Андрее! 

Славим и величаем апостольския труды твоя, сладце поминаем твое благословенное к нам пришествие, ублажаем честная страдания твоя, яже за Христа претерпел еси, лобызаем священныя мощи твоя, чтим святую память твою и веруем, яко жив Господь, жива же и душа твоя и с Ним во веки пребываеши на небеси, идеже и любиши ны тоюжде любовию, еюже возлюбил еси нас, егда Духом Святым прозрел еси наше, еже ко Христу обращение, и не точию любиши, но и молиши о нас Бога, зря во свете Его вся нужды наша. Тако веруем и тако сию веру нашу исповедуем во храме, иже во имя твое, святый Андрее, преславно создася, идеже святыя мощи твоя почивают: верующе же, просим и молим Господа и Бога и Спаса нашего Иисуса Христа, да молитвами твоими, яже присно послушает и приемлет, подаст нам вся потребная ко спасению нас грешных: да якоже ты абие по гласу Господа, оставль мрежи своя, неуклонно Ему последовал еси, сице и кийждо от нас да ищет не своих си, но еже к созиданию ближняго и о горнем звании да помышляет. Имуще же тя предстателя и молитвенника о нас, уповаем, яко молитва твоя много может пред Господем и Спасителем нашим Иисусом Христом, Емуже подобает всякая слава, честь и поклонение со Отцем и Святым Духом и во веки веков. Аминь.\mychapterending

\mychapter{Святителю Димитрию, Ростовскому чудотворцу}
%http://www.molitvoslov.com/text290.htm 
 



Святитель Димитрий родился в 1651 году в Киевской губернии. Родом из казаков. С юных лет избрал путь духовного подвига: учился в Киевской Духовной академии, вел строгую монашескую жизнь в монастырях Киева и Чернигова, обладал выдающимся даром проповедничества. 


В 1701 году св. Димитрий был возведен в сан митрополита, а в 1702 "--- назначен на Ростовскую кафедру, где и проходил святительское служение до смерти, последовавшей в 1709 году. 


Неоценимая заслуга св. Димитрия перед Церковью "--- создание знаменитых Четий-Миней (Житий святых) на весь год, борьба против раскольников, воспитание должного благочестия в священниках и их пастве. 


В богоотступническое время российской истории, каким в значительной мере был ХУШ век, святитель Димитрий явился иерархом, возрождающим культуру Церкви как высшее проявление русской национальной культуры. 


Мощи Святителя покоятся в Спасо-Яковлевом монастыре Ростова Великого.


\mysubsubsection{Молитва}


О предивный и преславный чудотворче Димитрие, исцеляяй недуги человеческия! Ты неусыпно молиши Господа Бога нашего о всех грешных: молю убо тя, буди ми ходатай пред Господем и помощник на преоборение страстей ненасытныя плоти моея и на одоление стрел сопротивника моего диавола, имиже уязвляет немощное сердце мое и, аки гладный и лютый зверь, алчет погубити душу мою. Ты, святителю Христов, моя ограда, ты мое заступление и оружие! За твоим содействием сокрушу вся во мне сопротивная воле Царя царствующих. Ты, великий чудотворче, во дни подвигов твоих в мире сем ревнуя о Православной Церкви Божией, яко истинный и добрый пастырь, неблазненно обличал еси грехи и невежествия людския, и уклонившихся от стези правды в ереси и расколы наставлял еси на путь истины. Споспешествуй убо и мне исправити кратковременный путь жизни моея, да непреткновенно пойду по стези заповедей Божиих и неленостно поработаю Господеви моему Иисусу Христу, яко единому Владыце моему, Искупителю и праведному Судии моему. К сим же припадая, молюся ти, угодниче Божий, егда изыти души моей из телесе моего, избави ю от темных мытарств: не имам бо добрых дел ко оправданию моему: не даждь сатане возгордиться победою над немощною душею моею. Избави ю от геенны, идеже плач и скрежет зубов, и святыми молитвами твоими сотвори мя причастника Небеснаго Царствия в Троице славимаго Бога, Отца и Сына и Святаго Духа. Аминь.
\longpage[2]{}\mychapterending

\mychapter{Преподобному Александру Свирскому}
%http://www.molitvoslov.com/text300.htm 
 
\myfigh{190}{16}

\mysubsubsection{Тропарь преподобному, глас 4-й:}

От юности, богомудре, желанием духовным в пустыню вселився, единаго Христа возжелал еси усердно стопам в след ходити. Темже и ангельстии чини зряще тя удивишася, како с плотию к невидимым кознем подвизався, премудре, победил еси полки страстей воздержанием, и явился еси равноангелен на земли, Александре, преподобне. Моли Христа Бога, да спасет души наша.


\mysubsubsection{Кондак преподобному, глас 8-й:}


Яко многосветлая звезда днесь в странах Российских возсиял еси, отче, вселився в пустыню, Христовым стопам последовати усердно возжелел еси, и Того святое иго на рамо твое взем честный крест, умертвил еси, труды подвиг твоих, телесная взыграния. Темже вопием ти: спаси стадо твое, еже собрал еси мудре, да зовем ти: радуйся, преподобне Александре, отче наш.


\mysubsubsection{Величание:}


Ублажаем тя, преподобне отче Александре, и чтим свяую память твою, наставник монахов и собеседник ангелов.


\mysubsubsection{Молитва:}


О, священная главо, ангеле земный и человече небесный, преподобне и богоносне отче наш Александре, изрядный угодниче Пресвятыя и Единосущныя Троицы, многия милости живущим во святей обители твоей и всем, с верою и любовию притекающим к тебе, являяй. Испроси нам вся к житию сему временному благопотребная, паче же к вечному спасению нашему нужная. Пособствуй предстательством твоим, угодниче Божий, правителем страны нашея России. И да в мире глубоце пребудет святая православная Церковь Христова. Буди всем нам, чудотворче святый, во всякой скорби и обстоянии скорый помощниче. Наипаче же в час кончины нашея явися нам, заступниче благосердый, да не предани будем на мытарствах воздушных власти злобнаго миродержца, но да сподобимся непреткновеннаго восхода в Царствие Небесное. Ей, отче, молитвенниче наш присный! Не посрами упования нашего, не презри смиренных молений наших, но присно о нас пред Престолом Живоначальныя Троицы предстательствуй, да сподобимся вкупе с тобою и со всеми святыми, аще и недостойны есмы, в селениих райских славити величие, благодать и милость Единаго в Троице Бога, Отца и Сына и Святаго Духа во веки веков. Аминь.\mychapterending

\mychapter{Святому преподобному Силуану Афонскому}
%http://www.molitvoslov.com/text294.htm 
 
Будущий старец Силуан родился в крестьянской семье Тамбовской губернии в 1866 году, Призванный на военную службу в Петербург, он «умом пребывая на Афоне и на Страшном Суде», получает у св. праведного Иоанна Кронштадтского благословение на монашеский подвиг и в 1892 году становится послушником русского Пантелеимонова монастыря на Афоне. 


Монашеские подвиги преподобного Силуана были тайными, но особую благодать, которая от него исходила, чувствовали все, от простых рабочих до церковных иерархов. 


После смерти старца, последовавшей в 1938 году, в Пантелеимонов монастырь стали приходить многочисленные письма, свидетельствующие о его небесном предстательстве за тех, кто обращался к нему с молитвой.



\mysubsubsection{Молитва}

\myfig[0.22]{521}

О предивный угодниче Божий отче Силуане! По благодати, тебе от Бога данной, слезно молится о всей вселенной "--- мертвых, живых и грядущих "--- не промолчи за нас ко Господу, к тебе усердно припадающих и твоего предстательства умильно просящих. Подвигни, о всеблаженне, на молитву Усердную Заступницу рода христианскаго, Преблагословенную Богородицу и Приснодеву Марию, чудно призвавшую тя быти верным делателем в Ея земном вертограде, идеже избранницы Божий о гресех наших милостива и долготерпелива быти Бога умоляют, во еже не помянута неправд и беззаконий наших, но по неизреченной благости Господа нашего Иисуса Христа ущедрити и спасти нас по велицей Его милости. 


Ей, угодниче Божий, с Преблагословенною Владычицею мира "--- Святейшею Игумению Афона и святыми подвижниками Ея земнаго жребия испроси у святых Святейшего Слова Святей Горе Афонской и боголюбивым пустынножителям ея от всех бед и наветов вражиих в мире сохранитися. Да Ангелы святыми от зол избавляеми и Духом Святым в вере и братолюбии укрепляеми, до скончания века о Единей, Святей, Соборней и Апостольстей Церкви молитвы творят и всем спасительный путь указуют, да Церковь Земная и Небесная непрестанно славословит Творца и Отца Светов, просвещающи и освещающи мир в вечной правде и благости Божией. 


Народом земли всей испроси благоденственное и мирное житие, дух смиренномудрия и братолюбия, добронравия и спасения, дух страха Божия. Да не злоба и беззаконие ожесточают сердца людския, могущие истребити любовь Божию в человецех и низвергнуть их в богопротивную вражду и братоубийство, но в силе Божественныя любве и правды, якоже на небеси и на земли да святится имя Божие, да будет воля Его святая в человецех, и да воцарится мир и Царствие Божие на земли. 


Такожде и земному Отечеству твоему "--- Земли Российстей испроси, угодниче Божий, вожделенный мир и небесное благословение, во еже всемощным омофором Матере Божия покрываему, избавитися ему от глада, губительства, труса, огня, меча, нашествия иноплеменников и междоусобныя брани и от всех враг видимых и невидимых, и тако святейшим домом Преблагословенныя Богородицы до скончания века ему пребыти Креста Животворящаго силою и в любви Божией неоскудеваему утвердитися.


Нам же всем, во тьму грехов погружаемым и покаяния тепла, ниже страха Божия не имущим и сице безмерно любящаго нас Господа непрестанно оскорбляющим, испроси, о всеблаженне, у Всещедраго Бога нашего, да Своею Всесильною благодатию божественне посетит и оживотворит души наша, и всяку злобу и гордость житейскую, уныние и нерадение в сердцах наших да упразднит. 


Еще молимся, о еже и нам, благодатию Всесвятаго Духа укрепляемым и любовию Божию согреваемым, в человеколюбии и братолюбии, смиренномудренном сраспинании друг за друга и за всех, в правде Божией утвердитися и в благодатней любви Божией благонравно укрепитися, и сынолюбне Тому приближитися. Да тако, творяще Его всесвятую волю, во всяком благочестии и чистоте временнаго жития путь непостыдно прейдем и со всеми святыми Небеснаго Царствия и Его Агнчаго брака сподобимся. 


Ему же от всех земных и небесных да будет слава, честь и поклонение, со Безначальным Его Отцем, Пресвятым и Благим и Животворящим Его Духом, ныне и присно и во веки веков. Аминь.
\mychapterending

\mychapter{Мученикам Адриану и Наталии}
%http://www.molitvoslov.com/text298.htm 
 
\myfigh{189_0}{15}

\mysubsubsection{Тропарь, глас 3-й}

Богатство неиждивущее вменил еси веру спасенную, треблаженне, оставль отеческое нечестие и Владыки стопами шествуя, Божественными даровании обогатился еси, Адриане славне; Христа Бога моли спастися душам нашим.


\mysubsubsection{Кондак, глас 4-й:}


Жены богомудрыя Божественная словеса в сердце положив, Адриане, мучениче Христов. К мучением усердно стремился еси. С супругою венец прием.


\mysubsubsection{Молитва:}


О священная двоице, святии мученицы Христовы Адриане и Наталие, блаженнии супрузи и доблии страдальцы! Услышите нас, молящихся вам со слезами, и низпослите на ны вся благопотребная душам и телесем нашим, и молите Христа Бога, да помилует нас и сотворит с нами по милости Своей, да не погибнем во гресех наших. Ей, святии мученицы! Приимите глас моления нашего, и избавите ны молитвами вашими от глада, губителъства, труса, потопа, огня, града, меча, нашествия иноплеменников и междоусобныя брани, от внезапныя смерти и от всех бед, печалей и болезней, да присно вашими молитвами и предстательством укрепляеми, прославим Господа Иисуса Христа, Емуже подобает всякая слава, честь и поклонение, со безначальным Его Отцем и Пресвятым Духом, во веки веков. Аминь.\mychapterending

\mychapter{Святому праведному Иоанну, Кронштадтскому чудотворцу}
%http://www.molitvoslov.com/text293.htm 
 
\myfig{184}Св. праведный Иоанн родился в 1829 году в селе Сура Архангельской губернии, в семье бедного сельского дьячка. С юных лет увидев и пережив материальную нужду, Иоанн стал чуток и сострадателен к чужим страданиям, без колебаний раздавая нуждающимся все, что имел. 


В 1855 году Иоанн становится священником Кронштадтского Андреевского собора, женясь на дочери протоиерея этого собора Елизавете Несвитской. Муж и жена всю жизнь прожили как брат с сестрой, храня телесную и душевную чистоту. 


Слава Кронштадтского молитвенника еще при жизни была столь велика, что на исповедь в Андреевский собор собирались ежедневно тысячи народа. Толпы людей сопровождали доброго пастыря везде, где бы он ни появлялся. 


В годы всеобщего духовного и нравственного оскудения, происходившего в России в предреволюционное смутное время, Иоанн Кронштадтский явился непримиримым обличителем лжи «освободительного движения», стал ярким примером несокрушимой веры в Бога, надежды на Его милость к кающемуся человеку, любви к своему страждущему народу. 


Чудеса, совершенные Богом по молитвам св. Иоанна Кронштадтского, изливались прежде и изливаются ныне нескончаемой рекой на Русскую Землю.


\mysubsubsection{Молитва}


О великий чудотворче и предивный угодниче Божий, богоносне отче Иоанне! Призри на нас и внемли благосердно молению нашему, яко великих дарований сподоби тя Господь, да ходатаем и присным молитвенником за нас будеши. Се бо страстьми греховными обуреваеми и злобою снедаеми, заповеди Божия пренебрегохом, покаяния сердечного и слез воздыхания не принесохом, сего ради многим скорбем и печалем достойнем явихомся. 


Ты же, отче праведный, велие дерзновение ко Господу и сострадание к ближним имея, умоли Всещедрого Владыку мира, да пробавит милость Свою на нас и потерпит неправдам нашим, не погубит нас грех ради наших, но время на покаяние милостивно нам дарует. 


О святче Божий, помози нам веру Православную непорочно соблюсти и заповеди Божии благочестно сохранити, да не обладает нами всякое беззаконие, ниже посрамится Правда Божия в неправдах наших, но да сподобимся достигнута кончины христианския, безболезненныя, непостыдныя, мирныя и Тайн Божиих Причастия. 


Еще молим тя, отче праведне, о еже Церкви нашей Святей до скончания века утвержденной быти, Отечеству же нашему мир и пребывание испроси, от всех зол сохрани, да тако народи наши, Богом храними, в единомыслии веры и во всяком благочестии и чистоте, в лепоте духовнаго братства, трезвении и согласии свидетельствуют: яко с нами Бог! В Нем же и движемся и есмы, и пребудем во веки. Аминь.\mychapterending

\mychapter{Святителю Николаю, Мирликийскому чудотворцу}
%http://www.molitvoslov.com/text297.htm 
 
\myfig{188_0}

Святой Николай родился от благородных и благочестивых родителей во времена императора Валериана (ок. 280 г.) в Ликийском городе Патаре. Благодать Божия почивала на нем явно с самого юного возраста, проявляясь в его строгом постничестве и горячей молитве. 

По достижении зрелого возраста св. Николай был поставлен пресвитером, а позже "--- избран епископом города Миры и для всех явился образцом несокрушимой веры, безмерного смирения, чистоты, человеколюбия и милосердия. При гонителе христиан Диоклетиане вместе с другими был подвергнут заключению и освобожден был только Константином Великим. 

Присутствуя в 325 году на I Вселенском Соборе, св. Николай обличил и посрамил лжеучение Ария. Еще при жизни св. Николай именовался «отцом сирот и утешителем страждущих» за свою глубокую христианскую любовь. 

В России почитание святителя Николая столь велико, что нет не только ни одного храма, но и ни одного православного дома, где не имелось бы иконы святого и где ему не возносились бы самые искренние и горячие молитвы.


\mysubsubsection{Молитва}


О всеблагий отче Николае, пастырю и учителю всех, верою притекающих к твоему заступлению и теплою молитвою тебе призывающих, скоро потщися и избави Христово стадо от волков, губящих е, и всяку страну христианскую огради и сохрани святыми твоими молитвами от мирскаго мятежа, труса, нашествия иноплеменников и междоусобныя брани, от глада, потопа, огня, меча и напрасныя смерти. И якоже помиловал еси триех мужей, в темнице седящих, и избавил еси их царева гнева и посечения мечнаго, тако помилуй и мене, умом, словом и делом во тьме грехов суща, и избави мя гнева Божия и вечныя казни, яко да твоим ходатайством и помощию, Своим же милосердием и благодатию, Христос Бог тихое и безгрешное житие даст ми пожити в веце сем и избавит мя шуйяго стояния, сподобит же деснаго со всеми святыми. Аминь.
\longpage[2]{}\mychapterending

\mychapter{Святителю Тихону, Патриарху Московскому и Всея Руси}
%http://www.molitvoslov.com/text292.htm 
 
\myfig{183}Святитель Тихон родился в 1865 году в семье сельского священника Псковской губернии. В 1891 году он принимает монашество, а в 1898 году возводится в сан епископа и направляется в далекую американскую епархию. 

Много сделав для Православной Церкви Америки, святитель Тихон снискал себе всеобщую любовь и уважение. 

В 1907 году он был назначен на Ярославскую кафедру, где также его окружали поклонение и благодарность пастырей и паствы. 

После февральской революции 1917 года архиепископа Тихона пригласили в число членов вновь сформированного Синода, а в ноябре того же года Поместный Собор Русской Православной Церкви избирает его Патриархом. Патриаршее служение Святителя, выпавшее на годы, поистине страшные для России, является для нас в XX веке образцом мужественной и взвешенной защиты родной Православной Церкви от натиска безбожного мира.


\mysubsubsection{Молитва}


О пастырю наш добрый, святый великий Патриарше Тихоне, яко град горний ты явился еси "--- добрая дела твоя и доныне светятся пред человеки. Вемы, яко ты, предстоя престолу Пресвятыя Троицы, велие имаши дерзновение в молитвах пред Господем. Воззри и ныне на нас, грешных и недостойных чад твоих, к тебе бо, яко имущему велие дерзновение пред Творцом всяческих, ныне припадаем и усердно молимся: умоли Господа, да подаст нам решимость стяжать благочестие отцев наших, егоже ты стяжал еси от юности твоея. Ты в житии своем ревностный защититель и хранитель истинныя веры был еси, помози и нам незыблемо соблюсти веру Православную. Тихая бодуша твоя зело преуспела в божественном смиренномудрии, научи и нас разум наш питати не многомятежной мудростию человеческой, но смиренным познанием воли Божией. Ты пред лицем лютых врагов Христовых Истиннаго Бога дерзновенно исповедал еси, молитвою своею укрепи нас, малодушных, да и мы всегда и всюду противостанем духу безбожия и льсти. 

Ей, угодниче Божий, не презри нас, молящихся тебе, ибо не токмо от бед и скорбей избавления просим, но силы и твердости, великодушия и любви просим, дабы переносить оныя напасти, востающия на ны. Испроси нам неослабное терпение даже до конца жития нашего, мир с Господом и грехов отпущение. 

Отче святый! Укроти в стране нашей ветры неверия и смуты, да водворит Господь на Земле Российстей тишину и благочестие и любовь нелицемерную. Молитвами твоими да сохранит ю от междоусобныя брани, да укрепит Святую Церковь нашу Православную, да не оскудеет она истинными пастырями, добрыми делателями, право правящими слово Евангельской истины. Упаси и заблуждения овцы стада Христова. Наипаче же моли Господа сил, да возродится Русская Земля святым покаянием и единым сердцем и едиными усты прославит Дивнего во святых Своих Бога в Троице славимаго, Отца и Сына и Святаго Духа во веки веков. Аминь.
\mychapterending

\mychapter{Преподобному Алексию, человеку Божию}
%http://www.molitvoslov.com/text302.htm 

\myfigh[0.23]{192}{16}

\mysubsubsection{Тропарь, глас 4-й:}


Возвысився на добродетель и ум очистив, к желанному и крайнему достигл еси, безстрастием же украсив житие твое и пощение изрядное восприим совестию чистою, в молитвах, яко безплотен, пребывая, возсиял еси, яко солнце, в мире, преблаженне Алексие.


\mysubsubsection{Кондак, глас 2-й:}


Дом родителей твоих, яко чужд, имев, водворился еси в нем нищеобразно и, по преставлении венец прием славы, дивен на земли явился еси, Алексие, человече Божий, Ангелом и человеком радование.


\mysubsubsection{Молитва:}


О, великий Христов угодниче, святый человече Божий Алексие, душею на небеси Престолу Господню предстояй, на земли же данною ти свыше благодатию различная совершаяй чудеса! Призри милостивно на предстоящий святей иконе твоей люди, умиленно молящиеся и просящие от тебе помощи и заступления. Простри молитвенно ко Господу Богу честнеи руце твои и испроси нам от Него оставление согрешений наших вольных и невольных, в недузех страждущим исцеление, напаствуемым заступление, скорбящим утешение, бедствующим скорую помощь, всем же чтущим тя мирную и христианскую живота кончину и добрый ответ на Страшнем Суде Христове. Ей, святче Божий, не посрами упования нашего, еже на тя по Бозе и Богородице возлагаем, но буди нам помощник и покровитель во спасение, да твоими молитвами получивше благодать и милость от Господа, прославим человеколюбие Отца и Сына и Святаго Духа, в Троице славимаго и покланяемаго Бога, и твое святое заступление ныне и присно, и во веки веков. Аминь.
\mychapterending

\mychapter{Mолитва Небесным печальникам Русской Земли}
%http://www.molitvoslov.com/text280.htm 
 


О великие печальники и заступники наши пред Богом за Землю Российскую! Святители Христовы: Петре, Алексие, Ионо и Филиппе, и священномучениче Ермогене, преподобные отцы наши Антоние и Феодосие, Зосиме и Савватие, Сергие и Никоне, и все новые чудотворцы, во дни наши Богом прославленные: Феодосие, Серафиме, Иоасафе, Питириме, Иоанне и Святые Патриархи Иове и Тихоне и прочий угодники Божий, в пределах Руси Православной просиявшие! Услышите многоскорбную молитву нас грешных русских людей: наша, некогда Святая Русь, ваше отечество, погибает от беззаконий сынов своих. Уже колеблется спасетельное православие, уже сверкает меч Божий над безплодным древом Руси грешной, уже слышится в совести нашей грозный приговор Господа: отьимется от вас Царствие Божие и дастся народу, творящему плоды его. Мы все же дерзаем именовать вас своими родными: вы родные нам по крови, по земному вашему отечеству, вы "--- наши самые близкие заступники пред Богом. Зрите скорбь нашу великую, зрите беду нашу лютую, общую беду родной вам земли… Придите же на помощь ей, умолите Владычицу мира "--- Матерь Божию, да смилуется Она, Милосердая, над нами грешными; тысячу лет Она покрывала, заступала Русь во всех бедах и напастях ея: да не отвратит и ныне лица Своего от народа Русскаго за грехи наши, да станет между нами и прогневанным Сыном Своим и Богом, и, воздевая пречистыя Своя руки, паки и паки да умолит Его милосердие преложити гнев на милость ради той крови, которой так много пролито нашими воинами за Божию правду, за братий своих, за веру православную и родную землю, ради горьких сирот, оставшихся безпомощными после убиенных наших братий, ради нищих и убогих, лишаемых хлеба насущнаго, ради тысяч искалеченных наших воинов, ради всех скорбящих и обремененных, чающих Христова утешения и Ея Матерняго заступления. О святые угодники Божий, наши по плоти сродники! Подвигните на молитву за нас и всемирнаго, столь Русью любимаго великаго Чудотворца Николая и всех апостолов, пророков, мучеников, святителей, преподобных и праведных; подвигните Архангелов и Ангелов и всю Церковь, на небесах торжествующаю: ваша родная Русь в великой опасности и в час грознаго Суда Божия над нею к вам взывает: спасите ее молитвенным заступлением вашим! Ведаем, грешные, что доколе мы живы, дотоле еще не затворились двери милосердия Божия, дотоле Господь готов еще принять и наше покаяние: испросите же нам у Него самое покаяние, да наказуя накажет нас, смерти же и погибели да не предаст народа нашего. Аминь.
\longpage{}\mychapterending

\mychapter{Преподобному Сергию Радонежскому,  Всея России чудотворцу}
%http://www.molitvoslov.com/text285.htm 
 
\myfig{177_0}

Преподобный Сергий Радонежский "--- один из самых прославленных русских святых. Основатель Троице-Сергиевой Лавры, учитель и наставник многих десятков русских святых, канонизированных Церковью, Преподобный стал поистине игуменом и заступником всей Русской Земли, образцом кротости и смирения для монахов и мирян.

Болезнуя сердцем за Отчизну в тяжкие годы татарского ига, Преподобный Сергий благословлял на Куликовскую битву св. князя Димитрия Донского, молился за победу, поминал денно и нощно тех, кто отдал свою жизнь за Родину.




Преподобный Сергий Радонежский "--- горячий и верный служитель Святой Троицы как образа Божественной Любви в истории Русской Церкви.


\mysubsubsection{Молитва}


О священная главо, преподобне и богоносне отче наш Сергие, молитвою твоею, и верою и любовию, яже к Богу, и чистотою сердца, еще на земли во обитель Пресвятыя Троицы душу твою устроивый, и ангельскаго общения и Пресвятыя Богородицы посещения сподобивыйся, и дар чудодейственныя благодати приемый, по отшествии же твоем от земных, наипаче к Богу приближивыйся и небесныя силы приобщивыйся, но и от нас духом любве твоея неотступивый, и честныя твоя мощи, яко сосуд благодати полный и преизливающийся, нам оставивый! Велие имея дерзновение ко всемилостивому Владыце, моли спасти рабы Его, сущей в тебе благодати Его верующия и к тебе с любовию притекающия. Помоги нам, да благоуправляемо будет Отечество наше в мире и благостоянии, и да покорятся под ноги его все сопротивнии. Испроси нам от великодаровитаго Бога нашего всякий дар, всем и коемуждо благопотребен: веры непорочны соблюдение, градов наших утверждение, мира умирение, от глада и пагубы избавление, от нашествия иноплеменных сохранение, скорбящим утешение, недугующим исцеление, падшим возставление, заблуждающим на путь истины и спасения возвращение, подвизающимся укрепление, благоделающим в делах благих преспеяние и благословение, младенцем воспитание, юным наставление, неверующим вразумление, сиротам и вдовицам заступление, отходящим от сего временнаго жития к вечному благое уготовление и напутствие, отшедшим блаженное упокоение, и вся ны споспешествующими твоими молитвами сподоби, в день Страшнаго Суда шуия части избавитися, десныя же страны общники быти и блаженный оный глас Владыки Христа услышати: приидите, благословеннии Отца Моего, наследуйте уготованное вам Царствие от сложения мира.
\mychapterending

\mychapter{Cвятому благоверному Великому Князю Александру Невскому}
%http://www.molitvoslov.com/text284.htm 
 
\myfig[0.39]{176}

Св. Александр родился в 1220 году; в 1240 году, когда шведы напали на Северную Русь, был Новгородским князем; одержал блестящую победу над шведами на Неве, за что и получил прозвание Невский. Спустя два года, в 1242 году, разбил немецких рыцарей-захватчиков при Чудском озере. 

Св. Александр был мудрым государственным деятелем, мирившим русских князей между собой: он старался смягчить для Руси иго татарских ханов, вождей Золотой Орды. 

Являя в личной жизни пример христианского благочестия и чистоты, св. Александр Невский был любим и почитаем русским народом как великий печальник и заступник Русской Земли.


\mysubsubsection{Молитва}

Скорый помощниче всех, усердно к тебе прибегающих и теплый наш пред Господем предстателю, святый благоверный великий княже Александре! Призри милостивно на ны недостойныя, многими беззаконии непотребны себе сотворшия, к раце мощей твоих (или иконе твоей) ныне притекающия и из глубины сердца к тебе взывающия: ты в житии своем ревнитель и защитник Православныя веры был еси, и нас в ней теплыми твоими к Богу молитвами непоколебимы утверди. Ты великое, возложенное на тя, служение тщательно проходил еси, и нас твоею помощию пребывати коегождо, в неже призван есть, настави. Ты, победив полки супостатов, от пределов Российских отгнал еси, и на нас ополчающихся всех видимых и невидимых врагов низложи. Ты, оставив тленный венец царства земнаго, избрал еси безмолвное житие, и ныне праведно венцем нетленным увенчанный, на небесех царствуеши, исходатайствуй и нам, смиренно молим тя, житие тихое и безмятежное и к вечному Царствию шествие неуклонное твоим предстательством устрой нам. Предстоя же со всеми святыми престолу Божию, молися о всех православных христианах, да сохранит их Господь Бог Своею благодатию в мире, здравии, долгоденствии и всяком благополучии в должайшая лета, да присно славим и благословим Бога, в Троице Святей славимаго Отца и Сына и Святаго Духа, ныне и присно и во веки веков. Аминь.\longpage[2]{}\mychapterending

\mychapter{Mолитва всем святым, в земле Российской просиявшим}
%http://www.molitvoslov.com/text279.htm 
 


О, преславнии и всехвальнии угодницы Божии, вси святии ведомыя и неведомыя в земле Российстей просиявшии! К вам, яко к теплым ходатаям и предстателям нашим с любовию прибегаем и смиренно молимся: умолите Господа Бога, да вашими благоприятными молитвами, Своим же человеколюбием да дарует нам (имена) тихое и благочестное житие в веце сем пожити, да избавит нас от искушений и соблазнов лукаваго диавола и от бед и напастей, и от всякаго зла; на Страшнем же суде Своем да сподобит нас деснаго стояния и наследники Царствия Своего Небеснаго да сотворит, яко благословися пречестное и великолепое имя Отца и Сына и Святаго Духа, ныне и присно и во веки веков. Аминь.
\mychapterending

\mychapter{Вере, Надежде, Любви и матери их Софии}
%http://www.molitvoslov.com/content/Vere-Nadezhde-Lyubvi-i-materi-ikh-Sofii 
 


О святыя и достохвальныя мученицы Веро, Надеждо и Любы, и доблестных дщерей мудрая мати Софие, к вам ныне притецем со усердною молитвою; что бо паче возможет предстательствовати за ны пред Господем, аще не вера, надежда и любы, три сия краеугольныя добродетели, в нихже образ нареченныя, самою вещию тыя явисте! Умолите Господа, да в скорбех и напастех неизреченною благодатию Своею покрыет ны, спасет и сохранит, яко благ есть и Человеколюбец. Того славу, яко солнце незаходимое, ныне зряще светолепну, споспешествуйте нам во смиренных молениих наших, да простит Господь Бог грехи и беззакония наша, и да помилует нас грешных и недостойных щедрот Его. Молите убо о нас, святыя мученицы, Господа нашего Иисуса Христа, Емуже славу возсылаем со Безначальным Его Отцем и Пресвятым и Благим и Животворящеим Его духом, ныне и присно и во веки веков. Аминь.
\mychapterending

\mychapter{Святым равноапостольным Кириллу и  Мефодию}
%http://www.molitvoslov.com/content/svyatym-ravnoapostolnym-kirillu-i-mefodiyu 
 


О святии равноапостольнии Мефодие и Кирилле! Ныне к вам усердно прибегаем и в сокрушении сердец наших молимся: молите Господа, да наставит и обратит нас, на путь спасения. Не оставите нас, недостойных чад ваших (имена), дадите нам, молитвами вашими, о православии ревность, да ею возгреваеми, отеческая предания добре сохраним, уставы и обычаи церковныя верно соблюдати  потщимся, всяких лжеучений странных отбежим, и тако в житии богоугоднем на земли преспевающе, жизни райския на небеси сподобимся, и тамо с вами вкупе Владыку всех, в Троице Единаго Бога, прославим во веки веков.
\mychapterending

\mychapter{Святой равноапостольной княгине Ольге}
%http://www.molitvoslov.com/content/svyatoi-ravnoapostolnoi-knyagine-olge 
 


О святая равноапостольная княгине Ольго, приими убо похвалу от нас, недостойных раб Божиих (имена), пред  честною твоею иконою молящихся и смиренно просящих: огради нас твоими молитвами и заступлением от напастей и бед,  и печалей, и лютых грехов; еще же и от будущих мук избави ны, честно творящия святую память твою и славящия прославльшаго тя Бога, во Святей Троице прославляемаго, Отца и Сына и Святаго Духа, ныне и присно и во веки веков
\mychapterending

\mychapter{Преподобному Даниилу Московскому}
%http://www.molitvoslov.com/content/prepodobnomu-daniilu-moskovskomu 
 


О преподобне княже Данииле, к иконе твоей притекающе, усердно молим тя: призри на нас (имена), с верою под  кров молитв твоих прибегающих.  Пролей теплое твое ходатайство ко Спасу всех,  яко да утвердит миром приход сей и храм сей добре да сохранит, благочестие и любовь в людех православных насаждая, злобу же, междоусобие и нравов развращение  искореняя; всем же нам вся благая ко временному животу и вечному спасению даруй молитвами твоими, яко да прославляем дивнаго во святых своих Христа Бога нашего вкупе со Отцем и Святым Духом во веки веков. Аминь.
\mychapterending

\mychapter{Святым равноапостольным Константину и Елене}
%http://www.molitvoslov.com/content/svyatym-ravnoapostolnym-konstantinu-i-elene 
 


О святии равноапостольнии Константине и Елено! Избавите приход сей и храм наш от всякаго навета вражия и не оставите заступлением вашим нас, немощных (имена), умолите благость Христа Бога нашего даровати нам помыслов мир, от пагубных страстей и всякия скверны воздержание , благочестие же нелицемерное.  Испросите нам, угодницы Божии, свыше дух кротости и смиренномудрия, дух терпения и покаяния,  да прочее время жития нашего в вере и сокрушении сердечнем поживем, и тако в час скончания нашего благодарне восхвалим прославльшаго вас Господа, Безначальнаго Отца, Единороднаго Его Сына и Единосущнаго Всеблагаго Духа, Троицу Нераздельную, во веки веков.
\mychapterending

\mychapter{Святому благоверному князю Вячеславу Чешскому}
%http://www.molitvoslov.com/content/svyatomu-blagovernomu-knyazyu-vyacheslavu-cheshskomu 
 


О святый княже Вячеславе! Усердно просим тя молитися за ны (имена), да простит Господь Бог наши согрешения вольные и невольные и очистит нас от всякия скверны плоти и духа, да избавит нас от козней диавола и сохранит от клеветы людския, да утвердит нас в истинной вере и благочестии, да соблюдет от суемудрых и душетленных учений, да сохранит сердца наша от соблазнов мира сего и да научит огребатися\footnote{Воздерживаться, отвращаться} от плотских страстей и похотей, да тако горняя мудрствовати, а не земная, прославляя Единосущную Троицу во веки и в век века. Аминь.
\mychapterending

\mychapter{Святому князю Олегу Брянскому}
%http://www.molitvoslov.com/content/svyatomu-knyazyu-olegu-bryanskomu 
 


О преподобне отче, благоверне княже Олеже! Воззри, преблаженне, милостивно на дом (приход) сей и люди, в нем живущия, и не отвержи требующих твоея помощи, не презри нас, молящихся тебе, но помози нам (имена) скорым предстательством твоим, да всяких бед, напастей и скорбей в сей жизни временней избежавше, кончину непостыдну обрящем, и тако на земли богоугодно поживше, жизни райския на небеси сподобимся, идеже вкупе с тобою прославим человеколюбие и щедроты в Троице славимаго Бога, Отца и Сына и Святаго Духа, во веки веков. Аминь.
\mychapterending

\mychapter{Святой равноапостольной Нине, просветительнице Грузии}
%http://www.molitvoslov.com/content/svyatoi-ravnoapostolnoi-nine-prosvetitelnitse-gruzii 
 


О всехвальная и предивная равноапостольная Нино, к тебе прибегаем и умильно тебе просим: огради нас (имена) от всяких зол и скорбей, вразуми врагов святыя Церкви Христовы и посрами противников благочестия и умоли Всеблагаго Бога Спасителя нашего, Емуже ты ныне предстоиши, да дарует народу православному мир, долгоденствие и во всяком добрем начинании поспешение, и да приведет Господь нас в Небесное Свое Царствие, идеже  вси святии славословят всесвятое Его имя, ныне и присно и во веки веков. Аминь.
\mychapterending

\mychapter{Молитва Святителю Иоанну Шанхайскому}
%http://www.molitvoslov.com/content/Molitva-Prepodobnomu-Serafimu-Sarovskomu 
 
\myfig{ioann_max_1}

\mysubsubsection{Святитель Иоанн (Максимович) Шанхайский и Сан-Францисский Чудотворец}


Строгий аскет, кормитель обездоленных, безмездный целитель, Христа ради юродивый, апостол последних времен, подвизавшийся в Азии, Европе, Америке. Никогда не ложился спать, постоянно бодрствовал, ночи его проходили в молитве. Сам Господь открывал ему, кто нуждается в помощи. Владыка проходил сквозь стены, изгонял бесов из одержимых, отвечал на незаданные вслух вопросы; исцелял безнадежно больных, обреченных на смерть людей; неожиданно являлся там, где был особенно необходим; заранее знал нужды и печали приходивших к нему. От него исходила сила, привлекавшая людей более, чем бесчисленные чудеса: то была сила любви Христовой. Еще при жизни на просьбы, обращенные к Владыке, он не только находил слова утешения, но и сразу же начинал действовать. Для верующих всего мира святитель Иоанн Чудотворец "--- скорый помощник всех сущих в бедах, болезнях, в скорбных и опасных обстоятельствах, хранитель путешествующих, утешитель страждущих.


\mysubsubsection{Тропарь, глас 5}


Попечение твое о пастве в странствии ея, / се прообраз и молитв твоих, за мир весь присно возносимых: / тако веруем, познав любовь твою, святителю и чудотворче Иоанне! / Весь от Бога освящен священнодействием пречистых Тайн, / имиже сам присно укрепляем, / поспешал еси ко страждущим, целителю отраднейший. / Поспеши и ныне в помощь нам, всем сердцем чтущим тя.


\mysubsubsection{Величание}


Величаем, величаем тя, святителю отче наш Иоанне, и чтим святую память твою. Ты бо молиши за нас Христа Бога нашего.


\mysubsubsection{Молитва}


О, святителю отче наш Иоанне, Пастырю добрый и тайновидец душ человеческих. Ныне у Престола Божия за нас молишися, яко же и сам посмертне изрек: «Хотя я и умер "--- но я жив». Умоли Всещедраго Бога прощение нам во гресех даровати, да смело воспрянем духом и уныние сего мира отряхнем и Богу возопиим о даровании нам смирения и Богодухновения, Богосознания и духа благочестия на всяких путях жизни нашея. Яко милостивый сиропитатель и опытный путеводитель на земле бывший, ныне буди нам вождем Моисеовым и в смутах церковных всеобъемлющее Христово вразумление. Услыши стенание смущенных юношей нашего лихолетия, обуреваемых беснованием вселукавым и отряхни лень уныния изнемогающих пастырей от натисков духа мира сего и томящихся в оцепенении праздном. Да слезно вопием ти, о теплый молитвенниче, посети нас сирых, во тьме страстей утопающих, ждущих твоего отеческаго наставления, да озаримся светом невечерним, идеже ты пребываеши и молишися за чад твоих, по лицу вселенныя рассеянных, но любовию слабою к свету все же тянущихся, идеже свет Христос Господь наш пребывает, Ему же честь и держава ныне и присно и во веки веков. Аминь.
\mychapterending
