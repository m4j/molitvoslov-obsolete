

\mypart{КАНОН ПРЕПОДОБНОМУ ПАИСИЮ ВЕЛИКОМУ ОБ ИЗБАВЛЕНИИ ОТ МУК УМЕРШИХ БЕЗ ПОКАЯНИЯ}
%http://www.molitvoslov.com/text903.htm



Преподобный Паисий Великий родился в Египте. С детства полюбил иноческую жизнь. Отрекшись от своей воли, он жил под духовным руководством святого Павмы, во всем исполняя его повеления. Святой подвижник прилежно читал духовные книги и особенно прославился подвигом поста и молитвы. Главный завет преподобного Паисия был один: ничего но делать по своей воле, а во всем исполнять волю своих наставников. Тяготясь нарушением безмолвия, преподобный удалился в более отдаленную пещеру. Однажды он был восхищен в райские обители и удостоился там причаститься невещественной Божественной пищи. Преподобный Паисий отличался великим смирением, совершал подвиги поста и молитвы, но, по возможности, скрывал их от посторонних. На вопрос иноков, какая добродетель выше всех, преподобный ответил: «Та, которая совершается в тайне и о которой никто не знает». Преподобный Паисий имеет от Бога благодать избавлять от вечной муки умерших без покаяния. Память преподобному Паисию Великому 19 июня (2 июля по гражданскому стилю). Канон взят из старинного канонника, на него ссылается как на поемый за избавление от муки без покаяния умерших в своей книге «О поминовении усопших по уставу Православной Церкви» авторитетный знаток уставного богослужения епископ Афанасий (Сахаров).





\bfseries Тропарь, глас 2-й\normalfont{}


Божественною любовию от юности распалаемь, преподобне, вся красная, яже в мире, возненавиде, Христа единаго возлюбил еси, сего ради в пустыню вселися,идеже сподобися Божественнаго посещения, Егоже неудобь зрети и ангельскима очима, паде, поклонися. Великий же Дародатель, яко Человеколюбец рече к тебе: не ужасайся, возлюбленный Мой, дела твоя угодна Мне. Се даю тебе дар: о коем-либо грешнице помолишися, отпустятся ему греси. Ты же в чистоте сердца твоего возгореся, прием воду и прикоснуся Неприкосновенному, умы нозе Его и, воду пив, даром чудес обогатися, болящия исцеляти, бесы от человек отгоняти и грешники от муки молитвою своею избавляти. О преподобне отче Паисие, молю тя, да умолиши и о мне, якоже тебе Бог обетова, ибо от сих грешник первый аз есмь, да даст ми Господь время покаяния и простит мое согрешение, яко Благий и Человеколюбец, да со всеми и аз воспою Ему: аллилуия. \itshape (Дважды)\normalfont{}


Слава Отцу и Сыну и Святому Духу. И ныне и присно и во веки веков. Аминь.


\itshape Богородичен.\normalfont{} Вся паче смысла, вся преславная Твоя, Богородице, таинства, чистотою запечатленна, и девством хранима, Мати разумеся неложна, Бога рождши истиннаго, Того моли спастися душам нашим.





\bfseries Канон, глас 6-й\normalfont{}





\bfseries Песнь 1\normalfont{}


\itshape Ирмо́с:\normalfont{} Помощник и Покровитель бысть мне во спасение, Сей ми есть Бог, и прославлю Его. Бог отца моего и вознесу Его, славно бо прославися.


\itshape Припев:\normalfont{} Преподобне отче Паисие Великий, моли Бога о нас.


Иже вся умудряющему Богу молися, преподобне о рабе своем, да отверзет ми устне недостоинии, и двигнет ми язык недоуменный. Тесноту же и худогласие, отче, разверзи благодатию, иже в тебе Святаго Духа, к пению чудес твоих.


\itshape Припев:\normalfont{} Преподобне отче Паисие Великий, моли Бога о нас.


Начну убо достохвальное и душеполезное житие твое от младенства. Иже древле Египет великаго во пророцех произнесе Моисея, иже к Богу присвоением и великими чудесы прославися. Тако и ныне вторицею прославлен показася Египет, тебе ради, отче Паисие, честным именем твоим и многими добродетельми обогатися, еже дарова тебе Господь, Егоже моли, да спасет душа наша.


Слава Отцу и Сыну и Святому Духу.


В Царство Небесное вшед узким путем и прискорбным по заповеди Владыки своего Христа, преподобне отче Паисие, широкий и пространный путь возненавидев, распространьшийся мрак ума моего отжени, да возмогу малое сие моление принести в пречестную память твою.


И ныне и присно и во веки веков. Аминь.


\itshape Богородичен.\normalfont{} Дух сокрушен подаждь ми, Благая, смиренно сердце, уму же чистоту и житию исправление и прегрешением оставление.


\itshape Катавасия:\normalfont{} Избави от муки вечныя усопшаго раба своего \itshape (назови имя)\normalfont{}, преподобне отче Паисие Великий, яко вси по Бозе к тебе прибегаем, ты бо молиши о нас Христа Бога нашего ( \itshape поклон\normalfont{}). Господи, помилуй ( \itshape трижды с поклонами\normalfont{}).


\itshape Или сия катавасия:\normalfont{} Избави от бед рабы своя, преподобне отче Паисие, яко вси по Бозе к тебе прибегаем, ты бо молиши о нас Христа Бога нашего (\itshape поклон\normalfont{}). Господи, помилуй ( \itshape трижды с поклонами\normalfont{}).





\bfseries Песнь 3\normalfont{}


\itshape Ирмо́с:\normalfont{} Утверди, Господи, на камени заповедей Ти подвигшееся сердце мое, яко Един Свят еси и Господь.


\itshape Припев:\normalfont{} Преподобне отче Паисие Великий, моли Бога о нас.


Каменю веры Петру апостолу уподобился еси, распятся мирови во всей жизни своей, преподобне Паисие, и нозе свои невозвратно к небесному шествию управил еси, и горняго Иерусалима достигл, со святыми предстоя Святей Троице, моли о мне единаго Благаго Человеколюбца.


\itshape Припев:\normalfont{} Преподобне отче Паисие Великий, моли Бога о нас.


Благочестиваго корени пресветлая отрасль, тебе избра Господь, ангел рече матери твоей, сей Богу угоден. Ты же от младенства крест свой взем, невозвратным путем Тому последова, и возрасташе леты и разумом купно и благодатию Божиею. Моли дати ми отпущение грехов.


Слава Отцу и Сыну и Святому Духу.


О, Владыко Господи Исусе Христе, кто не удивится Твоему человеколюбию, егда прииде желание преподобному иноческо житие получити к деланию заповедий Твоих, Твоею благодатию яко агнец непорочен иэведеся в пустыню, и достигл словесных овец, и введеся к пастырю блаженному Памве, и облечеся во иноческий образ, в немже утверди и мене Господи, молитвами преподобнаго Паисия, к деланию заповедий Твоих.


И ныне и присно и во веки веков. Аминь.


\itshape Богородичен.\normalfont{} Лютых мя мук, и тмы кромешныя и геенны, Твоими молитвами свободи, Дево, имаши бо волю и силу, Господа рождшая единаго Преблагаго.


\itshape Катавасия:\normalfont{} Избави от муки вечныя усопшаго раба своего ( \itshape назови имя\normalfont{}), преподобне отче Паисие Великий, яко вси по Бозе к тебе прибегаем, ты бо молиши о нас Христа Бога нашего ( \itshape поклон\normalfont{}). Господи, помилуй ( \itshape трижды с поклонами\normalfont{}).





\bfseries Седален, глас 2-й\normalfont{}


Душу связав любовию Христовою, земная возненавидев вся мудре, водворился еси, отче преподобне, в пустынях и горах, разумнаго древа вкуш славне, ангельски просиял еси. Темже и мрак прошед плоти своея, тму отгнал еси бесов Паисие, инокующым первый, моли Христа Бога, грехов оставление подати, чтущым любовию святую память твою.


Слава Отцу и Сыну и Святому Духу. И ныне и присно и во веки веков. Аминь.


\itshape Богородичен.\normalfont{} Скорый покров и помощь и милость покажи на рабе Своем, Чистая, и волны укроти суетных помышлений, и падшую ми душу возстави, Богородице, вем бо, вем яко можеши, елико хощеши.





\bfseries Песнь 4\normalfont{}


\itshape Ирмо́с:\normalfont{} Услыша пророк пришествие Твое, Господи, и убояся, яко хощеши от Девы родитися и человеком явитися, и глаголаше: услышах слух Твой и убояхся, слава силе Твоей, Господи!


\itshape Припев:\normalfont{} Преподобне отче Паисие Великий, моли Бога о нас.


Законострадальческое житие от мягких ногтей восприим богомудре Паисие, даже до конца пребыл еси, яко божествен доблественник венец приял еси победы от всех Царствующаго, молитвами своими избавляти от муки грешники, от них же первый есмь аз, не забуди мене.


\itshape Припев:\normalfont{} Преподобне отче Паисие Великий, моли Бога о нас.


Имуще молитвенника крепка блаженне Паисие, и обещника печальным, предстателя и поборника и ходатая благочестна, от всяких спасаемся бед, напастей и обстояний.


Слава Отцу и Сыну и Святому Духу.


Велию судеб бездну имеяй, Той тебе рече: се, бо даю ти дар, да все еже просиши от Отца Моего во имя Мое, дастся тебе, о коем либо грешнике помолишися, оставятся греси того: сего ради припадаю ти, отче Паисие, имея бездну согрешений, да отпустит твоими молитвами Иже тебе обетова, яко Благ и Человеколюбец.


И ныне и присно и во веки веков. Аминь.


\itshape Богородичен.\normalfont{} Глагол провещай, егда право имам судитися, Богородительнице, к Своему Сыну, Пренепорочная, да обрящу Тя тогда прибежище и заступление державнейшее, и мук всех избавляющи.


\itshape Катавасия:\normalfont{} Избави от муки вечныя усопшаго раба своего ( \itshape назови имя\normalfont{}), преподобне отче Паисие Великий, яко вси по Бозе к тебе прибегаем, ты бо молиши о нас Христа Бога нашего ( \itshape поклон\normalfont{}). Господи, помилуй ( \itshape трижды с поклонами\normalfont{}).





\bfseries Песнь 5\normalfont{}


\itshape Ирмо́с:\normalfont{} Из нощи утреннююща, Человеколюбче, просвети, молюся, и настави мя на повеления Твоя, и научи мя, Спасе, творити волю Твою.


\itshape Припев:\normalfont{} Преподобне отче Паисие Великий, моли Бога о нас.


Постом удручая тело свое отче Паисие, иногда по причащении Тела и Крови Христовы, яко ангел седмьдесят дней без пищи телесныя пребывая, божественную державу неизреченную имущи, и возмогши в себе животную содержати силу благодатию Твоею, паче пищнаго укрепления, слава державе Твоей, Господи.


\itshape Припев:\normalfont{} Преподобне отче Паисие Великий, моли Бога о нас.


Яко Иоанн Креститель узким и прискорбным путем ходити произволил еси. Но той на Иордане узрев Зиждителя своего, устрашися и вопияше: не смею приступити сено огню. Ты же, отче Паисие, в пустыни иногда явльшагося Господа, не могий пречистаго лица Его зрети, пад трепетом одержим. Он же ти рече: не устрашайся, сию пустыню наполню постниками тебе. С нимиже молим, отче, не забуди нас, молитвенник твоих, помилованных быти.


Слава Отцу и Сыну и Святому Духу.


Аз бо в начале моя пред Тобою зрю согрешения, и о моих беззакониих молю Твое милосердие, да простиши и покрыеши благоутробием Твоим множество грехов моих, и даждь ми прочее жизни сея время безгрешно проводити. Да удобь на спасения стезю потек, безпреткновенно к доброму концу достигну Твоею помощию, кроме бо Твоея помощи и наставления ничтоже благо совершити, и милость у Тебе получити кто может.


И ныне и присно и во веки веков. Аминь.


\itshape Богородичен.\normalfont{} Рабское приношу Ти моление неразумный аз, и к Твоему прибегаю благоутробному милосердию, не отврати мене, Чистая, посрамлена.


\itshape Катавасия:\normalfont{} Избави от муки вечныя усопшаго раба своего ( \itshape назови имя\normalfont{}), преподобне отче Паисие Великий, яко вси по Бозе к тебе прибегаем, ты бо молиши о нас Христа Бога нашего ( \itshape поклон.\normalfont{}). Господи, помилуй ( \itshape трижды с поклонами\normalfont{}).





\bfseries Песнь 6\normalfont{}


\itshape Ирмо́с:\normalfont{} Возопих всем сердцем моим к щедрому Богу, и услыша мя от ада преисподняго, и возведе от тли живот мой.


\itshape Припев:\normalfont{} Преподобне отче Паисие Великий, моли Бога о нас.


Бездушная тварь Иордан своего Творца устыдеся, вопияше: не могу Безгрешнаго омыти. Святый же виде Господа, приим воду, прикоснуся Неприкосновенному, умы нозе Его. И пив воду, и прия дар целити недуги и прогонити от человек бесы. Сего ради и мы припадаем ти, отче, молитвами твоими помилуй нас от всяких бесовских наведений.


\itshape Припев:\normalfont{} Преподобне отче Паисие Великий, моли Бога о нас.


По славе Господа нашего Исуса Христа, обретохом тя, отче Паисие, великое прибежище, и заступника и тепла молитвенника о гресех наших. Яко же первее инока преставльшася, и христианства отчуждившася и во глубину адову пагубным безверием низведеся, и егда ощутився притече к тебе, да умолиши Всемилостиваго. Щедрый же Господь явися тебе и глагола: о, угодниче Мой, добро есть Моей любви подобяся, печешися о грешных, изволяяй муки прияти за их избавление.


\itshape Припев:\normalfont{} Преподобне отче Паисие Великий, моли Бога о нас.


Принесеся молитва твоя и слезы к Богу, яко благовонный фимиам, и жертва непорочна и благоприятна, и мене не забуди ходатайством твоим ко Пресвятей Богородице, слезы умиления даси ми, омый грехов моих бездну, избави мя от смертныя глубины. И не надеющымся подаждь велию милость, и прости им согрешения.


Слава Отцу и Сыну и Святому Духу.


Весь недоумением одержим есмь, егда прииму во уме той час страшный испытания Судии и Бога, и плачуся и сетую и рыдаю, поминая бездну зол моих. Темже спаси, Человеколюбче, молитвами Твоего угодника преподобнаго Паисия, и избави мя муки, яко Благоутробен.


И ныне и присно и во веки веков. Аминь.


\itshape Богородичен.\normalfont{} От сердца стенания приношаю Ти, Пренепорочная, Твое прося благопребытное заступление. Помилуй всестрастную ми душу, умилосердися, Мати многомилостиваго Бога, избавити мя Суда и езера огненнаго.


\itshape Катавасия:\normalfont{} Избави от муки вечныя усопшаго раба своего ( \itshape назови имя\normalfont{}), преподобне отче Паисие Великий, яко вси по Бозе к тебе прибегаем, ты бо молиши о нас Христа Бога нашего ( \itshape поклон\normalfont{}). Господи, помилуй ( \itshape трижды с поклонами\normalfont{}).


Слава Отцу и Сыну и Святому Духу. И ныне и присно и во веки веков. Аминь.





\bfseries Кондак, глас 2-й.\normalfont{}


Житейских молв оставль, безмолвное житие возлюбил еси, Крестителю подобяся всеми образы, с нимже тя почитаем, отче отцев Паисие.





\bfseries Икос\normalfont{}


Христов глас услышав, шествовал еси во след Того заповедий, наг жития быв, отверг попечения и вся стяжания и имения, и братий своих и любве матере, богоносне Паисие, един в пустынях Богови беседуя разумом, дарования приял еси, еже ми посли в песнех поющему, отцев начальниче Паисие.





\bfseries Песнь 7\normalfont{}


\itshape Ирмо́с:\normalfont{} Согрешихом, беззаконновахом, неправдовахом пред Тобою, ниже соблюдохом, ниже сотворихом, якоже заповеда нам, но не предаждь нас до конца, отцев Боже.


\itshape Припев:\normalfont{} Преподобне отче Паисие Великий, моли Бога о нас.


Многих грехов исполнен аз, и твоя молитва яко кадило благовонно исправися отче. Бездну грехов моих потреби, и буяющее море злаго жития изсуши, и напоение гневное отжени, и целомудренный ум молитвами своими утверди, отче Паисие.


\itshape Припев:\normalfont{} Преподобне отче Паисие Великий, моли Бога о нас.


Наставника тя и молитвенника тепла имущи, и скораго помощника, яко стену тверду и забрало недвижимо и воеводу крепка и непобедима молим тя, не забуди молитвенник твоих, избавляя от веяния печали и навет вражиих.


\itshape Припев:\normalfont{} Преподобне отче Паисие Великий, моли Бога о нас.


О, чудо велие, во един от дний седящу ти, отче Паисие, в пещере, бысть ти глас глаголяй: мир тебе, возлюбленному Моему угоднику, ты же возстав со страхом, и трепетом одержимь, паде поклонися и рече: се, раб Твой, Господи. Сего ради молим тя, моли Человеколюбца, да спасет души наша.


Слава Отцу и Сыну и Святому Духу.


Даждь ми Христе разум и терпение, еже не осуждати согрешающих с кичением фарисейским, но яко мытарево покаяние приими, и яко блуднаго сына, Боже, вечери Твоей достойна мя яви, молитвами преподобнаго Паисия, и грехов прощение ми даруй.


И ныне и присно и во веки веков. Аминь.


\itshape Богородичен.\normalfont{}Возникни, о, страстная душе, возстани окаянная, бий из глубины в перси, и испусти слез источники, да тя помилует, окаянную, милосердая Мати Христа Бога.


\itshape Катавасия:\normalfont{} Избави от муки вечныя усопшаго раба своего ( \itshape назови имя\normalfont{}), преподобне отче Паисие Великий, яко вси по Бозе к тебе прибегаем, ты бо молиши о нас Христа Бога нашего ( \itshape поклон\normalfont{}). Господи, помилуй ( \itshape трижды с поклонами\normalfont{}).





\bfseries Песнь 8\normalfont{}


\itshape Ирмо́с:\normalfont{} Егоже вои Небеснии славят и трепещут Херувими и Серафими, всяко дыхание и тварь, пойте, благословите и превозносите Его во веки.


\itshape Припев:\normalfont{} Преподобне отче Паисие Великий, моли Бога о нас.


Благодарно приносимое тебе пение сие не презри, отче Паисие, но паче приими и радости духовныя исполни, да без закоснения поклонюся твоему образу, идеже есть написан, исцеления всем подавающ.


\itshape Припев:\normalfont{} Преподобне отче Паисие Великий, моли Бога о нас.


Все преподобному житию поревновав богоугодных отец, иже от века поживших, отче Паисие, во смирении и постничестве Христа ради всю жизнь свою мученически скончал еси, и от бесов нападения много претерпел еси, и сих победил. Сего ради молю тя, отче, буйство, ярость и небрежение молитвами твоими отрини от мене.


\itshape Припев:\normalfont{} Преподобне отче Паисие Великий, моли Бога о нас.


Владыко Христе Боже Всещедрый, имиже веси судьбами даждь ми ненавидети деяния лукаваго, Ты бо еси Бог наш рекий: просите и приимете. Даждь любовь от всея души моея, молитвами преподобнаго отца Паисия, творити волю Твою спасительную.


Слава Отцу и Сыну и Святому Духу.


Молютися, на мне грешнем долготерпи Владыко, и не посецы мене яко неплодное древо посечением смертным во огнь отсылая, но плодоносна мя сотвори молитвами преподобнаго умолен бывая, время покаяния ми даждь, яко Человеколюбец.


И ныне и присно и во веки веков. Аминь.


\itshape Богородичен.\normalfont{} Струи низпосли слез и стенания от души, сотвори ми, Чистая, внегда припадати к Твоему покрову, яко да обрящу разрешение грехов моих, Твоею молитвою.


\itshape Катавасия:\normalfont{} Избави от муки вечныя усопшаго раба своего ( \itshape назови имя\normalfont{}), преподобне отче Паисие Великий, яко вси по Бозе к тебе прибегаем, ты бо молиши о нас Христа Бога нашего ( \itshape поклон\normalfont{}). Господи, помилуй ( \itshape трижды с поклонами\normalfont{}).





\bfseries Песнь 9\normalfont{}


\itshape Ирмо́с:\normalfont{} Приимшая радость от Ангела, и рождшая Создателя Своего, Дево, спаси Тебе величающия.


\itshape Припев:\normalfont{} Преподобне отче Паисие Великий, моли Бога о нас.


Скорый помощник был еси преподобне, еще в жизни сей, иногда у преподобнаго старца ученику в преслушании умершу, и ведену во ад, он же припадаше прося, да умолиши Всещедраго о ученице его, ты же яко скорый послушник и любви хранитель, надежду возложи ко Всещедрому Богу простреся на молитву, и Той многомилостив и во обетех неложный, волю к боящымся Его сотвори, и молитву твою услыша, изведе душу от ада. Того ради и аз недостойный, припадая молю тя, отче Паисие, молитвами твоими избави мя муки и огня негасимаго.


\itshape Припев:\normalfont{} Преподобне отче Паисие Великий, моли Бога о нас.


Бывша мя бесом смех, человеком уничижение, праведным рыдание, Ангелом плач, осквернение воздуху и земли и водам. Тело окалях и ум оскверних, паче слова деяньми. Аз враг Богу бых. Увы мне, согреших, молитвами преподобнаго Паисия прости мя.


Слава Отцу и Сыну и Святому Духу.


Яко впадый в разбойники и уязвен, тако и аз впадох от многих грехов и уязвена ми есть душа. К кому прибегну повинный аз? Токмо к Тебе, милосердному душам Врачу, приими молебника тепла преподобнаго Паисия, и молитвами его излей на мя великую Твою милость.


И ныне и присно и во веки веков. Аминь.


\itshape Богородичен.\normalfont{} Прегрешении мои умножиша, прошению и суду достоин, Чистая, припадая зову Ти: прежде конца подаждь ми очищение и умиление, и нравом исправление.


\itshape Катавасия:\normalfont{} Избави от муки вечныя усопшаго раба своего ( \itshape назови имя\normalfont{}), преподобне отче Паисие Великий, яко вси по Бозе к тебе прибегаем, ты бо молиши о нас Христа Бога нашего ( \itshape поклон\normalfont{}). Господи, помилуй ( \itshape трижды с поклонами\normalfont{}).





\bfseries Молитва:\normalfont{}


Страстей победителя, душам помощника, о всех молебника, всем спасения ходатая и наставника, из глубины сердца воздыхая, усердно и пламенно молим тя, Паисие Преподобне! Внемли и помози нам, не отринь и не презри нас, но абие услыши в смирении сердца притекающих к тебе. Ты, преподобне, к спасению ближних прилежно стремился и многих грешников к свету спасения привел еси. Подвиги чрезмерные успокоением считал по себе, пречудне, и, любовью ко Господу всегда горя, явления Христа Спаса сподобился еси, и Ему, за людей Умершему, любовию подражая, и об отрекшихся от Христа молился еси. Услыши нас, Паисие прехвальне, ибо недостойны есмы молитися о даровании нам великой милости Господней, понеже грешны есмы, и уста оскверненныя и сердца отягощенныя имеем, и под бременем прегрешений страждем, и не достигает молитва наша до Господа. Сего ради помолися за нас мольбою твоею крепкою и богоприятною, святый Паисие, да избавлены будут скончавшиеся без покаяния сродники, ближние и знаемые наши от муки вечныя, и молитву твою во благоволении приимет Спас наш и милосердие Свое вместо добрых дел их даст им, свободит их, веруем, от страданий и вселит в селениях праведных, и нас в покаянии скончатися удостоит, да прославим вкупе Всесвятое и великолепое имя Отца и Сына и Святаго Духа, во веки веков. Аминь.





\bfseries Отпуст:\normalfont{}


Господи Иисусе Христе, Сыне Божий, молитв ради Пречистыя Твоея Матере, и преподобнаго отца нашего Паисия Великаго, и всех ради святых, помилуй и спаси нас, яко Благ и Человеколюбец. Аминь.


Господи, помилуй ( \itshape трижды\normalfont{}).


 