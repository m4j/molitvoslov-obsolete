

\mypart{ЧАСЫ ТРЕТИЙ И ШЕСТОЙ}
%http://www.molitvoslov.com/content/chasi-tretiy-shestoy

\bfseries Смотреть весь раздел &rarr;\normalfont{} 

\mychapter{Час Третий}
%http://www.molitvoslov.com/text909.htm 
 


\itshape Иерей:\normalfont{} Благословен Бог наш, всегда, ныне и присно, и во веки веков.


\itshape Чтец:\normalfont{} Аминь. Слава Тебе, Боже наш, слава Тебе.


Царю Небесный, Утешителю, Душе истины, Иже везде сый и вся исполняяй, Сокровище благих и жизни Подателю: прииди и вселися в ны, и очисти ны от всякия скверны, и спаси, Блаже, души наша.


Святый Боже, Святый Крепкий, Святый Безсмертный, помилуй нас.\itshape (Трижды)\normalfont{}


Слава Отцу, и Сыну, и Святому Духу, и ныне и присно, и во веки веков. Аминь.


Пресвятая Троице, помилуй нас; Господи, очисти грехи наша; Владыко, прости беззакония наша; Святый, посети и исцели немощи наша, имени Твоего ради.


Господи, помилуй. \itshape (Трижды)\normalfont{}


Слава Отцу, и Сыну, и Святому Духу, и ныне и присно, и во веки веков. Аминь.


Отче наш, Иже еси на небесех! Да святится имя Твое, да приидет Царствие Твое, да будет воля Твоя, яко на небеси и на земли. Хлеб наш насущный даждь нам днесь; и остави нам долги наша, якоже и мы оставляем должником нашим; и не введи нас во искушение, но избави нас от лукаваго.


\itshape Иерей:\normalfont{} Яко Твое есть Царство, и Сила, и Слава, Отца, и Сына, и Святаго Духа, ныне и присно, и во веки веков.


\itshape Чтец:\normalfont{} Аминь. Господи, помилуй \itshape (12)\normalfont{}. Слава Отцу, и Сыну, и Святому Духу, и ныне и присно, и во веки веков. Аминь.


Приидите, поклонимся Цареви нашему Богу.


Приидите, поклонимся и припадем Христу, Цареви нашему Богу.


Приидите, поклонимся и припадем Самому Христу, Цареви и Богу нашему.





\bfseries Псалом 16\normalfont{}


Услыши, Господи, правду мою, вонми молению моему, внуши молитву мою не во устнах льстивых. От Лица Твоего судьба моя изыдет, очи мои да видита правоты. Искусил еси сердце мое, посетил еси нощию, искусил мя еси, и не обретеся во мне неправда. Яко да не возглаголют уста моя дел человеческих, за словеса устен Твоих аз сохраних пути жестоки. Соверши стопы моя во стезях Твоих, да не подвижутся стопы моя. Аз воззвах, яко услышал мя еси, Боже, приклони ухо Твое мне и услыши глаголы моя. Удиви милости Твоя, спасаяй уповаюшия на Тя от противящихся деснице Твоей. Сохрани мя, Господи, яко зеницу ока; в крове крилу Твоею покрыеши мя, от лица нечестивых, острастших мя. Врази мои душу мою одержаша, тук свой затвориша; уста их глаголаша гордыню. Изгонящии мя ныне обыдоша мя, очи свои возложиша уклонити на землю. Объяша мя яко лев готов на лов, и яко скимен, обитаяй в тайных. Воскресни, Господи, предвари я, и запни им: избави душу мою от нечестиваго, оружие Твое от враг руки Твоея. Господи, от малых от земли, раздели я в животе их, и сокровенных Твоих исполнися чрево их. Насытишася сынов, и оставиши останки младенцем своим. Аз же правдою явлюся лицу Твоему, насыщуся, внегда явити ми ся славе Твоей.





\bfseries Псалом 24\normalfont{}


К Тебе, Господи, воздвигох душу мою. Боже мой, на Тя уповах, да не постыжуся во век, ниже да посмеют ми ся врази мои. Ибо вся терпящии Тя не постыдятся. Да постыдятся беззаконнующии вотще. Пути Твои, Господи, скажи ми, и стезям Твоим научи мя. Настави мя на истину Твою, и научи мя; яко Ты еси Бог, Спас мой, и Тебе терпех весь день. Помяни щедроты Твоя, Господи, и милости Твоя, яко отвека суть. Грех юности моея и неведения моего не помяни; по милости Твоей помяни мя Ты, ради благости Твоея, Господи. Благ и прав Господь, сего ради законоположит согрешающим на пути. Наставит кроткия на суд, научит кроткия путем Своим. Вся путие Господни милость и истина, взыскающим завета Его и свидения Его. Ради имене Твоего, Господи, и очисти грех мой, мног бо есть. Кто есть человек, бояйся Господа? Законоположит ему на пути, его же изволи. Душа его во благих водворится, и семя его наследит землю. Держава Господь боящихся Его, и завет Его явит им. Очи мои выну ко Господу, яко Той исторгнет от сети нозе мои. Призри на мя и помилуй мя, яко единород и нищ есмь аз. Скорби сердца моего умножишася, от нужд моих изведи мя. Виждь смирение мое и труд мой, и остави вся грехи моя. Виждь враги моя, яко умножишася, и ненавидением неправедным возненавидеша мя. Сохрани душу мою, и избави мя, да не постыжуся, яко уповах на Тя. Незлобивии и правии прилепляхуся мне, яко потерпех Тя, Господи. Избави, Боже, Израиля от всех скорбей его.





\bfseries Псалом 50\normalfont{}


Помилуй мя, Боже, по велицей милости Твоей, и по множеству щедрот Твоих очисти беззаконие мое. Наипаче омый мя от беззакония моего, и от греха моего очисти мя; яко беззаконие мое аз знаю, и грех мой предо мною есть выну. Тебе Единому согреших и лукавое пред Тобою сотворих, яко да оправдишися во словесех Твоих, и победиши внегда судити Ти. Се бо, в беззакониих зачат есмь, и во гресех роди мя мати моя. Се бо, истину возлюбил еси; безвестная и тайная премудрости Твоея явил ми еси. Окропиши мя иссопом, и очищуся; омыеши мя, и паче снега убелюся. Слуху моему даси радость и веселие; возрадуются кости смиренныя. Отврати лице Твое от грех моих и вся беззакония моя очисти. Сердце чисто созижди во мне, Боже, и дух прав обнови во утробе моей. Не отвержи мене от лица Твоего и Духа Твоего Святаго не отыми от мене. Воздаждь ми радость спасения Твоего и Духом владычним утверди мя. Научу беззаконыя путем Твоим, и нечестивии к Тебе обратятся. Избави мя от кровей, Боже, Боже спасения моего; возрадуется язык мой правде Твоей. Господи, устне мои отверзеши, и уста моя возвестят хвалу Твою. Яко аще бы восхотел еси жертвы, дал бых убо: всесожжения не благоволиши. Жертва Богу дух сокрушен; сердце сокрушенно и смиренно Бог не уничижит. Ублажи, Господи, благоволением Твоим Сиона, и да созиждутся стены Иерусалимския. Тогда благоволиши жертву правды, возношение и всесожегаемая; тогда возложат на oлтарь Твой тельцы.


Слава Отцу, и Сыну, и Святому Духу, и ныне и присно, и во веки веков. Аминь.


Аллилуиа, аллилуиа, аллилуиа, слава Тебе Боже. \itshape (Трижды)\normalfont{}


Господи, помилуй. \itshape (Трижды)\normalfont{}


Слава Отцу, и Сыну, и Святому Духу:





\bfseries Тропарь дня\normalfont{}


\itshape (О чтении тропарей на часах см. выше в последовании 9-го часа)\normalfont{}


И ныне и присно, и во веки веков. Аминь.


Богородице, Ты еси лоза истинная, возрастившая нам плод живота, Тебе молимся: молися, Владычице, со святыми апостолы, помиловати души наша.


\itshape [Если Великий пост "--- настоящий тропарь, глас 6:\normalfont{}


Господи, Иже Пресвятаго Твоего Духа в третий час апостолом Твоим ниспославый, Того, Благий, не отыми от нас, но обнови нас, молящих Ти ся.


Стих 1: Сердце чисто созижди во мне, Боже, и дух прав обнови во утробе моей.


Стих 2: Не отвержи мене от лица Твоего и Духа Твоего Святаго не отыми от мене.


Слава Отцу и Сыну и Святому Духу. И ныне и присно и во веки веков. Аминь.


\itshape Богородичен:\normalfont{} Богородице, Ты еси лоза истинная, возрастившая нам плод живота, Тебе молимся: молися, Владычице, со святыми апостолы, помиловати души наша.\itshape ]\normalfont{}


Господь Бог благословен, благословен Господь день дне, поспешит нам Бог спасений наших; Бог наш, Бог спасати.


Святый Боже, Святый Крепкий, Святый Безсмертный, помилуй нас.\itshape (Трижды)\normalfont{}


Слава Отцу, и Сыну, и Святому Духу, и ныне и присно, и во веки веков. Аминь.


Пресвятая Троице, помилуй нас; Господи, очисти грехи наша; Владыко, прости беззакония наша; Святый, посети и исцели немощи наша, имене Твоего ради.


Господи, помилуй. \itshape (Трижды)\normalfont{}


Слава Отцу, и Сыну, и Святому Духу, и ныне и присно, и во веки веков. Аминь.


Отче наш, Иже еси на небесех. Да святится имя Твое; да приидет Царствие Твое; да будет воля Твоя, яко на небеси и на земли. Хлеб наш насущный даждь нам днесь. И остави нам долги наша, якоже и мы оставляем должником нашим. И не введи нас во искушение, но избави нас от лукаваго.


\itshape Иерей:\normalfont{} Яко Твое есть Царство, и сила, и слава, Отца, и Сына, и Святаго Духа, ныне и присно, и во веки веков.


\itshape Чтец:\normalfont{} Аминь.





\bfseries Кондак праздника\normalfont{}


\itshape (О чтении кондаков на часах см. выше в последовании 9-го часа)\normalfont{}


\itshape [Если пост, или нет кондака, читаем эти тропари, глас 8-й:\normalfont{}


Благословен еси, Христе Боже наш, Иже премудры ловцы явлей, низпослав им Духа Святаго, и теми уловлей вселенную. Человеколюбче, слава Тебе.


Слава Отцу и Сыну и Святому Духу.


Скорое и известное даждь утешение рабом Твоим, Иисусе, внегда унывати духом нашим, не разлучайся от душ наших в скорбех, не удаляйся от мыслей наших во обстояниих, но присно нас предвари, приближися нам, приближися везде Сый, якоже со Апостолы Твоими всегда еси, сице и Тебе желающым соедини Себе, Щедре, да совокуплени Тебе поем и славословим Всесвятаго Духа Твоего.


И ныне и присно и во веки веков. Аминь.


Надежда, и предстательство и прибежище христиан, необоримая стена, изнемогающим пристанище небурное Ты еси, Богородице Пречистая; но яко мiр спасающая непрестаннною Твоею молитвою, помяни нас, Дево Всепетая.\itshape ]\normalfont{}


Господи, помилуй. \itshape (40 раз)\normalfont{}


Иже на всякое время и на всякий час, на небеси и на земли поклоняемый и славимый, Христе Боже, долготерпеливе, многомилостиве, многоблагоутробне, Иже праведныя любяй, и грешныя милуяй, Иже вся зовый ко спасению обещания ради будущих благ, Сам, Господи, приими и наша в час сей молитвы, и исправи живот наш к заповедем Твоим, души наша освяти, телеса очисти, помышления исправи, мысли очисти, и избави нас от всякия скорби, зол и болезней, огради нас святыми Твоими ангелы, да ополчением их соблюдаеми и наставляеми, достигнем в соединение веры и в разум неприступныя Твоея славы, яко благословен еси во веки веков. Аминь.


Господи, помилуй. \itshape (Трижды)\normalfont{}


Слава Отцу, и Сыну, и Святому Духу, и ныне и присно, и во веки веков. Аминь.


Честнейшую херувим и славнейшую без сравнения Серафим, без истления Бога Слова рождшую, сущую Богородицу, Тя величаем.


Именем Господним, благослови отче.


\itshape Иерей:\normalfont{} Молитвами святых отец наших, Господи Иисусе Христе, Сыне Божий, помилуй нас.


\itshape Чтец:\normalfont{} Аминь.





\bfseries Молитва святого Мардария\normalfont{}


Владыко Боже, Отче Вседержителю, Господи Сыне Единородный, Иисусе Христе, и Святый Душе, Едино Божество, едина сила, помилуй мя грешнаго, и имиже веси судьбами спаси мя недостойного раба Твоего; яко благословен еси во веки веков. Аминь.




\mychapterending

\mychapter{Час Шестой}
%http://www.molitvoslov.com/text910.htm 
 


\itshape Чтец:\normalfont{} Приидите, поклонимся Цареви нашему Богу.


Приидите, поклонимся и припадем Христу, Цареви нашему Богу.


Приидите, поклонимся и припадем Самому Христу, Цареви и Богу нашему.





\bfseries Псалом 53\normalfont{}


Боже, во имя Твое спаси мя, и в силе Твоей суди ми. Боже, услыши молитву мою, внуши глаголы уст моих. Яко чуждии восташа на мя, и крепцыи взыскаша душу мою, и не предложиша Бога пред собою. Се бо Бог помогает ми, и Господь Заступник души моей. Отвратит злая врагом моим, истиною Твоею потреби их. Волею пожру Тебе, исповемся имени Твоему, Господи, яко благо, яко от всякия печали избавил мя еси, и на враги моя воззре око мое.





\bfseries Псалом 54\normalfont{}


Внуши, Боже, молитву мою и не презри моления моего. Вонми ми и услыши мя; возскорбех печалию моею, и смятохся от гласа вражия и от стужения грешнича, яко уклониша на мя беззаконие, и во гневе враждоваху ми. Сердце мое смятеся во мне, и боязнь смерти нападе на мя. Страх и трепет прииде на мя, и покры мя тьма. И рех: кто даст ми криле, яко голубине, и полещу и почию? Се, удалихся, бегая, и подворихся в пустыни. Чаях Бога, спасающаго мя от малодушия и от бури. Потопи, Господи, и раздели языки их, яко видех беззаконие и пререкание во граде. Днем и нощию обыдет и по стенам его; беззаконие и труд посреде его, и неправда; и не оскуде от стогн его лихва и лесть. Яко аще бы враг поносил ми, претерпел бых убо; и аще бы ненавидяй мя на мя велеречевал, укрылбыхся от него. Ты же, человече равнодушне, владыко мой и знаемый мой; иже купно наслаждался еси со мною брашен, в дому Божии ходихом единомышлением. Да приидет же смерть на ня, и да снидут во ад живи; яко лукавство в жилищах их, посреде их. Аз к Богу воззвах, и Господь услыша мя. Вечер и заутра, и полудне повем и возвещу, и услышит глас мой. Избавит миром душу мою от приближающихся мне; яко бо мнозе бяху со мною. Услышит Бог и смирит я, Сый прежде век; несть бо им изменения, яко не убояшася Бога. Простре руку Свою на воздаяние, оскверниша завет Его. Разделишася от гнева лица Его, и приближишася сердца их; умякнуша словеса их паче елеа, и та суть стрелы. Возверзи на Господа печаль твою, и Той тя препитает, не даст в век молвы праведнику. Ты же, Боже, низведеши я в студенец истления. Мужие кровей и льсти не преполовят дней своих. Аз же, Господи, уповаю на Тя.





\bfseries Псалом 90\normalfont{}


Живый в помощи Вышняго, в крове Бога Небеснаго водворится. Речет Господеви: Заступник мой еси и Прибежище мое, Бог мой и уповаю на Него. Яко Той избавит тя от сети ловчи и от словесе мятежна, плещма Своима осенит тя и под криле Его надеешися: оружием обыдет тя истина Его. Не убоишися от страха нощнаго, от стрелы летящия во дни, oт веши во тьме преходящия, от сряща и беса полуденнаго. Падет от страны твоея тысяща, и тьма одесную тебе, к тебе же не приближится: oбаче очима Твоима смотриши, и воздаяние грешников узриши. Яко Ты, Господи, упование мое, Вышняго положил еси прибежище твое. Не приидет к тебе зло, и рана не приближится телеси твоему. Яко ангелом Своим заповесть о тебе, сохранити тя во всех путех твоих. На руках возмут тя, да не когда преткнеши о камень ногу твою. На аспида и василиска наступиши, и попереши льва и змия. Яко на Мя упова, и избавлю и; покрыю и, яко позна имя Мое. Воззовет ко Мне, и услышу его; с ним есмь в скорби, изму его и прославлю его; долготою дней исполню его, и явлю ему спасение Мое.


Слава Отцу, и Сыну, и Святому Духу, и ныне и присно, и во веки веков. Аминь.


Аллилуиа, аллилуиа, аллилуиа, слава Тебе Боже. \itshape (Трижды)\normalfont{}


Господи, помилуй. \itshape (Трижды)\normalfont{}


Слава Отцу, и Сыну, и Святому Духу:





\bfseries Тропарь дня\normalfont{}


\itshape (О чтении тропарей на часах см. выше в последовании 9-го часа)\normalfont{}


И ныне и присно, и во веки веков. Аминь.


\itshape Богородичен:\normalfont{} Яко не имамы дерзновения за премногия грехи наша, Ты Иже от Тебе рождшагося моли, Богородице Дево: много бо может моление Матернее ко благосердию Владыки; не презри грешных мольбы, Всечистая, яко милостив есть, и спасати Mогий, Иже и страдати о нас изволивый.


\itshape [Если Великий пост "--- настоящий тропарь, глас 2:\normalfont{}


Иже в шестый день же и час на кресте пригвождей в раи дерзновенный Адамов грех, и согрешений наших рукописание раздери, Христе Боже, и спаси нас.


Стих 1: Внуши, Боже, молитву мою, и не презри моления моего.


Стих 2: Аз к Богу воззвах, и Господь услыша мя.


Слава Отцу, и Сыну, и Святому Духу, и ныне и присно, и во веки веков. Аминь.


\itshape Богородичен:\normalfont{} Яко не имамы дерзновения за премногия грехи наша, Ты Иже от Тебе рождшагося моли, Богородице Дево: много бо может моление Матернее ко благосердию Владыки; не презри грешных мольбы, Всечистая, яко милостив есть, и спасати Mогий, Иже и страдати о нас изволивый.]


Скоро да предварят ны щедроты Твоя Господи, яко обнищахом зело; помози нам, Боже Спасе наш, славы ради имене Твоего, Господи, избави нас, и очисти грехи наша, имене ради Твоего.


Святый Боже, Святый Крепкий, Святый Безсмертный, помилуй нас.\itshape (Трижды)\normalfont{}


Слава Отцу, и Сыну, и Святому Духу, и ныне и присно, и по веки веков. Аминь.


Пресвятая Троице, помилуй нас; Господи, очисти грехи наша; Владыко, прости беззакония наша; Святый, посети и исцели немощи наша, имене Твоего ради.


Господи, помилуй. \itshape (Трижды)\normalfont{}


Слава Отцу, и Сыну, и Святому Духу, и ныне и присно, и во веки веков. Аминь.


Отче наш, Иже еси на небесех. Да святится имя Твое; да приидет Царствие Твое; да будет воля Твоя, яко на небеси и на земли. Хлеб наш насущный даждь нам днесь. И остави нам долги наша, якоже и мы оставляем должником нашим. И не введи нас во искушение, но избави нас от лукаваго.


\itshape Иерей:\normalfont{} Яко Твое есть Царство, и сила, и слава, Отца, и Сына, и Святаго Духа, ныне и присно, и во веки веков.


\itshape Чтец:\normalfont{} Аминь.





\bfseries Кондак праздника\normalfont{}


\itshape (О чтении кондаков на часах см. выше в последовании 9-го часа)\normalfont{}


\itshape [Если Великий пост "--- тропари, глас 2:\normalfont{}


Спасение соделал еси посреде земли, Христе Боже, на Кресте пречистеи руце Твои распростерл еси, собирая вся языки, зовущия: Господи, слава тебе.


Слава Отцу, и Сыну, и Святому Духу:


Пречистому Твоему образу покланяемся, Благий, просяще прощения прегрешений наших, Христе Боже, волею бо благоволил еси плотию взыти на Крест, да избавиши, яже создал еси, от работы вражия; тем благодарственно вопием Ти: радости исполнил еси вся, Спасе наш, пришедый спасти мир.


И ныне и присно, и во веки веков. Аминь.


\itshape В понедельник, вторник и четверг "--- Богородичен:\normalfont{}


Милосердия сущия источник, милости сподоби нас, Богородице, призри на люди согрешившия, яви яко присно силу Твою; на Тя бо уповающе, радуйся вопием Ти, якоже иногда Гавриил, безплотных архистратиг.


\itshape В среду и пятницу "--- Крестобогородичен:\normalfont{}


Препрославлена еси, Богородице Дево, поем Тя; Крестом бо Сына Твоего низложися ад, и смерть умертвися, умерщвленнии востахом, и живота сподобихомся, рай восприяхом древнее наслаждение. Тем благодаряще, славословим яко державного Христа Бога нашего, и единого Милостиваго.]


Господи, помилуй. \itshape (40 раз)\normalfont{}


Иже на всякое время и на всякий час, на небеси и на земли поклоняемый и славимый, Христе Боже, долготерпеливе, многомилостиве, многоблагоутробне, Иже праведныя любяй, и грешныя милуяй, Иже вся зовый ко спасению обещания ради будущих благ, Сам, Господи, приими и наша в час сей молитвы, и исправи живот наш к заповедем Твоим, души наша освяти, телеса очисти, помышления исправи, мысли очисти, и избави нас от всякия скорби, зол и болезней, огради нас святыми Твоими ангелы, да ополчением их соблюдаеми и наставляеми, достигнем в соединение веры и в разум неприступныя Твоея славы, яко благословен еси во веки веков. Аминь.


Господи, помилуй. \itshape (Трижды)\normalfont{}


Слава Отцу, и Сыну, и Святому Духу, и ныне и присно, и во веки веков. Аминь.


Честнейшую херувим и славнейшую без сравнения Серафим, без истления Бога Слова рождшую, сущую Богородицу, Тя величаем.


Именем Господним, благослови отче.


\itshape Иерей:\normalfont{} Молитвами святых отец наших, Господи Иисусе Христе, Сыне Божий, помилуй нас.


\itshape Чтец:\normalfont{} Аминь.





\bfseries Молитва св. Василия Великого\normalfont{}


Боже и Господи сил и всея твари Содетелю, Иже за милосердие безприкладныя милости Твоея Единороднаго Сына Твоего, Господа нашего Иисуса Христа, низпославый на спасение рода нашего, и честным Его Крестом рукописание грех наших растерзавый, и победивый тем начала и власти тьмы; Сам, Владыко Человеколюбче, приими и нас грешных благодарственныя сия и молебныя молитвы, и избави нас от всякаго всегубительнаго и мрачнаго прегрешения, и всех озлобити нас ищущих, видимых и невидимых враг. Пригвозди страху Твоему плоти наша, и не уклони сердец наших в словеса или помышления лукавствия; но любовию Твоею уязви души наша, да к Тебе всегда взирающе, и еже от Тебе светом наставляеми, Тебе непреступнаго и присносущнаго зряще Cвета, непрестанное Тебе исповедание и благодарение возсылаем, Безначальному Отцу со Единородным Твоим Сыном, и Всесвятым и Благим, и Животворящим Твоим Духом, ныне и присно и во веки веков. Аминь.




\mychapterending