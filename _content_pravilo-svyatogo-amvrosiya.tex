

\mypart{MОЛИТВЕННОЕ ПРАВИЛО ПРЕПОДОБНОГО AМВРОСИЯ OПТИНСКОГО}\label{_content_pravilo-svyatogo-amvrosiya}
%http://www.molitvoslov.com/content/pravilo-svyatogo-amvrosiya

 

\mychapter{1. Правило, читаемое во время скорби}
%http://www.molitvoslov.com/text648.htm 
 


\mysubsubsection{Псалом 3}


Господи, что ся умножиша стужающии ми? Мнози востают на мя, мнози глаголют души моей: несть спасения ему в Бозе его. Ты же. Господи, Заступник мой ecи, слава моя и возносяй главу мою. Гласом моим ко Господу воззвах, и услыша мя от горы святыя Своея. Аз уснух, и спах, востах, яко Господь заступит мя. Не убоюся от тем людей, окрест нападающих на мя. Воскресни, Господи, спаси мя, Боже мой, яко Ты поразил ecи вся враждующыя ми всуе: зубы грешников сокрушил ecu. Господне есть спасение, и на людех Твоих благословение Твое.


\mysubsubsection{Псалом 53}


Боже, во Имя Твое спаси мя, и в силе Твоей суди ми. Боже, услыши молитву мою, внуши глаголы уст моих. Яко чуждии восташа на мя и крепцыи взыскаша душу мою, и не предложиша Бога пред собою. Се бо Бог помогает ми, и Господь заступник души моей. Отвратит злая врагом моим, истиною Твоею потреби их. Волею пожру Тебе, исповемся имени Твоему, Господи, яко благо, яко от всякия печали избавил мя еси, и на враги моя воззре око мое.


\mysubsubsection{Псалом 58}


Изми мя от враг моих, Боже, и от востающих на мя избави мя. Избави мя от делающих беззаконие и от муж кровей спаси мя. Яко се уловиша душу мою, нападоша на мя крепцыи, ниже беззаконие мое, ниже грех мой, Господи. Без беззакония текох и исправих, востани в сретение мое и виждь. И ты, Господи Боже сил, Боже Израилев, вонми посетити вся языки, да не ущедриши вся делающыя беззаконие. Возвратятся на вечер и взалчут, яко пес, и обыдут град. Се тии отвещают усты своими, и меч во устнах их, яко кто слыша? И Ты, Господи, посмеешися им, уничижиши вся языки. Державу мою к Тебе сохраню, яко Ты, Боже, Заступник мой еси. Бог мой, милость Его предварит мя, Бог мой, явит мне на вразех моих. Не убий их, да не когда забудут закон Твой, расточи я силою Твоею и низведи я, Защитниче мой, Господи. Грех уст их, слово устен их, и яти да будут в гордыни своей, и от клятвы и лжи возвестятся в кончине. Во гневе кончины, и не будут и уведят, яко Бог владычествует Иаковом и концы земли. Возвратятся на вечер, и взалчут, яко пес, и обыдут град. Тии разыдутся ясти. Аще ли же не насытятся, и поропщут. Аз же воспою силу Твою и возрадуюся заутра о милости Твоей, яко был еси Заступник мой и Прибежище мое в день скорби моея. Помощник мой еси, Тебе пою: яко Бог Заступник мой еси, Боже мой, милость моя.


\mysubsubsection{Псалом 142}


Господи, услыши молитву мою, внуши моление мое во истине Твоей, услыши мя в правде Твоей и не вниди в суд с рабом Твоим, яко не оправдится пред Тобою всяк живый. Яко погна враг душу мою, смирил есть в землю живот мой, посадил мя есть в темных, яко мертвыя века. И уны во мне дух мой, во мне смятеся сердце мое. Помянух дни древния, поучихся во всех делех Твоих, в творениих руку Твоею поучахся. Воздех к Тебе руце мои, душа моя, яко земля безводная Тебе. Скоро услыши мя, Господи, исчезе дух мой, не отврати лица Твоего от мене, и уподоблюся низходящым в ров. Слышану сотвори мне заутра милость Твою, яко на Тя уповах. Скажи мне, Господи, путь воньже пойду, яко к Тебе взях душу мою. Изми мя от враг моих, Господи, к Тебе прибегох. Научи мя творити волю Твою, яко Ты еси Бог мой. Дух Твой Благий наставит мя на землю праву. Имене Твоего ради, Господи, живиши мя, правдою Твоею изведеши от печали душу мою. И милостию Твоею потребиши враги моя и погубиши вся стужающыя души моей, яко аз раб Твой есмь.


\mysubsubsection{Псалом 101}


Господи, услыши молитву мою, и вопль мой к Тебе да приидет. Не отврати лица Твоего от мене: воньже аще день скорблю, приклони ко мне ухо Твое: воньже аще день призову Тя, скоро услыши мя. Яко исчезоша яко дым дние мои, и кости моя яко сушило сосхошася. Уязвен бых яко трава, и изсше сердце мое, яко забых снести хлеб мой. От гласа воздыхания моего прильпе кость моя плоти моей. Уподобихся неясыти пустынней, бых яко нощный вран на нырищи. Бдех и бых яко птица особящаяся на зде. Весь день поношаху ми врази мои, и хвалящии мя мною кленяхуся. Зане пепел яко хлеб ядях, и питие мое с плачем растворях. От лица гнева Твоего и ярости Твоея: яко вознес низвергл мя еси. Дние мои яко сень уклонишася, и аз яко сено изсхох. Ты же, Господи, во век пребываеши, и память Твоя в род и род. Ты воскрес ущедриши Сиона, яко время ущедрити его, яко прииде время. Яко благоволиша раби Твои камение его, и персть его ущедрят. И убоятся языцы имене Господня, и вси царие земстии славы Твоея. Яко созиждет Господь Сиона, и явится во славе Своей. Призре на молитву смиренных, и не уничижи моления их. Да напишется сие в род ин, и людие зиждемии восхвалят Господа. Яко приниче с высоты святыя Своея, Господь с Небесе на землю призре, услышати воздыхание окованных, разрешити сыны умерщвленных, возвестити в Сионе Имя Господне, и хвалу Его во Иерусалиме: внегда собратися людем вкупе, и царем, еже работати Господеви. Отвеща ему на пути крепости его: умаление дней моих возвести ми. Не возведи мене в преполовение дней моих: в роде родов лета Твоя. В началех Ты, Господи, землю основал еси, и дела руку Твоею суть Небеса. Та погибнут, Ты же пребываеши: и вся, яко риза обетшают, и яко одежду свиеши я и изменятся. Ты же Тойжде еси, и лета твоя не оскудеют. Сынове раб Твоих вселятся, и семя их во век исправится.
\longpage[2]{}\mychapterending

\mychapter{2. Правило, читаемое во время искушений}
%http://www.molitvoslov.com/text649.htm 
 


\mysubsubsection{Псалом 36}


Не ревнуй лукавнующым, ниже завиди творящым беззаконие. Зане яко трава скоро изсшут, яко зелие злака скоро отпадут. Уповай на Господа и твори благостыню, и насели землю, и упасешися в богатстве ея. Насладися Господеви, и даст ти прошения сердца твоего. Открый ко Господу путь твой и уповай на Него, и Той сотворит: и изведет, яко свет, правду твою и судьбу твою, яко полудне. Повинися Господеви и умоли Его. Не ревнуй спеющему в пути своем, человеку, творящему законопреступление. Престани от гнева и остави ярость, не ревнуй еже лукавновати. Зане лукавнующии потребятся, терпящии же Господа, тии наследят землю. И еще мало, и не будет грешника, и взыщеши место его, и не обрящеши. Кротцыи же наследят землю и насладятся о множестве мира. Назирает грешный праведнаго и поскрежещет нань зубы своими. Господь же посмеется ему, зане прозирает, яко приидет день его. Мечь извлекоша грешницы, напрягоша лук свой, низложити убога и нища, заклати правыя сердцем. Мечь их да внидет в сердца их, и луцы их да сокрушатся. Лучше малое праведнику, паче богатства грешных многа. Зане мышцы грешных сокрушатся, утверждает же праведныя Господь. Весть Господь пути непорочных, и достояние их в век будет. Не постыдятся во время лютое, и во днех глада насытятся, яко грешницы погибнут. Врази же Господни, купно прославитися им и вознестися, исчезающе яко дым исчезоша. Заемлет грешный и не возвратит, праведный же щедрит и дает. Яко благословящии Его наследят землю, кленущии же Его потребятся. От Господа стопы человеку исправляются, и пути его восхощет зело. Егда падет, не разбиется, яко Господь подкрепляет руку его. Юнейший бых, ибо состарехся, и не видех праведника оставлена, ниже семени его просяща хлебы. Весь день милует и взаим дает праведный, и семя его во благословение будет. Уклонися от зла и сотвори благо, и вселися в век века. Яко Господь любит суд и не оставит преподобных Своих, во век сохранятся. Беззаконницы же изженутся, и семя нечестивых потребится. Праведницы же наследят землю и вселятся в век века на ней. Уста праведнаго поучатся премудрости, и язык его возглаголет суд. Закон Бога его в сердце его, и не запнутся стопы его. Сматряет грешный праведнаго и ищет еже умертвити его. Господь же не оставит его в руку его, ниже осудит его, егда судит ему. Потерпи Господа и сохрани путь Его, и вознесет тя, еже наследити землю, внегда потреблятися грешником узриши. Видех нечестиваго превозносящася и высящася, яко кедры Ливанския. И мимо идох, и се не бе, и взысках его, и не обретеся место его. Храни незлобие и виждь правоту, яко есть останок человеку мирну. Беззаконницы же потребятся вкупе: останцы же нечестивых потребятся. Спасение же праведных от Господа, и Защититель их есть во время скорби. И поможет им Господь, и избавит их, и измет их от грешник, и спасет их, яко уповаша на Него.


\mysubsubsection{Псалом 26}


Господь просвещение мое и Спаситель мой, кого убоюся? Господь Защититель живота моего, от кого устрашуся? Внегда приближатися на мя злобующым, еже снести плоти моя, оскорбляюшии мя, и врази мои, тии изнемогоша и падоша. Аще ополчится на мя полк, не убоится сердце мое, аще востанет на мя брань, на Него аз уповаю. Едино просих от Господа, то взыщу: еже жити ми в дому Господни вся дни живота моего, зрети ми красоту Господню и посещати храм святый Его. Яко скры мя в селении Своем в день зол моих, покры мя в тайне селения Своего, на камень вознесе мя. И ныне се вознесе главу мою, на враги моя: обыдох и пожрох в селении Его жертву хваления и воскликновения, пою и воспою Господеви. Услыши, Господи, глас мой, имже воззвах: помилуй мя и услыши мя. Тебе рече сердце мое, Господа взыщу. Взыска Тебе лице мое, лица Твоего, Господи, взыщу. Не отврати лица Твоего от мене и не уклонися гневом от раба Твоего: помощник мой буди, не отрини мене, и не остави мене, Боже Спасителю мой. Яко отец мой и мати моя остависта мя, Господь же восприят мя. Законоположи ми, Господи, в пути Твоем и настави мя на стезю правую враг моих ради. Не предаждь мене в душы стужающих ми, яко восташа на мя свидетеле неправеднии и солга неправда себе. Верую видети благая Господня на земли живых. Потерпи Господа, мужайся и да крепится сердце твое, и потерпи Господа.


\mysubsubsection{Псалом 90}


Живый в помощи Вышняго, в крове Бога Небеснаго водворится. Речет Господеви: Заступник мой еси и Прибежище мое, Бог мой, и уповаю на Него. Яко Той избавит тя от сети ловчи, и от словесе мятежна, плещма Своима осенит тя, и под криле Его надеешися: оружием обыдет тя истина Его. Не убоишися от страха нощнаго, от стрелы летящия во дни, от вещи во тме преходяшия, от сряща, и беса полуденнаго. Падет от страны твоея тысяща, и тма одесную тебе, к тебе же не приближится, обаче очима твоима смотриши, и воздаяние грешников узриши. Яко Ты, Господи, упование мое, Вышняго положил еси прибежище твое. Не приидет к тебе зло, и рана не приближится телеси твоему, яко aнгелом Своим заповесть о тебе, сохранити тя во всех путех твоих. На руках возмут тя, да не когда преткнеши о камень ногу твою, на аспида и василиска наступиши, и попереши льва и змия. Яко на Мя упова, и избавлю и: покрыю и, яко позна имя Мое. Воззовет ко Мне, и услышу его: с ним есмь в скорби, изму его, и прославлю его, долготою дней исполню его, и явлю ему спасение Мое.


\mysubsubsection{Псалом 39}


Терпя, потерпех Господа, и внят ми, и услыша молитву мою. И возведе мя от рова страстей, и от брения тины, и постави на камени нозе мои, и исправи стопы моя, и вложи во уста моя песнь нову, пение Богу нашему. Узрят мнози и убоятся, и уповают на Господа. Блажен муж, емуже есть имя Господне упование его, и не призре в суеты и неистовления ложная. Многа сотворил еси Ты, Господи, Боже мой, чудеса Твоя и помышлением Твоим несть кто уподобится Тебе. Bозвестих и глаголах, умножишася паче числа. Жертвы и приношения не восхотел еси, тело же свершил ми еси, всесожжений и о гресе не взыскал еси. Тогда рех: се прииду, в главизне книжне писано есть о мне: еже сотворити волю Твою, Боже мой, восхотех, и закон Твой посреде чрева моего. Благовестих правду в церкви велицей, се устнам моим не возбраню: Господи, Ты разумел еси. Правду Твою не скрых в сердце моем, истину Твою и спасение Твое рех, не скрых милость Твою и истину Твою от сонма многа. Ты же, Господи, не удали щедрот Твоих от мене: милость Твоя и истина Твоя выну да заступите мя. Яко одержаша мя злая, имже несть числа, постигоша мя беззакония моя, и не возмогох зрети, умножишася паче влас главы моея, и сердце мое остави мя. Благоволи, Господи, избавити мя: Господи, во еже помощи ми вонми. Да постыдятся и посрамятся вкупе ищущии душу мою изъяти ю, да возвратятся вспять и постыдятся хотящии ми злая. Да приимут абие студ свой глаголющии ми: благоже, благоже. Да возрадуются и возвеселятся о Тебе вси ищущии Тебе, Господи, и да рекут выну: да возвеличится Господь, любящии спасение Твое. Аз же нищ есмь и убог, Господь попечется о мне. Помощник мой и Защититель мой еси Ты, Боже мой, не закосни.
\mychapterending
