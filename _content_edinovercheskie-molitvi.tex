

\mypart{ЕДИНОВЕРЧЕСКИЕ МОЛИТВЫ}\label{_content_edinovercheskie-molitvi}
%http://www.molitvoslov.com/content/edinovercheskie-molitvi

 

\mychapter{Молитвенное правило}
%http://www.molitvoslov.com/content/molitvennoepravilo

 

\section{Молитвы утренние и вечерние}
%http://www.molitvoslov.com/text1000.htm 
 


\itshape Утром, вставши от сна, еще на постели, перекрестись с молитвою:\normalfont{}


Господи, Исусе Христе, Сыне Божии, помилуй мя грешнаго.


\medskip\itshape Вставши с постели и умыв лице свое, а вечером, отходя ко сну, встань с благоговением пред святыми иконами и, взирая на них, мысль свою устремляя к невидимому Богу и его святым, истово, не спеша, ограждая себя крестным знамением и кланяясь, произноси с умилением молитву мытаря:\normalfont{}


Боже, милостив буди мне грешному \itshape (поклон)\normalfont{}. Создавыи мя, Господи, и помилуй мя \itshape (поклон)\normalfont{}. Без числа согреших, Господи, помилуй и прости мя, грешнаго \itshape (поклон)\normalfont{}.


Достойно есть, яко воистину блажити Тя Богородице, присно блаженную и пренепорочную, и Матерь Бога нашего. Честнейшую херувим и славнейшую воистину серафим, без истления Бога Слова рождьшую, сущую Богородицу, Тя величаем \itshape (поклон всегда земной)\normalfont{}.


Слава Отцу и Сыну и Святому Духу \itshape (поклон)\normalfont{}. И ныне и присно и во веки веком, аминь \itshape (поклон)\normalfont{}.


Господи помилуй, Господи помилуй, Господи благослови \itshape (поклон)\normalfont{}. 


Господи, Исусе Христе, Сыне Божии, молитв ради Пречистыя Твоея Матере, силою Честнаго и Животворящаго Креста, и святаго Ангела моего хранителя, и всех ради святых, помилуй и спаси мя грешнаго, яко Благ и Человеколюбец. Аминь. \itshape (поклон земной, без крестнаго знамения\normalfont{}.


\medskip\itshape Эти молитвы называются «началом» или «приходными и исходными поклонами» потому, что они совершаются в начале и после всякого молитвенного правила.\normalfont{}


После этого повтори молитву мытаря с поклонами:


Боже, милостив буди мне грешному \itshape (поклон)\normalfont{}. Создавыи мя, Господи, и помилуй мя \itshape (поклон)\normalfont{}. Без числа согреших, Господи, помилуй и прости мя грешнаго \itshape (поклон)\normalfont{}. 


\medskip\itshape И с благоговением начинай утренние молитвы.\normalfont{}


\medskipЗа молитв святых отец наших, Господи Исусе Христе, Сыне Божии, помилуй нас. Аминь \itshape (поклон всегда поясной). Перекрестись и говори трижды:\normalfont{}


Слава Тебе, Боже наш, слава Тебе всяческих ради. 


\medskip\itshape Далее:\normalfont{} Боже, очисти мя грешнаго, яко николиже благо сотворих пред Тобою \itshape (поклон)\normalfont{}, но избави мя от лукаваго, и да будет во мне воля Твоя \itshape (поклон)\normalfont{}, да неосужденно отверзу уста моя недостойная и восхвалю имя Твое святое: Отца и Сына и Святаго Духа, ныне и присно и во веки веком. Аминь \itshape (поклон).

 Обведенные чертой молитвы вечером не читаются.\normalfont{} 


\medskip\itshape Далее перекрестись и читай: \normalfont{}


Царю небесныи, Утешителю, Душе истинныи, Иже везде сыи и вся исполняя, Сокровище благих и жизни Подателю, прииди и вселися в ны, и очисти ны от всякия скверны, и спаси, Блаже, душа наша.


Святыи Боже, Святыи Крепкии, Святыи Безсмертныи, помилуй нас \itshape (трижды с поклонами)\normalfont{}. Слава Отцу и Сыну и Святому Духу, и ныне и присно и во веки веком. Аминь. Пресвятая Троице, помилуй нас. Господи, очисти грехи наша. Владыко, прости беззакония наша. Святыи, посети и исцели немощи наша, Имене Твоего ради. Господи помилуй (трижды). Слава Отцу и Сыну и Святому Духу, и ныне и присно и во веки веком. Аминь. Отче наш, Иже еси на небесех, да святится имя Твое, да приидет царствие Твое, да будет воля Твоя, яко на небеси и на земли. Хлеб наш насущныи даждь нам днесь. И остави нам долги наша, яко же и мы оставляем должником нашим. И не введи нас во искушение, но избави нас от лукаваго. Господи Исусе Христе Сыне Божии, помилуй нас. Аминь. Господи помилуй \itshape (12 раз)\normalfont{}. 


\medskip\itshape Если утро "--- читается: \normalfont{}


От сна востав, благодарю Тя, Всесвятая Троице, яко многия ради благости и долготерпения не прогневася на мя грешнаго и лениваго раба Твоего, и не погубил еси мене со беззаконии моими, но человеколюбствова. И в нечаянии лежаща, воздвиже мя утреневати и славословити державу Твою непобедимую. И ныне, Владыко, Боже Пресвятыи, просвети очи сердца моего и отверзи ми устне поучатися словесем Твоим, и разумети заповеди Твоя, и творити волю Твою, и пети Тя во исповедании сердечнем. Воспевати же и славити пречестное и великолепое имя Твое: Отца и Сына и Святаго Духа, ныне и присно и во веки веком. Аминь.


\medskip\itshape Если вечер:\normalfont{}


Слава Отцу и Сыну и Святому Духу. И ныне, и присно, и во веки веком. Аминь.


\medskip\itshape Далее, как утром, так и вечером, читай:\normalfont{}


Приидите, поклонимся Цареви нашему Богу \itshape (поклон)\normalfont{}. Приидите, поклонимся Христу, Цареви и Богу нашему \itshape (поклон)\normalfont{}. Приидите, поклонимся и припадем к самому Господу Исусу Христу, Цареви и Богу нашему \itshape (поклон)\normalfont{}. 


\mysubsubsection{ПСАЛОМ 50 (покаянный)}


Помилуй мя, Боже, по велицей милости Твоей. И по множеству щедрот Твоих очисти беззаконие мое. Наипаче омыи мя от беззакония моего и от греха моего очисти мя. Яко беззаконие мое аз знаю и грех мой предо мною есть выну. Тебе единому согреших и лукавое пред Тобою сотворих. Яко да оправдишися в словесех Своих и победиши внегда судити. Се бо в беззакониих зачат есмь и во гресех роди мя мати моя. Се бо истину возлюбил еси; безвестная и тайная премудрости Твоея явил ми еси. Окропиши мя иссопом и очищуся. Омыеши мя и паче снега убелюся. Слуху моему даси радость и веселие: возрадуются кости смиренныя. Отврати лице Твое от грех моих и вся беззакония моя очисти. Сердце чисто созижди во мне, Боже, и дух прав обнови во утробе моей. Не отверзи мене от Лица Твоего, и Духа Твоего Святаго не отыми от мене. Воздаждь ми радость спасения Твоего и духом владычным утверди мя. Научу беззаконныя путем Твоим и нечестивии к Тебе обратятся. Избави мя от кровий, Боже, Боже спасения моего; возрадуется язык мой правде Твоей. Господи, устне мои отверзеши, и уста моя возвестят хвалу Твою. Яко аще бы восхотел жертвы, дал бых убо; всесожжения не благоволиши. Жертва Богу "--- дух сокрушен: сердце сокрушенно и смиренно Бог не уничижит. Ублажи, Господи, благоволением Твоим Сиона; и да созиждутся стены Иеросалимския. Тогда благоволиши жертву правды, возношение и всесожигаемая. Тогда возложат на олтарь Твой тельца. 

\medskip\itshape Ограждая себя благоговейно крестным знамением, произносим СИМВОЛ ВЕРЫ "--- слова святых отец Первого и Второго Вселенского собора (крестное знамение без поклона): \normalfont{}


\medskipВерую во единаго Бога Отца Вседержителя, Творца небу и земли, видимым же всем и невидимым. И во единаго Господа, Исуса Христа, Сына Божия, Единороднаго, Иже от Отца рожденнаго прежде всех век. Света от Света, Бога истинна от Бога истинна, рождена, а не сотворена, единосущна Отцу, Им же вся быша. Нас ради человек, и нашего ради спасения сшедшаго с небес, и воплотившагося от Духа Свята и Марии Девы вочеловечшася. Распятаго за ны при Понтийстем Пилате, страдавша и погребенна. И воскресшаго в третии день по писаниих. И возшедшаго на небеса, и седяща одесную Отца. И паки грядущаго со славою судити живым и мертвым, Его же царствию несть конца. И в Духа Святаго, Господа истиннаго и Животворящаго, Иже от Отца исходящаго, иже со Отцем и Сыном споклоняема и сславима, глаголавшаго пророки. И во едину святую соборную и апостольскую Церковь. Исповедую едино Крещение во оставление грехов. Чаю воскресения мертвым. И жизни будущаго века. Аминь.


\medskip\itshape Далее: \normalfont{}


Богородице Дево, радуися, обрадованная Марие, Господь с Тобою, благословенна Ты в женах и благословен Плод чрева Твоего, яко родила еси Христа Спаса, Избавителя душам нашим \itshape (с поклонами трижды)\normalfont{}.


О! Всепетая Мати, рождьшая всех святых святейшее Слово, нынешнее приношение приемши, от всякия напасти избави всех, и грядущия изми муки, вопиющия Ти: Аллилуия \itshape (трижды, с поклонами земными)\normalfont{}.


Непобедимая и божественная сила честнаго и животворящаго Креста Господня, не остави мя грешнаго, уповающаго на Тя \itshape (поклон)\normalfont{}. Пресвятая Владычице моя Богородице, помилуй мя, и спаси мя, и помози ми ныне, в жизни сей, и во исход души моея, и в будущем веце \itshape (поклон)\normalfont{}. Вся небесныя силы, святии ангели и архангели, херувими и серафими, помилуйте мя, и помолитеся о мне грешнем ко Господу Богу, и помозите ми ныне, в жизни сей, и во исход души моея, и в будущем веце \itshape (поклон)\normalfont{}. Ангеле Христов, хранителю мой святыи, помилуй мя и помолися о мне грешнем ко Господу Богу, и помози ми ныне, в жизни сей, и во исход души моея, и в будущем веце \itshape (поклон)\normalfont{}. Святыи великии Иоанне, пророче и Предотече Господень, помилуй мя, и помолися о мне грешнем ко Господу Богу, и помози ми ныне, в жизни сей, и во исход души моея, и в будущем веце \itshape (поклон)\normalfont{}. Святии славнии апостоли, пророцы и мученицы, святителие, преподобнии и праведнии и вси святии, помилуйте мя, и помолитеся о мне грешнем ко Господу Богу, и помозите ми ныне, в жизни сей, и во исход души моея, и в будущем веце \itshape (поклон)\normalfont{}. 


\medskip\itshape После сего молись по три раза с поклонами следующие молитвы.\normalfont{}


Пресвятая Троице Боже наш, слава Тебе.Господи Исусе Христе, Сыне Божии, помилуй мя грешнаго. Слава, Господи, Кресту Твоему Честному. Пресвятая Госпоже Богородице, спаси мя грешнаго раба Твоего. Ангеле Христов, хранителю мой святыи, спаси мя грешнаго раба Твоего. Святии архангели и ангели, молите Бога о мне грешнем. Святыи великии Иоанне, пророче и Предотече, Крестителю Господень, моли Бога о мне грешнем. Святыи славныи пророче Илие, моли Бога о мне грешнем. Святии праотцы, молите Бога о мне грешнем. Святии пророцы, молите Бога о мне грешнем. Святии апостоли, молите Бога о мне грешнем. Святии славнии апостоли и евангелисти: Матфее, Марко, Луко и Иоанне Богослове, молите Бога о мне грешнем. Святии славнии верховнии апостоли Петре и Павле, молите Бога о мне грешнем. Святии велиции трие святителие: Василие Великии, Григорие Богослове и Иоанне Златоусте, молите Бога о мне грешнем. Святителю Христов Николае, моли Бога о мне грешнем. Преподобныи отче Сергие, моли Бога о мне грешнем. Святыи священномучениче и исповедниче Аввакуме, моли Бога о мне грешнем. Святителю Христов и исповедниче Амвросие, моли Бога о мне грешнем. Преподобнии и богоноснии отцы наши, пастырие и учителие вселенней, молите Бога о мне грешнем. Святии вси, молите Бога о мне грешнем.


\itshape После этого помолись святому, имя которого носишь, и святому, празднуемому этого числа, также и другим святым, кому хочешь. Не забудь помолиться и епитимию, какие поклоны имеешь от духовного отца.\normalfont{}


\medskipПотом молись о здравии правящего епископа, духовного отца, родителей, родственников и близких, по три раза произнося с поклонами о здравии и о спасении: 


\medskipМилостиве Господи, спаси и помилуй раб Своих \itshape (поклон) (называй имена, за кого молишься)\normalfont{}. Избави их от всякия скорби, гнева и нужды \itshape (поклон)\normalfont{}. От всякия болезни душевныя и телесныя \itshape (поклон)\normalfont{}. И прости им всякое согрешение, вольное и невольное (поклон). И душам нашим полезная сотвори \itshape (поклон)\normalfont{}. 


\medskip\itshape Потом помолись за упокой духовных отцов, родителей и близких, и за кого имеешь усердие, по три раза произнося с поклонами: \normalfont{}


Покой, Господи, душа усопших раб Твоих \itshape (поклон) (называй имена, за кого молишься)\normalfont{}. И елика в житии сем яко человецы согрешиша, Ты же, яко Человеколюбец Бог, прости их и помилуй (поклон). Вечныя муки избави \itshape (поклон)\normalfont{}. Небесному царствию причастники учини \itshape (поклон)\normalfont{}. И душам нашим полезная сотвори \itshape (поклон)\normalfont{}.


\medskip\itshape Заканчивая молитвы, говори: \normalfont{}


Господи, или словом, или делом, или помышлением согреших во всей жизни моей, помилуй мя и прости мя, милости Твоея ради \itshape (поклон земной)\normalfont{}. Все упование мое к Тебе возлагаю, Мати Божия, сохрани мя во Своем си крове \itshape (поклон земной)\normalfont{}. Упование ми Бог, и прибежище мое Христос, и покровитель ми есть Дух Святыи \itshape (поклон земной)\normalfont{}.


Достойно есть, яко воистину блажити Тя Богородице, присно блаженную и пренепорочную, и Матерь Бога нашего. Честнейшую херувим и славнейшую воистину серафим, без истления Бога Слова рождьшую, сущую Богородицу, Тя величаем \itshape (поклон всегда земной)\normalfont{}.


Слава Отцу и Сыну и Святому Духу \itshape (поклон)\normalfont{}.И ныне и присно и во веки веком, аминь \itshape (поклон)\normalfont{}. Господи помилуй, Господи помилуй, Господи благослови \itshape (поклон)\normalfont{}. 


\mysubsubsection{И отпуст:}


Господи, Исусе Христе, Сыне Божии, молитв ради Пречистыя Твоея Матере и преподобных и богоносных отец наших и всех святых, помилуй и спаси мя грешнаго, яко Благ и Человеколюбец. Аминь.


\medskip\itshape И, наклонившись до земли, без осенения себя крестным знамением, читай прощение: \normalfont{}


Ослаби, остави, отпусти, Боже, согрешения моя, вольная и невольная, яже в слове и в деле, и яже в ведении и не в ведении, яже во уме и в помышлении, яже во дни и в нощи, вся ми прости, яко Благ и Человеколюбец. Аминь. 


\medskip\itshape Вставши, читай молитву сию с поклонами:\normalfont{}


Ненавидящих и обидящих нас прости, Господи Человеколюбче. Благотворящим благо сотвори, братиям и всем сродником нашим, иже и уединившимся, даруй им вся, яже ко спасению прошения и живот вечныи \itshape (поклон)\normalfont{}.В болезнех сущия посети и исцели, в темницах сущих свободи, по водам плавающим Правитель буди и иже в путех шествующим, исправи и поспеши \itshape (поклон)\normalfont{}.Помяни, Господи, и плененныя братию нашу, единоверных православныя веры, и избави их всякаго злаго обстояния \itshape (поклон)\normalfont{}. Помилуй, Господи, давших нам милостыню и заповедавших нам, недостойным, молитися о них, прости их и помилуй \itshape (поклон)\normalfont{}. Помилуй, Господи, труждающихся и служащих нам, милующих и питающих нас, и даруй им вся, яже ко спасению, прошения и живот вечныи \itshape (поклон)\normalfont{}. Помяни, Господи, прежде отшедшия отцы и братию нашу и всели их, идеже присещает свет лица Твоего \itshape (поклон)\normalfont{}. Помяни, Господи, и нашу худость и убожество, и просвети наш ум светом разума святаго Евангелия Твоего, и настави нас на стезю заповедей Твоих, молитвами Пречистыя Твоея Матере и всех святых Твоих, аминь \itshape (поклон)\normalfont{}.


\medskip\itshape Оканчиваются сии молитвы обычным семипоклонным началом (смотри «приходные и исходные поклоны» в начале). \normalfont{}


\medskipПо окончании молитв, как утром, так и вечером, ограждаясь своим нательным крестом, говори: Господи, Исусе Христе, Сыне Божии, благослови и освяти, и сохрани мя силою Живоноснаго Креста Твоего. 


\medskip\itshape После этого целуй крест.


И читай молитву Кресту, перекрестясь:\normalfont{}


Да воскреснет Бог, и разыдутся врази Его, и да бежат от лица Его ненавидящии Его, яко исчезает дым, да исчезнут. Яко тает воск от лица огня, тако да погибнут беси от лица любящих Бога, и знаменающихся крестным знамением, и да возвеселимся рекуще: радуися, Кресте Господень, прогоняя бесы силою на Тебе пропятаго Господа нашего Исуса Христа, во ад сшедшаго, и поправшаго силу диаволю, и давшаго нам Крест Свой Честныи на прогнание всякаго супостата.


О! Пречестныи и Животворящии Кресте Господень, помогай ми, с Пресвятою Госпожею Богородицею, и со всеми святыми небесными силами, всегда и ныне и присно и во веки веком. Аминь.


\section{Чин чтения 12-ти псалмов}
%http://www.molitvoslov.com/text1002.htm 
 


Чин, како подобает пети дванадесять псалмов особь, ихже пояху преподобнии отцы пустыннии во дни и в нощи, о нихже вспоминается в книгах отеческих и в житиях и мучениих святых многих.


Сей же чин принесе от святыя горы преподобный Досифей архимандрит Киево-Печерский.


\itshape По обычнем начале глаголем: \normalfont{}


За молитв святых отец наших, Господи, Исусе Христе, Сыне Божии, помилуй нас, аминь. [поклон]


\itshape Ограждая лице свое, читаем:\normalfont{}


Царю небесныи, Утешителю, Душе истинныи, Иже везде сыи и вся исполняя, сокровище благих и жизни Подателю, прииди и вселися в ны, и очисти ны от всякия скверны, и спаси Блаже, душа наша.

Святыи Боже, Святыи Крепкии, Святыи Безсмертныи, помилуй нас. (трижды с поклонами).

Слава Отцу и Сыну и Святому Духу, и ныне и присно и во веки веком, аминь.

Пресвятая Троице, помилуй нас. Господи, очисти грехи наша; Владыко, прости беззакония наша; Святыи, посети и исцели немощи наша; имене Твоего ради.

Господи помилуй (трижды)

Слава Отцу и Сыну и Святому Духу, и ныне и присно и во веки веком, аминь.

Отче наш, Иже еси на небесех; да святится имя Твое; да приидет царствие Твое; да будет воля Твоя, яко на небеси и на земли; хлеб наш насущныи даждь нам днесь; и остави нам долги наша, яко же и мы оставляем должником нашим; и не введи нас во искушение; но избави нас от лукаваго.

Господи Исусе Христе, Сыне Божии, помилуй нас.

Аминь.

Господи, помилуй, 12.

Слава Отцу и Сыну и Святому Духу, и ныне и присно и во веки веком, аминь.

Приидите, поклонимся Цареви нашему Богу (поклон).

Приидите, поклонимся Христу, Цареви и Богу нашему (поклон).

Приидите, поклонимся и припадем к Самому Господу Исусу Христу, Цареви и Богу нашему (поклон).


\mysubsubsection{Псалом 26}


Господь просвещение мое и Спаситель мой, кого ся убою? Господь Защититель животу моему, от кого ся устрашу? Внегда приближатися на мя злобующе, снести плотии моих, оскорбляющии мя, и врази мои, тии изнемогоша и падоша. Аще ополчится на мя полк, не убоится сердце мое, аще востанет на мя брань, на Нань аз уповаю. Едино просих от Господа, то взыщу: еже жити ми в дому Господни вся дни живота моего, зрети ми красоту Господню и посещати церковь святую Его. Яко скры мя в крове Своем в день зла моего, покры мя в тайне крова Своего. На камень вознесе мя, и ныне се вознесе главу мою на враги моя, обыдох и пожрох в крове Его жертву хваления и воскликновения, пою и воспою Господеви. Услыши, Господи, глас мой, им же воззвах, помилуй мя и услыши мя. Тебе рече сердце мое: Господа взыщу. Взыска Тебе лице мое, лица Твоего, Господи, взыщу. Не отврати лица Твоего от мене и не уклонися гневом от раба Твоего, помощник ми буди, не отрини мене, и не остави мене, Боже Спасителю мой. Яко отец мой и мати моя остависта мя, Господь же восприят мя. Законоположи ми, Господи, в пути Твоем и настави мя на путь правыи,  враг моих ради. Не предаждь мене в душах стужающим ми, яко восташа на мя свидетели неправеднии, и солга неправда себе. Верую видети благая Господня на земли живых. Потерпи Господа, мужайся, и да крепится сердце твое, и потерпи Господа.

\mysubsubsection{Псалом 31}


Блажени, им же отпустишася беззакония, и им же прикрышася греси. Блажен муж, ему же не вменит Господь греха, ниже есть во устех его лесть. Яко умолчах, обетшаша кости моя, зовущу ми весь день. Яко день и нощь отяготе на мне рука Твоя, возвратихся на страсть, егда унзе ми терн. Беззаконие мое познах и греха моего не покрых, рех: исповем на мя беззаконие мое Господеви, и Ты отпустил еси нечестие сердца моего. За то помолится к Тебе всяк преподобныи во время благопотребно, обаче в потопе вод мног, к нему не приближатся. Ты еси прибежище мое, от скорби одержащия мя, радосте моя, избави мя от обышедших мя. Вразумлю тя и наставлю тя на путь сей, вонь же пойдеши, утвержу на тя очи Мои. Не будите яко конь имск, имже несть разума, броздами и уздою челюсти их востягнеши, неприближающимся к тебе. Многи раны грешному, уповающаго же на Господа милость обыдет. Веселитеся о Господе, и радуйтеся, праведнии, и хвалитеся, вси правии сердцем.


\mysubsubsection{Псалом 56}


Помилуй мя, Боже, помилуй мя: яко на Тя упова душа моя. И на сень крилу Твоею надеюся, дондеже прейдет беззаконие. Воззову к Богу Вышнему, Богу, благодеявшему мне. Посла с Небесе и спасе мя, даде в поношение попирающыя мя. Посла Бог милость Свою и истину Свою, и изъят душу мою от среды скимен, поспах смущен. Сынове человечестии, зубы их оружия и стрелы, и язык их меч остр. Вознесися на Небеса, Боже, и по всей земли слава Твоя. Сеть уготоваша ногам моим, и смириша душу мою. Ископаша пред лицем моим яму и впадошася в ню. Готово сердце мое, Боже, готово сердце мое, пою и воспою во славе моей. Востани слава моя, востани псалтырю и гусли, востану рано. Исповемся Тебе в людех, Господи, пою Тебе во языцех. Яко возвеличися до Небес милость Твоя, и даже до облак истина Твоя. Вознесися на Небеса, Боже, и по всей земли слава Твоя.


\itshape Таже, Трисвятое, и по Отче наш:\normalfont{}


\mysubsubsection{Тропари, глас 1}


Объятия Отча отверсти ми потщися, блудно изжих мое житие, но на богатство неизживущее взираю щедрот Твоих, Спасе: ныне обнищавшее мое сердце не презри. Тебе бо, Господи, умилением зову: согреших, Отче, на небо и пред Тобою.


Слава: Егда приидеши, Боже, судити на землю со славою, и вострепещут всяческая, река же огненная пред судищем потечет, книги разгнутся, и тайная обличатся, тогда избави мя огня негасимаго, и сподоби мя одесную Тебе стати, Судие праведныи.


И ныне: Матерь Тя Божию молим вси Дево; в щедроты Твои прибегаем любовию и в благодать Твою; Тебе бо имамы грешнии спасение, и Тебе стяжахом в напастех Едину всенепорочную.


Посем: Господи, помилуй, 30.

Слава Отцу и Сыну и Святому Духу, и ныне и присно и во веки веком, аминь.

Честнейшую херувим, и славнейшую воистину серафим, без истления Бога, Слова рождьшую, сущую Богородицу Тя величаем.

И поклон.

Именем Господним благослови, отче.

За молитв святых отец наших, Господи, Исусе Христе, Сыне Божии, помилуй нас. Аминь.


\itshape В Великий пост (кроме суббот и воскресений), а также и в другие посты (и на масленнице в среду и пятницу):\normalfont{}


Господи помилуй, 40, Слава, и ныне. (И 17 поклонов с молитвою св. Ефрема Сирина).

\begin{enumerate}

\item[1.] Честнейшую херувим… (поклон земной великий). Именем Господним… За молитв святых отец наших…


\itshape И молитва св. Ефрема:\normalfont{}


\item[2.] Господи и Владыко животу моему, дух уныния, небрежения, сребролюбия и празднословия отжени от мене (земной великий поклон).


\item[3.] Дух же целомудрия, смирения, терпения и любве даруй ми, рабу Твоему (земной великий поклон).


\item[4.] Ей, Господи Царю, даждь ми зрети моя согрешения и еже не осуждати брата моего, яко благословен еси во веки, аминь (земной великий поклон).


\item[5--6.] Господи Исусе Христе, Сыне Божии, помилуй мя грешнаго (дважды с поклонами).

\item[7.] Боже, милостив буди мне грешному (поклон).


\item[8.] Боже, очисти грехи моя и помилуй мя (поклон).


\item[9.] Создавыи мя Господи, помилуй. (поклон).


\item[10.] Без числа согреших Господи, прости мя (поклон).


\item[11--16.] Повторяем молитвы 5--10 с земными поклонами.


\item[17.] Затем еще раз читаем всю молитву св. Ефрема Сирина и земной великий поклон.

\end{enumerate}

Приидите, поклонимся, трижды.


\mysubsubsection{Псалом 33}


Благословлю Господа на всяко время, выну хвала Его во устех моих. О Господе похвалится душа моя, да услышат кротцыи и возвеселятся. Возвеличите Господа со мною, и вознесем имя Его вкупе. Взысках Господа и услыша мя, и от всех скорбий моих избави мя. Приступите к Нему и просветитеся, и лица ваша не постыдятся. Сей нищий возва, и Господь услыша и, и от всех скорбий его спасе и. Ополчится Ангел Господень окрест боящихся Его, и избавит их. Вкусите и видите, яко благ Господь, блажен муж, иже уповает Нань. Бойтеся Господа, вси святии Его, яко несть лишения боящимся Его. Богатии обнищаша и взалкаша, взыскающии же Господа не лишатся всякаго блага. Приидете, чада, послушайте мене, страху Господню научу вас. Кто есть человек хотяи живот, любяи дни видети благи? Удержи язык свой от зла, и устне свои, еже не глаголати льсти. Уклонися от зла и сотвори благо, взыщи мир, и пожени и. Очи Господни на праведныя, и уши Его в молитву их. Лице же Господне, на творящия злая, еже потребити от земли память их. Возваша праведнии, и Господь услыша их, и от всех печалей их избави их. Близ Господь сокрушенных сердцем, и смиренныя духом спасет. Многи скорби праведным, и от всех их избавит я Господь. Хранит Господь вся кости их, ни едина от них сокрушится. Смерть грешников люта, и ненавидящии праведнаго прегрешат. Избавит Господь душа раб Своих, и не прегрешат вси, уповающии Нань.


\mysubsubsection{Псалом 38}


Рех: сохраню пути моя, еже не согрешати языком моим: положих устом моим хранило, внегда востати грешному предо мною. Онемех и смирихся, и умолчах от благ, и болезнь моя обновися. Согреяся сердце мое во мне, и в поучении моем разгорится огнь. Глаголах языком моим: скажи ми, Господи, кончину мою и число дний моих, кое есть, да разумею, чесо лишаюся аз? Се пядию измерены положил еси дни моя, и состав мой яко нивочто же пред Тобою; обаче всяческая суета всяк человек живыи. Иде убо образом ходит человек, обаче всуе мятется: сокровищует, и невесть, кому собирает я. И ныне кто терпение мое, не Господь ли? и состав мой от Тебе есть. От всех беззаконии моих избави мя, поношение безумному дал мя еси. Онемех и не отверзох уст моих, яко Ты сотвори. Отстави от мене раны Твоя, от крепости бо руки Твоея аз исчезох. Во обличении о беззаконии показал еси человека, и истаял еси, яко паучину, душу его; обаче всуе всяк человек. Услыши молитву мою Господи, и моление мое внуши; слез моих не премолчиши, яко пресельник аз от Тебе и пришлец, якоже вси отцы мои. Ослаби ми, да почию, прежде даже не отъиду, и к тому не буду.


\mysubsubsection{Псалом 40}


Блажен разумеваяи на нища и убога, в день лют избавит и Господь. Господь сохранит и, и живит и, и ублажит и на земли, и не предаст его в руки врагов его. Господь поможет ему на одре болезни его; все ложе его обратил еси в болезнь его. Аз рех: Господи, помилуй мя, исцели душу мою, яко согреших Ти. Врази мои реша мне злая: когда умрет и погибнет имя его? И вхождаше видети; всуе глаголаше: сердце его собра беззаконие ему; исхождаше вон, и глаголаше вкупе. На мя шептаху вси врази мои, на мя помышляху злая мне. Слово законопреступное возложиша на мя, еда спяи не приложит воскреснути. Ибо человек мира моего, нань же уповах, ядыи хлебы моя, возвеличи на мя пяту. Ты же, Господи, помилуй мя и воздвигни мя и воздам им. О сем познах, яко восхоте мя; яко не возрадуется враг мой о мне. Мене же за незлобие прият, и утвердил мя еси пред Собою в век. Благословен Господь Бог Израилев от века и до века, будет, будет.

Таже Трисвятое, и по Отче наш:


\mysubsubsection{Тропари, глас 4:}


Смиренную душу мою посети Господи, иже во гресех  житие все изжившу. Но яко  блудницу приими мене и спаси мя.

Слава: Все житие мое блудно изжих Господи, со блудницами окаянныи; яко же блудныи умилением зову: Отче Небесныи, согреших, очисти, и приими мя, и не отрини мене от Себе, от Тебе удалившагося, и неплодными делы ныне обнищавша.

И ныне: К Богородице прилежно ныне притецем грешнии, со смирением припадающе и покаянием, вопиюще из глубины душевныя: Владычице, помози, мсилосердовавши на ны, потщися, яко изгибаем от множества грехов, не отврати раб Своих тощь, Тебе бо едину помощницу имамы.


Посем: Господи, помилуй, 30. И далее, якоже пренаписахом.


Таже, Приидите, поклонимся, трижды.


\mysubsubsection{Псалом 69}


Боже, в помощь мою воньми, Господи, помощи ми потщися. Да постыдятся и посрамятся ищущии душу мою. Да возвратятся вспять и постыдятся, хотящии ми злая. Да возвратятся абие стыдящеся глаголющии ми: благо же, благо же. Да возрадуются и возвеселятся о Тебе, вси ищущии Тебе Боже, и да глаголют выну: да возвеличится Господь, любящии спасение Твое. Аз же нищ есмь и убог, Боже, помози ми. Помощник мой и Избавитель мой еси Ты, Господи, не закосни.


\mysubsubsection{Псалом 70}


На Тя, Господи, уповах, да не постыжуся в век. Правдою Твоею изми мя и избави мя. Приклони ко мне ухо Твое и спаси мя. Буди ми в Бог Защититель и в место крепко спасти мя, яко утверждение мое и прибежище мое еси Ты. Боже мой, изми мя из руки грешнаго, из руки законопреступнаго и обидящаго. Яко Ты еси терпение мое, Господи, Господи упование мое от юности моея. В Тебе утвердихся от утробы, от чрева матере моея Ты еси мой Покровитель. О Тебе пение мое выну, яко чудо бых многим, и Ты помощник мой крепок. Да исполнятся уста моя похвалы, яко да воспою славу Твою, весь день великолепие Твое. Не отверзи мене во время старости, внегда исчезати крепости моей, не остави мене. Яко реша врази мои мне, и стрегущии душу мою совещаша вкупе. Глаголюще: Бог оставил есть его, поженете и имете его, яко несть избавляяи. Боже мой, не удалися от мене, Боже мой, в помощь мою вонми. Да постыдятся и исчезнут оклеветающии душу мою, да облекутся в студ и срам ищущии злая мне. Аз же всегда уповаю на Тя и приложу на всяку похвалу Твою. Уста моя возвестят правду Твою, весь день спасение Твое. Яко не познах книжна, вниду в силе Господни, Господи, помяну правду Твою Единаго. Боже мой, яже научил мя еси от юности моея, и до ныне возвещу чудеса Твоя. И даже до старости и маторства, Боже мой, не остави мене, дондеже возвещу мышцу Твою роду всему грядущему. Силу Твою и правду Твою, Боже, даже до вышних, яже сотворил ми еси величия. Боже, кто подобен Тебе? Елики явил ми еси скорби многи и злы, и обращь оживил мя еси, и от бездн земли возведе мя. Умножил еси на мне величествие Твое, и обращь утешил мя еси, и от бездн земли паки возведе мя. Ибо аз исповемся Тебе в людех, Господи, в сосудех псаломских истину Твою, Боже, пою Тебе в гуслех, Святый Израилев. Возрадуетеся устне мои, егда пою Тебе, и душа моя, юже еси избавил. Еще же и язык мой весь день поучится правде Твоей, егда постыдятся и посрамятся ищущии злая мне.


\mysubsubsection{Псалом 76}


Гласом моим ко Господу воззвах, гласом моим к Богу, и внят ми. В день печали моея Бога взысках, руками моима нощию пред Ним, и не прельщен бых. Отвержеся утешитися душа моя, помянух Бога и возвеселихся, возскорбех и пренеможе дух мой. Предваристе стражбы очи мои, смутихся и не глаголах. Помыслих дни первыя, и лета вечная помянух, и поучихся. Нощию сердцем моим глумляхся, и тужаше дух мой. Еда во веки отринет Господь, и не приложит благоволити паки? Или до конца милость Свою отсечет, сконча глагол от рода в род? Еда забудет ущедрити Бог? Или удержит во гневе Своем щедроты Своя? И рех: ныне начах, си измена десница Вышняго. Помянух дела Господня, яко помяну от зачала чудес Твоих. И поучуся во всех делех Твоих, и в зачинаниих Твоих поглумлюся. Боже, во святем путь Твой; кто Бог велии яко Бог наш? Ты еси Бог творяи чудеса. Познал еси в людех силу Твою, избавил еси мышцею Твоею люди Твоя, сыны Ияковля и Иосифовы. Видеша Тя воды, Боже, видеша Тя воды, и убояшася; смутишася бездны множеством шума вод. Глас даша облацы, ибо стрелы Твоя преходят, глас грома Твоего в колеси. Осветиша молния Твоя вселенную, подвижася и трепетна бысть земля. В мори путие Твои, и стези Твоя в водах многих, и стопы Твоя не познаются. Наставил еси яко овца люди Твоя, рукою Моисеовою и Ааронею.

Таже, Трисвятое, и по Отче наш.

Молитва Исусова.


\mysubsubsection{Тропари, глас 6:}


Помышляю день страшныи и плачуся деянии моих лукавых; како отвещаю Безсмертному Царю? Коим ли дерзновением воззрю на Судию блудныи аз? Благоутробныи Отче, Сыне Единородныи, Душе Святыи, помилуй нас.

Слава. Во юдоли плачевне на месте идеже положи, егда сядеши Милостиве, сотворити праведныи суд, не обличи моя сокровенная, ниже посрами мене пред ангелы, но пощади мя, Боже, и помилуй мя.

И ныне: Милосердия двери отверзи нам, Благословенная Богородице Дево, надеющиися на Тя  не  погибнем, но да избавимся Тобою от бед, Ты бо еси спасение роду християнскому.

Посем: Господи, помилуй, 30, и далее якоже преднаписася.

Таже, Приидите, поклонимся, трижды.


\mysubsubsection{Псалом 101}


Господи, услыши молитву мою, и вопль мой к Тебе да приидет. Не отврати лица Твоего от мене, в онь же день аще скорблю. Приклони ко мне ухо Твое, в онь же  день аще призову Тя, скоро услыши мя. Яко исчезоша яко дым дние мои, и кости моя яко сушило сосхошася. Уязвен бых яко трава, и изсше сердце мое, яко забых снести хлеб мой. От гласа воздыхания моего прильпе кость моя плоти моей. Уподобихся неясыти пустынному, бых яко нощныи вран на нырищи. Забдех и бых яко птица особящаяся на зде. Весь день поношаху ми врази мои, и хвалящии мя мною кленяхуся. Зане пепел яко хлеб ядях, и питие мое с плачем растворях. От лица гнева Твоего и ярости Твоея: яко вознес низверже мя. Дние мои яко сень уклонишася, и аз яко сено изсхох. Ты же Господи, во веки пребываеши, и память Твоя в род и род. Ты воскрес ущедриши Сиона, яко время ущедрити его, яко прииде время. Яко благоволиша раби Твои камение его, и персть его ущедрят. И убоятся языцы имени Господня, и вси царие земстии славы Твоея. Яко созиждет Господь Сиона, и явится в славе Своей. Призре на молитву убогих, и не уничижи моления их. Да напишется сие в род ин, и людие зиждемии восхвалят Господа. Яко призре с высоты святыя Своея, Господь с Небесе на землю призре. Услышати воздыхания окованных, разрешити сыны умерщвленных. Возвестити в Сионе имя Господне, и хвалу Его во Иеросалиме. Внегда собратися людем вкупе, и царие  работати Господеви. Отвеща ему на пути крепости его, умаление дний моих возвести ми. Не возведи мене в преполовение дний моих, в род и род лета Твоя. В началех Ты Господи, землю основа, и дела рук Твоих суть небеса. Та погибнут, Ты же пребываеши; и вся, яко риза обетшают, и яко одежду  свиеши их, и изменятся. Ты же тожде еси, и лета твоя не оскудеют. Сынове раб Твоих вселятся, и семя их во век исправится.


\mysubsubsection{Молитва Манассии, царя Иудейска}


Господи Вседержителю, Боже отец наших, Авраамов и Исааков и Ияковль, и семене их праведнаго. Сотворивыи небо и землю со всею лепотою их, и сопныи море словом повеления Твоего. Затворивыи бездну, и запечатлев ю страшным и славным именем Твоим, Его же вся боятся, и трепещут от лица славы Твоея. Яко непостоянна велелепота славы Твоея, и не стерпим гнев еже на грешники прещения Твоего. Безчислена же и неизследованна милость обещания Твоего. Ты бо еси Господь вышнии, милосерд, долготерпелив и многомилостив, и каяся о злобах человеческих. Но Ты Господи, по множеству благости Твоея, обеща покаяние и оставление согрешшим к Тебе, и множеством щедрот Твоих нарече покаяние грешником во спасение. Ты убо Господи Боже праведных, неси положил покаяние праведным Твоим, Аврааму и Исааку и Иякову, не согрешившим пред Тобою, но положил еси покаяние мне грешному, зане согреших Ти паче числа песка морскаго. Умножишася беззакония моя Господи, умножишася, и несмь достоин воззрети и видети высоту небесную, от множества неправд моих связан есмь многими юзами железными. Яко не возвести ми главы моея, и несть ми восклонения. Зане прогневах ярость Твою, и лукавое пред Тобою сотворих. И не сотворих воли Твоея, не сохраних повелении Твоих. И ныне поклоняю колена сердца моего, и молю яже от Тебе благость. Согреших Господи, согреших, и беззакония моя аз свем. Но прошу и молюся Тебе: отради ми Господи, отради ми, и не погуби мене со беззаконьми моими. Ниже в век враждовав соблюдеши зол моих, и не осуди мене в преисподних земли. Зане Ты еси Боже, Бог кающихся, да и на мне явиши всю благость Твою, яко недостойна суща спасеши мя, по мнозей милости Твоей. И восхвалю Тя всегда, во вся дни живота моего. Яко Тебе поют вся силы небесныя, и Твоя есть слава во веки, аминь.

Слава в вышних Богу, и на земли мир, в человецех благоволение. Хвалим Тя, благословим Тя (поклон), кланяемтися, славословим Тя (поклон), благодарим Тя, великия ради славы Твоея (поклон). Господи, Царю небесныи. Боже Отче Вседержителю, и Господи Сыне Единородныи, Исусе Христе, и Святый Душе. Господи Боже Агньче Божии, Сыне Отечь, вземляи грех мiра, помилуй нас, Вземляи грехи мiра, приими молитвы наша; седяи одесную Отца, помилуй нас. Яко Ты еси Един Свят, Ты еси Един Господь, Исус Христос, в славу Богу Отцу, аминь.

На всяк день благословим Тя, и восхвалим  имя Твое во веки и в век века.

Господи, прибежище бысть нам в род и род. Аз рех: Господи, помилуй мя, и исцели душу мою, яко согреших Тебе. Господи, к Тебе прибегох, научи мя творити волю Твою, яко Ты еси Бог мой. Яко от Тебе есть источник живота, во свете Твоем узрим свет. Пробави милость Твою ведущим Тя.

Сподоби, Господи, в день сей без греха сохранитися нам. Благословен еси, Господи, Боже отец наших, и хвально, и прославлено имя Твое во веки, аминь.

Буди, Господи, милость Твоя на нас, якоже уповахом на Тя. Благословен еси, Господи, научи нас оправданием Твоим. Благословен еси, Владыко, вразуми нас оправданием Твоим. Благословен еси, Святыи, просвети нас оправданием Твоими.

Господи, милость Твоя во веки, и дела  руку Твоею не презри. Тебе подобает хвала, Тебе подобает пение, Тебе слава подобает, Отцу, и Сыну, и Святому Духу, ныне и присно и во веки веком, аминь.


\mysubsubsection{Молитва великаго Евстратия}

Величая величаю Тя Господи, яко призрел еси на смирение мое, и неси мене затворил в руках вражиих, но спасл еси от бед душу мою. И ныне Владыко, да покрыет мя рука Твоя, и да приидет на мя милость Твоя, яко смятеся душа моя, и болезненна есть во исхождении своем, от окаяннаго ми и смраднаго телесе сего. Да некогда лукавыи супостата совет срящет, и препнет ю во тьме, за неведомыя и ведомыя в житии сем бывшия ми грехи. Милостив ми буди, Владыко, и да не узрит душа моя темнаго взора лукавых бесов, но да приимут ю ангели Твои светлии  и пресветлии. Даждь славу имени Твоему святому, и Твоею силою возведи мя на божественныи Твой суд. Внегда судитимися, да не приимет мене рука князя мiра сего, исторгнути мя грешника во глубину адову. Но предстани ми, и буди мой Спаситель и Заступник. Телесныя бо сия муки, веселие суть рабом Твоим. Помилуй, Господи, осквернившуюся страстьми жттия сего душу мою, и чисту ю ради покаяния и исповедания приими. Яко благословен еси во веки веком, аминь.


Таже, Трисвятое, и по Отче наш.

Молитва Исусова.


\mysubsubsection{Тропари, глас 8}


Оком милосердым Господи, виждь мое смирение. Яко по малу жизнь моя скончевается, и от дел несть спасения. Сего ради молюся: оком благоутробным Ти Господи, виждь смирение мое и спаси мя.

Слава: Век мой скончавается, и страшный Твой престол готовится, житие мое мимо ходит, суд мене ждет, претя мне огненною мукою, и пламенем негасимым. Слезам тучу подаждь ми и угаси его силу, хотяи спастися всем человеком.

И ныне: Иже нас ради рождеися от Девы, и распятие претерпев, Благии, испровергии смертию смерть, и воскресение явлеи, яко Бог. Не презри их же созда рукою Своею. Яви человеколюбие Свое, Милостиве. Приими Рождьшую Тя Богородицу молящуюся за ны, и спаси Спасе наш, люди согрешшыя.

Таже, Господи помилуй, 30. 

Честнейшую херувим, и славнейшую воистину серафим, без истления Бога, Слова рождьшую, сущую Богородицу Тя величаем. (поклон).

Слава Отцу и Сыну и Святому Духу (поклон), и ныне и присно и во веки веком, аминь (поклон).

Господи помилуй, Господи помилуй, Господи благослови (поклон).

И отпуст: Господи Исусе Христе, Сыне Божии, молитв ради Пречистыя Твоея Матере, и преподобных и богоносных отец наших, и всех ради святых, помилуй и спаси мя грешнаго, яко Благ и Человеколюбец. Аминь.

Затем "--- прощение.

Аще ли пост, творим поклоны по преждеписанному.

По 17 поклонех Слава, и ныне. Господи помилуй (дважды), Господи благослови (без поклонов). И отпуст псалтырный: Господи Исусе Христе, Сыне Божии, молитв ради Пречистыя Твоея Матере, Силою Честнаго и Животворящаго Креста, и святых небесных сил безплотных, и преподобных и богоносных отец наших, и святаго пророка Давыда (и святаго, его же есть день) и всех святых, помилуй и спаси мя грешнаго, яко Благ и Человеколюбец. Аминь.

И прощение. Наклонившись до земли, не крестясь, читай:

Ослаби, остави, отпусти Боже, согрешения моя, вольная и невольная, яже в слове и в деле, и яже в ведении и не в ведении, яже во уме и в помышлении, яже во дни и в нощи, вся ми прости, яко Благ и Человеколюбец, аминь.

Вставши, читай молитву сию с поклонами:

Ненавидящих и обидящих нас прости, Господи Человеколюбче. Благотворящим благо сотвори, братиям и всем сродником нашим, иже и уединившимся, даруй им вся, яже ко спасению прошения и живот вечныи (поклон).

В болезнех сущия посети и исцели, в темницах сущих свободи, по водам плавающим Правитель буди и иже в путех шествующим исправи и поспеши (поклон).

Помяни Господи, и плененныя братию нашу, единоверных православныя веры, и избави их всякаго злаго обстояния (поклон).

Помилуй Господи, давших нам милостыню и заповедавших нам, недостойным, молитися о них, прости их и помилуй (поклон).

Помилуй Господи, труждающихся и служащих нам, милующих и питающих нас, и даруй им вся, яже ко спасению, прошения и живот вечныи (поклон).

Помяни Господи, прежде отшедшия отцы и братию нашу и всели их, идеже присещает свет лица Твоего (поклон).

Помяни Господи, и нашу худость и убожество, и просвети наш ум светом разума святаго Евангелия Твоего, и настави нас на стезю заповедей Твоих, молитвами Пречистыя Твоея Матере и всех святых Твоих, аминь (поклон).

Господи помилуй (трижды). 

И обычные исходные поклоны.


\section{Келейное правило}
%http://www.molitvoslov.com/text1001.htm 
 


Начало правилу келейному

Подобает ведати, како начати правило свое в келии. Став убо на обычном своем месте, сотвори со смирением сердца и сокрушенным помыслом поклоны до земли, глаголя сице:

\begin{enumerate}

\item Боже, очисти мя, грешнаго, и помилуй мя (поклон).

\item Создавыи мя, Господи, помилуй мя (поклон).

\item Без числа согреших, Господи, прости мя (поклон).

\item Боже, милости буди мне, грешному (поклон).

\item Боже, прости беззакония моя и согрешения (поклон).

\item Пресвятая Владычице моя Богородице, помилуй мя и спаси мя, и помози ми ныне в жизни сей, и во исход души моея, и в будущем веце (поклон).

\item Непобедимая и божественная сила, честнаго и животворящаго Креста Господня, не остави мя, грешнаго, уповающаго на тя (поклон).

\item Вся небесныя силы, святии ангели и архангели, херувими и серафими, помилуйте мя и помолитеся о мне, грешнем, ко Господу Богу, и помозите ми ныне, в жизни сей, и во исход души моея, и в будущем веце (поклон).

\item Ангеле Христов, хранителю мой святыи, помилуй мя и помолися о мне грешнем ко Господу Богу, и помози ми ныне, в жизни сей, и во исход души моея, и в будущем веце (поклон).

\item Святыи великии Иоанне, пророче и Предотече Господень, помилуй мя и помолися о мне грешнем ко Господу Богу, и помози ми ныне, в жизни сей, и во исход души моея, и в будущем веце (поклон).

\item Святии славнии апостоли, пророцы и мученицы, святителие, лреподобнии и праведнии и вси святии, помилуйте мя и помолитеся о мне грешнем ко Господу Богу, и помозите ми ныне, в жизни сей, и во исход души моея, и в будущем веце (поклон).

\item Господи, или словом, или делом, или помышлением, согреших во всей жизни моей, помилуй мя и прости мя милости Твоея ради (поклон).

\end{enumerate}

\mysubsubsection{Посем, молитва святаго Макария, 1-я}

Боже, очисти мя, грешнаго, яко николиже благо сотворих пред Тобою, но избави мя от лукаваго, и да будет во мне воля Твоя, да неосужденно отверзу уста моя недостойная, и восхвалю имя Твое святое, Отца и Сына и Святаго Духа, ныне и присно и во веки веком. Аминь.


\mysubsubsection{Молитва 2-я, святаго Исаака Сириянина}

Господи Исусе Христе Боже мой, посещаяи тварь Свою, Тебе явлены страсти моя, и немощь естества моего, и крепость супостата моего. Ты Сам, Владыко, покрый мя от злобы его, занеже сила его крепка и естество мое страстно, и сила немощна. Ты убо Благии, ведыи немощь мою, иже нося неудобьство безсилия моего, сохрани мя от помысла смущена, и потопа страстей, и достойна мя сотвори, сея службы святыя, да не како в страстех моих растлю сладость ея, и обрящуся безстуден пред Тобою и дерзостив. Но милостию Си помилуй мя, яко благословен еси во веки. Аминь.


\mysubsubsection{Молитва 3-я, святаго Стефана Фивейскаго}

Владыко Господи Исусе Христе, Боже мой, Ты Помощник ми буди, в руку Твоею аз, да не оставиши мене согрешати, яко прельщен есмь, да не оставиши мене последовати хотению моему лукавому, да не оставиши мене погибнути во гресех моих, ущедри создание Свое, не отверзи мене от лица Твоего грех ради моих, яко к Тебе прибегох, исцели душу мою, яко согреших Ти, пред Тобою суть вси оскорбляющие мя, и ищущие дущу мою изъяти ю, и несть ми прибежища иного, токмо к Тебе Господи, Господи, спаси мя милости Твоея ради, яко Ты еси Господи, сильныи во всяческих. Яко Твое есть Царство и сила и слава, Отца и Сына и Святаго Духа, ныне и присно и во веки веком. Аминь.


\mysubsubsection{Молитва 4-я, святаго Иоанна Златоустаго}

Господи, аще хощу, аще не хощу, спаси мя, понеже бо аз яко кал любовещныи, греховную скверну желаю, но Ты яко Благ и Всесилен, можеши ми возбранити. Аще бо праведнаго помилуеши, ничтоже велие, ащи чистаго спасеши, ничтоже дивно, достойни бо суть милости Твоея. Но на мне паче Владыко, окаяннем и грешнем и сквернем удиви милость Свою, покажи благоутробие Свое, Тебе бо оставлен есмь нищии, обнищав всеми благими делы. Господи, спаси мя милости Твоея ради, яко благословен еси во веки. Аминь.


\mysubsubsection{Молитва 5-я}

Владыко, помилуй мя благости Твоея ради, и не остави мене заблудити от Твоея воли. И не отверзи убогия молитвы моея от Твоего лица, но услыши Господи, глас молитвы моея, егда молюся Тебе во дни и в нощи, и приими яко избранно кадило, и не возбрани грехов моих ради благости Твоея, но спаси мя имене Твоего ради святаго. Твое бо есть единаго, еже миловати и спасати нас, и Тебе славу возсылаем, Отцу и Сыну и Святому Духу, ныне и присно, и во веки веком. Аминь.


Таже, помолився, и предочистив себя, целуй крест свой, иже носиши на персех своих, и знаменаяся им, глаголи:

Господи Исусе Христе, Сыне Божии, благослови и освяти и сохрани мя, силою Креста живоноснаго Ти.

\mysubsubsection{Посем:}

Слава Тебе, Боже наш, слава Тебе всяческих ради. (трижды).

За молитв святых отец наших, Господи Исусе Христе, Сыне Божии, помилуй нас. Аминь. Таже:

Царю Небесныи.

Трисвятое. По Отче наш:

Господи помилуй, 12. Слава, и ныне.

Приидите поклонимся, трижды.

Псалом 50: Помилуй мя, Боже:

Символ Веры: Верую во единаго Бога:

Таже, или каноны глаголи, или молитвы Исусовы, или поклоны.

По завершении канонов глаголи:

Достойно есть. Трисвятое. По Отче наш: тропари канонов.

Господи помилуй, 40.

Таже Честнейшую херувим.

Слава, и ныне, Господи помилуй, дважды, Господи, благослови (с поклонами).

И отпуст, по обычаю.

Неции же глаголют последи, Господи помилуй, 40. Молитвы канонов. И исповедание.

Таже: Честнейшую. И отпуст.
\longpage[2]\mychapterending[1]

\mychapter{Суточный круг}
%http://www.molitvoslov.com/content/sutochniykrug

 

\section{Шестый час}
%http://www.molitvoslov.com/text1013.htm 
 


Приидите, поклонимся (трижды, и поклоны три).


\mysubsubsection{Псалом 53}


Боже, во имя Твое спаси мя, и в силе Твоей суди ми. Боже, услыши молитву мою, внуши глаголы уст моих. Яко чуждии восташа на мя, и крепцыи взыскаша душу мою, и не предложиша Бога пред собою. Се бо Бог помогает ми, и Господь заступник души моей. Отвратит злая врагом моим, истиною Твоею потреби их. Волею пожру Тебе, исповемся имени Твоему, Господи, яко благо. Яко от всякия печали избави мя, и на враги моя воззре око мое.


\mysubsubsection{Псалом 54}


Внуши Боже, молитву мою, и не презри моления моего. Вонми ми, и услыши мя. Возскорбех печалию моею, и смутихся от гласа вражия, и от стужения грешнича. Яко уклониша на мя беззаконие, и во гневе враждоваху ми. Сердце мое смутися во мне, и страх смерти нападе на мя. Боязнь и трепет прииде на мя, и покры мя тма. И рех: кто даст ми криле, яко голуби, и полещу и почию. Се удалихся бегая, и водворихся в пустыни. Чаях Бога спасающаго мя, от малодушия и бури. Потопи Господи, и раздели языки их. Яко видех беззаконие и пререкание во граде. День и нощь обыдет и по стенам его, беззаконие и труд посреде его и неправда. И не оскуде от пути его лихва и лесть. Яко аще бы враг поносил ми, претерпел бых убо. И аще бы ненавидяи мя на мя велеречевал, укрылбыхся от него. Ты же человече равнодушне, владыко мой и знаемый мой, иже купно насладил мя еси брашна, во храме Божии ходихове единомышлением. Да приидет же смерть на ня, и снидут во ад живи, яко лукавство в жилищих их, посреде их. Аз к Богу возвах, и Господь услыша мя. Вечер и заутра и полудне, повем и возвещу, и услышит глас мой. Избавит миром душу мою от приближающихся мне, яко во мнозе бяху со мною. Услышит Бог и смирит их, Сыи прежде век. Несть бо им изменения, яко не убояшася Бога. Прострет руку свою на воздаяние, оскверниша завет его. Разделишася от гнева лица его, и приближишася сердца их, умякнуша словеса их паче елея, и та суть стрелы. Возверзи на Господа печаль твою, и Той тя препитает, не даст в век молвы праведнику. Ты же Боже, низведеши их в студенец истления. Мужие крове и льсти не преполовят дней своих, аз же Господи, уповаю на Тя.


\mysubsubsection{Псалом 90. }


Живыи в помощи Вышняго, в крове Бога небеснаго водворится. Речет Господеви: Заступник мой еси, и прибежище мое, Бог мой и уповаю Нань. Яко той избавит тя от сети ловчи, и от словесе мятежна. Плещьма Своима осенит тя, и под крыле Его надеешися. Оружие обыдет тя истина Его, не убоишися от страха нощнаго, от стрелы летящия во дне. От вещи во тме преходящия, от сряща и беса полуденнаго. Падет от страны твоея тысяща, и тма одесную тебе, к тебе же не приближится. Обаче очима своима смотриши, и воздаяние грешником узриши. Яко Ты Господи, упование мое; Вышняго положил еси прибежище твое. Не приидет к тебе зло, и рана не приближится к телеси твоему. Яко ангелом Своим заповесть о тебе, сохранити тя во всех путех твоих. На руках возмут тя, да некогда преткнеши о камень ноги твоея. На аспида и василиска наступиши, и попереши льва и змия. Яко на Мя упова, и избавлю и, покрыю и, яко позна имя мое. Воззовет ко Мне, и услышу и, с ним есмь в скорби, изму и, и прославлю его. Долготу дней исполню и, и явлю ему спасение Мое. 

Слава, и ныне. Аллилуия (трижды).


\itshape Тропарь по уставу.\normalfont{} 

Аще ли же пост, глаголем сии тропарь трижды: Иже в шестыи день же и час, на Кресте пригвождеи, Иже в раи дерзновенныи от Адама грех, и согрешении наших рукописание раздери, Христе Боже, и спаси нас.

\itshape Стих: \normalfont{}Аз к Богу возвах, и Господь услыша мя.

Стих: Вечер и заутра и полудне, повем и возвещу, и услышит глас мой.

Слава, и ныне, богородичен: Яко не имамы дерзновения, за премногия грехи наша, но Ты, иже от Тебе рождьшагося, моли Богородице Дево, много бо может молитва Матерня, на умоление Владыки. Не презри грешных мольбы Всечистая, яко милостив есть, и спасти могии, Иже страдати нас ради изволивыи. 

Аще ли пост, чтется паремия, и в лествице. Таже, Скоро да предварят ны щедроты Твоя Господи, яко обнищахом зело, помози нам, Боже Спасителю наш; славы ради имене Твоего Господи, избави нас, очисти грехи наша, имене Твоего ради. Трсвятое, и по Отче наш, кондак, по уставу. 

\itshape Аще ли же пост, глаголем тропари сия, глас 2: \normalfont{}Спасение содея посреде земли, Христе Боже, на Кресте пречистеи руце Свои простер, собирая вся языки вопиющия: Господи, слава Тебе. 

Слава. Пречистому Ти образу покланяемся Благии, просяще прощения прегрешением нашим, Христе Боже, волею бо благоизволил еси взыти на Крест, да избавиши их же созда от работы вражия. Тем благодаряще вопием Ти: радости вся исполнивыи, Спасе наш, пришедыи спасти мир (поклон великий). 

И ныне. Милосердия сущи источник, милости сподоби нас, Богородице, призри на люди согрешившия, яви яко присно силу Твою, на Тя бо уповающе: радуися, вопием Ти, якоже иногда Гавриил, безплотным архистратиг. 

Господи помилуй (40). Иже на всяко время… Господи помилуй (трижды). Слава, и ныне. Честнейшую херувим… Именем Господним благослови, отче. За молитв святых отец наших… И сотворяем три великия поклоны, глаголюще молитву: Господи и Владыко животу моему… И поклон един. Таже, Владыко Боже, Отче Вседержителю… 

\itshape Аще ли кроме поста, глаголем молитву сию:\normalfont{} Боже и Господи силам, и всея твари Содетелю, Иже за милосердие неизреченныя милости Твоея, единороднаго Сына Твоего, Господа нашего Исуса Христа, низпослав на спасение роду нашему, и честным Его Крестом, рукописание грех наших растерзав, и обличив тем начала и власти тмы. Сам, Владыко Человеколюбче, приими от нас грешных, благодарныя сия молитвы, и избави нас от всякаго всегубительнаго и мрачнаго грехопадения, и всех иже озлобити нас ищущих, видимых и невидимых враг. Пригвозди страсе Твоем плоти наша, и не уклони сердец наших в словеса и помышления лукавствия, но любовию Твоею уязви душа наша, да к Тебе всегда взирающе, и еже от Тебе светом наставляеми, Тебе неприкосновеннаго и присносущнаго зряще Света, непрестанное Тебе исповедание, и благодарение возсылаем, безначальному Отцу, со единородным Ти Сыном, и с Пресвятым и Благим и Животворящим Ти Духом, ныне и присно и во веки веком, аминь.


\section{Третий час}
%http://www.molitvoslov.com/text1012.htm 
 


Приходные поклоны 

За молитв святых отец наших, Господи Исусе Христе Сыне Божии, помилуй нас. Аминь 

Царю небесныи. Трисвятое. И по Отче наш. Молитва Исусова. Аминь. Господи помилуй (12). Слава, и ныне. 

Приидите поклонимся Цареви нашему Богу (поклон).

Приидите поклонимся Христу, Цареви и Богу нашему (поклон).

Приидите поклонимся и припадем к Самому Господу Исусу Христу, Цареви и Богу нашему (поклон).


\mysubsubsection{Псалом 16}


Услыши Господи правду мою, вонми моление мое. Внуши молитву мою, не во устнах льстивых. От лица Твоего судьба моя изыдет, очи мои да видите правоты. Искусил еси сердце мое, посетил еси нощию, искусил мя еси, и не обретеся во мне неправда. Яко да не возглаголют уста моя дел человеческих за словеса устен Твоих. Аз сохраних пути жестоки. Соверши стопы моя во стезях Твоих, да не подвижутся стопы моя. Аз возвах, яко услыша мя Боже, приклони ухо Твое ко мне, и услыши глаголы моя. Удиви милости Твоя, спасаяи уповающих на Тя. От противящихся деснице Твоей, сохрани мя Господи, яко зеницу ока. В крове крилу Твоею покрыеши мя, от лица нечестивых острастьших мя. Врази мои душу мою удержаша. Тук свой затвориша, уста их глаголаша гордыню. Изгонящии мя ныне обыдоша мя, очи свои возложиша уклонити на землю. Объяша мя яко лев готов на лов, и яко скимен обитая в сокровищих. Воскресни Господи, предвари я, и запни им; изми душу мою от нечестивых оружия Твоя, от рук враг Твоих. Господи, отмаль от земли раздели я в животе их, и сокровенных Твоих исполнися утроба их. Насытишася сынов, и оставиша останки младенцем своим. Аз же праводю явлюся лицу Твоему, насыщуся внегда явитмися слава Твоя.


\mysubsubsection{Псалом 24}


К Тебе Господи, воздвигох душу мою. Боже мой, на Тя уповах, да не постыжуся во веки. Ни посмеютмися врази мои, ибо вси терпящии Тя не постыдятся. Да постыдятся беззаконнующии всуе. Пути Твоя Господи, скажи ми, и стезям Твоим научи мя. Настави мя на истину Твою, и научи мя, яко Ты еси Бог Спас мой, и Тебе терпех весь день. Помяни щедроты Твоя Господи, и милости Твоя, яко от века суть. Грех юности моея и невежествия моего не помяни. По милости Твоей помяни мя Ты, ради благости Твоея, Господи. Благ и прав Господь, сего ради закон положит согрешающим на пути. Наставит кроткия на суд, и научит кроткия путем своим. Вси путие Господни, милость и истина, взыскающим завета Его и свидения Его. Ради имене Твоего Господи, очисти грех мой, мног бо есть. Кто есть человек бояися Господа? Закон положит ему на пути, иже изволи. Душа его во благих водворится, и семя его наследит землю. Держава Господь боящимся Его, и завет Его явит им. Очи мои выну ко Господу, яко Той исторгнет от сети нозе мои. Призри на мя и помилуй мя, яко единород и нищь есмь аз. Скорби сердца моего умножишася, и от бед моих изведи мя. Виждь смирение мое, и труд мой, и отпусти вся грехи моя. Виждь враги моя, яко умножишася, и ненавидением неправедным возненавидеша мя. Сохрани душу мою, и избави мя, да не постыжуся яко уповах на Тя. Незлобивии и правии прилепляхуся мне, яко потерпех Тя, Господи. Избави Боже, Израиля от всех скорбей его. 


\mysubsubsection{Псалом 50}


Помилуй мя Боже, по велицей милости Твоей. И по множеству щедрот Твоих очисти беззаконие мое. Наипаче омый мя от беззакония моего и от греха моего очисти мя. Яко беззаконие мое аз знаю и грех мой предо мною есть выну. Тебе единому согреших и лукавое пред Тобою сотворих. Яко да оправдишися в словесех Своих и победиши внегда судити. Себо в беззакониих зачат есмь и во гресех роди мя мати моя. Се бо истину возлюбил еси, безвестная и тайная премудрости Твоея явил ми еси. Окропиши мя иссопом и очищуся. Омыеши мя и паче снега убелюся. Слуху моему даси радость и веселие: возрадуются кости смиренныя. Отврати лице Твое от грех моих и вся беззакония моя очисти. Сердце чисто созижди во мне Боже, и дух прав обнови во утробе моей. Не отверзи мене от лица Твоего, и Духа Твоего Святаго не отыми от мене. Воздаждь ми радость спасения Твоего и духом владычным утверди мя. Научу беззаконныя путем Твоим и нечестивии к Тебе обратятся. Избави мя от кровий Боже, Боже спасения моего; возрадуется язык мой правде Твоей. Господи, устне мои отверзеши, и уста моя возвестят хвалу Твою. Яко аще бы восхотел жертвы, дал бых убо; всесожжения не благоволиши. Жертва Богу "--- дух сокрушен: сердце сокрушенно и смиренно Бог не уничижит. Ублажи Господи, благоволением Твоим Сиона; и да созиждутся стены Иеросалимския. Тогда благоволиши жертву правде, возношение и всесожигаемая. Тогда возложат на олтарь Твой тельца.

Слава, ныне. Аллилуия (трижды). Господи помилуй (трижды): Слава, тропарь дню, или святому. И ныне: Богородице, Ты еси лоза истинная…


\itshape Аще ли есть пост, глаголи сие, на глас 6, трижды:\normalfont{} Господи Иже Пресвятыи Свой Дух, в третии час апостолом Своим пославыи, того Благии не отими от нас, но обнови нас молящихтися.

Стих: Сердце чисто созижди во мне Боже, и Дух прав обнови во утробе моей.


\itshape Стих:\normalfont{} Не отверзи мене от лица Твоего, и Духа Твоего Святаго не отыми от мене. И по коемждо стисе: Господи, Иже Пресвятый Свой Дух… весь до конца. И поклоны три. Посем, Слава, и ныне. 

Богородице, Ты еси лоза истинная, возрастившая нам плод животный, Тебе молимся молися Владычице, со святыми апостолы, помиловатися душам нашим.


\itshape Аще ли велкий пост, бывает чтение в лествице.\normalfont{} По скончании же чтения, псаломщик глаголет: аминь. Господь Бог благословен, благословен Господь день дне. Поспешит нам Бог спасении наших, Бог наш, Бог спасати. Трисвятое. И по Отче наш, кондак дню, или святому, или празднику.


\itshape Аще ли же пост, глаголем тропари сия, глас 8.\normalfont{} Благословен еси, Христе Боже наш, Иже премудры ловцы явлеи, низпослав им Дух Святыи, и теми уловлеи вселенную. Человеколюбче, слава Тебе.

Слава. Егда сшед языки размеси, разделяше племена Вышнии. Егда же огненыя языки раздая, в совокупление вся призва, единогласно славити Всесвятыи Дух.

И ныне. Надеже и заступнице, и прибежище християном, и необоримая стено, изнемогающим и труждающимся в пучине, пристанище тихое Ты еси, Пречистая Богородице. Но яко мир избавляеши непрестанною Си молитвою, помяни нас, Дево Всепетая. Господи помилуй (40).


\itshape Посем молитва святаго Великаго Василия:\normalfont{} Иже на всяко время, и на всяк час, на небеси и на земли, покланяемыи и славимыи, Боже Благии, долготерпеливе и многомилостиве; Иже праведники любя и грешных милуя; Иже всех зовыи ко спасению, обещания ради будущих благ; сам Господи, приими молитвы наша в час сий, и в благости исправи живот наш, к заповедем Твоим. Душа наша освяти, и телеса очисти, помышления исправи, и мысли очисти, разум уцеломудри и истрезви, и избави нас от всякия скорби, зол и болезней, и душевных страстей, и огради нас святыми ангелы Твоими, яко да ополчением их соблюдаеми и наставляеми, достигнем во единство веры, и в разум неприкосновенныя Ти славы, яко благословен еси во веки, аминь. 

Таже, Господи помилуй (трижды). Слава, и ныне. Честнейшую херувим, и славнейшую воистину серафим, без истления Бога Слова рождьшую, сущую Богородицу Тя величаем (поклон). Именем Господним благослови, отче. За молитв святых отец наших, Господи Исусе Христе, Сыне Божии, помилуй нас. Аминь.


Аще есть пост, или кроме поста, егда поем Аллилуия, и творим три поклоны великия, глаголюще молитву святаго Ефрема: Господи и Владыко животу моему…. И прочих поклонов, 12.


\itshape И паки:\normalfont{}Господи и Владыко животу моему…, с поклоном. 

И по сем молитва святаго Великаго Василия. Владыко Боже, Отче Вседержителю, и Господи Сыне Единородныи Исусе Христе, и Святыи Душе. Едино Божество и едина сила, помилуй мя грешнаго, и ими же веси судьбами, спаси мя недостойнаго раба Твоего, яко благословен еси во веки, аминь.


\section{Первый час}
%http://www.molitvoslov.com/text1010.htm 
 


\itshape Начало Перваго часа 


Глаголет псаломщик:\normalfont{} Аминь.

Приидите, поклонимся (трижды, и поклоны три).


\mysubsubsection{Псалом 5}


Глаголы моя внуши Господи, разумей звание мое. Воньми глас молитвы моея. Царю мой и Боже мой. Яко к Тебе помолюся Господи, заутра услыши глас мой. Заутра предстану Ти, и узриши мя, яко Бог не хотя беззаконие Ты еси. Не приидет к Тебе лукавнуяи, не пребудут же законопреступницы пред очима Твоима. Возненавидел еси вся творящия беззаконие, погубиши вся глаголющия лжу. Мужа крови и льстива гнушается Господь. Аз же множеством милости Твоея, вниду в дом Твой. Поклонюся к церкви святей Твоей, во страсе Твоем. Господи, настави мя правдою Твоею, враг моих ради, исправи пред Собою путь мой. Яко несть во устех их истины, сердце их суетно. Гроб отверст гортань их, языки своими льщаху: суди им Боже. Да отпадут от мыслей своих, по множеству нечестия их изрини я, яко прогневаша Тя, Господи. И возвеселятся вси уповающии на Тя, во веки возрадуются и вселишися в них. И похвалятся о Тебе любящии имя Твое, яко Ты благословиши праведника. Господи, яко оружием благоволения венчал еси нас.


\mysubsubsection{Псалом 89}


Господи, прибежище бысть нам в род и род. Прежде даже горам не быти, и создатися земли и вселенней, от века и до века Ты еси. Не возврати человека во смирение и рекл еси: обратитеся сынове человечестии. Яко тысяща лет пред очима Твоима Господи, яко день вчерашнии, иже мимо иде, и стража нощная. Уничижения их еще будут. Утро яко трава мимо идет, утро процветет и прейдет, на вечер отпадет, ожестеет и исхнет. Яко исчезохом гневом Твоим, и яростию Твоею смутихомся. Положил еси беззакония наша пред Собою, век наш в просвещение лица Твоего. Яко вси дние наши оскудеша, и гневом Твоим исчезохом. Лета наша яко паучина паучахуся, дние лет наших, в них же седмьдесят лет. Аще ли же в силах, осмьдесят лет; и множае их труд и болезнь. Яко прииде кротость на ны, и накажемся. Кто свесть державу гнева Твоего, и от страха Твоего ярость Твою исчести; Десницу Твою тако скажи ми, и окованныя сердцем в мудрости. Обрати Господи, доколе; и умолен буди на рабы Своя. Исполнихомся заутра милости Твоея Господи, и возрадовахомся и возвеселихомся. Во вся дни наша возвеселихомся, задни в няже смирил ны еси; и лета в няже видехом злая. И призри на рабы Своя, и на дела Твоя и настави сыны их. И буди светлость Господа Бога нашего на нас, и дела рук наших исправи на нас, и дело рук наших исправи.


\mysubsubsection{Псалом 100}


Милость и суд воспою Тебе, Господи. Пою и разумею в пути непорочне, когда приидеши ко мне? Прехождах в незлобии сердца моего, посреде дому моего. Не предлагах пред очима моима вещь законопреступну, творящия преступление возненавидех. Не прильпе мне сердце лукаво, уклоняющагося от мене лукаваго не познах. Оклеветающаго тай искреняго своего, сего изгонях. Гордым оком и несытым сердцем, с сим не ядях. Очи мои на верныя земли, посаждати их с собою. Ходяи по пути непорочну, сей ми служаше; не живяше посреде дому моего, творяи гордыню. Глаголяи неправедная, не исправляше пред очима моима. Во утрия избивах вся грешныя земли, потребити от града Господня вся делающия беззаконие. 

Слава, и ныне. Аллилуия (трижды). Господи помилуй (трижды). Слава, тропарь дню или святому. И ныне, богородичен: Что Тя наречем…


\itshape Аще ли же пост, поем на глас 6:\normalfont{} Заутра услыши глас мой, Царю мой и Боже мой (трижды), и поклоны три.

Стих. яко к Тебе помолюся, Господи.

Стих. Глаголы моя внуши Господи, разумей звание мое.

Слава, и ныне, богородичен: Что Тя наречем, О! Обрадованная? Небо, яко восияла еси Солнце правде. Рай, яко возрастила еси Цвет нетления. Деву, яко пребыла еси нетленна, Чистая; Матерь, яко имела еси на руку Сына всех Бога. Того моли спастися душам нашим.


\itshape Аще ли пост, сия два тропари поем по трижды:\normalfont{} Стопы моя направи по словеси Твоему, и да не одолеет ми всяко беззаконие, избави мя от клеветы человеческия, и сохраню заповеди Твоя. Лице Твое просвети на раба Твоего, и научи мя оправданием Твоим.

Да исполнятся уста моя похвалы, яко да воспою славу Твою, весь день велелепоту Твою. 

Трисвятое. И по Отче наш, кондак святому, или дневи. Господи помилуй (40).


\itshape Таже, молитва Великаго Василия:\normalfont{} Иже на всяко время, и на всяк час, на небеси и на земли, покланяемыи и славимыи, Боже Благии, долготерпеливе и многомилостиве; Иже праведники любя и грешных милуя; Иже всех зовыи ко спасению, обещания ради будущих благ; сам Господи, приими молитвы наша в час сий, и в благости исправи живот наш, к заповедем Твоим. Душа наша освяти, и телеса очисти, помышления исправи, и мысли очисти, разум уцеломудри и истрезви, и избави нас от всякия скорби, зол и болезней, и душевных страстей, и огради нас святыми ангелы Твоими, яко да ополчением их соблюдаеми и наставляеми, достигнем во единство веры, и в разум неприкосновенныя Ти славы, яко благословен еси во веки, аминь. 

Таже, Господи помилуй (трижды). Слава, и ныне. Честнейшую херувим, и славнейшую воистину серафим, без истления Бога Слова рождьшую, сущую Богородицу Тя величаем (поклон). Именем Господним благослови, отче. За молитв святых отец наших, Господи Исусе Христе, Сыне Божии, помилуй нас. Аминь.


\itshape Иерей глаголет:\normalfont{} За молитв святых отец наших… И мы глаголем: Аминь. Таже молитва: Христе, Свете истинный… И конечный отпуст.


\itshape Аще ли пост, по Отче наш, глаголются тропари сия, глас 4.\normalfont{} Скоро предвари, даже не поработимся, врази бо хулят Тя, и претят нам, Христе Боже. Но порази Крестом Своим борющихся с нами, да разумеют, колико может православных вера, молитвами Богородицы, едине Человеколюбче.

Слава. Премудрости наставниче, и смыслу давче, немудрым наказателю, и нищим защитителю, утверди и вразуми сердце мое, Владыко. Ты даждь ми слово, еже отчее единородное Слово: се бо устнама моима не возбраню, еже звати Ти: Милостиве, помилуй мя падшаго.

И ныне. Преславную Христову Матерь, и святых ангел святейшу, немолчно воспоем сердцем и устнами, Богородицу Тя исповедающе, яко воистину рождьшую нам Бога воплощена, и молящуюся непрестанно о душах наших. 

Господи помилуй (40). Иже на всяко время… Господи помилуй (трижды). Слава, и ныне. Честнейшую херувим… (поклон великий). Именем Господним благослови, отче. За молитв святых отец наших… Аминь. И творим великия поклоны, с молитвою святаго Ефрема. И прочих 12 метаний. И паки вышеписаную молитву: Господи и Владыко животу моему… всю до конца. И поклон един. Посем, Трисвятое. И по Отче наш, Господи помилуй (12).


\itshape И молитва:\normalfont{} Христе Свете истинныи, иже просвещая и освещая всякаго человека грядущаго в мир, да знаменается на нас свет лица Твоего Господи, яко да в нем ходяще, узрим свет неприкосновенныя Ти славы, исправи стопы наша, к деланию заповедей Твоих, молитвами Пречистыя Ти Матере, и всех святых Твоих, яко благословен еси во веки, аминь. Таже, обычный отпуст дня того, и настоящаго святаго.


\section{Чин обедницы}
%http://www.molitvoslov.com/text1015.htm 
 


\itshape По начале Божественныя службы, и по первом возгласе, псаломщик глаголет псалом:\normalfont{}


Благослови душе моя Господа. 

\itshape Егда же во уставе укажет на литургии антифоны празднику, и тогда часы глаголются вси вкупе.\normalfont{}


\mysubsubsection{Псалом 102}


Благослови, душе моя, Господа, и вся внутреняя моя имя святое Его. Благослови, душе моя, Господа, и не забывай всех воздаянии Его. Очищающаго вся беззакония твоя, исцеляющаго вся недуги твоя. Избавляющаго от истления живот твой, венчающаго тя милостию и щедротами. Исполняющаго во благих желание твое, обновится яко орлу юность твоя. Творяи милостыню Господь, и судьбу всем обидимым. Сказа пути Своя Моисеови, сыновом израилевым хотения Своя. Щедр и милостив Господь, долготерпелив, и многомилостив. Не до конца прогневается, ни в век враждует. Не по беззаконием нашим сотворил есть нам, ни по грехом нашим воздал есть нам. Яко по высоте небесней от земли, утвердил есть Господь милость Свою на боящихся Его. Елико отстоят востоцы от запад, удалил есть от нас беззакония наша. Яко же щедрит отец сыны, ущедрит Господь боящихся Его. Яко Той позна создание наше, помяну яко персть есмы. Человек, яко трава, дние его, яко цвет сельныи, тако оцветет. Яко дух пройдет в нем и не будет, и не познает к тому места своего. Милость же Господня от века и до века, на боящихся Его. И правда Его на сынех сынов, хранящих завет Его, и помнящих заповеди Его творити я. Господь на небеси уготова престол Свой, и царство Его всеми обладает. Благословите Господа вси ангели Его, сильнии крепостию творящии слово Его, услышати глас словес Его. Благословите Господа вся силы Его, слуги Его, творящии волю Его. Благословите Господа вся дела Его, на всяком месте владычества Его, благослови, душе моя Господа.


\mysubsubsection{\textit{Аще ли Литургия, глаголем по 2-м возгласе,} псалом 145}


Хвали, душе моя, Господа, восхвалю Господа в животе моем, пою Богу моему дондеже есмь. Не надейтеся на князи на сыны человеческия, в них же несть спасения. Изыдет дух его, и возвратится в землю свою, в той день погибнут вся помышления его. Блажен ему же Бог Ияковль помощник его, упование его на Господа Бога своего. Сотворшаго небо и землю, море и вся яже в них. Хранящаго истину в век, творящаго суд обидимым, дающаго пищу алчущим. Господь решит окованных, Господь умудряет слепца, Господь возводит низверженныя, Господь любит праведники. Господь хранит пришельца, сира и вдову приемлет, и путь грешных погубит. Воцарится Господь во веки. Бог Твой Сионе в род и род. 

Слава, и ныне.


\itshape Клирицы поют:\normalfont{} Единородный Сыне, и Слово Божие, безсмертне сыи, изволивыи спасения нашего ради воплотитися от святыя Богородицы и присно Девы Марии непреложно вочеловечивыися; распятся, Христе Боже, и смертию на смерть наступивыи, Едине сыи Святыя Троицы, спрославляемыи Отцу и Святому Духу, спаси нас.


\itshape По скончании пения глаголет канархист блаженны, по уставу.\normalfont{}


Во царствии Си егда приидеши, помяни нас, Господи. Блажени нищии духом, яко тех есть царство небесное. Блажени плачущии, яко тии утешатся. Блажени кротцыи, яко тии наследят землю. Блажени алчущии и жаждущии правды, яко тии насытятся. Блажени милостивии, яко тии помиловани будут. Блажени чистии сердцем, яко тии Бога узрят. Блажени миротворцы, яко тии сынове Божии нарекутся. Блажени изгнани правды ради, яко тех есть царство небесное. Блажени есте егда поносят вам, и изженут вы, и рекут всяк зол глагол на вы лжуще Мене ради. Радуйтеся и веселитеся, яко мзда ваша многа на небесех. Слава, и ныне.


\itshape Аще ли кроме Божественныя литургии, посем чтем:\normalfont{} Апостол, и Евангелие.


\itshape Таже, глаголем:\normalfont{} Помяни нас Господи, егда приидеши во царствии Си (поклон).

Помяни нас Владыко, егда приидеши во царствии Си (поклон).

Помяни нас Святыи, егда приидеши во царствии Си (поклон).


\itshape Лик небесныи поет Тя и глаголет:\normalfont{} свят, свят, свят Господь Саваоф, исполнь небо и землю славы Твоея. Приступите к Нему и просветитеся, и лица ваша не постыдятся. Лик небесныи поет Тя и глаголет: свят, свят, свят Господь Саваоф, исполнь небо и землю славы Твоея. 

Слава. Собор святых ангел и архангел, со всеми небесными силами поет Тя и глаголет: свят, свят, свят Господь Саваоф, исполнь небо и землю славы Твоея. 

И ныне. Исповедание православныя веры, Перваго собора.

Верую во единаго Бога Отца, Вседержителя, Творца небу и земли, видимым же всем и невидимым.

И во единаго Господа Исуса Христа Сына Божия, Единороднаго, иже от Отца рожденнаго прежде всех век. Света от Света, Бога истинна от Бога истинна, рождена, а не сотворена, единосущна Отцу, Им же вся быша.

Нас ради человек, и нашего ради спасения сшедшаго с небес и воплотившагося от Духа Свята и Марии Девы вочеловечшася.

Распятаго за ны при понтийстем Пилате, страдавша и погребена.

И воскресшаго в третии день по писаниих.

И возшедшаго на небеса, и седяща одесную Отца.

И паки грядушаго со славою судити живым и мертвым, Его же царствию несть конца.

Втораго Собора:

И в Духа Святаго, Господа истиннаго и Животворящаго, иже от Отца исходящаго, иже со Отцем и Сыном спокланяема и сславима, глаголавшаго пророки.

И во едину святую соборную и апостольскую Церковь.

Исповедую едино Крещение во оставление грехов.

Чаю воскресение мертвым.

И жизни будущаго века. Аминь.


\itshape В литургии сие не глаголем:\normalfont{} Ослаби, остави, отпусти Боже, согрешения наша, вольная и невольная, яже в слове и в деле, и яже в ведении и не в ведении, яже во уме и в помышлении, яже во дни в нощи, вся ми прости, яко Благ и Человеколюбец.


\itshape Таже, Отче наш. И по возгласе или молитве Исусовой глаголются кондаки, преже храму, и потом дню: таже, святому.\normalfont{} 

Слава. Со святыми покой, Христе, душа раб Своих, идеже несть болезни, ни печали, ни воздыхания, но жизнь вечная.

И ныне, богородичен: Заступнице християном непостыдная, Ходатаице ко Творцу непреложная, не презри грешных моления гласы, но предвари яко Блага на помощь нашу, верно вопиющих Ти: ускори на молитву, и потщися на умоление, заступающи присно Богродице чтущих Тя. 

Господи помилуй (12). Таже Всесвятая Троице.


\itshape Аще ли пост великии, глаголется:\normalfont{} Господи помилуй (40). Слава, и ныне. Честнейшую херувим (поклон великий). Именем Господним благослови, отче. За молитв святых отец наших… Аминь. И сотворяем три великия поклоны с молитвою: Господи и Владыко животу моему. И прочих метаний (12). Посем, Трисвятое.


\itshape И по Отче наш:\normalfont{} Господи помилуй (12).


\itshape Таже, сию молитву:\normalfont{} Всесвятая Троице единосущная, неодержимая держава, и неразделимое царство, яже всем благим виновна, благоволи в настоящии сей час о мне грешнем, и вся ми омый скверны, и просвети ми смысл, яко да всегда воспеваю Тя, и славословлю и глаголю: един Свят, един Господь Исус Христос, в славу Богу Отцу, аминь. 

Буди имя Господне благословено от ныне и до века (трижды с поклонами). 

Слава, и ныне.


\mysubsubsection{\textit{Таже,} псалом 33}


Благословлю Господа на всяко время, выну хвала Его во устех моих. О Господе похвалится душа моя, да услышат кротцыи и возвеселятся. Возвеличите Господа со мною, и вознесем имя Его вкупе. Взысках Господа и услыша мя, и от всех скорбий моих избави мя. Приступите к Нему и просветитеся и лица ваша не постыдятся. Се нищии возва, и Господь услыша и, и от всех скорбий его спасе и. Ополчится ангел Господень окрест боящихся Его, и избавит их. Вкусите и видите, яко благ Господь, блажен муж, иже уповает Нань. Бойтеся Господа вси святии Его, яко несть лишения боящимся Его. Богатии обнищаша и взалкаша, взыскающии же Господа не лишатся от всякаго блага. Приидите чада послушайте мене, страху Господню научу вас. Кто есть человек хотяи живот, любяи дни видети благи; Удержи язык свой от зла, и устне свои еже не глаголати льсти. Уклонися от зла, и сотвори благо, взыщи мир, и пожени и. Очи Господни на праведныя, и уши Его в молитву их. Лице же Господне на творящия злая, еже потребити от земли память их. Возваша праведнии, и Господь услыша их, и от всех печалей их избави их. Близ Господь сокрушенных сердцем, и смиренныя духом спасет. Многи скорби праведным, и от всех их избавит я Господь. Хранит Господь вся кости их, и не едина от них сокрушится. Смерть грешником люта, и ненавидящии праведнаго прегрешат. Избавит Господь душа раб Своих, и не прегрешат вси уповающии Нань. 

Достойно есть, яко воистину блажити Тя, Богородице, присноблаженную и пренепорочную, и Матерь Бога нашего. Честнейшую херувим, и славнейшую воистину серафим, без истления Бога Слова рождьшую, сущую Богородицу, Тя величаем. И поклон до земли вси равно. Слава, и ныне. Господи помилуй (дважды). Господи благослови. И три поклона. 


\itshape Таже, отпуст почину. По отпусте же клирицы поют:\normalfont{} Аминь. Многолетны соблюди Господи, и помилуй державу Российскую, спаси Господи, и помилуй господина нашего Преосвященнейшего Митрополита (имя рек) и отец наших духовных, и вся христианы Господи, спаси. 

Господи помилуй (трижды).


\itshape И обычные исходные поклоны.\normalfont{}


\section{Утреня}
%http://www.molitvoslov.com/text1009.htm 
 


\itshape По обычнем начале глаголем:\normalfont{}


За молитв святых отец наших, Господи Исусе Христе, Сыне Божии, помилуй нас. Аминь. Трисвятое. И по Отче наш. Молитва Исусова. Аминь. Господи помилуй, (12). Слава и ныне. Приидите, поклонимся (трижды с поклонами).


\mysubsubsection{Псалом 19}


Услышит тя Господь в день печали, защитит тя имя Бога Ияковля. Послет ти помощь от Святаго, и от Сиона заступит тя. Помянет всяку жертву твою, и вся сожжения твоя тучна буди. Даст ти Господь по сердцу твоему, и весь совет твой исполнит. Возрадуемся о спасении твоем, и во имя Господа Бога нашего возвеличимся. Исполнит Господь вся прошения твоя, ныне познах, яко спасл есть Господь христа Своего. Услышит его с Небесе святаго Своего, в силах спасение десница Его. Сии на колесницах и сии на конех, мы же во имя Господа Бога нашего призовем. Тии спяти быша и падоша, мы же востахом и исправихомся. Господи, спаси царя и услыши ны, вонь же день аще призовем Тя. 


\mysubsubsection{Псалом 20}


Господи, силою Твоею возвеселится царь, и о спасении Твоем возрадуется зело. Желание сердца его дал еси ему, и хотения устну его неси лишил его. Яко предварил еси его благословением благостынным, положил еси на главе его венец от камене честна. Живота просил есть у Тебе, и дал еси ему долготу дний, в век века. Велия слава его спасением Твоим, славу и велелепие возложиши нань. Яко даси ему благословение в век века, возвеселиши его радостию с лицем Твоим. Яко царь уповает на Господа, и милостию Вышняго не подвижится. Обрящется рука Твоя всем врагом Твоим, десница Твоя обрящет вся ненавидящия Тебе. Яко положиши их яко пещь огненую во время лица Твоего, Господь гневом Своим смутит я, и снесть их огнь. Плод их от земли погубиши, и семя их от сынов человеческих. Яко уклониша на Тя злая, помыслиша советы, их же не возмогоша составити. Яко положиши я хребет во избытцех Своих, уготоваеши лице их. Вознесися, Господи, силою Твоею, воспоем и поем силы Твоя. 

Слава, и ныне. Трисвятое. И по Отче наш. Тропари сия, глас 1. Спаси Господи, люди Своя и благослови достояние Свое, победы державе Российстей на сопротивныя даруй, и Своя сохраняя Крестом люди.

Слава. Вознесыися на Крест волею, тезоименитому ныне граду Твоему, щедроты Твоя даруй, Христе Боже. Возвесели силою Своею державу Российскую, победы дая ей на сопостаты, пособие имущу Твое оружие, миру непобедимую победу.

И ныне. Предстательнице страшная и непостыдная, не презри Благая молитв наших. Всепетая Богородице, утверди православных житие и спаси державу Российскую, подай же ей небесную победу Богом, Его же родила еси, едина Благословенная. Господи помилуй (12). Слава Отцу и Сыну и Святому Духу, и ныне, и присно, и во веки веком. Аминь.

Слава в вышних Богу, и на земли мир, в человецех благоволение (трижды без поклонов).


\itshape Таже:\normalfont{} Господи, устне мои отверзеши и уста моя возвестят хвалу Твою (дважды). 

И глаголет екса псалмы, легко, с тихостию и со всяким вниманием и страхом Божиим, яко к самому Богу беседуя невидимо, и о своих гресех умоляя Его. Братия же стоят руце имуше согбени к персем, главы же мало преклонены.


\mysubsubsection{Псалом 3}


Господи, чтося умножиша стужающии ми, мнози востают на мя. Мнози глаголют души моей: несть спасения ему о Бозе его. Ты же Господи, заступник мой еси, слава моя, и вознося главу мою. Гласом моим ко Господу возвах, и услыша мя от горы святыя Своея, Аз уснух и спах; востах, яко Господь заступит мя. Не убоюся от тем людей, окрест нападающих на мя. Воскресни Господи, спаси мя, Боже мой. Яко Ты порази вся враждующия ми всуе, зубы грешником сокрушил еси. Господне есть спасение, и на людех Твоих благословение Твое. Аз уснух и спах; востах, яко Господь заступит мя.


\mysubsubsection{Псалом 37}


Господи, не яростию Твоею обличи мене, ни гневом Твоим покажи мене. Яко стрелы Твоя унзоша ми, и утвердил еси на мне руку Твою. Несть исцеления в плоти моей, от лица гнева Твоего; несть мира в костех моих, от лица грех моих. Яко беззакония моя превзыдоша главу мою, яко бремя тяжко отяготеша на мне, возсмердеша и согнишася раны моя, от лица безумия моего. Пострадах и смирихся до конца, весь день сетуя хождах. Яко лядвия моя наполнишася поругании, и несть исцеления в плоти моей. Озлоблен бых и смирихся до зела, рыках от воздыхания сердца моего. Господи, пред Тобою все желание мое, и воздыхание мое от Тебе не утаися. Сердце мое смутися, остави мя сила моя, и свет очию моею, и той несть со мною. Друзи мои, и искрении мои, прямо мне приближишася, и сташа. И ближнии мои отдалече мене сташа, и нуждахуся, ищущии душу мою, и ищущии злая мне: глаголаху суетная и льстивная, весь день поучахуся. Аз же яко глух: не слышах; и яко нем: не отверзая уст своих. И бых яко человек не слыша, и не имыи во устех своих обличения. Яко на Тя Господи, уповах, Ты услышиши, Господи Боже мой. Яко рех: да некогда порадуютмися врази мои, и внегда подвижатися ногам моим, на мя велеречеваша. Яко аз на раны готов, и болезнь моя предо мною есть выну. Яко беззаконие мое аз возвещу, и попекуся о гресе моем. Врази же мои живут, и укрепишася паче мене, и умножишася ненавидящии мя без правды. Воздающии ми зла возблагая, оболгаху мя: зане гонях благостыню. Не остави мене Господи Боже мой, не отступи от мене. Воньми в помощь мою, Господи спасения моего. Не остави мене, Господи Боже мои, не отступи от мене. Воньми в помощь мою, Господи спасения моего.

Не остави мене, Господи Боже мои, не отступи от мене. Воньми в помощь мою, Господи спасения моего.


\mysubsubsection{Псалом 62}


Боже, Боже мой, к Тебе утренюю, возжада Тебе душа моя, коль множицею Тебе плоть моя, в земли пусте и непроходне и безводне. Тако во святем явихся Тебе, видети силу Твою и славу Твою. Яко лучши милость Твоя паче живот, устне мои похвалите Тя. Тако благословлю Тя в животе моем, о имени Твоем воздежу руце мои. Яко от тука и масти исполнися душа моя, и устне радости восхвалят Тя уста моя. аще поминах Тя на постели моей, на утрених поучахся в Тя. Яко бысть Помощник мой, и в крове крилу Твоею возрадуюся. Прильпе душа моя по Тебе, мене же прият десница Твоя. Тии же всуе искаша душу мою, внидут в преисподняя земли. Предадятся в руки оружию, части лисовом будут. Царь же возвеселится о Бозе, похвалится всяк кленыися Им. Яко заградишася уста глаголющих неправду.

На утрених поучахся в Тя. Яко бысть Помощник мой, и в крове крилу Твоею возрадуюся. Прильпе душа моя по Тебе, мене же прият десница Твоя. 

Слава, и ныне. Аллилуия, Аллилуия, слава Тебе, Боже (трижды, без поклонов). Господи помилуй (трижды). Слава, и ныне.


\mysubsubsection{Псалом 87}


Господи Боже спасения моего, во дни возвах и в нощи пред Тобою. Да внидет пред Тя молитва моя, приклони ухо Твое к молению моему. Яко исполнися зол душа моя, и живот мой аду приближися. Привменен бых с низходящими в ров, бых яко человек без помощи в мертвых свободь. Яко язвении спящии во гробех, их же не помянул еси к тому и тии от руки Твоея отриновени быша. Положиша мя в рове преисподнем, в темных и сени смертней. На мне утвердися ярость Твоя, и вся волны Твоя наведе на мя. Удалил еси знаемых моих от мене, положиша мя мерзость себе, предан бых и не исхождах. Очи мои изнемогосте от нищеты. Возвах к Тебе Господи, весь день, воздех к Тебе руце мои. Еда мертвыми твориши чудеса? ли врачеве воскресят и исповедятся Тебе? Еда повесть кто во гробе милость Твою, и истину Твою в погибели? Еда познана будут во тме чудеса Твоя, и правда Твоя в земли забвенней? Аз к Тебе Господи, возвах и утро молитва моя предварит Тя. Вскую Господи, отрееши душу мою? отвращаеши лице Свое от мене. Нищь есмь аз, и в трудех от юности моея, вознес же смирихся и изнемогох. На мне преидоша гневи Твои, устрашения Твоя возмутиша мя. Обыдоша мя яко вода, весь день одержаша мя вкупе. Удалил еси от мене друга и искреняго, и знаемых моих от страстей.

Господи Боже спасения моего, во дни возвах и в нощи пред Тобою. Да внидет пред Тя молитва моя, приклони ухо Твое к молению моему.


\mysubsubsection{Псалом 102}


Благослови, душе моя, Господа, и вся внутреняя моя имя святое Его. Благослови, душе моя, Господа, и не забывай всех воздаянии Его. Очищающаго вся беззакония твоя, исцеляющаго вся недуги твоя. Избавляющаго от истления живот твой, венчающаго тя милостию и щедротами. Исполняющаго во благих желание твое, обновится яко орлу юность твоя. Творяи милостыню Господь, и судьбу всем обидимым. Сказа пути Своя Моисеови, сыновом израилевым хотения Своя. Щедр и милостив Господь, долготерпелив, и многомилостив. Не до конца прогневается, ни в век враждует. Не по беззаконием нашим сотворил есть нам, ни по грехом нашим воздал есть нам. Яко по высоте небесней от земли, утвердил есть Господь милость Свою на боящихся Его. Елико отстоят востоцы от запад, удалил есть от нас беззакония наша. Яко же щедрит отец сыны, ущедрит Господь боящихся Его. Яко Той позна создание наше, помяну яко персть есмы. Человек, яко трава, дние его, яко цвет сельныи, тако оцветет. Яко дух пройдет в нем и не будет, и не познает к тому места своего. Милость же Господня от века и до века, на боящихся Его. И правда Его на сынех сынов, хранящих завет Его, и помнящих заповеди Его творити я. Господь на небеси уготова престол Свой, и царство Его всеми обладает. Благословите Господа вси ангели Его, сильнии крепостию творящии слово Его, услышати глас словес Его. Благословите Господа вся силы Его, слуги Его, творящии волю Его. Благословите Господа вся дела Его, на всяком месте владычества Его, благослови, душе моя Господа.

На всяком месте владычествия Его, благослови душе моя Господа.


\mysubsubsection{Псалом 142}


Господи, услыши молитву мою, внуши моление мое воистине Твоей, услыши мя в правде Твоей. И не вниди в суд с рабом Твоим, яко не оправдится пред Тобою всяк живыи. Яко погна враг душу мою, и смирил есть в земли живот мой. Посадил мя есть в темных, яко мертвыя веку и уны во мне дух мой, во Мне смутися сердце мое. Помянух дни древния, поучихся во всех делех Твоих, и в делех руку Твоею поучахся, воздех к Тебе руце мои, душа моя, яко земля безводная Тебе. Скоро услыши мя Господи, исчезе дух мой. Не отврати лица Твоего от мене, и уподоблюся низходящим в ров. Слышану сотвори мне заутра милость Твою, яко на Тя уповах. Скажи мне Господи, путь, воньже пойду, яко к Тебе взях душу мою. Изми мя от враг моих, Господи, к Тебе прибегох. Научи мя творити волю Твою, яко Ты еси Бог мой. Дух Твой благии наставит мя на землю праву. Имене Твоего ради Господи, живиши мя, правдою Твоею изведеши от печали душу мою. И милостию Твоею потребиши враги моя, и погубиши вся стужающия души моей, яко аз раб Твой есмь.

Услыши мя в правде Твоей Господи, услыши, и не вниди в суд с рабом Твоим. Дух Твой Благии наставит мя на землю праву. 

Слава, и ныне. Аллилуия, Аллилуия, слава Тебе, Боже (трижды, с поклонами). Господи помилуй (12). Слава, и ныне. 

Таже, Бог Господь и явися нам, благословен грядыи во имя Господне.

Стих 1. Исповедайтеся Господеви, яко благ, яко в век милость Его.


\itshape Стих 2.\normalfont{}


Обышедше обыдоша мя, и именем Господним противляхся им.


\itshape Стих 3.\normalfont{}


Не умру, но жив буду, и повем дела Господня.


\itshape Стих 4.\normalfont{}


Камень, Его же небрегоша зиждущии, Сей бысть во главу углу, от Господа бысть Сей, и есть дивна во очию нашею. 

Таже тропарь дню, или святому, или празднику.


\itshape Аще ли пост, поется Аллилуия, на глас октая. Стихи же глаголем:\normalfont{}


\itshape Стих 1.\normalfont{}


От нощи утренюет дух мой к Тебе, Боже.


\itshape Стих 2.\normalfont{}


Правде научитеся живущии на земли.


\itshape Стих 3.\normalfont{}


Господи Боже наш, мир даждь нам, вся бо воздал еси нам.


\itshape Стих 4.\normalfont{}


Тако быхом возлюбленному Твоему.


Таже троичны во октаи, и обычная стихология, и седальны, и прочая по уставу.


\mysubsubsection{\textit{Таже,} псалом 50}


Помилуй мя Боже…


\itshape Посем песни пророческия и каноны. На 9 песни канона поем песнь Пресвятей Богородице (от Луки, зачало 4):\normalfont{} Величит душа моя Господа, и возрадовася дух мой о Бозе Спасе моем (правый лик). И по коемждо стисе припеваем: Честнейшую херувим, и славнейшую воистину серафим, без истления Бога Слова рождьшую, сущую Богородицу Тя величаем. И поклон до земли. Аще ли неделя, или празднество, поклоны творим малыя, токмо последнии поклон земныи.

Яко призре на смирение Рабы Своея, се бо от ныне блажат Мя вси роди (левый лик). Яко сотвори мне величие Сильныи, и свято имя Его, и милость Его в род и род на боящихся Его (правый лик). 

Сотвори державу мышцею Своею, расточи гордыя мысли сердца их (левый лик). Низложи сильныя со престол и вознесе смиренныя, алчущия исполни благ, и богатящияся отпусти тща (правый лик). 

Восприят Израиля отрока Своего, помянути милости, якоже глагола ко отцем нашим, Аврааму и семени его до века (левыи лик).


\itshape Таже, песнь 9-я канонов. По 9-й песни оба лика совокупльшеся поем:\normalfont{} Достойно есть, и творим поклон вси вкупе до земли. Таже, ектения малая. Аще ли неделя; поем на глас октая: Свят Господь Бог наш (трижды), таже светилен.


\mysubsubsection{\textit{Посем}, псалом 148}


Хвалите Господа с небес, хвалите Его в вышних. Хвалите Его вси ангели Его. Хвалите Его вся силы Его. Хвалите Его солнце и луна, хвалите Его вся звезды и свет. Хвалите Его небеса небес и вода яже превыше небес, да восхвалят имя Господне. Яко Той рече, и быша; Той повеле, и создашася. Постави я в век и в век века. Повеление положи, и не мимо идет. Хвалите Господа от земли, змиеве и вся бездны. Огнь, град, снег, голоть, дух бурен, творящая слово Его. Горы и вси холми, древа плодоносна и вси кедри. Зверие и вси скоти, гади и птицы пернаты. Царие земстии и вси людие, князи и вси судии земстии. Юноши и девы, старцы с юнотами, да восхвалят имя Господне. Яко вознесеся имя Того единаго. Исповедание Его на земли и на небеси, и вознесет рог людей Своих. Песнь всем преподобным Его, сыновом израилевом, людем приближающимся Ему.


\mysubsubsection{Псалом 149}


Воспойте Господеви песнь нову, хваления Его в церкви преподобных. Да возвеселится Израиль о сотворшем Его, и сынове Сиони возрадуются о Цари своем. Да восхвалят имя Его в лице, в тимпане и псалтыри да поют Ему. Яко благоволит Господь в людех Своих, и вознесет кроткия во спасение. Восхвалятся преподобнии во славе, и возрадуются на ложах своих. Возношения Божия в гортани их, и мечи обоюду остры в руках их. Сотворити отмщение во языцех, обличение в людех. Связати царя их путы, и славныя их ручными оковы железными. Сотворити в них суд написан, слава си есть всем преподобным Его.


\mysubsubsection{Псалом 150}


Хвалите Бога во святых Его, хвалите Его во утвержении сил Его. Хвалите Его на силах Его, хвалите Его по премногому величествию Его. Хвалите Его во гласе трубнем, хвалите Его во псалтыри и гуслех. Хвалите Его в тимпане и лице, хвалите Его во струнах и органех. Хвалите Его в кимвалех доброгласных, хвалите Его в кимвалех восклицания. Всяко дыхание да хвалит Господа.


\itshape Таже, стихеры хвалитныя, и Слава, и И ныне по уставу. Посем глаголем:\normalfont{} Слава показавшему нам свет, и прочая. Аще ли на хвалитех стихер нет и мы глаголем: Тебе слава подобает, Господи Боже наш, и Тебе славу возсылаем, Отцу и Сыну и Святому Духу, и ныне и присно и во веки веком, аминь. Слава показавшему нам свет. 

Слава в вышних Богу, и на земли мир, в человецех благоволение. Хвалим Тя, благословим Тя (поклон), кланяемтися, славословим Тя (поклон), благодарим Тя великия ради славы Твоея (поклон). Господи Царю небесныи, Боже Отче Вседержителю, и Господи Сыне Единородныи Исусе Христе, и Святыи Душе: Господи Боже, Агньче Божии, Сыне Отечь, вземляи грехи мира, помилуй нас. Вземляи грехи мира, приими молитвы наша. Седяи одесную Отца, помилуй нас. Яко Ты един свят, Ты един Господь Исус Христос, в славу Богу Отцу, аминь. На всяку нощь благословлю Тя, и восхвалю имя Твое во веки, и в век века. Господи, Прибежище бысть нам, в род и род. Аз рех: Господи, помилуи мя и исцели душу мою, яко согреших Тебе. Господи, к Тебе прибегох, научи мя творити волю Твою, яко Ты еси Бог мой, яко от Тебе есть источник живота, во свете Твоем узрим свет. Пробави милость Твою ведущим Тя. Сподоби Господи, в нощь сию без греха сохранитися нам. Благословен еси, Господи Боже отец наших, и хвально и прославлено имя Твое во веки, аминь. Буди Господи, милость Твоя на нас, якоже уповахом на Тя. Благословен еси Господи, научи нас оправданием Твоим. Благословен еси Владыко, вразуми нас оправданием Твоим. Благословен еси Святыи, просвети нас оправданием Твоим. Господи, милость Твоя во веки, и дела руку Твоею не презри. Тебе подобает хвала, Тебе подобает пение. Тебе слава подобает: Отцу и Сыну и Святому Духу, ныне и присно и во веки веком, аминь.


\itshape Посем ектения.\normalfont{} Исполним заутрения. Таже, на стиховне стихеры. Посем: Слава, и ныне, Богородичен.


\itshape Таже:\normalfont{} Благо есть исповедатися Господеви, и пети имени Твоему, Вышнии, возвещати заутра милость Твою, и истину Твою на всяку нощь. Аще есть пост, глаголем дважды. Таже, Трисвятое. И по Отче наш, тропарь дню, или святому по уставу, и ектения и по возгласе глаголет псаломщик: Аминь. Приидите, поклонимся, трижды. И псалмы Перваго часа. И прочее. И отпуст.


\itshape Аще ли пост, то глаголем по Отче наш тропари сии:\normalfont{} В церкви стояще славы Твоея, на небеси стояти мнимся, Богородице дверь небесная, отверзи нам двери милости Твоея. Господи помилуй (40). Господи благослови. Господи Исусе Христе Сыне Божии, помилуй нас. Глаголем: Аминь. Небесныи Царю, державу нашу укрепи, веру утверди, языки укроти, мир умири, и святыи храм сей добре сохрани, и прежде отшедшия отцы и братию нашу в кровех с праведными учини и нас в православней вере и в покаянии, Господи, приими и помилуй, яко Благ и Человеколюбец. Таже, Господи помилуй (трижды). Слава, и ныне. Честнейшую херувим, и славнейшую воистину серафим, без истления Бога Слова рождьшую, сущую Богородицу Тя величаем (поклон великий). Именем Господним благослови, отче.


\itshape И творим великия поклоны, с молитвою святаго Ефрема, якоже предуказася на вседневней полунощнице. 

И абие начинаем первый час.\normalfont{}


\section{Девятый час}
%http://www.molitvoslov.com/text1014.htm 
 


Приидите, поклонимся (трижды, и поклоны три)


\mysubsubsection{Псалом 83}


Коль возлюблена села Твоя, Господи сил, желает и скончевается душа моя во дворы Господня. Сердце мое и плоть моя возрадовастася о Бозе живе. Ибо птица обрете себе храмину, и горлица гнездо себе, идеже положи птенцы своя. Олтари Твоя, Господи Боже сил, Царю мой и Боже мой. Блажени живущии в дому Твоем, в век века восхвалят Тя. Блажен муж ему же есть заступление его от Тебе. Восхождения в сердцы своем положи, во юдоли плачевне, в месте идеже положи. Ибо благословение даст закон даяи. Пойдут от силы в силу, явится Бог богом в Сионе. Господи Боже сил, услыши молитву мою, внуши Боже Ияковль. Защитниче наш, виждь Боже, и призри на лице Христа Твоего. Яко лучше день един во дворех Твоих, паче тысящ. Изволих приметатися в дому Бога моего паче, неже жити ми в селех грешничих. Яко милость и истину любит Господь. Бог благодать и славу даст. Господь не лишит блага, ходящих незлобием. Господи Боже сил, блажен человек уповаяи на Тя.


\mysubsubsection{Псалом 84}


Благоволил еси Господи, землю Твою, возвратил еси плен Ияковль Оставил еси беззакония людем Твоим, покрыл еси вся грехи их. Укротил еси весь гнев Твой, возвратил еси от гнева ярости Твоея. Возврати нас, Боже спасении наших, и возврати ярость Твою от нас. Еда во веки прогневаешися на ны, или простреши гнев свой от рода в род. Боже, Ты обращ живиши ны, и людие Твои возвеселятся о Тебе. Яви нам Господи, милость Твою, и спасение Твое даждь нам. Услышу, что речет о мне Господь Бог; яко речет мир на люди Своя и на преподобныя Своя, и на обращающия сердца к Нему. Обаче близ боящихся Его, спасение Его, вселити славу в землю нашу. Милость и истина сретостеся, правда и мир облобызастася. Истина от земли восия, и правда с небесе приниче. Ибо Господь даст благость, и земля наша даст плод свой. Правда пред Ним предъидет, и положит в путь стопы своя.


\mysubsubsection{Псалом 85}


Приклони Господи, ухо Твое и услыши мя, яко нищ и убог есмь аз. Сохрани душу мою, яко преподобен есмь, спаси раба Твоего, Боже мой, уповающаго на Тя. Помилуй мя Господи, яко к Тебе воззову весь день. Возвесели душу раба Твоего, яко к Тебе взях душу мою. Яко Ты Господи, благ, и долготерпелив, и многомилостив всем призывающим Тя. Внуши Господи, молитву мою, и воньми гласу моления моего. В день печали моея возвах к Тебе, яко услыша мя. Несть подобен Тебе в бозех Господи, и несть по делом Твоим. Вси языцы елико сотвори приидут, и поклонятся пред Тобою Господи, и прославят имя Твое. Яко велии еси Ты, и творяи чудеса, Ты еси Бог един. Настави мя Господи, на путь Твой, и пойду во истине Твоей: да возвеселится сердце мое, боятися имене Твоего. Исповемся Тебе, Господи Боже мой, всем сердцем моим, и прославлю имя Твое в век. Яко милость Твоя велия на мне, и избави душу мою от ада преисподняго. Боже, законопреступницы восташа на мя, и сонм державных взыскаша душу мою, и не предложиша Тебе пред собою. И Ты Господи Боже мой, щедр и милостив, долготерпелив и многомилостив и истинен. Призри на мя и помилуй мя, даждь державу Твою отроку Твоему, и спаси сына рабы Твоея. Сотвори со мною знамение во благо, и да видят ненавидящии мя и постыдятся; яко Ты Господи, поможе ми, и утешил мя еси.

Слава, и ныне. Аллилуия (трижды). Господи помилуй (трижды). Тропарь по уставу.


\itshape Аще ли же пост, глаголем сий тропарь, глас 8:\normalfont{} Иже в девятыи час, нас ради плотию смерть вкусивыи, умертви плоти нашея мудрование, Христе Боже, спаси нас.


\itshape Стих 1:\normalfont{}


Да приближится молитва моя пред Тя Господи, по словеси Твоему вразуми мя.


\itshape Стих 2:\normalfont{}


Да внидет прошение мое пред Тя Господи, по словеси Твоему избави мя.


Слава, и ныне. Иже нас ради рождеися от Девы, и распятие претерпев Благии, испровергии смертию смерть, и воскресение явль яко Бог, не презри их же созда рукою Своею, яви человеколюбие Свое Милостиве, приими рождьшую Тя Богородицу, молящуся заны, и спаси, Спасе наш, люди согрешьшия.


\itshape Аще ли пост, бывает чтение. Таже:\normalfont{} Не предаждь нас до конца, имене Твоего ради, и не разори завета Твоего, и не отстави милости Твоея от нас, Авраама ради возлюбленнаго от Тебе, и за Исаака раба Твоего, и Израиля святаго Твоего. Таже, Трисвятое. И по Отче наш, кондак дню, или святому.


\itshape Аще ли же пост, глаголем тропари сия, глас 8:\normalfont{} Видя разбойник Начальника жизни на Кресте висяща, и глаголаше: аще небы Бог был воплощься, Иже с нами распныися, небы солнце лучи свои потаило, ниже земля трепещущи тряслася, но Иже за вся терпяи, помяни мя Спасе, егда приидеши во царствии Си. 

Слава. Посреди двою разбойнику мерило праведное обретеся Крест Твой: овому убо низходящу во ад тяготою хуления, другому же легчащуся от грех к разуму богословия. Христе Боже, слава Тебе. 

И ныне. Агньца и пастыря и Спаса миру, на Кресте зрящи рождьшая Тя, глаголаше слезящии: мир убо радуется, приемля Тобою избавление, утроба же Моя горит, зрящи Твое распятие, еже за всех терпиши, Сыне и Боже мой. 


\itshape Таже,\normalfont{} Господи помилуй (40). Иже на всяко время… Господи помилуй (трижды). Слава, и ныне. Честнейшую херувим… Именем Господним благослови, отче. За молитв святых отец наших… Аминь. И сотворяем три великия поклоны, со обычною молитвою: Господи и Владыко животу моему… И прочих метаний (12). Посем молитва. Владыко Боже, Отче Вседержителю…


\itshape Аще ли великий пост, и мы по поклонех молитвы:\normalfont{} Владыко Боже, Отче… не глаголем, но поем на оба лика: Во царствии Си…


\itshape Аще ли кроме поста, глаголем молитву сию:\normalfont{} Владыко, Господи Исусе Христе, Боже наш, долготерпеливе о наших согрешениих, даже и до нынешняго часа приведыи нас, воньже на животворящем древе вися, благоразумному разбойнику, Иже в рай путь сотворил еси вход, и смертию смерть разрушил еси, очисти нас грешных, и недостойных раб Твоих. Согрешихом бо и беззаконновахом, и несмы достойни возвести очию нашею, и возрети на высоту небесную, зане же оставихом путь правды Твоея, и ходихом в волях сердец наших, но молим Твою благость: пощади нас Господи, по множеству милости Твоея, и спаси нас имене Твоего ради святаго, яко исчезоша в суете дние наши. Изми нас из руки сопротивнаго, и остави нам прегрешения наша, и умертви плотьское наше мудрование, да ветхаго отложивше человека, и в новаго облецемся, и Тебе поживем, нашему Владыце и Благодателю, и тако Твоим повелением последующе, в вечныи покой достигнем, идеже есть всем веселящимся жилище. Ты бо еси воистину истинное веселие, и радость любящим Тя, Христе Боже наш, и Тебе славу возсылаем, со безначальным Ти Отцем, и с Пресвятым и Благим и Животворящим Ти Духом, ныне и присно и во веки веком, аминь.


\section{Павечерница великая}
%http://www.molitvoslov.com/text1006.htm 
 


Начало Великия павечерницы. 

Певаема бывает в Великии пост, и в пост святых Апостол, и в пост Святыя Богородицы, и в пост Рожества Христова, егда есть Аллилуия, с поклоны.

По обычнем начале глаголем:

За молитв святых отец наших, Господи Исусе Христе, Сыне Божии, помилуй нас.

Аминь.

Царю Небесныи. Трисвятое. И по Отче наш. 

Молитва Исусова. 

Господи помилуй (12).

Слава, и ныне. 

Приидите, поклонимся (трижды).


\mysubsubsection{\textit{Аще ли есть первая неделя Великаго поста, глаголем,} псалом 69:}


Боже, в помощь мою воньми. Господи, помощи ми потщися. Да постыдятся и посрамятся ищущии душу мою. Да возвратятся вспять и постыдятся, мыслящии ми злая. Да возвратятся абие стыдящеся, глаголющии ми: благо же, благо же. Да возрадуются и возвеселятся о Тебе, вси ищущии Тебе Боже, и глаголют выну: да возвеличится Господь, любящии спасение Твое. Аз же нищь есмь и убог, Боже помози ми. Помощник мой и Избавитель мой еси Ты. Господи, не закосни.

Посем канон святаго Андрея Критскаго.

По каноне же глаголем псалом, 4-й Внегда возвах.

Аще ли кроме первыя недели святаго поста по изглаголании: Приидите, поклонимся, глаголем сия псалмы.

Внегда возвах, услыша мя Боже правды моея, в скорби распространил мя еси. Ущедри мя, и услыши молитву мою. Сынове человечестии, доколе тяжкосердии; вскую любите суетная, и ищете лжи; И уведите, яко удиви Господь преподобнаго Своего, Господь услышит мя внегда взову к Нему. Гневайтеся и не согрешайте, яже глаголете в сердцах ваших, на ложах ваших умилитеся. Пожрите жертву правды, и уповайте на Господа, мнози глаголют: кто явит нам благая? 3наменася на нас свет лица Твоего Господи, дал еси веселие в сердцы моем. От плода пшеницы, вина и елея своего умножишася. В мире вкупе усну и почию, яко Ты Господи, единаго на упование вселил мя еси.


\mysubsubsection{Псалом 6}


Господи, да не яростию Твоею обличиши мене, ниже гневом Твоим покажеши мене. Помилуй мя Господи, яко немощен есмь, исцели мя Господи, яко смутишася кости моя. И душа моя смутися зело, и Ты Господи, доколе? Обрати Господи, изми душу мою, спаси мя ради милости Твоея. Яко несть в смерти поминаяи Тебе, во аде же кто исповестьтися? Утрудихся воздыханием моим, измыю на всяку нощь ложе мое, слезами моими постелю мою омочу. Смутися от ярости око мое, обетшах во всех вразех моих. Отступите от мене вси творящии беззаконие, яко услыша Господь глас плача моего. Услыша Господь моление мое, Господь молитву мою прият. Да постыдятся и смятутся вси врази мои, возвратятся и устыдятся зело вскоре.


\mysubsubsection{Псалом 12}


Доколе Господи, забудеши мя до конца? Доколе отвращаеши лице Свое от мене? Доколе положу советы в души моей, болезни в сердцы моем день и нощь? Доколе вознесется враг мой на мя? Призри и услыши мя, Господи Боже мой. Просвети очи мои да некогда усну в смерть, да некогда речет враг мой: укрепихся нань. Стужающии ми возрадуются, аще ся подвижу, аз же на милость Твою уповах. Возрадуется сердце мое о спасении Твоем, воспою Господеви благодеявшему мне, и пою имени Господню Вышнему.

Слава, и ныне. 

Аллилуия, Аллилуия, слава Тебе, Боже (трижды и поклоны три). 

Господи помилуй (трижды) . Слава, и ныне.


\mysubsubsection{Псалом 24}


К Тебе Господи, воздвигох душу мою, Боже мой, на Тя уповах, да не постыжуся во веки. Ни посмеютмися врази мои, ибо вси терпящии Тя не постыдятся. Да постыдятся беззаконнующии всуе. Пути Твоя Господи, скажи ми, и стезям Твоим научи мя. Настави мя на истину Твою, и научи мя, яко Ты еси Бог Спас мой, и Тебе терпех весь день. Помяни щедроты Твоя Господи, и милости Твоя, яко от века суть. Грех юности моея, и невежествия моего не помяни, по милости Твоей помяни мя Ты, ради благости Твоея, Господи. Благ и прав Господь, сего ради закон положит согрешающим на пути. Наставит кроткия на суд, и научит кроткия путем Своим. Вси путие Господни милость и истина, взыскающим завета Его и свидения Его. Ради имене Твоего Господи, и очисти грех мой, мног бо есть. Кто есть человек бояися Господа; закон положит ему на пути иже изволи. Душа его во благих водворится, и семя его наследит землю. Держава Господь боящимся Его, и завет Его явит им. Очи мои выну ко Господу, яко Той исторгнет от сети нозе мои. Призри на мя и помилуй мя, яко единород и нищь есмь аз. Скорби сердца моего умножишася, и от бед моих изведи мя. Виждь смирение мое и труд мой, и отпусти вся грехи моя. Виждь враги моя, яко умножишася, и ненавидением неправедным возненавидеша мя. Сохрани душу мою и избави мя, да не постыжуся, яко уповах на Тя. Незлобивии и правии прилепляхуся мне, яко потерпех Тя, Господи. Избави Боже, Израиля от всех скорбий его.


\mysubsubsection{Псалом 30}


На Тя Господи, уповах, да не постыжуся во веки, правдою Твоею избави мя, и изми мя. Приклони ко мне ухо Твое, ускори изъяти мя. Буди ми в Бог защититель, и в дом прибежища спасти мя. Яко держава моя и прибежище мое еси Ты, и имене Твоего ради наставиши мя, и препитаеши мя. Исторгнеши мя от сети сея, юже скрыша ми, яко Ты еси Защититель мой, Господи. В руце Твои предаю дух мой, избавил мя еси, Господи Боже истине. Возненавидел еси хранящия суетная затщее, аз же на Господа уповах. Возрадуюся и возвеселюся о милости Твоей, яко призрел еси на смирение мое, спасл еси от бед душу мою. И неси мене затворил в руках вражииих, поставил еси на пространне нозе мои. Помилуй мя Господи, яко скорблю, смутися от ярости око мое, душа моя и утроба моя. Яко исчезе в болезни живот мой, и лета моя в воздыханиих, изнеможе нищетою крепость моя, и кости моя смутишася. От всех враг моих бых поношение, и соседом моим зело, и страх знаемым моим. Видящии мя вон бежаша от мене, забвен бых яко мертв от сердца. Бых яко сосуд погублен, яко слышах гаждение многих живущих окрест, внегда собратися им вкупе на мя, прияти душу мою совещаша. Аз же на Тя Господи, уповах, рек: Ты еси Бог мой, в руку Твоею жребии мои. Избави мя из руки враг моих, и от гонящих мя. Просвети лице Твое на раба Твоего, спаси мя милостию Твоею Господи, да не постыжуся, яко призвах Тя. Да постыдятся нечестивии, и снидут во ад. Немы да будут устны льстивыя, глаголющия на праведнаго беззаконие, гордынею и уничижением. Коль много множество благости Твоея Господи, юже скрыл еси боящимся Тебе, соделал еси уповающим на Тя, пред сыны человеческими. Скрыеши их в тайне лица Твоего, от мятежа человеческа, покрыеши их в крове от пререкания язык. Благословен Господь, яко удиви милость Свою во граде обстояния. Аз же рех во изступлении моем: отвержен есмь от лица очию Твоею; сего ради услыша глас молитвы моея, внегда возвах к Тебе. Возлюбите Господа вси преподобнии Его, яко истины взыскает Господь, и воздает излиха творящим гордыню. Мужайтеся и да крепится сердце ваше, вси уповающии на Господа.


\mysubsubsection{Псалом 90}


Живыи в помощи Вышняго, в крове Бога небеснаго водворится. Речет Господеви: Заступник мой еси, и прибежище мое, Бог мой, и уповаю Нань. Яко Той избавит тя от сети ловчи, и от словесе мятежна. Плещьма Своима осенит тя, и под криле Его надеешися. Оружие обыдет тя истина Его, не убоишися от страха нощнаго, от стрелы летящия во дни. От вещи во тме преходящия, от сряща и беса полуденнаго. Падет от страны твоея тысяща, и тма одесную тебе, к тебе же не приближится. Обаче очима своима смотриши, и воздаяние грешииком узриши. Яко Ты Господи, упование  мое; Вышняго положил еси прибежище твое. Не приидет к тебе зло, и рана не приближится телеси твоему. Яко ангелом Своим заповесть о тебе, сохранити тя во всех путех твоих. На руках возмут тя, да некогда преткнеши о камень ноги твоея. На аспида и василиска наступиши, и попереши льва и змия. Яко на Мя упова, и избавлю и, покрыю и, яко позна имя Мое. Воззовет ко Мне, и услышу и, с ним есмь в скорби, изму и, и прославлю его. Долготу дний исполню и, и явлю ему спасение Мое.


Посем: Слава, и ныне. 

Аллилуия, Аллилуия, слава Тебе, Боже (трижды и поклоны три).

Господи помилуй (трижды). Слава, и ныне.

Таже, песнь святаго пророка Исаии.

С нами Бог, разумейте языцы и покаряйтеся, яко с нами Бог.

Услышите и до последних земли, яко с нами Бог. Могущии покаряйтеся, яко с нами Бог. Аще бо паки возможете, и паки побеждени будете, яко с нами Бог. И иже аще совет совещаваете, и разорит и Господь, яко с нами Бог. И слово еже аще возглаголете, не имать пребывати в вас, яко с нами Бог. Страха же вашего не имам убоятися, ниже соблазнитися, яко с нами Бог. Господа же Бога нашего святите, и Той будет вам в боязнь, яко с нами Бог. И иже аще Нань надеешися, и Той будет ти во освящение, яко с нами Бог. И уповающе будем Нань, и спасемся Его ради, яко с нами Бог. Се аз и дети, яже ми дал Бог, яко с нами Бог. Людие ходящии во тме, видеша свет велии, яко с нами Бог. Живущии во стране и сени смертней, свет восияет на вы, яко с нами Бог. Яко Отроча родися нам, Сын и дастся нам, яко с нами Бог. Ему же власть бысть на раме Его, яко с нами Бог. И миру Его несть предела, яко с нами Бог. И нарицается имя Его велика совета Ангел, яко с нами Бог. Чуден Советник, яко с нами Бог. Бог крепок, Владыка, Князь смирения, яко с нами Бог. Отец будущаго века, яко с нами Бог.

Слава: Яко с нами Бог.

И ныне: Яко с нами Бог. 

Таже, сей стих глаголем трижды: С нами Бог, разумейте языцы и покаряйтеся, яко с нами Бог. 

Таже: День пребыв благодарю Тя  Господи, вечер молюся с нощию, безнаветие подаждь ми Спасе, и спаси мя.

Слава. День изшед славословлю Тя Владыко, вечер молюся с нощию, безсоблазньство подаждь ми Спасе, и спаси мя.

И ныне. День прешед песнословя Тя Святыи, вечер молюся с нощию, без греха сохрани мя Спасе, и спаси мя.

Безплотное естество херувимское, немолчными песньми Тя славословят, шестокрыльная и животная серафим, непрестанными гласы Тя превозносят, ангельская же вся воинства, трисвятыми песньми Тя восхваляют. Прежде бо всех еси сыи Отец, и собезначальна имаши Своего Сына, и равночестен нося Дух жизни, Троицу являеши нераздельну.

Пресвятая Дево, Мати Божия, и иже Слову самовидцы и слуги, пророк же и мученик, и вси ликове, яко безсмертну имуще жизнь, о всех молитеся прилежно, яко вси есмы в бедах, да льсти избавльшеся лукаваго, ангельскую вопием песнь: Святыи, Святыи, Святыи, Трисвятыи Господи, помилуй и спаси нас. 

Таже, исповедание православныя веры, перваго Собора:

Верую во единаго Бога Отца, Вседержителя, Творца небу и земли, видимым же всем и невидимым. И во единаго Господа Исуса Христа, Сына Божия, Единороднаго, Иже от Отца рожденнаго прежде всех век. Света от света, Бога истинна от Бога истинна, рожденна, а не сотворена, единосущна Отцу, Им же вся быша. Нас ради человек, и нашего ради спасения, сшедшаго с небес, и воплотившагося от Духа Свята, и Марии Девы вочеловечьшася. Распятаго за ни при понтийстем Пилате, страдавша и погребена, и воскресшаго в третий день по писаниих. И возшедшаго на небеса, и седяща одесную Отца. И паки грядущаго со славою, судити живым и мертвым, Его же царствию несть конца.

Втораго Собора: И в Духа Святаго, Господа истиннаго и животворящаго, Иже от Отца исходящаго, Иже со Отцем и Сыном спокланяема и сславима, глаголавшаго пророки. И во едину святую соборную и апостольскую Церковь. Исповедую едино Крещение, во оставление грехов. Чаю воскресения мертвым. И жизни будущаго века, аминь.

Посем глаголет, начинаяи лик:

О! Пресвятая Госпоже Владычице Богородице Всемилостивая, моли Бога о нас грешных. 

И творим метание на кииждо стих. И другии лик той же стих глаголет. Наченыи же глаголет: 

Вся небесныя силы, святии ангели и архангели, молите Бога о нас грешных.

Святыи великии Иоанне пророче и Предотече, Крестителю Господень, моли Бога о нас грешных.

Святии славнии апостоли, пророцы и мученицы, и вси святии, молите Бога о нас грешных.

Преподобнии и богоноснии отцы наши, пастырие и учителие вселенней, молите Бога о нас грешных.

Зде рцы святаго, его же есть храм.

Непобедимая и божественная сила, честнаго и животворящаго Креста Господня, не остави нас грешных, уповающих на Тя. 

Боже, милостив буди нам грешным. 

Егда же достигнем сего стиха: Боже, очисти грехи наша, и помилуй нас, глаголем его трижды.

Посем, Трисвятое, и поклоны обычныя (3). 

По Отче наш. Молитва Исусова.


\itshape В понедельник и среду вечер, тропари сия глас 2.\normalfont{}

Просвети очи мои Христе Боже, да некогда усну в смерть, да некогда речет враг мой: укрепихся нань.

Стих. Призри и услыши мя, Господи Боже мой. 

Паки той же тропарь: Просвети очи мои Христе Боже, да некогда усну в смерть, да некогда речет враг мой: укрепихся нань.

Слава. Заступник души моей буди Боже, яко посреде хожду сетей многих. Избави мя от них, и спаси мя Блаже, яко Человеколюбец.

И ныне, Богородичен. Яко не имамы дерзновения, за премногия грехи наша, но Ты, Иже от Тебе рождьшагося, моли Богородице Дево. Много бо может молитва Матерня, на умоление Владыки. Не презри грешных мольбы Всечистая, яко милостив есть, и спасти могии, Иже страдати нас ради изволивыи.


\itshape Ины тропари, во вторник и в четверток вечер, глас 8.\normalfont{}

Сия обои тропари пременяти чрез день.

Невидимых враг моих неусыпание свеси Господи, и окаянныя плоти моея неможение веси, Создавыи мя. Тем же в руце Твои предаю дух мой, покрый мя крилома Твоея благости, да некогда усну в смерть. И умнии мои очи просвети, наслаждением божественных словес Твоих, и воздвигни мя во время  благопотребно, к Твоему славословию, яко един Благ и Человеколюбец.

Стих. Призри и услыши мя, Господи Боже мой. 

Яко страшен суд Твой Господи, ангелом предстоящим, человеком вводимым, книгам разгибаемым, делом испытаемым, помыслом истязуемым. Кии суд будет мне, зачатому во гресех? кто ми пламень угасит? кто ми тму просветит? аще не Ты Господи, помилуеши мя Блаже, яко Человеколюбец.

Слава. Слезы ми даждь Боже, якоже древле жене грешнице. И сподоби мя омочити нозе Твои, яже мя от пути льстиваго свободившая и мvро благоухания Тебе приносити, житие чисто, покаянием ми стяжанно. Да услышу и аз благаго Твоего гласа: вера твоя спасе тя, иди в мир.

И ныне, Богородичен. Непостыдную, Богородице, надежду Твою имея спасуся, и заступление Твое стяжав Пречистая, не убоюся, но пожену враги моя, и побежду я, во едино оболкся, яко во броня, в кров Твой, и всемогущую Твою помощь. И моляся вопию Ти: Владычице, спаси мя мольбами Си, и воздвигни мя от мрачнаго сна, ко Твоему славословию, силою из Тебе воплотившагося Христа Бога.

Господи помилуй (40). 

Слава, и ныне.

Честнейшую херувим (поклон един).

Именем Господним благослови, отче.

За молитв святых отец наших. 

Аминь.


\itshape Таже, молитва Великаго Василия.\normalfont{}

Господи, Господи, избави нас от всякия стрелы летящия во дни. И избави нас от всякия вещи во тме приходящия, приими жертву вечернюю рук наших воздеяние. Сподоби же нас и нощное поприще без порока преити, неискусныи от злых, и избави нас от всякаго смущения и страха, иже от диявола нам прибывающаго. Даруй душам нашим умиление, и помыслом нашим попечение, еже на страшнем и праведнем Твоем суде испытания. Пригвозди страсе Твоем плоти наша, и умертви уды наша сущия на земли. Яко да и сонным безмолвием, просветимся зрением судеб Твоих. Отими от нас всяко мечтание неподобно, и похоть вредну. Востави же нас во время молитвы, утвержены в вере, и преспевающих в заповедех Твоих, благоволением и благостию, единороднаго Сына Твоего, с Ним же благословен еси, и с пресвятым и животворящим Ти Духом, ныне и присно и во веки веком, аминь.

Приидите, поклонимся (трижды, и поклоны три).


\mysubsubsection{Псалом 50}


Помилуй мя Боже, по велицей милости Твоей. И по множеству щедрот Твоих очисти беззаконие мое. Наипаче омый мя от беззакония моего и от греха моего очисти мя. Яко беззаконие мое аз знаю и грех мой предо мною есть выну. Тебе единому согреших и лукавое пред Тобою сотворих. Яко да оправдишися в словесех Своих и победиши внегда судити. Се бо в беззакониих зачат есмь и во гресех роди мя мати моя. Се бо истину возлюбил еси, безвестная и тайная премудрости Твоея явил ми еси. Окропиши мя иссопом и очищуся. Омыеши мя и паче снега убелюся. Слуху моему даси радость и веселие: возрадуются кости смиренныя. Отврати лице Твое от грех моих и вся  беззакония моя очисти. Сердце чисто созижди во мне Боже, и дух прав обнови во утробе моей. Не отверзи мене oт лица Твоего, и Духа Твоего Святаго не отыми от мене. Воздаждь ми радость спасения Твоего и Духом Владычным утверди мя. Научу беззаконныя путем Твоим и нечестивии к Тебе обратятся. Избави мя от кровий Боже, Боже спасения моего; возрадуется язык мой правде Твоей. Господи, устне мои отверзеши, и уста моя возвестят хвалу Твою. Яко аще бы восхотел жертвы, дал бых убо; всесожжения не благоволиши. Жертва Богу "--- дух сокрушен: сердце сокрушенно и смиренно Бог не уничижит. Ублажи Господи, благоволением Твоим Сиона; и да созиждутся стены Иеросалимския. Тогда благоволиши жертву правде, возношение и всесожигаемая. Тогда возложат на олтарь Твой тельца.


\mysubsubsection{Псалом 101}


Господи, услыши молитву мою, и вопль мой к Тебе да приидет. Не отврати лица Твоего от мене, вонь же день аще скорблю, приклони ко мне ухо Твое, вонь же день, аще призову Тя, скоро услыши мя. Яко исчезоша, яко дым дние мои, и кости моя, яко сушило сосхошася. Уязвен бых, яко трава, и изсше сердце мое, яко забых снести хлеб мой. От гласа воздыхания моего, прильпе кость моя плоти моей. Уподобихся неясыти пустынному, бых яко нощныи вран на нырищи. Забдех, и бых яко птица особящаяся назде. Весь день поношаху ми врази мои, и хвалящии мя мною кленяхуся. Зане пепел, яко хлеб ядях, и питие мое с плачем растворях. От лица гнева Твоего, и ярости Твоея, яко вознес низверже мя. Дние мои, яко сень уклонишася, и аз яко сено изсхох. Ты же Господи, во веки пребываеши, и память Твоя в род и род. Ты воскрес ущедриши Сиона, яко время ушедрити его, яко прииде время. Яко благоволиша раби Твои камение его, и персть его ущедрят. И убоятся языцы имене Господня, и вси царие земстии славы Твоея. Яко созиждет Господь Сиона, и явится в славе Своей. Призре на молитву убогих, и не уничижи моления их. Да напишутся сии в род их, и людие зиждемии восхвалят Господа. Яко призре с высоты святыя Своея, Господь с небесе на землю призре. Услышати воздыхание окованных, разрешити сыны умерщвенных. Возвестити в Сионе имя Господне, и хвалу Его во Иеросалиме. Внегда собратися людем вкупе, и царие работати Господеви. Отвеща ему на пути крепости его, умаление днии моих возвести ми. Не возведи мене в преполовение днии моих, в род и род лета Твоя. В началех Ты Господи, землю основа, и дела рук Твоих суть небеса. Та погибнут, Ты же пребываеши, и вся яко риза обетшают, и яко одежду свиеши их, и изменятся. Ты же тожде еси, и лета Твоя не оскудеют. Сынове раб Твоих вселятся, и семя их в век исправится


\itshape Таже молитва Манасии, царя иудейска:\normalfont{} Господи Вседержителю, Боже отец наших, Авраамов и Исааков и Ияковль, и семене их праведнаго. Сотворивыи небо и землю со всею лепотою их, и сопныи море словом повеления Твоего. Затворивыи бездну, и запечатлев ю, страшным и славным именем Твоим. Его же вся боятся, и трепещут от лица славы Твоея. Яко непостоянна велелепота славы Твоея, и не стерпим гнев еже на грешники прещения Твоего. Безчислена же и неизследованна милость обещания Твоего. Ты бо еси Господь вышнии, милосерд, долготерпелив и многомилостив, и каяся о злобах человеческих. Но Ты Господи, по множеству благости Твоея, обеща покаяние и оставление согрешшим к Тебе, и множеством щедрот Твоих нарече покаяние грешником во спасение. Ты убо Господи Боже праведных, неси положил покаяние праведным Твоим, Аврааму и Исааку и Иякову, несогрешившим пред Тобою, но положил еси покаяние мне грешному, зане согреших Ти паче числа песка морскаго. Умножишася беззакония моя Господи, умножишася, и несмь достоин воззрети и видети высоту небесную, от множества неправд моих связан есмь многими юзами железными. Яко не возвести ми главы моея, и несть ми восклонения. Зане прогневах ярость Твою, и лукавое пред Тобою сотворих. И не сотворих воли Твоея, ни сохраних повелении Твоих. И ныне поклоняю колена сердца моего, и молю яже от Тебе благость. Согреших Господи, согреших, и беззакония моя аз свем. Но прошу и молюся Тебе: отради ми Господи, отради ми, и не погуби мене со беззаконьми моими. Ниже в век враждовав соблюдеши зол моих, и не осуди мене в преисподних земли. Зане Ты еси Боже, Бог кающихся, да и на мне явиши всю благость Твою, яко недостойна суща спасеши мя, по мнозей милости Твоей. И восхвалю Тя всегда, во вся дни живота моего. Яко Тебе поют вся силы небесныя, и Твоя есть слава во веки, аминь.


\itshape Таже: Трисвятое. 

По Отче наш. Молитва Исусова.\normalfont{}

Аминь.


\mysubsubsection{\textit{Глаголем} тропари сия, глас 6:}


Помилуй нас Господи, помилуй нас, всякаго бо ответа недоумеюще, сию Ти молитву, яко Владыце, грешнии раби Твои приносим. Помилуй нас Господи, помилуй нас.

Слава. Господи, помилуй нас, на Тя бо уповахом. Не прогневайся на ны зело, ниже помяни беззаконии наших. Но призри на ны, яко Милосерд, и избави нас от враг наших. Ты бо еси Бог наш, и мы людие Твои, и вси дела руку Твоею, и имя Твое призываем.

И ныне, Богородичен. Милосердия двери отверзи нам, благословеная Богородице Дево, надеющиися на Тя не погибнем, но да избавимся Тобою от бед. Ты бо еси спасение роду християнскому.

Господи помилуй (40), Слава, и ныне.

Честнейшую херувим (поклон великий).

Именем Господним благослови, отче.

За молитв святых отец наших.

Аминь.


\itshape Таже молитва Великаго Василия.\normalfont{}


Владыко Боже Отче Вседержителю, и Господи Сыне Единородныи Исусе Христе, и Святыи Душе, едино Божество, и едина сила, помилуй мя грешнаго и ими же веси судьбами, спаси мя недостойнаго раба Твоего, яко благословен еси во веки, аминь.

Приидите, поклонимся (трижды, и поклоны три).


\mysubsubsection{\textit{Аще первая неделя Великаго поста, глаголем} псалом 142:}


Господи, услыши молитву мою.


\mysubsubsection{\textit{В прочая же недели глаголем} псалом 69}


Боже, в помощь мою вонми. Господи, помощи ми потщися. Да постыдятся и посрамятся ищущии душу мою. Да возвратятся вспять и постыдятся мыслящии ми злая. Да возвратятся абие стыдящеся, глаголющии ми: благо же, благо же. Да возрадуются и возвеселятся о Тебе, вси ищущии Тебе Боже, и глаголют выну: да возвеличится Господь, любящии спасение Твое. Аз же нищь есмь и убог, Боже помози ми. Помощник мой и Избавитель мой еси Ты, Господи, не закосни.


\mysubsubsection{Псалом 142}


Господи, услыши молитву мою, внуши моление мое во истине Твоей, услыши мя в правде Твоей. И не вниди в суд с рабом Твоим, яко не оправдится пред Тобою всяк живыи. Яко погна враг душу мою, и смирил есть в земли живот мой. Посадил мя есть в темных, яко мертвыя веку, и уны во мне дух мой, во мне смутися сердце мое. Помянух дни древния, поучихся во всех делех Твоих, и в делех руку Твоею поучахся. Воздех к Тебе руце мои, душа моя яко земля безводная Тебе. Скоро услыши мя Господи, исчезе дух мой. Не отврати лица Твоего от мене, и уподоблюся низходящим в ров. Слышану сотвори мне заутра милость Твою, яко на Тя уповах. Скажи мне Господи, путь, воньже поиду, яко к Тебе взях душу мою. Изми мя от враг моих Господи, к Тебе прибегох. Научи мя творити волю Твою, яко Ты еси Бог мой. Дух Твой Благии наставит мя на землю праву. Имене Твоего ради Господи, живиши мя, правдою Твоею изведеши от печали душу мою. И милостию Твоею потребиши враги моя и погубиши вся стужающия души моей, яко аз раб Твой есмь.

Слава в вышних Богу, и на земли мир, в человецех благоволение. Хвалим Тя, благословим Тя (поклон), кланяемтися, славословим Тя (поклон), благодарим Тя великия ради славы Твоея (поклон). Господи Царю Небесныи, Боже Отче Вседержителю, и Господи Сыне Единородныи Исусе Христе, и Святыи Душе. Господи Боже Агньче Божии, Сыне Отечь. Вземляи грехи мiра, помилуй нас. Вземляи грехи мiра, приими молитвы наша. Седяи одесную Отца, помилуй нас. Яко Ты един свят, Ты един Господь Исус Христос, в славу Богу Отцу, аминь.

На всяку нощь благословим Тя, и восхвалим имя Твое во веки, и в век века. Господи, прибежище бысть нам, в род и род. Аз рех: Господи, помилуй мя, и исцели душу мою, яко согреших Тебе. Господи, к Тебе прибегох, научи мя творити волю Твою, яко Ты еси Бог мой. Яко от Тебе есть источник живота. Во свете Твоем узрим свет, пробави милость Твою ведущим Тя. Сподоби Господи, в нощь сию, без греха сохранитися нам. Благословен еси Господи Боже отец наших, и хвально и прославлено имя  Твое во веки, аминь. Буди  Господи, милость Твоя на нас, яко же уповахом на Тя. Благословен еси Господи, научи нас оправданием Твоим. Благословен еси Владыко, вразуми нас оправданием Твоим. Благословен еси Святыи, просвети нас оправданием Твоим. Господи, милость Твоя во веки, и дела руку Твоею не презри. Тебе подобает хвала. Тебе подобает пение. Тебе слава подобает, Отцу и Сыну и Святому Духу, ныне и присно и во веки веком, аминь.

Аще ли первая неделя святаго поста глаголем:

Трисвятое. 

По Отче наш. Молитва Исусова.

Аминь.

И поем тропари: Господи сил с нами.

В прочая же недели и посты, чтем канон Пресвятей Богородице. Таже, поем: Достойно есть. И поклон до земли. Трисвятое, и по Отче наш.


\itshape Аще есть пост Великии, начинает первый лик тропарь, глас 6:\normalfont{}

Господи сил с нами буди, иного бо разве Тебе помощника в скорбех не имамы, Господи сил помилуй нас.

И другии лик поет той же стих: Господи сил с нами.


\itshape Посем первый лик поет:\normalfont{} Хвалите Бога во святых Его, хвалите Его во утвержении сил Его.


\itshape И абие вторыи лик поет тропарь:\normalfont{} Господи сил с нами.


\itshape Тако и прочая стихи правыи лик поет. А тропарь поет левыи лик.\normalfont{}


\itshape Стих 2.\normalfont{}


Хвалите Его на силах Его, хвалите Его по премногому величествию Его. Господи сил.


\itshape Стих 3.\normalfont{}


Хвалите Его во гласе трубнем, хвалите Его во псалтыри и гуслех. Господи сил.


\itshape Стих 4.\normalfont{}


Хвалите Его в тимпане и лице, хвалите Его во струнах и органех. Господи сил.


\itshape Стих 5.\normalfont{}


Хвалите Его в кимвалех доброгласных, хвалите Его в кимвалех восклицания. Всяко дыхание да хвалит Господа. Господи сил.


\itshape Таже, оба лика совокупльшеся вкупе, сий стих поют:\normalfont{} Хвалите Бога во святых Его, хвалите Его во утвержении сил Его. Господи сил.


\medskip\itshape Слава, глас 6.\normalfont{}


Господи, аще не быхом святыя Твоя имели молитвеники, и благостыню Твою милующую нас, како смели быхом Спасе, пети Тя, Его же славословят безпрестани ангели. Сердцеведче, пощади душа наша.

И ныне. Многая премножества моих Богородице, согрешении, к Тебе прибегох Пречистая, спасения требуя. Посети болезнующую мою душу, и моли Сына Своего и Бога нашего, подати ми оставление, их же содеях зле, едина Благословеная.

Всесвятая Богородице, во время живота моего не остави мене, и человечестей помощи не ввери мене, но Сама мя заступи и помилуй мя.

Все упование мое к Тебе возлагаю, Мати Божия, сохрани мя во Своем си крове.

Господи помилуй (40). Посем молитва Великаго Василия:

Иже на всяко время, и на всяк час, на небеси и на земли, покланяемыи и славимыи, Боже Благии, долготерпеливе и многомилостиве. Иже праведники любя и грешных милуя, Иже всех зовыи ко спасению, обещания ради будущих благ; сам Господи, приими молитвы наша в час сей, и в благости исправи живот наш к заповедем Твоим. Душа наша освяти и телеса очисти, помышления исправи и мысли очисти. Разум уцеломудри и истрезви, и избави нас от всякия скорби, зол и болезней, и душевных страстей. И огради нас святыми ангелы Твоими, яко да ополчением их соблюдаеми и наставляеми, достигнем во единство веры, и в разум неприкосиовенныя Ти славы, яко благословен еси во веки, аминь.

Господи помилуй (трижды).

Слава, и ныне.

Честнейшую херувим, и славнейшую воистину серафим, без истления Бога Слова рождьшую, сущую Богородицу, Тя величаем (поклон великий).

Именем Господним благослови, отче.

За молитв святых отец наших. Аминь.

И творим великия поклоны с молитвою святаго Ефрема, якоже явлено в Вечерни.


\itshape По сих, молитва Пресвятей Богородице, Великаго Василия:\normalfont{}


Нескверная, Неблазная, Нетленная, Пречистая Богоневесто Владычице, яже Бога Слова человеком преславным Си Рожеством соединивши, и падшее естество рода нашего Небесным совокупльши, яже ненадежным едина Надежда, и побеждаемым Помощница; готовое Заступление к Тебе прибегающим, и всем християном Прибежище; не гнушайся мене грешнаго и сквернаго, иже скверными помыслы и словесы, и делы и деяньми, всего себе непотребна сотворша, и сластем уныния, и лености нравом раба бывша. Но яко человеколюбиваго Бога Мати, человеколюбне умилосердися о мне грешнем и блуднем, и приими мое еже от скверных устен приносимое Тебе моление, и Своего Сына и нашего Владыку и Господа, Матерьним Си дерзновением обьемши умоли, яко да отверзет и мне человеколюбное милосердие Своея благости, и презрев моя безчисленая прегрешения, обратит мя на покаяние, и Своим заповедем делателя искусна явит мя; предстани ми присно милостивая и милосердая и человеколюбивая в настоящей сей жизни, теплая Предстательнице и Помощнице, сопротивных полки отгоняющи и ко спасению наставляющи мя, и во время исхода моего окаянную ми душу соблюдающи, и темныя образы лукавых бесов далече от нея отревающи. В страшныи же день праведнаго суда вечныя мя избавляющи муки, и неизреченней славе Твоего Сына и Бога нашего наследника мя показающи. Ея же да улучу, Владычице моя Пресвятая Богородице, Твоего ради ходатайства и заступления, благодатию и человеколюбием, единороднаго Сына Твоего, Господа Бога и Спаса нашего Исуса Христа, Ему же подобает всяка слава, честь и покланяние, со безначальным Его Отцем, и с Пресвятым и Благим и Животворящим Духом, ныне и присно и во веки веком, аминь.


\itshape Аще ли есть Великии пост, глаголем:\normalfont{} Трисвятое. По Отче наш, возгласа несть. Господи помилуй (12).


\itshape Таже, молитва святаго Антиоха мниха:\normalfont{} Даждь нам Владыко, на сон грядущим, покой души и телу, и сохрани нас от мрачнаго сна греховнаго, и всякаго темнаго и нощнаго сладострастия. Утиши стремление страстем и угаси телесное разжжение и стрелы лукаваго, яже на ны лукавно движимыя. И востания плоти нашея утоли и всяко земное и вещное мудрование успи, и даруй нам Боже, бодр ум и целомудр помысл, сердце трезвящеся, сон легок, и всякаго сатанина мечтания неискусен. Востави же нас во время молитвы утвержены в заповедех Твоих, и память судеб Твоих в себе выну имуща,  и всенощное славословие нам даруй, пети и благословити и славити пречестное и великолепое имя Твое, Отца и Сына н Святаго Духа, и ныне и присно и во веки веком, аминь.

Преславная и присно Дево Богородице Марие, Мати Христа Бога нашего, приими молитвы наша, и принеси я Сыну Своему и Богу нашему, да спасет и просветит Тебе ради душа наша.

Упование нам Бог и прибежище наше Христос, и Покровитель нам есть Дух Святыи.

Слава, и ныне. 

Господи помилуй, Господи помилуй, Господи благослови.


\itshape И отпуст седмичный, иже в субботу.\normalfont{}


\itshape Таже, прощение: Ослаби, остави. 

И молитва:\normalfont{} Ненавидящих и обидящих нас прости, Господи Человеколюбче.

Господи помилуй (трижды). 

\itshape И исходные поклоны.\normalfont{}


\section{Павечерница малая}
%http://www.molitvoslov.com/text1005.htm 
 


\itshape По обычнем начале глаголем:\normalfont{}


За молитв святых отец наших, Господи Исусе Христе, Сыне Божии, помилуй нас. Аминь.

Царю небесныи, Утешителю, Душе истинныи. Иже везде сыи, и вся исполняя, сокровище благих, и жизни Подателю, прииди и вселися в ны, и очисти ны от всякия скверны, и спаси Блаже, душа наша.

Святыи Боже. Святыи Крепкии, Святыи безсмертныи, помилуй нас (трижды, и поклоны три).

Слава Отцу и Сыну и Святому Духу, ныне и присно и во веки веком, аминь.

Пресвятая Троице, помилуй нас. Господи, очисти грехи наша: Владыко, прости беззакония наша: Святыи, посети и исцели немощи наша, имене твоего ради. 

Господи помилуй (трижды). Слава, и ныне.

Отче наш, Иже еси на небесах. да святится имя Твое. да приидет царствие Твое. да будет воля Твоя, яко на небеси и на земли. хлеб наш насущныи даждь нам днесь. и остави нам долги наша, якоже и мы оставляем должником нашим. и не введи нас во искушение. но избави нас от лукаваго.

Господи Исусе Христе, Сыне Божии, помилуй нас.

Аминь.

Господи помилуй (12). Слава, и ныне.

Приидите, поклонимся Цареви нашему Богу (поклон).

Приидите, поклонимся Христу, Цареви и Богу нашему (поклон).

Приидите, поклонимся и припадем к Самому Господу Исусу Христу, Цареви и Богу нашему (поклон).


\mysubsubsection{Псалом 50}


Помилуй мя Боже, по велицей милости Твоей, И по множеству щедрот Твоих, очисти беззаконие мое. Наипаче омый мя от беззакония моего, и от греха моего очисти мя. Яко беззаконие мое аз знаю, и грех мой предо мною есть выну. Тебе единому согреших, и лукавое пред Тобою сотворих. Яко да оправдишися в словесех Своих, и победиши внегда судити. Се бо в беззакониих зачат есмь, и во гресех роди мя мати моя. Се бо истину возлюбил еси, безвестная и тайная премудрости Твоея явил ми еси. Окропиши мя иссопом и очищуся, омыеши мя и паче снега убелюся. Слуху моему даси радость и веселие, возрадуются кости смиренныя. Отврати лице Твое от грех моих, и вся беззакония моя очисти. Сердце чисто созижди во мне Боже, и дух прав обнови во утробе моей. Не отверзи мене от лица Твоего и Духа Твоего Святаго не отыми от мене. Воздаждь ми радость спасения Твоего, и Духом Владычным утверди мя. Научу беззаконныя путем Твоим, и нечестивии к Тебе обратятся. Избави мя от кровий Боже, Боже спасения моего; возрадуется язык мой правде Твоей. Господи, устне мои отверзеши, и уста моя возвестят хвалу Твою. Яко аще бы восхотел жертвы, дал бых убо: всесожжения не благоволиши. Жертва Богу дух сокрушен: сердце сокрушено и смирено Бог не уничижит. Ублажи, Господи, благоволением Твоим Сиона, и да созиждутся стены иеросалимския. Тогда благоволиши жертву правды, возношение и всесожегаемая. Тогда возложат на олтарь Твой тельца.


\mysubsubsection{Псалом 69}


Боже, в помощь мою воньми. Господи, помощи ми потщися. Да постыдятся и посрамятся ищущии душу мою. Да возвратятся вспять и постыдятся мыслящии ми злая. Да возвратятся абие стыдящеся глаголющии ми: благо же, благо же. Да возрадуются и возвеселятся о Тебе, вси ищущии Тебе, Боже, и глаголют выну: да возвеличится Господь, любящии спасение Твое. Аз же нищь есмь и убог, Боже помози ми. Помощник мой, и Избавитель мой еси Ты  Господи, не закосни.


\mysubsubsection{Псалом 142}


Господи, услышы молитву мою, внуши моление мое воистине Твоей, услыши мя в правде Твоей. И не вниди в суд с рабом Твоим, яко не оправдится пред Тобою всяк живыи. Яко погна враг душу мою, и смирил есть в земли живот мой. Посадил мя есть в темных, яко мертвыя веку и уны во мне дух мой, во Мне смутися сердце мое. Помянух дни древния, поучихся во всех делех Твоих. и в делех руку Твоею поучахся, воздех к Тебе руце мои, душа моя, яко земля безводная Тебе. Скоро услыши мя Господи, исчезе дух мой. Не отврати лица Твоего от мене, и уподоблюся низходящим в ров. Слышану сотвори мне заутра милость Твою, яко на Тя уповах. Скажи мне Господи, путь, воньже пойду, яко к Тебе взях душу мою. Изми мя от враг моих, Господи, к Тебе прибегох. Научи мя творити волю Твою, яко Ты еси Бог мой. Дух Твой Благии наставит мя на землю праву. Имене Твоего ради Господи, живиши мя, правдою Твоею изведеши от печали душу мою. И милостию Твоею потребиши враги моя, и погубиши вся стужающия души моей, яко аз раб Твой есмь.


Слава в вышних Богу, и на земли мир, в человецех благоволение. Хвалим Тя, благословим Тя (поклон), кланяемтися, славословим Тя (поклон), благодарим Тя великия ради славы Твоея (поклон). Господи Царю небесныи, Боже Отче Вседержителю, и Господи Сыне Единородныи Исусе Христе, и Святыи Душе: Господи Боже, Агньче Божии, Сыне Отечь, вземляи грехи мiра, помилуй нас. Вземляи грехи мiра. приими молитвы наша. Седяи одесную Отца, помилуй нас. Яко Ты един свят, Ты един Господь Исус Христос, в славу Богу Отцу, аминь. На всяку нощь благословлю Тя, и восхвалю имя Твое во веки, и в век века. Господи, Прибежище бысть нам, в род и род. Аз рех: Господи, помилуи мя и исцели душу мою, яко согреших Тебе. Господи, к Тебе прибегох, научи мя творити волю Твою, яко Ты еси Бог мой, яко от Тебе есть источник живота, во свете Твоем узрим свет. Пробави милость Твою ведущим Тя. Сподоби Господи, в нощь сию без греха сохранитися нам. Благословен еси, Господи Боже отец наших, и хвально и прославлено имя Твое во веки,  аминь. Буди Господи, милость Твоя на нас, якоже уповахом на Тя. Благословен еси Господи, научи нас оправданием Твоим. Благословен еси Владыко, вразуми нас оправданием Твоим. Благословен еси Святыи, просвети нас оправданием Твоим. Господи, милость Твоя во веки, и дела руку Твоею не презри. Тебе подобает хвала, Тебе подобает пение. Тебе слава подобает: Отцу и Сыну и Святому Духу, ныне и присно и во веки веком, аминь.

Исповедание православныя веры, перваго Собора (перекрестись без поклона): 

Верую во единаго Бога Отца, Вседержителя, Творца небу и земли, видимым же всем и невидимым.

И во единаго Господа Исуса Христа, Сына  Божия, Единороднаго, Иже от Отца рожденнаго прежде всех век. Света от Света, Бога истинна от Бога истинна, рождена, а не сотворена, единосущна Отцу, Им же вся быша.

Нас ради человек, и нашего ради спасения сшедшаго с небес, и воплотившагося от Духа Свята и Марии Девы вочеловечьшася.

Распятаго за ны при понтийстем Пилате, страдавша и погребена.

И воскресшаго в третии день по писаниих.

И возшедшаго на небеса, и седяща одесную Отца.

И паки грядущаго со славою судити живым и мертвым, Его же царствию несть конца.

Втораго собора:

И в Духа Святаго, Господа истиннаго и Животворящаго, Иже от Отца исходящаго, Иже со Отцем и Сыном спокланяема  и сславима, глаголавшаго пророки.

И во едину святую соборную и апостольскую Церковь.

Исповедую едино Крещение во оставление грехов.

Чаю воскресения мертвым.

И жизни будущаго века. Аминь.


\itshape Посем, канон Пресвятей Богородице, во октаи гласу или ин, по уставу. В  келиях глаголем канон Пресвятей Богородице, творение Феофаново, глас8. \normalfont{}


\mysubsubsection{Песнь 1, ирмос}


Воду прошед, яко по суху, из египетска зла избежав. Израильтянин вопияше: Избавителю Богу нашему поем.


\itshape Запев:\normalfont{} Пресвятая Госпоже Богородице, спаси нас (поклон).

Многими содержим напастьми, к Тебе прибегаю, спасении искии, О! Мати Слову и Дево, от лютых и нужных мя спаси.


\itshape Запев.\normalfont{}


Страстей мя смущают прилози, и многия печали исполняют ми душу; умири Отроковице, тишиною Сына и Бога Твоего, Всенепорочная.


Слава: Спаса рождшую Тя и Бога, молю Девице избавити мя от лют. К Тебе бо ныне прибегаю, простирая душу и помышление.

И ныне. В скорби мя суща и в тузе, посещения божественаго промышления, еже от Тебе сподоби, едина Богомати, яко блага, Благому же Родительнице.


\mysubsubsection{Песнь 3, ирмос}


Небесному кругу, Верхотворче Господи, и церкви Зиждителю, Ты мене утверди в любви Своей, желанием сыи край, и верным утвержение,  едине Человеколюбче.


\itshape Запев.\normalfont{}


Заступление и покров жизни моей положих Тя, Богородительнице Дево, Ты мене окорми  к пристанищу Своему, благим виновна, и верным утвержение, едина всепетая.


\itshape Запев.\normalfont{}


Молю Девице, душевное смущение, и печали бурю разорити ми. Ты бо, Богоневестная, Начальника тишине Христа родила еси, едина Пречистая.


Слава.


Благодетеля рождьши добрым виновнаго, благодеяния богатство всем источи. Вся бо можеши яко сильна в крепости, Христа рождьши, Богоблаженная.

И ныне. Лютыми недуги, и болезненными страстьми истязаему, Дево, Ты ми помози, исцелением бо неоскудно Тя знаю сокровище, некрадомо и неиздаемо.

Господи помилуй (трижды). 

Слава, и ныне:


\mysubsubsection{Седален, глас 8}


Яко Всенепорочная Невесто Творцу, яко Неискусомужная Мати Избавителя, приятелище яко сущи Утешителя всепетая, беззаконию мя суща скверно жилище и бесом игралище в разуме бывша, потщися сих злодействия мя избавити, светло жилище добродетельми соделавши, светодательная и нетленная, разжени


\mysubsubsection{Песнь 4, ирмос}


Услышах Господи, смотрения Твоего таиньство, разумех дела Твоя, и прославих Твое Божество.

Запев. Страстей моих смущение, яже кормчию  рождьши Господа, бурю утиши моих прегрешении, Богоневестная.

Запев. Милосердия бездну призывающу Твою подаждь ми, яже благосердаго рождьши, и Спаса всем поющим Тя.

Слава. Надежу и утверждение, и спасения стену недвижиму, стяжавше Тя Всенепорочная, злолютых всех избавляемся.

И ныне. На одре болезни моея греховне слежаща мя, яко человеколюбива, помози ми Богородице, Мати присно Дево.


\mysubsubsection{Песнь 5, ирмос}


Просвети нас повелением Си Господи, и мышцею высокою Твоего мира подаждь нам, Едине Человеколюбче.

Запев. Исполни Чистая, веселия живота нашего, Твою нетленную дающи радость, веселия рождьши сущаго Вину.

Запев. Избави нас от бед, Богородице Чистая, вечное рождьши избавление, и мира всяк ум преимущаго.

Слава. Разреши мглу прегрешении моих, Богоневесто, светом Твоея светлости, яже Свет рождьшая Божественыи и Превечныи.

И ныне. Исцели Чистая, души моея неможение посещением Си убеждьшися, и здравие мольбами Си подаждь ми.


\mysubsubsection{Песнь 6 , ирмос}


Молитву пролию ко Господу, и Тому возвещу печаль мою, яко зол душа моя наполнися, и живот мой аду приближися. Но молюся яко Иона: от тли, Боже мой, возведи мя.

Запев. Смерти и тли яко спасл есть, Сам ся издав на смерть, тлением и смертию мое естество ято бывшее, Дево, моли Господа и Сына Своего, врагов злодействия мя избавити.

Запев. Заступницу Тя животу свем, и хранительницу тверду Девице, напастей решащу молвы, и налоги бесом отгонящу. И молюся всегда, страстей мя избавити, Всенепорочная.

Слава. Яко стену и прибежище стяжахом, и душам Тя совершеное спасение, и пространство в скорбех Отроковице, и светом Ти присно радуемся, О! Владычице, ныне нас от страстей и бед спаси.

И ныне. На одре ныне немощию низлежу, и несть исцеления плоти моей. Но яже Бога и Спаса мiру, и Избавителя недугом рождьшая, Тебе молюся благой: от истлення болезней востави мя.

Господи помилуй (трижды). 

Слава, и ныне.


\mysubsubsection{Кондак, глас 6}


Заступнице християном непостыдная, Ходатаице ко Творцу непреложная, не презри грешных моления гласы. Но предвари яко блага на помощь нашу, верно вопиющих Ти: Ускори на молитву, и потщися на умоление, заступающи присно, Богородице, чтущих Тя.

Икос. Простри длани Свои, имаже всех Владыку, яко Младенца прият, за множество благости, не остави нас всегда надеющихся на Тя. Бодреною Си молитвою, и неисчетною простынею ущедри нас, и подаждь душам нашим милосердие Свое во веки источающи. Тебе бо имамы грешнии заступницу, от находящих на ны бед и зол. Но яко имущи милосердия щедроты, ускори на молитву, и потщися на умоление, заступающи присно Богородице чтущих Тя.


\mysubsubsection{Песнь 7, ирмос}


Иже от Июдеи дошедше отроцы в Вавилон древле, верою утвержении, пламень пещныи попраша глаголюще: отец наших Боже,   благословен еси.

Запев. Наше спасение якоже восхотел, Спасе, устроити, во чрево Девыя вселися, юже мiру Заступницу показа. Отец наших Боже, благословен еси.

3апев. Волителя милости, Егоже роди Мати, ныне умоли, избавити мя от прегрешении и душевныя скверны, Сыну Твоему зовуща: отец наших Боже, благословен еси.

Слава. Сокровище спасения, и источник нетления, Тебе рождьшую, столп утвержения, и дверь покаяния, вопиющим показа: отец наших Боже, благословен еси.

И ныне. Телесныя недуги, и душевныя грехи, Богородительнице, любовию приступающим к крову Твоему Девице, испелити сподоби, яже Спаса Христа нам рождьшая.


\mysubsubsection{Песнь 8, ирмос}


Царя Небесного, Его же поют вои ангельстии, хвалите, и превозносите Его во веки.

Запев. Помощи яже от Тебе требующия, не презри Дево, поющия и превозносящия Христа во веки.

Запев. Неможение души моея исцеляеши и телесныя болезни Дево, яко да Тя поем, и превозносим Чистую во веки.

Слава. Исцелением богатство изливаеши, верно поющим Тя Девице, и превозносящим Христа во веки.

И ныне. Напастныя Ты прилоги отгоняеши, и страстныя находы Дево, яко да Тя поем и превозносим Чистую во веки.


\mysubsubsection{Песнь 9, ирмос}


Воистину Богородицу Тя исповедающе, спасении Тобою, Девице Чистая, с небесными вои Тя величаем.

Запев (поклон) Тока слез моих не отвратися, яже от всякаго лица вся слезы отъемшаго, Девице, Христа  рождьшая.

Запев. Радости мое сердце исполни Дево, яже радости приемши исполнение, греховную печаль потребляющи.

Слава. Света Твоего зарями просвети Дево, мрак неведения отгонящи, благоверно Богородицу Тя исповедающих.

И ныне. В место озлобления, немощи низлежащаго, Дево исцели, из нездравия во здравие претворяющи.

По каноне же, Достойно есть. Трисвятое. И по Отче наш. Молитва Исусова. Тропарь дню и святому, его же есть храм. Аще ли храм Христов, или Богородицы, глаголи преже храму.


\itshape Таже:\normalfont{} Боже отец наших, творяи присно с нами по Твоему смотрению, не отстави милости Твоея от нас, но тех молитвами, во смирении устрои живот наш.


\itshape Ин тропарь, глас той же:\normalfont{} Иже во всем мiре мученик Твоих Господи, яко багряницею и виссом кровьми их, Церкви Твоя украсившися, теми вопиет Ти, Христе Боже: людем Твоим щедроты Твоя низпосли, и мир граду Твоему даруй, и душам нашим велию милость.

Слава. Со святыми покой, Христе, душа раб Своих, идеже несть болезни, ни печали, ни воздыхания, но жизнь вечная.

И ныне. Молитвами Господи, всех святых и Богородицы, Твой мир даждь нам, и помилуй нас, яко един Щедр.


\itshape Аще в пяток вечер, глаголются тропари сия:\normalfont{} Апостоли, пророцы и мученицы, святителие, преподобнии и праведнии, иже добре подвиг скончавше, и веру соблюдше, дерзновение имуще ко Спасу, молим, вы, о нас Того молите яко Блага, спасти душа наша.

Таже тропарь, его же церковь.

Слава. Со святыми покой.


\itshape И ныне, кондак, глас 8.\normalfont{} Яко начатки естества Содетелю твари, вселенная приносит Ти Господи, богоносныя мученики. Тех молитвами, во смирении глубоце церковь Свою, и град Свой, Богородицы ради соблюди, едине Многомилостиве.


\itshape Аще ли господьскии праздник, или нарочитаго святаго, глаголем кондак его.\normalfont{} Господи помилуй, 40.


\itshape Таже, молитва Великаго Василия:\normalfont{} Иже на всяко время, и на всяк час, на небеси и на земли, покланяемыи и славимыи, Боже Благии, долготерпеливе и многомилостиве, Иже праведники любя, и грешных милуя, Иже всех зовыи ко спасению, обещания ради будущих благ; сам Господи, приими молитвы наша в час сей, и в благости исправи живот наш,  к заповедем Твоим. Душа наша освяти, и телеса очисти,  помышления исправи, и мысли очисти, разум уцеломудри, и истрезви, и избави нас от всякия скорби, зол и болезней, и душевных страстей. И огради нас святыми ангелы Твоими, яко да ополчением их соблюдаеми и наставляеми, достигнем во единство веры, и в разум неприкосновенныя Ти славы, яко благословен еси во веки, аминь.

Господи помилуй (трижды). Слава, и ныне. Честнейшую херувим.

Именем Господним благослови, отче. 

За молитв святых отец наших, Господи Исусе Христе, Сыне Божии, помилуй нас. Аминь.


\itshape Молитва Пресвятей Богородице, святаго Великаго Василия:\normalfont{} Нескверная, Неблазная, Нетленная, Пречистая, Богоневесто Владычице, яже Бога Слова человеком, преславным Си Рожеством соединивши. И падшее естество рода нашего, небесным совокупльши. Яже ненадежным едина надежда, и побеждаемым Помощница. Готовое заступление к Тебе прибегающим, и всем християном прибежище, не гнушайся мене грешнаго и сквернаго. Иже скверными помыслы и словесы, и делы и деяньми, всего себе непотребна сотворша, и сластем уныния, и лености нравом раба бывша. Но яко человеколюбиваго Бога Мати, человеколюбне умилосердися о мне грешнем и блуднем, и приими  мое еже от скверных устен, приносимое Тебе моление. И Своего Сына и нашего Владыку и Господа, матерним Си дерзновением объемши умоли, яко да отверзет и мне человеколюбное милосердие Своея благости. И презрев, моя безчисленая прегрешения, обратит мя на покаяние. И Своим заповедем делателя искусна явит мя. Предстани ми присно милостивая, и милосердая и человеколюбивая в настоящей сей жизни, теплая Предстательнице и Помощнице, сопротивных полки отгоняющи, и ко спасению наставляющи мя, и во время исхода моего, окаянную ми душу соблюдающи, и темныя образы лукавых бесов, далече от нея отревающи. В страшныи же день праведнаго суда, вечныя мя избавляющи муки. И неизреченней славе Твоего Сына и Бога нашего, наследника мя показающи. Ея же да улучу, Владычице моя Пресвятая Богородице, Твоего ради ходатайства и заступления, благодатию и человеколюбием, единороднаго Сына Твоего, Господа Бога и Спаса нашего Исуса Христа. Ему же подобает всяка слава, честь и покланяние, со Безначальным Его Отцем, и с Пресвятым и Благим, и Животворящим Духом, ныне и присно и во веки веком, аминь.

Молитва святаго Антиоха мниха: Даждь нам Владыко, на сон грядущим, покой души и телу, и сохрани нас от мрачнаго сна греховнаго, и всякаго темнаго и нощнаго сладострастия. Утиши стремление страстем, и угаси телесное разжжение, и стрелы лукаваго, яже на ны лукавно движимыя, и востания плоти нашея утоли, и всяко земное и вещное мудрование успи. И даруй нам Боже, бодр ум и целомудр помысл, сердце трезвящеся, сон легок, и всякаго сатанина мечтания неискусен. Востави же нас во время молитвы, утвержены в заповедех Твоих. И память судеб Твоих в себе выну имуща. И всенощное славословие нам даруй, пети и благословити и славити, пречестное и великолепое имя Твое, Отца и Сына и Святаго Духа, ныне и присно и во веки веком, аминь.

Преславная и присно Дево Богородице Марие, Мати Христа Бога нашего, приими молитвы наша, и принеси я Сыну Своему и Богу нашему, да спасет и просветит Тебе ради душа наша.

Упование нам Бог, и прибежище наше Христос, и покровитель нам есть Дух Святыи.

Слава, и ныне. Господи помилуй (дважды), Господи благослови.


\itshape И отпуст:\normalfont{} Господи Исусе Христе, Сыне Божии, молитв ради Пречистыя Ти Матере, и преподобных и богоносных отец наших, и всех святых, помилуй и спаси мя грешнаго, яко Благ и Человеколюбец. Аминь. 

Таже прошение: Ослаби, остави, отпусти Боже, согрешения моя, вольная и невольная, яже в слове и в деле, и яже в ведении и не в ведении, яже во уме и в помышлении, яже во дни и в нощи, вся ми прости, яко Благ и Человеколюбец, аминь.

Посем вместо ектении глаголем молитву сию: Ненавидящих и обидящих нас прости, Господи Человеколюбче. Благотворящим благо сотвори, братиям и всем сродником нашим, иже и уединившимся, даруй им вся, яже ко спасению прошения и живот вечныи (поклон).

В болезнех сущия посети и исцели, в темницах сущих свободи, по водам плавающим Правитель буди и иже в путех шествующим исправи и поспеши (поклон).

Помяни Господи, и плененныя братию нашу, единоверных православныя веры, и избави их всякаго злаго обстояния (поклон).

Помилуй Господи, давших нам милостыню и заповедавших нам, недостойным, молитися о них, прости их и помилуй (поклон).

Помилуй Господи, труждающихся и служащих нам, милующих и питающих нас, и даруй им вся, яже ко спасению, прошения и живот вечныи (поклон).

Помяни Господи, прежде отшедшия отцы и братию нашу и всели их, идеже присещает свет лица Твоего (поклон).

Помяни Господи, и нашу худость и убожество, и просвети наш ум светом разума святаго Евангелия Твоего, и настави нас на стезю заповедей Твоих, молитвами Пречистыя Твоея Матере и всех святых Твоих, аминь (поклон).

Господи помилуй (трижды). И обычные исходные поклоны.


\section{Вечерня}
%http://www.molitvoslov.com/text1004.htm 
 


\itshape По обычнем начале глаголем:\normalfont{}


За молитв святых отец наших, Господи Исусе Христе, Сыне Божии, помилуй нас.

Псаломщик глаголет кротким и равным гласом с тихостию, и со всяким вниманием и со страхом Божиим во услышание всем: Аминь.

Царю Небесныи, Утешителю, Душе Истинныи, Иже везде сыи, и вся исполняя, сокровище благих, и жизни Подателю, прииди и вселися в ны, и очисти ны от всякия скверны, и спаси Блаже, душа наша.

Святыи Боже, Святыи Крепкии, Святыи Безсмертныи, помилуй нас (трижды, и поклоны три). 

Слава Отцу и Сыну и Святому Духу, и ныне и присно и во веки веком, аминь.

Пресвятая Троице, помилуй нас. Господи, очисти грехи наша. Владыко, прости беззакония наша. Святыи, посети и исцели немощи наша, имене Твоего ради.

Господи,помилуй (трижды).

Слава Отцу и Сыну и Святому Духу, ныне и присно и во веки веком, аминь.

Отче наш, Иже еси на небесех, да святится имя Твоеда приидет царствие Твое, да будет воля Твоя, яко на небеси и на земли, хлеб наш насущныи даждь Нам днесь, и остави нам долги наша, яко же и мы оставляем должником нашим, и не введи нас во искушение, но избави нас от лукаваго.

Господи Исусе Христе, Сыне Божии, помилуй нас. Аминь.

Господи помилуй, 12. Слава, и ныне.

Приидите, поклонимся Цареви нашему Богу (поклон).

Приидите, поклонимся Христу, Цареви и Богу нашему (поклон).

Приидите, поклонимся и припадем к самому Господу Исусу Христу, Цареви и Богу нашему (поклон).


\mysubsubsection{\textit{Таже,} псалом 103}


Благослови душе моя Господа. Господи Боже мой возвеличился еси зело. Во исповедание и в велелепотуся облече. Одеяися светом яко ризою, пропинаяи небо яко кожу. Покрываяи водами превыспреняя Своя, полагаяи облаки на восхождение Свое. Ходяи на крылу ветреню. Творяи ангелы Своя духи, и слуги Своя огнь палящь. Основаяи землю на тверди Своей, не преклонится в век века. Бездна яко риза одеяние ея. На горах станут воды. От запрещения Твоего побегнут, от гласа грома Твоего устрашатся. Восходят горы, и низходят поля, в место еже основал еси им. Предел положи, его же не прейдут, ниже обратятся покрыти землю. Посылая источники в дебрех, посреде гор пройдут воды. Напаяют вся звери сельныя. Ждут онагри  в жажду свою. На ты птицы небесныя привитают, от  среды камения дадят глас. Напаяя горы от превыспрених Своих, от плода дел Твоих насытится земля. Прозябая пажити скотом, и траву на службу человеком. Извести хлеб от земли, и вино веселит сердце человеку, умастити лице елеом. И хлеб сердце человеку укрепит. Насытятся древа польская, кедри ливаньстии их же еси насадил. Ту птицы вогнездятся, еродиево жилище обладает ими. Горы высокия еленем, камень прибежище заяцем. Сотворил еси луну во времена, солнце позна запад свой. Положи тму и бысть нощь, в ней же пройдут вси зверие дубравнии. Скимни рыкающе восхитити, и испросити от Бога пищу себе. Восия солнце, и собрашася, и в ложах своих лягут. Изыдет человек на дело свое, и на делание свое до вечера. Яко возвеличишася дела Твоя Господи, вся премудростию сотворил еси. Исполнися земля твари Твоея. Се море велико и пространно, ту гади, им же несть числа, животна малая с великими. Ту корабли преплавают, змий сей, его же созда ругатися ему. Вся к Тебе чают, дати пищу им во благо время. Давшу Тебе им, соберут. Отверзшу Тебе руку, всяческая исполнятся благости. Отвращьшу же Тебе лице, возмятутся. Отимеши духи их, и исчезнут, и в персть свою возвратятся. Послеши дух Свой и созиждутся, и обновиши лице земли. Буди слава Господня во веки, возвеселится Господь о делех Своих. Призираяи на землю, и творя ю трястися. Прикасаяся горах и воздымятся. Воспою Господеви в животе моем, пою Богу моему дондеже есмь. Да насладится ему беседа моя, аз же возвеселася о Господе. Скончаются грешницы от земли, и беззаконницы, яко не быти им. Благослови душе моя Господа. 

По конце псалма рцы стих: Яко возвеличишася дела Твоя Господи, вся премудростию сотворил еси. 

Таже, Слава, и ныне.

Аллилуия, Аллилуия, слава Тебе, Боже (три и поклоны три).

Господи помилуй (12), Слава, и ныне. Кафисма рядовая.

Егда же бывает малая вечерня, и по псалме глаголем: Слава, и ныне. Аллилуия, Аллилуия, слава Тебе, Боже (трижды). Господи помилуй (трижды). Слава, и ныне.


\mysubsubsection{\textit{Таже} псалом 140}


Господи возвах к Тебе, услыши мя. Воньми глас молитвы моея, егда воззову к Тебе. Да ся исправит молитва моя, яко кадило пред Тобою, воздеяние руку моею: жертва вечерняя. Положи Господи, хранение устом моим, и дверь ограждения о устнах моих. Не уклони сердце мое в словеса лукавствия, непщевати вины о гресех. С человеки творящими беззаконие, и не счетаюся со избранными их. Покажет мя праведник милостию, и обличит мя. Елей же грешнаго, да не намастит главы моея. Яко еще и молитва моя во благоволениих их. Пожерты быша при камени судии их. Услышатся глаголи мои, яко возмогоша, яко толща земли проседеся на земли, разсыпашася кости их при аде. Яко к Тебе Господи, Господи очи мои, на Тя уповах, не отъими душу мою. Сохрани мя от сети юже составиша ми, и от соблазн делающих беззаконие. Впадутся во мрежу свою грешницы, един есмь аз дондеже преиду.


\mysubsubsection{Псалом 141}


Гласом моим ко Господу возвах, гласом моим ко Господу помолюся. Пролию пред Ним молитву мою, и печаль мою пред ним возвещу. Внегда исчезает дух мой, и Ты позна стези мои. На пути сем, по нему же хождах, скрыша сеть мне. Смотрях одесную и возглядах, и небе знаяи мене. Погибе бегство от мене, и несть взыскаяи душу мою. Возвах к Тебе Господи, рек: Ты еси упование мое, часть моя еси на земли живых. Воньми молитву мою, яко смирихся зело. Избави мя от гонящих мя, яко укрепишася паче мене. Изведи из темницы душу мою, исповедатися имени Твоему. Мене ждут праведницы, дондеже воздаси мне.


\mysubsubsection{Псалом 129}


Из глубины возвах к Тебе Господи, Господи, услыши глас мой. Будете уши Твои, внемлюще глас молитвы моея. Аще беззакония назриши Господи, Господи, кто постоит, яко от Тебе очищение есть. Имене ради Твоего потерпех Тя Господи, потерпе душа моя в слово Твое, упова душа моя на Господа. От стражи утрения до нощи, от стражи утрения, да уповает Израиль на Господа. Яко от Господа милость, и много от Него избавление: и Той избавит Израиля от всех беззаконий его.


\mysubsubsection{Псалом 116}


Хвалите Господа вси языцы, похвалите Его вси людие. Яко утвердися милость Его на нас, и истина Господня пребывает во веки.

Таже, стихеры настоянного дня по уставу. Посем, сей стих, творение святаго священномученика Анфиногена.

Свете тихии, святыя славы, безсмертнаго Отца небеснаго, святаго блаженнаго, Исуса Христа, Сына Божия, пришедшаго на запад солнцу, видевше свет вечернии, поем Отца и Сына и Святаго Духа, Бога. Достоин еси во вся времена, пет быти гласы преподобными, Сыне Божии, живот даяи всему мiру: его же ради весь мiр славит Тя.

Таже, прокимны дневныя. В субботу вечер прокимен: Господь воцарися, в лепотуся облече. 

Стих первыи. Облечеся Господь в силу, и препоясася.

Стих вгорыи. Ибо утверди вселенную, яже не подвижится.

Стих третии. Дому Твоему подобает, святыни Господи в долготу днии.


\itshape В неделю вечер:\normalfont{} Се ныне благословите Господа вси раби Господни.

Стих. Стоящии в храме Господни, во дворех дому Бога нашего.


\itshape В понедельник вечер:\normalfont{} Господь услышит мя, егда взову к Нему.

Стих. Внегда возвах, услыша мя Боже правды моея.


Во вторник вечер: Милость Твоя Господи, поженет мя, во вся дни живота моего.

Стих. Господь пасет мя, и ничто же мя лишит.


\itshape В среду вечер:\normalfont{} Боже, во имя Твое спаси мя, и в силе Твоей суди ми.

Стих. Боже, услыши молитву мою, внуши глаголы уст моих.


\itshape В четверток вечер:\normalfont{} Помощь моя от Господа, сотворшаго небо и землю.

Стих. Возведох очи мои в горы, отнюду же приидет помощь моя.


\itshape В пяток вечер:\normalfont{} Боже, заступник мой еси Ты, и милость Твоя предварит мя.

Стих. Изми мя от враг моих Боже, и от востающих на мя избави мя.


\mysubsubsection{\itshape Егда же несть тропаря святому, тогда поем Аллилуию, на глас 6. Стихи же глаголем сия:\normalfont{}}


\itshape В понедельник вечер:\normalfont{} Господи, не яростию Твоею обличи мене, ни гневом Твоим покажеши мене.


\itshape Во вторник и в четверток вечер:\normalfont{} Возносите Господа Бога нашего, и кланяйтеся подножию ногу Его, яко свято есть.


\itshape В среду вечер:\normalfont{} Во всю землю изыде вещание их, и в концы вселенныя глаголы их.


\itshape Аще ли суббота за упокой, поем:\normalfont{} Аллилуия, Аллилуия, Аллилуия, на глас 8.

Стих. Блажени яже избра и прият я Господь.

Стих. Память их от рода в род.

Стих. Душа их во благих водворятся.

И еще стих. Аллилуия, Аллилуия, Аллилуия.


\medskip\itshape По прокимне, или по ектении, или по аллилуии псаломшик глаголет:\normalfont{} Сподоби Господи, в вечер сей, без греха сохранитися нам (поклон). Благословен еси Господи Боже отец наших (поклон), и хвально и прославлено имя Твое во веки, аминь (поклон). Буди Господи, милость Твоя на нас, якоже уповахом на Тя. Благословен еси  Господи, научи нас оправданием Твоим; благословен еси Владыко, вразуми нас оправданием Твоим; благословен еси Святыи, просвети нас оправданием Твоим. Господи, милость Твои во веки, и дела руку Твоею не презри. Тебе подобает хвала, Тебе подобает пение. Тебе слава подобает, Отцу и Сыну и Святому Духу, ныне и присно и во веки веком, аминь.


\medskip\itshape Таже ектения. И на стиховне стихеры, по уставу.

Таже, молитва святаго Симеона Богоприимца:\normalfont{} Ныне отпущаеши раба Твоего Владыко, по глаголу Твоему с миром, яко видеста очи мои спасение Твое, еже еси уготовал пред лицем всех людей, свет во откровение языком, и славу людий Твоих Израиля.

\itshape Таже. Трисвятое, и по Отче наш.\normalfont{}

Исусова молитва.

Аминь.

\itshape Тропарь по уставу. Посем ектения. И отпуст.

Аще ли есть пост, или егда поем Аллилуиа, глаголем тропари сия, глас 4.\normalfont{}

Богородице Дево, радуися, обрадованная Марие, Господь с Тобою, благословена Ты в женах, и благословен плод чрева Твоего, яко родила еси Христа Спаса, Избавителя душам нашим (поклон земной).

Слава. Крестителю Христов, тебе молимся: всех нас помяни, да избавимся от беззаконии наших. Тебе бо дана бысть благодать, молитися за мы (поклон земной).

И ныне. Молите за ны святии апостоли, пророцы, и мученицы, и вси святии, да избавимся от бед и скорбей, вас бо теплыя предстатели ко Спасу вси стяжахом (поклон земной).

Таже, 6огородичен. Под Твою милость прибегаем, Богородице Дево, молитв наших не презри в скорбех. Но от бед избави нас, едина Чистая и Благословеная (без поклона).

Господи помилуй (40), кротким и тихим гласом. Господи благослови.

\itshape Исусова молитва.\normalfont{} Аминь. 

Небесныи Царю, державу нашу укрепи, веру утверди, языки укроти, мiр умири, и святыи храм сей добре сохрани, и прежде отшедшия, отцы и братию нашу, в кровех с праведными учини. И нас в православней вере и в покаянии Господи, приими и помилуй, яко Благ и Человеколюбец. 

Таже, Господи помилуй, трижды.

Слава, и ныне.

Честнейшую херувим, и славнейшую воистину серафим, без истления Бога Слова рождьшую, сущую Богородицу Тя величаем (поклон земной).

Именем Господним благослови, отче.

За молитв святых отец наших, Господи Исусе Христе, Сыне Божии, помилуй нас. Аминь. 

И глаголем молитву святаго Ефрема Сирина, творя поклоны земные.

Господи и Владыко животу моему, дух уныния, небрежения, сребролюбия и празднословия, отжени от мене (поклон великий).

Дух же целомудрия, смирения, терпения и любве, даруй ми рабу Твоему (поклон великий).

Ей, Господи Царю, даждь ми зрети моя согрешения, и еже не осуждати брата моего, яко благословен еси во веки, аминь (поклон великий).

И прочих поклонов (12), глаголюще в себе: Господи Исусе Христе, Сыне Божии, помилуй мя грешнаго (дважды с поклоны); Боже, милостив буди мне грешному (поклон). Боже, очисти грехи моя и помилуй мя (поклон). Создавыи мя Господи, помилуй (поклон). Без числа согреших Господи, прости мя (поклон). И паки, скончав поклоны, глаголем молитву всю выше писаную: Господи и Владыко животу моему (поклон великий един).

И по сем, Трисвятое, и по Отче наш. Господи помилуй (12). Слава, и ныне. Господи помилуй (дважды), Господи благослови. \itshape И отпуст седмичный.\normalfont{}

Господи помилуй (трижды). исходные поклоны.


\section{Полунощница повседневная}
%http://www.molitvoslov.com/text1007.htm 
 


\itshape По обычнем начале глаголем со умилением и сокрушением сердца:\normalfont{} За молитв святых отец наших, Господи Исусе Христе Сыне Божии, помилуй нас. (поклон). Аминь. 


Слава Тебе Боже наш, слава Тебе Всяческих ради. (трижды). Боже очисти мя грешнаго, яко николиже благо сотворих пред Тобою (поклон). Но избави мя от лукаваго, и да будет во мне воля Твоя (поклон). Да не осужденно отверзу уста моя недостойная, и восхвалю имя Твое святое, Отца и Сына, и Святаго Духа, ныне и присно, и во веки веком, аминь (поклон). 

Царю небесныи, Утешителю, Душе истинныи, иже везде сыи, и вся исполняя, сокровище благих, и жизни Подателю, прииди и вселися в ны, и очисти ны от всякия скверны, и спаси Блаже душа наша. 

Святыи Боже, Святыи Крепкии, Святыи Безсмертныи, помилуй нас (трижды, и поклоны три). Слава Отцу и Сыну и Святому Духу, и ныне и присно, и во веки веком, аминь. 

Пресвятая Троице помилуй нас. Господи очисти грехи наша. Владыко прости беззакония наша. Святыи посети, и исцели немощи наша имене Твоего ради. 

Господи помилуй (трижды). 

Слава Отцу и Сыну и Святому Духу, и ныне и присно, и во веки веком, аминь. 

Отче наш, иже еси на небесех. Да святится имя Твое. Да приидет царствие Твое. Да будет воля твоя, яко на небеси и на земли. Хлеб наш насущныи, даждь нам днесь. И остави нам долги наша, якоже и мы оставляем должником нашим. И не введи нас во искушение. Но избави нас от лукаваго. 

Господи Исусе Христе, Сыне Божии, помилуй нас. Аминь. 

Господи помилуй (12). 

\itshape Посих, молитва:\normalfont{} От сна восстав, благодарю тя Всесвятая Троице, яко многия ради благости и долготерпения, не прогневася на мя грешнаго, и лениваго раба твоего, и не погубил еси мене со беззаконии моими, но человеколюбствова. И в нечаянии лежаща воздвиже мя, утреневати и славословити державу Твою непобедимую. И ныне Владыко Боже пресвятыи, просвети очи сердца моего. И отверзи ми устне поучатися словесем Твоим, и разумети заповеди Твоя, и творити волю Твою, и пети Тя во исповедании сердечнем. Воспевати же и славити пречестное и великолепое имя Твое, Отца и Сына и Святаго Духа, ныне и присно и во веки веком, аминь. 

Приидите поклонимся Цареви нашему Богу (поклон).

Приидите поклонимся Христу, Цареви и Богу нашему (поклон).

Приидите поклонимся и припадем к Самому Господу Исусу Христу, Цареви и Богу нашему (поклон). \itshape Таже, Псалом 50\normalfont{}. Помилуй мя Боже. \itshape Таже, кафисма 17\normalfont{}.


\medskipАллилуия, \bfseries псалом 118\normalfont{}.


Блажени непорочнии в путь, ходящии в законе Господни. Блажени испытающии свидения Его, всем сердцем взыщут Его. Неделающии бо беззакония в путех Его ходиша. Ты заповеда заповеди Твоя сохранити зело. Еда исправилися быша путие мои, сохранити оправдания Твоя. Тогда не постыжуся, егда призрю на вся заповеди Твоя. Исповемся Тебе в правости сердца, внегда научитимися судбам правды Твоея. Оправдания Твоя сохраню, не остави мене до зела. В чесом исправит юныи путь свой, внегда сохранити словеса Твоя. Всем сердцем моим взысках Тебе, не отрини мене от заповедей Твоих. В сердцы моем скрых словеса Твоя, да не согрешу Тебе. Благословен еси Господи, научи мя оправданием Твоим. Устнама моима возвестих вся судьбы уст Твоих. На пути свидении Твоих насладихся, яко о всяком богатьстве. В заповедех Твоих поглумлюся, и разумею пути Твоя. Во оправданиих Твоих поучуся, не забуду словес Твоих. Воздаждь рабу Твоему, живи мя, и сохраню словеса Твоя. Открый очи мои, и разумею чудеса от закона Твоего. Пришлец аз есмь на земли, не скрый от мене заповеди Твоя. Возлюби душа моя вожделети судьбы Твоя на всяко время. Запретил еси гордым, прокляти уклоняющиися от заповедей Твоих. Отъими от мене понос и уничижение, яко свидении Твоих взысках. Ибо седоша князи, и на мя клеветаху, раб же Твой глумляшеся во оправданиих Твоих. Ибо свидения Твоя поучение мое есть, и совети мои оправдания Твоя. Прильпе земли душа моя, живи мя по словеси Твоему. Пути моя исповедах, и услыша мя, научи мя оправданием Твоим. Пути оправдании Твоих вразуми мя, и поглумлюся в чудесех Твоих. Воздрема душа моя от уныния, утверди мя в словесех Твоих. Путь неправды отстави от мене, и законом Твоим помилуй мя. Путь истинныи изволих, и судьбы Твоя не забых. Прилепихся свидении Твоих, Господи не посрами мене. Путь заповедей Твоих текох, егда разширил еси сердце мое. Законоположи мне Господи, путь оправдании Твоих, и взыщу и выну. Вразуми мя, и испытаю закон Твой, и сохраню и всем сердцем моим. Настави мя на путь заповедей Твоих, яко той восхотех. Приклони сердце мое во свидения Твоя, а не в лихоимство. Отврати очи мои не видети суеты, в пути Твоем живи мя. Постави рабу Твоему слово Твое в страх Твой. Отъими поношение мое, еже непщевах, яко будьбы Твоя благи. Се вожделех заповеди Твоя, в правде Твоей живи мя. И да приидет на мя милость Твоя Господи, спасение Твое по словеси Твоему. И отвещаю поношающим ми слово, яко уповах на словеса Твоя. И не отъими от уст моих словесе истинна до зела, яко на судьбы Твоя уповах. И сохраню закон Твой всегда: в век и в век века. И хождах в широте, яко заповеди Твоя взысках. И глаголах о свидениих Твоих пред цари и не стыдяхся. И поучахся в заповедех Твоих, их же возлюбих зело. И воздвигох руце мои к заповедем Твоим их же возлюбих, и глумляхся во оправданиих Твоих. Помяни словес Твоих рабу Твоему, их же упование дал ми еси. То мя утеши во смирении моем, яко слово Твое живи мя. Гордии законопреступноваху зело, от закона же Твоего не уклонихся. Помянух судьбы Твоя от века Господи, и утешихся. Печаль прият мя от грешник, оставляющих закон Твой. Пета бяху мне оправдания Твоя, на месте пришельствия моего. Помянух в нощи имя Твое Господи, и сохраних закон Твой. Си бысть мне, яко оправдании Твоих взысках. Часть моя еси Господи, рех: сохранити закон Твой. Помолихся лицу Твоему всем сердцем моим, помилуй мя по словеси Твоему. Помыслих пути Твоя, и возвратих нозе мои, во свидения Твоя. Уготовихся и не смутихся сохранити заповеди Твоя. Южа грешник обязашася мне, и закона Твоего не забых. Полунощи востах, исповедатися Тебе на судьбы правды Твоея. Прчастник аз есмь всем боящимся Тебе, и хранящим заповеди Твоя. Милости Твоея Господи, исполнь земля, оправданием Твоим научи мя. Благость сотворил еси с рабом Твоим Господи, по словеси Твоему. Благости и наказанию и разуму научи мя, яко заповедем Твоим веровах. Прежде даже не смирихся, аз прегреших, сего ради слово Твое сохраних. Благ еси Ты Господи, и благостию Твоею научи мя оправданием Твоим. Умножися на мя неправда гордых, аз же всем сердцем моим испытаю заповеди Твоя. Усырися яко млеко сердце их, аз же закону Твоему поучихся. Благо мне, яко смирил мя еси, да научуся оправданием Твоим. Благ мне закон уст Твоих, паче тысящ злата и сребра.


\medskipСлава, и ныне. Аллилуия, Аллилуия, слава Тебе Боже (трижды). Господи помилуй (трижды). Слава, и ныне.


\medskipРуце Твои сотвористе мя, и создасте мя, вразуми мя, и испытаю заповеди Твоя. Боящиися Тебе узрят мя и возвеселятся, яко на словеса Твоя уповах. Разумех Господи, яко правда повеления Твоя, и воистину смирил мя еси. Буди же милость Твоя, да утешит мя по словеси Твоему раба Твоего. Да приидут мне щедроты Твоя и жив буду, яко закон Твой поучение мое есть. Да постыдятся гордии, яко без правды беззаконноваша на мя, аз же поучуся в заповедех Твоих. Да обратят мя боящиися Тебе, и ведящии свидения Твоя. Буди сердце мое непорочно во оправданиих Твоих, яко да ся не постыжу. Исчезает во спасение Твое душа моя, в слово Твое уповах. Исчезоша очи мои в слово Твое глаголюще, когда утешиши мя; Яко бых яко мех на слане, оправдании Твоих не забых. Колико есть дний раба Твоего, когда сотвориши от гонящих мя суд; Поведаша мне законопреступницы глумления, но не яко закон Твой, Господи. Вся заповеди Твоя истина, без правды погнаша мя помози ми. Мала нескончаша мене на земли, аз же не оставих заповеди Твоя. По милости Твоей живи мя, и сохраню свидения уст Твоих. Во веки Господи, слово Твое пребывает на небеси. В род и род истина Твоя. Основал еси землю и пребывает. Учинением Твоим пребывает день, яко всяческая работна Тебе. Яко аще не закон Твой поучение мое есть, тогда убо погибл бых во смирении моем. Во веки не забуду оправдании Твоих, яко в них живил мя еси.

Твой есмь аз спаси мя, яко оправдании Твоих взысках. Мене ждаша грешницы, погубити мя. Свидения Твоя разумех. Всякия кончины видех конец, широка заповедь Твоя зело. Коль возлюбих закон Твой Господи, весь день поучение мое есть. Паче враг моих умудрил мя еси заповедию Твоею, яко в век моя есть. Паче всех учащих мя разумех, яко свидения Твоя поучение мое есть. Паче старец разумех, яко заповеди Твоя взысках. От всякаго пути лукава, возбраних ногам моим, яко да сохраню словеса Твоя. От судеб Твоих не удалихся, яко Ты законоположил ми еси. Коль сладка гортани моему словеса Твоя; паче меда устом моим. От заповедей Твоих разумех, сего ради возненавидех всяк путь неправды. Светильник ногам моим закон Твой, и свет стезям моим. Кляхся и поставих, сохранити судьбы правды Твоея. Смирихся до зела Господи, живи мя по словеси Твоему. Вольная уст моих благоволи же Господи, и судьбам Твоим научи мя. Душа моя в руку Твоею выну, и закона Твоего не забых. Положиша грешницы сеть мне, и от заповедей Твоих не заблудих. Наследовах свидения Твоя в век, яко радование сердца моего суть. Обратих сердце мое сотворити оправдания Твоя, в век за воздаяние. Законопреступныя возненавидех, закон же Твой возлюбих. Помощник мой и заступник мой еси Ты, на словеса Твоя уповах. Уклонитеся от мене лукавнующии, и испытаю заповеди Бога моего. Заступи мя по словеси Твоему, и жив буду, и не посрами мене от чаяния моего. Помози ми и спасуся, и поучуся во оправданиих Твоих выну. Уничижил еси вся отступающия от оправдании Твоих, яко неправедно помышление их. Преступающия непщевах вся грешныя земли, сего ради возлюбих свидения Твоя. Пригвозди от страха Твоего плоти моя, от судеб бо Твоих убояхся. Сотворих суд и правду, не предаждь мене обидящим мя. Восприими раба Твоего во благо, да не оклеветают мене гордии. Очи мои исчезосте во спасение Твое, и в слово правды Твоея. Сотвори с рабом Твоим по милости Твоей, и судьбам Твоим научи мя. Раб Твой есмь аз, вразуми мя, и научуся свидением Твоим. Время сотворити Господеви, разориша закон Твой. Сего ради возлюбих заповеди Твоя, паче злата и топазия. Сего ради ко всем заповедем Твоим направляхся, всяк путь неправды возненавидех. Дивна свидения Твоя, сего ради испытает я душа моя. Явление словес Твоих просвещает и вразумляет младенца. Уста моя отверзох и превлекох дух, яко заповеди Твоя желах. 

Слава, и ныне. Аллилуия, Аллилуия, слава Тебе Боже (трижды). Господи помилуй (трижды). Слава, и ныне.


\medskipПризри на мя и помилуй мя, по суду любящих имя Твое. Стопы моя направи по словеси Твоему, и да не одолеет ми всяко беззаконие. Избави мя от клеветы человеческия, и сохраню заповеди Твоя. Лице Твое просвети на раба Твоего, и научи мя оправданием Твоим. Исходища водная изведосте очи мои, понеже не сохраних закон Твой. Праведен еси Господи, и прави суди Твои. Заповеда правду свидения Твоя, и истину зело. Истаяла мя есть жалость Твоя, яко забыша словеса Твоя врази мои. Разжжено слово Твое зело, и раб Твой возлюби е. Юноша аз есмь и уничижен, оправдании Твоих не забых. Правда Твоя правда в век, и закон Твой истина. Скорби и нужды обретоша мя, заповеди Твоя поучение мое. Правда свидения Твоя в век, вразуми мя и жив буду. Возвах всем сердцем моим, услыши мя Господи, оправдании Твоих взыщу. Возвах Ти, спаси мя и сохраню свидения Твоя. Предварих в безгодии и возвах, на словеса Твоя уповах. Предваристе очи мои ко утру, поучитися словесем Твоим. Глас мой услыши Господи, по милости Твоей и по судбе Твоей живи мя. Приближиша гонящии мя беззаконие, от закона же Твоего удалишася. Близ еси Ты Господи, и вси путие Твои истина. Исперва познах от свидении Твоих, яко в век основал я еси. Виждь смирение мое, и изми мя, яко закона Твоего не забых. Суди суд мой, и избави мя, по словеси Твоему живи мя. Далече от грешник спасение, яко оправдании Твоих не взыскаша. Щедроты Твоя многи Господи, по судбе Твоей живи мя. Мнози изгонящии мя и стужающии ми, от свидении Твоих не уклонихся. Видех неразумевающия, и истаях, яко словес Твоих не сохраниша. Виждь, яко заповеди Твоя возлюбих Господи, по милости Твоей живи мя. Начало словес Твоих истина, и в век вся судбы правды Твоея. Князи погнаша мя туне, и от словес Твоих устрашися сердце мое. Возрадуюся аз о словесех Твоих, яко обретаяи корысть многу. Неправду возненавидех, и омерзе ми, закон же Твой возлюбих. Седмицею днем хвалих Тя, о судьбах правды Твоея. Мир мног любящим закон Твой, и несть им соблазны. Чаях спасение Твое Господи, и заповеди Твоя возлюбих. Сохрани душа моя свидения Твоя, и возлюби я зело. Сохраних заповеди Твоя, и свидения Твоя, яко вси путие мои пред Тобою, Господи. Да приближится молитва моя пред Тя Господи, по словеси Твоему вразуми мя. Да внидет прошение мое пред Тя Господи, по словеси Твоему избави мя. Отрыгнут устне мои пение, егда научиши мя оправданием Твоим. Провещает язык мой словеса Твоя, яко вся заповеди Твоя правда. Буди рука Твоя спасти мя, яко заповеди Твоя изволих. Вожделех спасение Твое Господи, и закон Твой поучение мое есть. Жива будет душа моя и восхвалит Тя, и судьбы Твоя помогут мне. Заблудих, яко овча погибшее, взыщи раба Твоего, яко заповеди Твоя не забых. 


Слава, и ныне, без Аллилуии. 

\itshape Таже, исповедание православныя веры.\normalfont{} Верую во единаго Бога. Трисвятое. И по Отче наш. 


\mysubsubsection{Тропари сия, глас 8:}


Се Жених грядет в полунощи, и блажен раб, его же обрящет бдяща. Недостоин же паки, его же обрящет ленящася. Блюди убо душе моя, да не сном отяготишися и да не смерти предана будеши, и царствия вне затворишися, но воспряни зовущи: свят, свят, свят еси, Боже, Богородицы ради, помилуй нас. 

Слава. День он страшныи помышляющи, душе моя побди, вжигающи свещу свою, и маслом просвещающи, не веси бо когда приидет к тебе глас глаголющии: се Жених. Блюди убо, душе моя, да не воздремлеши, и пребудеши вне толкущи, яко пять дев; но бодрено побди, яко да усрящеши Христа милостива, и даст ти чертог божественыя славы Своея. 

И ныне. Тебе необоримую стену имуще вернии, и спасения утвержение, Богородице Дево молим: сопротивных советы разори и людей Своих печаль на радость преложи. Мир Свой умири и православных утверди, и о смирении мира молися, яко Ты еси Богородице, упование наше. Господи помилуй (40). 


\medskip\itshape Таже молитва Великаго Василия:\normalfont{} Иже на всяко время, и на всяк час, на небеси и на земли, покланяемыи и славимыи, Боже Благии, долготерпеливе и многомилостиве; Иже праведники любя и грешных милуя; Иже всех зовыи ко спасению, обещания ради будущих благ; сам Господи, приими молитвы наша в час сий, и в благости исправи живот наш, к заповедем Твоим. Душа наша освяти, и телеса очисти, помышления исправи, и мысли очисти, разум уцеломудри и истрезви, и избави нас от всякия скорби, зол и болезней, и душевных страстей, и огради нас святыми ангелы Твоими, яко да ополчением их соблюдаеми и наставляеми, достигнем во единство веры, и в разум неприкосновенныя Ти славы, яко благословен еси во веки, аминь. 

\itshape Таже, Господи помилуй\normalfont{} (трижды). Слава, и ныне. 

Честнейшую херувим, и славнейшую воистину серафим, без истления Бога Слова рождьшую, сущую Богородицу Тя величаем (поклон). Именем Господним благослови, отче. За молитв святых отец наших, Господи Исусе Христе, Сыне Божии, помилуй нас. Аминь. 

\itshape Аще есть пост, или кроме поста, егда поем Аллилуия, и творим три поклоны великия, глаголюще молитву святаго Ефрема\normalfont{}. Господи и Владыко животу моему, дух уныния, небрежения, сребролюбия и празднословия, отжени от мене (поклон).Дух же целомудрия, смирения, терпения и любве, даруй ми рабу Твоему (поклон).Ей, Господи Царю, даждь ми зрети моя согрешения, и еже не осуждати брата моего, яко благословен еси во веки, аминь (поклон).

\itshape И прочих поклонов 12, глаголюще в себе:\normalfont{} Господи Исусе Христе, Сыне Божии, помилуй мя грешнаго (дважды с поклонами).

Боже, милостив буди мне грешному (поклон).

Боже, очисти грехи моя и помилуй мя (поклон).

Создавыи мя Господи, помилуй (поклон).

Без числа согреших Господи, прости мя (поклон).

\itshape И паки скончав поклоны, глаголем молитву всю выше писаную:\normalfont{} Господи, и Владыко животу моему… И поклон един. 


И по изглаголании молитвы, или по Честнейшую херувим, глаголем сию молитву Великаго Василия: Владыко Боже, Отче Вседержителю, и Господи Сыне Единородныи Исусе Христе, и Святыи Душе, едино Божество, и едина сила, помилуй мя грешнаго, и ими же веси судьбами, спаси мя недостойнаго раба Твоего, яко благословен еси во веки, аминь. 


Таже, Приидите, поклонимся (трижды). И поклоны три. 


\mysubsubsection{Псалом 120}


Возведох очи мои в горы, отнюдуже приидет помощь моя. Помощь моя от Господа, сотворшаго небо и землю. Не даждь во смятение ноги твоея, ниже воздремлет храняи тя. Се не воздремлет, ни уснет храняи Израиля. Господь сохранит тя, Господь покров Твой на руку десную твою. Во дни солнце не ожжет тебе, ни луна нощию, Господь сохранит тя от всякаго зла, сохранит душу твою Господь. Господь сохранит вхождение твое, и исхождение твое от ныне и до века.


\mysubsubsection{Псалом 133}


Се ныне благословите Господа вси раби Господни, стоящии в храме Господни, во дворех дому Бога нашего. В нощех воздежите руки ваша во святая, и благословите Господа. Благословит тя Господь от Сиона, сотворивыи небо и землю. 

Слава, и ныне. Трисвятое, и поклоны три. По Отче наш, тропари сия: Помяни Господи, яко благ, рабы Своя, и елика в житии сем согрешиша, прости. Никтоже бо без греха, токмо Ты могии преставленым дати покой. Иже глубинами мудрости, человеколюбне вся строя, и еже на пользу всем подавая, едине Содетелю, покой Господи, душа усопших раб Своих, на Тя бо упование возложиша, Творца и Зиждителя и Бога нашего. 

Слава. Со святыми покой, Христе, душа раб Своих, идеже несть болезни, ни печали, ни воздыхания, но жизнь вечная. 

И ныне, богородичен: Блажим, Тя вси роди, Богородице Дево, в Тя бо невместимыи Христос Бог наш, вместитися изволи, блажени есмы и мы помощницу Тя имуще, день бо и нощь молишися о нас, и державы царствия Твоими молитвами утвержаются. Тем благодаряще вопием Ти: радуися, Обрадованная, Господь с Тобою. 

Господи помилуй (12).


\itshape Таже, молитву сию:\normalfont{} Помяни Господи, иже в надежи воскресения, и жизни вечныя, усопшия отцы и братию нашу, и всех иже во благочестивей вере скончавшихся, и прости им всяко согрешение, вольное же и невольное, словом и делом, и помышлением согрешенное ими и всели их в места светла, в места прохладна, в места покойна, отнюдуже отбеже всяка болезнь, и печаль и воздыхание, идеже присещает свет лица Твоего, и веселит вся иже от века святыя Твоя. И даруй им царствие Твое, и причастие неизреченных и вечных Твоих благ, и Твоея безконечныя, и блаженныя жизни наслаждение. Ты бо еси воскресение и живот и покой усопшим рабом Твоим, Христе Боже наш, и Тебе славу возсылаем, со безначальным Ти Отцем, и с Пресвятым и Благим и Животворящим Ти Духом, ныне и присно и во веки веком, аминь. 

\itshape Таже, Слава, и ныне. Господи помилуй (дважды). Господи благослови. 

И отпуст:\normalfont{} Господи Исусе Христе, Сыне Божии, молитв ради Пречистыя Твоея Матере, преподобных и богоносных отец наших и всех ради святых, помилуй и спаси нас, яко Благ и Человеколюбец, аминь. 

\itshape И прощение:\normalfont{} Ослаби, остави, отпусти Боже, согрешения наша, вольная и невольная, яже в слове и в деле, и яже в ведении и не в ведении, яже во уме и в помышлении, яже во дни в нощи, вся ми прости, яко Благ и Человеколюбец, аминь.


\itshape И молитва с поклонами:\normalfont{} Ненавидящих и обидящих нас прости, Господи Человеколюбче. Благотоворящим благо сотвори, братиям и всем сродником нашим, иже и уединившимся, даруй им вся, яже ко спасению прошения и живот вечныи (поклон). 

В болезнех сущия посети и исцели, в темницах сущия свободи, по водам плавающим Правитель буди и иже в путех шествующим исправи и поспеши (поклон). 

Помяни Господи, и плененныя братию нашу, единоверных православныя веры, и избави их всякаго злаго обстояния (поклон). 

Помилуй Господи, давших нам милостыню и заповедавших нам, недостойным, молитися о них, прости их и помилуй (поклон). 

Помяни Господи, прежде отшедшия отцы и братию нашу и всели их, идеже присещает свет лица Твоего (поклон). 

Помяни Господи, и нашу худость и убожество, и просвети наш ум светом разума святаго Евангелия Твоего, и настави нас на стезю заповедей Твоих, молитвами Пречистыя Твоея Матере и всех святых Твоих, аминь (поклон). 

Господи помилуй (трижды),

и обычные исходные поклоны.
\mychapterending
