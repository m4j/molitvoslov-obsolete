

\mypart{МОЛИТВЫ О ЗАКЛЮЧЕННЫХ}
%http://www.molitvoslov.com/content/za-zakluchennih

 

\mychapter{Молитва заключенного к святому Иоанну Предтече}
%http://www.molitvoslov.com/text512.htm 
 


Предтече и Крестителю Христов Иоанне, проповедниче покаяния! Ты неповинне всажден был еси в темницу: аз же, вверженный в злоключение сие, достойное по делом моим приемлю, яко преступник правды и закона. Всели в мое сердце чувство покаяния о гресех моих! Несть бо ни единыя злобы ни беззакония, ихже аз, окаянный, не содеях; престрашни греси мои. Учителю правды! научи мя право глаголати о мне самом пред судиями. Не преставаяй и в темнице обличати беззаконнаго Ирода, даруй ми, да наипаче зде обличает мене совесть моя, да от обличении ея не возмогу на долзе времени утаити мое преступление. Аще же осужден буду понести наказание, даруй ми быти терпеливу, якоже ты сам терпеливно несл еси усекновение главы твоея, желанное от Иродиады. Ей, Крестителю Христов! Простри ми, рабу твоему, руку, крестившую Христа Спасителя моего, да мя извлечеши из глубины погибели. Ты еси больший всех в рожденных женами, ты еси первый по Богородице, праведник между человеки. Сего ради прибегаю к тебе аз, имеяй потребу в велицем ходатае, яко велик есмь грешник. Убо и да осенит мене, недостойнаго, благодать твоя, Предтече Господень. 


\mychapterending

\mychapter{Преподобному Петру Афонскому}
%http://www.molitvoslov.com/text515.htm 
 
\myfig{img/380.jpg}

\bfseries Тропарь:\normalfont{}


Мир оставил еси, Петре, Господа ради своего и крест восприим на рамо свое, на гору достигл еси Афонскую, якоже древле фесвитянин Илия. И в ней пребыв пощением, и молитвами, и бдением, Бога милостива творя, и Пречистую Богородицу Молебницу предлагая, сего ради вопием ти: моли Христа Бога, да спасет души наша.


\medskip


\bfseries Кондак:\normalfont{}


Удалив себе человеческаго сожительства в пещерах каменных и разселинах пожил еси желанием Божественным и любовию, Петре, Господа твоего, от Негоже венец приял еси. Моли непрестано спастися нам.


\mychapterending

\mychapter{Молитва великомученице Анастасии, именуемой &quot;Узорешительница&quot;}
%http://www.molitvoslov.com/text514.htm 
 
\myfig{img/379.jpg}


О многострадальная и премудрая великомученице Христова Анастасие! Ты душею на небеси у Престола Господня предстоиши, на земли же, данною тебе благодатию, различная совершаеши исцеления: призри убо милостивно на предстоящия люди  и молящиеся  пред мощами твоими, просящия твоей помощи: простри ко Господу святыя молитвы твоя о нас, и испроси нам оставление согрешений наших, недужным исцеление, скорбящим и бедствующим скорую помощь: умоли Господа, да подаст всем нам христианскую кончину и добрый ответ на Страшнем Суде Своем, да сподобимся и мы купно с тобою славити Отца и Сына и Святаго Духа во веки веков. Аминь. 





\mychapterending

\mychapter{Молитва заключенного к Ангелу хранителю}
%http://www.molitvoslov.com/text513.htm 
 



Ангеле Божий, предстояй ми от купели святаго крещения! Колико аз, преогорчевах тебе злыми помыслы, скверными словесы и постыдными деянями в житии моем! Ктому же, не памятовах, яко имам тя даннаго ми от Бога хранителя. Егда аз упивахся вином, или безстудно плясах, или в блуде, хищениях и злоприобретениях иждивах житие мое: и в сии дни ты не отлучался еси от мене, но токмо, яко чистый дух, печален был еси о мне, непотребнем рабе. Коль множицею аз бых близ самыя смерти, творяй преступныя дела и отнюдь не помышляяй предстати внезапу Богу-Судии! Смерть же не постиже мене тогда: и вем, яко ты отвратил еси ю от мене. Тако выну охраняеши мя, и не един аз есмь, внегда един жительствую во узилищи сем, но с тобою, хранителю мой. Аще и не вижду тебе телесными очима, ты еси присно спребываяй окаянной моей душе и телу. Якоже солнце, преходя скверныя места, не оскверняется, сице и ты, светоносный Ангеле Божий, не возгнушайся моея смрадности, но николиже отлучайся от мене и в сей темнице. Сохрани мя здрава нощию и неприступна во сне диавольскому искушению. Отжени от мене безсоние. Сопутствуй ми, во дни, на кийждо час, во всех входех и исходех моих. Даруй ми крепость телесных сил, еже понести горькия работы, в наказание мне налагаемыя, горькия же паче по вине их. Заступи мя от злобы бесов, люте нападающих на мою душу: ибо от единаго мановения десницы твоея бежат беси. Паче же всего отжени от мене смятение душевное и ропот вредный: сый бо смущен сердцем, не хощу ни на кого и ни на что благоприятно взирати и присно зде огорчеваюся. Даруй очам моим слезы о гресех, устам же тихое глаголание. Не ступи от мене, святый Ангеле, и егда душа моя начнет разлучатися от тела, но предстани ми у смертнаго моего одра тих и радостен; ибо имам тогда видети тебе близ суща. Не остави и послежде мою душу, яже ти от Бога предана бысть непорочне, но яже по смерти моея имать судитися, яко оскверненная многими грехми. О, хранителю мой! Призови тогда помолитися о мне и прочия безплотныя силы, да твоим заступлением помилован буду, грешный раб (\itshape имя\normalfont{}), во веки веков, аминь! 





\mychapterending

\mychapter{Молитва заключенного к Божией Матери}
%http://www.molitvoslov.com/text511.htm 
 
\myfig{img/376.jpg}

О, Пресвятая Дево, Мати Господа и Спасителя моего! Вси человецы притекают к Тебе и никто притекаяй отходит от Тебе всуе, аще о добре просит. Молят Тя архиерее и священницы, яко Матерь еси Архиерея Великаго, прошедшаго небеса. Молят Тя иноки и инокини, яко Сама еси Пречистая Дева. Молят Тя отцы и матери, иже соболезнуют о своих детях, в немощех лежащих или далече живущих, яко Ты Сама болезнующи искала Сына своего, егда Он, дванадесятилетний отрок, оста во Иерусалиме. Молят Тя обидимии, яко претерпела еси гонение от Ирода с Божественным Младенцем. Возносят к Тебе последнюю свою молитву умирающии, яко Ты предстояла еси у Креста умирающаго Сына Твоего. Молят Тя праведнии и вси хотящии благочестно жити: и никтоже от них достиже праведности, не имеяй Тебе молитвенницу. Тако вси имут Тя заступницу усердну и всем подаеши скору помощь.


 Един ли аз, грешник, в темнице седяй, забвен буду от Тебе? Мене ли единаго молитву не услышиши? Мене ли, прибегающаго к Тебе, не приимеши под кров Твой? О, никакоже помышляю сему быти. Понеже Матерь еси Спасителя погибающих грешников. Понеже яко несть иного Бога разве Иисуса Христа, Сына Твоего, со Отцем и Святым Духом во Едином Божестве покланяемаго. Тако несть инаго у Престола Божия ближайшаго ходатая, якоже Ты, Мати Божия! Матернее моление много может ко благосердию Владыки. Пред Ним Тебе вся возможна суть. Сего ради на Тя по Бозе все упование мое возлагаю. Сего ради храню аз пречистый образ Твой во узилищи моем, памятуяй близость Твою ко всем скорбящим и заключенным. 


Обаче Ты, Матерь Божия, и являла еси доныне ко мне многая милости по моим прошением к Тебе! Чимже аз явихся пред Тобою? Увы мне злонравному! Аз же непамятлив бых Твоих благодеяний. Но, о милосердая Царице Богородице, прости моя измены и неразумие, не отвергни мене, паки возвращающагося к Тебе, яви ко мне древняя Твоя милости и приими мя на Свои мощныи руки, аки боляща, аки еле жива суща от язв греховных. Приведи мою душу грешную к покаянию. Беззакония бо моя превзыдоша главу мою. Колико доселе прогневах аз Бога моего лжами, хищениями, пиянством, крамолами, жестокосердием, безсовестием. Безпрестани согрешаю и в сей темнице леностию, мечтаниями, обманами, непокорством, досаждениями соузникам моим и развращением их, татьбами и роптанием на мою участь. Ктому не бысть у мене доныне ни слез, ни умиления, или же аз плаках токмо от досады моея. 


Примири убо мене, Благодатная, с Богом, Сыном Твоим. Сподоби мя получити разрешение во всех гресех моих в тайне покаяния, и абие причаститися Святаго Тела и Крове Сына Твоего. Помози ми отныне возненавидети злыя мои страсти и крепко воинствовати противу греха; зане аще в чесом и покаюся, по малом часе таяжде творю, и сице лжив пред Богом обретаюся. Испроси ми, Владычице Богородице, благодать прочее время живота моего в твердом покаянии скончати. Милующая грешники, не презри мене перваго из них. Зриши моя беды, зриши мою горьку жизнь в сей темнице. Помози ми претерпети до конца позор мой, тесноту и прочия темничныя лишения, да не впаду в отчаяние, да не дерзну возносити хулы на Промысл Божий, но токмо скорблю и плачу о гресех моих, токмо от печали моея о чадах моих и сродниках, от нихже разлучен есмь, обращаюся ко утешению в молитве. Несть у меня терпения и кийждо день мню, яко наста конец моему терпению. О, печалъных недоведомое утешение! Утеши моего отца и матерь со сродники, иже скорбят о мне, для нихже аз содеяхся яко мертв. Наипаче же, о чадолюбивая Мати! не лиши Твоего покровительства жену и чада моя, ихже оставих аз, яко преступный отец, зане они неповинни суть во гресех моих. Препитай их и сохрани Твоею молитвою здравы и богобоязненны. Даруй ми, напоследок, видети свободу и со слезами радости облобызати моих домашних. Из глубины души моея вопию к Тебе, Царице Небесная, от одного образа Твоего с предвечным Младенцем на руку Твоею держимым чувствуяй в сердце моем отраду: не отрини мене, буди ми покровительница во всяких моих нуждах, во всяком моем страхе и унынии, днем и нощию, во здравии и болезни. Облегчи моя душевныя и телесныя скорби, яко Ты еси всех скорбящих радость. Милостива буди ми не зде токмо, но и тамо, в стране вечней. Не остави мене без Твоего заступления в день Страшнаго Втораго Пришествия Господа моего. Ей, Пресвятая Богородице! Не забуди тогда моея веры, яже к Тебе, и недостойных моих молитв пред Тобою. Да тако, быв на вечныя времена помилован Тобою, в безконечныя же веки буду благодарити Тебе и славословити Отца и Сына и Святаго Духа, аминь. 





\mychapterending

\mychapter{Молитва узника в темнице заключеннаго}
%http://www.molitvoslov.com/text507.htm 
 



\itshape (Творение святителя Филарета Московского)


\normalfont{}


Господи Боже, Создателю и Спасителю мой! Благословенно да будет имя Твое Святое. Благодарение и слава Тебе, Господи, о всех благих, яже прият от Тебе в житии сем. Ныне же скорбь и болезнь обретох, и Имя Твое призываю. Поношения нападоша на мя. Положиша мя в рове преисподнем, в темных и сени смертней. Скорблю о сем; и по сей скорби разумеваю, яко согреших пред Тобою, и по грехом моим приидоша на мя беды;  ибо праведники Твои не унывающе и в темницах пояху Тебе, и во страданиих радовахуся. И аще беззакония назриши, Господи, Господи, кто постоит?  Яко несть человек, иже не согреши. Но Ты, Господи, грехи всего мира носиши, и покаянием очищаеши. Верую, яко и мене грешнаго не отвержеши от лица Твоего. За весь мир Единородный Сын Твой излия Свою Божественную Кровь. Верую, яко и мене от грехов моих омыти может и хощет. Сего ради с Давидом глаголю: исповем на мя беззаконие мое. Ты же, яко Благ, остави нечестие сердца моего. Страшуся суда и осуждения человеческого: но наипаче да будет мне в сердце страх Твой неумытный суд и вечное осуждение. Аще неправда возстанет на мя, дерзаю словом Давидовым молитися Тебе, услыши, Господи, правду мою и вонми суду моему, и правдою Твоею избави мя. Аще же неправду сотворих: милосердием Твоим неправду мою уврачуй. Не попусти уклонитися сердцу моему во словеса лукавствия, к сокрытию истины, и к ложному оправданию. Помози мне и уразумети, и возненавидети неправду мою: возлюбити же правду, и во истине обрести облегчение души моей. Облегчи бремя бедствия моего. А еже понести мне суждено, да понесу с терпением, ради очищения грехов моих, и ради умилостивления Твоего правосудия. Аще и стыд покрыет мене пред некими человеками: да потерплю со смирением, да умилостивлю Тебе, Господи, да не постыжен буду пред лицем сего мира на Страшном Суде Твоем.


Прихожду к Тебе скорбный и печальный: не лиши мене духовного утешения. 


Прихожду к Тебе омраченный: яви свет упования спасения. 


Припадаю к Тебе изнемогший: возстави и утверди мене благодатию Твоею. 


Паче всего даруй мне желание, и помози мне, Господи, творити во всем Твою волю, да в мире совести прославлю Имя Твое Святое, Отца и Сына и Святаго Духа. Аминь.


\mychapterending

\mychapter{Молитва о в узах сущих}
%http://www.molitvoslov.com/text506.htm 
 



Господи Иисусе Христе Боже наш, святаго апостола Твоего Петра от уз и темницы без всякаго вреда свободивый, приими, смиренно молим Ти ся, жертву сию милостивно во оставление грехов рабов Твоих (раба Твоего/рабы Твоея)  (\itshape имярек\normalfont{}) в темницу всажденных  (всажденнаго), и молитвами их (его), яко Человеколюбец, всесильною Твоею Десницею от всякаго злаго обстояния избави и на свободу изведи.





\mychapterending