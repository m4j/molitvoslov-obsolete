

\mypart{МОЛИТВЫ ОБ ИСЦЕЛЕНИИ ДУШЕВНЫХ И ДУХОВНЫХ НЕДУГОВ}\label{_content_obiscelenii-nedugov}
%http://www.molitvoslov.com/content/obiscelenii-nedugov

 

\mychapter{При обуревании души неверием}
%http://www.molitvoslov.com/content/pri-oburevanii-dushi-neveriem

 

\section{Первоверховному апостолу Павлу}
%http://www.molitvoslov.com/text745.htm 
 


\mysubsubsection{Тропарь, глас 4-й:}


Апостолов первопрестольницы, и вселенныя учителие, Владыку всех молите; мир вселенней даровати, и душам нашим велию милость.


\mysubsubsection{Кондак, глас 2-й:}


Твердыя и боговещанныя проповедатели, верх апостолов Твоих, Господи, приял еси в наслаждение благих Твоих и покой; болезни бо онех и смерть приял еси паче всякаго всеплодия, Едине сведый сердечная.


\mysubsubsection{Молитва:}


О, святый верховный апостоле Павле, сосуде избранный Христов, небесных таин сказателю, всех языков учителю, церковная трубо, пресловущий ветие, многия беды за имя Христово претерпевый, море измеривый и землю обшедый и нас от лести идольский обративый! Тя молю и к тебе вопию: не гнушайся мене сквернаго, возстави падшаго греховною леностию, якоже в Листрех хромаго от чрева матерня возставил еси; и якоже Евтиха мертва бывша оживил еси, воскреси и мене от мертвых дел; и якоже молитвою твоею основание темницы некогда потрясл еси и узники разрешил еси, сице исторгни мя творити волю Божию. Вся бо можеши данною ти властию от Христа Бога, Емуже подобает всякая слава, честь и поклонение, со Безначальным Его Отцем, и со Пресвятым и Благим и Животворящим Его Духом, ныне и присно и во веки веков. Аминь.


\section{Апостолу Фоме}
%http://www.molitvoslov.com/text744.htm 
 


\mysubsubsection{Тропарь, глас 2-й:}


Мученик Христов быв, Божественнаго собора апостольскаго сопричастник, неверствием бо Христово Воскресение известив и Того пречистую страсть осязанием уверив, Фомо всехвальне, и ныне нам проси мира и велия милости.


\mysubsubsection{Кондак, глас 4-й:}


Премудрости благодати исполнен, Христов апостол и служитель истинный, в покаянии вопияше Тебе: Ты мой еси Бог же и Господь.


\mysubsubsection{Молитва:}


О, святый апостоле Фомо! Молим тя: сохрани и соблюди нас молитвами твоими от искушений диавольских и падений греховных, и испроси нам, рабам Божиим (\itshape имена\normalfont{}), свыше помощь во время неверия, да не преткнемся о камень соблазна, но неуклонно шествуем спасительным путем заповедей Христовых, дондеже достигнем оных блаженных обителей райских. Ей, апостоле Спасов! Не посрами нас, но буди нам помощник и покровитель во всем житии нашем и помози нам благочестно и богоугодно житие сие временное скончати, христианскую кончину получити и добраго ответа сподобитися на Страшнем Суде Христове; да прославим великолепое имя Отца, и Сына, и Святаго Духа во веки веков. Аминь.
\mychapterending

\mychapter{В отчаянии}
%http://www.molitvoslov.com/content/%D0%B2-%D0%BE%D1%82%D1%87%D0%B0%D0%BD%D0%B8%D0%B8

 

\section{Святителю Иоанну Златоустому, архиепископу Константинопольскому}
%http://www.molitvoslov.com/text747.htm 
 


\mysubsubsection{Тропарь, глас 8-й:}


Уст твоих, якоже светлость огня возсиявши благодать, вселенную просвети: не сребролюбия мирови сокровища сниска, высоту нам смиренномудрия показа. Hо твоими словесы наказуя, отче Иоанне Златоусте, моли Слова Христа Бога спастися душам нашим.


\mysubsubsection{Кондак, глас 6-й:}


От Hебес приял еси Божественную благодать и твоими устнами вся учиши покланятися в Троице Единому Богу, Иоанне Златоусте, всеблаженне, преподобне, достойно хвалим тя: еси бо наставник, яко Божественная являя.


\mysubsubsection{Молитва:}


О, святителю великий Иоанне Златоусте! Многая и различная дарования от Господа приял еси и, яко раб благий и верный, вся данныя тебе таланты добре умножил еси: сего ради воистинну вселенский учитель был еси, яко всяк возраст и всяко звание от тебе поучается. Се бо отроком послушания образ явился еси, юным "--- целомудрия светило, мужем "--- трудолюбия наставник, старым "--- незлобия учитель, иноком "--- воздержания правило, молящимся "--- вождь от Бога вдохновенный, мудрости ищущим "--- ума просветитель, витиям доброглаголивым "--- слова живаго источник неисчерпаемый, благотворящим "--- милосердия звезда, начальствующим "--- правления мудраго образ, правды ревнителем "--- дерзновения вдохновитель, правды ради гонимым "--- терпения наставник: всем вся был еси, да всяко некия спасеши. Над всеми же сими стяжая еси любовь, яже есть соуз совершенства, и тою, яко силою Божественною, вся дарования во единем лице твоем во едино совокупил еси, и туюжде любовь, разделенная примиряющую, в толковании словес апостольских всем верным проповедал еси. Мы же грешнии, по единому кийждо свое дарование имуще, единения духа в союзе мира не имамы, бываем тщеславни, друг друга раздражающе, друг другу завидяще; сего ради дарования наша разделенная не в мир и спасение, но во вражду и осуждение нам приложишася. Темже к тебе, святителю Божий, припадаем, раздором обуреваеми, и в сокрушении сердца просим: молитвами твоими отжени от сердец наших всяку гордость и зависть, нас разделяющия, да во мнозех удех едино тело церковное невозбранно пребудем, да по словеси твоему молитвенному возлюбим друг друга и единомыслием исповемы Отца и Сына и Святаго Духа, Троицу Единосущную и Нераздельную, ныне и присно и во веки веков. Аминь.
\mychapterending

\mychapter{В унынии}
%http://www.molitvoslov.com/content/v-uninii

 

\section{Праведному царю Давиду, псалмопевцу}
%http://www.molitvoslov.com/node/367 
 


\mysubsubsection{Тропарь, глас 2-й:}


Благовествуй, Иосифе, Давиду чудеса Богоотцу: Деву видел еси рождшую, с пастыри славословил еси, с волхвы поклонился еси, Ангелом весть прием. Моли Христа Бога спасти души наша.


\mysubsubsection{Тропарь, глас 2-й:}


Пророка Твоего Давида, Господи, память празднующе, тем Тя молим: спаси души наша.


\mysubsubsection{Кондак, глас 3-й:}


Веселия днесь Давид исполняется Божественный, Иосиф же хваление со Иаковом приносит: венец бо сродством Христовым приемше, радуются, и неизреченно на земли Рождшагося воспевают, и вопиют: Щедре, спасай Тебе чтущия.


\mysubsubsection{Кондак, глас 4-й:}


Просветившееся Духом чистое твое сердце, пророчества бысть светлейшаго приятелище: зриши бо яко настоящая далече сущая: сего ради тя почитаем, пророче блаженне, Давиде славне.


\mysubsubsection{Молитва:}


Помяни, Господи, царя Давида и всю кротость его.


\mysubsubsection{Молитва:}


О, прехвальный и пречудный пророче Божий Давиде! Услыши нас, грешных и непотребных, в час сей предстоящих пред святою твоею иконою и усердно прибегающих к ходатайству твоему. Моли о нас Человеколюбца Бога, да подаст нам дух покаяния и сокрушения о гресех наших и всесильною Своею благодатию да поможет нам оставити пути нечестия, приспевати же во всяком деле блазе, да укрепит нас в борьбе со страстьми и похотьми нашими; да всадит в сердце наше дух смирения и кротости, дух братолюбия и незлобия, дух терпения и целомудрия, дух ревности к славе Божией и спасению ближних. Упраздни молитвами твоими, пророче, злыя обычаи мира, паче же погибельный и тлетворный дух века сего, заражающий христианский род неуважением к Божественней Православней вере, к уставам святыя Церкви и к заповедем Господним, непочтением к родителем и влаcтем предержащим, и низвергающий людей в бездну нечестия, развращения и погибели. Отврати от нас, пречудне пророче, предстательством твоим праведный гнев Божий, и избави вся грады и веси царства нашего от бездождия и глада, от страшных бурь и землетрясений, от смертоносных язв и болезней, от нашествия врагов и междоусобныя брани. Укрепи твоими молитвами православных людей, благопоспешествуй им во всех благих деяниих и начинаниях к водворению мира и правды в державе их. Пособствуй Всероссийкому Христолюбивому воинству во бранех со врагами нашими. Испроси, пророче Божий, от Господа пастырем нашим святую ревность по Бозе, сердечное попечение о спасении пасомых, мудрость в учении и управлении, благочестие и крепость во искушениих, судиям испроси нелицеприятие и безкорыстие, правоту и сострадание к обидимым, всем начальствующим попечение о подчиненных, милость и правосудие, подчиненным же покорность и послушание ко властем и усердное исполнение своих обязанностей; да, тако в мире и благочестии поживше в сем веце, сподобимся причастия вечных благ в Царствии Господа и Спаса нашего Иисуса Христа, Емуже подобает честь и поклонение, со Безначальным Его Отцем и Пресвятым Духом, во веки веков. Аминь.


\section{Преподобному Ефрему Сирину}
%http://www.molitvoslov.com/text782.htm 
 


\mysubsubsection{Тропарь, глас 8-й:}


В тебе, отче, известно спасеся еже по образу: приим бо Крест последовал еси Христу, и дея учил еси презирати убо плоть, преходит бо, прилежати же о души, вещи бессмертней. Темже и со ангелы срадуется, преподобне Ефреме, дух твой.


\mysubsubsection{Тропарь, глас 8-й:}


Слез твоих теченьми пустыни безплодное возделал еси, и иже из глубины воздыханьми во сто трудов уплодоносил еси, и был еси светильник вселенней, сияя чудесы, Ефреме, отче наш, моли Христа Бога спастися душам нашим.


\mysubsubsection{Кондак, глас 2-й:}


Час присно провидя суда, рыдал еси горько, Ефреме, яко любобезмолвный, делателен же был еси в делех учитель, преподобне. Темже, отче всемирный, ленивыя воздвизаеши к покаянию.


\mysubsubsection{Молитва:}


О угодниче Христов, отче наш Ефреме! Принеси молитву нашу к милостивому и всесильному Богу и испроси нам, рабам Божиим (\itshape имена\normalfont{}), у благости Его вся яже на пользу душам и телесем нашим: веру праву, надежду несумненну, любовь нелицемерну, кротость и незлобие, во искушениях мужество, в злостраданиях терпение, во благочестии преспеяние. Да не во зло обратим дары Всеблагого Бога. Не забуди, чудотворче святый, и святый храм (дом) сей и приход наш: сохрани и соблюди их молитвами твоими от всякого зла. Ей, святче Божий, сподоби нас кончину благую улучити и Царствие Небесное унаследити, да прославим дивнаго во святых Своих Бога, Емуже подобает всякая слава, честь и держава, во веки веков. Аминь.


\section{Святителю Тихону, епископу Воронежскому, Задонскому чудотворцу}
%http://www.molitvoslov.com/text749.htm 
 


\mysubsubsection{Тропарь, глас 8-й:}


От юности возлюбил еси Христа, блаженне, образ был еси словом, житием, любовию, духом, верою, чистотою и смирением, темже и вселися в Небесныя обители, идеже предстоя Престолу Пресвятыя Троицы, моли, святителю Тихоне, спастися душам нашим.


\mysubsubsection{Кондак, глас 8-й:}


Апостолов приемниче, святителей украшение, Православныя Церкве учителю, Владыце всех молися мир вселенней даровати и душам нашим велию милость.


\mysubsubsection{Молитва:}


О, всехвальный святителю и угодниче Христов, отче наш Тихоне! Ангельски на земли пожив, ты яко ангел благий явился еси и в дивнем твоем прославлении. Веруем от всея души и помышления, яко ты, благосердный наш помощниче и молитвенниче, твоими неложными ходатайствы и благодатию, от Господа обильно тебе дарованною, присно способствуеши нашему спасению. Приими убо, ублажаемый угодниче Христов, и в час сей наша недостойная моления: свободи ны твоим заступлением от облегающего нас суесловия и суемудрия, неправоверия и зловерия человеческаго. Потщися, скорый о нас предстателю, благоприятным твоим ходатайством умолити Господа, да пробавит Своя великия и богатыя милости на нас, грешных и недостойных рабех Своих, да уврачует Своею благодатию неисцельныя язвы и струпы растленных душ и телес наших, да растворит окаменелая сердца наша слезами умиления и сокрушения о премногих согрешениих наших, и да избавит ны от вечных мук и огня геенскаго: всем же верным людем Своим да дарует в нынешнем веце мир и тишину, здравие и спасение и во всем благое поспешение, да тако, тихое и безмолвное житие поживше во всяком благочестии и чистоте, сподобимся со ангелы и со всеми святыми славити и воспевати Всесвятое имя Отца и Сына и Святаго Духа во веки веков. Аминь.
\mychapterending

\mychapter{О даровании покаяния}
%http://www.molitvoslov.com/content/o-darovanii-pokayaniya

 

\section{Преподобной Марии Египетской}
%http://www.molitvoslov.com/text758.htm 
 
\myfigh{img/62593__8.jpg}{17}

\mysubsubsection{Тропарь, глас 8-й:}


В тебе, мати, известно спасеся еже по образу: приимши бо крест, последовала еси Христу, и деющи учила еси презирати убо плоть, преходит бо, прилежати же о души, вещи безсмертней. Темже и со aнгелы срадуется, преподобная Марие, дух твой.


\mysubsubsection{Кондак, глас 4-й:}


Греха мглы избежавши, покаяния светом озаривши твое сердце, славная, пришла еси ко Христу, Сего всенепорочную и святую Матерь, молитвенницу милостивную принесла еси. Отонудуже и прегрешений обрела еси оставление, и со ангелы присно срадуешися.


\mysubsubsection{Молитва:}


Услыши недостойную молитву нас, грешных \itshape (имена),\normalfont{} избави нас, преподобная мати, от страстей, воюющих на души наша, от всякия печали и находящия напасти, от внезапныя смерти и от всякого зла, в час же разлучения души и тела отжени, святая угодница, всякую лукавую мысль и лукавые бесы, яко да приимет души наша с миром в место светло Христос Господь Бог наш, яко от него очищение грехов, и Той есть спасение душ наших, Емуже подобает всякая слава, честь; и поклонение со Отцем и Святым Духом во веки веков. Аминь.


\section{Пресвятой Богородице в честь Ее иконы «Споручница грешных»}
%http://www.molitvoslov.com/text754.htm 
 


\mysubsubsection{Тропарь, глас 4-й:}


Умолкает ныне всякое уныние и страх отчаяния исчезает, грешницы в скорби сердца обретают утешение и Hебесною любовию озаряются светло: днесь бо Матерь Божия простирает нам спасающую руку и от Пречистаго образа Своего вещает, глаголя: Аз Споручница грешных к Моему Сыну, Сей дал Мне за них руце слышати Мя выну. Темже людие, обремененнии грехи многими, припадите к подножию Ея иконы со слезами вопиюще: Заступнице мира, грешным Споручнице, умоли Матерними молитвами Избавителя всех, да Божественным всепрощением покрыет грехи наша, и светлыя двери райския отверзет нам, Ты бо еси предстательство и спасение рода христианскаго.


\mysubsubsection{Кондак, глас 1-й:}


Честное Жилище бывши неизреченнаго естества Божественнаго выше слова и паче ума и грешным еси Споручница, подаваеши благодать и исцеления, яко Мати всех Царствующаго, моли Сына Твоего получити нам милость в День судный.


\mysubsubsection{Молитва:}


О, Владычице Преблагословенная, Защитнице рода христианскаго, Прибежище и спасение притекающих к Тебе! Вем, воистинну вем, яко зело согреших и прогневах, Премилостивая Госпоже, рожденнаго плотию от Тебе Сына Божия, но имам многия образы прежде мене прогневавших Его благоутробие: мытари, блудницы и прочия грешники, имже дадеся прощение грехов их, покаяния ради и исповедания. Тыя убо образы помилованных очесем грешныя души моея представляя и на толикое Божие милосердие, онех приемшее, взирая, дерзнух и аз грешный прибегнути с покаянием ко Твоему благоутробию. О, Всемилостивая Владычице, да подаси ми руку помощи и испросиши у Сына Твоего и Бога матерними и святейшими Твоими молитвами тяжким моим грехом прощение. Верую и исповедую, яко Той, Егоже родила еси, Сын Твой, есть воистинну Христос, Сын Бога Живаго, Судия живых и мертвых, воздаяй комуждо по делом его. Верую же паки и исповедую Тебе быти истинную Богородицу, милосердия источник, утешение плачущих, взыскание погибших, сильную и непрестающую к Богу Ходатаицу, зело любящую род христианский, и Споручницу покаяния. Воистинну бо несть человеком иныя помощи и покрова, разве Тебе, Госпоже Премилостивая, и никтоже уповая на Тя постыдится когда, и Тобою умоляя Бога, никтоже оставлен бысть. Того ради молю Твою неисчетную благость: отверзи двери милосердия Твоего мене заблуждшему и падшему в тимение глубины, не возгнушайся мене, сквернаго, не презри грешнаго моления моего, не остави мене, окаяннаго, яко в погибель злобный враг похитити мя ищет, но умоли о мне Рожденнаго от Тебе милосердаго Сына Твоего и Бога, да простит великия моя грехи, и избавит мя от пагубы моея; яко да и аз со всеми получившими прощение воспою и прославлю безмерное милосердие Божие и Твое непостыдное о мне заступление в жизни сей и в нескончаемем веце. Аминь.


\section{Преподобному Ефрему Сирину}
%http://www.molitvoslov.com/text757.htm 
 


\mysubsubsection{Тропарь, глас 8-й:}


В тебе, отче, известно спасеся еже по образу: приим бо Крест последовал еси Христу, и дея учил еси презирати убо плоть, преходит бо, прилежати же о души, вещи бессмертней. Темже и со ангелы срадуется, преподобне Ефреме, дух твой.


\mysubsubsection{Тропарь, глас 8-й:}


Слез твоих теченьми пустыни безплодное возделал еси, и иже из глубины воздыханьми во сто трудов уплодоносил еси, и был еси светильник вселенней, сияя чудесы, Ефреме, отче наш, моли Христа Бога спастися душам нашим.


\mysubsubsection{Кондак, глас 2-й:}


Час присно провидя суда, рыдал еси горько, Ефреме, яко любобезмолвный, делателен же был еси в делех учитель, преподобне. Темже, отче всемирный, ленивыя воздвизаеши к покаянию.


\mysubsubsection{Молитва:}


О угодниче Христов, отче наш Ефреме! Принеси молитву нашу к милостивому и всесильному Богу и испроси нам, рабам Божиим (\itshape имена\normalfont{}), у благости Его вся яже на пользу душам и телесем нашим: веру праву, надежду несумненну, любовь нелицемерну, кротость и незлобие, во искушениях мужество, в злостраданиях терпение, во благочестии преспеяние. Да не во зло обратим дары Всеблагого Бога. Не забуди, чудотворче святый, и святый храм (дом) сей и приход наш: сохрани и соблюди их молитвами твоими от всякого зла. Ей, святче Божий, сподоби нас кончину благую улучити и Царствие Небесное унаследити, да прославим дивнаго во святых Своих Бога, Емуже подобает всякая слава, честь и держава, во веки веков. Аминь.


\section{Святителю  Андрею, архиепископу Критскому}
%http://www.molitvoslov.com/text756.htm 
 


\mysubsubsection{Тропарь, глас 1-й:}


Христову Церковь цевницею языка твоего, песнословя умильно, возвеселил еси, богословием же Препетыя Троицы славу всем сказал еси ясно, тем тя, яко тайноглагольника, поем, Андрее, пастырю Критский, и величаем память твою, Христа славяще дивнаго во святых Своих.


\mysubsubsection{Кондак, глас 2-й:}


Вострубив ясно Божественная сладкопения, явился еси светильник мира светлейший, светом сияя Троицы, Андрее преподобне. Темже вси вопием ти: не престай моляся о всех нас.


\mysubsubsection{Молитва:}


О, пречестная и священная главо и благодати Святаго Духа исполненная, Спасово со Отцем обиталище, великий архиерее, теплый наш заступниче, святителю Андрее! Предстоя у Престола всех Царя и наслаждаяся света Единосущныя Троицы и херувимски со ангелы возглашая песнь трисвятую, великое же и неизследованное дерзновение имея ко Всемилостивому Владыце, моли спастися паствы Христовы людем, благостояние святых церквей утверди, архиереи благолепием святительства украси, монашествующия к подвигом добраго течения укрепи, град сей и вся грады страны добре сохрани и веру святую непорочну соблюсти умоли, мир весь предстательством твоим умири, от глада и пагубы избави ны, и от нападения иноплеменных сохрани, старыя утеши, юныя настави, безумныя умудри, вдовицы помилуй, сироты заступи, младенцы возрасти, плененныя возврати, немощствующия исцели, и везде тепле призывающия тя и с верою припадающия и молящияся тебе от всяких напастей и бед ходатайством твоим свободию Моли о нас Всещедраго и Человеколюбиваго Христа Бога нашего, да и в день страшнаго пришествия Его от шуияго стояния избавит нас и радости святых причастники сотворит со всеми святыми во веки веков. Аминь.


\section{Когда заметишь за собой какой грех. Молитва св. Иоанна Кронштадтского.}
%http://www.molitvoslov.com/content/Kogda-zametish-za-soboi-kakoi-grekh-Molitva-sv-Ioanna-Kronshtadtskogo 
 


Господи! Грехи наши на кресте пригвоздивый, пригвозди ко кресту Твоему и настоящий мой грех и помилуй мя по велицей Твоей милости. (\itshape И наложить на себя крестное знамение.\normalfont{})
\mychapterending

\mychapter{От гордыни и самомнения}
%http://www.molitvoslov.com/content/ot-gordyni_isamomneniya

 
\vspace{-\baselineskip}
\section{Преподобному Алексию, человеку Божию}
%http://www.molitvoslov.com/content/text760.htm 
 


\mysubsubsection{Тропарь, глас 4-й:}


Возвысився на добродетель и ум очистив, к желанному и крайнему достигл еси, безстрастием же украсив житие твое, и пощение изрядное восприим совестию чистою, в молитвах, яко безплотен, пребывая, возсиял еси, яко солнце, в мире, преблаженне Алексие.


\mysubsubsection{Кондак, глас 2-й:}


Дом родителей твоих яко чужд, имев, водворился еси в нем нищеобразно и, по преставлении венец прием славы, дивен на земли явился еси, Алексие, человече Божий, ангелом и человеком радование.


\mysubsubsection{Молитва:}


О, великий Христов угодниче, святый человече Божий Алексие, душею на Небеси Престолу Господню предстояй, на земли же данною ти свыше благодатию различная совершаяй чудеса! Призри милостивно на предстоящия святей иконе твоей люди, умиленно молящияся и просящия от тебе помощи и заступления. Простри молитвенно ко Господу Богу честнии руце твои, и испроси нам от Него оставление согрешений наших вольных и невольных, в недузех страждущим исцеление, напаствуемым заступление, скорбящим утешение, бедствующим скорую помощь, всем же чтущим тя мирную и христианскую живота кончину и добрый ответ на страшнем суде Христове. Ей, угодниче Божий, не посрами упования нашего, еже на тя по Бозе и Богородице возлагаем, но буди нам помощник и покровитель во спасение, да, твоими молитвами получивше благодать и милость от Господа, прославим человеколюбие Отца и Сына и Святаго Духа, в Троице славимаго и покланяемаго Бога, и твое святое заступление, ныне и присно и во веки веков. Аминь.


\section{При тщеславных помыслах. Молитва св. Иоанна Кронштадтского.}
%http://www.molitvoslov.com/content/Pri-tshcheslavnykh-pomyslakh-Molitva-sv-Ioanna-Kronshtadtskogo 
 


Господи, не дай мне возмечтать о себе как бы о лучшем кого-либо из людей, но думать как о худшем всех и никого не осуждать, а себя судить строго. Аминь.
\longpage{}\mychapterending

\mychapter{От сребролюбия}
%http://www.molitvoslov.com/content/ot-srebrolyubiya

 

\section{Преподобномученикам Феодору и Василию Печерским}
%http://www.molitvoslov.com/text762.htm 
 


\mysubsubsection{Тропарь, глас 1-й:}


Союзом любве связавшеся, преподобнии, всяк союз козней вражиих попрасте, страдание же и смерть неповинне от сребролюбиваго князя претерпесте добле, тем молим вас, вкупе поживших и венцы мучения приемших: молитеся Господеви о нас, яко да во мнозей любви, вере и надежди добле поживше, всегда ублажаем вас, Феодоре и Василие добропобеднии.


\mysubsubsection{Кондак, глас 2-й:}


Добр светильник явлься блаженному Феодору, Богомудре Василие, избавил еси того советом своим от прелести диаволи и ко свету Богоразумия наставль. С темже последе блаженную приял еси кончину, устрелен неправедно быв во утробу от сребролюбиваго князя. И ныне, предстоя Господеви, молися непрестанно о всех нас.


\mysubsubsection{Молитва:}


Преподобнии отцы Феодоре и Василие! Воззрите на нас милостивно и к земли приверженных возведите к высоте небесней. Вы горе на небеси, мы на земли низу, удалены от вас, не толико местом, елико грехми своими и беззаконии, но к вам прибегаем и взываем: наставите нас ходити путем вашим, вразумите и руководствуйте. Вся ваша святая жизнь бысть зерцалом всякия добродетели. Не престаните, угодницы Божии, о нас вопия ко Господу. Испросите предстательством своим у Всемилостиваго Бога нашего мир Церкви Его, под знамением креста воинствующей, согласие в вере и единомудрие, суемудрия же и расколов истребление, утверждение во благих делех, больным исцеление, печальным утешение, обиженным заступление, бедствующим помощь. Не посрамите нас, к вам с верою притекающих. Вси православнии христиане, вашими чудесы исполненнии и милостями облагодетельствованнии, исповедуют вас быти своих покровителей и заступников. Явите древния милости ваша, и ихже отцем всепомоществовали есте, не отрините и нас, чад их, стопами их к вам шествующих. Предстояще всечестней иконе вашей, яко вам живым сущим, припадаем и молимся: приимите моления наша и вознесите их на жертвенник благоутробия Божия, да приимем вами благодать и благовременную в нуждех наших помощь. Укрепите наше малодушие и утвердите нас в вере, да несомненно уповаем получити вся благая от благосердия Владыки молитвами вашими. О, превеликие угодницы Божии! Всем нам, с верою притекающим к вам, помозите предстательством вашим ко Господу, и всех нас управите в мире и покаянии скончати живот наш и преселитися со упованием в блаженныя недра Авраамова, идеже вы радостно во трудех и подвизех ныне почиваете, прославляя со всеми святыми Бога, в Троице славимаго, Отца и Сына и Святаго Духа, ныне и присно и во веки веков. Аминь.
\mychapterending

\mychapter{От недуга пьянства}
%http://www.molitvoslov.com/content/ot-neduga-pyanstva

 

\section{Святому мученику Вонифатию}
%http://www.molitvoslov.com/text766.htm 
 


\mysubsubsection{Тропарь, глас 4-й:}


К сословию послан мучеников, мученик был еси истинен пострадав за Христа крепчайше, всехвальне, мощми же возвратился еси верою пославшей тя, Вонифатие блаженне, моли Христа Бога прияти нам грехов прощение.


\mysubsubsection{Кондак, глас 4-й:}


Священие непорочное самовольно Тебе привел еси, Иже от Девы тебе ради родитися Хотящему, святе венченосче, премудре Вонифатие.


\mysubsubsection{Молитва:}


О, многострадальный и всехвальный мучениче Вонифатие! Ко твоему заступлению ныне прибегаем, молений нас, поющих тебе, не отвержи, но милостивно услыши нас. Виждь братию и сестры наша, тяжким недугом пианства одержимыя, виждь того ради от матери своея, Церкве Христовой, и вечнаго спасения отпадающия. О, святый мучениче Вонифатие, коснися сердцу их данною ти от Бога благодатию, скоро возстави от падений греховных и ко спасительному воздержанию приведи их. Умоли Господа Бога, Егоже ради страдал еси, да простив нам согрешения наша, не отвратит милости Своея от сынов Своих, но да укрепит в нас трезвение и целомудрие, да поможет Своею десницею трезвящимся спасительный обет свой до конца сохранити во дни и нощи, в Нем бодрствующе и о Нем добрый ответ на страшное судилище дати. Приими, угодниче Божий, молитвы матерей, о чадех своих слезы проливающих; жен честных, о мужех своих рыдающих, чад сирых и убогих, от пианиц оставленных, всех нас, к иконе твоей припадающих, и да приидет сей вопль наш молитвами ко Престолу Всевышняго даровати всем по молитвам их здравие и спасение душ и телес, наипаче же Царство Небесное. Покрый и соблюди нас от лукаваго ловления и всех козней вражиих, в страшный час исхода нашего помози прейти непреткновенно воздушныя мытарства и молитвами твоими избави вечнаго осуждения. Умоли же Господа даровати нам к отечеству нашему любовь нелицемерную и непоколебимую, пред врагами Церкве Святыя видимыми и невидимыми, да покроет нас милость Божия в нескончаемыя веки веков. Аминь.


\section{Молитва святого праведного Иоанна Кронштадского}
%http://www.molitvoslov.com/text770.htm 
 


Господи, призри милостиво на раба Твоего (\itshape имя\normalfont{}), прельщенного лестью чрева и плотского веселья. Даруй ему (\itshape имя\normalfont{}), познать сладость воздержания в посте и проистекающих от него плодов Духа. Аминь.

\itshape 

Для избавления от страсти пьянства рекомендуется также читать ежедневно главу 15 Евангелия от Иоанна.\normalfont{}


\section{Молитва Пресвятой Богородице и Приснодеве Марии пред  Ее иконой «Неупиваемая Чаша»}
%http://www.molitvoslov.com/text765.htm 
 


\mysubsubsection{Тропарь, глас 4-й:}


Днесь притецем вернии к Божественному и пречудному образу Пресвятыя Богоматери, напaяющей верных сердца небесною Неупиваемою Чашею Своего милосердия, и людем верным чудеса показующей. Яже мы видяще и слышаще духовно празднуем и тепле вопием: Владычице Премилостивая, исцели наша недуги и страсти, молящи Сына Твоего, Христа Бога нашего, спасти души наша.


\mysubsubsection{Кондак:}


Избранное и дивное избавление нам даровася "--- Твой образ честный, Владычице Богородице, яко избавльшеся явлением его от недугов душевных и телесных и скорбных обстояний, благодарственная хваления приносим Ти, Всемилостивая Заступнице. Ты же, Владычице, Неупиваемою Чашею нами именуемая, приклонися благоутробно к нашим воздыханиям и воплем сердечным, и избавление подаждь страждущим недугом пианства, да с верою воззовем Ти: Радуйся, Владычице, Неупиваемая Чаше, духовную жажду нашу утоляющая.


\mysubsubsection{Молитва:}


О, Премилосердая Владычице! К Твоему заступлению ныне прибегаем, молений наших не презри, но милостивно услыши нас: жен, детей, матерей и тяжким недугом пианства одержимых, и того ради от матере своея "--- Церкви Христовой и спасения отпадающих, братьев и сестер, и сродник наших исцели. О, Милостивая Мати Божия, коснися сердец их и скоро возстави от падений греховных, ко спасительному воздержанию приведи их. Умоли Сына Своего, Христа Бога нашего, да простит нам согрешения наша и не отвратит милости Своея от людей Своих, но да укрепит нас в трезвении и целомудрии. Приими, Пресвятая Богородице, молитвы матерей, о чадех своих слезы проливающих, жен, о мужех своих рыдающих, чад, сирых и убогих, заблудшими оставленных, и всех нас, к иконе Твоей припадающих. И да приидет сей вопль наш, молитвами Твоими, ко Престолу Всевышняго. Покрый и соблюди нас от лукаваго ловления и всех козней вражиих, в страшный же час исхода нашего помоги пройти непреткновенно воздушныя мытарства, молитвами Твоими избави нас вечнаго осуждения, да покрыет нас милость Божия в нескончаемые веки веков. Аминь.


\section{Преподобному Моисею Мурину}
%http://www.molitvoslov.com/text769.htm 
 


\mysubsubsection{Тропарь, глас 8-й:}


В тебе, отче, известно спасеся еже по образу: приим бо Крест последовал еси Христу, и дея учил еси презирати убо плоть, преходит бо, прилежати же о души, вещи бессмертней. Темже и со ангелы срадуется, преподобне Моисее, дух твой.


\mysubsubsection{Кондак, глас 4-й:}


Мурины заушив, и лица демонов поплевав, мысленно просиял еси, якоже солнце светло, светом жития твоего и учением наставляя души наша.


\mysubsubsection{Молитва:}


О, великая сила покаяния! О, неизмеримая глубина милосердия Божия! Ты, преподобне Моисей, был прежде разбойником, но потом ужаснулся своих грехов, возскорбел о них, и в раскаянии пришел в монастырь, и там в великом плаче о своих бывших беззакониих и в трудных подвигах проводил дни свои до кончины, и удостоился Христовой благодати прощения и дара чудотворения. О, преподобне, ты от тяжких грехов достиг пречудных добродетелей! Помози и молящимся тебе рабом Божиим \itshape (имена),\normalfont{} влекомым в погибель от того, что предаются безмерному, вредному для души и тела употреблению вина. Склони на них свой милостивый взор и не презри их, но внемли им, прибегающим к тебе. Моли, святе Моисей, Владыку Христа, чтобы не отверг их Он, Милосердный, и да не возрадуется диавол их погибели, но да пощадит Господь этих безсильных и несчастных \itshape (имена),\normalfont{} которыми овладела пагубная страсть пьянства, ведь мы все Божии создания и искуплены Пречистой Кровию Сына Его. Услышь же, преподобный Моисей, их молитву, отгони от них диавола, даруй им силу победить их страсть, помоги им, простри твою руку, выведи их на путь добра, освободи от рабства страстей и избавь от их от винопития, чтобы они, обновленные, в трезвении и светлом уме, возлюбили  воздержание и благочестие, вечно прославляли Всеблагого Бога, всегда спасающего Свои создания. Аминь.


\section{Молитва от недуга пьянства}
%http://www.molitvoslov.com/text764.htm 
 


\mysubsubsection{Молитва:}


Спаси, Господи, и помилуй рабов Твоих (\itshape имярек\normalfont{}) словесами Божественного Евангелия Твоего, читаемыми о спасении рабов Твоих сих (\itshape имярек\normalfont{}). Попали, Господи, терние всех согрешений их, вольных и невольных, и да вселится в них благодать Твоя, просвещающая, опаляющая, очищающая всего человека. Во имя Отца и Сына и Святаго Духа. Аминь.


\section{Святителю Вонифатию Милостивому}
%http://www.molitvoslov.com/text768.htm 
 


\mysubsubsection{Тропарь, глас 4-й:}


Правило веры и образ кротости, воздержания учителя яви тя стаду твоему яже вещей истина: сего ради стяжал еси смирения высокая, нищетою богатая. Отче священноначальниче Вонифатиe, моли Христа Бога спастися душам нашим.


\mysubsubsection{Кондак, глас 2-й:}


Божественный гром, труба духовная, веры насадителю и отсекателю ересей, Троицы угодниче, великий святителю Вонифатиe, со ангелы предстоя присно,  моли непрестанно о всех нас.


\mysubsubsection{Молитва:}


О, всесвятый Вонифатие, милостивый раб Милосердаго Владыки! Услыши прибегающих к тебе, одержимых пагубным пристрастием к винопитию, и, как в своей земной жизни ты никогда не отказывал в помощи просящим тя, так и теперь избави этих несчастных (\itshape имена\normalfont{}). Некогда, богомудрый отец, град побил твой виноградник, ты же, воздав благодарение Богу, велел немногие сохранившиеся грозды положйти в точиле и позвати нищих. Затем, взяв новое вино, ты разлил его по каплям во все сосуды, бывшие в епископии, и Бог, исполняющий молитву милостивых, совершил преславное чудо: вино в точиле умножилось, и нищие наполнили свои сосуды. О, святителю Божий! Как по твоей молитве умножилось вино для нужд церкви и для пользы убогих, так ты, блаженный, уменьши его теперь там, где оно приносит вред, избави от пристрастия к нему предающихся постыдной страсти винопития (\itshape имена\normalfont{}), исцели их от тяжкого недуга, освободи от бесовского искушения, утверди их, слабых, дай им, немощным, крепость и силу благоу спешно перенести это искушение, возврати их к здоровой и трезвой жизни, направи их на путь труда, вложи в них стремление к трезвости и духовной бодрости. Помоги им, угодник Божий Вонифатие, когда жажда вина станет жечь их гортань, уничтожи их пагубное желание, освежи их уста небесною прохладою, просвети их очи, постави их ноги на скале веры и надежды, чтобы, оставив свое душевредное пристрастие, влекущее за собой отлучение от Небеснаго Царствия, они, утвердившися в благочестии, удостоились непостыдной мирной кончины и в вечном свете бесконечного Царства Славы достойно прославляли Господа нашего Иисуса Христа со Безначальным Его Отцем и с Пресвятым и Животворящим Его Духом во веки веков. Аминь.


\section{Молитва вторая Святому мученику Вонифатию}
%http://www.molitvoslov.com/text767.htm 
 


О, святый страстотерпче Христов, воине Царя Небеснаго, земная сладострастия презревый и к Горнему Иерусалиму страданием возшедый, мучениче Вонифатие! Услыши мя грешнаго, приносящаго от сердца молебная пения, и умоли Господа нашего Иисуса Христа простити вся моя согрешения, в ведении и в неведении мною содеянная. Ей, мучениче Христов, образ покаяния собою показавый грешным! Буди ми на злаго сопротивника диавола твоими молитвами к Богу помощник и заступник: много бо поревахся избегнути сетей его лукавых, но удицею греховною уловлен и крепко влачимь от него, не могу избавитися, аще не ты предстанеши ми, во обстояниих горькая терпящему, и колико покушахся покаятися, но ложь пред Богом бых. Сего ради к тебе прибегаю и молюся: спаси мя, святче Божий, от всех зол твоим предстательством, благодатию же Всесильнаго Бога, в Троице Святей славимаго и покланяемаго, Отца и Сына и Святаго Духа, ныне и присно и во веки веков. Аминь. 
\mychapterending

\mychapter{При обуревании плотской страстию}
%http://www.molitvoslov.com/content/Pri-oburevanii-plotskoy-strastiyu

 

\section{Святителю Евфимию, архиепископу Новгородскому, чудотворцу}
%http://www.molitvoslov.com/text772.htm 
 


\mysubsubsection{Тропарь, глас 4-й:}


Избран быв Богови от юности, святителю Евфимие, и, сего ради архиерейства саном почтен быв, упасл еси люди, иже тебе Богом врученныя; темже и по преставлении чудес дарования от Господа приял еси, исцеляти различныя недуги. Того моли о нас, совершающих честную память твою, да тебе вси непрестанно ублажаем.


\mysubsubsection{Кондак, глас 8-й:}


Яко архиереем сопрестольник и святителем изрядный поборник был еси, святителю Евфимие, не престай сохраняя отечество твое, град же и люди, иже тебе верою почитающия и честным мощем твоим покланяющияся, да велегласно тебе вси вопием: радуйся, святителю Богомудре!


\mysubsubsection{Молитва:}


Отче святителю Евфимие! Ты от юности твоея Христа возлюбил еси и благодатию Его укрепляем, вся плотская мудрования умертвил еси, чистым житием и кротким нравом угодил еси Господеви, и по благодати Его на престоле святительства подвизался, явился еси пастве Христове Великаго Новаграда пастырь добрый, полагаяй душу твою о стаде твоем. Темже и по смерти твоей прослави тя Пастыреначальник Христос подаянием чудес, от тебе истекающих, образ добродетельнаго жития в Тебе нам показуя. Темже и аз грешный и унылый, к к раце мощей Твоих припадая, молюся тебе усердно: даждь ми руку помощи, воздвигни мя молитвами твоими из глубины греховныя, се бо волнами страстей плотских и уныния, и иными тьмочисленными житейскими треволнении обуреваюся и погибаю. Буди убо мне, многогрешному, теплый ко Христу предстатель и молитвенник, да и аз, от потопления греховнаго избавлен, ко пристанищу непорочнаго жития благодатию Христовою достигну, и чистым житием прославлю Спасителя моего, кровию Своею мене искупившаго: и тако житие скончав, получу живот вечный во Царствии Небеснем, идеже славится пречестное и великолепое имя Отца и Сына и Святаго Духа, ныне и присно и во веки веков. Аминь. 


\section{Преподобной Марии  Египетской}
%http://www.molitvoslov.com/text773.htm 
 


\mysubsubsection{Тропарь, глас 8-й:}


В тебе, мати, известно спасеся еже по образу: приимши бо крест, последовала еси Христу, и деющи учила еси презирати убо плоть, преходит бо, прилежати же о души, вещи безсмертней. Темже и со aнгелы срадуется, преподобная Марие, дух твой.


\mysubsubsection{Кондак, глас 4-й:}


Греха мглы избежавши, покаяния светом озаривши твое сердце, славная, пришла еси ко Христу, Сего всенепорочную и святую Матерь, молитвенницу милостивную принесла еси. Отонудуже и прегрешений обрела еси оставление, и со ангелы присно срадуешися.


\mysubsubsection{Молитва:}


Услыши недостойную молитву нас, грешных \itshape (имена),\normalfont{} избави нас, преподобная мати, от страстей, воюющих на души наша, от всякия печали и находящия напасти, от внезапныя смерти и от всякого зла, в час же разлучения души и тела отжени, святая угодница, всякую лукавую мысль и лукавые бесы, яко да приимет души наша с миром в место светло Христос Господь Бог наш, яко от него очищение грехов, и Той есть спасение душ наших, Емуже подобает всякая слава, честь; и поклонение со Отцем и Святым Духом во веки веков. Аминь.


\section{Мученице Фомаиде Египетской}
%http://www.molitvoslov.com/text774.htm 
 


\mysubsubsection{Тропарь, глас 4-й:}

 Агница Твоя, Иисусе, Фомаида, зовет велиим гласом: Тебе, Женише мой, люблю, и Тебе ищущи страдальчествую и сраспинаюся, и спогребаюся Крещению Твоему, и стражду Тебе ради, яко да царствую в Тебе, и умираю за Тя, да и живу с Тобою; но яко жертву непорочную приими мя, с любовию пожершуюся Тебе. Тоя молитвами, яко Милостив, спаси души наша.


\mysubsubsection{Кондак, глас 2-й:}

 Храм твой всечестный яко цельбу душевную обретше, вси вернии велегласно вопием ти: дево мученице, Фомаидо великоименитая, Христа Бога моли непрестанно о всех нас.


\mysubsubsection{Молитва:}

 О, всехвальная мученице Фомаидо! За чистоту супружества даже до крове подвизавшися и целомудрия ради душу свою положивши, достойна пред Господем обрелася еси, во еже в лице преподобных дев почтенней тебе быти. Услыши нас молящихся тебе, и якоже древле боримии от плоти целительницу тя имеяху, по дарованной тебе от Бога благодати, сице и ныне прибегающим ко предстательству твоему подаждь отраду и свобождение от плотския брани, и целомудренное житие и незазорное в супружестве и девстве пребывание твоими богоприятными молитвами всем исходатайствовати потщися, яко да телеса наша храм живущаго в нас Святаго Духа будут. О, преизбранная в женах и верная рабо Христова! Помози нам, да не погибнем со страстьми и похотьми нашими, но да управимся умом нашим и укрепимся сердцем нашим во всяком благочестии и чистоте, славяще твою помощь и предстательство, благодать же и милость Триединаго Бога, Отца и Сына и Святаго Духа, во веки веков. Аминь. 
\mychapterending

\mychapter{В душевной болезни}
%http://www.molitvoslov.com/content/%D0%B2-%D0%B4%D1%83%D1%88%D0%B5%D0%B2%D0%BD%D0%BE%D0%B9-%D0%B1%D0%BE%D0%BB%D0%B5%D0%B7%D0%BD%D0%B8

 

\section{Святой праведной Иулиании Ольшанской}
%http://www.molitvoslov.com/text776.htm 
 


\mysubsubsection{Тропарь:}


Яко непорочная невеста Нетленнаго Жениха Христа, праведная дево Иулиание,  со светлою свещею добрых дел вошла еси в чертог Его Небесный и тамо со святыми блаженства вечнаго наслаждаешися. Тем же моли, Его же возлюбила еси и Ему же девство твое обручила еси, спастися душам нашим.


\mysubsubsection{Молитва:}


О, святая праведная дево Иулиание, княжно Ольшанская, помощнице всем жаждущим спасения исцеления от болезней душ и телес! О, святая агнице Божие, яко имея дар мнози болезни целити и от всех козней врагов ограждати, исцели убо наша страсти душевныя и облегчи тяжкия болезни телесныя, отраду в скорбех даруй и избави нас от всяких бед и напастей. Призри на вся предстоящия честным твоим мощем (иконе) просящим твоея помощи сердцем сокрушенным и духом смиренным, да принесем во всем житии нашем плоды духовныя: любовь, благость, милосердие, веру, кротость, воздержание, да сподобимся жизни вечныя и да любовию твоею ограждаеми, поем прославльшему тя Господу Иисусу Христу. Ему же подобает всякая слава и честь со Безначальным Его Отцем и Пресвятым Животворящим Его Духом, ныне и присно, и во веки веков. Аминь.
\mychapterending

\mychapter{Об укреплении в перенесении поста}
%http://www.molitvoslov.com/content/ob-ukreplenii-v-perenesenii-posta

 

\section{Великомученику Феодору Тирону}
%http://www.molitvoslov.com/text778.htm 
 


\mysubsubsection{Тропарь, глас 2-й:}


Велия веры исправления, во источнице пламене, яко на воде упокоения, святый мученик Феодор радовашеся: огнем бо всесожегся, яко хлеб сладкий Троице принесеся. Того молитвами, Христе Боже, спаси души наша.


\mysubsubsection{Кондак, глас 8-й:}


Веру Христову, яко щит, внутрь приим в сердце твоем, противныя силы попрал еси, многострадальче, и венцем Небесным венчался еси вечно, Феодоре, яко непобедимый.


\mysubsubsection{Стихира на стиховне, глас 2-й:}


Божественных даров тезоименита тя, треблаженне, почитаю, Феодоре: света бо Божественнаго незаходимое светило явлься, просветил еси страданьми твоими всю тварь, и огня крепчайший явився, пламень угасил еси, и льстиваго змия главу сокрушил еси. Темже во страданиих твоих Христос приклонився, венча Божественную главу твою, великомучениче страдальче. Яко имея дерзновение к Богу, прилежно молися о душах наших.


\mysubsubsection{Молитва:}


Святый великомучениче Феодоре! Призри с небеснаго чертога на требующих твоея помощи и не отвергни прошений наших, но, яко присный благодетель и ходатай наш, моли Христа Бога, да, человеколюбив и многомилостив сый, сохранит нас от всякаго лютаго обстояния: от труса, потопа, огня, меча, нашествия иноплеменников и междоусобныя брани. Да не осудит нас грешных по беззаконием нашим, и да не во зло обратим благая, даруемая нам от Всещедраго Бога, но во славу святаго имене Его и в прославление крепкаго твоего заступления. Да молитвами твоими даст нам Господь мир помыслов, воздержание от пагубных страстей и от всякия скверны и да укрепит во всем мире Свою Едину Святую, Соборную и Апостольскую Церковь, юже стяжал есть честною Своею Кровию. Молися прилежно, святий мучениче, да благословит Христос Бог державу, да утвердит во святей Своей Православней Церкви живый дух правыя веры и благочестия, да вси члены ея, чистии от суемудрия и суеверия, духом и истиною покланяются Ему и усердно пекутся о соблюдении Его заповедей, да мы вси в мире и благочестии поживем в настоящем веце и достигнем блаженныя вечныя жизни на небеси, благодатию Господа нашего Иисуса Христа, Емуже подобает всякая слава, честь и держава со Отцем и Святым Духом, ныне и присно и во веки веков. Аминь.
\mychapterending

\mychapter{Об избавлении от мыслей о самоубийстве}
%http://www.molitvoslov.com/content/ob-izbavlenii-ot-mysley-o-samoubiystve

 

\section{Пресвятой Богородице перед Ее иконой «Страстная»}
%http://www.molitvoslov.com/text780.htm 
 
\myfigh{img/643.jpg}{15}

\mysubsubsection{Тропарь, глас 4-й:}


Днесь возсия неизреченно царствующему граду нашему Москве икона Богоматере, и, яко светозарным солнцем, пришествием тоя озарися весь мир, Небесныя Силы и души праведных мысленно торжествуют, радующеся, мы же, на ню взирающе, к Богородице со слезами вопием: о Всемилостивая Госпоже, Владычице Богородице, молися из Тебе воплощенному Христу, Богу нашему, да подаст мир и здравие всем христианом по велицей и неизреченной Своей милости.


\mysubsubsection{Кондак, глас 3-й:}


Благодать прияхом нетленния, еже даровала еси нам, спасительного Твоего целения чудотворным Твоим честным, Богородице Дево, темже вопием Ти и с радостью зовем, Госпоже Царице, молим Ти ся умильно, грешнии, со слезами глаголюще: о Пресвятая Владычице, скорое нам яви заступление и помощь, спаси ны от супостат наших и от всякия скорби соблюди, землю нашу миром огради, и вся люди Твоя покрый, и соблюди на Тя уповающия, потщися избавити, да не погибнем зле, раби Твои, но да зовем Ти: радуйся, Невесто Неневестная.


\mysubsubsection{Молитва:}


О, Пресвятая Госпоже Владычице Богородице, высши еси всех Ангел и Архангел и всея твари честнейши. Помощница еси обидимых, ненадеющихся надение, убогих Заступница, печальных утешение, алчущих Кормительница, нагих одеяние, больных исцеление, грешных спасение, христиан всех поможение и заступление. 


О, Всемилостивая Госпоже, Дево Богородице, Владычице, милостию Твоею спаси и помилуй раб Твоих, Великого Господина и отца нашего Святейшаго Патриарха (\itshape имя\normalfont{}), и преосвященныя митрополиты, архиепископы и епископы, и весь священнический и иноческий чин, богохранимую страну нашу, военачальники, градоначальники и христолюбивое воинство и доброхоты, и вся православныя христианы ризою Твоею честною защити, и умоли, Госпоже, из Тебе без Семене воплотившагося Христа Бога нашего, да препояшет нас силою Своею свыше на невидимыя и видимыя враги наша. 

О, Всемилостивая Госпоже Владычице Богородице, воздвигни нас из глубины греховныя и избави нас от глада, губительства, от труса и потопа, от огня и меча, от нахождения иноплеменных и междоусобныя брани, и от напрасныя смерти, и от нападения вражия, и от тлетворных ветр, и от смертоносныя язвы, и от всякаго зла. Подаждь, Госпоже, мир и здравие рабом Твоим, всем православным христианом, и просвети им ум и очи сердечныя, еже ко спасению, и сподоби ны, грешныя рабы Твоя, Царствия Сына Твоего, Христа Бога нашего, яко держава Его благословена и препрославлена, со Безначальным Его Отцем и с Пресвятым и Благим и Животворящим Его Духом, ныне и присно и во веки веков. Аминь.
\mychapterending

\mychapter{О возвращении Церкви отпавших от нее по навету диавольскому}
%http://www.molitvoslov.com/content/%D0%BE-%D0%B2%D0%BE%D0%B7%D0%B2%D1%80%D0%B0%D1%89%D0%B5%D0%BD%D0%B8%D0%B8-%D1%86%D0%B5%D1%80%D0%BA%D0%B2%D0%B8-%D0%BE%D1%82%D0%BF%D0%B0%D0%B2%D1%88%D0%B8%D1%85-%D0%BE%D1%82-%D0%BD%D0%B5%D0%B5-%D0%BF%D0%BE-%D0%BD%D0%B0%D0%B2%D0%B5%D1%82%D1%83-%D0%B4%D0%B8%D0%B0%D0%B2%D0%BE%D0%BB%D1%8C%D1%81%D0%BA%D0%BE%D0%BC%D1%83

 

\section{Преподобному Симеону Столпнику}
%http://www.molitvoslov.com/text785.htm 
 


\mysubsubsection{Тропарь, глас 1-й:}


Терпения столп был еси, ревновавый праотцем, преподобне, Иову во страстех, Иосифу во искушениих, и безплотных жительству, сый в телеси, Симеоне отче наш, моли Христа Бога спастися душам нашим. 


\mysubsubsection{Кондак, глас 2-й:}


Вышних ищай, нижним совокупляяйся, и колесницу огненную столп соделавый: тем собеседник ангелом был еси, преподобне, с ними Христу Богу моляся непрестанно о всех нас.


\mysubsubsection{Седален, глас 8-й:}


Крест Господень взем мудре, и Тому до конца последовав, умом не возвратился еси в мир, богомудре: воздержанием и труды страсти умертвив, и храм уготовил себе Господу Твоему. Темже и дарований возмездия приял еси, исцеляти недужныя, и духи отгоняти, Симеоне преподобнейшиий, моли Христа Бога грехов оставление подати празднующим любовию святую память твою.


\mysubsubsection{Молитва:}


Преподобне отче Симеоне! Воззри на нас милостивно и к земли приверженных возведи к высоте небесней. Ты горе на небеси, мы на земли низу, удалены от тебе, не толико местом, елико грехми своими и беззаконии, но к тебе прибегаем и взываем: настави нас ходити путем твоим, вразуми и руководствуй. Вся твоя святая жизнь бысть зерцалом всякия добродетели. Не престани, угодниче Божий, о нас вопия ко Господу. Испроси предстательством своим у Всемилостиваго Бога нашего мир Церкви Его, под знамением креста воинствующей, согласие в вере и единомудрие, суемудрия же и расколов истребление, утверждение во благих делех, больным исцеление, печальным утешение, обиженным заступление, бедствующим помощь. Не посрами нас, к тебе с верою притекающих. Вси православнии христиане, твоими чудесы исполненнии и милостями облагодетельствованнии, исповедуют тя быти своего покровителя и заступника. Яви древния милости твоя, и ихже отцем всепомоществовал еси, не отрини и нас, чад их, стопами их к тебе шествующих. Предстояще всечестней иконе твоей, яко тебе живу сущу, припадаем и молимся: приими моления наша и вознеси их на жертвенник благоутробия Божия, да приимем тобою благодать и благовременную в нуждех наших помощь. Укрепи наше малодушие и утверди нас в вере, да несомненно уповаем получити вся благая от благосердия Владыки молитвами твоими. О, превеликий угодниче Божий! Всем нам, с верою притекающим к тебе, помози предстательством твоим ко Господу, и всех нас управи в мире и покаянии скончати живот наш и преселитися со упованием в блаженныя недра Авраамова, идеже ты радостно во трудех и подвизех ныне почиваеши, прославляя со всеми святыми Бога, в Троице славимаго, Отца и Сына и Святаго Духа, ныне и присно и во веки веков. Аминь.
\mychapterending

\mychapter{Молитвы святым об исцелении от беснования}
%http://www.molitvoslov.com/content/molitvi-svyatym-ob-iscelenii-ot-besnovaniya

 

\section{Преподобному Савве Крыпецкому, чудотворцу}
%http://www.molitvoslov.com/text791.htm 
 


\mysubsubsection{Тропарь, глас 8-й:}


В тебе, отче, известно спасеся еже по образу: приим бо Крест последовал еси Христу, и дея учил еси презирати убо плоть, преходит бо, прилежати же о души, вещи бессмертней. Темже и со ангелы срадуется, преподобне Савве, дух твой.


\mysubsubsection{Тропарь иной, глас 8-й:}


Пустыни явился еси доброе насаждение, преподобне, от юности бо и в мире изволил еси чистое житие; ревнуя же большему, и дом отеческий оставил еси, и, вся красная мира сего презрев, от чуждых стран пришел еси во обители Богоспасаемаго града Пскова, и послушанию преподобнаго Евфросина предался еси; таже простирался умом еще к вящшему, вселился еси в пустыню, в нейже обитель пречестну воздвигл еси, и множество инок собрал еси, и их наставил еси на путь истинный. Темже и Христос, Емуже в жертву себе принесл еси, видев твоя благия труды, яко пресветла тя светильника, и по преставлении чудесы обогати, Савво, отче наш. Молися убо присно Ему о стране нашей, еже дароватися ей на враги победам, стаду же твоему от враг ненаветну пребыти, и всем нам спастися.


\mysubsubsection{Кондак, глас 2-й:}


Чистотою душевною Божественне вооружився, и непрестанныя молитвы яко копие вручив крепко, пробол еси бесовская ополчения, Савве, отче наш, моли непрестанно о всех нас.


\mysubsubsection{Молитва:}


Преподобне отче Савво, пустынный подвижниче, в телеси ангеле, воине Христов доблественне, веры правило, церкве утверждение, благочестия образе, добродетелей светило, трудов, поста, воздержания, чистоты и целомудрия предводителю, помощниче всем нуждающимся, странствующим вождю, плавающим кормчию, обуреваемым пристанище, ослабевающим помоще, обидимым заступление, печальным утешение, вдовицам и сирым покрове, алчущим питание, нагим одеяние, плачущим утешение, болящим исцеление, ненадеющимся надеяние, христиан всех ходатайство и защищение; призри милостивно на ны, припадающия к святым мощем твоим, внемли молитвам нашим и подаждь коемуждо от нас благопотребная, просимая от тебе. Вемы, святче Божий, яко имаше дерзновение ко Владыце Христу Богу, и елика аще воспросиши у Него дастся нам. Аще же и недостойни есмы благодати Его, обаче твоя молитва да сотворит ны ея достойны, много бо может прилежная молитва праведного якоже рече апостол. Сам убо ты воздвигни нас из глубины греховныя, неверныя просвети, еретики обрати, заблуждшия в нас от православия к познанию истинныя веры настави, всей же Церкви исходатайствуй мир и святыню, едимомыслие в вере, умов благопостояние, соревнование в благочестии, единодушие в братолюбии, верность к Отечеству, родителем благопопечение о чадех, чадом же страх и послушание, девствующим чистоту и целомудрие, брачным же верность н непорочность ложа, всем же грешным покаяние, кающимся прощение, в злоключениих же терпение и упование, в надежде же спасение даруй. Избави же всех нас от глада, губительства, от труса и потопа, от огня и меча, от нахождения иноплеменных и междоусобныя брани, и от нападения вражия, и от тлетворных ветр, и от смертоносныя язвы, и от напрасные смерти, и от всякаго зла, да тако вси тобою просвещаеми, наставляеми, утешаеми, утверждаеми и сохраняеми, славим прославившаго тя Бога, Отца и Сына и Святаго Духа и твое предстательство, всегда, ныне и присно и во веки веков. Аминь.


\section{Святителю Феодосию, архиепископу Углицкому и  Черниговскому}
%http://www.molitvoslov.com/text795.htm 
 


\mysubsubsection{Тропарь, глас 4-й:}


Преудобрен во архиереех, святителю Феодосие, был еси светило своему стаду, таже преставился еси в вечныя oбители. Умоли у Престола Царя cлавы избавитися нам от находящих на ны зол и спастися душам нашим, святе, молитвами твоими.


\mysubsubsection{Кондак, глас 4-й:}


Пастырей Начальнику Христу трудился еси, святителю Феодосие, на пажити духовней питая словесныя твоя овцы, и приял еси от Христа Спаса целебен дар целити от немощи душевныя и телесныя всякого с верою к тебе приходящаго к целебным твоим мощем. Моли, святе, о призывающих имя твое, от наветов вражиих спастися душам нашим.


\mysubsubsection{Молитва:}


Пастырю добрый стада Христова, святителю отче наш Феодосие! Припадающе ко святей иконе твоей, молим тя с верою и любовию: буди нам помощник и покровитель в житии сем земнем многоскорбнем и многомятежнем, руководствуя нас невидимо к верному исполнению заповедей Христовых и стяжанию добродетелей Евангельских, нужных ко спасению душевному нашему. Буди заступник и охранитель Отечеству нашему, граду твоему, в немже святыми мощами твоими почивати благоизволил еси, и всем нам к тебе усердно притекающим и тя на помощь призывающим, ограждая нас покровом молитв твоих от всяких бед и скорбей и болезней, и даруя нам вся, яже на пользу душ и телес наших. Ей чудотворче святый и милостивый! Веруем несомненно, яко вся сия можеши нам испросити  у великодаровитаго Бога нашего и яко мощно есть предстательство твое пред лицем Его. Сего ради просим тя: умоли Его благость за ны грешныя и недостойныя, и подажть нам твое святое архипастырское благословление, да оным осеняеми, тихое и богоугодное житие поживем, благую христианскую кончину получити сподобимся и вкупе ц тобою Царствие Небесное наследуем, славяще великое милосердие Отца и Сына и Святаго Духа, в Троице славимаго и покланяемого Бога, и твое святое заступление, во веки веков. Аминь.


\section{Mученику Трифону}
%http://www.molitvoslov.com/text794.htm 
 


%\medskip\bfseries Тропарь, глас 8-й:\normalfont{}\nopagebreak

\subsubsection{Тропарь, глас 8-й:}


Мученик Твой, Господи, Трифон во страдании своем венец прият нетленный от Тебе, Бога нашего: имеяй бо крепость Твою, мучителей низложи, сокруши и демонов немощныя дерзости. Того молитвами спаси души наша.


\mysubsubsection{Кондак, глас 8-й:}


Троическою твердостию многобожие разрушил еси от конец, всеславне, честен во Христе быв, и, победив мучители во Христе Спасителе, венец приял еси мученичества твоего и дарования Божественных исцелений, яко непобедим.


\mysubsubsection{Молитва:}


О, святый мучениче Христов Трифоне, скорый помощниче всем, к тебе прибегающим, и молящимся пред святым твоим образом, скоропослушный предстателю! Услыши убо ныне и на всякий час моление нас, недостойных рабов твоих, почитающих святую память твою, и предстательствуй о нас пред Господем на всяком месте, ты бо, угодниче Христов, святый мучениче и чудотворче Трифоне, в великих чудесех возсиявый, прежде исхода твоего от жития сего тленнаго молился еси ко Господу и испросил у Него дар сей: аще кто в коей-либо нужде, беде, печали и болезни душевней или телесней призывати начнет святое имя твое, той да избавлен будет от всякаго прилога злаго, и якоже ты иногда дщерь цареву,  во граде Риме от диавола мучиму, исцелил еси, сице и нас от лютых его козней сохрани во вся дни живота нашего, наипаче же в день страшный последняго нашего издыхания предстательствуй о нас, буди нам тогда помошник, и скорый прогонитель лукавых духов, и к Царствию Небесному предводитель, идеже ты ныне предстоиши с лики святых у Престола Божия, моли Господа, да сподобит и нас причастниками быти присносущнаго веселия и радости, да с тобою купно удостоимся славити Отца и Сына и Святаго Утешителя Духа во веки веков. Аминь.


\section{Преподобному Иринарху Ростовскому}
%http://www.molitvoslov.com/text793.htm 
 


\mysubsubsection{Тропарь, глас 4-й:}


Яко мученика добропроизвольна и преподобных удобрение, звезду Ростовскую, в затворе, узах и веригах Господеви благоугодившаго и чудес благодать от Него приемшаго, Иринарха предивнаго песньми хвалебными почтим и, к нему припадающе, умильно глаголем: отче преподобне, моли Христа Бога спастися душам нашим.


\mysubsubsection{Кондак, глас 2-й:}


Волнений множество жестоким житием преходя, изгнания, затвор и узы железныя претерпел еси мужески, Иринарше терпеливодушне, нам оставль образ злострадания и терпения твоего, озаряя чудес блистаньми с верою приходящих к честному гробу твоему, у негоже, яко почесть победную, вериги твоя тяжкия видим, от нихже подаеши исцеления недужным. Сего ради зовем ти: радуйся, Иринарше, отче предивный.


\mysubsubsection{Молитва:}


Преподобный Иринарх, великий в подвигах! Много обид от иноков монастыря, в котором ты жил, и преломление руки ты претерпел. Часто будучи изгоняем, все это кротко и любовно переносил ты Бога ради. В затворе, приковав себя цепями и обложив веригами, много лет пребыл ты, досточудный, и трудом рук твоих заработанные деньги ты раздавал нищим. Христу сраспявшись, изнурял ты тело свое, дух же укрепил и очистил, и власть против бесов от Бога принял и дар врачевать бесноватых. Поэтому, припадая к тебе, с верою и надеждою молим тебя, святый угодник Божий: прогони духов зла из людей, наказуя бесов, дерзнувших войти в создания Божии. Умилосердись, Иринарх проподобный, над несчастными и безумными, и исцели их, чтобы выздоровев, в добродетели течение жизни своей совершили, прославляя дивного во святых Своих милостивого Бога, теперь и всегда, и во веки веков. Аминь.


\section{Святителю Иоанну Новгородскому}
%http://www.molitvoslov.com/text789.htm 
 



\mysubsubsection{Тропарь, глас 8-й:}


  Днесь светло красуется славнейший великий Новаград, имея мощи твоя в себе, святителю Иоанне, яко солнечныя лучи испущающия, и подающия исцеления притекающим с верою к раце мощей твоих: молися Христу Богу, избавити град сей невредим от варварскаго пленения, и междоусобныя брани, и огненнаго запаления, святителю богомудре и чудоносче, небесный человече и земный ангеле: да сошедшеся любовию в память твою, светло празднуем, в песнех и пениих радующеся и Христа славяще, тебе таковую благодать даровавшаго исцелений, и Bеликому Новуграду заступление и утверждение. 


\mysubsubsection{Кондак, глас 4-й:}


  Возвеселися явленно Честная Церковь Христова, в память днесь приснословущаго святителя Иоанна, от великаго Новаграда возсиявшаго: и всю страну удивившаго преславными чудодеянии, всеми добродетельми украсившагося. И по преставлении бо честное тело его обретеся нетленно, источающее велия чудеса. Темже зовем ему: о, всеблаженне! Моли Христа Бога непрестанно о всех нас. 


\mysubsubsection{Славник на стиховне, глас 6-й:}


 Чистое житие твое похваляем, пречестнейший Иоанне блаженне: сосуд избранный Божий, муж желаний духовных, и Духа Святаго был еси обиталище: духи же нечистыя поправ, и сих без вести сотворил еси. И по преставлении твоем к Богу, яко жив являяся, и о полезных наказуя. И ныне предстоя Святей Троице молися, мир миру даровати, и душам нашим велию милость.


\mysubsubsection{Молитва:    }


О, пречестная и священная главо и благодати Святаго Духа исполненная, Спасово со Отцем обиталище, великий архиерее, теплый наш заступниче, святителю Иоанне, предстоя у Престола всех Царя и наслаждаяся света Единосущныя Троицы и херувимски со ангелы возглашая песнь трисвятую, великое же и неизследованное дерзновение имея ко Всемилостивому Владыце, моли спастися паствы Христовы людем, благостояние святых церквей утверди, архиереи благолепием святительства украси, монашествующия к подвигом добраго течения укрепи, царствующий град и вся грады страны добре сохрани и веру святую непорочну соблюсти умоли, мир весь предстательством твоим умири, от глада и пагубы избави ны, и от нападения иноплеменных сохрани, старыя утеши, юныя настави, безумныя умудри, вдовицы помилуй, сироты заступи, младенцы возрасти, плененныя возврати, немощствующия исцели, и везде тепле призывающия тя и с верою припадающия и молящияся тебе от всяких напастей и бед ходатайством твоим свободи, моли о нас Всещедраго и Человеколюбиваго Христа Бога нашего, да и в день страшнаго пришествия Его от шуияго стояния избавит нас и радости святых причастники сотворит со всеми святыми во веки веков. Аминь.


\section{Преподобному Макарию Великому, Египетскому}
%http://www.molitvoslov.com/text788.htm 
 


\mysubsubsection{Тропарь, глас 1-й:}


Пустынный житель, и во плоти Ангел, и чудотворец явился еси, Богоносне отче наш Макарие: постом, бдением, молитвою Небесныя дарования приим, исцеляеши недужныя и души верою приходящих ти. Слава Давшему тебе крепость, слава Венчавшему тя, слава Действующему тобою всем исцеления.


\mysubsubsection{Кондак, глас 1-й:}


Блаженную жизнь скончав в житии с мученическими лики, в земли кротких достойно водворишися, богоносне Макарие, и пустыню, якоже град, населив, благодать приял еси от Бога чудес, темже тя почитаем.


\mysubsubsection{Молитва:}


О, преподобне отче Макарие!  Молим тя мы, недостойные, испроси предстательством твоим у Всемилостиваго Бога нашего нам здравие душевное и телесное, тихое и богоугодное житие  и добрый ответ на Страшнем Суде Христове. Угаси молитвами твоими разжжены стрелы диавольския, да не прекоснется нам злоба греховная, да благочестно скончавше временное житие, сподобимся наследовати Царствие Небесное и купно с тобою славити Отца и Сына и Святаго Духа во веки веков. Аминь.
\mychapterending

\mychapter{При зависти ближнему}
%http://www.molitvoslov.com/content/Pri-zavisti-blizhnemu

 

\section{При зависти ближнему и сетовании о своих неудачах}
%http://www.molitvoslov.com/content/Pri-zavisti-blizhnemu-i-setovanii-o-svoikh-neudachakh 
 


О Ты, преблагий, прещедрый, Господи Иисусе Христе! Всякое благо приходит от Тебя и чрез Тебя, из сокровища Твоего вечнаго непреходящаго богатства; Ты уделяешь каждому свое по воле Твоей. Не допусти мне чрез вредоносную зависть сравняться с диаволом. Излей в сердце мое Твою благость, Твою любовь, Твою верность, чтобы я сердечно радовался дарам Твоим, которые Ты из щедрой благости разделяешь между нами, и Твоему милосердию над нами; чтобы мы не отягощали бы друг друга завистью, ложью, хулою и клеветою, но чтобы мы все, Тобою нам данное, употребляли к хвале, чести и славе Твоей, познавали Тебя в Твоих благодеяниях и хвалили, чтили и прославляли во всю вечность.
\mychapterending
