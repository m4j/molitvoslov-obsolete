

\mypart{МОЛИТВЫ}\label{_content_molitvi}
%http://www.molitvoslov.com/content/molitvi

\mychapter{Молитва Господня. Отче наш}\begin{mymulticols}

%http://www.molitvoslov.com/node/37

\myfigure{Spasnatrone}

Отче наш, Иже еси на небесех! Да святится имя Твое, да приидет Царствие Твое, да будет воля Твоя, яко на небеси и на земли. Хлеб наш насущный даждь нам днесь; и остави нам долги наша, якоже и мы оставляем должником нашим; и не введи нас во искушение, но избави нас от лукаваго.

Отче наш, сущий на небесах! да святится имя Твое; да приидет Царствие Твое; да будет воля Твоя и на земле, как на небе; хлеб наш насущный дай нам на сей день; и прости нам долги наши, как и мы прощаем должникам нашим; и не введи нас в искушение, но избавь нас от лукавого. Ибо Твое есть Царство и сила и слава во веки. Аминь. (Матф. 6:9--13)


\end{mymulticols}

\mychapterending


\mychapter{Иисусова молитва}
%http://www.molitvoslov.com/text596.htm

\myfigure[0.5]{456}

\begin{center}
Господи Иисусе Христе, Сыне Божий, помилуй мя, грешнаго.
\end{center}

\mychapterending


\mychapter{Благодарение за всякое благодеяние Божие}\begin{mymulticols}
%http://www.molitvoslov.com/text889.htm

\myfigure{777}

\mysubsubsection{Тропарь, глас 4-й}

Благодарни суще недостойнии раби Твои, Господи, о Твоих великих благодеяниих на нас бывших, славяще Тя хвалим, благословим, благодарим, поем и величаем Твое благоутробие, и рабски любовию вопием Ти: Благодетелю Спасе наш, слава Тебе.

\mysubsubsection{Кондак, глас 3-й}

Твоих благодеяний и даров туне, яко раби непотребнии, сподобльшеся, Владыко, к Тебе усердно притекающе, благодарение по силе приносим, и Тебе яко Благодетеля и Творца славяще, вопием: слава Тебе, Боже Всещедрый.

\slavainynen

\Bogorodichen{Богородице, христианом Помощнице, Твое предстательство стяжавше раби Твои, благодарно Тебе вопием: радуйся, Пречистая Богородице Дево, и от всех нас бед Твоими молитвами всегда избави, Едина вскоре предстательствующая.}

\end{mymulticols}

\mychapterending


\mychapter{Молитвы утренние}\begin{mymulticols}
%http://www.molitvoslov.com/text893.htm

\myfigure{795}

\footnote{Напечатанное \emph{курсивом} (\myemph{пояснения} и \myemph{\textbf{названия молитв}}) не читается во время молитвы.}\myemph{  Востав от сна, прежде всякого другого дела, стань благоговейно, представляя себя пред Всевидящим Богом, и, совершая крестное знамение, произнеси:}

Во имя Отца, и Сына, и Святаго Духа, Аминь.

\medskip\myemph{ Затем немного подожди, пока все чувства твои не придут в тишину и мысли твои не оставят все земное, и тогда произноси следующие молитвы, без поспешности и со вниманием сердечным:}

\mysubsubsection{Молитва мытаря \myemph{(Евангелие от Луки, глава 18, стих 13)}}

Боже, милостив буди мне грешному. \myemph{ (Поклон)}

\mysubsubsection{Молитва предначинательная}

Господи Иисусе Христе, Сыне Божий, молитв ради Пречистыя Твоея Матере и всех святых, помилуй нас. Аминь.

Слава Тебе, Боже наш, слава Тебе.

\mysubsubsection{Молитва Святому Духу}

Царю Небесный, Утешителю, Душе истины, Иже везде сый и вся исполняяй, Сокровище благих и жизни Подателю, прииди и вселися в ны, и очисти ны от всякия скверны, и спаси, Блаже, души наша\footnote{От Пасхи до Вознесения вместо этой молитвы читается тропарь: «Христос воскресе из мертвых, смертию смерть поправ, и сущим во гробех живот даровав». \myemph{(Трижды)} От Вознесения до Троицы начинаем молитвы со «Святый Боже…», опуская все предшествующие.

Это замечание относится и к молитвам на сон грядущим.
}.

\mysubsubsection{Трисвятое}

Святый Боже, Святый Крепкий, Святый Безсмертный, помилуй нас. \myemph{ (Читается трижды, с крестным знамением и поясным поклоном.) }

Слава Отцу и Сыну и Святому Духу, и ныне и присно и во веки веков. Аминь.

\mysubsubsection{Молитва ко Пресвятой Троице}

Пресвятая Троице, помилуй нас; Господи, очисти грехи наша; Владыко, прости беззакония наша; Святый, посети и исцели немощи наша, имене Твоего ради.

Господи, помилуй. \myemph{ (Трижды)}.

Слава Отцу и Сыну и Святому Духу, и ныне и присно и во веки веков. Аминь\footnote{Когда написано «Слава», «И ныне», надо читать полностью: «Слава Отцу и Сыну и Святому Духу», «И ныне и присно и во веки веков. Аминь»}.

\mysubsubsection{Молитва Господня}

Отче наш, Иже еси на небесех! Да святится имя Твое, да приидет Царствие Твое, да будет воля Твоя, яко на небеси и на земли. Хлеб наш насущный даждь нам днесь; и остави нам долги наша, якоже и мы оставляем должником нашим; и не введи нас во искушение, но избави нас от лукаваго.

\mysubsubsection{Тропари Троичные}

Воставше от сна, припадаем Ти, Блаже, и ангельскую песнь вопием Ти, Сильне: Свят, Свят, Свят еси, Боже, Богородицею помилуй нас.

\slavan

 От одра и сна воздвигл мя еси, Господи, ум мой просвети и сердце, и устне мои отверзи, во еже пети Тя, Святая Троице: Свят, Свят, Свят еси, Боже, Богородицею помилуй нас.

\inynen

 Внезапно Судия приидет, и коегождо деяния обнажатся, но страхом зовем\footnote{В церковнославянском языке нет звука ё, а поэтому надо читать «зовем», а не «зовём», «твое», а не «твоё», «мое», а не «моё» и т.~д.} в полунощи: Свят, Свят, Свят еси, Боже, Богородицею помилуй нас.

Господи, помилуй. \myemph{ (12 раз)}

\mysubsubsection{Молитва ко Пресвятой Троице}

От сна востав, благодарю Тя, Святая Троице, яко многия ради Твоея благости и долготерпения не прогневался еси на мя, лениваго и грешнаго, ниже погубил мя еси со беззаконьми моими; но человеколюбствовал еси обычно и в нечаянии лежащаго воздвигл мя еси, во еже утреневати и славословити державу Твою. И ныне просвети мои очи мысленныя, отверзи моя уста поучатися словесем Твоим, и разумети заповеди Твоя, и творити волю Твою, и пети Тя во исповедании сердечнем, и воспевати всесвятое имя Твое, Отца и Сына и Святаго Духа, ныне и присно и во веки веков. Аминь.

Приидите, поклонимся Цареви нашему Богу. \myemph{ (Поклон)}

Приидите, поклонимся и припадем Христу, Цареви нашему Богу. \myemph{ (Поклон)}

Приидите, поклонимся и припадем Самому Христу, Цареви и Богу нашему. \myemph{ (Поклон)}

\mysubsubsection{Псалом 50}

\PsalmFifty

\mysubsubsection{Символ веры}

\SymbolOfFaith
  
\mysubsubsection{Молитва первая, святого Макария Великого}

Боже, очисти мя грешнаго, яко николиже сотворих благое пред Тобою; но избави мя от лукаваго, и да будет во мне воля Твоя, да неосужденно отверзу уста моя недостойная и восхвалю имя Твое святое, Отца и Сына и Святаго Духа, ныне и присно и во веки веков Аминь.

\mysubsubsection{ Молитва вторая, того же святого}

От сна востав, полунощную песнь приношу Ти, Спасе, и припадая вопию Ти: не даждь ми уснути во греховней смерти, но ущедри мя, распныйся волею, и лежащаго мя в лености ускорив возстави, и спаси мя в предстоянии и молитве, и по сне нощнем возсияй ми день безгрешен, Христе Боже, и спаси мя.

\mysubsubsection{Молитва третья, того же святого}

К Тебе, Владыко Человеколюбче, от сна востав прибегаю, и на дела Твоя подвизаюся милосердием Твоим, и молюся Тебе: помози ми на всякое время, во всякой вещи, и избави мя от всякия мирския злыя вещи и диавольскаго поспешения, и спаси мя, и введи в Царство Твое вечное. Ты бо еси мой Сотворитель и всякому благу Промысленник и Податель, о Тебе же все упование мое, и Тебе славу возсылаю, ныне и присно и во веки веков. Аминь.

\mysubsubsection{Молитва четвертая, того же святого}

Господи, Иже многою Твоею благостию и великими щедротами Твоими дал еси мне, рабу Твоему, мимошедшее время нощи сея без напасти прейти от всякаго зла противна; Ты Сам, Владыко, всяческих Творче, сподоби мя истинным Твоим светом и просвещенным сердцем творити волю Твою, ныне и присно и во веки веков. Аминь.

\mysubsubsection{Молитва пятая, святого Василия Великого}

Господи Вседержителю, Боже сил и всякия плоти, в вышних живый и на смиренныя призираяй, сердца же и утробы испытуяй и сокровенная человеков яве предведый, Безначальный и Присносущный Свете, у Него же несть пременение, или преложения осенение; Сам, Безсмертный Царю, приими моления наша, яже в настоящее время, на множество Твоих щедрот дерзающе, от скверных к Тебе устен творим, и остави нам прегрешения наша, яже делом, и словом, и мыслию, ведением, или неведением согрешенная нами; и очисти ны от всякия скверны плоти и духа. И даруй нам бодренным сердцем и трезвенною мыслию всю настоящаго жития нощь прейти, ожидающим пришествия светлаго и явленнаго дне Единороднаго Твоего Сына, Господа и Бога и Спаса нашего Иисуса Христа, в оньже со славою Судия всех приидет, комуждо отдати по делом его; да не падше и обленившеся, но бодрствующе и воздвижени в делание обрящемся готови, в радость и Божественный чертог славы Его совнидем, идеже празднующих глас непрестанный, и неизреченная сладость зрящих Твоего лица доброту неизреченную. Ты бо еси истинный Свет, просвещаяй и освящаяй всяческая, и Тя поет вся тварь во веки веков. Аминь.

\mysubsubsection{Молитва шестая, того же святого}

Тя благословим, вышний Боже и Господи милости, творящаго присно с нами великая же и неизследованная, славная же и ужасная, ихже несть числа, подавшаго нам сон во упокоение немощи нашея, и ослабление трудов многотрудныя плоти. Благодарим Тя, якo не погубил еси нас со беззаконьми нашими, но человеколюбствовал еси обычно, и в нечаянии лежащия ны воздвигл еси, во еже славословити державу Твою. Темже молим безмерную Твою благость, просвети наша мысли, очеса, и ум наш от тяжкаго сна лености возстави: отверзи наша уста, и исполни я Твоего хваления, яко да возможем непоколеблемо пети же и исповедатися Тебе, во всех, и от всех славимому Богу, Безначальному Отцу, со Единородным Твоим Сыном, и Всесвятым и Благим и Животворящим Твоим Духом, ныне и присно и во веки веков. Аминь.

\mysubsubsection{Молитва седьмая, ко Пресвятой Богородице}

Воспеваю благодать Твою, Владычице, молю Тя, ум мой облагодати. Ступати право мя настави, путем Христовых заповедей. Бдети к песни укрепи, уныния сон отгоняющи. Связана пленицами грехопадений, мольбами Твоими разреши, Богоневесто. В нощи мя и во дни сохраняй, борющих враг избавляющи мя. Жизнодателя Бога рождшая, умерщвлена мя страстьми оживи. Яже Свет невечерний рождшая, душу мою ослепшую просвети. О дивная Владычня палато, дом Духа Божественна мене сотвори. Врача рождшая, уврачуй души моея многолетныя страсти. Волнующася житейскою бурею, ко стези мя покаяния направи. Избави мя огня вечнующаго, и червия же злаго, и тартара. Да мя не явиши бесом радование, иже многим грехом повинника. Нова сотвори мя, обетшавшаго нечувственными, Пренепорочная, согрешении. Странна муки всякия покажи мя, и всех Владыку умоли. Небесная ми улучити веселия, со всеми святыми, сподоби. Пресвятая Дево, услыши глас непотребнаго раба Твоего. Струю давай мне слезам, Пречистая, души моея скверну очищающи. Стенания от сердца приношу Ти непрестанно, усердствуй, Владычице. Молебную службу мою приими, и Богу благоутробному принеси. Превышшая Ангел, мирскаго мя превышша слития сотвори. Светоносная Сене небесная, духовную благодать во мне направи. Руце воздею и устне к похвалению, осквернены скверною, Всенепорочная. Душетленных мя пакостей избави, Христа прилежно умоляющи; Емуже честь и поклонение подобает, ныне и присно и во веки веков. Аминь.

\mysubsubsection{Молитва восьмая, ко Господу нашему Иисусу Христу}

Многомилостиве и Всемилостиве Боже мой, Господи Иисусе Христе, многия ради любве сшел и воплотился еси, яко да спасеши всех. И паки, Спасе, спаси мя по благодати, молю Тя; аще бо от дел спасеши мя, несть се благодать, и дар, но долг паче. Ей, многий в щедротах и неизреченный в милости! Веруяй бо в Мя, рекл еси, о Христе мой, жив будет и не узрит смерти во веки. Аще убо вера, яже в Тя, спасает отчаянныя, се верую, спаси мя, яко Бог мой еси Ты и Создатель. Вера же вместо дел да вменится мне, Боже мой, не обрящеши бо дел отнюд оправдающих мя. Но та вера моя да довлеет вместо всех, та да отвещает, та да оправдит мя, та да покажет мя причастника славы Твоея вечныя. Да не убо похитит мя сатана, и похвалится, Слове, еже отторгнути мя от Твоей руки и ограды; но или хощу, спаси мя, или не хощу, Христе Спасе мой, предвари скоро, скоро погибох: Ты бо еси Бог мой от чрева матере моея. Сподоби мя, Господи, ныне возлюбити Тя, якоже возлюбих иногда той самый грех; и паки поработати Тебе без лености тощно, якоже поработах прежде сатане льстивому. Наипаче же поработаю Тебе, Господу и Богу моему Иисусу Христу, во вся дни живота моего, ныне и присно и во веки веков. Аминь.

\mysubsubsection{Молитва девятая, к Ангелу хранителю}

Святый Ангеле, предстояй окаянной моей души и страстной моей жизни, не остави мене грешнаго, ниже отступи от мене за невоздержание мое. Не даждь места лукавому демону обладати мною, насильством смертнаго сего телесе; укрепи бедствующую и худую мою руку и настави мя на путь спасения. Ей, святый Ангеле Божий, хранителю и покровителю окаянныя моея души и тела, вся мне прости, еликими тя оскорбих во вся дни живота моего, и аще что согреших в прешедшую нощь сию, покрый мя в настоящий день, и сохрани мя от всякаго искушения противнаго, да ни в коем гресе прогневаю Бога, и молися за мя ко Господу, да утвердит мя в страсе Своем, и достойна покажет мя раба Своея благости. Аминь.

\mysubsubsection{Молитва десятая, ко Пресвятой Богородице}

Пресвятая Владычице моя Богородице, святыми Твоими и всесильными мольбами отжени от мене, смиреннаго и окаяннаго раба Твоего, уныние, забвение, неразумие, нерадение, и вся скверная, лукавая и хульная помышления от окаяннаго моего сердца и от помраченнаго ума моего; и погаси пламень страстей моих, яко нищ есмь и окаянен. И избави мя от многих и лютых воспоминаний и предприятий, и от всех действ злых свободи мя. Яко благословена еси от всех родов, и славится пречестное имя Твое во веки веков. Аминь.

\mysubsubsection{Молитвенное призывание святого, имя которого носишь}

Моли Бога о мне, святый угодниче Божий \myemph{ (имя)}, яко аз усердно к тебе прибегаю, скорому помощнику и молитвеннику о душе моей.

\mysubsubsection{Песнь Пресвятой Богородице}

Богородице Дево, радуйся, Благодатная Марие, Господь с Тобою; благословена Ты в женах и благословен плод чрева Твоего, яко Спаса родила еси душ наших.

\mysubsubsection{Тропарь Кресту и молитва за отечество}

Спаси, Господи, люди Твоя, и благослови достояние Твое, победы православным христианом на сопротивныя даруя, и Твое сохраняя Крестом Твоим жительство.

\mysubsubsection{Молитва о живых}

Спаси, Господи, и помилуй отца моего духовнаго \myemph{ (имя)}, родителей моих \myemph{ (имена)}, сродников \myemph{ (имена)}, начальников, наставников, благодетелей \myemph{ (имена их)} и всех православных христиан.

\mysubsubsection{Молитва о усопших}

Упокой, Господи, души усопших раб Твоих: родителей моих, сродников, благодетелей \myemph{(имена их)}, и всех православных христиан, и прости им вся согрешения вольная и невольная, и даруй им Царствие Небесное.

\myemph{ Если можешь, читай вместо кратких молитв о живых и усопших этот помянник:}

\mysubsubsection{О живых}

Помяни, Господи Иисусе Христе, Боже наш, милости и щедроты Твоя от века сущия, ихже ради и вочеловечился еси, и распятие и смерть, спасения ради право в Тя верующих, претерпети изволил еси; и воскрес из мертвых, вознеслся еси на небеса и седиши одесную Бога Отца, и призираеши на смиренныя мольбы всем сердцем призывающих Тя: приклони ухо Твое, и услыши смиренное моление мене, непотребнаго раба Твоего, в воню благоухания духовнаго, Тебе за вся люди Твоя приносящаго. И в первых помяни Церковь Твою Святую, Соборную и Апостольскую, юже снабдел еси честною Твоею Кровию, и утверди, и укрепи, и разшири, умножи, умири, и непреобориму адовыми враты во веки сохрани; раздирания Церквей утиши, шатания языческая угаси, и ересей востания скоро разори и искорени, и в ничтоже силою Святаго Твоего Духа обрати. \myemph{ (Поклон)}

Спаси, Господи, и помилуй Богом хранимую страну нашу, власти и воинство ея, огради миром державу их, и покори под нозе Православных всякаго врага и супостата, и глаголи мирная и благая в сердцах их о Церкви Твоей Святей, и о всех людех Твоих: да тихое и безмолвное житие поживем во правоверии, и во всяком благочестии и чистоте. \myemph{ (Поклон)}

Спаси, Господи, и помилуй Великаго Господина и Отца нашего Святейшего Патриарха Кирилла, преосвященныя митрополиты, архиепископы и епископы православныя, иереи же и диаконы, и весь причет церковный, яже поставил еси пасти словесное Твое стадо, и молитвами их помилуй и спаси мя грешнаго. \myemph{ (Поклон)}

Спаси, Господи, и помилуй отца моего духовнаго \myemph{ (имя его)}, и святыми его молитвами прости моя согрешения. \myemph{ (Поклон)}

Спаси, Господи, и помилуй родители моя \myemph{ (имена их)}, братию и сестры, и сродники моя по плоти, и вся ближния рода моего, и други, и даруй им мирная Твоя и премирная благая. \myemph{ (Поклон)}

Спаси, Господи, и помилуй по множеству щедрот Твоих вся священноиноки, иноки же и инокини, и вся в девстве же и благоговении и постничестве живущия в монастырех, в пустынях, в пещерах, горах, столпех, затворех, разселинах каменных, островех же морских, и на всяком месте владычествия Твоего правоверно живущия, и благочестно служащия Ти, и молящияся Тебе: облегчи им тяготу, и утеши их скорбь, и к подвигу о Тебе силу и крепость им подаждь, и молитвами их даруй ми оставление грехов. \myemph{ (Поклон)}

Спаси, Господи, и помилуй старцы и юныя, нищия и сироты и вдовицы, и сущия в болезни и в печалех, бедах же и скорбех, обстояниих и пленениих, темницах же и заточениих, изряднее же в гонениих, Тебе ради и веры православныя, от язык безбожных, от отступник и от еретиков, сущия рабы Твоя, и помяни я, посети, укрепи, утеши, и вскоре силою Твоею ослабу, свободу и избаву им подаждь. \myemph{ (Поклон) }

Спаси, Господи, и помилуй благотворящия нам, милующия и питающия нас, давшия нам милостыни, и заповедавшия нам недостойным молитися о них, и упокоевающия нас, и сотвори милость Твою с ними, даруя им вся, яже ко спасению прошения, и вечных благ восприятие. \myemph{ (Поклон)}

Спаси, Господи, и помилуй посланныя в службу, путешествующия, отцы и братию нашу, и вся православныя христианы. \myemph{ (Поклон)}

Спаси, Господи, и помилуй ихже аз безумием моим соблазних, и от пути спасительнаго отвратих, к делом злым и неподобным приведох; Божественным Твоим Промыслом к пути спасения паки возврати. \myemph{ (Поклон) }

Спаси, Господи, и помилуй ненавидящия и обидящия мя, и творящия ми напасти, и не остави их погибнути мене ради, грешнаго. \myemph{ (Поклон)}

Отступившия от православныя веры и погибельными ересьми ослепленныя, светом Твоего познания просвети и Святей Твоей Апостольстей Соборней Церкви причти. \myemph{ (Поклон) }

\mysubsubsection{О усопших}

Помяни, Господи, от жития сего отшедшия правоверныя цари и царицы, благоверныя князи и княгини, святейшия патриархи, преосвященныя митрополиты, архиепископы и епископы православныя, во иерейстем же и в причте церковнем, и монашестем чине Тебе послужившия, и в вечных Твоих селениих со святыми упокой. \myemph{ (Поклон.)}

Помяни, Господи, души усопших рабов Твоих, родителей моих \myemph{ (имена их)}, и всех сродников по плоти; и прости их вся согрешения вольная и невольная, даруя им Царствие и причастие вечных Твоих благих и Твоея безконечныя и блаженныя жизни наслаждение. \myemph{ (Поклон) }

Помяни, Господи, и вся в надежди воскресения и жизни вечныя усопшия, отцы и братию нашу, и сестры, и зде лежащия и повсюду, православныя христианы, и со святыми Твоими, идеже присещает свет лица Твоего, всели, и нас помилуй, яко Благ и Человеколюбец. Аминь. \myemph{ (Поклон) }

Подаждь, Господи, оставление грехов всем прежде отшедшим в вере и надежди воскресения, отцем, братиям и сестрам нашим и сотвори им вечную память. \myemph{ (Трижды)}

\mysubsubsection{Окончание молитв}

Достойно есть яко воистину блажити Тя Богородицу, Присноблаженную и Пренепорочную и Матерь Бога нашего. Честнейшую Херувим и славнейшую без сравнения Серафим, без истления Бога Слова рождшую, сущую Богородицу Тя величаем\footnote{От Пасхи до Вознесения вместо этой молитвы читается припев и ирмос 9-й песни пасхального канона:

\begin{quote}

«Ангел вопияше Благодатней: Чистая Дево, радуйся! И паки реку: радуйся! Твой Сын воскресе тридневен от гроба и мертвыя воздвигнувый; людие, веселитеся!

Светися, светися, новый Иерусалиме, слава бо Господня на тебе возсия. Ликуй ныне и веселися, Сионе. Ты же, Чистая, красуйся, Богородице, о востании Рождества Твоего».

\end{quote}

Это замечание относится и к вечерним молитвам. }.

\slavainynen

 Господи, помилуй. \myemph{ (Трижды)}

Господи, Иисусе Христе, Сыне Божий, молитв ради Пречистыя Твоея Матере, преподобных и богоносных отец наших и всех святых помилуй нас. Аминь.

\end{mymulticols}

\mychapterending


\mychapter{Молитвы на сон грядущим}\begin{mymulticols}
%http://www.molitvoslov.com/text2.htm

\myfigure{1_1}

Во имя Отца, и Сына, и Святаго Духа. Аминь.

Господи Иисусе Христе, Сыне Божий, молитв ради Пречистыя Твоея Матере, преподобных и богоносных отец наших и всех святых, помилуй нас. Аминь.

Слава Тебе, Боже наш, слава Тебе.

Царю Небесный, Утешителю, Душе истины, Иже везде сый и вся исполняяй, Сокровище благих и жизни Подателю, прииди и вселися в ны, и очисти ны от всякия скверны, и спаси, Блаже, души наша.

Святый Боже, Святый Крепкий, Святый Безсмертный, помилуй нас.\myemph{  (Трижды)}

Слава Отцу и Сыну и Святому Духу, и ныне и присно и во веки веков. Аминь.

Пресвятая Троице, помилуй нас; Господи, очисти грехи наша; Владыко, прости беззакония наша; Святый, посети и исцели немощи наша, имене Твоего ради.

Господи, помилуй. \myemph{ (Трижды)}

Слава Отцу и Сыну и Святому Духу, и ныне и присно и во веки веков. Аминь.

Отче наш, Иже еси на небесех! Да святится имя Твое, да приидет Царствие Твое, да будет воля Твоя, яко на небеси и на земли. Хлеб наш насущный даждь нам днесь; и остави нам долги наша, якоже и мы оставляем должником нашим; и не введи нас во искушение, но избави нас от лукаваго.

\mysubsubsection{Тропари}

\TroparPomilujNas

Господи, помилуй. \myemph{ (12 раз)}

\mysubsubsection{Молитва 1-я, святого Макария Великого, к Богу Отцу}

Боже вечный и Царю всякаго создания, сподобивый мя даже в час сей доспети, прости ми грехи, яже сотворих в сей день делом, словом и помышлением, и очисти, Господи, смиренную мою душу от всякия скверны плоти и духа. И даждь ми, Господи, в нощи сей сон прейти в мире, да востав от смиреннаго ми ложа, благоугожду пресвятому имени Твоему, во вся дни живота моего, и поперу борющия мя враги плотския и безплотныя. И избави мя, Господи, от помышлений суетных, оскверняющих мя, и похотей лукавых. Яко Твое есть царство, и сила и слава, Отца и Сына и Святаго Духа, ныне и присно и во веки веков. Аминь.

\mysubsubsection{Молитва 2-я, святого Антиоха, ко Господу нашему Иисусу Христу}

Вседержителю, Слово Отчее, Сам совершен сый, Иисусе Христе, многаго ради милосердия Твоего никогдаже отлучайся мене, раба Твоего, но всегда во мне почивай. Иисусе, добрый Пастырю Твоих овец, не предаждь мене крамоле змиине, и желанию сатанину не остави мене, яко семя тли во мне есть. Ты убо, Господи Боже покланяемый, Царю Святый, Иисусе Христе, спяща мя сохрани немерцающим светом, Духом Твоим Святым, Имже освятил еси Твоя ученики. Даждь, Господи, и мне, недостойному рабу Твоему, спасение Твое на ложи моем: просвети ум мой светом разума святаго Евангелия Твоего, душу любовию Креста Твоего, сердце чистотою словесе Твоего, тело мое Твоею страстию безстрастною, мысль мою Твоим смирением сохрани, и воздвигни мя во время подобно на Твое славословие. Яко препрославлен еси со Безначальным Твоим Отцем и с Пресвятым Духом во веки. Аминь.

\mysubsubsection{Молитва 3-я, ко Пресвятому Духу}

Господи, Царю Небесный, Утешителю, Душе истины, умилосердися и помилуй мя грешнаго раба Твоего, и отпусти ми недостойному, и прости вся, елика Ти согреших днесь яко человек, паче же и не яко человек, но и горее скота, вольныя моя грехи и невольныя, ведомыя и неведомыя: яже от юности и науки злы, и яже суть от нагльства и уныния. Аще именем Твоим кляхся, или похулих е в помышлении моем; или кого укорих; или оклеветах кого гневом моим, или опечалих, или о чем прогневахся; или солгах, или безгодно спах, или нищ прииде ко мне, и презрех его; или брата моего опечалих, или свадих, или кого осудих; или развеличахся, или разгордехся, или разгневахся; или стоящу ми на молитве, ум мой о лукавствии мира сего подвижеся, или развращение помыслих; или объядохся, или опихся, или без ума смеяхся; или лукавое помыслих, или доброту чуждую видев, и тою уязвлен бых сердцем; или неподобная глаголах, или греху брата моего посмеяхся, моя же суть безчисленная согрешения; или о молитве не радих, или ино что содеях лукавое, не помню, та бо вся и больша сих содеях. Помилуй мя, Творче мой Владыко, унылаго и недостойнаго раба Твоего, и остави ми, и отпусти, и прости мя, яко Благ и Человеколюбец, да с миром лягу, усну и почию, блудный, грешный и окаянный аз, и поклонюся, и воспою, и прославлю пречестное имя Твое, со Отцем, и Единородным Его Сыном, ныне и присно и во веки. Аминь.

\mysubsubsection{Молитва 4-я, святого Макария Великого}

Что Ти принесу, или что Ти воздам, великодаровитый Безсмертный Царю, щедре и человеколюбче Господи, яко ленящася мене на Твое угождение, и ничтоже благо сотворша, привел еси на конец мимошедшаго дне сего, обращение и спасение души моей строя? Милостив ми буди грешному и обнаженному всякаго дела блага, возстави падшую мою душу, осквернившуюся в безмерных согрешениих, и отыми от мене весь помысл лукавый видимаго сего жития. Прости моя согрешения, едине Безгрешне, яже Ти согреших в сей день, ведением и неведением, словом, и делом, и помышлением, и всеми моими чувствы. Ты Сам, покрывая, сохрани мя от всякаго сопротивнаго обстояния Божественною Твоею властию, и неизреченным человеколюбием, и силою. Очисти, Боже, очисти множество грехов моих. Благоволи, Господи, избавити мя от сети лукаваго, и спаси страстную мою душу, и осени мя светом лица Твоего, егда приидеши во славе, и неосужденна ныне сном уснути сотвори, и без мечтания, и несмущен помысл раба Твоего соблюди, и всю сатанину детель отжени от мене, и просвети ми разумныя очи сердечныя, да не усну в смерть. И посли ми Ангела мирна, хранителя и наставника души и телу моему, да избавит мя от враг моих; да востав со одра моего, принесу Ти благодарственныя мольбы. Ей, Господи, услыши мя грешнаго и убогаго раба Твоего, изволением и совестию; даруй ми воставшу словесем Твоим поучитися, и уныние бесовское далече от мене отгнано быти сотвори Твоими Ангелы; да благословлю имя Твое святое, и прославлю, и славлю Пречистую Богородицу Марию, Юже дал еси нам грешным заступление, и приими Сию молящуюся за ны; вем бо, яко подражает Твое человеколюбие, и молящися не престает. Тоя заступлением, и Честнаго Креста знамением, и всех святых Твоих ради, убогую душу мою соблюди, Иисусе Христе Боже наш, яко Свят еси, и препрославлен во веки. Аминь.

\mysubsubsection{Молитва 5-я}

Господи Боже наш, еже согреших во дни сем словом, делом и помышлением, яко Благ и Человеколюбец прости ми. Мирен сон и безмятежен даруй ми. Ангела Твоего хранителя посли, покрывающа и соблюдающа мя от всякаго зла, яко Ты еси хранитель душам и телесем нашим, и Тебе славу возсылаем, Отцу и Сыну и Святому Духу, ныне и присно и во веки веков. Аминь.

\mysubsubsection{Молитва 6-я}

Господи Боже наш, в Негоже веровахом, и Егоже имя паче всякаго имене призываем, даждь нам, ко сну отходящим, ослабу души и телу, и соблюди нас от всякаго мечтания, и темныя сласти кроме; устави стремление страстей, угаси разжжения востания телеснаго. Даждь нам целомудренно пожити делы и словесы; да добродетельное жительство восприемлюще, обетованных не отпадем благих Твоих, яко благословен еси во веки. Аминь.

\mysubsubsection{Молитва 7-я, святого Иоанна Златоуста (24 молитвы, по числу часов дня и ночи)}

Господи, не лиши мене небесных Твоих благ.

Господи, избави мя вечных мук.

Господи, умом ли или помышлением, словом или делом согреших, прости мя.

Господи, избави мя всякаго неведения и забвения, и малодушия, и окамененнаго нечувствия.

Господи, избави мя от всякаго искушения.

Господи, просвети мое сердце, еже помрачи лукавое похотение.

Господи, аз яко человек согреших, Ты же яко Бог щедр, помилуй мя, видя немощь души моея.

Господи, посли благодать Твою в помощь мне, да прославлю имя Твое святое.

Господи Иисусе Христе, напиши мя раба Твоего в книзе животней и даруй ми конец благий.

Господи, Боже мой, аще и ничтоже благо сотворих пред Тобою, но даждь ми по благодати Твоей положити начало благое.

Господи, окропи в сердце моем росу благодати Твоея.

Господи небесе и земли, помяни мя грешнаго раба Твоего, студнаго и нечистаго, во Царствии Твоем. Аминь.

Господи, в покаянии приими мя.

Господи, не остави мене.

Господи, не введи мене в напасть.

Господи, даждь ми мысль благу.

Господи, даждь ми слезы и память смертную, и умиление.

Господи, даждь ми помысл исповедания грехов моих.

Господи, даждь ми смирение, целомудрие и послушание.

Господи, даждь ми терпение, великодушие и кротость.

Господи, всели в мя корень благих, страх Твой в сердце мое.

Господи, сподоби мя любити Тя от всея души моея и помышления и творити во всем волю Твою.

Господи, покрый мя от человек некоторых, и бесов, и страстей, и от всякия иныя неподобныя вещи.

Господи, веси, яко твориши, якоже Ты волиши, да будет воля Твоя и во мне грешнем, яко благословен еси во веки. Аминь.

\mysubsubsection{Молитва 8-я, ко Господу нашему Иисусу Христу}

Господи Иисусе Христе, Сыне Божий, ради честнейшия Матере Твоея, и безплотных Твоих Ангел, Пророка же и Предтечи и Крестителя Твоего, богоглаголивых же апостол, светлых и добропобедных мученик, преподобных и богоносных отец, и всех святых молитвами, избави мя настоящаго обстояния бесовскаго. Ей, Господи мой и Творче, не хотяй смерти грешнаго, но якоже обратитися и живу быти ему, даждь и мне обращение окаянному и недостойному; изми мя от уст пагубнаго змия, зияющаго пожрети мя и свести во ад жива. Ей, Господи мой, утешение мое, Иже мене ради окаяннаго в тленную плоть оболкийся, исторгни мя от окаянства, и утешение подаждь души моей окаянней. Всади в сердце мое творити Твоя повеления, и оставити лукавая деяния, и получити блаженства Твоя: на Тя бо, Господи, уповах, спаси мя.

\mysubsubsection{Молитва 9-я, ко Пресвятой Богородице, Петра Студийского}

К Тебе Пречистей Божией Матери аз окаянный припадая молюся: веси, Царице, яко безпрестани согрешаю и прогневляю Сына Твоего и Бога моего, и многажды аще каюся, лож пред Богом обретаюся, и каюся трепеща: неужели Господь поразит мя, и по часе паки таяжде творю; ведущи сия, Владычице моя Госпоже Богородице, молю, да помилуеши, да укрепиши, и благая творити да подаси ми. Веси бо, Владычице моя Богородице, яко отнюд имам в ненависти злая моя дела, и всею мыслию люблю закон Бога моего; но не вем, Госпоже Пречистая, откуду яже ненавижду, та и люблю, а благая преступаю. Не попущай, Пречистая, воли моей совершатися, не угодна бо есть, но да будет воля Сына Твоего и Бога моего: да мя спасет, и вразумит, и подаст благодать Святаго Духа, да бых аз отселе престал сквернодейства, и прочее пожил бых в повелении Сына Твоего, Емуже подобает всякая слава, честь и держава, со Безначальным Его Отцем, и Пресвятым и Благим и Животворящим Его Духом, ныне и присно, и во веки веков. Aминь.

\mysubsubsection{Молитва 10-я, ко Пресвятой Богородице}

Благаго Царя благая Мати, Пречистая и Благословенная Богородице Марие, милость Сына Твоего и Бога нашего излей на страстную мою душу и Твоими молитвами настави мя на деяния благая, да прочее время живота моего без порока прейду и Тобою рай да обрящу, Богородице Дево, едина Чистая и Благословенная.

\mysubsubsection{Молитва 11-я, ко святому Ангелу хранителю}

Ангеле Христов, хранителю мой святый и покровителю души и тела моего, вся ми прости, елика согреших во днешний день, и от всякаго лукавствия противнаго ми врага избави мя, да ни в коемже гресе прогневаю Бога моего; но моли за мя грешнаго и недостойнаго раба, яко да достойна мя покажеши благости и милости Всесвятыя Троицы и Матере Господа моего Иисуса Христа и всех святых. Аминь.

\mysubsubsection{Кондак Богородице}

Взбранной Воеводе победительная, яко избавльшеся от злых, благодарственная восписуем Ти раби Твои, Богородице, но яко имущая державу непобедимую, от всяких нас бед свободи, да зовем Ти; радуйся, Невесто Неневестная.

Преславная Приснодево, Мати Христа Бога, принеси нашу молитву Сыну Твоему и Богу нашему, да спасет Тобою души наша.

Все упование мое на Тя возлагаю, Мати Божия, сохрани мя под кровом Твоим.

Богородице Дево, не презри мене, грешнаго, требующа Твоея помощи и Твоего заступления, на Тя бо упова душа моя, и помилуй мя.

\mysubsubsection{Молитва святого Иоанникия}

Упование мое Отец, прибежище мое Сын, покров мой Дух Святый: Троице Святая, слава Тебе.

\Chestneyshuyu

Слава Отцу и Сыну и Святому Духу, и ныне и присно и во веки веков. Аминь.

Господи, помилуй. \myemph{ (Трижды)}

Господи Иисусе Христе, Сыне Божий, молитв ради Пречистыя Твоея Матере, преподобных и богоносных отец наших и всех святых, помилуй нас. Аминь.

\mysubsubsection{Молитва святого Иоанна Дамаскина}

Владыко Человеколюбче, неужели мне одр сей гроб будет, или еще окаянную мою душу просветиши днем? Се ми гроб предлежит, се ми смерть предстоит. Суда Твоего, Господи, боюся и муки безконечныя, злое же творя не престаю: Тебе Господа Бога моего всегда прогневляю, и Пречистую Твою Матерь, и вся Небесныя силы, и святаго Ангела хранителя моего. Вем убо, Господи, яко недостоин есмь человеколюбия Твоего, но достоин есмь всякаго осуждения и муки. Но, Господи, или хощу, или не хощу, спаси мя. Аще бо праведника спасеши, ничтоже велие; и аще чистаго помилуеши, ничтоже дивно: достойни бо суть милости Твоея. Но на мне грешнем удиви милость Твою: о сем яви человеколюбие Твое, да не одолеет моя злоба Твоей неизглаголанней благости и милосердию: и якоже хощеши, устрой о мне вещь.

Просвети очи мои, Христе Боже, да не когда усну в смерть, да не когда речет враг мой: укрепихся на него.

\slavan

 Заступник души моея буди, Боже, яко посреде хожду сетей многих; избави мя от них и спаси мя, Блаже, яко Человеколюбец.

\inynen

 Преславную Божию Матерь, и святых Ангел Святейшую, немолчно воспоим сердцем и усты, Богородицу сию исповедающе, яко воистинну рождшую нам Бога воплощенна, и молящуюся непрестанно о душах наших.

\mysubsubsection{Знаменуй себя крестом и говори молитву Честному Кресту:}

Да воскреснет Бог, и расточатся врази Его, и да бежат от лица Его ненавидящии Его. Яко исчезает дым, да исчезнут; яко тает воск от лица огня, тако да погибнут беси от лица любящих Бога и знаменующихся крестным знамением, и в веселии глаголющих: радуйся, Пречестный и Животворящий Кресте Господень, прогоняяй бесы силою на тебе пропятаго Господа нашего Иисуса Христа, во ад сшедшаго и поправшего силу диаволю, и даровавшаго нам тебе Крест Свой Честный на прогнание всякаго супостата. О, Пречестный и Животворящий Кресте Господень! Помогай ми со Святою Госпожею Девою Богородицею и со всеми святыми во веки. Аминь.

\myemph{ Или кратко:}

Огради мя, Господи, силою Честнаго и Животворящаго Твоего Креста, и сохрани мя от всякаго зла.

\mysubsubsection{Молитва}

Ослаби, остави, прости, Боже, прегрешения наша, вольная и невольная, яже в слове и в деле, яже в ведении и не в ведении, яже во дни и в нощи, яже во уме и в помышлении: вся нам прости, яко Благ и Человеколюбец.

\mysubsubsection{Молитва}

Ненавидящих и обидящих нас прости, Господи Человеколюбче. Благотворящим благосотвори. Братиям и сродником нашим даруй яже ко спасению прошения и жизнь вечную. В немощех сущия посети и исцеление даруй. Иже на мори управи. Путешествующим спутешествуй. Православным христианом споб\'{о}рствуй. Служащим и милующим нас грехов оставление даруй. Заповедавших нам недостойным молитися о них помилуй по велицей Твоей милости. Помяни, Господи, прежде усопших отец и братий наших и упокой их, идеже присещает свет лица Твоего. Помяни, Господи, братий наших плененных и избави я от всякаго обстояния. Помяни, Господи, плодоносящих и доброделающих во святых Твоих церквах, и даждь им яже ко спасению прошения и жизнь вечную. Помяни, Господи, и нас, смиренных и грешных и недостойных раб Твоих, и просвети наш ум светом разума Твоего, и настави нас на стезю заповедей Твоих, молитвами Пречистыя Владычицы нашея Богородицы и Приснодевы Марии и всех Твоих святых: яко благословен еси во веки веков. Аминь.

\mysubsubsection{Исповедание грехов повседневное}

Исповедаю Тебе Господу Богу моему и Творцу, во Святей Троице Единому, славимому и покланяемому, Отцу и Сыну и Святому Духу, вся моя грехи, яже содеях во вся дни живота моего, и на всякий час, и в настоящее время, и в прешедшия дни и нощи, делом, словом, помышлением, объядением, пиянством, тайноядением, празднословием, унынием, леностию, прекословием, непослушанием, оклеветанием, осуждением, небрежением, самолюбием, многостяжанием, хищением, неправдоглаголанием, скверноприбытчеством, мшелоимством, ревнованием, завистию, гневом, памятозлобием, ненавистию, лихоимством и всеми моими чувствы: зрением, слухом, обонянием, вкусом, осязанием и прочими моими грехи, душевными вкупе и телесными, имиже Тебе Бога моего и Творца прогневах, и ближняго моего онеправдовах: о сих жалея, винна себе Тебе Богу моему представляю, и имею волю каятися: точию, Господи Боже мой, помози ми, со слезами смиренно молю Тя: прешедшая же согрешения моя милосердием Твоим прости ми, и разреши от всех сих, яже изглаголах пред Тобою, яко Благ и Человеколюбец.

\mysubsubsection{\myemph{ Когда отходишь ко сну, произноси:}}

В руце Твои, Господи Иисусе Христе, Боже мой, предаю дух мой: Ты же мя благослови, Ты мя помилуй и живот вечный даруй ми. Аминь.

\end{mymulticols}

\mychapterending


\mychapter{Канон покаянный ко Господу нашему Иисусу Христу}\begin{mymulticols}
%http://www.molitvoslov.com/text3.htm 

\myfigure{748}

\mysubsubsection{Глас 6-й, Песнь 1}

\irmos{Яко по суху пешешествовав Израиль, по бездне стопами, гонителя фараона видя потопляема, Богу победную песнь поим, вопияше.}

\pripev{Помилуй мя, Боже, помилуй мя.}

Ныне приступих аз грешный и обремененный к Тебе, Владыце и Богу моему; не смею же взирати на небо, токмо молюся, глаголя: даждь ми, Господи, ум, да плачуся дел моих горько.

\pripev{Помилуй мя, Боже, помилуй мя.}

О, горе мне грешному! Паче всех человек окаянен есмь, покаяния несть во мне; даждь ми, Господи, слезы, да плачуся дел моих горько.

\slava

Безумне, окаянне человече, в лености время губиши; помысли житие твое, и обратися ко Господу Богу, и плачися о делех твоих горько.

\inyne

Мати Божия Пречистая, воззри на мя грешного, и от сети диаволи избави мя, и на путь покаяния настави мя, да плачуся дел моих горько.

\mysubsubsection{Песнь 3}

\irmos{Несть свят, якоже Ты, Господи Боже мой, вознесый рог верных Твоих, Блаже, и утвердивый нас на камени исповедания Твоего.}

\pripev{Помилуй мя, Боже, помилуй мя.}

Внегда поставлени будут престоли на судищи страшнем, тогда всех человек дела обличатся; горе тамо будет грешным, в муку отсылаемым; и то ведущи, душе моя, покайся от злых дел твоих.

\pripev{Помилуй мя, Боже, помилуй мя.}

Праведницы возрадуются, а грешнии восплачутся, тогда никтоже возможет помощи нам, но дела наша осудят нас, темже прежде конца покайся от злых дел твоих.

\slava

Увы мне великогрешному, иже делы и мысльми осквернився, ни капли слез имею от жестосердия; ныне возникни от земли, душе моя, и покайся от злых дел твоих.

\inyne

Се, взывает, Госпоже, Сын Твой, и поучает нас на доброе, аз же грешный добра всегда бегаю; но Ты, Милостивая, помилуй мя, да покаюся от злых моих дел.

\mysubsubsection{Седален, глас 6-й}

Помышляю день страшный и плачуся деяний моих лукавых: како отвещаю Безсмертному Царю, или коим дерзновением воззрю на Судию, блудный аз? Благоутробный Отче, Сыне Единородный и Душе Святый, помилуй мя.

Слава Отцу и Сыну и Святому Духу. И ныне и присно и во веки веков. Аминь.

\Bogorodichen{Связан многими ныне пленицами грехов и содержим лютыми страстьми и бедами, к Тебе прибегаю, моему спасению, и вопию: помози ми, Дево, Мати Божия.}

\mysubsubsection{Песнь 4}

\irmos{Христос моя сила, Бог и Господь, честная Церковь боголепно поет, взывающи от смысла чиста, о Господе празднующи.}

\pripev{Помилуй мя, Боже, помилуй мя.}

Широк путь зде и угодный сласти творити, но горько будет в последний день, егда душа от тела разлучатися будет: блюдися от сих, человече, Царствия ради Божия.

\pripev{Помилуй мя, Боже, помилуй мя.}

Почто убогаго обидиши, мзду наемничу удержуеши, брата твоего не любиши, блуд и гордость гониши? Остави убо сия, душе моя, и покайся Царствия ради Божия.

\slava

О, безумный человече, доколе углебаеши, яко пчела, собирающи богатство твое? Вскоре бо погибнет, яко прах и пепел: но более взыщи Царствия Божия.

\inyne

Госпоже Богородице, помилуй мя грешного, и в добродетели укрепи, и соблюди мя, да наглая смерть не похитит мя неготоваго, и доведи мя, Дево, Царствия Божия.

\mysubsubsection{Песнь 5}

\irmos{Божиим светом Твоим, Блаже, утренюющих Ти души любовию озари, молюся, Тя ведети, Слове Божий, истиннаго Бога, от мрака греховнаго взывающа.}

\pripev{Помилуй мя, Боже, помилуй мя.}

Воспомяни, окаянный человече, како лжам, клеветам, разбою, немощем, лютым зверем, грехов ради порабощен еси; душе моя грешная, того ли восхотела еси?

\pripev{Помилуй мя, Боже, помилуй мя.}

Трепещут ми уди, всеми бо сотворих вину: очима взираяй, ушима слышай, языком злая глаголяй, всего себе геенне предаяй; душе моя грешная, сего ли восхотела еси?

\slava

Блудника и разбойника кающася приял еси, Спасе, аз же един леностию греховною отягчихся и злым делом поработихся, душе моя грешная, сего ли восхотела еси?

\inyne

Дивная и скорая помощнице всем человеком, Мати Божия, помози мне недостойному, душа бо моя грешная того восхоте.

\mysubsubsection{Песнь 6}

\irmos{Житейское море, воздвизаемое зря напастей бурею, к тихому пристанищу Твоему притек, вопию Ти: возведи от тли живот мой, Многомилостиве.}

\pripev{Помилуй мя, Боже, помилуй мя.}

Житие на земли блудно пожих и душу во тьму предах, ныне убо молю Тя, Милостивый Владыко: свободи мя от работы сея вражия, и даждь ми разум творити волю Твою.

\pripev{Помилуй мя, Боже, помилуй мя.}

Кто творит таковая, якоже аз? Якоже бо свиния лежит в калу, тако и аз греху служу. Но Ты, Господи, исторгни мя от гнуса сего и даждь ми сердце творити заповеди Твоя.

\slava

Воспряни, окаянный человече, к Богу, воспомянув своя согрешения, припадая ко Творцу, слезя и стеня; Той же, яко милосерд, даст ти ум знати волю Свою.

\inyne

Богородице Дево, от видимаго и невидимаго зла сохрани мя, Пречистая, и приими молитвы моя, и донеси я Сыну Твоему, да даст ми ум творити волю Его.

\mysubsubsection{Кондак}

Душе моя, почто грехами богатееши, почто волю диаволю твориши, в чесом надежду полагаеши? Престани от сих и обратися к Богу с плачем, зовущи: милосерде Господи, помилуй мя грешнаго.

\mysubsubsection{Икос}

Помысли, душе моя, горький час смерти и страшный суд Творца твоего и Бога: Ангели бо грознии поймут тя, душе, и в вечный огнь введут: убо прежде смерти покайся, вопиющи: Господи, помилуй мя грешнаго.

\mysubsubsection{Песнь 7}

\irmos{Росодательну убо пещь содела Ангел преподобным отроком, халдеи же опаляющее веление Божие, мучителя увеща вопити: благословен еси, Боже отец наших.}

\pripev{Помилуй мя, Боже, помилуй мя.}

Не надейся, душе моя, на тленное богатство и на неправедное собрание, вся бо сия не веси кому оставиши, но возопий: помилуй мя, Христе Боже, недостойнаго.

\pripev{Помилуй мя, Боже, помилуй мя.}

Не уповай, душе моя, на телесное здравие и на скоромимоходящую красоту, видиши бо, яко сильнии и младии умирают; но возопий: помилуй мя, Христе Боже, недостойнаго.

\slava

Воспомяни, душе моя, вечное житие, Царство Небесное, уготованное святым, и тьму кромешную и гнев Божий злым, и возопий: помилуй мя, Христе Боже, недостойнаго.

\inyne

Припади, душе моя, к Божией Матери и помолися Той, есть бо скорая помощница кающимся, умолит Сына Христа Бога, и помилует мя недостойнаго.

\mysubsubsection{Песнь 8}

\irmos{Из пламене преподобным росу источил еси и праведнаго жертву водою попалил еси: вся бо твориши, Христе, токмо еже хотети. Тя превозносим во вся веки.}

\pripev{Помилуй мя, Боже, помилуй мя.}

Како не имам плакатися, егда помышляю смерть, видех бо во гробе лежаща брата моего, безславна и безобразна? Что убо чаю, и на что надеюся? Токмо даждь ми, Господи, прежде конца покаяние. \myemph{ (Дважды)}

\slava

Верую, яко приидеши судити живых и мертвых, и вси во своем чину станут, старии и младии, владыки и князи, девы и священницы; где обрящуся аз? Сего ради вопию: даждь ми, Господи, прежде конца покаяние.

\inyne

Пречистая Богородице, приими недостойную молитву мою и сохрани мя от наглыя смерти, и даруй ми прежде конца покаяние.

\mysubsubsection{Песнь 9}

\irmos{Бога человеком невозможно видети, на Негоже не смеют чини Ангельстии взирати; Тобою же, Всечистая, явися человеком Слово Воплощенно, Егоже величающе, с небесными вои Тя ублажаем.}

\pripev{Помилуй мя, Боже, помилуй мя.}

Ныне к вам прибегаю, Ангели, Архангели и вся небесныя силы, у Престола Божия стоящии, молитеся ко Творцу своему, да избавит душу мою от муки вечныя.

\pripev{Помилуй мя, Боже, помилуй мя.}

Ныне плачуся к вам, святии патриарси, царие и пророцы, апостоли и святителие и вси избраннии Христовы: помозите ми на суде, да спасет душу мою от силы вражия.

\slava

Ныне к вам воздежу руце, святии мученицы, пустынницы, девственницы, праведницы и вси святии, молящиися ко Господу за весь мир, да помилует мя в час смерти моея.

\inyne

Мати Божия, помози ми, на Тя сильне надеющемуся, умоли Сына Своего, да поставит мя недостойнаго одесную Себе, егда сядет судяй живых и мертвых, аминь.

\mysubsubsection{Молитва}

Господи Иисусе Христе, Сыне Божий, помилуй мя грешнаго.

Владыко Христе Боже, Иже страстьми Своими страсти моя исцеливый и язвами Своими язвы моя уврачевавый, даруй мне, много Тебе прегрешившему, слезы умиления; сраствори моему телу от обоняния Животворящаго Тела Твоего, и наслади душу мою Твоею Честною Кровию от горести, еюже мя сопротивник напои; возвыси мой ум к Тебе, долу поникший, и возведи от пропасти погибели: яко не имам покаяния, не имам умиления, не имам слезы утешительныя, возводящия чада ко своему наследию. Омрачихся умом в житейских страстех, не могу воззрети к Тебе в болезни, не могу согретися слезами, яже к Тебе любве. Но, Владыко Господи Иисусе Христе, сокровище благих, даруй мне покаяние всецелое и сердце люботрудное во взыскание Твое, даруй мне благодать Твою и обнови во мне зраки Твоего образа. Оставих Тя, не остави мене; изыди на взыскание мое, возведи к пажити Твоей и сопричти мя овцам избраннаго Твоего стада, воспитай мя с ними от злака Божественных Твоих Таинств, молитвами Пречистыя Твоея Матере и всех святых Твоих. Аминь.

\end{mymulticols}

\mychapterending


\mychapter{Канон молебный ко Пресвятой Богородице}\begin{mymulticols}
%http://www.molitvoslov.com/text4.htm 

\myfigure{3}

\mysubsubsection{Поемый во всякой скорби душевной и обстоянии.}

\mysubsubsection{Tворение Феостирикта монаха.}

\mysubsubsection{Тропaрь Богородице, глас 4-й}

К Богородице прилежно ныне притецем, грешнии и смиреннии, и припадем, в покаянии зовуще из глубины души: Владычице, помози, на ны милосердовавши, потщися, погибaем от множества прегрешений, не отврати Твоя рабы тщи, Тя бо и едину надежду имамы. \myemph{ (Дважды)}

Слава Отцу и Сыну и Святому Духу. И ныне и присно и во веки веков. Аминь.

Не умолчим никогда, Богородице, силы Твоя глаголати, недостойнии: aще бо Ты не бы предстояла молящи, кто бы нас избaвил от толиких бед, кто же бы сохранил до ныне свободны? Не отступим, Владычице, от Тебе: Твоя бо рабы спасaеши присно от всяких лютых.

\mysubsubsection{Псалом 50}

\PsalmFifty

\mysubsubsection{Канон ко Пресвятой Богородице, глас 8-й}

\mysubsubsection{Песнь 1}

\irmos{Воду прошед яко сушу, и египетскаго зла избежaв, изрaильтянин вопияше: избaвителю и Богу нашему поим.}

\pripev{Пресвятая Богородице, спаси нас.}

Многими содержимь напaстьми, к Тебе прибегаю, спасения иский: о, Мaти Слова и Дево, от тяжких и лютых мя спаси.

\pripev{Пресвятая Богородице, спаси нас.}

Страстей мя смущaют прилози, многаго уныния исполнити мою душу; умири, Отроковице, тишиною Сына и Бога Твоего, Всенепорочная.

\slava

Спaса рождшую Тя и Бога, молю, Дево, избaвитися ми лютых: к Тебе бо ныне прибегaя, простирaю и душу и помышление.

\inyne

Недугующа телом и душею, посещения Божественнаго и промышления от Тебе сподоби, едина Богомaти, яко благая, Благaго же Родительница.

\mysubsubsection{Песнь 3}

\irmos{Небеснаго круга Верхотворче, Господи, и Церкве Зиждителю, Ты мене утверди в любви Твоей, желaний крaю, верных утверждение, едине Человеколюбче.}

\pripev{Пресвятая Богородице, спаси нас.}

Предстaтельство и покров жизни моея полагaю Тя, Богородительнице Дево: Ты мя окорми ко пристaнищу Твоему, благих виновна; верных утверждение, едина Всепетая.

\pripev{Пресвятая Богородице, спаси нас.}

Молю, Дево, душевное смущение и печали моея бурю разорити: Ты бо, Богоневестная, начальника тишины Христа родилa еси, едина Пречистая.

\slava

Благодетеля рождши добрых виновнаго, благодеяния богатство всем источи, вся бо можеши, яко сильнаго в крепости Христа рождши, Богоблаженная.

\inyne

Лютыми недуги и болезненными страстьми истязaему, Дево, Ты ми помози: исцелений бо неоскудное Тя знаю сокровище, Пренепорочная, неиждивaемое.

Спаси от бед рабы Твоя, Богородице, яко вси по Бозе к Тебе прибегaем, яко нерушимей стене и предстaтельству.

Призри благосердием, всепетая Богородице, на мое лютое телесе озлобление, и исцели души моея болезнь.

\mysubsubsection{Тропарь, глас 2-й}

Моление теплое и стенa необоримая, милости источниче, мирови прибежище, прилежно вопием Ти: Богородице Владычице, предвари, и от бед избaви нас, едина вскоре предстaтельствующая.

\mysubsubsection{Песнь 4}

\irmos{Услышах, Господи, смотрения Твоего тaинство, разумех дела Твоя и прослaвих Твое Божество.}

\pripev{Пресвятая Богородице, спаси нас.}

Страстей моих смущение, кормчию рождшая Господа, и бурю утиши моих прегрешений, Богоневестная.

\pripev{Пресвятая Богородице, спаси нас.}

Милосердия Твоего бездну призывaющу подaждь ми, яже Благосердаго рождшая и Спaса всех поющих Тя.

\pripev{Пресвятая Богородице, спаси нас.}

Наслаждaющеся, Пречистая, Твоих даровaний, благодaрственное воспевaем пение, ведуще Тя Богомaтерь.

\slava

На одре болезни моея и немощи низлежaщу ми, яко Благолюбива, помози, Богородице, едина Приснодево.

\inyne

Надежду и утверждение и спасения стену недвижиму имуще Тя, Всепетая, неудобства всякаго избавляемся.

\mysubsubsection{Песнь 5}

\irmos{Просвети нас повелении Твоими, Господи, и мышцею Твоею высокою Твой мир подaждь нам, Человеколюбче.}

\pripev{Пресвятая Богородице, спаси нас.}

Исполни, Чистая, веселия сердце мое, Твою нетленную дающи радость, веселия рождшая виновнаго.

\pripev{Пресвятая Богородице, спаси нас.}

Избaви нас от бед, Богородице чистая, вечное рождши избавление, и мир, всяк ум преимущий.

\slava

Разреши мглу прегрешений моих, Богоневесто, просвещением Твоея светлости, Свет рождшая Божественный и превечный.

\inyne

Исцели, Чистая, души моея неможение, посещения Твоего сподобльшая, и здрaвие молитвами Твоими подaждь ми.

\mysubsubsection{Песнь 6}

\irmos{Молитву пролию ко Господу, и Тому возвещу печали моя, яко зол душа моя исполнися, и живот мой аду приближися, и молюся яко Иона: от тли, Боже, возведи мя.}

\pripev{Пресвятая Богородице, спаси нас.}

Смерти и тли яко спасл есть, Сам Ся издaв смерти, тлением и смертию мое естество, ято бывшее, Дево, моли Господа и Сына Твоего, врагов злодействия мя избaвити.

\pripev{Пресвятая Богородице, спаси нас.}

Предстaтельницу Тя живота вем и хранительницу тверду, Дево, и напaстей решaщу молвы, и налоги бесов отгоняющу; и молюся всегда, от тли страстей моих избaвити мя.

\slava

Яко стену прибежища стяжaхом, и душ всесовершенное спасение, и прострaнство в скорбех, Отроковице, и просвещением Твоим присно рaдуемся: о, Владычице, и ныне нас от страстей и бед спаси.

\inyne

На одре ныне немощствуяй лежу, и несть исцеления плоти моей: но, Бога и Спaса миру и Избaвителя недугов рождшая, Тебе молюся, Благой: от тли недуг возстaви мя.

\mysubsubsection{Кондaк, глас 6-й}

Предстaтельство христиан непостыдное, ходaтайство ко Творцу непреложное, не презри грешных молений глaсы, но предвари, яко Благaя, на помощь нас, верно зовущих Ти; ускори на молитву, и потщися на умоление, предстaтельствующи присно, Богородице, чтущих Тя.

\mysubsubsection{Другой кондaк, глас тот же}

 Не имамы иныя помощи, не имамы иныя надежды, разве Тебе, Пречистая Дево. Ты нам помози, на Тебе надеемся, и Тобою хвaлимся, Твои бо есмы рабы, да не постыдимся.

\mysubsubsection{Стихира, глас тот же}

Не ввери мя человеческому предстaтельству, Пресвятая Владычице, но приими моление раба Твоего: скорбь бо обдержит мя, терпети не могу демонскаго стреляния, покрова не имам, ниже где прибегну, окаянный, всегда побеждaемь, и утешения не имам, разве Тебе, Владычице мира, уповaние и предстaтельство верных, не презри моление мое, полезно сотвори.

\mysubsubsection{Песнь 7}

\irmos{От Иудеи дошедше отроцы, в Вавилоне иногдa, верою Троическою плaмень пещный попрaша, поюще: отцев Боже, благословен еси.}

\pripev{Пресвятая Богородице, спаси нас.}

Наше спасение якоже восхотел еси, Спaсе, устроити, во утробу Девыя вселился еси, Юже миру предстaтельницу показал еси: отец наших Боже, благословен еси.

\pripev{Пресвятая Богородице, спаси нас.}

Волителя милости, Егоже родилa еси, Мaти чистая, умоли избaвитися от прегрешений и душевных скверн верою зовущим: отец наших Боже, благословен еси.

\slava

Сокровище спасения и Источник нетления, Тя рождшую, и столп утверждения, и дверь покаяния, зовущим показал еси: отец наших Боже, благословен еси.

\inyne

Телесныя слабости и душевныя недуги, Богородительнице, любовию приступaющих к крову Твоему, Дево, исцелити сподоби, Спaса Христа нам рождшая.

\mysubsubsection{Песнь 8}

\irmos{Царя Небеснаго, Егоже поют вои aнгельстии, хвалите и превозносите во вся веки.}

\pripev{Пресвятая Богородице, спаси нас.}

Помощи яже от Тебе требующия не презри, Дево, поющия и превозносящия Тя во веки.

\pripev{Пресвятая Богородице, спаси нас.}

Неможение души моея исцеляеши и телесныя болезни, Дево, да Тя прослaвлю, Чистая, во веки.

\slava

Исцелений богатство изливaеши верно поющим Тя, Дево, и превозносящим неизреченное Твое рождество.

\inyne

Напaстей Ты прилоги отгоняеши и страстей находы, Дево: темже Тя поем во вся веки.

\mysubsubsection{Песнь 9}

\irmos{Воистинну Богородицу Тя исповедуем, спасеннии Тобою, Дево чистая, с безплотными лики Тя величaюще.}

\pripev{Пресвятая Богородице, спаси нас.}

Тока слез моих не отвратися, Яже от всякаго лица всяку слезу отъемшаго, Дево, Христа рождшая.

\pripev{Пресвятая Богородице, спаси нас.}

Радости мое сердце исполни, Дево, Яже радости приемшая исполнение, греховную печаль потребляющи.

\pripev{Пресвятая Богородице, спаси нас.}

Пристaнище и предстaтельство к Тебе прибегaющих буди, Дево, и стена нерушимая, прибежище же и покров и веселие.

\slava

Света Твоего зарями просвети, Дево, мрак неведения отгоняющи, благоверно Богородицу Тя исповедающих.

\inyne

На месте озлобления немощи смирившагося, Дево, исцели, из нездрaвия во здрaвие претворяющи.

\mysubsubsection{Стихиры, глас 2-й}

Высшую небес и чистшую светлостей солнечных, избaвльшую нас от клятвы, Владычицу мира песньми почтим.

От многих моих грехов немощствует тело, немощствует и душа моя; к Тебе прибегaю, Благодaтней, надеждо ненадежных, Ты ми помози.

Владычице и Мaти Избaвителя, приими моление недостойных раб Твоих, да ходaтайствуеши к Рождшемуся от Тебе; о, Владычице мира, буди Ходaтаица!

Поем прилежно Тебе песнь ныне, всепетой Богородице, рaдостно: со Предтечею и всеми святыми моли, Богородице, еже ущедрити ны.

Вся aнгелов воинства, Предтече Господень, апостолов двоенадесятице, святии вси с Богородицею, сотворите молитву, во еже спастися нам.

\mysubsubsection{Молитвы ко Пресвятой Богородице}

Пресвятая Богородице, спаси мя.

Царице моя преблагaя, надеждо моя Богородице, приятелище сирых и странных предстaтельнице, скорбящих рaдосте, обидимых покровительнице! Зриши мою беду, зриши мою скорбь, помози ми яко немощну, окорми мя яко стрaнна. Обиду мою веси, разреши ту, яко волиши: яко не имам иныя помощи разве Тебе, ни иныя предстaтельницы, ни благия утешительницы, токмо Тебе, о Богомaти, яко да сохраниши мя и покрыеши во веки веков. Аминь.

К кому возопию, Владычице? К кому прибегну в горести моей, aще не к Тебе, Царице Небесная? Кто плач мой и воздыхaние мое приимет, aще не Ты, Пренепорочная, надеждо христиан и прибежище нам, грешным? Кто пaче Тебе в напaстех защитит? Услыши убо стенaние мое, и приклони ухо Твое ко мне, Владычице Мaти Бога моего, и не презри мене, требующаго Твоея помощи, и не отрини мене, грешнаго. Вразуми и научи мя, Царице Небесная; не отступи от мене, раба Твоего, Владычице, за роптaние мое, но буди мне Мaти и заступница. Вручaю себе милостивому покрову Твоему: приведи мя, грешнаго, к тихой и безмятежной жизни, да плaчуся о гресех моих. К кому бо прибегну повинный аз, aще не к Тебе, уповaнию и прибежищу грешных, надеждою на неизреченную милость Твою и щедроты Твоя окриляемь? О, Владычице Царице Небесная! Ты мне уповaние и прибежище, покров и заступление и помощь. Царице моя преблагaя и скорая заступнице! Покрый Твоим ходaтайством моя прегрешения, защити мене от враг видимых и невидимых; умягчи сердца злых человек, возстающих на мя. О, Мaти Господа моего Творцa! Ты еси корень девства и неувядaемый цвет чистоты. О, Богородительнице! Ты подaждь ми помощь немощствующему плотскими страстьми и болезнующему сердцем, едино бо Твое и с Тобою Твоего Сына и Бога нашего имам заступление; и Твоим пречудным заступлением да избaвлюся от всякия беды и напaсти, о пренепорочная и преслaвная Божия Мaти Марие. Темже со уповaнием глаголю и вопию: радуйся, благодaтная, радуйся, обрaдованная; радуйся, преблагословенная, Господь с Тобою.

\end{mymulticols}

\mychapterending


\mychapter{Канон Ангелу Хранителю}\begin{mymulticols}
%http://www.molitvoslov.com/text5.htm

\myfigure{8}

\mysubsubsection{Тропарь, глас 6-й}

Ангеле Божий, хранителю мой святый, живот мой соблюди во страсе Христа Бога, ум мой утверди во истиннем пути, и к любви горней уязви душу мою, да тобою направляемь, получу от Христа Бога велию милость.

Слава Отцу и Сыну и Святому Духу. И ныне и присно и во веки веков. Аминь.

\mysubsubsection{Богородичен}

Святая Владычице, Христа Бога нашего Мати, яко всех Творца недоуменно рождшая, моли благость Его всегда, со хранителем моим ангелом, спасти душу мою, страстьми одержимую, и оставление грехов даровати ми.

\mysubsubsection{Канон, глас 8-й}

\mysubsubsection{Песнь 1}

\irmos{Поим Господеви, проведшему люди Своя сквозе Чермное море, яко един славно прославися.}

\pripev[Иисусу:]{Господи Иисусе Христе Боже мой, помилуй мя.}

Песнь воспети и восхвалити, Спасе, Твоего раба достойно сподоби, безплотному Aнгелу, наставнику и хранителю моему.

\pripev{Святый Aнгеле Божий, хранителю мой, моли Бога о мне.}

Един аз в неразумии и в лености ныне лежу, наставниче мой и хранителю, не остави мене, погибающа.

\slava

Ум мой твоею молитвою направи, творити ми Божия повеления, да получу от Бога отдание грехов, и ненавидети ми злых настави мя, молюся ти.

\inyne

Молися, Девице, о мне, рабе Твоем, ко Благодателю, со хранителем моим Aнгелом, и настави мя творити заповеди Сына Твоего и Творца моего.

\mysubsubsection{Песнь 3}

\irmos{Ты еси утверждение притекающих к Тебе, Господи, Ты еси свет омраченных, и поет Тя дух мой.}

\pripev{Святый Aнгеле Божий, хранителю мой, моли Бога о мне.}

Все помышление мое и душу мою к тебе возложих, хранителю мой; ты от всякия мя напасти вражия избави.

\pripev{Святый Aнгеле Божий, хранителю мой, моли Бога о мне.}

Враг попирает мя, и озлобляет, и поучает всегда творити своя хотения; но ты, наставниче мой, не остави мене погибающа.

\slava

Пети песнь со благодарением и усердием Творцу и Богу даждь ми, и тебе, благому Aнгелу хранителю моему: избавителю мой, изми мя от враг озлобляющих мя.

\inyne

Исцели, Пречистая, моя многонедужныя струпы, яже в души, прожени враги, иже присно борются со мною.

\mysubsubsection{Седален, глас 2-й}

От любве душевныя вопию ти, хранителю моея души, всесвятый мой Aнгеле: покрый мя и соблюди от лукаваго ловления всегда, и к жизни настави небесней, вразумляя и просвещая и укрепляя мя.

Слава Отцу и Сыну и Святому Духу. И ныне и присно и во веки веков. Аминь.

\mysubsubsection{Богородичен:}

Богородице безневестная Пречистая, Яже без семени рождши всех Владыку, Того со Aнгелом хранителем моим моли, избавити ми ся всякаго недоумения, и дати умиление и свет души моей и согрешением очищение, Яже едина вскоре заступающи.

\mysubsubsection{Песнь 4}

\irmos{Услышах, Господи, смотрения Твоего таинство, разумех дела Твоя, и прославих Твое Божество.}

\pripev{Святый Aнгеле Божий, хранителю мой, моли Бога о мне.}

Моли Человеколюбца Бога ты, хранителю мой, и не остави мене, но присно в мире житие мое соблюди и подаждь ми спасение необоримое.

\pripev{Святый Aнгеле Божий, хранителю мой, моли Бога о мне.}

Яко заступника и хранителя животу моему прием тя от Бога, Aнгеле, молю тя, святый, от всяких мя бед свободи.

\slava

Мою скверность твоею святынею очисти, хранителю мой, и от части шуия да отлучен буду молитвами твоими и причастник славы явлюся.

\inyne

Недоумение предлежит ми от обышедших мя зол, Пречистая, но избави мя от них скоро: к Тебе бо единей прибегох.

\mysubsubsection{Песнь 5}

\irmos{Утренююще вопием Ти: Господи, спаси ны; Ты бо еси Бог наш, разве Тебе иного не вемы.}

\pripev{Святый Aнгеле Божий, хранителю мой, моли Бога о мне.}

Яко имея дерзновение к Богу, хранителю мой святый, Сего умоли от оскорбляющих мя зол избавити.

\pripev{Святый Aнгеле Божий, хранителю мой, моли Бога о мне.}

Свете светлый, светло просвети душу мою, наставниче мой и хранителю, от Бога данный ми Aнгеле.

\slava

Спяща мя зле тяготою греховною, яко бдяща сохрани, Aнгеле Божий, и возстави мя на славословие молением твоим.

\inyne

Марие, Госпоже Богородице безневестная, надеждо верных, вражия возношения низложи, поющия же Тя возвесели.

\newpage\mysubsubsection{Песнь 6}

\irmos{Ризу ми подаждь светлу, одеяйся светом яко ризою, многомилостиве Христе Боже наш.}

\pripev{Святый Aнгеле Божий, хранителю мой, моли Бога о мне.}

Всяких мя напастей свободи, и от печалей спаси, молюся ти, святый Aнгеле, данный ми от Бога, хранителю мой добрый.

\pripev{Святый Aнгеле Божий, хранителю мой, моли Бога о мне.}

Освети ум мой, блаже, и просвети мя, молюся ти, святый Aнгеле, и мыслити ми полезная всегда настави мя.

\slava

Устави сердце мое от настоящаго мятежа, и бдети укрепи мя во благих, хранителю мой, и настави мя чудно к тишине животней.

\inyne

Слово Божие в Тя вселися, Богородице, и человеком Тя показа небесную лествицу; Тобою бо к нам Вышний сошел есть.

\mysubsubsection{Кондак, глас 4-й}

Явися мне милосерд, святый Aнгеле Господень, хранителю мой, и не отлучайся от мене, сквернаго, но просвети мя светом неприкосновенным и сотвори мя достойна Царствия Небеснаго.

\mysubsubsection{Икос}

Уничиженную душу мою многими соблазны, ты, святый предстателю, неизреченныя славы небесныя сподоби, и певец с лики безплотных сил Божиих, помилуй мя и сохрани, и помыслы добрыми душу мою просвети, да твоею славою, Aнгеле мой, обогащуся, и низложи зломыслящия мне враги, и сотвори мя достойна Царствия Небеснаго.

\mysubsubsection{Песнь 7}

\irmos{От Иудеи дошедше отроцы, в Вавилоне иногда, верою Троическою пламень пещный попраша, поюще: отцев Боже, благословен еси.}

\pripev{Святый Aнгеле Божий, хранителю мой, моли Бога о мне.}

Милостив буди ми, и умоли Бога, Господень Aнгеле, имею бо тя заступника во всем животе моем, наставника же и хранителя, от Бога дарованнаго ми во веки.

\pripev{Святый Aнгеле Божий, хранителю мой, моли Бога о мне.}

Не остави в путь шествующия души моея окаянныя убити разбойником, святый Aнгеле, яже ти от Бога предана бысть непорочне; но настави ю на путь покаяния.

\slava

Всю посрамлену душу мою привожду от лукавых ми помысл и дел: но предвари, наставниче мой, и исцеление ми подаждь благих помысл, уклоняти ми ся всегда на правыя стези.

\inyne

Премудрости исполни всех и крепости Божественныя, Ипостасная Премудросте Вышняго, Богородицы ради, верою вопиющих: отец наших Боже, благословен еси.

\mysubsubsection{Песнь 8}

\irmos{Царя Небеснаго, Егоже поют вои ангельстии, хвалите и превозносите во вся веки.}

\pripev{Святый Aнгеле Божий, хранителю мой, моли Бога о мне.}

От Бога посланный, утверди живот мой, раба твоего, преблагий Aнгеле, и не остави мене во веки.

\pripev{Святый Aнгеле Божий, хранителю мой, моли Бога о мне.}

Ангела тя суща блага, души моея наставника и хранителя, преблаженне, воспеваю во веки.

\slava

Буди ми покров и забрало в день испытания всех человек, воньже огнем искушаются дела благая же и злая.

\inyne

Буди ми помощница и тишина, Богородице Приснодево, рабу Твоему, и не остави мене лишена быти Твоего владычества.

\mysubsubsection{Песнь 9}

\irmos{Воистинну Богородицу Тя исповедуем, спасеннии Тобою, Дево чистая, с безплотными лики Тя величающе.}

\pripev[Иисусу:]{Господи Иисусе Христе Боже мой, помилуй мя.}

Помилуй мя, едине Спасе мой, яко милостив еси и милосерд, и праведных ликов сотвори мя причастника.

\pripev{Святый Aнгеле Божий, хранителю мой, моли Бога о мне.}

Мыслити ми присно и творити, Господень Aнгеле, благая и полезная даруй, яко сильна яви в немощи и непорочна.

\slava

Яко имея дерзновение к Царю Небесному, Того моли, с прочими безплотными, помиловати мя, окаяннаго.

\inyne

Много дерзновение имущи, Дево, к Воплощшемуся из Тебе, преложи мя от уз и разрешение ми подаждь и спасение, молитвами Твоими.

\mysubsubsection{Молитва к Aнгелу Xранителю}

Святый Aнгеле Божий, хранителю мой, моли Бога о мне.

Ангеле Христов святый, к тебе припадая молюся, хранителю мой святый, приданный мне на соблюдение души и телу моему грешному от святаго крещения, аз же своею леностию и своим злым обычаем прогневах твою пречистую светлость и отгнах тя от себе всеми студными делы: лжами, клеветами, завистию, осуждением, презорством, непокорством, братоненавидением, и злопомнением, сребролюбием, прелюбодеянием, яростию, скупостию, объядением без сытости и опивством, многоглаголанием, злыми помыслы и лукавыми, гордым обычаем и блудным возбешением, имый самохотение на всякое плотское вожделение. О, злое мое произволение, егоже и скоти безсловеснии не творят! Да како возможеши воззрети на мя, или приступити ко мне, аки псу смердящему? Которыма очима, ангеле Христов, воззриши на мя, оплетшася зле во гнусных делех? Да како уже возмогу отпущения просити горьким и злым моим и лукавым деянием, в няже впадаю по вся дни и нощи и на всяк час? Но молюся ти припадая, хранителю мой святый, умилосердися на мя грешнаго и недостойнаго раба твоего \myemph{ (имя)}, буди ми помощник и заступник на злаго моего сопротивника, святыми твоими молитвами, и Царствия Божия причастника мя сотвори со всеми святыми, всегда, и ныне и присно и во веки веков. Аминь.

\end{mymulticols}

\mychapterending


\mychapter{Последование ко Святому Причащению}\begin{mymulticols}
%http://www.molitvoslov.com/text207.htm

\myfigure{132}

\MolitvamiSviatyhOtecNashih

Царю Небесный, Утешителю, Душе истины, Иже везде сый и вся исполняяй, Сокровище благих и жизни Подателю, прииди и вселися в ны, и очисти ны от всякия скверны, и спаси, Блаже, души наша.

Святый Боже, Святый Крепкий, Святый Безсмертный, помилуй нас. \myemph{ (Tрижды)}

Слава Отцу и Сыну и Святому Духу, и ныне и присно и во веки веков. Аминь.

Пресвятая Троице, помилуй нас; Господи, очисти грехи наша; Владыко, прости беззакония наша; Святый, посети и исцели немощи наша, имене Твоего ради.

Господи, помилуй. \myemph{ (Трижды)}

Слава Отцу и Сыну и Святому Духу, и ныне и присно и во веки веков. Аминь.

Отче наш, Иже еси на небесех! Да святится имя Твое, да приидет Царствие Твое, да будет воля Твоя, яко на небеси и на земли. Хлеб наш насущный даждь нам днесь; и остави нам долги наша, якоже и мы оставляем должником нашим; и не введи нас во искушение, но избави нас от лукаваго.

Господи, помилуй.\myemph{  (12раз)}

Приидите, поклонимся Цареви нашему Богу. \myemph{ (Поклон)}

Приидите, поклонимся и припадем Христу, Цареви нашему Богу. \myemph{ (Поклон)}

Приидите, поклонимся и припадем Самому Христу, Цареви и Богу нашему.\myemph{ (Поклон)}

\mysubsubsection{Псалом 22}

Господь пасет мя, и ничтоже мя лишит. На месте злачне, тамо всели мя, на воде покойне воспита мя. Душу мою обрати, настави мя на стези правды, имене ради Своего. Аще бо и пойду посреде сени смертныя, не убоюся зла, яко Ты со мною еси, жезл Твой и палица Твоя, та мя утешиста. Уготовал еси предо мною трапезу сопротив стужающим мне, умастил еси елеом главу мою, и чаша Твоя упоявающи мя, яко державна. И милость Твоя поженет мя вся дни живота моего, и еже вселити ми ся в дом Господень, в долготу дний.

\mysubsubsection{Псалом 23}

Господня земля, и исполнение ея, вселенная, и вси живущии на ней. Той на морях основал ю есть, и на реках уготовал ю есть. Кто взыдет на гору Господню? Или кто станет на месте святем Его? Неповинен рукама и чист сердцем, иже не прият всуе душу свою, и не клятся лестию искреннему своему. Сей приимет благословение от Господа, и милостыню от Бога, Спаса своего. Сей род ищущих Господа, ищущих лице Бога Иаковля. Возмите врата князи ваша, и возмитеся врата вечная; и внидет Царь Славы. Кто есть сей Царь Славы? Господь крепок и силен, Господь силен в брани. Возмите врата князи ваша, и возмитеся врата вечная, и внидет Царь Славы. Кто есть сей Царь Славы? Господь сил, Той есть Царь Славы.

\mysubsubsection{Псалом 115}

Веровах, темже возглаголах, аз же смирихся зело. Аз же рех во изступлении моем: всяк человек ложь. Что воздам Господеви о всех, яже воздаде ми? Чашу спасения прииму, и имя Господне призову, молитвы моя Господеви воздам пред всеми людьми Его. Честна пред Господем смерть преподобных Его. О, Господи, аз раб Твой, аз раб Твой и сын рабыни Твоея; растерзал еси узы моя. Тебе пожру жертву хвалы, и во имя Господне призову. Молитвы моя Господеви воздам пред всеми людьми Его, во дворех дому Господня, посреди тебе, Иерусалиме.

Слава Отцу и Сыну и Святому Духу, и ныне и присно и во веки веков. Аминь.

Аллилуия. \myemph{ (Трижды с тремя поклонами)}

\mysubsubsection{Тропари, глас 8-й}

Беззакония моя презри, Господи, от Девы рождейся, и сердце мое очисти, храм то творя пречистому Твоему Телу и Крови, ниже отрини мене от Твоего лица, без числа имеяй велию милость.

\slava

Во причастие святынь Твоих како дерзну [вниду], недостойный? Аше бо дерзну к Тебе приступити с достойными, хитон мя обличает, яко несть вечерний, и осуждение исходатайствую многогрешной души моей. Очисти, Господи, скверну души моея, и спаси мя, яко Человеколюбец.

\inyne

Многая множества моих, Богородице, прегрешений, к Тебе прибегох, Чистая, спасения требуя: посети немощствующую мою душу, и моли Сына Твоего и Бога нашего, дати ми оставление, яже содеях лютых, Едина благословенная.

\mysubsubsection{[Во Святую же Четыредесятницу:}

Егда славнии ученицы на умовении вечери просвещахуся, тогда Иуда злочестивый сребролюбием недуговав омрачашеся, и беззаконным судиям Тебе праведнаго Судию предает. Виждь, имений рачителю, сих ради удавление употребивша: бежи несытыя души, Учителю таковая дерзнувшия. Иже о всех благий Господи, слава Тебе. ]

\mysubsubsection{Псалом 50}

\PsalmFifty

\mysubsubsection{Канон, глас 2-й}

\mysubsubsection{Песнь 1}

\irmos{Грядите людие, поим песнь Христу Богу, раздельшему море, и наставльшему люди, яже изведе из работы египетския, яко прославися.}

\pripev{Сердце чисто созижди во мне, Боже, и дух прав обнови во утробе моей.}

Хлеб живота вечнующаго да будет ми Тело Твое Святое, благоутробне Господи, и Честная Кровь, и недуг многообразных исцеление.

\pripev{Не отвержи мене от лица Твоего, и Духа Твоего Святаго не отыми от мене.}

Осквернен делы безместными окаянный, Твоего Пречистаго Тела и Божественныя Крове недостоин есмь, Христе, причащения, егоже мя сподоби.

\pripev{Пресвятая Богородице, спаси нас.}

\Bogorodichen{Земле благая, благословенная Богоневесто, клас прозябшая неоранный и спасительный миру, сподоби мя сей ядуща спастися.}

\mysubsubsection{Песнь 3}

\irmos{На камени мя веры утвердив, разширил еси уста моя на враги моя. Возвесели бо ся дух мой, внегда пети: несть свят, якоже Бог наш, и несть праведен паче Тебе, Господи.}

\pripev{Сердце чисто созижди во мне, Боже, и дух прав обнови во утробе моей.}

Слезныя ми подаждь, Христе, капли, скверну сердца моего очищающия: яко да благою совестию очищен, верою прихожду и страхом, Владыко, ко причащению Божественных Даров Твоих.

\pripev{Не отвержи мене от лица Твоего, и Духа Твоего Святаго не отыми от мене.}

Во оставление да будет ми прегрешений Пречистое Тело Твое, и Божественная Кровь, Духа же Святаго общение, и в жизнь вечную, Человеколюбче, и страстей и скорбей отчуждение.

\pripev{Пресвятая Богородице, спаси нас.}

\Bogorodichen{Хлеба животнаго Tрапеза Пресвятая, свыше милости ради сшедшаго, и мирови новый живот дающаго, и мене ныне сподоби недостойнаго, со страхом вкусити сего, и живу быти.}

\mysubsubsection{Песнь 4}

\irmos{Пришел еси от Девы, не ходатай, ни Ангел, но Сам, Господи, воплощься, и спася еси всего мя человека. Тем зову Ти: слава силе Твоей, Господи.}

\pripev{Сердце чисто созижди во мне, Боже, и дух прав обнови во утробе моей.}

Восхотел еси, нас ради воплощься, Многомилостиве, заклан быти яко овча, грех ради человеческих: темже молю Тя, и моя очисти согрешения.

\pripev{Не отвержи мене от лица Твоего, и Духа Твоего Святаго не отыми от мене.}

Исцели души моея язвы, Господи, и всего освяти: и сподоби, Владыко, яко да причащуся тайныя Твоея Божественныя вечери, окаянный.

\pripev{Пресвятая Богородице, спаси нас.}

\Bogorodichen{Умилостиви и мне сущаго от утробы Твоея, Владычице, и соблюди мя нескверна раба Твоего и непорочна, яко да прием умнаго бисера, освящуся.}

\mysubsubsection{Песнь 5}

\irmos{Света Подателю и веков Творче, Господи, во свете Твоих повелений настави нас; разве бо Тебе иного бога не знаем.}

\pripev{Сердце чисто созижди во мне, Боже, и дух прав обнови во утробе моей.}

Якоже предрекл еси, Христе, да будет убо худому рабу Твоему, и во мне пребуди, якоже обещался еси: се бо Тело Твое ям Божественное, и пию Кровь Твою.

\pripev{Не отвержи мене от лица Твоего, и Духа Твоего Святаго не отыми от мене.}

Слове Божий и Боже, угль Тела Твоего да будет мне помраченному в просвещение, и очищение оскверненной души моей Кровь Твоя.

\pripev{Пресвятая Богородице, спаси нас.}

\Bogorodichen{Марие, Мати Божия, благоухания честное селение, Твоими молитвами сосуд мя избранный соделай, яко да освящений причащуся Сына Твоего.}

\mysubsubsection{Песнь 6}

\irmos{В бездне греховней валяяся, неизследную милосердия Твоего призываю бездну: от тли, Боже, мя возведи.}

\pripev{Сердце чисто созижди во мне, Боже, и дух прав обнови во утробе моей.}

Ум, душу и сердце освяти, Спасе, и тело мое, и сподоби неосужденно, Владыко, к страшным Тайнам приступити.

\pripev{Не отвержи мене от лица Твоего, и Духа Твоего Святаго не отыми от мене.}

Да бых устранился от страстей, и Твоея благодати имел бы приложение, живота же утверждение, причащением Святых, Христе, Таин Твоих.

\pripev{Пресвятая Богородице, спаси нас.}

\Bogorodichen{Божие, Боже, Слово Святое, всего мя освяти, ныне приходящаго к Божественным Твоим Тайнам, Святыя Матере Твоея мольбами.}

\mysubsubsection{Кондак, глас 2-й}

Хлеб, Христе, взяти не презри мя, Тело Твое, и Божественную Твою ныне Кровь, пречистых, Владыко, и страшных Твоих Таин причаститися окаяннаго, да не будет ми в суд, да будет же ми в живот вечный и безсмертный.

\mysubsubsection{Песнь 7}

\irmos{Телу златому премудрыя дети не послужиша, и в пламень сами поидоша, и боги их обругаша, среди пламене возопиша, и ороси я Ангел: услышася уже уст ваших молитва.}

\pripev{Сердце чисто созижди во мне, Боже, и дух прав обнови во утробе моей.}

Источник благих, причащение, Христе, безсмертных Твоих ныне Таинств да будет ми свет, и живот, и безстрастие, и к преспеянию же и умножению добродетели Божественнейшия ходатайственно, едине Блаже, яко да славлю Тя.

\pripev{Не отвержи мене от лица Твоего, и Духа Твоего Святаго не отыми от мене.}

Да избавлюся от страстей, и врагов, и нужды, и всякия скорби, трепетом и любовию со благоговением, Человеколюбче, приступаяй ныне к Твоим безсмертным и Божественным Тайнам, и пети Тебе сподоби: благословен еси, Господи, Боже отец наших.

\pripev{Пресвятая Богородице, спаси нас.}

\Bogorodichen{Спаса Христа рождшая паче ума, Богоблагодатная, молю Тя ныне, раб Твой, Чистую нечистый: хотящаго мя ныне к пречистым Тайнам приступити, очисти всего от скверны плоти и духа.}

\mysubsubsection{Песнь 8}

\irmos{В пещь огненную ко отроком еврейским снизшедшаго, и пламень в росу преложшаго Бога, пойте дела яко Господа, и превозносите во вся веки.}

\pripev{Сердце чисто созижди во мне, Боже, и дух прав обнови во утробе моей.}

Небесных, и страшных, и святых Твоих, Христе, ныне Таин, и Божественныя Твоея и тайныя вечери общника быти и мене сподоби отчаяннаго, Боже, Спасе мой.

\pripev{Не отвержи мене от лица Твоего, и Духа Твоего Святаго не отыми от мене.}

Под Твое прибег благоутробие, Блаже, со страхом зову Ти: во мне пребуди, Спасе, и аз, якоже рекл еси, в Тебе; се бо дерзая на милость Твою, ям Тело Твое, и пию Кровь Твою.

\pripev{Пресвятая Троице, Боже наш, слава Тебе.}

\myemph{ Троичен:} Трепещу, приемля огнь, да не опалюся яко воск и яко трава; oле страшнаго таинства! oле благоутробия Божия! Како Божественнаго Тела и Крове брение причащаюся, и нетленен сотворяюся?

\mysubsubsection{Песнь 9}

\irmos{Безначальна Родителя Сын, Бог и Господь, воплощся от Девы нам явися, омраченная просветити, собрати расточенная: тем всепетую Богородицу величаем.}

\pripev{Сердце чисто созижди во мне, Боже, и дух прав обнови во утробе моей.}

Христос есть, вкусите и видите: Господь нас ради, по нам бо древле бывый, единою Себе принес, яко приношение Отцу Своему, присно закалается, освящаяй причащающияся.

\pripev{Не отвержи мене от лица Твоего, и Духа Твоего Святаго не отыми от мене.}

Душею и телом да освящуся, Владыко, да просвещуся, да спасуся, да буду дом Твой причащением священных Таин, живущаго Тя имея в себе со Отцем и Духом, Благодетелю Многомилостиве.

\pripev{Воздаждь ми радость спасения Твоего и Духом Владычним утверди мя.}

Якоже огнь да будет ми, и яко свет, Тело Твое и Кровь, Спасе мой, пречестная, опаляя греховное вещество, сжигая же страстей терние, и всего мя просвещая, покланятися Божеству Твоему.

\pripev{Пресвятая Богородице, спаси нас.}

\Bogorodichen{Бог воплотися от чистых кровей Твоих; темже всякий род поет Тя, Владычице, умная же множества славят, яко Тобою яве узреша всеми Владычествующаго, осуществовавшагося человечеством.}

\mysubsubsection{Далее}

Достойно есть яко воистину блажити Тя Богородицу, Присноблаженную и Пренепорочную и Матерь Бога нашего. Честнейшую Херувим и славнейшую без сравнения Серафим, без истления Бога Слова рождшую, сущую Богородицу Тя величаем.

Святый Боже, Святый Крепкий, Святый Безсмертный, помилуй нас. \myemph{ (Tрижды)}

Слава Отцу и Сыну и Святому Духу, и ныне и присно и во веки веков. Аминь.

Пресвятая Троице, помилуй нас; Господи, очисти грехи наша; Владыко, прости беззакония наша; Святый, посети и исцели немощи наша, имене Твоего ради.

Господи, помилуй. \myemph{ (Трижды)}

Слава Отцу и Сыну и Святому Духу, и ныне и присно и во веки веков. Аминь.

Отче наш, Иже еси на небесех! Да святится имя Твое, да приидет Царствие Твое, да будет воля Твоя, яко на небеси и на земли. Хлеб наш насущный даждь нам днесь; и остави нам долги наша, якоже и мы оставляем должником нашим; и не введи нас во искушение, но избави нас от лукаваго.

\myemph{ Если неделя, тропарь воскресный по гласу. Если же нет, настоящие тропари, глас 6-й:}

Помилуй нас, Господи, помилуй нас; всякаго бо ответа недоумеюще, сию Ти молитву, яко Владыце, грешнии приносим: помилуй нас.

\slava

Господи, помилуй нас, на Тя бо уповахом; не прогневайся на ны зело, ниже помяни беззаконий наших, но призри и ныне яко благоутробен, и избави ны от враг наших. Ты бо еси Бог наш, и мы людие Твои, вси дела руку Твоею, и имя Твое призываем.

\inyne

Милосердия двери отверзи нам, благословенная Богородице, надеющиися на Тя да не погибнем, но да избавимся Тобою от бед: Ты бо еси спасение рода христианскаго.

Господи, помилуй. \myemph{ (40 раз) И поклоны, сколько хочешь.}

\myemph{ И стихи:}

Хотя ясти, человече, Тело Владычне,

Страхом приступи, да не опалишися: огнь бо есть.

Божественную же пия Кровь ко общению,

Первее примирися тя опечалившим.

Таже дерзая, таинственное брашно яждь.

Прежде причастия страшныя жертвы,

Животворящаго Тела Владычня,

Сим помолися образом со трепетом:

\mysubsubsection{Молитва 1-я, Василия Великого}

Владыко Господи Иисусе Христе, Боже наш, Источниче жизни и безсмертия, всея твари видимыя и невидимыя Содетелю, безначальнаго Отца соприсносущный Сыне и собезначальный, премногия ради благости в последния дни в плоть оболкийся, и распныйся, и погребыйся за ны неблагодарныя и злонравныя, и Твоею Кровию обновивый растлевшее грехом естество наше, Сам, Безсмертный Царю, приими и мое грешнаго покаяние, и приклони ухо Твое мне, и услыши глаголы моя. Согреших бо, Господи, согреших на небо и пред Тобою, и несмь достоин воззрети на высоту славы Твоея: прогневах бо Твою благость, Твоя заповеди преступив, и не послушав Твоих повелений. Но Ты, Господи, незлобив сый, долготерпелив же и многомилостив, не предал еси мя погибнути со беззаконьми моими, моего всячески ожидая обращения. Ты бо рекл еси, Человеколюбче, пророком Твоим: яко хотением не хощу смерти грешника, но еже обратится и живу быти ему. Не хощеши бо, Владыко, создания Твоею руку погубити, ниже благоволиши о погибели человечестей, но хощеши всем спастися, и в разум истины приити. Темже и аз, аще и недостоин есмь небесе и земли, и сея привременныя жизни, всего себе повинув греху, и сластем поработив, и Твой осквернив образ; но творение и создание Твое быв, не отчаяваю своего спасения окаянный, на Твое же безмерное благоутробие дерзая, прихожду. Приими убо и мене, Человеколюбче Господи, якоже блудницу, яко разбойника, яко мытаря и яко блуднаго, и возми мое тяжкое бремя грехов, грех вземляй мира, и немощи человеческия исцеляяй, труждающияся и обремененныя к Себе призываяй и упокоеваяй, не пришедый призвати праведныя, но грешныя на покаяние. И очисти мя от всякия скверны плоти и духа, и научи мя совершати святыню во страсе Твоем: яко да чистым сведением совести моея, святынь Твоих часть приемля, соединюся святому Телу Твоему и Крови, и имею Тебе во мне живуща и пребывающа, со Отцем, и Святым Твоим Духом. Ей, Господи Иисусе Христе, Боже мой, и да не в суд ми будет причастие пречистых и животворящих Таин Твоих, ниже да немощен буду душею же и телом, от еже недостойне тем причащатися, но даждь ми, даже до конечнаго моего издыхания, неосужденно восприимати часть святынь Твоих, в Духа Святаго общение, в напутие живота вечнаго, и во благоприятен ответ на Страшнем судищи Твоем: яко да и аз со всеми избранными Твоими общник буду нетленных Твоих благ, яже уготовал еси любящим Тя, Господи, в нихже препрославлен еси во веки. Аминь.

\mysubsubsection{Молитва 2-я , святого Иоанна Златоустого}

Господи Боже мой, вем, яко несмь достоин, ниже доволен, да под кров внидеши храма души моея, занеже весь пуст и пался есть, и не имаши во мне места достойна еже главу подклонити: но якоже с высоты нас ради смирил еси Себе, смирися и ныне смирению моему; и якоже восприял еси в вертепе и в яслех безсловесных возлещи, сице восприими и в яслех безсловесныя моея души, и во оскверненное мое тело внити. И якоже не неудостоил еси внити, и свечеряти со грешники в дому Симона прокаженнаго, тако изволи внити и в дом смиренныя моея души, прокаженныя и грешныя; и якоже не отринул еси подобную мне блудницу и грешную, пришедшую и прикоснувшуюся Тебе, сице умилосердися и о мне грешнем, приходящем и прикасающем Ти ся; и якоже не возгнушался еси скверных ея уст и нечистых, целующих Тя, ниже моих возгнушайся сквернших оныя уст и нечистших, ниже мерзких моих и нечистых устен, и сквернаго и нечистейшаго моего языка. Но да будет ми угль пресвятаго Твоего Тела, и честныя Твоея Крове, во освящение и просвещение и здравие смиренней моей души и телу, во облегчение тяжестей многих моих согрешений, в соблюдение от всякаго диавольскаго действа, во отгнание и возбранение злаго моего и лукаваго обычая, во умерщвление страстей, в снабдение заповедей Твоих, в приложение Божественныя Твоея благодати, и Твоего Царствия присвоение. Не бо яко презираяй прихожду к Тебе, Христе Боже, но яко дерзая на неизреченную Твою благость, и да не на мнозе удаляяйся общения Твоего, от мысленнаго волка звероуловлен буду. Темже молюся Тебе: яко един сый Свят, Владыко, освяти мою душу и тело, ум и сердце, чревеса и утробы, и всего мя обнови, и вкорени страх Твой во удесех моих, и освящение Твое неотъемлемо от мене сотвори; и буди ми помощник и заступник, окормляя в мире живот мой, сподобляя мя и одесную Тебе предстояния со святыми Твоими, молитвами и моленьми Пречистыя Твоея Матере, невещественных Твоих служителей и пречистых сил, и всех святых, от века Тебе благоугодивших. Аминь.

\mysubsubsection{Молитва 3-я, Симеона Метафраста}

Едине чистый и нетленный Господи, за неизреченную милость человеколюбия наше все восприемый смешение, от чистых и девственных кровей паче естества рождшия Тя, Духа Божественнаго нашествием, и благоволением Отца присносущнаго, Христе Иисусе, премудросте Божия, и мире, и сило; Твоим восприятием животворящая и спасительная страдания восприемый, крест, гвоздия, копие, смерть, умертви моя душетленныя страсти телесныя. Погребением Твоим адова пленивый царствия, погреби моя благими помыслы лукавая советования, и лукавствия духи разори. Тридневным Твоим и живоносным воскресением падшаго праотца возставивый, возстави мя грехом поползшагося, образы мне покаяния предлагая. Преславным Твоим вознесением плотское обоживый восприятие, и сие десным Отца седением почтый, сподоби мя причастием святых Твоих Таин десную часть спасаемых получити. Снитием Утешителя Твоего Духа сосуды честны священныя Твоя ученики соделавый, приятелище и мене покажи Того пришествия. Хотяй паки прийти судити вселенней правдою, благоволи и мне усрести Тя на облацех, Судию и Создателя моего, со всеми святыми Твоими: да безконечно славословлю и воспеваю Тя, со безначальным Твоим Отцем, и Пресвятым и Благим и Животворящим Твоим Духом, ныне и присно, и во веки веков. Аминь.

\mysubsubsection{Молитва 4-я, святого Иоанна Дамаскина}

Владыко Господи Иисусе Христе, Боже наш, едине имеяй власть человеком оставляти грехи, яко благ и Человеколюбец презри моя вся в ведении и не в ведении прегрешения, и сподоби мя неосужденно причаститися Божественных, и преславных, и пречистых, и животворящих Твоих Таин, не в тяжесть, ни в муку, ни в приложение грехов, но во очищение, и освящение, и обручение будущаго Живота и царствия, в стену и помощь, и в возражение сопротивных, во истребление многих моих согрешений. Ты бо еси Бог милости, и щедрот, и человеколюбия, и Тебе славу возсылаем, со Отцем, и Святым Духом, ныне и присно, и во веки веков. Аминь.

\mysubsubsection{Молитва 5-я, святого Василия Великого}

Вем, Господи, яко недостойне причащаюся пречистаго Твоего Тела и честныя Твоея Крове, и повинен есмь, и суд себе ям и пию, не разсуждая Тела и Крове Тебе Христа и Бога моего, но на щедроты Твоя дерзая прихожду к Тебе рекшему: ядый Мою плоть, и пияй Мою кровь, во Мне пребывает, и Аз в нем. Умилосердися убо, Господи, и не обличи мя грешнаго, но сотвори со мною по милости Твоей; и да будут ми святая сия во исцеление, и очищение, и просвещение, и сохранение, и спасение, и во освящение души и тела; во отгнание всякаго мечтания, и лукаваго деяния, и действа диавольскаго, мысленнe во удесех моих действуемаго, в дерзновение и любовь, яже к Тебе; во исправление жития и утверждение, в возращение добродетели и совершенства; во исполнение заповедей, в Духа Святаго общение, в напутие живота вечнаго, во ответ благоприятен на Страшнем судищи Твоем: не в суд или во осуждение.

\mysubsubsection{Молитва 6-я,святого Симеона Нового Богослова}

От скверных устен, от мерзкаго сердца, от нечистаго языка, от души осквернены, приими моление, Христе мой, и не презри моих ни словес, ниже образов, ниже безстудия. Даждь ми дерзновенно глаголати, яже хощу, Христе мой, паче же и научи мя, что ми подобает творити и глаголати. Согреших паче блудницы, яже уведе, где обитаеши, миро купивши, прииде дерзостне помазати Твои нозе, Бога моего, Владыки и Христа моего. Якоже ону не отринул еси пришедшую от сердца, ниже мене возгнушайся, Слове: Твои же ми подаждь нозе, и держати и целовати, и струями слезными, яко многоценным миром, сия дерзостно помазати. Омый мя слезами моими, очисти мя ими, Слове. Остави и прегрешения моя, и прощение ми подаждь. Веси зол множество, веси и струпы моя, и язвы зриши моя, но и веру веси, и произволение зриши, и воздыхание слышиши. Не таится Тебе, Боже мой, Творче мой, Избавителю мой, ниже капля слезная, ниже капли часть некая. Несоделанное мое видесте очи Твои, в книзе же Твоей и еще несодеянная написана Тебе суть. Виждь смирение мое, виждь труд мой елик, и грехи вся остави ми, Боже всяческих: да чистым сердцем, притрепетною мыслию, и душею сокрушенною, нескверных Твоих причащуся и пресвятых Таин, имиже оживляется и обожается всяк ядый же и пияй чистым сердцем; Ты бо рекл еси, Владыко мой: всяк ядый Мою Плоть, и пияй Мою Кровь, во Мне убо сей пребывает, в немже и Аз есмь. Истинно слово всяко Владыки и Бога моего: божественных бо причащаяйся и боготворящих благодатей, не убо есмь един, но с Тобою, Христе мой, Светом трисолнечным, просвещающим мир. Да убо не един пребуду кроме Тебе Живодавца, дыхания моего, живота моего, радования моего, спасения миру. Сего ради к Тебе приступих, якоже зриши, со слезами, и душею сокрушенною избавления моих прегрешений прошу прияти ми, и Твоих живодательных и непорочных Таинств причаститися неосужденно, да пребудеши, якоже рекл еси, со мною треокаянным: да не кроме обрет мя Твоея благодати, прелестник восхитит мя льстивне, и прельстив отведет боготворящих Твоих словес. Сего ради к Тебе припадаю, и тепле вопию Ти: якоже блуднаго приял еси, и блудницу пришедшую, тако приими мя блуднаго и сквернаго, Щедре. Душею сокрушенною, ныне бо к Тебе приходя, вем, Спасе, яко иный, якоже аз, не прегреши Тебе, ниже содея деяния, яже аз содеях. Но сие паки вем, яко не величество прегрешений, ни грехов множество превосходит Бога моего многое долготерпение, и человеколюбие крайнее; но милостию сострастия тепле кающияся, и чистиши, и светлиши, и света твориши причастники, общники Божества Твоего соделоваяй независтно, и странное и Ангелом, и человеческим мыслем, беседуеши им многажды, якоже другом Твоим истинным. Сия дерзостна творят мя, сия вперяют мя, Христе мой. И дерзая Твоим богатым к нам благодеянием, радуяся вкупе и трепеща, огневи причащаюся трава сый, и странно чудо, орошаем неопально, якоже убо купина древле неопальне горящи. Ныне благодарною мыслию, благодарным же сердцем, благодарными удесы моими, души и тела моего, покланяюся и величаю, и славословлю Тя, Боже мой, яко благословенна суща, ныне же и во веки.

\mysubsubsection{Молитва 7-я, святого Иоанна Златоустого}

Боже, ослаби, остави, прости ми согрешения моя, елика Ти согреших, аще словом, аще делом, аще помышлением, волею или неволею, разумом или неразумием, вся ми прости яко благ и Человеколюбец, и молитвами Пречистыя Твоея Матере, умных Твоих служителей и святых сил, и всех святых от века Тебе благоугодивших, неосужденно благоволи прияти ми святое и пречистое Твое Тело и честную Кровь, во исцеление души же и тела, и во очищение лукавых моих помышлений. Яко Твое есть царство и сила и слава, со Отцем и Святым Духом, ныне и присно, и во веки веков. Аминь.

\mysubsubsection{Его же, 8-я}

Несмь доволен, Владыко Господи, да внидеши под кров души моея; но понеже хощеши Ты, яко Человеколюбец, жити во мне, дерзая приступаю; повелеваеши, да отверзу двери, яже Ты един создал еси, и внидеши со человеколюбием якоже еси, внидеши и просвещаеши помраченный мой помысл. Верую, яко сие сотвориши: не бо блудницу, со слезами пришедшую к Тебе, отгнал еси; ниже мытаря отвергл еси покаявшася; ниже разбойника, познавша Царство Твое, отгнал еси; ниже гонителя покаявшася оставил еси, еже бе: но от покаяния Тебе пришедшия вся, в лице Твоих другов вчинил еси, Един сый благословенный всегда, ныне и в безконечныя веки. Аминь.

\mysubsubsection{Его же, 9-я}

Господи Иисусе Христе Боже мой, ослаби, остави, очисти и прости ми грешному, и непотребному, и недостойному рабу Твоему, прегрешения, и согрешения, и грехопадения моя, елика Ти от юности моея, даже до настоящего дне и часа согреших: аще в разуме и в неразумии, аще в словесех или делех, или помышлениих и мыслех, и начинаниих, и всех моих чувствах. И молитвами безсеменно рождшия Тя Пречистыя и Приснодевы Марии, Матере Твоея, единыя непостыдныя надежды и предстательства и спасения моего, сподоби мя неосужденно причаститися пречистых, безсмертных, животворящих и страшных Твоих Таинств, во оставление грехов и в жизнь вечную: во освящение, и просвещение, крепость, исцеление, и здравие души же и тела, и в потребление и всесовершенное погубление лукавых моих помыслов, и помышлений, и предприятий, и нощных мечтаний, темных и лукавых духов; яко Твое есть царство, и сила, и слава, и честь, и поклонение, со Отцем и Святым Твоим Духом, ныне и присно, и во веки веков. Аминь.

\mysubsubsection{Молитва 10-я, святого Иоанна Дамаскина}

Пред дверьми храма Твоего предстою, и лютых помышлений не отступаю; но Ты, Христе Боже, мытаря оправдивый, и хананею помиловавый, и разбойнику рая двери отверзый, отверзи ми утробы человеколюбия Твоего и приими мя приходяща и прикасающася Тебе, яко блудницу, и кровоточивую: ова убо края ризы Твоея коснувшися, удобь исцеление прият, ова же пречистеи Твои нозе удержавши, разрешение грехов понесе. Аз же, окаянный, все Твое Тело дерзая восприяти, да не опален буду; но приими мя, якоже оныя, и просвети моя душевныя чувства, попаляя моя греховныя вины, молитвами безсеменно Рождшия Тя, и Небесных сил; яко благословен еси во веки веков. Аминь.

\mysubsubsection{Молитва святого Иоанна Златоустого}

Верую, Господи, и исповедую, яко Ты еси воистинну Христос, Сын Бога живаго, пришедый в мир грешныя спасти, от нихже первый есмь аз. Еще верую, яко сие есть самое пречистое Тело Твое, и сия самая есть честная Кровь Твоя. Молюся убо Тебе: помилуй мя, и прости ми прегрешения моя, вольная и невольная, яже словом, яже делом, яже ведением и неведением, и сподоби мя неосужденно причаститися пречистых Твоих Таинств, во оставление грехов, и в жизнь вечную. Аминь.

\mysubsubsection{Приходя же причаститься, произноси мысленно эти стихи Метафраста:}

Се приступаю к Божественному Причащению.

Содетелю, да не опалиши мя приобщением:

Огнь бо еси, недостойныя попаляяй.

Но убо очисти мя от всякия скверны.

\mysubsubsection{Затем:}

Вечери Твоея тайныя днесь, Сыне Божий, причастника мя приими; не бо врагом Твоим тайну повем, ни лобзания Ти дам, яко Иуда, но яко разбойник исповедаю Тя: помяни мя, Господи, во Царствии Твоем.

\mysubsubsection{И стихи:}

Боготворящую Кровь ужаснися, человече, зря:

Огнь бо есть, недостойныя попаляяй.

Божественное Тело и обожает мя и питает:

Обожает дух, ум же питает странно.

\mysubsubsection{Потом тропари:}

Усладил мя еси любовию, Христе, и изменил мя еси Божественным Твоим рачением; но попали огнем невещественным грехи моя, и насытитися еже в Тебе наслаждения сподоби: да ликуя возвеличаю, Блаже, два пришествия Твоя.

Во светлостех Святых Твоих како вниду недостойный? Аще бо дерзну совнити в чертог, одежда мя обличает, яко несть брачна, и связан извержен буду от Ангелов. Очисти, Господи, скверну души моея, и спаси мя, яко Человеколюбец.

\mysubsubsection{Также молитву:}

Владыко Человеколюбче, Господи Иисусе Христе Боже мой, да не в суд ми будут Святая сия, за еже недостойну ми быти: но во очищение и освящение души же и тела, и во обручение будущия жизни и царствия. Мне же, еже прилеплятися Богу, благо есть, полагати во Господе упование спасения моего.

\mysubsubsection{И еще:}

Вечери Твоея тайныя днесь, Сыне Божий, причастника мя приими; не бо врагом Твоим тайну повем, ни лобзания Ти дам, яко Иуда, но яко разбойник исповедаю Тя: помяни мя, Господи, во Царствии Твоем.

\end{mymulticols}

\myparsep[0.25]

\medskip Желающий причаститься должен достойно приготовится к этому святому таинству. Приготовление это (в церковной практике оно называется говением) продолжается несколько дней и касается как телесной, так и духовной жизни человека. Телу предписывается воздержание, т.~е. телесная чистота (воздержание от супружеских отношений) и ограничение в пищи (пост). В дни поста исключается пища животного происхождения "--- мясо, молоко, яйца и, про строгом посте, рыба. Хлеб, овощи, фрукты употребляются в умеренном количестве. Ум не должен рассеиваться по мелочам житейским и развлекаться.

В дни говения надлежит посещать богослужения в храме, если позволят обстоятельства, и более прилежно выполнять домашнее молитвенное правило: кто читает обычно не все утренние и вечерние молитвы, пусть читает все полностью, кто не читает каноны, пусть в эти дни читает хотя бы по одному канону. Накануне причащения надо быть на вечернем богослужении и прочитать дома, кроме обычных молитв на сон грядущим, канон покаянный, канон Богородице и Ангелу хранителю. Каноны читают или один за другим полностью, или соединяя таким образом: читается ирмос первой песни покаянного канона («Яко по суху петешествовав Израиль, по бездне стопами, гонителя фараона видя потопляема, Богу победную песнь поим, вопияше») и тропари, затем тропари первой песни канона Богородице («Многими содержимь напaстьми, к Тебе прибегаю, спасения иский: о, Мaти Слова и Дево, от тяжких и лютых мя спаси»), опуская ирмос «Воду прошед…», и тропари канона Ангелу хранителю, тоже без ирмоса («Поим Господеви, проведшему люди Своя сквозе Чермное море, яко един славно прославися»). Так же читают и следующие песни. Тропари перед каноном Богородице и Ангелу хранителю, а также стихиры после канона Богородице в таком случае опускаются.

Читается также канон ко причащению и, кто пожелает,"--- акафист Иисусу Сладчайшему. После полуночи уже не едят и не пьют, ибо принято приступать к Таинству Причащения натощак. Утром прочитываются утренние молитвы и все последование ко Святому Причащению, кроме канона, прочитанного накануне.

Перед причащением необходима исповедь "--- вечером ли, или утром, перед литургией.

\mychapterending

\mychapter{Акафист Иисусу Сладчайшему}\begin{mymulticols}
%http://www.molitvoslov.com/text11.htm

\myfigure{B-2757_01-IV}

\mysubsubsection{Кондак 1}

Возбранный  Воеводо и Господи, ада победителю, яко избавлься от вечныя смерти, похвальная восписую Ти, создание и раб Твой; но, яко имеяй милосердие неизреченное, от всяких мя бед свободи, зовуща: Иисусе, Сыне Божий, помилуй мя.

\mysubsubsection{Икос 1}


Ангелов
Творче и Господи cил, отверзи ми недоуменный ум и язык на похвалу пречистаго Твоего имене, якоже глухому и гугнивому древле слух и язык отверзл еси, и, глаголаше зовый таковая: Иисусе пречудный, aнгелов удивление; Иисусе пресильный, прародителей избавление. Иисусе пресладкий, патриархов величание; Иисусе преславный, царей укрепление. Иисусе прелюбимый, пророков исполнение; Иисусе предивный, мучеников крепосте. Иисусе претихий, монахов радосте; Иисусе премилостивый,
пресвитеров сладосте. Иисусе премилосердый, постников воздержание; Иисусе пресладостный, преподобных радование. Иисусе пречестный, девственных целомудрие; Иисусе предвечный, грешников спасение. Иисусе, Сыне Божий, помилуй мя.

\mysubsubsection{Кондак 2}

Видя
вдовицу зельне плачущу, Господи, якоже бо тогда умилосердився, сына ея на погребение несома воскресил еси; сице и о мне умилосердися, Человеколюбче, и грехми умерщвленную мою душу воскреси, зовущую: Аллилуиа.

\mysubsubsection{Икос 2}

Разум
неуразуменный разумети Филипп ища, Господи, покажи нам Отца, глаголаше; Ты же к нему: толикое время сый со Мною, не познал ли еси, яко Отец во Мне, и Аз во Отце есмь? Темже, Неизследованне, со страхом зову Ти: Иисусе, Боже предвечный; Иисусе, Царю пресильный. Иисусе, Владыко долготерпеливый; Иисусе, Спасе премилостивый. Иисусе, хранителю мой преблагий; Иисусе, очисти грехи моя. Иисусе, отыми беззакония моя; Иисусе, отпусти неправды моя. Иисусе, надеждо моя, не остави мене; Иисусе, помощниче мой, не отрини мене. Иисусе, Создателю мой, не забуди
мене; Иисусе, Пастырю мой, не погуби мене. Иисусе, Сыне Божий, помилуй мя.

\mysubsubsection{Кондак 3}

Силою
свыше апостолы облекий, Иисусе, во Иерусалиме седящия, облецы и мене, обнаженнаго от всякаго благотворения, теплотою Духа Святаго Твоего и даждь ми с любвью пети Тебе: Аллилуиа.

\mysubsubsection{Икос 3}

Имеяй
богатство милосердия, мытари и грешники, и неверныя призвал еси, Иисусе; не презри и мене ныне, подобнаго им, но, яко многоценное миро, приими песнь сию: Иисусе, сило непобедимая; Иисусе, милосте
безконечная. Иисусе, красото пресветлая; Иисусе, любы неизреченная. Иисусе, Сыне Бога Живаго; Иисусе, помилуй мя грешнаго. Иисусе, услыши мя в беззакониих зачатаго; Иисусе, очисти мя во гресех рожденнаго. Иисусе, научи мя непотребнаго; Иисусе, освети мя темнаго. Иисусе, очисти мя сквернаго; Иисусе, возведи мя блуднаго. Иисусе, Сыне Божий, помилуй мя.

\mysubsubsection{Кондак 4}

Бурю
внутрь имеяй помышлений сумнительных, Петр утопаше; узрев же во плоти Тя суща, Иисусе, и по водам ходяща, позна Тя Бога истиннаго и, руку спасения получив, рече: Аллилуиа.

\mysubsubsection{Икос 4}

Слыша
слепый мимоходяща Тя, Господи, путем вопияше: Иисусе, Сыне Давидов, помилуй мя! И, призвав, отверзл еси очи его. Просвети убо милостию Твоею очи мысленныя сердца и мене, вопиюща Ти и глаголюща: Иисусе, вышних Создателю; Иисусе, нижних Искупителю. Иисусе, преисподних потребителю; Иисусе, всея твари украсителю. Иисусе, души моея утешителю; Иисусе, ума моего просветителю. Иисусе, сердца моего веселие; Иисусе, тела моего здравие. Иисусе, Спасе мой, спаси мя; Иисусе, свете мой, просвети мя. Иисусе, муки всякия избави мя; Иисусе, спаси мя, недостойнаго. Иисусе, Сыне Божий, помилуй мя.

\mysubsubsection{Кондак 5}

Боготочною
Кровию якоже искупил еси нас древле от законныя клятвы, Иисусе, сице изми нас от сети, еюже змий запят ны страстьми плотскими, и блудным наваждением, и злым унынием, вопиющия Ти: Аллилуиа.

\mysubsubsection{Икос 5}

Видевше
отроцы еврейстии во образе человечестем Создавшаго рукою человека, и Владыку разумевше Его, потщашася ветвьми угодити Ему, осанна вопиюще. Мы же песнь приносим Ти, глаголюще: Иисусе, Боже истинный; Иисусе, Сыне Давидов. Иисусе, Царю преславный; Иисусе, Агнче непорочный. Иисусе, Пастырю предивный; Иисусе, хранителю во младости моей. Иисусе, кормителю во юности моей; Иисусе, похвало в старости моей. Иисусе, надежде в смерти моей; Иисусе, животе по смерти моей. Иисусе, утешение мое на суде Твоем; Иисусе, желание мое, не посрами мене тогда. Иисусе, Сыне Божий, помилуй мя.

\mysubsubsection{Кондак 6}

Проповедник
богоносных вещание и глаголы исполняя, Иисусе, на земли явлься и с человеки Невместимый пожил еси, и болезни наша подъял еси, отнюдуже ранами Твоими мы исцелевше, пети навыкохом: Аллилуиа.

\mysubsubsection{Икос 6}

Возсия
вселенней просвещение истины Твоея, и отгнася лесть бесовская: идоли бо, Спасе наш, не терпяще Твоея крепости, падоша. Мы же, спасение получивше, вопием Ти: Иисусе, истино, лесть отгонящая; Иисусе, свете, превышший всех светлостей. Иисусе, Царю, премогаяй всех крепости; Иисусе, Боже, пребываяй в милости. Иисусе, Хлебе Животный, насыти мя алчущаго; Иисусе, источниче разума, напой мя жаждущаго. Иисусе, одеждо веселия, одей мя тленнаго; Иисусе, покрове радости, покрый мя недостойнаго. Иисусе, подателю просящим, даждь мне плач за грехи моя; Иисусе, обретение ищущим, обрящи душу мою. Иисусе, отверзителю толкущим, отверзи сердце мое окаянное; Иисусе, Искупителю грешных, очисти беззакония моя. Иисусе, Сыне Божий, помилуй мя.

\mysubsubsection{Кондак 7}

Хотя
сокровенную тайну от века открыти, яко овча на заколение веден был еси, Иисусе, и яко агнец прямо стригущаго его безгласен, и яко Бог из мертвых воскресл еси, и со славою на небеса вознеслся еси, и нас совоздвигл еси, зовущих: Аллилуиа.

\mysubsubsection{Икос 7}

Дивную
показа тварь, явлейся Творец нам: без семене от Девы воплотися, из гроба, печати не рушив, воскресе, и ко апостолом, дверем затворенным, с плотию вниде. Темже чудящеся, воспоим: Иисусе, Слове необыменный; Иисусе, Слове несоглядаемый. Иисусе, сило непостижимая; Иисусе,
мудросте недомыслимая. Иисусе, Божество неописанное; Иисусе, господство неисчетное. Иисусе, царство непобедимое; Иисусе, владычество безконечное. Иисусе, крепосте высочайшая; Иисусе, власте вечная. Иисусе, Творче мой, ущедри мя; Иисусе, Спасе мой, спаси мя. Иисусе, Сыне Божий, помилуй мя.

\mysubsubsection{Кондак 8}

Странно
Бога вочеловечшася видяще, устранимся суетнаго мира и ум на Божественная возложим. Сего бо ради Бог на землю сниде, да нас на небеса возведет, вопиющих Ему: Аллилуиа.

\mysubsubsection{Икос 8}

Весь
бе в нижних, и вышних никакоже отступи Неисчетный, егда волею нас ради пострада, и смертию Своею нашу смерть умертви, и воскресением живот дарова поющим: Иисусе, сладосте сердечная; Иисусе, крепосте телесная. Иисусе, светлосте душевная; Иисусе, быстрото умная. Иисусе, радосте совестная; Иисусе, надеждо известная. Иисусе, памяте предвечная; Иисусе, похвало высокая. Иисусе, славо моя превознесенная; Иисусе, желание мое, не отрини мене. Иисусе, Пастырю мой, взыщи мене; Иисусе, Спасе мой, спаси мене. Иисусе, Сыне Божий, помилуй мя.

\mysubsubsection{Кондак 9}

Все
естество ангельское безпрестани славит пресвятое имя Твое, Иисусе, на небеси: Свят, Свят, Свят, вопиюще; мы же, грешнии на земли бренными устнами вопием: Аллилуиа.

\mysubsubsection{Икос 9}

Ветия
многовещанны, якоже рыбы безгласныя видим о Тебе, Иисусе, Спасе наш: недоумеют бо глаголати, како Бог непреложний и человек совершенный пребываеши? Мы же таинству дивящеся, вопием верно: Иисусе, Боже предвечный; Иисусе, Царю царствующих. Иисусе, Владыко владеющих; Иисусе, Судие живых и мертвых. Иисусе, надеждо ненадежных; Иисусе, утешение плачущих. Иисусе, славо нищих; Иисусе, не осуди мя по делом моим. Иисусе, очисти мя по милости Твоей; Иисусе, отжени от мене уныние. Иисусе, просвети моя мысли сердечныя; Иисусе, даждь ми память смертную. Иисусе, Сыне Божий, помилуй мя.

\mysubsubsection{Кондак 10}

Спасти
хотя мир, Восточе востоком, к темному западу "--- естеству нашему пришед, смирился еси до смерти; темже превознесеся имя Твое паче всякаго имене, и от всех колен небесных и земных слышиши: Аллилуиа.

\mysubsubsection{Икос 10}

Царю
Превечный, Утешителю, Христе истинный, очисти ны от всякия скверны, якоже очистил еси десять прокаженных, и исцели ны, якоже исцелил еси сребролюбивую душу Закхеа мытаря, да вопием Ти, во умилении зовуще: Иисусе, сокровище нетленное; Иисусе, богатство неистощимое. Иисусе, пище крепкая; Иисусе, питие неисчерпаемое. Иисусе, нищих одеяние; Иисусе, вдов заступление. Иисусе, сирых защитниче; Иисусе, труждающихся помоще. Иисусе, странных наставниче; Иисусе, плавающих кормчий. Иисусе, бурных отишие; Иисусе Боже, воздвигни мя падшаго. Иисусе, Сыне Божий, помилуй мя.

\mysubsubsection{Кондак 11}

Пение
всеумиленное приношу Ти недостойный, вопию Ти яко хананеа: Иисусе, помилуй мя; не дщерь бо, но плоть имам страстьми люте бесящуюся и яростию палимую, и исцеление даждь вопиющу Ти: Аллилуиа.

\mysubsubsection{Икос 11}

Светоподательна
светильника сущим во тьме неразумия, прежде гоняй Тя Павел, богоразумнаго гласа силу внуши и душевную быстроту уясни; сице и мене темныя зеницы душевныя просвети, зовуща: Иисусе, Царю мой прекрепкий; Иисусе, Боже мой пресильный. Иисусе, Господи мой пребезсмертный; Иисусе, Создателю мой преславный. Иисусе, Наставниче мой предобрый; Иисусе, Пастырю мой прещедрый. Иисусе, Владыко мой премилостивый; Иисусе, Спасе мой премилосердый. Иисусе, просвети моя чувствия, потемненныя страстьми; Иисусе, исцели мое тело, острупленное грехми. Иисусе, очисти мой ум от помыслов суетных; Иисусе, сохрани сердце мое от похотей лукавых. Иисусе, Сыне Божий, помилуй мя.

\mysubsubsection{Кондак 12}

Благодать
подаждь ми, всех долгов решителю, Иисусе, и приими мя кающася, якоже приял еси Петра, отвергшагося Тебе, и призови мя унывающаго, якоже древле Павла, гоняща Тя, и услыши мя, вопиюща Ти: Аллилуиа.

\mysubsubsection{Икос 12}

Поюще
Твое вочеловечение, восхваляем Тя вси, и веруем со Фомою, яко Господь и Бог еси, седяй со Отцем и хотяй судити живым и мертвым. Тогда убо сподоби мя деснаго стояния, вопиющаго: Иисусе, Царю предвечный, помилуй мя; Иисусе, цвете благовонный, облагоухай мя. Иисусе, теплото любимая, огрей мя; Иисусе, храме предвечный, покрый мя. Иисусе, одеждо светлая, украси мя; Иисусе, бисере честный, осияй мя. Иисусе, каменю драгий, просвети мя; Иисусе, солнце правды, освети мя. Иисусе, свете святый, облистай мя; Иисусе, болезни душевныя и телесныя избави мя. Иисусе, из руки сопротивныя изми мя; Иисусе, огня неугасимаго и прочих вечных мук свободи мя. Иисусе, Сыне Божий, помилуй мя.

\mysubsubsection{Кондак 13}

О,
пресладкий и всещедрый Иисусе! Приими ныне малое моление сие наше, якоже приял еси вдовицы две лепте, и сохрани достояние Твое от враг видимых и невидимых, от нашествия иноплеменних, от недуга и глада, от всякия скорби и смертоносныя раны, и грядущия изми муки всех, вопиющих Ти: Аллилуиа, aллилуиа, aллилуиа.

\mysubsubsection{(Kондак читается трижды)}

\mysubsubsection{Икос 1}

Ангелов
Творче и Господи cил, отверзи ми недоуменный ум и язык на похвалу пречистаго Твоего имене, якоже глухому и гугнивому древле слух и язык отверзл еси, и, глаголаше зовый таковая: Иисусе пречудный, aнгелов удивление; Иисусе пресильный, прародителей избавление. Иисусе пресладкий, патриархов величание; Иисусе преславный, царей укрепление. Иисусе прелюбимый, пророков исполнение; Иисусе предивный, мучеников крепосте. Иисусе претихий, монахов радосте; Иисусе премилостивый, пресвитеров сладосте. Иисусе премилосердый, постников воздержание; Иисусе пресладостный, преподобных радование. Иисусе пречестный, девственных целомудрие; Иисусе предвечный, грешников спасение. Иисусе, Сыне Божий, помилуй мя.

\mysubsubsection{Кондак 1}

Возбранный
Воеводо и Господи, ада победителю, яко избавлься от вечныя смерти, похвальная восписую Ти, создание и раб Твой; но, яко имеяй милосердие неизреченное, от всяких мя бед свободи, зовуща: Иисусе, Сыне Божий, помилуй мя.

\mysubsubsection{Молитва}

Владыко
Господи Иисусе Христе Боже мой, иже неизреченнаго ради Твоего человеколюбия на конец веков во плоть оболкийся от Приснодевы Марии, славлю о мне Твое спасительное промышление, раб Твой, Владыко;
песнословлю Тя, яко Тебе ради Отца познах; благословлю Тя, Егоже ради и Дух Святый в мир прииде; покланяюся Твоей по плоти Пречистой Матери, таковей страшней тайне послужившей; восхваляю Твоя Ангельская ликостояния, яко воспеватели и служители Твоего величествия; ублажаю Предтечу Иоанна, Тебе крестившаго, Господи; почитаю и провозвестившия Тя пророки, прославляю апостолы Твоя святыя; торжествую же и мученики, священники же Твоя славлю; покланяюся преподобным Твоим, и вся Твоя праведники пестунствую. Таковаго и толикаго многаго и неизреченнаго лика Божественнаго в молитву привожду Тебе, всещедрому Богу, раб Твой, и сего ради прошу моим согрешением прощения, еже даруй ми всех Твоих ради святых, изряднее же святых Твоих щедрот, яко благословен еси во веки. Аминь.


\end{mymulticols}

\mychapterending


\mychapter{Акафист Пресвятой Богородице}\begin{mymulticols}
%http://www.molitvoslov.com/text13.htm

\myfigure{5_0}

\mysubsubsection{Кондак 1}

Взбранной Воеводе победительная, яко избавльшеся от злых, благодарственная восписуем Ти раби Твои, Богородице; но яко имущая державу непобедимую, от всяких нас бед свободи, да зовем Ти: радуйся, Невесто Неневестная.

\mysubsubsection{Икос 1}

Ангел предстатель с небесе послан бысть рещи Богородице: радуйся, и со безплотным гласом воплощаема Тя зря, Господи, ужасашеся и стояше, зовый к Ней таковая: Радуйся, Еюже рaдocть возсияет; радуйся, Еюже клятва изчезнет. Радуйся, падшаго Адама воззвание; радуйся, слез Евиных избавление. Радуйся, высото неудобовосходимая человеческими помыслы; радуйся, глубино неудобозримая и ангельскима очима. Радуйся, яко еси Царево седалище; радуйся, яко носиши Носящаго вся. Радуйся, Звездо, являющая Солнце; радуйся, утробо Божественнаго воплощения. Радуйся, Еюже обновляется тварь; радуйся, Еюже покланяемся Творцу. Радуйся, Невесто Неневестная.

\mysubsubsection{Кондак 2}

Видящи Святая Себе в чистоте, глаголет Гавриилу дерзостно: преславное твоего гласа неудобоприятельно души Моей является: безсеменнаго бо зачатия рождество како глаголеши, зовый: Аллилуиа.

\mysubsubsection{Икос 2}

Разум недоразумеваемый разумети Дева ищущи, возопи к служащему: из боку чисту, Сыну како есть родитися мощно, рцы Ми? К Нейже он рече со страхом, обаче зовый сице: Радуйся, совета неизреченнаго Таиннице; радуйся, молчания просящих веро. Радуйся, чудес Христовых начало; радуйся, велений Его главизно. Радуйся, лествице небесная, Еюже сниде Бог; радуйся, мосте, преводяй сущих от земли на небо. Радуйся, Ангелов многословущее чудо; радуйся, бесов многоплачевное поражение. Радуйся, Свет неизреченно родившая; радуйся, еже како, ни единаго же научившая. Радуйся, премудрых превосходящая разум; радуйся, верных озаряющая смыслы. Радуйся, Невесто Неневестная.

\mysubsubsection{Кондак 3}

Сила Вышняго осени тогда к зачатию Браконеискусную, и благоплодная Тоя ложесна, яко село показа сладкое, всем хотящим жати спасение, всегда пети сице: Аллилуиа.

\mysubsubsection{Икос 3}

Имущи Богоприятную Дева утробу, востече ко Елисавети: младенец же оноя абие познав Сея целование, радовашеся, и играньми яко песньми вопияше к Богородице: Радуйся, отрасли неувядаемыя розго; радуйся, Плода безсмертнаго стяжание. Радуйся, Делателя делающая Человеколюбца; радуйся, Садителя жизни нашея рождшая. Радуйся, ниво, растящая гобзование щедрот; радуйся, трапезо, носящая обилие очищения. Радуйся, яко рай пищный процветаеши; радуйся, яко пристанище душам готовиши. Радуйся, приятное молитвы кадило; радуйся, всего мира очищение. Радуйся, Божие к смертным благоволение; радуйся, смертных к Богу дерзновение. Радуйся, Невесто Неневестная.

\mysubsubsection{Кондак 4}

Бурю внутрь имея помышлений сумнительных, целомудренный Иосиф смятеся, к Тебе зря небрачней, и бракоокрадованную помышляя, Непорочная; уведев же Твое зачатие от Духа Свята, рече: Аллилуиа.

\mysubsubsection{Икос 4}

Слышаша пастырие Ангелов поющих плотское Христово пришествие, и текше яко к Пастырю видят Сего яко агнца непорочна, во чреве Мариине упасшася, Юже поюще реша: Радуйся, Агнца и Пастыря Мати; радуйся, дворе словесных овец. Радуйся, невидимых врагов мучение; радуйся, райских дверей отверзение. Радуйся, яко небесная срадуются земным; радуйся, яко земная сликовствуют небесным. Радуйся, апостолов немолчная уста; радуйся, страстотерпцев непобедимая дерзосте. Радуйся, твердое веры утверждение; радуйся, светлое благодати познание. Радуйся, Еюже обнажися ад; радуйся, Еюже облекохомся славою. Радуйся, Невесто Неневестная.

\mysubsubsection{Кондак 5}

Боготечную звезду узревше волсви, тоя последоваша зари, и яко светильник держаще ю, тою испытаху крепкаго Царя, и достигше Непостижимаго, возрадовашася, Ему вопиюще: Аллилуиа.

\mysubsubsection{Икос 5}

Видеша отроцы халдейстии на руку Девичу Создавшаго руками человеки, и Владыку разумевающе Его, аще и рабий прият зрак, потщашася дарми послужити Ему, и возопити Благословенней: Радуйся, Звезды незаходимыя Мати; радуйся, заре таинственнаго дне. Радуйся, прелести пещь угасившая; радуйся, Троицы таинники просвещающая. Радуйся, мучителя безчеловечнаго изметающая от начальства; радуйся, Господа Человеколюбца показавшая Христа. Радуйся, варварскаго избавляющая служения; радуйся, тимения изымающая дел. Радуйся, огня поклонение угасившая; радуйся, пламене страстей изменяющая. Радуйся, верных наставнице целомудрия; радуйся, всех родов веселие. Радуйся, Невесто Неневестная.

\mysubsubsection{Кондак 6}

Проповедницы богоноснии, бывше волсви, возвратишася в Вавилон, скончавше Твое пророчество, и проповедавше Тя Христа всем, оставиша Ирода яко буесловна, не ведуща пети: Аллилуиа.

\mysubsubsection{Икос 6}

Возсиявый во Египте просвещение истины, отгнал еси лжи тьму: идоли бо его, Спасе, не терпяще Твоея крепости, падоша, сих же избавльшиися вопияху к Богородице: Радуйся, исправление человеков; радуйся, низпадение бесов. Радуйся, прелести державу поправшая; радуйся, идольскую лесть обличившая. Радуйся, море, потопившее фараона мысленнаго; радуйся, каменю, напоивший жаждущия жизни. Радуйся, огненный столпе, наставляяй сущия во тьме; радуйся, покрове миру, ширший облака. Радуйся, пище, манны приемнице; радуйся, сладости святыя служительнице. Радуйся, земле обетования; радуйся, из неяже течет мед и млеко. Радуйся, Невесто Неневестная.

\mysubsubsection{Кондак 7}

Хотящу Симеону от нынешняго века преставитися прелестнаго, вдался еси яко младенец тому, но познался еси ему и Бог совершенный. Темже удивися Твоей неизреченней премудрости, зовый: Аллилуиа.

\mysubsubsection{Икос 7}

Новую показа тварь, явлься Зиждитель нам от Него бывшим, из безсеменныя прозяб утробы, и сохранив Ю, якоже бе, нетленну, да чудо видяще, воспоем Ю, вопиюще: Радуйся, цвете нетления; радуйся, венче воздержания. Радуйся, воскресения образ облистающая; радуйся, ангельское житие являющая. Радуйся, древо светлоплодовитое, от негоже питаются вернии; радуйся, древо благосеннолиственное, имже покрываются мнози. Радуйся, во чреве носящая Избавителя плененным; радуйся, рождшая Наставника заблудшим. Радуйся, Судии праведнаго умоление; радуйся, многих согрешений прощение. Радуйся, одеждо нагих дерзновения; радуйся, любы, всякое желание побеждающая. Радуйся, Невесто Неневестная.

\mysubsubsection{Кондак 8}

Странное рождество видевше, устранимся мира, ум на небеса преложше: сего бо ради высокий Бог на земли явися смиренный человек, хотяй привлещи к высоте Тому вопиющия: Аллилуиа.

\mysubsubsection{Икос 8}

Весь бе в нижних и вышних никакоже отступи неописанное Слово: снизхождение бо Божественное, не прехождение же местное бысть, и рождество от Девы Богоприятныя, слышащия сия: Радуйся, Бога невместимаго вместилище; радуйся, честнаго таинства двери. Радуйся, неверных сумнительное слышание; радуйся, верных известная похвало. Радуйся, колеснице пресвятая Сущаго на Херувимех; радуйся, селение преславное Сущаго на Серафимех. Радуйся, противная в тожде собравшая; радуйся, девство и рождество сочетавшая. Радуйся, Еюже разрешися преступление; радуйся, Еюже отверзеся рай. Радуйся, ключу Царствия Христова; радуйся, надеждо благ вечных. Радуйся, Невесто Неневестная.

\mysubsubsection{Кондак 9}

Всякое естество ангельское удивися великому Твоего вочеловечения делу; неприступнаго бо яко Бога, зряще всем приступнаго Человека, нам убо спребывающа, слышаща же от всех: Аллилуиа.

\mysubsubsection{Икос 9}

Ветия многовещанныя, яко рыбы безгласныя видим о Тебе, Богородице, недоумевают бо глаголати, еже како и Дева пребываеши, и родити возмогла еси. Мы же, таинству дивящеся, верно вопием: Радуйся, премудрости Божия приятелище, радуйся, промышления Его сокровище. Радуйся, любомудрыя немудрыя являющая; радуйся, хитрословесныя безсловесныя обличающая. Радуйся, яко обуяша лютии взыскателе; радуйся, яко увядоша баснотворцы. Радуйся, афинейская плетения растерзающая; радуйся, рыбарския мрежи исполняющая. Радуйся, из глубины неведения извлачающая; радуйся, многи в разуме просвещающая. Радуйся, кораблю хотящих спастися; радуйся, пристанище житейских плаваний. Радуйся, Невесто Неневестная.

\mysubsubsection{Кондак 10}

Спасти хотя мир, Иже всех Украситель, к сему самообетован прииде, и Пастырь сый, яко Бог, нас ради явися по нам человек: подобным бо подобное призвав, яко Бог слышит: Аллилуиа.

\mysubsubsection{Икос 10}

Стена еси девам, Богородице Дево, и всем к Тебе прибегающим: ибо небесе и земли Творец устрои Тя, Пречистая, вселься во утробе Твоей, и вся приглашати Тебе научи: Радуйся, столпе девства; радуйся, дверь спасения. Радуйся, начальнице мысленнаго наздания; радуйся, подательнице Божественныя благости. Радуйся, Ты бо обновила еси зачатыя студно; радуйся, Ты бо наказала еси окраденныя умом. Радуйся, тлителя смыслов упраждняющая; радуйся, Сеятеля чистоты рождшая. Радуйся, чертоже безсеменнаго уневещения; радуйся, верных Господеви сочетавшая. Радуйся, добрая младопитательнице девам; радуйся, невестокрасительнице душ святых. Радуйся, Невесто Неневестная.

\mysubsubsection{Кондак 11}

Пение всякое побеждается, спростретися тщащееся ко множеству многих щедрот Твоих: равночисленныя бо песка песни аще приносим Ти, Царю Святый, ничтоже совершаем достойно, яже дал еси нам, Тебе вопиющим: Аллилуиа.

\mysubsubsection{Икос 11}

Светоприемную свещу, сущим во тьме явльшуюся, зрим Святую Деву, невещественный бо вжигающи огнь, наставляет к разуму Божественному вся, зарею ум просвещающая, званием же почитаемая, сими: Радуйся, луче умнаго Солнца; радуйся, светило незаходимаго Света. Радуйся, молние, души просвещающая; радуйся, яко гром враги устрашающая. Радуйся, яко многосветлое возсияваеши просвещение; радуйся, яко многотекущую источаеши реку. Радуйся, купели живописующая образ; радуйся, греховную отъемлющая скверну. Радуйся, бане, омывающая совесть; радуйся, чаше, черплющая рaдocть. Радуйся, обоняние Христова благоухания; радуйся, животе тайнаго веселия. Радуйся, Невесто Неневестная.

\mysubsubsection{Кондак 12}

Благодать дати восхотев, долгов древних, всех долгов Решитель человеком, прииде Собою ко отшедшим Того благодати, и раздрав рукописание, слышит от всех сице: Аллилуиа.

\mysubsubsection{Икос 12}

Поюще Твое Рождество, хвалим Тя вси, яко одушевленный храм, Богородице: во Твоей бо вселився утробе содержай вся рукою Господь, освяти, прослави и научи вопити Тебе всех: Радуйся, селение Бога и Слова; радуйся, святая святых большая. Радуйся, ковчеже, позлащенный Духом; радуйся, сокровище живота неистощимое. Радуйся, честный венче людей благочестивых; радуйся, честная похвало иереев благоговейных. Радуйся, церкве непоколебимый столпе; радуйся, Царствия нерушимая стено. Радуйся, Еюже воздвижутся победы; радуйся, Еюже низпадают врази. Радуйся, тела моего врачевание; радуйся, души моея спасение. Радуйся, Невесто Неневестная.

\mysubsubsection{Кондак 13}

О, Всепетая Мати, рождшая всех святых Святейшее Слово! Нынешнее приемши приношение, от всякия избави напасти всех, и будущия изми муки, о Тебе вопиющих: Аллилуиа, aллилуиа, aллилуиа.

\myemph{ (Kондак читается трижды)}

\mysubsubsection{Икос 1}

Ангел предстатель с небесе послан бысть рещи Богородице: радуйся, и со безплотным гласом воплощаема Тя зря, Господи, ужасашеся и стояше, зовый к Ней таковая: Радуйся, Еюже рaдocть возсияет; радуйся, Еюже клятва изчезнет. Радуйся, падшаго Адама воззвание; радуйся, слез Евиных избавление. Радуйся, высото неудобовосходимая человеческими помыслы; радуйся, глубино неудобозримая и ангельскима очима. Радуйся, яко еси Царево седалище; радуйся, яко носиши Носящаго вся. Радуйся, Звездо, являющая Солнце; радуйся, утробо Божественнаго воплощения. Радуйся, Еюже обновляется тварь; радуйся, Еюже покланяемся Творцу. Радуйся, Невесто Неневестная.

\mysubsubsection{Кондак 1}

Взбранной Воеводе победительная, яко избавльшеся от злых, благодарственная восписуем Ти раби Твои, Богородице; но яко имущая державу непобедимую, от всяких нас бед свободи, да зовем Ти: радуйся, Невесто Неневестная.

\mysubsubsection{Молитвы}

О, Пресвятая Госпоже Владычице Богородице, вышши еси всех Ангел и Архангел, и всея твари честнейши, помощнице еси обидимых, ненадеющихся надеяние, убогих заступнице, печальных утешение, алчущих кормительнице, нагих одеяние, больных исцеление, грешных спасение, христиан всех поможение и заступление. О, Всемилостивая Госпоже, Дево Богородице Владычице, милостию Твоею спаси и помилуй святейшия патриархи православныя, преосвященныя митрополиты, архиепископы и епископы и весь священнический и иноческий чин, и вся православныя христианы ризою Твоею честною защити; и умоли, Госпоже, из Тебе без семене воплотившагося Христа Бога нашего, да препояшет нас силою Своею свыше, на невидимыя и видимыя враги наша. О, Всемилостивая Госпоже Владычице Богородице! Воздвигни нас из глубины греховныя и избави нас от глада, губительства, от труса и потопа, от огня и меча, от нахождения иноплеменных и междоусобныя брани, и от напрасныя смерти, и от нападения вражия, и от тлетворных ветр, и от смертоносныя язвы, и от всякаго зла. Подаждь, Госпоже, мир и здравие рабом Твоим, всем православным христианом, и просвети им ум, и очи сердечнии, еже ко спасению; и сподоби ны, грешныя рабы Твоя, Царствия Сына Твоего, Христа Бога нашего; яко держава Его благословена и препрославлена, со Безначальным Его Отцем, и с Пресвятым, и Благим, и Животворящим Его Духом, ныне и присно, и во веки веков. Аминь.

О, Пресвятая Дево Мати Господа, Царице Небесе и земли! Вонми многоболезненному воздыханию души нашея, призри с высоты святыя Твоея на нас, с верою и любовию покланяющихся пречистому образу Твоему. Се бо грехми погружаемии и скорбьми обуреваемии, взирая на Твой образ, яко живей Ти сущей с нами, приносим смиренныя моления наша. Не имамы бо ни иныя помощи, ни инаго предстательства, ни утешения, токмо Тебе, о, Мати всех скорбящих и обремененных. Помози нам немощным, утоли скорбь нашу, настави на путь правый нас, заблуждающих, уврачуй и спаси безнадежных, даруй нам прочее время живота нашего в мире и тишине проводити, подаждь христианскую кончину, и на страшнем суде Сына Твоего явися нам милосердая Заступница, да всегда поем, величаем и славим Тя, яко благую Заступницу рода христианскаго, со всеми угодившими Богу. Аминь.


\end{mymulticols}

\mychapterending


\mychapter{Благодарственные молитвы по Святом Причащении}\begin{mymulticols}
%http://www.molitvoslov.com/text206.htm

Слава Тебе, Боже. Слава Тебе, Боже. Слава Тебе, Боже.

\mysubsubsection{Благодарственная молитва, 1-я}

Благодарю тя, Господи, Боже мой, яко не отринул мя еси грешнаго, но общника мя быти святынь Твоих сподобил еси. Благодарю Тя, яко мене недостойнаго причаститися Пречистых Твоих и Небесных Даров сподобил еси. Но Владыко Человеколюбче, нас ради умерый же и воскресый, и даровавый нам страшная сия и животворящая Таинства во благодеяние и освящение душ и телес наших, даждь быти сим и мне во исцеление души же и тела, во отгнание всякаго сопротивнаго, в просвещение очию сердца моего, в мир душевных моих сил, в веру непостыдну, в любовь нелицемерну, во исполнение премудрости, в соблюдение заповедей Твоих, в приложение Божественныя Твоея благодати и Твоего Царствия присвоение; да во святыни Твоей теми сохраняемь, Твою благодать поминаю всегда, и не к тому себе живу, но Тебе, нашему Владыце и Благодетелю; и тако сего жития исшед о надежди живота вечнаго, в присносущный достигну покой, идеже празднующих глас непрестанный, и безконечная сладость, зрящих Твоего лица доброту неизреченную. Ты бо еси истинное желание, и неизреченное веселие любящих Тя, Христе Боже наш, и Тя поет вся тварь во веки. Аминь.

\mysubsubsection{Молитва 2-я, святого Василия Великого}

Владыко Христе Боже, Царю веков, и Содетелю всех, Благодарю Тя о всех, яже ми подал благих, и о причащении пречистых и животворящих Твоих Таинств. Молю убо Тя, Блаже и Человеколюбче: сохрани мя под кровом Твоим, и в сени крилу Твоею; и даруй ми чистою совестию, даже до последняго моего издыхания, достойно причащатися святынь Твоих, во оставление грехов, и в жизнь вечную. Ты бо еси хлеб животный, источник святыни, Податель благих, и Тебе славу возсылаем, со Отцем и Святым Духом, ныне и присно, и во веки веков. Аминь.

\mysubsubsection{Молитва 3-я, Симеона Метафраста}

Давый пищу мне плоть Твою волею, огнь сый и опаляяй недостойныя, да не опалиши мене, Содетелю мой; паче же пройди во уды моя, во вся составы, во утробу, в сердце. Попали терние всех моих прегрешений. Душу очисти, освяти помышления. Составы утверди с костьми вкупе. Чувств просвети простую пятерицу. Всего мя спригвозди страху Твоему. Присно покрый, соблюди же, и сохрани мя от всякаго дела и слова душетленнаго. Очисти и омый, и украси мя; удобри, вразуми, и просвети мя. Покажи мя Твое селение единаго Духа, и не ктому селение греха. Да яко Твоего дому, входом причащения, яко огня мене бежит всяк злодей, всяка страсть. Молитвенники Тебе приношу вся святыя, чиноначалия же безплотных, Предтечу Твоего, премудрыя Апостолы, к сим же Твою нескверную чистую Матерь, ихже мольбы Благоутробне приими, Христе мой, и сыном света соделай Твоего служителя. Ты бо еси освящение и единый наших, Блаже, душ и светлость; и Тебе лепоподобно, яко Богу и Владыце, славу вси возсылаем на всяк день.

\mysubsubsection{Молитва 4-я}

Тело Твое Святое, Господи, Иисусе Христе, Боже наш, да будет ми в живот вечный, и Кровь Твоя Честная во оставление грехов: буди же ми благодарение сие в радость, здравие и веселие; в страшное же и второе пришествие Твое сподоби мя грешнаго стати одесную славы Твоея, молитвами Пречистыя Твоея Матере, и всех святых.

\mysubsubsection{Молитва 5-я, ко Пресвятой Богородице}

Пресвятая Владычице Богородице, свете помраченныя моея души, надеждо, покрове, прибежище, утешение, радование мое, благодарю Тя, яко сподобила мя еси недостойнаго, причастника быти Пречистаго Тела и Честныя Крове Сына Твоего. Но Рождшая истинный Свет, просвети моя умныя очи сердца; Яже Источник безсмертия рождшая, оживотвори мя умерщвленнаго грехом; Яже милостиваго Бога любоблагоутробная Мати, помилуй мя, и даждь ми умиление и сокрушение в сердце моем, и смирение в мыслех моих, и воззвание в пленениих помышлений моих; и сподоби мя до последняго издыхания, неосужденно приимати пречистых Таин освящение, во исцеление души же и тела. И подаждь ми слезы покаяния и исповедания, во еже пети и славити Тя во вся дни живота моего, яко благословенна и препрoславленна еси во веки. Аминь.

Ныне отпущаеши раба Твоего, Владыко, по глаголу Твоему, с миром: яко видеста очи мои спасение Твое, еже еси уготовал пред лицем всех людей, свет во откровение языков и славу людей Твоих, Израиля.

Святый Боже, Святый Крепкий, Святый Безсмертный, помилуй нас. \myemph{ (Tрижды)}

Слава Отцу и Сыну и Святому Духу, и ныне и присно и во веки веков. Аминь.

Пресвятая Троице, помилуй нас; Господи, очисти грехи наша; Владыко, прости беззакония наша; Святый, посети и исцели немощи наша, имене Твоего ради.

Господи, помилуй. \myemph{ (Tрижды)}

Слава Отцу и Сыну и Святому Духу, и ныне и присно и во веки веков. Аминь.

Отче наш, Иже еси на небесех! Да святится имя Твое, да приидет Царствие Твое, да будет воля Твоя, яко на небеси и на земли. Хлеб наш насущный даждь нам днесь; и остави нам долги наша, якоже и мы оставляем должником нашим; и не введи нас во искушение, но избави нас от лукаваго.

\mysubsubsection{Тропарь св. Иоанну Златоустому, глас 8-й:}

Уст твоих, якоже светлость огня, возсиявши благодать, вселенную просвети: не сребролюбия мирови сокровища сниска, высоту нам смиренномудрия показа, но твоими словесы наказуя, отче Иоанне Златоусте, моли Слова Христа Бога спастися душам нашим.

\mysubsubsection{Кондак, глас 6-й:}

\slava

От небес приял еси Божественную благодать, и твоими устнама вся учиши покланятися в Троице единому Богу, Иоанне Златоусте, всеблаженне преподобне, достойно хвалим тя: еси бо наставник, яко божественная являя.

\inyne

\Bogorodichen{Предстательство христиан непостыдное, ходатайство ко Творцу непреложное, не презри грешных молений гласы, но предвари, яко Благая, на помощь нас, верно зовущих Ти: ускори на молитву, и потщися на умоление, предстательствующи присно, Богородице, чтущих Тя.}

\myemph{ Если совершалась литургия святого Василия Великого, читай}

\mysubsubsection{Тропарь Василию Великому, глас 1-й:}

Во всю землю изыде вещание твое, яко приемшую слово твое, имже боголепно научил еси, естество сущих уяснил еси, человеческия обычаи украсил еси, царское священие, отче преподобне, моли Христа Бога, спастися душам нашим.

\vspace{-\baselineskip}\mysubsubsection{Кондак, глас 4-й:}

\slava

Явился еси основание непоколебимое церкве, подая всем некрадомое господство человеком, запечатлея твоими веленьми, небоявленне Василие преподобне.

\inyne

\Bogorodichen{Предстательство христиан непостыдное, ходатайство ко Творцу непреложное, не презри грешных молений гласы, но предвари, яко Благая, на помощь нас, верно зовущих Ти: ускори на молитву, и потщися на умоление, предстательствующи присно, Богородице, чтущих Тя.}

\myemph{ Если совершалась Литургия Преждеосвященных Даров, читай}

\mysubsubsection{Тропарь святому Григорию Двоеслову, глас 4-й:}

Иже от Бога свыше божественную благодать восприем, славне Григорие, и Того силою укрепляемь, евангельски шествовати изволил еси, отонудуже у Христа возмездие трудов приял еси всеблаженне: Егоже моли, да спасет души наша.

\vspace{-\baselineskip}\mysubsubsection{Кондак, глас 3-й:}

\slava

{\smallПодобоначальник показался еси Начальника пастырем Христа, иноков чреды, отче Григорие, ко ограде небесней наставляя, и оттуду научил еси стадо Христово заповедем Его: ныне же с ними радуешися, и ликуеши в небесных кровех.}

\inyne

\Bogorodichen{Предстательство христиан непостыдное, ходатайство ко Творцу непреложное, не презри грешных молений гласы, но предвари, яко Благая, на помощь нас, верно зовущих Ти: ускори на молитву, и потщися на умоление, предстательствующи присно, Богородице, чтущих Тя.}

Господи, помилуй. \myemph{ (12 раз)}

Слава Отцу и Сыну и Святому Духу, и ныне и присно и во веки веков. Аминь.

Честнейшую херувим и славнейшую без сравнения Серафим, без истления Бога Слова рождшую, сущую Богородицу Тя величаем.

\myemph{После Причащения да пребывает каждый в чистоте, воздержании и немногословии, чтобы достойно сохранить в себе принятого Христа.}

\end{mymulticols}

\mychapterending

\mychapter{Символ веры}\begin{mymulticols}
%http://www.molitvoslov.com/content/Simvol-very 
 
\SymbolOfFaith

\end{mymulticols}

\mychapterending

