

\mypart{В ПОМОЩЬ КАЮЩЕМУСЯ}
%http://www.molitvoslov.com/content/vpomoch-kayuchemusya

 

\mychapter{Семь Таинств Церкви}
%http://www.molitvoslov.com/text163.htm 
 


1. Крещение


2. Миропомазание


3. Евхаристия


4. Покаяние


5. Священство


6. Брак


7. Елеосвящение (Соборование)


\mychapterending

\mychapter{Десять заповедей}
%http://www.molitvoslov.com/text547.htm 
 


1. Аз есмь Господь Бог твой; да не будут тебе бози инии, разве Мене\itshape  (кроме Меня)\normalfont{}.

2. Не сотвори себе кумира и всякаго подобия, елика на небеси горе, и елика на земли низу, и елика \itshape (что)\normalfont{} в водах под землею: да не поклонишися им, ни послужиши им.

3. Не приемли \itshape (не произноси)\normalfont{} имене Господа Бога твоего всуе\itshape  (напрасно)\normalfont{}.

4. Помни день субботний, еже святити \itshape (чтобы святить) \normalfont{}его: шесть дней делай, и сотвориши в них вся дела твоя, в день же седьмый "--- суббота Господу Богу твоему.

5. Чти отца твоего и матерь твою, да благо ти будет, и да долголетен будеши на земли.

6. Не убий.

7. Не прелюбы сотвори.

8. Не укради.

9. Не послушествуй на друга твоего свидетельства ложна \itshape (не произноси ложного свидетельства на ближнего своего)\normalfont{}.

10. Не пожелай жены искренняго твоего, не пожелай дому ближняго твоего, ни села его, ни раба его, ни рабыни его, ни вола его, ни осла его, ни всякаго скота его, ни всего, елика суть ближняго твоего.


Суть этих заповедей Господь Иисус Христос изложил так: «Возлюби Господа Бога твоего всем сердцем твоим, и всею душею твоею, и всем разумением твоим. Сия есть первая и наибольшая заповедь. Вторая же, подобная ей: возлюби ближнего твоего, как самого себя».

(Мф. 22 , 37 –39.) 





\mychapterending

\mychapter{Блаженства Евангельские}
%http://www.molitvoslov.com/text162.htm 
 


(\itshape Евангелие от Матфея. Глава 5, стихи 3-12) 

\normalfont{}

Блажени нищии духом, яко тех есть Царство Небесное. 

Блажени плачущии, яко тии утешатся. 

Блажени кротцыи, яко тии наследят землю. 

Блажени алчущии и жаждущии правды, яко тии насытятся. 

Блажени милостивии, яко тии помилованы будут. 

Блажени чистии сердцем, яко тии Бога узрят. 

Блажени миротворцы, яко тии сынове Божии нарекутся. 

Блажени изгнани правды ради, яко тех есть Царство Небесное. 

Блажени есте, егда поносят вам, и ижденут, и рекут всяк зол глагол на вы лжуще Мене ради. 

Радуйтеся и веселитеся, яко мзда ваша многа на небесех. 


\mychapterending

\mychapter{Определение грехов по десяти заповедям}
%http://www.molitvoslov.com/text164.htm 
 


\bfseries \itshape По первой заповеди: 

\normalfont{}\normalfont{}1. Имеешь ли постоянную память о Боге и страх Божий в сердце?

2. Не колеблется ли вера твоя в Бога маловерием, сомнениями?

3. Не сомневался ли в святых догматах Веры Православной?

4. Молишь ли Господа Бога укрепить твою веру?

5. Не терял ли надежду на Божие милосердие и помощь?

6. Молишься ли ты каждый день, утром, вечером Богу? Усердна ли твоя молитва?

7. Всегда ли, когда можно посещаешь церковныя богослужения? Не опускаешь ли их без важных причин?

8. Любишь ли читать религиозно-нравственныя книги и читаешь ли их? Не читал ли из грешнаго любопытства книг безбожных и еретических?

9. Жертвуешь ли охотно, по призыву Церкви, на благотворительныя цели и на св. храм?

10. Не прибегал ли к гаданиям? Не участвовал ли в сеансах общения с нечистой силой?

11. Не забываешь ли о главном, о жизни по Закону Божию, о приготовлении к вечности и ответу пред Богом, предаваясь суете, лени, удовольствиям, безпечности? 

\bfseries \itshape 

По второй заповеди

\normalfont{}\normalfont{}1. На первом ли месте у тебя Бог? Может быть, не Бог у тебя на первом месте, а что-либо иное, например: собирание денег, приобретение имущества, удовольствия, развлечения, пища, напитки, одежда, украшения, стремление обращать на себя внимание, играть первую роль, получать похвалы, проводить время в разсеянии, в чтении пустых книг и т. д.?

2. Не отвлекает ли тебя от Господа Бога увлечение телевидением, кино, театром, картами?

3. Может быть, из-за забот о себе, о своей семье забываешь Бога и не живешь по Его заповедям и не исполняешь правила нашей Матери-Церкви?

4. Если так, то, значит, служишь ты своему «кумиру», своему идолу, он у тебя на первом месте, а не Господь Бог и Его учение.

5. Может быть, искусство, спорт, наука занимают у тебя первое место? Может быть, какая-либо страсть (сребролюбие, чревоугодие, плотская любовь и др.) завладела твоим сердцем.

6. Не делаешь ли из себя самого «кумира» по гордости, по эгоизму? Проверь себя. 

\bfseries \itshape 

По третьей заповеди

\normalfont{}\normalfont{}1. Не божился ли в обыкновенных, житейских разговорах, не употреблял ли легкомысленно, без благоговения имя Божие или, что еще хуже, святыню не обращал ли в шутку? Или, не дай Бог, в припадке ожесточения, злобы, отчаяния не позволил ли себе дерзко роптать на Бога или даже хулить Бога?

2. Или дав какую-либо клятву или присягу, потом нарушил ее?

3. Не предавался ли унынию?

4. Не бывает ли разсеянною, невнимательной твоя молитва Богу? 

\bfseries \itshape 

По четвёртой заповеди

\normalfont{}\normalfont{}1. Не нарушаешь ли святости воскресных дней и великих праздников, установленных св. Церковью?

2. Не работаешь ли в эти дни ради выгоды, прибыли?

3. Вместо праздничных богослужений не проводил ли время на каком-либо увеселении, на балу, в театре, кино или на каком бы то ни было собрании, где нет речи о Боге, где нет молитвы, коей надлежит встречать праздничный день? Не устраивал ли сам такия увеселения и собрания, отвлекая таким образом людей от посещения церкви?

4. Посещаешь ли аккуратно церковныя богослужения? Не приходишь ли в церковь с большим опозданием, к средине или концу богослужения?

5. Посещаешь ли в воскресенье и праздничные дни больных? Помогаешь ли бедным, нуждающимся?

6. Не нарушал ли св. постов?

7. Не упивался ли спиртными напитками? 

\bfseries \itshape 

По пятой заповеди

\normalfont{}\normalfont{}1. Не было ли случаев непочитательнаго отношения к родителям, невнимательнаго отношения к их заботе и их советам? Заботился ли о них в болезнях их, в старости?

2. Если умерли родители твои, часто-ли молишься об упокоении их душ в церкви и в домашней молитве?

3. Не было ли случаев непочтительнаго отношения к пастырям Церкви?

4. Не осуждал ли их? Не озлоблялся ли на них, когда они напоминали о вечности, о приготовлении к ней, о спасении души, о грехах? Когда они призывают к следованию Церкви и ея учению?

5. Не оскорбил ли кого старше себя и особенно благодетелей? 

\bfseries \itshape 

По шестой заповеди

\normalfont{}\normalfont{}1. Ты никого не убил физически в прямом и буквальном смысле, но, может быть, был причиною смерти кого-либо косвенным образом: мог помочь бедному или больному и не сделал этого, алчущаго не накормил, жаждущаго не напоил, странника не принял, нагого не одел, больного и находящагося в темнице не посетил (Матфея 25:34-36)?

2. Не совершил ли ты духовнаго убийства, то есть, не совратил ли кого с добраго пути жизни, не увлек ли в ересь или в раскол церковный, не соблазнил ли на грех?

3. Не убил ли кого греховно проявлением злобы и ненависти к нему?

4. Прощаешь ли обижающих тебя? Не таишь ли долго злобу в сердце и обиду?

5. Винишь ли во всем себя или только других?

6. Не прибегала ли к недозволенным операциям, что тоже есть убийство, грех и жены, и мужа? 

\bfseries \itshape 

По седьмой заповеди

\normalfont{}\normalfont{}1. Не сожительствовал ли с лицом другого пола, находясь с ним в плотских отношениях, без церковнаго брака или довольствуясь только гражданским браком? Не упорствуешь ли в этом, уклоняясь от церковнаго брака?

2. Не позволяешь ли себе легкомысленно обращаться с лицами другого пола?

3. Не оскверняешься ли, допуская себе предаваться нечистым и развратным мыслям и вожделениям? Чтению нечистых книг, разсматриванию нечистых картин?

4. Вспомни грешныя песни, страстные танцы, шутки, сквернословия, нескромныя зрелища, наряды, пьянство и подобные грехи.

5. Помни, христианин, что до тех пор, пока ты не исповедуешься в грехе незаконнаго сожительства или будешь довольствоваться только гражданским браком, без церковнаго, пока ты не прекратишь этого греха разлучением или вступлением в церковный брак, ты вообще не смеешь приступать к причащению Св. Христовых Таин, как и не имеешь никакого голоса в церковных делах.

6. Наиболее гибнет людей из-за нарушения 7-ой заповеди, так как люди стыдятся исповедывать свои грехи против этой заповеди, как это видно из слов Ангела преподобной Феодоре, при прохождении ею мытарств. После прохождения преподобною Феодорою 16, 17, 18 мытарств, ангел сказал Феодоре: Ты видела страшныя, отвратительныя блудныя мытарства, знай, что редкая душа минует их свободно, весь мир погружен во зле соблазнов и скверн, все человеки сластолюбивы. Большая часть, дошедши сюда, гибнет: лютые истязатели блудных грехов похищают души блудников и низводят их в ад.

7. Будь же мужествен, христианин, и покайся, пока жив, пока еще не поздно. 

\bfseries \itshape 

По восьмой заповеди

\normalfont{}\normalfont{}1. Не присваивал ли себе чужой собственности прямым или косвенным образом? Обманом, разными хитростями, комбинациями? Может, не исполнял, как должно того, что обязан был исполнять за полученное тобою вознаграждение?

2. Не пристращался ли чрезмерно к земным благам, не желая делиться ими с другими, нуждающимися в них?

3. Не овладевает ли твоею душою скупость?

4. Не принимал ли краденнаго? По совести ли распоряжался чужим добром, если оно тебе было доверено? 

\bfseries \itshape 

По девятой заповеди

\normalfont{}\normalfont{}1. Не клеветал ли на ближняго твоего? Не осуждал ли часто других, злословил, поносил их, за действительные ли их грехи и пороки или только за кажущиеся?

2. Не любишь ли слушать о ком-либо дурную молву, а потом охотно разносишь ее, увлекаясь всякими сплетнями, пересудами, празднословием?

3. Не прибегаешь ли иногда ко лжи, неправде? Стараешься ли быть всегда правдивым? 

\bfseries \itshape 

По десятой заповеди

\normalfont{}\normalfont{}1. Не завидуешь ли другим? Если ты завидуешь тому, что есть хорошаго или ценнаго у других людей, то это чувство может довести тебя до какого-либо тяжкаго преступления самым делом. Помни, что злобная зависть книжников и фарисеев возвела на крест Самого Сына Божия, пришедшаго на землю спасти людей.

2. Зависть всегда приводит к злобе и ненависти и способна бывает довести до самых безумных поступков, вплоть до убийства.





\mychapterending

\mychapter{Молитва перед исповедью}
%http://www.molitvoslov.com/text166.htm 
 


\itshape (Преподобного Симеона Нового Богослова) 

\normalfont{}

Боже и Господи всех! Всякаго дыхания и души имый власть, Един исцелити мя могий, услыши моление мя, окаяннаго, и гнездящагося во мне змия наитием Всесвятаго и Животворящаго Духа умертвив потреби: и мене нища и нага всякия добродетели суща, к ногам святаго моего отца (духовнаго) со слезами припасти сподоби, и святую его душу к милосердию, еже миловати мя, привлецы. И даждь, Господи, в сердце моем смирение и помыслы благи, подобающия грешнику, согласившемуся Тебе каятися, и да не вконец оставиши душу едину, сочетавшуюся Тебе и исповедавшую Тя, и вместо всего мира избравшую и предпочетшую Тя: веси бо, Господи, яко хощу спастися, аще и лукавый мой обычай препятствием бывает: но возможна Тебе, Владыко, суть вся, елика невозможно суть от человека. Аминь. 


\mychapterending

\mychapter{Молитва вторая перед исповедью}
%http://www.molitvoslov.com/text167.htm 
 


Господи, помоги мне чистосердечно покаяться. 


\mychapterending

\mychapter{Исповедь}
%http://www.molitvoslov.com/text168.htm 
 
Исповедаю аз многогрешный (имя) Господу Богу и Спасу нашему Иисусу Христу и тебе, честный отче, вся согрешения моя и вся злая моя дела, яже содеял во все дни жизни моей, яже помыслил даже до сего дня. 



Согрешил: Обеты Св. Крещения не соблюл, иноческого обещания не сохранил, но во всем солгал и непотребна себе пред Лицем Божиим сотворил. 



Прости нас, Милосердный Господи \itshape (для народа).

\normalfont{}Прости мя, честный отче \itshape (для одиноких).\normalfont{} 



Согрешил: пред Господом маловерием и замедлением в помыслах, от врага всеваемых против веры и св. Церкви; неблагодарностью за все Его великия и непрестанные благодеяния, призыванием Имени Божия без нужды "--- всуе. 



Прости мя, честный отче. 



Согрешил: неимением ко Господу любви ниже страха, неисполнением св. воли Его и св. Заповедей, небрежным изображением на себе крестнаго знамения, неблагоговейным почитанием св. икон; не носил креста, стыдился крестить и исповедывать Господа. 



Прости мя, честный отче. 



Согрешил: любви к ближнему не сохранил, не питал алчущих и жаждущих, не одевал нагих, не посещал больных и в темницах заключенных; Закону Божию и св. отцов преданиям от ленности и небрежения не поучался. 



Прости мя, честный отче. 



Согрешил: церковного и келейного правила неисполнением, хождением в храм Божий без усердия, с леностию и небрежением; оставлением утренних, вечерних и других молитв; во время церковной службы "--- согрешил празднословием, смехом, дреманием, невниманием к чтению и пению, рассеянностию ума, исхождением из храма во время службы и нехождением в храм Божий по лености и нерадению. 



Прости мя, честный отче. 



Согрешил: дерзая в нечистоте ходить в храм Божий и всякия святыни прикасатися. 



Прости мя, честный отче. 



Согрешил: непочитанием праздников Божиих; нарушением св. постов и нехранением постных дней "--- среды и пятницы; невоздержанием в пище и питии, многоядением, тайноядением, разноядением, пьянством, недовольством пищей и питием, одеждой, тунеядством; своея воли и разума исполнением, самонравием, самочинием и самооправданием; не должным почитанием родителей, не воспитанием детей в православной вере, проклинанием детей своих и ближних. 



Прости мя, честный отче. 



Согрешил: неверием, суеверием, сомнением, отчаянием, унынием, кощунством, божбою ложною, плясанием, курением, игрой в карты, гаданием, колдовством, чародейством, сплетнями, поминал живых за упокой, ел кровь животных. 



Прости мя, честный отче. 



Согрешил: гордостию, самомнением, высокоумием, самолюбием, честолюбием, завистию, превозношением, подозрительностию, раздражительностию. 



Прости мя, честный отче. 



Согрешил: осуждением всех людей "--- живых и мертвых, злословием и гневом, памятозлобием, ненавистию, зло за зло воздаянием, оклеветанием, укорением, лукавством, леностию, обманом, лицемерием, пересудами, спорами, упрямством, нежеланием уступить и услужить ближнему; согрешил злорадством, зложелательством, злосетованием, оскорблением, надсмеянием, поношением и человекоугодием. 



Прости мя, честный отче. 



Согрешил: невоздержанием душевных и телесных чувств; нечистотою душевною и телесною, услаждением и медлением в нечистых помыслах, пристрастием, сладострастием, нескромным воззрением на жен и юношей; во сне блудным ночным; осквернением, невоздержанием в супружеской жизни. 



Прости мя, честный отче. 



Согрешил: нетерпением болезней и скорбей, люблением удобств жизни сей, пленением ума и окаменением сердца, непонуждением себя на всякое доброе дело. 



Прости мя, честный отче. 



Согрешил: невниманием к внушениям совести своей, нерадением, леностию к чтению Слова Божия и нерадением к стяжанию Иисусовой молитвы. Согрешил любостяжанием, сребролюбием, неправедным приобретением, хищением, воровством, скупостью, привязанностию к разного рода вещам и людям. 



Прости мя, честный отче. 



Согрешил: осуждением и ослушанием отцов духовных, ропотом и обидой на них и неисповеданием пред ними грехов своих по забвению, нерадению и по ложному стыду. 



Прости мя, честный отче. 



Согрешил: немилосердием, презрением и осуждением нищих; хождением в храм Божий без страха и благоговения, уклоняясь в ересь и сектантское учение. 



Прости мя, честный отче. 



Согрешил: леностию, расслаблением негою, люблением телеснаго покоя, многоспанием, сладострастными мечтаниями, пристрастными воззрениями, бесстыдными телодвижениями, прикосновениями, блудом, прелюбодеянием, растлением, рукоблудием, невенчанными браками, тяжко согрешили те, кто делали аборты себе или другим или склоняли кого-нибудь к этому великому греху "--- детоубийству. Проводил время в пустых и праздных занятиях, в пустых разговорах, шутках, смехе и других постыдных грехах. 



Прости мя, честный отче. 



Согрешил: унынием, малодушием, нетерпением, ропотом, отчаянием в спасении, неимением надежды на милосердие Божие, бесчувствием, невежеством, наглостию, бесстыдством. 



Прости мя, честный отче. 



Согрешил: клеветою на ближнего, гневом, оскорблением, раздражением и осмеянием, непримирением, враждой и ненавистию, прекословием, подсматриванием чужих грехов и подслушиванием чужих разговоров. 



Прости мя, честный отче. 



Согрешил: холодностию и бесчувственностию на исповеди, умалением грехов, обвинением ближних, а не себя осуждением. 



Прости мя, честный отче. 



Согрешил: против Животворящих и Св. Тайн Христовых, приступая к ним без должного приготовления, без сокрушения и страха Божия. 



Прости мя, честный отче. 



Согрешил: словом, помышлением и всеми моими чувствами: зрением, слухом, обонянием, вкусом, осязанием "--- волею или неволею, ведением или неведением, в разуме и неразумии, и не перечислить всех грехов моих по множеству их. Но во всех сих, так и в неизреченных по забвению, раскаиваюсь и жалею, и впредь с помощию Божиею обещаюсь блюстись. 



Ты же, честный отче, прости мя и разреши от всех сих и помолись о мне грешном, а в оный судный день засвидетельствуй пред Богом об исповеданных мною грехах. Аминь. 

/$>$\bfseries *\normalfont{} Данный текст читается священником перед исповедью и приводится здесь, как один из образцов исповеди, в целях лучшей подготовки верующего к этому таинству, а так же лучшего осознания своей греховности и возбуждения покаянных чувств при домашней молитве. 

 

 

 

 



\mychapterending

\mychapter{Молитва после исповедания грехов}
%http://www.molitvoslov.com/text170.htm 
 


Как слабому и вовсе безсильному самому по себе на дела благия, смиренно со слезами молю Тебя, Господи, Спасителю мой, помози мне утвердиться в моем намерении: жить прочее время жизни для Тебя, возлюбленнаго Бога моего, богоугодно, а прошедшия согрешения моя прости милосердием Своим и разреши от всех моих, сказанных пред Тобою, грехов, яко благий Человеколюбец. Также смиренно молю Тебя, Пресвятая Богородице, и вас, небесные силы и все угодники Божии, помогите мне исправить мою жизнь.\itshape  

\normalfont{} 


\mychapterending

\mychapter{Молитва о прощении забытых грехов}
%http://www.molitvoslov.com/text169.htm 
 


\itshape (Преподобного Варсонофия Великого) 

\normalfont{}

Владыко Господи, поскольку и забыть свои прегрешения есть грех, то я во всем согрешил Тебе, Единому Сердцеведу; Ты и прости мне все по Твоему человеколюбию; тем-то и проявляется великолепие славы Твоей, когда Ты не воздаешь грешникам по делам их, ибо Ты препрославлен во веки. Аминь. 


\mychapterending