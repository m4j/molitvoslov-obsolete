

\mypart{О МОЛИТВЕ}\label{_o-molitve}
%http://www.molitvoslov.com/o-molitve

 

\mychapter{Как должно молиться в церкви}
%http://www.molitvoslov.com/text149.htm 
 





\myemph{ Православные христиане приняли от Святых Отец и исполняют во всем мире следующие обычаи: }



1. Войдя в храм и осеняя себя крестным знамением, творят три малых поклона, произнося:

\begin{quote}\bfseries 

«Создавый мя, Господи, помилуй». 

«Боже, милостив буди мне грешному». 

«Без числа согреших, Господи, прости мя».

}
\end{quote}


2. Затем, поклонившись направо и налево, стоят на месте и слушают псалмы и молитвы, читаемые в церкви, но не говорят про себя иных, собственных молитв, и не читают их по книжкам отдельно от церковного пения, ибо таких осуждает св. апостол Павел, как удаляющихся от церковного собрания (Евр. 10. 25). 



3. Поклоны малые и великие должно творить не по своему произволению, а по установлению св. апостолов и св. отец. Именно: при чтении Трисвятого («Святый Боже»), «Приидите, поклонимся» и троекратного «аллилуиа» трижды осенить себя крестным знамением, совершая малые поклоны; так же и при чтении «Сподоби, Господи», а равно и в начале великого славословия («Слава в вышних Богу») и после слов священника: «Слава Тебе, Христе Боже, упование наше». После каждого возгласа священника, а также при чтении чтецом «Честнейшую Херувим» осенять себя крестным знамением и творить малый поклон. 



Во дни будничные творить земные поклоны на литургии:

%\renewcommand{\theenumi}{\Asbuk{enumi}}

\begin{enumerate}

\item[а)] при начале пения «Достойно и праведно»; 

\item[б)] когда оканчивается молитва «Тебе поем»; 

\item[в)] в конце молитвы «Достойно есть» или Задостойника;

\item[г)] в начале молитвы «Отче наш»;

\item[д)] при изнесении св. Даров для причастия;

\item[е)] и при словах «Всегда, ныне и присно». 


\end{enumerate}


Hа утрени или всенощной, когда возглашается: «Богородицу и Матерь Света в песнех возвеличим». 



Во дни воскресные, а также от дня св. Пасхи до вечера дня св. Троицы, а равно от дня Рождества Христова по день Крещения, также в день Преображения и Воздвижения святые апостолы \myemph{ воспретили} вовсе преклонять колена и творить \myemph{ земные} поклоны, как о том свидетельствует св. Василий Великий в послании к блаженному Амфилохию. То же самое утвердили и Вселенские соборы I и VI; ибо воскресные и прочие Господские праздники содержат воспоминание о нашем примирении с Богом, по слову Апостола: «Уже неси раб, но сын» (Гал. 4, 7); cынам же не подобает рабское поклонение творити. 



4. Православным христианам не свойственно стоять на коленях, поднявши голову, но при словах священника: «Паки и паки, преклонше колена» и проч. повергаться ниц на землю; обычай же становиться на колени по собственному произволению, складывать руки и бить себя в грудь воспринят от западных еретиков, а в Православной Церкви недопускается. Православные христиане, согласно Уставу церковному, в \myemph{ положенное} время творят земные поклоны, повергаясь ниц и снова становясь на ноги. 



5. Когда в церкви осеняют народ крестом или Евангелием, образом или Чашей, то все крестятся, преклоняя главу, а когда осеняют свечами или благословляют рукой, или кадят к предстоящим, то православным христианам не должно креститься, а только наклонить голову; лишь в Светлую седмицу Пасхи, когда кадит священник с Kрестом в руке, то все крестятся и говорят: «Воистину воскресе». Так должно различать поклонение пред святыней и пред людьми, хотя и в священном сане. 



6. Принимая благословение священника или епископа, христиане целуют его десницу, но не крестятся перед этим. Не должно целовать у духовных лиц левую руку, ибо сие свойственно только иудеям, но \myemph{ правую}, через которую передается благословение. 



7. Крестное же знамение, по учению cвятых oтец, должно совершать так: сложив троеперстно правую руку, возлагать ее на лоб, на чрево, на правое плечо и на левое, и потом уже, положив на себя крест, наклоняться;  o тех же, которые знаменуют себя всей пятерней или кланяются, не окончив еще креста, или машут рукой по воздуху или по груди своей, сказано в Златоусте: «Тому неистовому маханию беси радуются». Напротив, крестное знамение, совершаемое истово с верою и благоговением, устрашает бесов, утишает греховные страсти и привлекает Божественную благодать. 




\mychapterending

\mychapter{Правила о поклонах и крестном знамении}
%http://www.molitvoslov.com/text148.htm 
 


\subsection*{Креститься \myemph{ без поклонов:}

\begin{enumerate}

\item В середине шестопсалмия на «аллилуиа» три раза. 

\item В начале «Верую». 

\item На отпусте «Христос, истинный Бог наш». 

\item В начале чтения Священного Писания: Евангелия, Апостола и паремий. 

\end{enumerate}

\subsection*{Креститься \myemph{ с поясным поклоном:}

\begin{enumerate}

\item При входе в храм и при выходе из него "--- три раза. 

\item При каждом прошении ектении после пения «Господи, помилуй», «Подай, Господи», «Тебе, Господи». 

\item При возгласе священнослужителя, воздающего славу Святой Троице. 

\item При возгласах «Приимите, ядите», «Пийте от нея вси», «Твоя от Твоих». 

\item При словах «Честнейшую Херувим». 

\item При каждом словe «поклонимся», «поклонение», «припадем». 

\item Во время слов «Аллилуиа», «Святый Боже» и «Приидите, поклонимся» и при возгласе «Слава Тебе, Христе Боже», перед отпустом "--- по три раза. 

\item На каноне на 1-й и 9-й песни при первом взывании к Господу, Божией Матери или святым.

\item После каждой стихиры (причем, крестится тот клирос, который оканчивает петь).

\item На литии после каждого из первых трех прошений ектении "--- по 3 поклона, после двух остальных "--- по одному.

\end{enumerate}

\subsection*{Креститься \myemph{ с земным поклоном:}

\begin{enumerate}
\item В пост при входе в храм и при выходе из него "--- 3 раза. 

\item В пост после каждого припева к песне Богородицы «Тя величаем». 

\item В начале пения «Достойно и праведно есть». 

\item После «Тебе поем». 

\item После «Достойно есть» или Задостойника. 

\item При возгласе: «И сподоби нас, Владыко». 

\item При выносе Святых Даров, при словах «Со страхом Божиим и верою приступите», и второй раз "--- при словах «Всегда, ныне и присно». 

\item В Великий пост, на Великом повечерии ,при пении «Пресвятая Владычице» "--- на каждом стихе; при пении «Богородице Дево, радуйся» и проч. на великопостной вечерне совершаются три поклона. 

\item В пост, при молитве «Господи и Владыко живота моего». 

\itemВ пост при заключительном пении: «Помяни мя, Господи, егда приидеши во Царствии Твоем». Всего 3 земных поклона.

\end{enumerate}




\subsection*{Поясной поклон \myemph{ без крестного знамения}

\begin{enumerate}

\item При словах священника «Мир всем»

\item «Благословение Господне на вас»,

\item «Благодать Господа нашего Иисуса Христа», 

\item «И да будут милости Великаго Бога» и

\item При словах диакона «И во веки веков» (после возгласа священника «Яко свят еси, Боже наш» перед пением Трисвятого). 


\end{enumerate}





\subsection*{Креститься не положено}


\begin{enumerate}


\item Во время псалмов.

\item Вообще во время пения.

\item Во время ектений тому клиросу, который поет ектенийные припевы

\item Креститься и класть поклоны нужно по окончании пения, а никак не про последних словах.


\end{enumerate}






\subsection*{Не допускается земных поклонов:}






Во дни воскресные, в дни от Рождества Христова до Крещения, от Пасхи до Пятидесятницы, в праздник Преображения и Воздвижения (в сей день три земных поклона Кресту). Поклоны прекращаются от вечернего входа под праздник до «Сподоби, Господи» на вечерне в самый день праздника.
\mychapterending

\mychapter{Поучение святителя Игнатия Брянчанинова о молитвенном правиле}
%http://www.molitvoslov.com/text147.htm 
 


Войдя в комнату твою, и, затворив дверь твою, помолись Отцу твоему, Который втайне; и Отец твой, видящий тайное, воздаст тебе явно… (Мф. 6, 6).

Господь, заповедавший уединенную молитву, очень часто Сам, во время Своего земного странствования, как повествует Евангелие, пребывал в ней. Он не имел где главу подклонить: и потому часто заменяли для него безмолвную, спокойную келлию безмолвные вершины гор и тенистые виноградники.

Темнота ночи закрывает предметы от любопытных взоров, тишина безмолвия не развлекает слуха. В безмолвии и ночью можно молиться внимательнее. Господь избирал для молитвы Своей преимущественное уединение и ночь, избирал их с тем, чтобы мы не только повиновались Его заповеди о молитве, но и следовали Его примеру. Для Самого Господа нужна ли была молитва? Пребывая, как человек, с нами на Земле, Он, как Бог, неразлучно был с Отцом и Духом, имел с Ними единую Божественную волю и Божественную власть.

«Войдя в комнату твою, и, затворив дверь твою, помолись Отцу твоему, Который втайне». Пусть о молитве твоей не знает никто: ни друг твой, ни родственник, ни само тщеславие, сожительствующее сердцу твоему и подстрекающее высказать кому-нибудь о молитвенном подвиге твоем, намекнуть о нем.

Затвори двери келлии твоей от людей приходящих для пустословия, для похищения у тебя молитвы; затвори двери ума от посторонних помышлений, которые предстанут, чтобы отвлечь тебя от молитвы; затвори двери сердца от ощущений греховных, которые будут покушаться смутить и осквернить тебя, и помолись.

Не дерзни приносить Богу многоглагольных и красноречивых молитв, тобой сочиненных, как бы они не казались тебе сильны и трогательны: они "--- произведение падшего разума и, будучи жертвой оскверненной, не могут быть приняты на духовный жертвенник Божий. А ты, любуясь изящными выражениями сочиненных тобою молитв и признавая утонченное действие тщеславия и сладострастия за утешение совести, и даже благодати, увлечешься далеко от молитвы; увлечешься далеко от молитвы в то самое время, когда тебе будет представляться, что ты молишься обильно и уже достиг некоторой степени богоугождения.

Душа, начинающая путь Божий, погружена в глубокое неведение всего Божественного и духовного, хотя бы она была и богата мудростию этого мира. По причине неведения она не знает, как и сколько надо ей молиться. Для вспомоществования младенчествующей душе Святая Церковь установила молитвенные правила.

Молитвенное правило есть собрание нескольких молитв, сочиненных Боговдохновенными святыми отцами, приспособленное к известному обстоятельству и времени.

Цель правила "--- доставить душе недостающее ей количество молитвенных мыслей и чувств, притом мыслей и чувств правильных, святых, истинно богоугодных. Такими мыслями и чувствами наполнены благодатные молитвы святых отцов.

Для молитвенного упражнения утром имеется особое собрание молитв, называемое утренними молитвами, или утренним правилом; для ночного моления пред отшествием ко сну "--- другое собрание молитв, именуемое молитвами на сон грядущим, или вечерним правилом. Особенное собрание молитв прочитывается готовящимися ко причащению Святых Христовых Таин и называется правилом ко Святому Причащению. Посвятившие большую часть своего времени благочестивым упражнениям (монахи) прочитывают около третьего часа пополудни особенное собрание молитв, называемое ежедневным, или иноческим правилом. Иные прочитывают ежедневно по нескольку кафизм, по нескольку глав из Нового Завета, полагают несколько поклонов "--- все это называется правилом.

Правило! Какое точное название, заимствованное из самого действия, производимого на человека молитвами, называемым правилом! Молитвенное правило направляет правильно и свято душу, научает ее поклоняться Богу Духом и Истиною (Ин. 4, 23) , между тем как душа, будучи предоставлена самой себе, не могла бы идти правильно путем молитвы. По причине своего повреждения и омрачения грехом,она совращалась бы непрестанно в стороны, нередко в пропасти, то в рассеянность, то в мечтательность, то в различные пустые и обманчивые призраки высоких молитвенных состояний, сочиняемых ее тщеславием и сластолюбием.

Молитвенные правила удерживают молящегося в спасительном расположении смирения и покаяния, научая его непрестанному самоосуждению, питая его умилением, укрепляя надеждой на Всеблагого и Всемилосердого Бога, увеселяя миром Христовым, любовию к Богу и ближним.

Как возвышенны и глубоки молитвы ко Святому Причащению! Какое превосходное приготовление они доставляют приступающему к Святым Христовым Тайнам! Они убирают и украшают дом души чудными помышлениями и ощущениями, столь благоугодными Господу. Величественно изображено и объяснено в этих молитвах величайшее из Таинств христианских; в противоположность этой высоте живо и верно исчислены недостатки человека, показаны его немощь и недостоинство. Из них сияет, как солнце с неба, непостижимая благость Бога, по причине которой Он благоволит тесно соединяться с человеком, несмотря на ничтожность человека.

Утренние молитвы так и дышат бодростью, свежестью утра: увидевший свет чувственного солнца и свет земного дня научается желать зрения высшего, духовного Света и Дня бесконечного, производимых Солнцем Правды "--- Христом.

Краткое успокоение сном во время ночи "--- образ положительного сна во мраке могилы. И напоминают нам молитвы на сон грядущим о нашем переселении в вечность, обозревают всю нашу деятельность в течение дня, научают приносить Богу исповедание соделанных согрешений и покаяние в них.

Молитвенное чтение акафиста Сладчайшему Иисусу, кроме собственного своего достоинства, служит превосходным приготовлением к упражнению Молитвой Иисусовой, которая читается так: «Господи Иисусе Христе, Сыне Божий, помилуй мя, грешнаго». Эта молитва составляет почти единственно упражнение преуспевших подвижников, достигших (христианской) простоты и чистоты, для которых всякое многомышление и многословие служат обременительным развлечением. Акафист показывает, какими мыслями может быть сопровождаема Молитва Иисусова, представляющаяся для новоначальных крайне сухой. Он (акафист) изображает только прошение грешника о помиловании Господом Иисусом Христом, но этому прошению даны разноообразные формы, сообразно младенчественности ума новоначальных. Так младенцам дают пищу, предварительно размягченную.

В акафисте Божией Матери воспето воплощение Бога Слова и величие Божией Матери, Которую за рождение Ею вочеловечившегося Бога «ублажают все роды» (Лк. 1, 48). Как бы на большой картине бесчисленными дивными чертами, красками, оттенками изображено в акафисте великое Таинство вочеловечения Бога Слова. Удачным освещением оживляется всякая картина "--- и необыкновенным светом благодати озарен акафист Божией Матери. Свет этот действует сугубо: им просвещается ум, он него сердце исполняется радости и извещения. Непостижимое приемлется как вполне постигнутое, по чудному действию, производимому (словами акафиста) на ум и сердце.

Многие благоговейные христиане, особенно иноки, совершают очень продолжительное вечернее правило, пользуясь тишиной и мраком ночи. К молитвам на сон грядущим они присоединяют чтение кафизм, чтение Евангелия, Апостола, чтение акафистов и поклоны с Молитвой Иисусовой… рабы Христовы плачут в тишине своих келлий, изливая усердные молитвы пред Господом… В веселии и бодрости духа, в сознании и ощущении необыкновенной способности к богомыслию и ко всем благим делам встречают рабы Божии тот день, которому предшествующую ночь они провели в молитвенном подвиге.

Господь повергался на колени во время молитвы Своей "--- и ты не должен пренебрегать коленопреклонениями, если имеешь достаточно сил для совершения их. Поклонением до лица земли, по объяснению отцов, изображается наше падение, а восстанием с земли "--- наше искупление (Слова св. Феолипта. Добротолюбие, ч.2). Пред началом вечернего правила особенно полезно положить посильное число поклонов, чтобы приготовиться к усердному и внимательному чтению правила.

При совершении правила и поклонов никак нельзя спешить; надо совершать и правила, и поклоны с возможной неспешностью и вниманием. Лучше меньше прочитать молитв и меньше положить поклонов, но со вниманием, чем много и без внимания.

Избери себе правило, соответствующее силам. Сказанное Господом о субботе, что она для человека, а не человек для нее (Мк. 2, 27), можно и нужно отнести ко всем благочестивым подвигам, а также и к молитвенному правило. Молитвенное правило "--- для человека, а не человек "--- для правила: оно должно способствовать к достижению человеком духовного преуспеяния, а не служить бременем неудобоносимым (тягостной обязанностью), сокрушающим телесные силы и смущающим душу. Тем более оно не должно служить поводом к гордостному и пагубному самомнению, к пагубному осуждению и унижению ближних.

Благоразумно избранное молитвенное правило, соответственно силам и роду жизни, является большим подспорьем для подвизающегося о своем спасении. Совершение его в положенные часы обращается в навык (от постоянства), в необходимую естественную потребность. Стяжавший этот блаженный навык, как только приближается к обычному месту совершения правила, так душа его уже исполняется молитвенным настроением: он не успел еще произнести ни одного слова из читаемых им молитв, а уже сердце наполняется умилением, и весь ум углубляется во внутренюю клеть (сердце).

«Предпочитаю,"--- сказал великий отец Матой,"--- непродолжительное правило, но постоянно исполняемое, продолжительному, но в скором времени оставляемому». А такую участь всегда имеют молитвенные правила, несоразмерные силам: при первом порыве горячности подвижник выполняет их, некоторое время, конечно, обращая больше внимания на количество, чем на качество, потом изнеможение, производимое подвигом, превосходящим силы, постепенно принуждает его сокращать и сокращать правило.

Часто подвижники, безрассудно установившие для себя обременительное правило, переходят от многотрудного правила к оставлению всякого правила. По оставлении правила, и даже при одном сокращении его, непремено нападает на подвижника смущение. От смущения он начинает чувствовать душевное расстройство. От расстройства рождается уныние. Усилившись, оно производит расслабление и исступление, а от действия их безрассудный подвижник предается праздной, рассеянной жизни, с равнодушием впадает в самые тяжкие согрешения.

Избрав для себя соразмерное силам и душевной потребности молитвенное правило, старайся тщательно и постоянно исполнять его: это нужно для поддержания нравственных сил души твоей, как нужно для поддержания телесных сил ежедневное, в известые часы, достаточное употребление здоровой пищи.

«Не за оставление псалмов осудит нас Бог в день Суда Своего,"--- говорит святой Исаак Сирин,"--- не за оставление молитвы, но за последующий оставлению их вход в нас бесов. Бесы, когда найдут место, войдут и затворят двери очей наших, тогда исполняют нами, их орудиями, насильственно и нечисто, с лютейшим отмщением, все воспрещенное Богом. И по причине оставления малого (правила), за которое (мы) сподобляемся заступления Христова, мы делаемся подвластными (бесам), как написано некоторым премудрым отцом: ,,Непокоряющий воли своей Богу, подчинится сопернику своему``. Эти (правила), кажущиеся тебе малыми, соделаются для тебя стенами против старающихся пленить нас. Совершение этих (правил) внутри келлии премудро установлено учредителями Церковного Устава, по откровению свыше, для хранения жития нашего» (Исаак Сирин, Слово 71).

Великие отцы, пребывавшие от обильного действия благодати Божией в непрестанной молитве, не оставляли и правил своих, которые навыкли они совершать в известные часы нощеденствия (ночных и дневных молитв). Многие доказательства этого видим в их житиях: преподобный Антоний Великий, совершая правило девятого часа "--- церковный девятый час соответствует третьему часу пополудни,"--- сподобился Божественного откровения; когда Преподобный Сергий Радонежский занимался молитвенным чтением акафиста Божией Матери, явилась ему Пресвятая Дева в сопровождении апостолов Петра и Иоанна.

Возлюбленные! Покорим свою свободу правилу: оно, лишив нас свободы пагубной, свяжет нас только для того, чтоб доставить нам свободу духовную, свободу во Христе. Цепи сначала покажутся тягостными, потом сделаются драгоценными для связанного ими. Все святые Божии приняли на себя и несли благое иго молитвенного правила; подражанием им и мы последуем в этом случае Господу нашему Иисусу Христу, Который вочеловечившись и указав нам Собой образ поведения, действовал так, как действовал Отец Его (Ин. 5, 19), говорил то, что заповедал Ему Отец (Ин. 12, 49), имел целью исполнить во всем волю Отца (Ин. 5, 30). Воля Отца и Сына и Святаго Духа "--- одна. По отношению к людям она заключается в спасении людей.

Всесвятая Троице, Боже наш! Слава Тебе! Аминь.

\bigskip

(Епископ Игнатий Брянчанинов. Сочинения. Аскетические опыты.

СПб., 1865 т.2, с. 181--191. Публикуется в сокращении.)

 


\mychapterending

\mychapter{Чин келейного чтения канонов и акафистов}
%http://www.molitvoslov.com/text894.htm 
 





 \myemph{Перед началом всякого правила и по окончании его кладутся следующие поклоны (земные или поясные), кои называются седмипоклонный начал.




}




Боже, милостив буди мне грешному. \myemph{ (Поклон) }

Боже, очисти мя грешнаго и помилуй мя. \myemph{ (Поклон)


}




Создавый мя, Господи, помилуй мя. \myemph{ (Поклон)


}




Без числа согреших, Господи,прости мя. \myemph{ (Поклон)


}




Владычица моя, Пресвятая Богородице, спаси мя грешнаго. \myemph{ (Поклон)


}




Ангеле, хранителю мой, от всякого зла сохрани мя. \myemph{ (Поклон)


}




Святой апостоле (или мучениче, или преподобный отче,  \myemph{ имя}) моли Бога о мне. \myemph{ (Поклон)




}







\bfseries 


Также}: \MolitvamiSviatyhOtecNashih Слава Тебе, Боже наш, слава Тебе. Царю Небесный: Святый Боже: \myemph{ (Трижды}), Слава, и ныне: Пресвятая Троице: Господи, помилуй \myemph{ (Трижды}), Слава, и ныне: Отче наш: Господи, помилуй \myemph{ (12 раз}), 





Слава и ныне: Приидите, поклонимся \myemph{: (Трижды}), 

псалом 50-й, Помилуй мя, Боже: Верую: \myemph{ и чти каноны c акафисты.






}


\bfseries 


Посем}, Достойно есть \myemph{: (Поклон}). 

Трисвятое и по Отче наш: тропари: Помилуй нас, Господи, помилуй нас: и молитвы на сон грядущим. 






 \myemph{


Готовящимся ко св. Причащению нужно обязательно вечером прочитать три канона: Спасителю, Божией Матери и Ангелу-хранителю, и акафист Спасителю и Божией Матери. Желающие же ежедневно выполнять это вечернее правило и получают от этого великую духовную пользу.

 

}


\mychapterending

\mychapter{О молитве Иисусовой}
%http://www.molitvoslov.com/text146.htm 
 
\myemph{ Иисусова молитва творится с благословения и под контролем духовника.}






У апостола Павла в первом послании к Солунянам (5, 16) сказано: «непрестанно молитесь». Как же это непрестанно молиться? "--- Часто творить молитву Иисусову: «Господи Иисусе Христе, Сыне Божий, помилуй мя». Если кто навыкнет этому призыванию, тот будет ощущать великое утешение и потребность всегда творить эту молитву и она как бы сама собой в нем будет твориться. 

Хотя вначале враг рода человеческого будет мешать в этом, наводить большую тяготу, лень, скуку, одолевающий сон, но, преодолевши все это, с помощью Божией, получишь покой душе твоей, духовную радость, расположение к людям, умиротворение помыслов, благодарение Богу. 

В самом имени Иисуса Христа заключается великая благодатная сила. 

Многие святые и праведные люди советуют как можно чаще, почти непрерывно, творить молитву Иисусову. 

Св. Иоанн Златоуст говорит: «Должно всякому, если он, пьет ли, сидит ли, служит ли, путешествует ли, или другое что делает, непрестанно вопить: ,,Господи, Иисусе Христе, Сыне Божий, помилуй мя``, да имя Господа Иисуса Христа, сходя в глубину сердечную, смирит змия пагубного, душу же спасет и оживотворит». 

Преподобный Серафим Саровский: «,,Господи Иисусе Христе, Сыне Божий, помилуй мя грешного``:  в этом да будет все твое внимание и обучение. Ходя, сидя, делая и в церкви до богослужения стоя, входя и выходя, сие непрестанно держи в устах и в сердце твоем. С призыванием имени Божия таким образом ты найдешь покой, достигнешь чистоты духовной и телесной и вселится в тебя Святый Дух, Источник всех благ, и управит он тебя в святыне, во всяком благочестии и чистоте». 

Епископ Феофан Затворник: «Чтобы удобнее навыкнуть памятованию о Боге, для этого для христиан ревностных есть особый прием, именно "--- непрестанно повторять коротенькую "--- слова в два-три "--- молитовку. Большей частью это есть: ,,Господи, помилуй! "--- Господи, Иисусе Христе, помилуй мя грешнаго (или грешную)``. Если вы этого еще не сделали, так вот слышьте, и если так не делали, так начинайте делать с этих пор». 

«Истинно решивший служить Господу Богу должен упражняться в памяти Божией и непрестанной молитве Иисусу Христу, говоря умом: ,,Господи, Иисусе Христе, Сыне Божий, помилуй мя грешнаго``. 

Таковым упражнением, при охранении себя от рассеяния и при соблюдении мира совести, можно приблизиться к Богу и соединиться с Ним. Ибо, по словам св. Исаака Сирина, без непрестанной молитвы мы приблизиться к Богу не можем» (Преп. Серафим Саровский). 

О. Иоанн Кронштадский также часто советовал творить молитву Иисусову.


\longpage[2]{}\mychapterending[1]

\mychapter{О даровании молитвы}
%http://www.molitvoslov.com/text145.htm 
 





Научи мя, Господи, усердно молиться Тебе со вниманием и любовию, без которых молитва не бывает услышана! Да не будет у меня небрежной молитвы во грех мне!

 


\mychapterending

\mychapter{Молитва  человека,  страдающего  рассеянием, невниманием, нерадением  в молитве}
%http://www.molitvoslov.com/text144.htm 
 





Рассеянный ум мой собери, Господи, и оледеневшее сердце очисти, яко Петру даяй мне покаяние, яко мытарю "--- воздыхание, и яко блуднице "--- слезы, да велиим гласом зову Ти, Боже, спаси мя, яко Един Благоутробен и Человеколюбец. 




\mychapterending

\mychapter{Молитва Оптинских старцев о даровании молитвы Иисусовой}
%http://www.molitvoslov.com/text137.htm 
 





Господи Иисусе Христе, Сыне Божий! Имени Твоему покланяются ангели и человецы, Твоего имени трепещут адские силы, Твое имя верное оружие на прогнание супостата, Твое имя попаляет грехи и страсти, Твое имя подает силу в подвигах, собирает воедино рассеянный ум и, во исполнении заповедей Твоих, обогащает добродетелями, Твое имя творит чудеса и соединяет нас с Тобою, дарует мир и радость о Духе Святом, а в жизни будущей "--- Царство Небесное. Сего ради я, недостойный раб Твой, молюся Тебе: прожени от нас неведение духовное, просвети познанием Божественной истины и научи нас незаблудно, во смирении, внимательно, с чувством покаяннаго сокрушения, устами, умом и сердцем творить непрестанно молитву сию: \bfseries «Господи Иисусе Христе, Сыне Божий, помилуй мя, грешнаго»}. Ты бо рекл еси, Господи, пречистыми устами Твоими: «Аще что просите во имя Мое, Аз сотворю». Се, молитвами Пречистыя Матери Твоея, святителя Иоасафа Белградскаго, святителя Николая Мирликийскаго, преподобнаго Серафима Саровскаго и всех преподобных отец наших о даровании прошу молитвы Иисусовой,  молитвы  Пресвятаго и Всемогущаго Имени Твоего. Услыши мя, обещавый услышать всех призывающих Тя во истине. Твое бо есть еже миловати и спасати, и даровати просимое молящемуся во славу Твою со Отцем и Святым Духом. Аминь. 






\myemph{ 


(Если молитва прочитана невнимательно, причитать еще раз.)

}


\mychapterending

\mychapter{Правило преподобного Серафима Саровского для мирян}
%http://www.molitvoslov.com/text132.htm 
 





Молитву преподобный Серафим Саровский считал для жизни столь же необходимой, как  воздух. Он просил и требовал от своих духовных детей, чтобы они непрестанно молились, и заповедал им молитвенное правило, оставшееся под именем «Правила отца Серафима». 

Пробудившись от сна и ставши на избранном месте, всякий должен оградить себя крестным знамением и, став на избранном месте, читать ту спасительную молитву, которую людям передал Сам Господь, то есть «Отче наш» (трижды), потом «Богородице Дево, радуйся» (трижды) , и, наконец, единожды Символ веры. Совершив это утреннее правило, всякий христианин пусть отходит на свое дело и, занимаясь дома или находясь в пути, должен читать тихо про себя: «Господи Иисусе Христе, Сыне Божий, помилуй мя грешнаго». Если же окружают люди, то, занимаясь делом, говорить только умом: «Господи, помилуй», и так продолжать до самого обеда. Перед обедом совершить утреннее правило. 

После обеда, исполняя свое дело, всякий должен читать тихо: «Пресвятая Богородице, спаси мя грешнаго», что продолжать до самой ночи. 

Когда же случится проводить время в уединении, нужно читать: «Господи Иисусе Христе, Богородицею помилуй мя грешнаго», а ложась спать на ночь, всякий христианин должен повторить утреннее правило, и после него с крестным знамением засыпать. При этом св. старец говорил, указывая на опыт св. отeц, что, если христианин будет держаться этого малого правила, как спасительного якоря среди волн мирской суеты, со смирением исполняя его, то может достигнуть высокой меры духовной, ибо эти молитвы суть  основание христианства: первая "--- как слово Самого Господа и поставленная Им в образец всех молитв, вторая принесена с неба Архангелом как приветствие Пресвятой Деве, Матери Господа. Последняя же заключает все догматы веры. 

Имеющий время пусть читает Евангелие, Апостол, другие молитвы, акафисты, каноны. Если же кому-то невозможно выполнять и это правило "--- слуге, подневольному человеку "--- то мудрый старец советовал выполнять это правило и лежа, и при ходьбе, и при деле, помня слова Писания: «Всяк, иже призовет имя Господне, спасется». 




\mychapterending

\mychapter{О молитве. Св.Иоанн Златоуст}
%http://www.molitvoslov.com/text152.htm 
 





Молитва имеет два вида: первый "--- славословие со смиренномудрием, а второй, низший "--- прошение. Посему, молясь, не вдруг приступай к прошению… Начиная молитву, оставь себя самого, жену, детей, расстанься с землей, минуй небо, оставь всякую тварь видимую и невидимую, и начни славословием все Сотворившего; и когда будешь славословить, не блуждай умом туда и сюда, не баснословь по-язычески, но выбирай слова из Святых Писаний… Когда же кончишь славословие… тогда начни со смиренномудрием и говори: недостоин я, Господи, говорить пред Тобою, потому что я весьма грешен,"--- более всех грешников грешен я. Так молись со страхом и смиренномудрием. Когда же совершишь обе эти части славословия и смиренномудрия, тогда проси уже, чего ты должен просить, то есть не богатства, не славы земной, не здравия телесного, потому что Он Сам знает, что полезно каждому; но, как поверено тебе, проси Царствия Божия.

\bigskip
Свт. Иоанн Златоуст

 


\mychapterending

\mychapter{О Храмовой молитве}
%http://www.molitvoslov.com/text151.htm 
 


В общих церковных богослужениях не исключается, конечно, частная молитва, но главное внимание молящихся все же должно быть сосредоточено на совершающемся богослужении. Молитва в храме "--- молитва соборная; в ней участвуют и клирики, и миряне, т.~е. вся земная Церковь. Поэтому и сила молитвеннаго духа в такой молитве сильнее, чем в частной домашней молитве. В наших храмовых богослужениях каждый найдет моления о том, что ему именно нужно, а кроме того, приносятся еще моления о всех ближних, о властях предержащих, о всей Церкви, благодарения, славословия Господу за Его милости, как учит ап.Павел: «Итак прежде всего прошу совершать молитвы, прошения, моления, благодарения за всех человеков, за царей и за всех начальствующих, дабы проводит нам жизнь тихую и безмятежную во всяком благочестии и чистоте» (Тим.II, 1--2). Все это есть в нашем богослужении, требуется лишь внимание ума и сердца со стороны молящихся и толковое чтение и пение со стороны клира. Разумное и сердечное участие в храмовом богослужении "--- самая действенная, самая живая жизнь в Церкви "--- этом таинственном Теле Христовом (I Кор. 12,27). В самом деле: вот мы все "--- клирики и миряне "--- составляющие Церковь видимую, земную; вот "--- видимые образы невидимо присутствующей Церкви прославленной, невидимой, во главе с Царицей неба и земли "--- Пречистой Богоматерью; все сонмы ангелов и святых. А на св. Престоле в алтаре воистину "--- в Своей Плоти и Крови "--- восседает Сам Глава всей Церкви, Пастыреначальник Господь Иисус Христос.

Перед лицем такого высочайшаго собора "--- с каким благоговением и трепетом должны стоять мы! С каким вниманием и страхом должны клирики совершать свое служение! И как-то странно в этом таинственном и связанном собрании Церкви видимой и невидимой взывать ко Господу со своими личными нуждами. Только очень уже большая, невыносимая тогу может заставить сделать это. А то хочется забыть себя, свою суетную жизнь и безраздельно слиться в священном трепете с ликованием невидимых, но чувствуемых каждой верующей душой, сонмов святых и ангелов, «Всякое ныне житейское отложим попечение» "--- взывает к нам на литургии невидимый сонм херувимов, предстоящих огненному Престолу Господа славы, Господа силы!..

Да и нужно отложить все! Нужно все силы ума и сердца направить на то, чтобы войти в жизнь, в ликование Церкви. А для этого нужно ясно сознавать смысл и цель каждаго движения. Маловеры упрекают Православие в том, что оно все ушло в обрядность. Но если сердцем и умом уметь следить за ходом богослужения, связывая тексты с действиями, тот открывается такая дивная, несказуемая жизнь, которую можно только переживать, но о которой нельзя говорить; это частица того, о чем ап. Павел писал: «Но вы приступили к горе Сиону и ко граду Бога живаго, к небесному Иерусалиму и тьмам Ангелов. К торжествующему собору и Церкви первенцев, написанных на небесах, и к Судии всех Богу, и к духам праведников, достигших совершества, и к ходатаю новаго завета Иисуса» (Евр.XII, 22--24), и еще: не видел того глаз, не слышало того ухо, и не приходило то на сердце человеку, что приготовил Бог любящим Его» (1 Кор. 11, 9). Поэтому справедливо утверждение, что храм "--- рай на земле.

\bigskip

Сер.Л. "--- Лит. оч. "--- № 35, 1930

«Приходская жизнь», декабрь 1980

 


\mychapterending
