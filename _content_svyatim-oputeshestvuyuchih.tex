

\mypart{Молитвы некоторым святым о путешествующих}\label{_content_svyatim-oputeshestvuyuchih}
%http://www.molitvoslov.com/content/svyatim-oputeshestvuyuchih

 

\mychapter{Святителю Николаю}
%http://www.molitvoslov.com/text883.htm 
 
\myfig{765}

О, всесвятый Николае, угодниче преизрядный Господень, теплый наш заступниче, и везде в скорбех скорый помощниче! Помози мне грешному и унылому в настоящем сем житии, умоли Господа Бога даровати ми оставление всех моих грехов, елико согреших от юности моея, во всем житии моем, делом, словом, помышлением и всеми моими чувствы; и во исходе души моея помози ми окаянному, умоли Господа Бога, всея твари Содетеля, избавити мя воздушных мытарств и вечного мучения: да всегда прославляю Отца и Сына и Святаго Духа, и твое милостивное предстательство, ныне и присно и во веки веков. Аминь.


\mysubsubsection{Тропарь, глас 4}


Правило веры и образ кротости, воздержания учителя яви тя стаду твоему яже вещей истина; сего ради стяжал еси смирением высокая, нищетою богатая, отче священноначальниче Николае, моли Христа Бога спастися душам нашим.


\mysubsubsection{Кондак, глас 3}


В Мирех, святе, священнодействитель показался еси: Христово бо, преподобне, Евангелие исполнив, положил душу твою о людех твоих, и спасл еси неповинныя от смерти; сего ради освятился еси, яко великий таинник Божия благодати. 
\mychapterending

\mychapter{Преподобным Кириллу и Марии, родителям преподобного Сергия Радонежского}
%http://www.molitvoslov.com/text884.htm 
 


О раби Божии, схимонаше Кирилле и схимонахине Марие! Внемлите смиренному молению нашему. Аще бо временное житие ваше скончали есте, но духом от нас не отступаете, присно по заповедем Господним шествовати нас научающе и крест свой терпеливо носити нам пособствующе. Се бо вкупе со преподобным и богоносным отцем нашим Сергием, вашим возлюбленным сыном, дерзновение ко Христу Богу и ко Пречистой Его Матери стяжали есте. Темже и ныне будите молитвенницы и ходатаи о нас, недостойных рабах Божиих\itshape  (имена)\normalfont{}. Будите нам заступницы крепции, да верою живуще, заступлением вашим сохраняеми, невредими от бесов и от человек злых пребудем, славяще Святую Троицу, Отца и Сына и Святаго Духа, ныне и присно и во веки веков. Аминь. 
\mychapterending

\mychapter{Святому великомученику Прокопию}
%http://www.molitvoslov.com/text885.htm 
 



О святый страстотерпче Христов Прокопие! Услыши нас грешных, предстоящих ныне пред Святою иконою твоею и умильно молящих тя: помолимся о нас\itshape  (имена\normalfont{}) к Иисусу Христу Богу нашему и Его рождшей Матери, Владычице нашей Богородице, еже отпустити нам согрешения наша, яже содеяхом. Испроси у Господа к пользе душевней и телесней милость, мир, благослове ние, во еже избавитися нам всем в день судный страшный, шуия части спастися, стати же одесную со избранными Его к наследию Царствия Небеснаго, яко Тому подобает всякая слава, честь и поклонение со Безначальным Его Отцем и Пресвятым и Благим и Животворящим Его Духом, ныне и присно и во веки веков. Аминь.
\mychapterending

\mychapter{Святым сорока мученикам Севастийским}
%http://www.molitvoslov.com/text886.htm 
 


О страстотерпцы Христовы, во граде севастийстем мужественно пострадавшии, к вам, яко молитвенникам нашим, усердно прибегаем и просим: испросите у Всещедраго Бога прощение согрешений наших и жития нашего исправление, да в покаянии и нелицемерней любви друг ко другу поживше, со дерзновением предстанем страшному судищу Христову и вашим предстательством одесную Праведного Судии предстанем. Ей, угодницы Божии, будите нам, рабам Божиим\itshape  (имена)\normalfont{}, защитницы от всех враг видимых и невидимых, да под кровом святых ваших молитв избавимся от всех бед, зол и напастей до последнего дне жизни нашея, и тако прославим Великое и Достопокланяемое имя Вседетельныя Троицы, Отца и Сына и Святаго Духа, ныне и присно и во веки веков. Аминь.



\mychapterending