\chapter*{Список изменений}

Этот раздел содержит информацию об изменениях и исправлениях на сайте \url{www.molitvoslov.com}, вошедших в эту и предыдущие публикации. Информация о новых версиях располагается в начале.
 
%\begin{center}
%\begin{tabular}{lp{0.7\textwidth}l}
%\toprule
%Версия & Изменения\\
%\midrule
%20110606 &
%Обновление текстов с сайта от 06 мая 2011 г.
%\\
%20110503.02 & Содержимое сайта от 03 мая 2011 г. \\
%\bottomrule
%\end{tabular}
%\end{center}

\small

\section*{Версия 20110606}

Обновление текстов с сайта от 06 июня 2011 г.
\begin{itemize}

\item В разделе «НАПУТСТВИЕ ХРИСТИАНИНА ПЕРЕД СМЕРТЬЮ И ЗАУПОКОЙНЫЕ МОЛИТВЫ»
\begin{itemize}
\item Исправлен «Чин литии, совершаемой мирянином дома и на кладбище» "--- убрана «Разрешительная молитва, читаемая при смерти\ldots»
\item Исправлен тропарь прп. Паисию Великому в молитве «Об ослаблении вечных мук умерших без покаяния», а также в разделе «КАНОН ПРЕПОДОБНОМУ ПАИСИЮ ВЕЛИКОМУ ОБ ИЗБАВЛЕНИИ ОТ МУК УМЕРШИХ БЕЗ ПОКАЯНИЯ».
\item Исправлен «Канон молебный ко Господу Иисусу Христу и Пречистой Богородице Матери Господни при разлучении души от тела всякого правоверного».
\end{itemize}

\item В разделе «МОЛИТВЫ СВЯТЫМ»
\begin{itemize}
\item Убраны дубли молитв прп. Александру Свирскому и прп. Серафиму Саровскому.
\item Добавлены молитвы св. первомученику архидиакону Стефану и св. царице Грузинской Тамаре.
\end{itemize}

\item В разделе «МОЛИТВЫ В СКОРБЯХ И ИСКУШЕНИЯХ ТВОРИМЫЕ»
\begin{itemize}
\item Добавлена «Молитва преследуемого человеками (свт. Игнатия Брянчанинова)».
\end{itemize}

\item В разделе «МОЛИТВЫ ЗА БОЛЯЩИХ»
\begin{itemize}
\item Исправлена опечатка в «Каноне за болящего, глас 3-й», Песнь 9.
\end{itemize}


\end{itemize}


\section*{Версия 20110503.02}
Содержимое сайта от 03 мая 2011 г.

\normalfont
