

\mychapter{Cвятому равноапостольному Великому Князю Владимиру}
%http://www.molitvoslov.com/text283.htm 
 
\myfig{img/175.jpg}Внук святой княгини Ольги, Владимир, поначалу, был язычником, но, чувствуя внутренне пустоту и ложь идолопоклонства, пережил обращение. В 988 году он принял Святое Крещение от греков и крестил свой народ. 


Св. Владимир строил церкви и школы, заботился о нищих и сиротах, являл собой пример милосердия и гостеприимства. 


Православная Церковь при св. Владимире становится духовным руководителем и просветителем русского народа, тем самым, именно князем Владимиром Святым был заложен несокрушимый духовный фундамент тысячелетней российской истории.


\medskip
\bfseries Молитва первая\normalfont{}
О великий угодниче Божий, богоизбранный и богопрославленный, равноапостольный княже Владимире! Ты отринул еси зловерие и нечестие языческое, уверовал еси во Единаго Истиннаго Триипостаснаго Бога и, восприяв Святое Крещение, просветил еси светом Божественныя веры и благочестия всю страну Русскую. Славяще убо и благодаряще Премилосердаго Творца и Спасителя нашего, славим, благодарим и тя, просветителю наш и отче, яко тобою познахом спасительную веру Христову и крестихомся во Имя Пресвятыя и Пребожественныя Троицы: тою верою избавихомся от праведнаго осуждения Божия, вечнаго рабства диаволя и адова мучительства: тою верою восприяхом благодать всыновления Богу и надежду наследования Небеснаго блаженства. Ты еси первый вождь наш к Начальнику и Совершителю нашего вечнаго спасения Господу Иисусу Христу; ты еси теплый молитвенник и ходатай о стране Русской, о воинстве и о всех людех. Не может язык наш изобразити величие и высоту благодеяний, тобою излиянных на землю нашу, отцев и праотцев наших и на нас, недостойных. О всеблагий отче и просветителю наш! Призри на немощи наша и умоли премилосердаго Царя Небеснаго, да не прогневается на ны зело, яко по немощем нашим по вся дни согрешаем, да не погубит нас со беззаконьми нашими, но да помилует и спасет нас, по милости Своей, да всадит в сердце наше спасительный страх Свой, да просветит Своею благодатию ум наш, во еже разумети нам пути Господни, оставити стези нечестия и заблуждений, тщатися же во стезях спасения и истины, неуклоннаго исполнения заповедей Божиих и уставов Святыя Церкве. Моли, благосерде, Человеколюбца Господа, да пробавит нам великую милость Свою, да избавит нас от нашествия иноплеменных, от внутренних нестроений, мятежей и раздоров, от глада, смертоносных болезней и от всякаго зла, да подаст нам благорастворение воздуха и плодоносие земли, да даст пастырем ревность о спасении пасомых, всем же людем споспешение о еже усердно службы своя исправляти, любовь между собою и единомыслие имети, на благо же Отечества и Святыя Церкве верне подвизатися, да возсияет свет спасительныя веры в стране нашей во всех концех ея, да упразднятся вся ереси и расколы, да тако поживше в мире на земли, сподобимся с тобою вечнаго блаженства, хваляще и превозносяще Бога во веки веков. Аминь.


\medskip
\bfseries Молитва вторая\normalfont{}
О великий угодниче Божий, равноапостольный княже Владимире! Призри на немощи наши и умоли Премилосердаго Царя Небеснаго, да не прогневается на ны зело и да не погубит нас со беззаконьми нашими, но да помилует и спасет нас по милости Своей, да всадит в сердца наша покаяние и спасительный страх Божий, да просветит Своею благодатию ум наш, во еже оставити нам стези нечестия и на путь спасения обратитися, неуклонно же заповеди Божия творити и уставы Святыя Церкве соблюдати. Моли, благосерде, Человеколюбца Бога, да явит нам великую милость Свою: да избавит нас от смертоносных болезней и от всякаго зла, да сохранит и спасет рабов Божиих (имена) от всех козней и наветов вражиих и да все мы сподобимся с тобою вечнаго блаженства, хваляще и превозносяще Бога во веки веков.