

\mypart{МОЛИТВЫ ОБ ИСЦЕЛЕНИИ ТЕЛЕСНЫХ НЕДУГОВ}
%http://www.molitvoslov.com/content/obiscelenii

 

\mychapter{О терпении болезней и несчастий}
%http://www.molitvoslov.com/content/o-terpenii-bolezney-i-neschastiy

 

\section{Праведному Иову Многострадальному}
%http://www.molitvoslov.com/text618.htm 
 


\bfseries Тропарь, глас 2-й:\normalfont{}


Память праведнаго Твоего Иова, Господи, празднующе, тем Тя молим: избави нас от наветов и сетей лукаваго диавола и спаси души наша, яко Человеколюбец.


\medskip


\bfseries Кондак, глас 8-й:\normalfont{}


Яко истинен и праведен, Богочестив и непорочен, освящен же явлься, всеславне, Божий угодниче истинный, просветил еси мир в терпении твоем, терпеливейший и доблейший. Темже вси, Богомудре, память твою воспеваем.


\medskip


\bfseries Стихира, глас 6-й:\normalfont{}


Дал еси нам образ терпения и мужества, Преблагий Господи, угодника Твоего Иова, во всех приключшихся ему злостраданиих ничимже согрешивша пред Тобою, ниже устнама своима, и не давша безумия Тебе, Богу нашему. Того молитвами и нас сотвори победители многоразличных искушений и спаси души наша, яко Человеколюбец.


\medskip


\bfseries Молитва:\normalfont{}


Боже святый и во святых почиваяй, трисвятым гласом на небеси от ангел воспеваемый, на земли от человек во святых Своих хвалимый, давый Святым Твоим Духом коемуждо благодать по мере дарования Христова, и тою поставивый Церкви Твоей Святей овы апостолы, овы пророки, овы же благовестники, овы пастыри и учители, ихже словом проповеди, Тебе Самому действующему вся во всех, мнози совершишася святии в коемждо роде и роде, различными благодетельми благоугодившии Тебе, и к Тебе, нам образ добрых подвигов своих оставивше, в радости прешедший, готови, в немже сами искушсни быша, и нам напаствуемым помогати. Сих святых всех и святого праведного Иова  воспоминая и их богоугодное похваляя житие, Тебе Самаго, в них действовавшаго, восхваляю, и онех благотворения Твоя дарования быти веруя, прилежно молю Тя, Святе святых, даждь ми грешному последовати их учению, житию, любви, вере, долготерпению, и их молитвенною помощию, паче же Твоею вседействующею благодатию, небесныя с ними сподобитися славы, хваляще Пресвятое имя Твое, Отца и Сына и Святаго Духа во веки. Аминь.


\mychapterending

\mychapter{От разных недугов и болезней}
%http://www.molitvoslov.com/content/ot-raznyh-nedugov-i-bolezney

 

\section{Священномученику Ермолаю, наставнику Пантелеимона, иерею Никомидийскому}
%http://www.molitvoslov.com/text661.htm 
 


\bfseries Тропарь, глас 1-й:\normalfont{}


Миром благодати помазан, преблаженне отче Ермолае, иерей Бога Вышняго явился еси и благочестия светлый проповедник, мученик же твердый и безмездный врач притекающих к тебе. Слава Давшему тебе крепость, слава Венчавшему тя, слава Действующему тобою всем исцеления.


\medskip


\bfseries Кондак, глас 4-й:\normalfont{}


Яко святитель благочестно пожив и священномучения венец приял еси. Идольския жертвы погасив, добрый пастырь Христова стада был еси, премудре, и Пантелеимону истинный учитель. Сего ради почитаем тя песньми, вопиюще: от бед избави нас присно молитвами твоими, Ермолае, отче наш.


\medskip


\bfseries Стихира, глас 6-й:\normalfont{}


Веселися, граде Никомидие, радуйся, соборе верных, светло торжествующе в память Ермолая священномученика, скорбящих сострадательнаго утешителя, обидимых теплаго заступника, болящих безмезднаго врача, молящагося о спасении душ наших.


\medskip


\bfseries Молитва:\normalfont{}


О, преславный священномучениче Ермолае и скорый помощниче христианом в болезнех! Верую от всея души и помышления, яко дадеся тебе от Господа дар болящия врачевати и расслабленныя укрепляти. Сего ради к тебе, яко благодатному врачу болезней, аз немощный прибегаю и, твой досточтимый образ с благоговением лобызая, молюся: твоим предстательством у Царя Небеснаго испроси мне болящему исцеление от удручающия мя болезни, аще бо и недостоин есмь тебе, благостнейшаго отца и приснаго заступника моего, но ты, быв подражатель человеколюбия Божия, сотвори мя достойна твоего заступления чрез мое обращение от злых дел к благому житию, уврачуй обильно дарованною тебе благодатию язвы и струпы души и тела моего, даруй ми здравие и спасение и во всем благое поспешение, да тако, тихое и безмолвное житие пожив во всяком благочестии и чистоте, сподоблюся со всеми святыми славити Всесвятое имя Отца и Сына и Святаго Духа. Аминь.


\section{Преподобному Сампсому странноприимцу}
%http://www.molitvoslov.com/text659.htm 
 


\bfseries Тропарь, глас 8-й:\normalfont{}


В терпении твоем стяжал еси мзду твою, отче преподобне, в молитвах непрестанно терпевый, нищия возлюбивый и сия удовливый, но молися Христу Богу, Сампсоне милостиве блаженне, спастися душам нашим.


\medskip


\bfseries Кондак, глас 8-й:\normalfont{}


Яко врача всеизрядна и молитвенника благоприятна, к раце твоей притекающе, Сампсоне богомудре преподобне, сошедшеся любовию, псалмы и песньми, радующеся, Христа прославляем, таковую тебе благодать подавающа исцелений.


\medskip


\bfseries Молитва:\normalfont{}


О, теплый молитвенниче, благостный отче, преподобне Сампсоне! Моли Бога о мне, грешнем, и низпосли от Всеблагаго Владыки помощь ми и избавление, привременна бо есть жизнь моя и исполнена труда, скорбей и болезней. Укрепи сердце мое, да возмогу понести тяготу мою, и не попусти, да превозмогут искушения многая моя малыя силы, но споспешествуй мне заступлением твоим и посреде обстояний и бед управи путь мой в Царство Небесное, да славлю Прославльшагося в тебе Господа во веки. Аминь.


\section{Молитва Архистратигу Михаилу}
%http://www.molitvoslov.com/text664.htm 
 


\bfseries Тропарь, глас 4-й:\normalfont{}


Небесных воинств Архистратиже, молим тя присно мы недостойнии, да твоими молитвами оградиши нас кровом крил невещественныя твоея славы; сохраняюще нас, припадающих прилежно и вопиющих: от бед избави нас, яко чиноначальник вышних сил.


\medskip


\bfseries Кондак, глас 2-й:\normalfont{}


Архистратизи Божии, служителие Божественным славы, Ангелов начальниче, и человеков наставницы полезное нам проси и велию милость, яко безплотных Архистратизи.


\section{Молитва Архангелу Рафаилу}
%http://www.molitvoslov.com/text663.htm 
 


О, святый великий Архангеле Рафаиле, престолу Божию предстояй! Ты благодатию, от Всемогущаго Врача душ и телес наших, тебе данною, праведнаго мужа Товита от слепоты телесныя исцелил еси, и сына его Товию, спутешествуя ему, от злаго духа сохранил еси. Всеусердно молю тя: буди мне путеводитель в жизни моей. Cохрани от всех видимых и невидимых враг, исцели душевныя и телесныя болезни моя, управи жизнь мою к покаянию во гресех и ко творению добрых дел. О, святый великий Рафаиле Архангеле! Услыши мене грешнаго, молящагося тебе, и сподоби в здешней и будущей жизни благодарити и славити общаго Создателя нашего в безконечныя веки веков. Аминь. 


\section{Мученику Трифону}
%http://www.molitvoslov.com/text656.htm 
 


\bfseries Тропарь, глас 8-й:\normalfont{}


Мученик Твой, Господи, Трифон во страдании своем венец прият нетленный от Тебе, Бога нашего: имеяй бо крепость Твою, мучителей низложи, сокруши и демонов немощныя дерзости.  Того молитвами спаси души наша.


\medskip


\bfseries Кондак, глас 8-й:\normalfont{}


Троическою твердостию многобожие разрушил еси от конец, всеславне, честен во Христе быв, и, победив мучители во Христе Спасителе, венец приял еси мученичества твоего и дарования Божественных исцелений, яко непобедим.


\medskip


\bfseries Молитва:\normalfont{}


О, святый мучениче Христов Трифоне, скорый помощниче всем, к тебе прибегающим, и молящимся пред святым твоим образом, скоропослушный предстателю! Услыши убо ныне и на всякий час моление нас, недостойных рабов твоих, почитающих святую память твою, и предстательствуй о нас пред Господем на всяком месте, ты бо, угодниче Христов, святый мученече и чудотворче Трифоне, в великих чудесех возсиявый, прежде исхода твоего от жития сего тленнаго молился еси ко Господу и испросил у Него дар сей: аще кто в коей-либо нужде, беде, печали и болезни душевней или телесней призывати начнет святое имя твое, той да избавлен будет от всякаго прилога злаго, и якоже ты иногда дщерь цареву, во граде Риме от диавола мучиму, исцелил еси, сице и нас от лютых его козней сохрани во вся дни живота нашего, наипаче же в день страшный последняго нашего издыхания предстательствуй о нас, буди нам тогда помошник, и скорый прогонитель лукавых духов, и к Царствию Небесному предводитель, идеже ты ныне предстоиши с лики святых у Престола Божия, моли Господа, да сподобит и нас причастниками быти присносущнаго веселия и радости, да с тобою купно удостоимся славити Отца и Сына и Святаго Утешителя Духа во веки веков. Аминь.


\section{Преподобному Иову Почаевскому}
%http://www.molitvoslov.com/text658.htm 
 


\bfseries Тропарь, глас 4-й:\normalfont{}


Возложь на ся иго Христово от юности, преподобне отче Иове, многолетне свято подвизался еси на поприще благочестия во обители Угорницкой и на острове Дубенстем, и пришед к горе Почаевской, знаменанной цельбоносною стопою Пресвятыя Богородицы, в тесной пещере каменней богомыслия ради и молитвы многократно заключался еси, и, благодатию Божиею укрепляяся, мужественно потрудился еси на пользу обители твоея, купно же и противу врагов Православия и благочестия христианскаго. И наставив сицевому ополчению иночествующих, победители тех представил еси Владыце своему и Богу. Того моли спастися душам нашим.


\medskip


\bfseries Кондак, глас 8-й:\normalfont{}


Возсия от спуда земнаго сокровище нетленное мощей твоих, угодниче Божий, яко, благочестно пожив в вере Христа Бога нашего, достигл еси добродетелей совершенства, и, оставив сладость жития преходящаго, в пещере горы Почаевския в пощениих, молитвах и трудех свято подвизался еси, и теми тело твое увядил еси. Ныне же, пришед к Богу в безмятежный и вечный покой, молишися о всех с верою к тебе прибегающих. Радуйся, Иове, преславный угодниче Божий и обители Почаевския украшение.


\medskip


\bfseries Молитва:\normalfont{}


О, всесвятый и преславный угодниче Божий, преподобне отче наш Иове, присный о нас ко Господу молитвенниче и теплый предстателю о душах наших, к тебе всеумиленно ныне притекаем, и поминающе подвиги и чудеса твоя, яже сотворил еси и твориши на земли, просим и молим твою благостыню: якоже твердо и неизменно потрудился еси в вере Христа Бога нашего, и сию до конца в себе и во всех присных твоих целу и невредиму сохранивый от всяких приражений вражиих и ересей тлетворных, сице и нас в Православии и единомыслии укрепи, отгоняя молитвами твоими всякую тьму неверия и неправомыслия от сердец и помышлений наших; послуживый Господу и Богу твоему делы благими и неизреченным самоотвержением в трудех, бдениих и пощениих, настави нас на путь всякия добредетели и благостыни, избавляя от искушений и грехов, удаляющих нас от Бога и повергающих в бездну зла все житие наше; явивыйся иногда с Пречистою Девою Богородицею верху горы Почаевския во спасение обители твоея от нашествия и обложения агарянскаго, и ныне ускори на помощь всей православней и боголюбивей державе нашей противу всех врагов наших внешних и внутренних, утверждая мир и тишину в земли нашей, да твоими молитвами и предстательством тихое и безмолвное житие поживем во всяком благочестии и чистоте; 

и всем, к тебе притекающим, и припадающим к раце честных и многоцелебных мощей твоих и твоея помощи и заступления требующим неоскудныя милости независтно подаваяй, не остави и нас, сирых и безпомощных, тебе молящихся, избавляя от всякия скорби, гнева и нужды, от глада, губительства, труса, потопа, огня, меча, нашествия иноплеменник и междоусобныя брани. 

Ей, угодниче Божий, призри милостивно от Престола Царя славы, Емуже ты ныне предстоиши со архангелы и ангелы и со всеми святыми, на обитель твою сию Почаевскую, еюже ты древле мудро правил еси, наздав ю всехвальным и дивным житием твоим, и сохрани ю молитвами твоими, и всякий град, и страну, и всех отовсюду, на мори и на суши, в пустынех и в злоключениих многоразличных тебе призывающих, от всех зол видимых и невидимых, да, тако твоею помощию и ходатайством спасаеми, в веце сем и по скончании живота нашего сподобимся купно с тобою славити и воспевати Всечестное имя Отца и Сына и Святаго Духа во веки веков. Аминь.


\section{Святому и праведному Иоанну Кронштадскому}
%http://www.molitvoslov.com/text657.htm 
 


\bfseries Тропарь, глас 1-й:\normalfont{}


Православныя веры поборниче, земли Российския печальниче, пастырем правило и образе верным, покаяния и жизни во Христе проповедниче, Божественных Тайн благоговейный служителю и дерзновенный о людех молитвенниче, отче праведный Иоанне, целителю и предивный чудотворче, граду Кронштадту похвало и Церкви нашея украшение, моли Всеблагаго Бога умирити мир и спасти души наша.


\medskip


\bfseries Кондак, глас 3-й:\normalfont{}


Днесь пастырь Кронштадтский предстоит Престолу Божию и усердно молит о верных Христа Пастыреначальника, обетование давшаго: "Созижду Церковь Мою и врата адова не одолеют ей".


\medskip


\bfseries Молитва:\normalfont{}


О, великий угодниче Христов, святый праведный отче Иоанне Кронштадтский, пастырю дивный, скорый помощниче и милостивый предстателю! 

Вознося славословие Триединому Богу, ты молитвенно взывал: 

"Имя Тебе – Любовь: не отвергни меня заблуждающагося;

Имя Тебе – Сила: укрепи меня изнемогающаго и падающаго;

Имя Тебе – Свет: просвети душу мою, омраченную житейскими страстями;

Имя Тебе – Мир: умири мятущуюся душу мою. 

Имя Тебе – Милость: не переставай миловать меня." 

Ныне благодарная твоему предстательству всероссийская паства молится тебе: 

Христоименитый и праведный угодниче Божий! 

Любовию твоею озари нас, грешных и немощных, сподоби нас принести достойные плоды покаяния и неосужденно причащатися Святых Христовых Таин;

cилою твоею веру в нас укрепи, в молитве поддержи, недуги и болезни исцели, от напастей, врагов видимых и невидимых избави; 

cветом лика твоего служителей и предстоятелей Aлтаря Христова на святые подвиги пастырскаго делания подвигни, младенцам воспитание даруй, юность настави, старость поддержи, святыни храмов и святые обители озари; 

умири, чудотворче и провидче преизряднейший, народы страны нашея, благодатию и даром Святаго Духа избави от междоусобныя брани, расточенныя собери, прелыщенныя обрати и совокупи Святей Соборной и Апостольской Церкви; 

Милостию твоею супружества в мире и единомыслии соблюди, монашествующим в делах благих преуспеяние и благословение даруй, малодушныя утеши, страждущих от духов нечистых свободи, в нуждах и обстояниях сущих помилуй и всех нас на путь спасения настави. 

Во Христе живый, отче наш Иоанне! 

Приведи нас к невечернему свету жизни вечныя, да сподобимся с тобою вечнаго блаженства, хваляще и превозносяще Бога во веки веков. Аминь.


\section{Святителю Спиридону чудотворцу, епископу Тримифунтскому}
%http://www.molitvoslov.com/text651.htm 
 


\bfseries Тропарь, глас 4-й:\normalfont{}


Собора перваго показался еси поборник и чудотворец, богоносне Спиридоне, отче наш. Темже мертву ты во гробе возгласив, и змию в злато претворил еси, и внегда пети тебе святыя молитвы, ангелы сослужащия тебе имел еси, священнейший. Слава Давшему тебе крепость, слава Венчавшему тя, слава Действующему тобою всем исцеления.


\medskip


\bfseries Кондак, глас 2-й:\normalfont{}


Любовию Христовою уязвився, священнейший, ум вперив зарею Духа, детельным видением твоим деяние обрел еси, Богоприятне, жертвенник Божественный быв, прося всем Божественнаго сияния.


\medskip


\bfseries Молитва:\normalfont{}


О, великий и пречудный святителю Христов и чудотворче Спиридоне, Керкирская похвале, всея вселенныя светильниче пресветлый, теплый к Богу молитвенниче и всем, к тебе прибегающим и с верою молящимся, скоропредстательный заступниче! Ты веру Православную на Никейстем Соборе посреде отцев преславно изъяснил еси, ты единство Святыя Троицы чудесною силою показал еси и еретиков до конца посрамил еси. Услыши, святителю Христов, нас грешных, молящихся тебе, и сильным твоим предстательством у Господа избави нас от всякаго злаго обстояния: от глада, потопа, огня и смертоносныя язвы. Ты бо во временней жизни своей от всех сих бедствий избавлял еси люди твоя: от нашествия Aгарян и от глада страну твою сохранил еси, царя от неисцельнаго недуга избавил и многая грешники к покаянию привел еси, за святость же жития твоего ангелы невидимо в церкви поющия и сослужащия тебе имел еси. Сице убо прослави тебе, вернаго Своего раба, Владыка Христос, яко вся тайная человеческая деяния дарова тебе разумети и живущия же неправедно обличати. Многим, в скудости и недостаточестве живущим, ты усердно помогал еси, люди убогия изобильно во время глада напитал еси, и иная многая знамения силою в тебе живущаго Духа Божия сотворил еси. Сице и нас не остави, святителю Христов, поминай нас, чад своих, у Престола Вседержителя, и умоли Господа, да подаст многих наших грехов прощение, безбедное и мирное житие да дарует нам, кончины же живота непостыдныя и мирныя и блаженства вечнаго в будущем веце сподобит нас, да непрестанно возсылаем славу и благодарение Отцу и Сыну и Духу Святому, ныне и присно и во веки веков. Аминь.


\section{Святому великомученику и целителю Пантелеимону}
%http://www.molitvoslov.com/text621.htm 
 
\myfig{img/478.jpg}

\bfseries Тропарь, глас 3-й:\normalfont{} 


Страстотерпче святый и целебниче Пантелеимоне, моли Милостивого Бога, да прегрешений оставление подаст душам нашим.


\bfseries  Кондак, глас 5-й:\normalfont{}


Подражатель сый Милостиваго, и исцелений благодать от Него прием, страстотерпче и мучениче Христа Бога, молитвами твоими душевныя наша недуги исцели, отгоняя присно борца соблазны от вопиющих верно: спаси ны, Господи.


\bfseries Молитва:\normalfont{}


Святый великомучениче и целителю Пантелеимоне, Бога милостиваго подражателю! Призри благосердием и услыши нас, грешных, пред святою твоей иконою усердне молящихся. Испроси нам у Господа Бога, Емуже со Ангелы предстоиши на небеси, оставление грехов и прегрешений наших: исцели болезни душевныя же и телесныя рабов Божиих, ныне поминаемых, зде предстоящих и всех христиан православных, к твоему заступлению притекающих: cе бо мы, по грехом нашим люте одержимы есмы многими недуги и не имамы помощи и утешения: к тебе же прибегаем, яко дадеся ти благодать молитися за ны и целити всяк недуг и всяку болезнь; даруй убо всем нам святыми молитвами твоими здравие и благомощие души и тела, преспеяние веры и благочестия и вся к житию временному и ко спасению потребная, яко да, сподобившися тобою великих и богатых милостей, прославим тя и Подателя всех благ, дивнаго во святых, Бога нашего, Отца, и Сына, и Святаго Духа. Аминь.





\section{Святому великомученику и целителю Пантелеимону молитва вторая, наедине от лица болящего читаемая}
%http://www.molitvoslov.com/text622.htm 
 


О великий угодниче Христов, страстотерпче и врачу многомилостивый, Пантелеимоне! Умилосердися надо мною, грешным рабом, услыши стенание и вопль мой, умилостиви небеснаго, верховнаго Врача душ и телес наших, Христа Бога нашего, да дарует ми исцеление от гнетущаго мя недуга. Приими недостойное моление грешнейшаго паче всех человек, посети мя благодатным посещением, не возгнушайся греховных язв моих, помажи елеем милости твоея и исцели мя; да, здрав сый душею и телом, остаток дний моих, благодатию Божиею, возмогу провести в покаянии и угождении Богу и сподоблюся восприяти благий конец жития моего. Ей, угодниче Божий! Умоли Христа Бога, да предстательством твоим дарует ми здравие телу и спасение души моей. Аминь.





\section{Пресвятой Богородице перед Ее иконой  &quot;Всех скорбящих радость&quot;}
%http://www.molitvoslov.com/text620.htm 
 


\bfseries Тропарь, глас 2-й:\normalfont{}


Всех скорбящих радосте и обидимых заступнице, и алчущих питательнице, странных утешение, обуреваемых пристанище, больных посещение, немощных покрове и заступнице, жезле старости, Мати Бога Вышняго Ты еси, Пречистая: потщися, молимся, спастися рабом Твоим.


\medskip


\bfseries Кондак, глас 6-й:\normalfont{}


Не имамы иныя помощи, не имамы иные надежды, разве Тебе, Владычице. Ты нам помози, на Тебе надеемся и Тобою хвалимся, Твои бо есмы рабы, да не постыдимся.


\medskip


\bfseries Молитва:\normalfont{}


О, Пресвятая Владычице Богородице, Преблагословенная Мати Христа Бога Спасителя нашего, всех скорбящих Радосте, больных посещение, немощных покрове и заступнице, вдовиц и сирых покровительнице, матерей печальных всенадежная утешительнице, младенцев немощных крепосте, и всем беспомощным всегда готовая помоще и верное прибежище! Тебе, о, Всемилостивая, дадеся от Всевышняго благодать во еже всех заступати и избавляти от скорби и болезней, зане Сама лютыя скорби и болезни претерпела еси, взирающи на вольное страдание Сына Твоего возлюбленнаго и Того на кресте распинаема зрящи, егда оружие Симеоном предреченное сердце Твое пройде. Темже убо, о Мати чадолюбивая, вонми гласу моления нашего, утеши нас в скорби сущих, яко верная радости Ходатаица: предстоящи престолу Пресвятыя Троицы, одесную Сына Твоего, Христа Бога нашего, можеши, аще восхощеши, вся нам полезная испросити. Сего ради с верою сердечною и любовию от души припадаем к Тебе яко Царице и Владычице и псаломски вопити Тебе дерзаем: слыша, Дщи, и виждь, и приклони ухо Твое, услыши моление наше, и избави нас от обстоящих бед и скорбей; Ты бо прошения всех верных, яко скорбящих радость, исполняeши, и душам их мир и утешение подавши. Се зриши беду нашу и скорбь: яви нам милость Твою, посли утешение уязвленному печалию сердцу нашему, покажи и удиви на нас грешных богатство милосердия Твоего, подаждь нам слезы покаяния ко очищению грехов наших и утолению гнева Божия, да с чистым сердцем, совестию благою и надеждою несумненною прибегаем ко Твоему ходатайству и заступлению: приими, всемилостивая наша Владычице Богородице, усердное моление наше Тебе приносимое, и не отрини нас, недостойных, от Твоего благосердия, но подаждь нам избавление от скорби и болезни, защити нас от всякаго навета вражия и клеветы человеческая, буди нам помощница неотступная  во все дни жизни нашея, яко да под Твоим матерним покровом всегда пребудем цели и сохранена Твоим заступлением и молитвами к Сыну Твоему и Богу Спасителю нашему, Ему же подобает всякая слава, честь и поклонение, со безначальным Его Отцом и Святым Духом, ныне и присно и во веки веков. Аминь.


\section{Бессребреникам и чудотворцам Косме и Дамиану}
%http://www.molitvoslov.com/text654.htm 
 
\bfseries Тропарь, глас 5-й:\normalfont{}


Святии безсребренницы и чудотворцы, Космо и Дамиане, посетите немощи наша: туне приясте, туне дадите нам.


\medskip
\bfseries Кондак, глас 2-й:\normalfont{}


Благодать приимше исцелений, простираете здравие сущим в нуждах, врачеве, чудотворцы преславнии, но вашим посещением ратников дерзости низложисте, мир исцеляюще чудесы.


\medskip
\bfseries Молитва:\normalfont{}


К вам, святии безсребреницы и чудотворцы Космо и Дамиане, яко к скорым помощником и теплым молитвенником о спасении нашем, мы недостойнии, преклонше колена, прибегаем и, припадающе, усердно вопием: не презрите моления нас грешных, немощных, во многая беззакония впадших и по вся дни и часы согрешающих. 

Умолите Господа, да пробавит нам, недостойным рабом Своим, великия и богатыя Своя милости, избавьте нас от всякия скорби и болезни, вы бо прияли есте от Господа и Спаса нашего Иисуса Христа неоскудную благодать исцелений, ради твердыя веры безмезднаго врачевания и мученическия кончины вашея. Паки, припадающе, прилежно молим: испросите нам от Господа вся благополезная, яже в животе нашем временном, наипаче же ко спасению вечному належащая, да сподобимся молитвами вашими улучити кончину христианскую, безболезненну, непостыдну, мирну, и да избавимся от козней диавольских и вечныя муки, безконечнаго же и блаженнаго Царствия Небеснаго наследницы будем. Ей, угодницы Божии, не престайте молящеся за ны, с верою к вам притекающия, аще бо по множеству грехов наших и несмы достойни милосердия вашего, обаче вы, вернии подражателие человеколюбия Божия суще, сотворите, да принесем плоды достойны покаяния и в вечный покой достигнем, хваляще и благословяще дивнаго во святых Своих Господа и Бога и Спаса нашего Иисуса Христа, и Пречистую Матерь Его, и ваше теплое заступление, всегда, ныне и присно и во веки веков. Аминь.

\mychapterending

\mychapter{Головные болезни}
%http://www.molitvoslov.com/golovnie_bolezni

 

\section{Пророку, Предтече и Крестителю Господню Иоанну}
%http://www.molitvoslov.com/content/proroku-predteche-i-krestitelyu-Gospodnyu-Ioannu 
 


\bfseries Тропарь, глас 2-й:\normalfont{}


Память праведного с похвалами, тебе же довлеет свидетельство Господне, Предтече: показал бо ся еси воистинну и пророков честнейший, яко и в струях крестити сподобился еси Проповеданнаго. Темже за истину пострадав радуяся, благовестил еси и сущим во аде Бога явльшагося плотию, вземлющаго грех мира и подающаго нам велию милость


\medskip


\bfseries Кондак, глас 5-й:\normalfont{}


Предтечево славное усекновение, смотрение бысть некое Божественное, да и сущим во аде Спасово проповесть пришествие; да рыдает убо Иродиа, беззаконное убийство испросивши: не закон бо Божий, ни живый век возлюби, но притворный, привременный.


\medskip


\bfseries Молитва:\normalfont{}


Крестителю Христов, проповедниче покаяния, кающагося не презри мене, но совокупляяся с вои небесными, молися ко Владыце за мене недостойнаго, унылаго, немощнаго и печальнаго, во многия беды впадшаго, утружденнаго бурными помыслы ума моего: аз бо есмь вертеп злых дел, отнюдь не имеяй конца греховному обычаю; пригвожден бо есть ум мой земным вещем. Что сотворю, не вем, и к кому прибегну, да спасена будет душа моя? Токмо к тебе, святый Иоанне, благодати тезоимените, яко тя пред Господем, по Богородице, вем больша быти рожденных всех, ибо ты сподобился есй коснутися верху Царя Христа, вземлющаго грехи мира, Агнца Божия: Егоже моли за грешную мою душу, да поне отныне, в первыйнадесять час, понесу тяготу благую и прииму мзду с последними.


Ей, Крестителю Христов, честный Предтече, крайний пророче, первый во благодати мучениче, постников и пустынников наставниче, чистоты учителю и ближний друже Христов, тя молю, к тебе прибегаю, не отрини мене от твоего заступления, но возстави мя, падшагося многими грехи; обнови душу мою покаянием, яко вторым крещением, понеже обоего начальник еси: крещением омываяй грех, покаяние же проповедуяй во очищение коегождо дел скверных; очисти мя грехми оскверненнаго и понуди внити, аможе ничтоже скверно входит, во Царствие Небесное. Аминь.


\mychapterending

\mychapter{Грудные болезни}
%http://www.molitvoslov.com/content/grudnie-bolezni

 

\section{Святому Димитрию, митрополиту Ростовскому}
%http://www.molitvoslov.com/text698.htm 
 


\bfseries Тропарь, глас 4-й:\normalfont{}


Православия ревнителю, и раскола искоренителю, Российский целебниче и новый к Богу молитвенниче, списаньми твоими буих уцеломудрил еси; цевнице духовная, Димитрие блаженне, моли Христа Бога, спастися душам нашим.


\medskip


\bfseries Кондак, глас 8-й:\normalfont{}


Звезду Российскую от Киева возсиявшую, и чрез Новград Северский в Ростов достигшую, всю же страну сию ученьми и чудесы озарившую, ублажим златословеснаго Димитрия, той бо всем вся написа, яже к наставлению, да всех приобрящет, якоже Павел Христу, и спасет правоверием души наша.


\medskip


\bfseries Молитва:\normalfont{}


О, предивный и преславный чудотворче Димитрие, исцеляяй недуги человеческия! Ты неусыпно молиши Господа Бога нашего о всех грешных: молю убо тя, буди ми ходатай пред Господем и помощник на преоборение страстей ненасытныя плоти моея и на одоление стрел сопротивника моего диавола, имиже уязвляет немощное сердце мое и, аки гладный и лютый зверь, алчет погубити душу мою. 

Ты, святителю Христов, моя ограда, ты мое заступление и оружие! Ты, великий чудотворче, во дни подвигов твоих в мире сем ревнуя, о Православной Церкви Божией, яко истинный и добрый пастырь, неблазненно обличал еси грехи и невежествия людския, и от стези правды в ереси и расколы уклонившихся на путь истины наставлял еси. Споспешествуй убо и мне кратковременный путь жизни моея исправити, да непреткновенно пойду по стези заповедей Божиих и неленостно поработаю Господеви моему Иисусу Христу, яко Единому Владыце моему, Искупителю и праведному Судии моему. К сим же припадая, молюся ти, угодниче Божий, егда изыти души моей от бренного сего телесе, избави ю от темных мытарств: не имам бо добрых дел ко оправданию моему, не даждь сатане возгордитися победою над немощною душею моею: избави ю от геенны, идеже плач и скрежет зубов, и святыми молитвами твоими сотвори мя причастника Небеснаго Царствия в Троице славимаго Бога, Отца и Сына и Святаго Духа, во веки веков. Аминь.


\mychapterending

\mychapter{Глазные болезни, слепота}
%http://www.molitvoslov.com/content/glaznie-bolezni-slepota

 

\section{Святому мученику Лонгину Сотнику, иже при Кресте Господни}
%http://www.molitvoslov.com/text684.htm 
 


\bfseries Тропарь, глас 4-й:\normalfont{}


Мученик Твой, Господи, Лонгин,  во страдании своем венец прият нетленный от Тебе, Бога нашего:  имеяй бо крепость Твою,  мучителей низложи, сокруши и демонов немощныя дерзости,  Того молитвами спаси души наша.


\medskip


\bfseries Кондак, глас 4-й:\normalfont{}


Весело возрадовася Церковь, в память днесь приснопамятнаго страдальца Лонгина, взывающи: Ты моя держава, Христе, и утверждение.


\medskip


\bfseries Молитва:\normalfont{}


О, святый мучениче Лонгине! Пилатом повелено тебе было с воинами стать на страже при Кресте Господа Иисуса. Ты же, недуговав очима и изцелев от капли крови Господа, капнувшей тебе в очи, обрел и духовное прозрение и, видя чудеса, при распятии Господа происходившие, и трус, и солнечное затмение, и мертвых из гробов воскресение, исповедал явно Иисуса Христа Сыном Божиим. И с кустодиею у гроба Господня находясь, и Воскресение Христово в трепете зря, сребренники, даваемые синедрионом за сокрытие воскресения, ты отверг и, Христа проповедав, во главу усечен быв. И по усечении главы вдове некой, очима неугодовавшей, во сне явившись, и ей, обретшей честную главу твою, прозрение даровал еси. 


Молим тя, святый мучениче Христов Лонгине, болящим  очима яви скорую помощь твою и исцели их, дабы, освободившись от недуга своего, не захотели они видеть ничего, распаляющаго сластолюбие, но к созерцанию духовной красоты устремились, славя Бога. Аминь.


\section{Святому равноапостольному великому князю Владимиру, во Святом Крещении Василию, Крестителю Руси}
%http://www.molitvoslov.com/text685.htm 
 


\bfseries Тропарь, глас 4-й:\normalfont{}


Уподобился еси купцу, ищущему добраго бисера, славнодержавный Владимире, на высоте стола сидя матере градов, Богоспасаемаго Киева, испытуя же и посылая к царскому граду уведети Православную веру, и обрел еси безценный бисер, Христа, избравшаго тя, яко втораго Павла, и отрясшаго слепоту во святей купели, душевную вкупе и телесную. Темже празднуем твое успение, людие твои суще: моли спастися державы твоея Российския начальником и множеству владомых.


\medskip


\bfseries Кондак, глас 8-й:\normalfont{}


Подобствовав великому апостолу Павлу, в сединах, всеславне Владимире, вся, яко младенческая, мудрования, яже о идолех тщания оставль, яко муж совершенный, украсился еси Божественнаго Крещения багряницею, и ныне Спасу Христу в веселии предстоя, моли спастися державы Российския начальником и множеству владомых.


\medskip


\bfseries Молитва:\normalfont{}


О великий и преславный угодниче Божий, богоизбранный и богопрославленный, равноапостольный княже Владимире, святое и великодейственное орудие всеблагаго Промысла о спасении народа Российскаго! Ты отринул еси зловерие и нечестие языческое, уверовал еси во Единаго Истиннаго Триипостаснаго Бога, и восприяв святое Крещение, просветил еси светом божественныя веры и благочестия царство Российское. Славяще убо и благодаряще Премилосердаго Творца и Спасителя нашего, славим и благодарим и тя, великий пастырю и отче наш, яко тобою познахом спасительную веру Христову, и крестихомся во имя Пресвятыя и Пребожественныя Троицы: тоюже верою избавихомся от праведнаго осуждения Божия, вечнаго рабства диаволя и адова мучительства: тою верою восприяхом благодать всыновления Богу и надежду наследования небеснаго блаженства. Ты еси первейший наш вождь к начальнику и совершителю нашего вечнаго спасения, Господу Иисусу Христу: ты еси ближайший предстоятель пред престолом Царя царствующих и теплый молитвенник и ходатай о царстве Всероссийстем, о народоправителех его и о всех людех: ты еси первейший виновник благословений и милостей Божиих, почивающих на нем. И что еще речем? Не может язык наш изобразити и высоту благодеяний твоих, излиянных на нас, недостойных. Но, о неразумия и ослепления нашего! Приемше толикия благодеяния, ни во что же вменихом я и отщетихомся спасительных плодов их. Омывшеся бо от греха в купели крещения и облекшеся в одежду чистоты и невинности, осквернихом сию богоданную одежду студными деяньми и помышленьми нашими: отрекшеся сатаны и ангелов его, паки порабощаемся ему, служаще идолам страстей наших, миру, плоти и злым обычаем века: сочетавшеся Христу, выну оскорбляем Его беззаконьми нашими, многообразными язвами гордости, зависти, злобы, злословия, невоздержания и презорства ко Святей Церкви: прилепихомся всецело к суетным благам, аки мняще во веки пребывати на земли: не помышляем о небе, о душе, о смерти, о Суде, о нескончаемой вечности. Сего ради воздвизаем на ся праведный гнев и осуждение Божие, купно же оскорбляем и преогорчаем твою отчую любовь и попечение о нас: ты бо просветил еси нас Святым Крещением, во еже способствовати нам к получению небеснаго блаженства и земнаго благоденствия: мы же, неразумнии, злым произволением нашим сами себе подвергаем адовым мукам и временным бедствиям! Но, о всеблагий отче и просветителю наш! Милостив буди к нашим немощем, долготерпелив ко грехам и неправдам нашим: умоли Премилосердаго Царя Небеснаго, да не прогневается на ны зело и не погубит нас со беззаконьми нашими, но да помилует и спасет нас, имиже весть судьбами: да всадит в сердце наше спасительный страх Свой, да просветит Своею благодатию ум наш, во еже узрети нам бездну погибели, в нюже стремимся, оставити стези нечестия и заблуждений, обратитися же на путь спасения и истины, неуклонно исполняти заповеди Божии и уставы Святыя Церкве. Моли, благосерде, человеколюбца Бога, да явит нам великую милость Свою: да избавит нас от нашествия иноплеменников, от внутренних нестроений, мятежей и раздоров, от глада, смертоносных болезней и от всякаго зла: да подаст нам благорастворение воздуха и плодоносие земли: да сохранит и спасет Отечество наше от всех козней и наветов вражиих: да дарует нам победу над врагами, да исполнит вся благая желания наша: да оградит державу нашу мудрыми и верными деятелями, да сохранит в судящих и начальствующих правду и милость: да даст духовным пастырем непорочность жития и ревность о спасении пасомых, всем же людем усердие в исполнении своих обязанностей, взаимную любовь и единомыслие, стремление ко благу Отечества и Святыя Церкве: да распространит свет спасительныя веры в царстве нашем во всех концах его: да обратит к правоверию неверующих: да упразднит вся ереси и расколы, да, тако поживше в мире на земли, сподобимся с тобою вечнаго блаженства, хваляще и превозносяще Бога во веки веков. Аминь.


\section{Святому великомученнику Димитрию Солунскому}
%http://www.molitvoslov.com/text667.htm 
 


\bfseries Тропарь, глас 3-й:\normalfont{}


Велика обрете в бедах тя поборника вселенныя, страстотерпче, языки побеждающа, якоже убо Лиеву низложил еси гордыню, на подвиг дерзновенна сотворив Нестора, тако, святе Димитрие, Христу Богу молися, даровати нам велию милость.


\medskip


\bfseries Кондак, глас 2-й:\normalfont{}


Кровей твоих струями, Димитрие, Церковь Бог обагри, давый тебе крепость непобедимую, и соблюдая град твой невредим; того бо еси утверждение.


\medskip


\bfseries Молитва:\normalfont{}


Святый и славный великомучениче Христов Димитрие, скорый помощниче и теплый заступниче с верою притекающих к тебе! Предстоя со дерзновением Небесному Царю, испроси у Него прощение согрешений наших, и еже избавитися нам от всегубительныя язвы, труса, потопа, огня, меча и вечныя казни: моли благость Его, еже ущедрити град сей (\itshape обитель сию\normalfont{}) и всякую страну христианскую, исходатайствуй у Царя царствующих на враги победу народу русскому и всей же державе нашей мир, тишину, твердость в вере и преспеяние во благочестии; нам же, чтущим честную память твою, испроси благодатное укрепление на дела благая, да, благоугодное Владыце нашему Христу Богу зде творяще, сподобимся молитвами твоими наследовати Царствие Небесное и тамо прославляти Его со Отцем и Святым Духом во веки веков. Аминь.


\section{Пресвятой Богородице перед Ее иконой &quot;Казанская&quot;}
%http://www.molitvoslov.com/text666.htm 
 


\bfseries Тропарь, глас 4-й:\normalfont{}


Заступнице усердная,  Мати Господа Вышняго,  за всех молиши Сына Твоего Христа Бога нашего,  и всем твориши спастися,  в державный Твой покров прибегающим.  Всех нас заступи, о Госпоже Царице и Владычице,  иже в напастех и в скорбех, и в болезнех, обременённых грехи многими,  предстоящих и молящихся Тебе умиленною душею и сокрушенным сердцем,  пред пречистым Твоим образом со слезами  и невозвратно надежду имущих на Тя,  избавления всех зол,  всем полезная даруй  и вся спаси, Богородице Дево:  Ты бо еси Божественный Покров рабом Твоим.


\medskip


\bfseries Кондак, глас 8-й:\normalfont{}


Притецем, людие, к тихому сему и доброму пристанищу, скорой Помощнице, готовому и теплому спасению, покрову Девы; ускорим на молитву и потщимся на покаяние: источает бо нам неоскудныя милости Пречистая Богородица, предваряет на помощь и избавляет от великих бед и зол благонаравныя и богобоящияся рабы Своя.


\medskip


\bfseries Молитва:\normalfont{}


О пресвятая Госпоже, Владычице Богородице! Со страхом, верою и любовию пред честною и чудотворною иконою Твоею, молим Тя: не отврати лица Твоего от прибегающих к Тебе, умоли, милосердная Мати, Сына Твоего и Бога нашего, Господа Иисуса Христа, да сохранит мирну страну нашу, Церковь же Свою святую непоколебиму да соблюдет, и от неверия, ересей и раскола да избавит. Не имамы бо иныя помощи, не имамы иныя надежды, разве Тебе, пречистая Дево: Ты еси всесильная христиан Помощница и Заступница: избави же и всех, с верою Тебе молящихся, от падений греховных, от навета злых человек, от всяких искушений, скорбей, болезней, бед и от внезапныя смерти: даруй нам дух сокрушения, смирение сердца, чистоту помышлений, исправление греховныя жизни и оставление прегрешений, да вси, благодарне воспевающе величия и милости Твоя, являемыя над нами зде на земли, сподобимся и Небеснаго Царствия, и тамо со всеми святыми прославим пречестное и великолепное имя Отца и Сына и Святого Духа во веки веков.


\section{Святителю Алексию, митрополиту Московскому, всея Руси чудотворцу}
%http://www.molitvoslov.com/text683.htm 
 


\bfseries Тропарь, глас 8-й:\normalfont{}


Яко Апостолом сопрестольна и врача предобра, и служителя благоприятна, к раце твоей честней притекающе, святителю Алексие, богомудре чудотворче, сошедшеся любовию в память твою, светло празднуем, в песнех и пениих радующеся и Христа славяще, таковую благодать тебе даровавшаго исцелений и граду твоему великое утверждение.


\medskip


\bfseries Кондак, глас 8-й:\normalfont{}


Божественнаго и пречестнаго святителя Христова, новаго чудотворца Алексия верно вси поюще, людие, любовию да ублажим, яко пастыря великаго, служителя же и учителя премудра земли Российстей: днесь в память его притекше, радостно возопием песнь Богоносному, яко имея дерзновение к Богу, многообразных нас избави обстояний, да зовем ти: радуйся, утверждение граду нашему.


\medskip


\bfseries Молитва:\normalfont{}


О, пречестная и священная главо и благодати Святаго Духа исполненная, Спасово со Отцем обиталище, великий архиерее, теплый наш заступниче, святителю Алексие! Предстоя у Престола всех Царя и наслаждаяся света Единосущныя Троицы и херувимски со ангелы возглашая песнь трисвятую, великое же и неизследованное дерзновение имея ко всемилостивому Владыце, молися паствы Христовы спасти люди, единородный ти язык: благостояние святых церквей утверди; архиереи благолепием святительства украси; монашествующия к подвигом добраго течения укрепи; царствующий град сей (\itshape и святую обитель сию\normalfont{}) и вся грады и страны добре сохрани, и веру святую непорочну соблюсти умоли; мир весь предстательством твоим умири, от глада и пагубы избави ны, и от нападения иноплеменных сохрани; старыя утеши, юныя настави, безумныя умудри, вдовицы помилуй, сироты заступи, младенцы возрасти, плененныя возврати, немощствующия исцели и везде тепле призывающих тя и с верою притекающих к раце честных и многоцелебных мощей твоих, усердно припадающих и молящихся тебе; от всяких напастей и бед ходатайством твоим свободи, да зовем ти: богоизбранный пастырю, звездо всесветлая мысленныя тверди, тайнаго Сиона необоримый столпе, миродохновенный цвете райский, всезлатая уста слова, московская похвало, всея России украшение! Моли о нас Всещедраго и Человеколюбиваго Христа, Бога нашего, да в день страшнаго пришествия Его шуияго стояния избавит нас, и радости святых причастники сотворит со всеми святыми во веки. Аминь.


\section{Святому мученику и архидиакону Лаврентию Римскому}
%http://www.molitvoslov.com/text682.htm 
 


\bfseries Тропарь, глас 4-й:\normalfont{}


Мученик Твой, Господи, Лаврентий,  во страдании своем венец прият нетленный от Тебе, Бога нашего:  имеяй бо крепость Твою,  мучителей низложи,  сокруши и демонов немощныя дерзости,  Того молитвами спаси души наша.


\medskip


\bfseries Кондак, глас 4-й:\normalfont{}


Огнем Божественным распалив сердце твое, огнь страстей до конца испепелил еси, страдальцев утверждение, богоносне мучениче Лаврентие, и в страданиих вопиял еси верно: ничтоже мя разлучит любве Христовы.


\medskip


\bfseries Молитва:\normalfont{}


О, пресвятый и предивный стратотерпче Христов, архидиаконе Лаврентие! Восхваляюще веру и страдания твоя, почитаем победу и венценосное прохождение твое чрез горящее углие от тьмы века сего ко свету невечернему Престола Величествия Божия,  темже убо молим тя: якоже древле притекавшия с верою к покрову твоему чудесы твоими утвердил еси, тако да приимеши и нас под покров твой, и в болезни и печалех наших заступник нам да будеши: и якоже от слепоты телесных очес Крискентиана преложением рук исцелил еси, тако да исцелиши предстательством твоим у Престола Божия и нашу душевную слепоту; призри на разслабление наше телесное и душевное, и укрепи нас бодренностию противу враг наших видимых и невидимых, удручающих нас напастьми.  Да твоею, помощию путь маловременныя жизни сея прешедше, непобежден и от бед и скорбей и всякаго лютаго обстояния, достигнем неприступныя славы величествия Божия, идеже предстоя, молишися со дерзновением за люди прибегающия с верою к тебе и воспевающия дивнаго во святых Бога Израилева во веки веков. Аминь.


\mychapterending

\mychapter{Зубная боль}
%http://www.molitvoslov.com/content/zubnaya-bol

 

\section{Священномученику Антипе, епископу Пергамскому}
%http://www.molitvoslov.com/text687.htm 
 


\bfseries Тропарь, глас 4-й:\normalfont{}


И нравом причастник, и престолом наместник апостолом быв, деяние обрел еси, богодухновение, в видения восход: сего ради слово истины исправляя, веры ради пострадал еси даже до крове, священномучениче Антипо, моли Христа Бога спастися душам нашим.


\medskip


\bfseries Кондак, глас 4-й:\normalfont{}


Апостолов сопрестольник и святителей украшение был еси, блаженне. Мученически прославився, возсиял еси. Якоже солнце, всех просвещая, Антипо священне, разрушил еси безбожия нощь глубокую. Сего ради тя почитаем, яко Божественнаго суща священномученика и целеб подателя.


\medskip


\bfseries Молитва:\normalfont{}


О, преславный священномучениче Антипо и скорый помощниче христианом в болезнех! Верую от всея души и помышления, яко дадеся тебе от Господа дар болящия врачевати и разслабленныя укрепляти, сего ради к тебе, яко благодатному врачу болезней, аз немощный (\itshape или немощная\normalfont{}) прибегаю и, твой досточтимый образ со благоговением лобызая (\itshape или лобызающи\normalfont{}), молюся: твоим предстательством у Царя Небеснаго испроси мне болящему (\itshape или болящей\normalfont{}) исцеление от удручающия мя зубныя болезни, аще бо и недостоин (\itshape или недостойна\normalfont{}) есмь тебе благостнейшаго отца и приснаго заступника моего: но ты, быв подражатель человеколюбия Божия, сотвори мя достойна (\itshape или достойну\normalfont{}) твоего заступления чрез мое обращение от злых дел к благому житию: уврачуй обильно дарованною тебе благодатию язвы и струпы души и тела моего, даруй ми здравие и спасение и во всем благое поспешение, да тако тихое и безмолвное житие пожив (\itshape или поживши\normalfont{}) во всяком благочестии и чистоте, сподоблюся со всеми святыми славити Всесвятое имя Отца и Сына и Святаго Духа. Аминь.


\mychapterending

\mychapter{Болезни сердца}
%http://www.molitvoslov.com/content/bolezni-serdca

 

\section{Святителю Иоасафу, епископу Белгородскому}
%http://www.molitvoslov.com/text689.htm 
 


\bfseries Тропарь, глас 3-й:\normalfont{}


Святителю Христу Богу возлюбленне, правило веры и образ милосердия людем был еси, бдением же, постом и молитвою яко светильник пресветлый просиял еси, и прославлен от Бога явился еси: телом убо в нетлении почивая, духом же Престолу Божию предстоя, чудеса преславная источаеши. Моли Христа Бога, да утвердит Отечество наше в Православии и благочестии и спасет души наша.


\medskip


\bfseries Кондак, глас 8-й:\normalfont{}


Многоразличныя подвиги жития твоего кто исповесть? Многообразныя милости Божия тобою явленныя кто исчислит? Дерзновение же твое у Пречистыя Богородицы и Всещедраго Бога добре ведуще, во умилении сердечнем зовем ти: не лиши и нас твоея помощи и заступления, святителю Христов и чудотворец Иоасафе.


\medskip


\bfseries Молитва:\normalfont{}


О, угодниче Божий святителю Иоасафе! Из глубины сердца взываем к тебе раби Божии (\itshape имена\normalfont{}), огради нас от соблазна, ересей и расколов, научи горняя мудрствовати, рассеянный ум наш просвети и на путь истины направи, охладевшее сердце согрей любовию ко ближнему и ревностию ко исполнению велений Божиих, грехом и нерадением ослабленную волю нашу оживотвори благодатию Духа Всесвятого. Да твоему пастырскому гласу последующе, сохраним в чистоте и правде души наша, и тако Богу помогающу, Небесного Царствия достигнем, идеже купно с тобою воспрославим пречестное и великолепное имя Отца и Сына и Святаго Духа во веки веков. Аминь.


\mychapterending

\mychapter{Немота}
%http://www.molitvoslov.com/content/nemota

 

\section{Преподобному Иоанну Рыльскому}
%http://www.molitvoslov.com/text691.htm 
 


\bfseries Тропарь, глас 1-й:\normalfont{}


Покаяния основание, прописание умиления, образ утешения, духовнаго совершения, равноангельское житие твое бысть, преподобне. В молитвах убо и пощениих и в слезах пребывавый, отче Иоанне, моли Христа Бога о душах наших.


\medskip


\bfseries Кондак, глас 8-й:\normalfont{}


Ангельскому житию поревновав, преподобне, вся земная оставив, ко Христу притекл еси, и Того заповедьми ограждаяся, явился еси столп непоколебим от вражиих нападений. Тем зовем ти: радуйся, отче Иоанне, светило пресветлое.


\medskip


\bfseries Молитва:\normalfont{}


О, преподобне и богоносне отче наш Иоанне, зрителю неизреченныя славы в Небеснем Царствии и помощниче благодатный всем к тебе притекающим, не остави места сего, на немже труды и подвиги о Христе подъял еси и землю слезами твоими оросил еси! Да пребудет обитель твоя, в нейже благоволи Бог святым твоим мощем почивати, незыблема от ухищрений врагов видимых и невидимых. Укрепи в доброделании чада твоя, ихже во обитель твою собрал еси, яко да познают звание свое и последуют стопам твоим. Страну нашу в мире соблюди. Православным людем во бранех на сопротивныя споборствуй. Посети с высоты небесныя всех, верно к тебе притекающих и просящих твоего крепкаго заступления; в мори плавающих управи, житейскаго моря волнение утиши, церковныя раздоры укроти, мир всему миру у Господа испроси, болящих уврачуй спасительным посещением, скорбных утеши благодатию, всем буди вся, якоже был еси в земнем житии твоем.

Сотвори о нас к Богу молитву, дана бо ти есть благодать молитися за ны, яко да сподобимся твоим ходатайством улучити вечное блаженство купно с тобою и всеми святыми во Христе Иисусе, Господе нашем, Емуже подобает всякая слава, честь и поклонение со Отцем и Святым Духом во веки веков. Аминь.


\mychapterending

\mychapter{Потеря слуха и болезни ушей}
%http://www.molitvoslov.com/content/poterya-sluha-i-bolezni-ushej

 

\section{Пресвятой Богородице перед Ее иконой &quot;Нечаянная радость&quot;}
%http://www.molitvoslov.com/text696.htm 
 


\bfseries Тропарь, глас 4-й:\normalfont{}


Днесь вернии людие духовно торжествуем, прославляюще Заступницу усердную рода христианскаго и притекающе к Пречистому Ея образу, взываем сице: о, Премилостивая Владычице Богородице, подаждь нам нечаянную радость, обремененным грехи и скорбьми многими, и избави нас от всякаго зла, молящи Сына Твоего, Христа Бога нашего, спасти души наша. 

\bfseries 

Кондак, глас 6-й:\normalfont{}


Не имамы иныя помощи, не имамы иныя надежды, разве Тебе, Владычице, Ты нам помози, на Тебя надеемся и Тобою хвалимся, Твои бо есмы рабы, да не постыдимся. 

\bfseries 

Молитва:\normalfont{}


О, Пресвятая Дево, Всеблагаго Сына Мати Всеблагая, града сего покрове, всех сущих во гресех, скорбех, бедах и болезнех верная Предстательнице и Заступнице! Приими молебное пение сие от нас, недостойных рабов Твоих Тебе возносимое; и якоже древле грешника, на всяк день многажды пред честною иконою Твоею молившагося, не презрела еси, но нечаянную радость покаяния тому даровала еси, и усердным к Сыну Твоему ходатайством Сего ко прощению грешника преклонила еси, тако и ныне не презри моления нас, недостойных раб Твоих, но умоли Сына Твоего и Бога нашего, да и всем нам, с верою и умилением пред цельбоносным образом Твоим покланяющимся, по коегождо потребе нечаянную радость дарует; да вси на небеси и на земли видят Тя, яко твердую и непостыдную Предстательницу рода христианского, и сие ведуще славят Тя и Тобою Сына Твоего со Безначальным Его Отцем и Единосущным Его Духом, ныне и присно и во веки веков. Аминь.





\mychapterending

\mychapter{Болезни чревные, грыжа}
%http://www.molitvoslov.com/content/bolezni-chrevnie-grizha

 

\section{Великомученику Артемию}
%http://www.molitvoslov.com/text703.htm 
 


\bfseries Тропарь, глас 4-й:\normalfont{}


Истинною бо Христовою верою утверждься, страстотерпче, мучителя злочестива царя победил еси со идольским того возношением. Тем от Царя Великаго, вечно царствующаго, пресветлым венцем победным одарен бысть, вся исцеляяй болящия и призывающия тя, Артемие великий, моли Христа Бога спасти души наша.


\medskip


\bfseries Кондак, глас 2-й:\normalfont{}


Благочестиваго и венценоснаго мученика, на враги победы вземшаго одоление, сошедшеся, достойно песньми восхвалим Артемия превеликаго в мученицех, чудес же дателя пребогатаго, молится бо Господу о всех нас.


\medskip


\bfseries Славник, глас 2-й:\normalfont{}


Разумного светильника веры, Артемия почтим. Яко обличи царя мерзкого, и кровию мучения его Церковь Бог обагри. Тем и прият исцелений благодать независтную исцеляти недуго верно приходящих к раце мощей его.


\medskip


\bfseries Молитва:\normalfont{}


Святый мучениче Артемие! Призри с небеснаго чертога на требующих твоея помощи и не отвергни прошений наших, но, яко присный благодетель и ходатай наш, моли Христа Бога, да, человеколюбив и многомилостив сый, сохранит нас от всякаго лютаго обстояния: от труса, потопа, огня, меча, нашествия иноплеменников и междоусобныя брани. Да не осудит нас грешных по беззаконием нашим, и да не во зло обратим благая, даруемая нам от Всещедраго Бога, но во славу святаго имене Его и в прославление крепкаго твоего заступления. Да молитвами твоими даст нам Господь мир помыслов, воздержание от пагубных страстей и от всякия скверны и да укрепит во всем мире Свою Едину Святую, Соборную и Апостольскую Церковь, юже стяжал есть честною Своею Кровию. Молися прилежно, святий мучениче, да благословит Христос Бог державу, да утвердит во святей Своей Православней Церкви живый дух правыя веры и благочестия, да вси члены ея, чистии от суемудрия и суеверия, духом и истиною покланяются Ему и усердно пекутся о соблюдении Его заповедей, да мы вси в мире и благочестии поживем в настоящем веце и достигнем блаженныя вечныя жизни на небеси, благодатию Господа нашего Иисуса Христа, Емуже подобает всякая слава, честь и держава со Отцем и Святым Духом, ныне и присно и во веки веков. Аминь.


\section{Преподобному Серафиму Саровскому}
%http://www.molitvoslov.com/text702.htm 
 


\bfseries Тропарь, глас 4-й:\normalfont{}


От юности Христа возлюбил еси, блаженне, и Тому Единому работати пламенне вожделев, непрестанною молитвою и трудом в пустыни подвизался еси, умиленным же сердцем любовь Христову стяжав, избранник возлюблен Божия Матери явился еси. Сего ради вопием ти: спасай нас молитвами твоими, Серафиме, преподобне отче наш.


\medskip


\bfseries Кондак, глас 2-й:\normalfont{}


Мира красоту и яже в нем тленная оставив, преподобне, в Саровскую обитель вселился еси; и тамо ангельски пожив, многим путь был еси ко спасению. Сего ради и Христос тебе, отче Серафиме, прослави, и даром исцелений и чудес обогати. Темже вопием ти: радуйся, Серафиме, преподобне отче наш.


\medskip


\bfseries Молитва:\normalfont{}


О, пречудный отче Серафиме, великий Саровский чудотворче, всем прибегающим к тебе скоропослушный помощниче! Во дни земнаго жития твоего никтоже от тебя тощь и неутешен отыде, но всем в сладость бысть видение лика твоего и благоуветливый глас словес твоих. К сим же и дар исцелений, дар прозрений, дар немощных душ врачевания обилен в тебе явися. Егда же призва тя Бог от земных трудов к небесному упокоению, николиже любовь твоя преста от нас, и невозможно есть исчислити чудеса твоя, умножившаяся, яко звезды небесныя: се бо по всем концем земли нашея людем Божиим являешися и даруеши им исцеления. Темже и мы вопием ти: о, претихий и кроткий угодниче Божий, дерзновенный к Нему молитвенниче, николиже призывающия тя отреваяй! Вознеси о нас благомощную твою молитву ко Господу сил, да дарует нам вся благотребная в жизни сей и вся к душевному спасению полезная, да оградит нас от падений греховных и истинному покаянию да научит нас, во еже безпреткновенно внити нам в Вечное Небесное Царство, идеже ты ныне в незаходимей сияеши славе, и тамо воспевати со всеми святыми Живоначальную Троицу во веки веков. Аминь.


\section{Преподобному Феодору Студиту}
%http://www.molitvoslov.com/text701.htm 
 


\bfseries Тропарь, глас 8-й:\normalfont{}


Православия наставниче, благочестия учителю и чистоты, вселенныя светильниче, монашествующих богодухновенное удобрение, Феодоре премудре, ученьми твоими вся просветил еси. Цевнице духовная, моли Христа Бога спастися душам нашим.


\medskip


\bfseries Кондак, глас 2-й:\normalfont{}


Постническое и равноангельское житие твое страдальческими уяснил еси подвиги и Ангелом совсельник, богоблаженне, явился еси, Феодоре. С ними Христу Богу моляся не престай о всех нас. 


\medskip


\bfseries Молитва:\normalfont{}


О, святый Феодоре, исповедник Православия, закона Божия ревнитель, наставник правой веры, помози нам и призри на нас, с верою и усердием прибегающих к тебе! От царей иконоборцев за почитание святых икон оковы, изгнание, заточение претерпел ты, преподобный, и множество тяжких ран и жестоких ударов, от которых и болезнь нестерпимую получил. 

Великий и досточудный Феодоре, до конца в скорбех претерпевый, и нам, обуреваемым напастями и треволнениями житейскими, подай крепость душевную и телесную для борьбы с бедами и для одоления страстей и козней вражиих. 

Моли о нас Милосерднаго Господа, верный раб Божий, Феодоре, да избавит Он нас от ожесточения и хладности сердечной, и дарует нам кротость, безгневие, незлобие, смирение и любовь. Видя помощь Твою, прославим мы Господа и Спаса нашего Иисуса Христа, и тебя, благодетеля нашего, возблагодарим, с благоговением почитая память твою. 

О, всехвальне, преподобне Феодоре, учением и песньми своими Церковь Божию напоивший и обрадовавший, и нас, ослепленных умом от соблазнов и суеты мира сего, просвети познанием истины и избавь от заблуждений и сомнений, чтобы не сообразоваться нам веку сему. 

О, угодниче Христов, Феодоре, получивший от Бога благодать исцеления недугующих, а наипаче страждущих желудком, уврачуй и нас, одержимых различными недугами и исцели болящих желудком и подай нам здравие, бодрость и мир душевный. Не отрини нас, притекающих к тебе, но заступи и помози нам. Предстательствуй за нас пред Спасителем нашим, внемли мольбам нашим и исполни прошения наши, да тобою укрепляемы, в трудах и добрых делах совершим мы течение жизни нашей и сподобимся войти в Царствие Небесное, Да вместе с тобою восхвалим и прославим Господа и Спаса нашего Иисуса Христа, со Безначальным Его Отцем и с Пресвятым Духом, ныне, и присно, и во веки веков. Аминь.


\section{Праведному Артемию Веркольскому}
%http://www.molitvoslov.com/text700.htm 
 


\bfseries Тропарь, глас 2-й:\normalfont{}


Вышняго повелением, тученосным облаком, небо помрачившим и молниям блистающим, грому же возшумевшему с прещением, испустил еси душу твою в руце Господеви, премудре Артемие, и ныне предстоиши Престолу Владыки всех, иже верою и любовию приходящим к тебе, подая исцеление всем неотложно и моляся Христу Богу спастися душам нашим.


\medskip


\bfseries Кондак, глас 8-й:\normalfont{}


Возсия днесь пресветлая память премудраго Артемия; богоданная благодать, яко реки, изливает от целебных мощей его дивная исцеления, имиже от многоразличных недуг избавляемся, с верою приемлюще я и взывающе: радуйся, Артемие богомудре.


\medskip


\bfseries Молитва:\normalfont{}


Святый угодниче Божий, праведный Артемие, присный хранителю святыя веры Православныя и близкий защитниче всего севернаго края Российския страны! 

Призри милостивно на усердную молитву нас, грешных, и твоим благомощным предстательством испроси нам у Господа прощение согрешений наших, преспеяние в вере и благочестии и ограждение от козней диавольских. 

Моли Господа, да хранит во здравии и непременяемом благополучии верных людей своих, да подаст стране нашей мир и тишину, а нам нелицемерное послушание; да сподобит всех нас получити, по христианстей кончине, Небесное Царствие, идеже вси праведнии вместе с тобою, вечно славят Отца и Сына и Святаго Духа. Аминь.


\mychapterending

\mychapter{Болезни ног}
%http://www.molitvoslov.com/content/bolezni-nog

 

\section{Святому праведному Симеону Верхотурскому}
%http://www.molitvoslov.com/text705.htm 
 


\bfseries Тропарь, глас 4-й:\normalfont{}


Мирскаго мятежа бегая, все желание обратил еси к Богу, да в видения восход обрящеши горе, отнюдуже не уклонився в лукавствия сердца, но очистив душу и тело, приял еси благодать точити цельбы верным и неверным, притекающим к раце мощей твоих, праведный Симеоне! Темже, по данному ти дару, испроси у Христа Бога исцеление нам, болящим душевными страстьми, и моли спасти души наша.


\medskip


\bfseries Кондак, глас 2-й:\normalfont{}


Мира суетнаго отверглся еси, да блага вечныя жизни наследши, возлюбив незлобие и чистоту души и тела. Снискал еси, еже возлюбил, свидетельствуют бо о сем гроб и нетление мощей твоих, и благодать чудотворения наипаче. Точиши бо цельбы всем притекающим к тебе и непросвещенным, Симеоне блаженне, чудотворче предивный.


\medskip


\bfseries Молитва:\normalfont{}


О, святый и праведный Симеоне, чистою душею твоею в небесных обителех в лице святых водворяяйся, на земли же телом твоим нетленно почиваяй по данной ти благодати от Господа молитися о нас. Милостивно призри на нас многогрешных, аще и недостойне, обаче с верою и упованием ко святым и цельбоносным мощем твоим притекающих, и испроси нам от Бога прощение согрешений наших, в няже впадаем множицею во вся дни жития нашего. И якоже прежде овым убо от очныя зельныя болезни ни мало зрети могущим исцеление очес, овым же близ смерти бывшим от лютых недугов врачевание, и иным иная многая преславная благодеяния даровал еси: сице избави и нас от недугов душевных и телесных и от всякия скорби и печали, и вся благая к настоящему житию нашему и к вечному спасению благопотребная нам от Господа испроси, да тако твоим предстательством и молитвами стяжавше вся нам полезная, аще и недостойнии, благодарне восхваляюще тя, прославим Бога, дивнаго во святых Своих, Отца и Сына и Святаго Духа, и ныне и присно и во веки веков. Аминь.


\mychapterending

\mychapter{Увечье и боль рук}
%http://www.molitvoslov.com/content/pri-uvechii-i-boli-ruk

 

\section{Преподобному Иоанну Дамаскину}
%http://www.molitvoslov.com/text708.htm 
 


\bfseries Тропарь, глас 8-й:\normalfont{}


В тебе, отче, известно спасеся еже по образу: приим бо Крест последовал еси Христу, и дея учил еси презирати убо плоть, преходит бо, прилежати же о души, вещи бессмертней. Темже и со ангелы срадуется, преподобне Иоанне, дух твой.


\medskip


\bfseries Тропарь иной , глас 8-й:\normalfont{}


Православия наставниче, благочестия учителю и чистоты, вселенныя светильниче, монашествующих Богодухновенное удобрение, Иоанне премудре, ученьми твоими вся просветил еси, цевнице духовная, моли Христа Бога спастися душам нашим.


\medskip


\bfseries Кондак, глас 4-й:\normalfont{}


Песнописца и честнаго Богоглагольника, Церкве наказателя и учителя и врагов сопротивоборца Иоанна воспоим: оружие бо взем "--- Крест Господень, всю отрази ересей прелесть и яко теплый предстатель к Богу всем подает прегрешений прощение.


\medskip


\bfseries Молитва:\normalfont{}


Преподобне отче Иоанне! Воззри на нас милостивно и к земли приверженных возведи к высоте небесней. Ты горе на небеси, мы на земли низу, удалены от тебе, не толико местом, елико грехми своими и беззаконии, но к тебе прибегаем и взываем: настави нас ходити путем твоим, вразуми и руководствуй. Вся твоя святая жизнь бысть зерцалом всякия добродетели. Не престани, угодниче Божий, о нас вопия ко Господу. Испроси предстательством своим у Всемилостиваго Бога нашего мир Церкви Его, под знамением креста воинствующей, согласие в вере и единомудрие, суемудрия же и расколов истребление, утверждение во благих делех, больным исцеление, печальным утешение, обиженным заступление, бедствующим помощь. Не посрами нас, к тебе с верою притекающих. Вси православнии христиане, твоими чудесы исполненнии и милостями облагодетельствованнии, исповедуют тя быти своего покровителя и заступника. Яви древния милости твоя, и ихже отцем всепомоществовал еси, не отрини и нас, чад их, стопами их к тебе шествующих. Предстояще всечестней иконе твоей, яко тебе живу сущу, припадаем и молимся: приими моления наша и вознеси их на жертвенник благоутробия Божия, да приимем тобою благодать и благовременную в нуждех наших помощь. Укрепи наше малодушие и утверди нас в вере, да несомненно уповаем получити вся благая от благосердия Владыки молитвами твоими. О, превеликий угодниче Божий! Всем нам, с верою притекающим к тебе, помози предстательством твоим ко Господу, и всех нас управи в мире и покаянии скончати живот наш и преселитися со упованием в блаженныя недра Авраамова, идеже ты радостно во трудех и подвизех ныне почиваеши, прославляя со всеми святыми Бога, в Троице славимаго, Отца и Сына и Святаго Духа, ныне и присно и во веки веков. Аминь.


\section{Пресвятой Богородице в честь Ее иконы &quot;Троеручица&quot;}
%http://www.molitvoslov.com/text707.htm 
 


\bfseries Тропарь, глас 4-й:\normalfont{}


Днесь всемирная радость возсия нам велия: даровася святей горе Афонстей цельбоносная Твоя, Владычице Богородице, икона, со изображением тричисленно и нераздельно пречистых рук Твоих, в прославление Святыя Троицы, созываеши бо верных и молящихся Тебе о сем познати, яко двема имаши Сына и Господа держиши, третию же яви на прибежище и покров чтущим Тя от всяких напастей и бед избавляти, да вси, притекающий к Тебе верою, приемлют неоскудно от всех зол свобождение, от врагов защищение, сего ради и мы вкупе с Афоном вопием: радуйся, Благодатная, Господь с Тобою. 


\medskip


\bfseries Кондак, глас 8-й:\normalfont{}


Днесь веселое наста ныне Твое торжество, Богомати Пречистая, вси вернии исполнишася радости и веселия, яко сподобльшеся изрядно воспевати предивное явление честнаго образа Твоего и рождшагося от Тебе Младенца, истинна же Бога, Егоже двема рукама объемлеши, и третиею от напастей и бед нас изымаеши и избавляеши от всех зол и обстояний.


\medskip


\bfseries Молитва:\normalfont{}


О, Пресвятая Госпоже Владычице Богородице, велие чудо святому Иоанну Дамаскину явившая, яко веру истинную "--- надежду несумненную показавшему! Услыши нас, грешных, пред чудотворною Твоею иконою усердно молящихся и просящих Твоея помощи: не отрини моления сего многих ради прегрешений наших, но, яко Мати милосердия и щедрот, избави нас от болезней, скорбей и печалей, прости содеянныя нами грехи, исполни радости и веселия всех, чтущих святую икону Твою, да радостно воспоем и любовию прославим имя Твое, яко Ты еси от всех родов избранная и благословенная во веки веков. Аминь.


\mychapterending

\mychapter{Горячка, лихорадка, жар}
%http://www.molitvoslov.com/content/%D0%B3%D0%BE%D1%80%D1%8F%D1%87%D0%BA%D0%B0-%D0%BB%D0%B8%D1%85%D0%BE%D1%80%D0%B0%D0%B4%D0%BA%D0%B0-%D0%B6%D0%B0%D1%80

 

\section{Первоверховному апостолу Петру}
%http://www.molitvoslov.com/text710.htm 
 


\bfseries Тропарь, глас 4-й:\normalfont{}


Апостолов первопрестольницы, и вселенныя учителие, Владыку всех молите, мир вселенней даровати, и душам нашим велию милость.


\medskip


\bfseries Кондак, глас 2-й:\normalfont{}


Твердыя и боговещанныя проповедатели, верх апостолов Твоих, Господи, приял еси в наслаждение благих Твоих и покой; болезни бо онех и смерть приял еси паже всякого бесплодия. Едине сведый сердечная.


\medskip


\bfseries Молитва:\normalfont{}


О, святый Петре, великий апостоле, самовидче и сотаинниче Божий, всемощною десницею Учителя твоего из вод волнующихся приятый и от конечнаго потопления свободивыйся! Не забуди и нас убогих, в тине греховней погрязших и волнами житейскаго моря обуреваемых: подаждь нам твою руку крепкую, помози нам и удержи нас от потопления во страстех, похотех, лжах и клеветах. Сотвори и ты с нами милость, тебе от Господа явленную, да не в сомнении и маловерии изгибнем. Научи нас, учителю наш, проливати слезы покаяния, да плачем горько о деяниих наших в веце сем. И аще твоя слезы, в покаянии излиянныя, милостию Своею покры Господь и Учитель твой, испроси и нам, со дерзновением апостольским, прощения во гресе ежечаснаго, от Христа отреченныя злыми помыслы и делы нашими. Да тихое и безмолвное житие поживем в веце сем до часа, в оньже имать призвати нас века Господь, наш Судия нелицеприятный. Ты же, о всехвальный апостоле, не отвержи вопля нашего и стенаний к тебе, но заступи нас пред Христом твоим Учителем, да непрестанно славим Его милосердие к нам, со Отцем и Святым Духом, во веки веков. Аминь.


\mychapterending

\mychapter{Женские немощи}
%http://www.molitvoslov.com/content/zhenskie-nemochi

 

\section{Священномученику Ипатию, епископу Гангрскому}
%http://www.molitvoslov.com/text792.htm 
 


\bfseries Тропарь, глас 4-й:\normalfont{}


Изволением Божественнаго разума вперив ум свой в Небесныя обители и равно Ангелом житие пожив на земли, сего ради мучения красоты венец на своем версе утвердив, священномучениче Ипатие, моли Христа Бога спастися душам нашим.


\medskip


\bfseries Другой тропарь, глас 3-й:\normalfont{}


В прекрасном Церкве Апостольския вертограде из юна возраста всесильною десницею Божиею насадился еси, богоблаженне отче Ипатие, и в ней, яко верен превосвященник, верно Богоначальное и Всетворящее Троическое Единство проповедал еси во вселенском Собрании, Духа же святаго в добродетелех исполнен быв, велия чудеса сотворив, мучением за веру Христову скончався, пришел еси в Небесный град, идеже Трипостасного Бога зря, моли о чествующих тя, да в Горнем Иерусалиме со святыми жити сподобимся.


\medskip


\bfseries Кондак, глас 2-й:\normalfont{}


Римскому отечеству светильник, богозарными лучами концы светя, ныне просвети твое празднество поющия, Ипатие, прося свыше мира и грехов оставления, молитвами защити ны, Ипатие многострадальне.


\medskip


\bfseries Другой кондак, глас 2-й:\normalfont{}


Обитель Триипостасного Божества, достоблаженне отче Ипатие, быв во вселенную Всетворца Бога воскликнул еси. Благодарственне же чудодействуя, вшел еси в Небесныя обители, тамо моли Всесвятую Троицу, молим тя, исцели телеса и души наша в богосозданную нашу красоту.


\medskip


\bfseries Молитва:\normalfont{}


Трисолнечного Света светозарная звездо, вводящая в мир Трисолнечный Свет апостольскими твоими, богоевангельскаго проповедания священноучении, священномучениче, иерарше Христов, Богоблаженне Ипатие. Ты приял еси от Святые Троицы равноапостольную благодать, серафимскую в горечестии к Богу любовь, яко же верховнейший апостолов Петр, и херувимскую многоочитую во учении премудрость, яко же вторый Павел. Просветивый премудрым твоим благочестия учением тьмочисленныя народы, присноусердно подражая всемирнаго Учителя, Господа Бога и Спаса нашего Иисуса Христа. Его же ради всесмиренно молю богоподражательное отеческое твое благоутробие: призри милостивно богосветлыми очима твоима на сущее недостоинство мое, яко всечадолюбивый по Бозе отец, и присноусердный нашего спасения наставниче, и всех в Горних правонеизменный строитель, адамантная Православию стена и пресветлый столп, вводящий второноваго Израиля во всепросветлейший Горний Сион. Потщися вскоре помоществовати на бранех воинству нашему. Проси же присно у Святыя Троицы всем православным христианом мира и тишины, душевного спасения и многолетняго в телеси здравия, воздуха благорастворения, земли благоплодия, неплодствующим благочадия, и в законе Господнем воспитания, и всех благих умножения. А после жития коемуждо жизни сея сподоби святыми твоими молитвами улучити христианскую кончину благу, непостыдну и мирну, со всепреподобнейшим исповеданием и со причастием Святых Бессмертных Небесных и Животворяхцих Христовых Тайн, и с молитвомаслием беспрепятное воздушных мытарств прошествие, наследие со святыми всерадостныя присносущныя жизни, в ней же купно со ангельскими чинми непрестанныя хвалы вознесем Отцу со Единородным Его Сыном и с Пресвятым, Благим и Животворящим Его Духом, и тебе, великое твое отеческое милостивое заступление, ныне, и присно, и во веки веков. Аминь.


\mychapterending

\mychapter{Проказа}
%http://www.molitvoslov.com/content/prokaza

 

\section{Преподобному Антонию Великому}
%http://www.molitvoslov.com/text715.htm 
 


\bfseries Тропарь, глас 4-й:\normalfont{}


Ревнителя Илию нравы подражая, Крестителю правыми стезями последуя, отче Антоние, пустыни был еси житель и вселенную утвердил еси молитвами твоими. Темже моли Христа Бога спастися душам нашим.


\medskip


\bfseries Кондак, глас 2-й:\normalfont{}


Житейския молвы отринув, безмолвно житие скончал еси, Крестителя подражаяй всяким образом, преподобнейший, с ним убо тя почитаем, отцев начальниче, Антоние.


\medskip


\bfseries Молитва:\normalfont{}


О, великий угодниче Божий, преподобне отче Антоние! Яко имея дерзновение ко Владыце Христу и ко Пречистей Его Матери, буди о нас, недостойных, молитвенник теплый, заступая нас от всяких бед и напастей, да твоими молитвами невредимы от враг видимых и невидимых пребудем. Моли милосердие Божие, да спасет нас от прегрешений наших имиже веси судьбами. Моли благость Его, еже милостивно храму (дому) сему потребная даровати, жизнь нашу умирити и вся прихожаны храма сего помиловати и спасти души наша, да непрестанно славим, хвалим, поем и величаем Пречестное и Великолепое имя Отца и Сына и Святаго Духа, ныне и присно и во веки веков. Аминь.


\mychapterending

\mychapter{Бессоница}
%http://www.molitvoslov.com/content/bessonica

 

\section{Cвятым седми отрокам, иже во Ефесе: Максимилиану, Иамвлиху, Мартиниану, Иоанну, Дионисию, Ексакустодиану и Антонину}
%http://www.molitvoslov.com/text718.htm 
 


\bfseries Тропарь, глас 8-й:\normalfont{}


Благочестия проповедники и воскресения умерших изобразители, Церкви столпы седмочисленныя, отроки всеблаженныя песньми восхвалим: тии бо по многих летех нетления, аки от сна воставше, всем возвестиша яве мертвых востание.


\medskip


\bfseries Кондак, глас 4-й:\normalfont{}


Прославивый на земли святыя Твоя, прежде втораго и страшнаго пришествия Твоего, Христе, преславным востанием отроков показал еси Воскресение неведящим е, нетленна одеяния и телеса явив, и царя уверил еси вопити: воистинну есть мертвых востание.


\medskip


\bfseries Молитва:\normalfont{}


О, пречуднии святии седмочисленнии отроцы, Ефеса града похвало и всея вселенныя упование! Воззрите с высоты небесныя славы на нас, любовию память вашу чтущих, наипаче же на младенцы христианския, вашему заступлению родителей своих препорученныя: низведите на ны благословение Христа Бога, рекшаго: "Оставите детей приходити ко Мне". Болящия убо в них исцелите, скорбящия утешите; сердца их в чистоте соблюдите, кротостию исполните я, и в земли сердец их зерно исповедания Божия насадите и укрепите, во еже от силы в силу им возрастати; и всех нас, святей иконе вашей предстоящих и тепле вам молящихся, сподобите Царствие Небесное улучити и немолчными гласы радования тамо прославляти Великолепое имя Пресвятыя Троицы, Отца и Сына и Святаго Духа, во веки веков. Аминь.


\section{Cвятому преподобному Иринарху, затворнику Ростовскому}
%http://www.molitvoslov.com/text717.htm 
 


\bfseries Тропарь, глас 4-й:\normalfont{}


Яко мученика добропроизвольна и преподобных удобрение, звезду Ростовскую, в затворе, узах и веригах Господеви благоугодившаго и чудес благодать от Него приемшаго, Иринарха предивнаго песньми хвалебными почтим и, к нему припадающе, умильно глаголем: отче преподобне, моли Христа Бога спастися душам нашим. 

\bfseries 

Кондак, глас 2-й:\normalfont{}


 Волнений множество жестоким житием преходя, изгнания, затвор и узы железныя претерпел еси мужески, Иринарше терпеливодушне, нам оставль образ злострадания и терпения твоего, озаряя чудес блистаньми с верою приходящих к честному гробу твоему, у негоже, яко почесть победную, вериги твоя тяжкия видим, от нихже подаеши исцеления недужным. Сего ради зовем ти: радуйся, Иринарше, отче предивный. 

\bfseries 

Молитва:\normalfont{}


О, преподобне отче Иринарше! Се мы молим тя усердно: буди ходатай наш присный, испроси нам от Христа Бога мир, тишину, благоденствие, здравие и спасение, и от всех врагов видимых и невидимых ограждение, покрый же нас ходатайством твоим от нахождения всяких бед и скорбей, паче же от искушения врага темнаго, да вси мы с тобою прославляем всесвятое имя Отца и Сына и Святаго Духа, ныне и присно и во веки веков. Аминь. 


\mychapterending

\mychapter{Паралич}
%http://www.molitvoslov.com/content/paralich

 

\section{Преподобному Иакову, игумену Железноборовскому}
%http://www.molitvoslov.com/text720.htm 
 


\bfseries Тропарь, глас 4-й:\normalfont{}


Небеснаго желая, земная возненавидел еси и, взем крест свой, последовал еси Христу и от Него прием дарования чудес, исцеляти недужныя; но яко имея дерзновение ко Святей Троице, испроси православным здравие и спасение, на враги же победу, и не забуди, посещая чад своих, припадающих к целебному гробу твоему, Иакове преподобие, отче наш.


\medskip


\bfseries Кондак, глас 8-й:\normalfont{}


Сердечная очи, досточудне, ко Господу вперив и телесныя страсти исторгая, неболезненную жизнь в болезнех восприял еси, преподобне, болезням же лютым даеши исцеление просящим с верою у тебе, темже молим тя, досточудне, исцели наша болезни душевныя и телесныя, да зовем ти: радуйся, Иакове Богомудре, отче наш.


\medskip


\bfseries Молитва:\normalfont{}


Преподобне и Богоносне отче наш Иакове! Приими ныне нас, тебе усердно молящихся и припадающих ко всечестному и многоцелебному твоему гробу, идеже святое и многотрудное покоится тело. Духом же на Небесех предстоя Святей Троице со ангелы и преподобных отцев лики, молися о нас, чадех твоих, отче, да избавимся от всяких скорбей, болезней, бед и обстояний, и благочестно поживем в настоящем житии, ходяще в заповедех и оправданиих Господних безпорочно, и да явимся последователи святому и равноангельскому житию твоему. 

Ей, преподобне отче, молим тя, испроси намже и всем, с верою к тебе притекающим, грехов прощение, телесем здравие, исправление жития и вечное спасение, яко да твоим предстательством спасеннии, славу возсылати сподобимся в Троице славимому Богу, Отцу и Сыну и Святому Духу, ныне и присно и во веки веков. Аминь.


\mychapterending

\mychapter{Болезни крови}
%http://www.molitvoslov.com/content/bolezni-krovi

 

\section{Мученику и чудотворцу Иоанну-воину}
%http://www.molitvoslov.com/text723.htm 
 


\bfseries Тропарь, глас 8-й:\normalfont{}


Блаженство Евангельское возлюбив, Богомудре Иоанне, чистоту сердца почтил еси. Темже суету мира сего пренебрегл, устремился еси зрети Бога, Иже тя прослави чудесы во врачевании различне страждущих. Сего ради молим тя: проси нам у Христа Господа всяких скорбей избавления и получения Царства Небеснаго.


\medskip


\bfseries Кондак, глас 6-й:\normalfont{}


Благочестиваго воина Христова, победившаго враги душевныя и телесныя богомудренно, Иоанна мученика, достодолжно песньми восхвалим, чудодействуя бо, подает обильная исцеления страждущим людем, и молится Господу Богу от всяких бед спасти правоверныя.


\medskip


\bfseries Молитва:\normalfont{}


О, великий Христов мучениче Иоанне, правоверных поборниче, врагов прогонителю и обидимых заступниче. Услыши нас, в бедах и скорбях молящихся тебе, яко дана тебе бысть благодать от Бога печальныя утешати, немощным помогати, неповинныя от напрасныя смерти избавляти и за всех зле страждущих молитися. Буди убо и нам поборник крепок на вся видимыя и невидимыя враги наша, яко да твоею помощию и поборством по нас посрамятся вси являющии нам злая. Умоли Господа нашего, да сподобит ны грешныя и недостойныя рабы Своя получити от Него неизреченная благая, яже уготова любящим Его, в Троице Святей славимаго Бога, всегда, ныне и присно и во веки веков. Аминь.


\mychapterending

\mychapter{Нарывы}
%http://www.molitvoslov.com/content/narivi

 

\section{Святым чудотворцам и бессребреникам, мученикам Киру и Иоанну}
%http://www.molitvoslov.com/text725.htm 
 


\bfseries Тропарь, глас 5-й:\normalfont{}


Святии славнии Кире и Иоанне, безсребреницы, мученицы и чудотворцы, милосердием своим, яко стеною необоримою, нас оградите от всех бед и от врагов видимых и невидимых сохраните непрестанною вашею молитвою ко Господу.


\medskip


\bfseries Кондак, глас 3-й:\normalfont{}


От Божественныя благодати дар чудес приемше, святии, чудодействуйте непрестанно, вся наша страсти рукодейством секуще невидимым, Кире Богомудре со Иоанном славным: вы бо Божественнии врачеве есте.


\medskip


\bfseries Молитва:\normalfont{}


О, святые чудотворцы и врачи безмездные Кире и Иоанне, прибегаем к вам, от болезней наших страдая люто. Во дни жизни вашея проповедовали вы, яко недуг души тяжелее всех недугов тела и, егда душа грехами болезнует, часто и тело в болезнь впадает. Знаем мы, что болезнями от грехов очищаемся, умолите же, святые чудотворцы, Господа, да дарует Он прощение согрешений нам, слабым и беспомощным, в болезнях изможденным, да не погибнем, впавши в уныние. Ускорите на помощь нам и исцелите болезни наши, и мы в здравии, покаянии дни наши проведя, в мире окончим жизнь нашу и в день Судный с чистым сердцем предстанем пред очами Судии вселенной, Сердцеведца Бога. Егоже поем и славим во веки веков. Аминь.


\mychapterending

\mychapter{Кожные болезни и язвы}
%http://www.molitvoslov.com/content/kozhnie-bolezni-i-yazvy

 

\section{Праведному Артемию Веркольскому, чудотворцу}
%http://www.molitvoslov.com/text727.htm 
 


\bfseries Тропарь, глас 2-й:\normalfont{}


Вышняго повелением, тученосным облаком, небо помрачившим и молниям блистающим, грому же возшумевшему с прещением, испустил еси душу твою в руце Господеви, премудре Артемие, и ныне предстоиши Престолу Владыки всех, иже верою и любовию приходящим к тебе, подая исцеление всем неотложно и моляся Христу Богу спастися душам нашим.


\medskip


\bfseries Кондак, глас 8-й:\normalfont{}


Возсия днесь пресветлая память премудраго Артемия; богоданная благодать, яко реки, изливает от целебных мощей его дивная исцеления, имиже от многоразличных недуг избавляемся, с верою приемлюще я и взывающе: радуйся, Артемие богомудре.


\medskip


\bfseries Молитва:\normalfont{}


Святый угодниче Божий, праведный Артемие, присный хранителю святыя веры Православныя и близкий защитниче всего севернаго края Российския страны! Призри милостивно на усердную молитву нас, грешных, и твоим благомощным предстательством испроси нам у Господа прощение согрешений наших, преспеяние в вере и благочестии и ограждение от козней диавольских. 

Моли Господа, да хранит во здравии и непременяемом благополучии верных людей своих, да подаст стране нашей мир и тишину, а нам нелицемерное послушание; да сподобит всех нас получити, по христианстей кончине, Небесное Царствие, идеже вси праведнии вместе с тобою, вечно славят Отца и Сына и Святаго Духа. Аминь.


\mychapterending

\mychapter{Водянка}
%http://www.molitvoslov.com/content/vodyanka

 

\section{Преподобному Ипатию, игумену Руфианскому}
%http://www.molitvoslov.com/text729.htm 
 


\bfseries Тропарь, глас 8-й:\normalfont{}


В тебе, отче, известно спасеся еже по образу: приим бо Крест последовал еси Христу, и дея учил еси презирати убо плоть, преходит бо, прилежати же о души, вещи бессмертней.  Темже и со ангелы срадуется, преподобне Ипатиe, дух твой.


\medskip


\bfseries Кондак, глас 2-й:\normalfont{}


Чистотою душевною Божественне вооружився, и непрестанныя молитвы яко копие вручив крепко, пробол еси бесовская ополчения, преподобне Ипатиe, отче наш, моли непрестанно о всех нас.


\medskip


\bfseries Молитва:\normalfont{}


Прославляется Бог в совете правых, ибо возсиял пред людьми свет добродетелей святого Ипатия, прибегая к которому, из глубины сердца взываем: инок совершенный был ты, Ипатий преподобный, многими трудами к вратам небесным очищавший себе путь; в необитаемую киновию Руфима придя с учениками, ты очистил ее и в прежнее благолепие привел. Там, власяницы делая, в подвигах великих ты пребыл. Дарование получил ты от Господа, Ипатий достохвальный, различные недуги исцелять, а более всего водянку. Призри же на страждущих тяжкою болезнею этой, святый Ипатий, и умилостиви твоею благоприятною молитвою Господа, и престань немощным невидимою чудодейственною силою Божию, исцеляя их по великой милости Господней, и пусть, почувствовав облегчение в теле и бодрость духа, отвергнут они нерадение и ревностно пойдут по лестнице добродетелей, простираясь на преуспеяние в богоугодных делах, и славя Отца и Сына, и Святаго Духа, ныне и присно, и во веки веков. Аминь.


\mychapterending

\mychapter{Расслабление тела с бессонницей, потерей аппетита и лишением каких-либо членов}
%http://www.molitvoslov.com/content/%D1%80%D0%B0%D1%81%D1%81%D0%BB%D0%B0%D0%B1%D0%BB%D0%B5%D0%BD%D0%B8%D0%B5-%D1%82%D0%B5%D0%BB%D0%B0-%D1%81-%D0%B1%D0%B5%D1%81%D1%81%D0%BE%D0%BD%D0%BD%D0%B8%D1%86%D0%B5%D0%B9-%D0%BF%D0%BE%D1%82%D0%B5%D1%80%D0%B5%D0%B9-%D0%B0%D0%BF%D0%BF%D0%B5%D1%82%D0%B8%D1%82%D0%B0-%D0%B8-%D0%BB%D0%B8%D1%88%D0%B5%D0%BD%D0%B8%D0%B5%D0%BC-%D0%BA%D0%B0%D0%BA%D0%B8%D1%85-%D0%BB%D0%B8%D0%B1%D0%BE-%D1%87%D0%BB%D0%B5%D0%BD%D0%BE%D0%B2

 

\section{Преподобному Никите Столпнику, Переяславскому чудотворцу}
%http://www.molitvoslov.com/text731.htm 
 


\bfseries Тропарь, глас 4-й:\normalfont{}


Христову мученику тезоименит был еси, преподобне, многи подвиги и труды претерпел еси Христа ради, Егоже ради вериги носил еси, блаженне: Того ныне о нас моли, Никито преподобне, душевныя наша и телесныя страсти уврачевати, верою и любовию почитающих присно память твою.


\medskip


\bfseries Кондак, глас 8-й:\normalfont{}


Христа ради от твоих рабов нуждную смерть претерпел еси и венец нетления от Него восприял еси, приходящим же с верою от честнаго твоего гроба подаеши цельбы, о Никито преподобне, молитвенниче о душах наших.


\medskip


\bfseries Молитва:\normalfont{}


О, всечестная главо, преподобне отче, преблаженне Никито преподобномучениче! Не забуди нищих твоих до конца, но поминай нас всегда во святых своих и благоприятных молитвах к Богу и не забуди посещати чад своих. Моли за ны, отче благий и избранниче Христов, яко имеяй дерзновение к Небесному Царю, и не премолчи за ны ко Господу, и не презри нас, верою и любовию чтущих тя. 

Поминай нас недостойных у Престола Вседержителева и не престай моляся за нас ко Христу Богу: тебе бо дана бысть благодать молитися за ны. Не мним бо тя мертва суща, аще бо и телом преставился еси от нас, но и по смерти убо жив пребываеши. Не отступай от нас духом, сохраняя и соблюдая нас от стрел вражиих и всякия прелести бесовския, заступниче и молитвенниче наш добрый. 

Аще бо и мощей твоих рака пред очима нашима видима есть всегда: но святая твоя душа со ангельскими воинствы, со безплотными лики, с небесными силами у Престола Вседержителя Бога достойно веселится. Ведуще убо тя воистину и по смерти жива суща, тебе припадаем, и тебе молимся и мили ся ти деем, еже моли ти ся о нас Всесильному Богу о пользе душ наших, и испроси нам время на покаяние, и невозбранно приити от земли на Небо, и горьких мытарств, и воздушных князей, и вечныя муки избавитися нам, и Небесному Царствию наследником быти со всеми праведными, от века угодившими Ему, Господу нашему Иисусу Христу, Емуже подобает всякая слава, честь и поклонение, со Безначальным Его Отцем и Святым Духом, ныне и присно и во веки веков. Аминь.


\mychapterending

\mychapter{Рак}
%http://www.molitvoslov.com/content/%D1%80ri-rake

 

\section{Пресвятой Богородице в честь Ее иконы &quot;Всецарица&quot; (Пантанасса)}
%http://www.molitvoslov.com/text735.htm 
 


\bfseries Тропарь, глас 4-й:\normalfont{}


Образом радостотворным честныя Всецарицы, желанием теплым взыскающих благодати Твоея, спаси, Владычице; избави от обстояний к Тебе прибегающих; от всякия напасти огради стадо Твое, к заступлению Твоему взывающее присно.


\medskip


\bfseries Кондак, глас 8-й:\normalfont{}


Новоявленней Твоей иконе предстояще вернии умиленно, воспеваем Ти, Всецарице, раби Твои; низпосли цельбы к Тебе притекающим ныне рабом Твоим. Да вси радостно зовем Ти: Радуйся, Всецарице, недуги наша благодатию исцеляющая.


\medskip


\bfseries Молитва:\normalfont{}


Всеблагая, досточудная Богородице, Пантанасса, Всецарице! Несмь достоин, да внидеши под кров мой! Но яко милостиваго Бога любоблагоутробная Мати, рцы слово, да исцелится душа моя и укрепится немощствующее тело мое. Имаши бо державу непобедимую, и не изнеможет у Тебе всяк глагол, о Всецарице! Ты за мя упроси! Ты за мя умоли. Да прославляю преславное имя Твое всегда, ныне и в бесконечныя веки. Аминь.


\section{Пресвятой Богородице в честь Ее иконы &quot;Всецарица&quot; (Пантанасса). Молитва вторая.}
%http://www.molitvoslov.com/text736.htm 
 


О Пречистая Богомати, Всецарице! Услыши многоболезненное воздыхание наше пред чудотворною иконою Твоею, из Афонского удела в Россию пренесенною, призри на чад Твоих, неисцельными недуги страждущих, ко святому образу Твоему с верою припадающих! Якоже птица крилома покрывает птенцы своя, тако и Ты ныне, присно жива сущи, покрый нас многоцелебным Твоим омофором. Тамо, идеже надежда исчезает, несумненною Надеждою буди. Тамо, идеже лютыя скорби превозмогают, Терпением и Ослабою явися. Тамо, идеже мрак отчаяния в души вселися, да возсияет неизреченный свет Божества! Малодушныя утеши, немощныя укрепи, ожесточенным сердцам умягчение и просвещение даруй. Исцели болящия люди Твоя, о всемилостивая Царице! Ум и руки врачующих нас благослови; да послужат орудием Всемощнаго Врача Христа Спаса нашего. Яко живей Ти, сущей с нами, молимся пред иконою Твоею, о Владычице! Простри руце Твои, исполненныя исцеления и врачбы, Радосте скорбящих, в печалех Утешение, да чудотворною помощь скоро получив, прославляем Живоначальную и Нераздельную Троицу, Отца и Сына и Святаго Духа, во веки веков. Аминь.


\section{Пресвятой Богородице в честь Ее иконы &quot;Скоропослушница&quot;}
%http://www.molitvoslov.com/text734.htm 
 


\bfseries Тропарь, глас 4-й:\normalfont{}


К Богородице притецем, сущии в бедах, и святей иконе Ея ныне припадем, с верою зовуще из глубины души: скоро наше услыши моление, Дево, яко Скоропослушница нарекшаяся. Тебе бо раби Твои в нуждах готовую помощницу имамы.


\medskip


\bfseries Кондак, глас 8-й:\normalfont{}


В море житейстем обуреваемии, треволнению подпадаем страстей и искушений. Подаждь убо нам, Госпоже, руку помощи, якоже Петрови Сын Твой, и ускори от бед избавити ны, да зовем Ти: радуйся, всеблагая Скоропослушнице.


\medskip


\bfseries Молитва:\normalfont{}


Преблагословенная Владычице, Приснодево Богородице, Бога Слова паче всякаго слова на спасение наше рождшая и благодать Его преизобильно паче всех приявшая, море явльшаяся Божественных дарований и чудес приснотекущая река, изливающая благость всем, с верою к Тебе прибегающим! Чудотворному Твоему образу припадающе, молимся Тебе, всещедрей Матери Человеколюбиваго Владыки: удиви на нас пребогатыя милости Твоя и прошения наша, приносимая Тебе, Скоропослушнице, ускори исполнити все, еже на пользу, во утешение и спасение коемуждо устрояющи. Посети, Преблагая, рабы Твоя благодатию Твоею и подаждь недугующим цельбу и совершенное здравие, обуреваемым тишину, плененным свободу и различными образы страждущих утеши. Избави, Всемилостивая Госпоже, всяк град и страну от глада, язвы, труса, потопа, огня, меча и иныя казни временныя и вечныя, Матерним Твоим дерзновением отвращающи гнев Божий; и душевнаго разслабления, обуревания страстей и грехопадений свободи рабы Твоя, яко да непреткновенно во всяком благочестии поживше в сем веце, и в будущем вечных благ сподобимся благодатию и человеколюбием Сына Твоего и Бога, Емуже подобает всякая слава, честь и поклонение со Безначальным Его Отцем и Пресвятым Духом, ныне и присно, и во веки веков. Аминь.


\section{Cвятителю Нектарию Эгинскому}
%http://www.molitvoslov.com/text737.htm 
 


\bfseries Тропарь, глас 1-й:\normalfont{}


Силиврии отрасль и эгины хранителя, в последняя лета явльшагося, добродетели друга искренняго, Нектария почтим вернии, яко божественнаго служителя Христова: точит бо цельбы многоразличныя благочестно вопиющим: слава Прославльшему тя Христу, слава Давшему ти чудес благодать, слава Действующему тобою всем исцеления.


\medskip


\bfseries Кондак, глас 2-й:\normalfont{}


Божественный гром, труба духовная, веры насадителю и отсекателю ересей, Троицы угодниче, великий святителю Нектарие, со ангелы предстоя присно, моли непрестанно о всех нас.


\medskip


\bfseries Молитва:\normalfont{}


О, Святителю Нектарие, отче Богомудрый! Приими, блюстителю веры православныя, исповедание устен людей христоименитых, собранных днесь в храме благодатию Божиею, в тебе живущею. Весть бо достиже до предел Российских, яко ты, великий во святых угодниче Христов, во всех концех вселенныя призывающим имя твое являешися и от раковыя болезни исцеление даруеши.


Слышахом об иерее, тебе тезоименитом и храм во имя твое созидавшем, со скорбьми велиими. Раковою убо язвою грудною поражен бе, крови на всяк день точащейся, и страждаше люте; дела же своего святаго не оставляше.


Внезапу ты, Святителю многомилостиве, с небесе сшед, ему во храме предстал еси в видимом образе. Он же, непщуя тя собрата, едина от смертных быти, прошаше молитв твоих и рече: «Боляй есмь вельми, обаче хощу святый олтарь наздати, да поне единожды святую литургию вкупе с прихожаны совершу; послежде н умрети готов есмь, смерть бо мя не страшит».


Ты же, отче, безплотен сый, слезами лице свое орошаше! и объем страдальца, лобызаше и, глаголя: «Не тужи, чадо мое, яко, болезнию испытан, здрав будеши. Вси убо о чудеси сем ведати имут». Он же, исцелен быв, асбие разуме, с кем беседоваше, тебе невидиму бывшу.

О, великий угодниче Христов Нектарие! Храм оный свершен есть ныне и чудеса твои, аки море преизливающееся, умножишася! Мы же познахом, яко молитве праведника должно поспешествуемой быти усердием нашим к службе Божией и решимостию за Христа умирати, да здрави обрящемся. Молят тя, отче праведный, болящая чада твоя: да свершается с нами воля Божия, благая, угодная и совершенная, не хотящи смерти грешнику, но еже обратитися и живу быти ему.


Ты же, провозвестниче воли Божией, исцели ны благодатным явлением своим, да велик будет Бог на небеси и на земли во веки веков! Аминь.


\mychapterending

\mychapter{Молитва первая об исцелении болящего}
%http://www.molitvoslov.com/text158.htm 
 


Владыко, Вседержителю, святый Царю, наказуяй и не умерщвляяй, утверждаяй низпадающия и возводяй низверженные, телесныя человеков скорби исправляяй, молимся Тебе, Боже наш, раба Твоего \itshape (имя рек)\normalfont{} немоществующа посети милостию Твоею, прости ему всякое согрешение вольное и невольное. Ей, Господи, врачебную Твою силу с небесе низпосли, прикоснися телеси, угаси огневицу, укроти страсть и всякую немощь таящуюся, буди врач раба Твоего \itshape (имя рек),\normalfont{} воздвигни его от одра болезненнаго и от ложа озлобления цела и всесовершенна, даруй его Церкви Твоей благоугождающа и творяща волю Твою. Твое бо есть, еже миловати и спасати ны, Боже наш, и Тебе славу возсылаем, Отцу и Сыну и Святому Духу, ныне и присно и во веки веков. Аминь.





\mychapterending

\mychapter{Mолитва вторая об исцелении болящего}
%http://www.molitvoslov.com/text616.htm 
 


О, Премилосердый Боже, Отче, Сыне и Святый Душе, в нераздельной Троице поклоняемый и славимый, призри благоутробно на раба Твоего (\itshape имя\normalfont{}), болезнею одержимаго; отпусти ему вся согрешения его; подай ему исцеление от болезни; возрати ему здравие и силы телесныя; подай ему долгоденственное и благоденственное житие, мирные Твои и примирные благая, чтобы он вместе с нами приносил благодарные мольбы Тебе, Всещедрому Богу и Создателю моему. 


Пресвятая Богородица, всесильным заступлением Твоим помоги мне умолить Сына Твоего, Бога моего, об исцелении раба Божия (\itshape имя\normalfont{}). 


Все святые и ангелы Господни, молите Бога о больном рабе Его (\itshape имя\normalfont{}). Аминь.





\mychapterending